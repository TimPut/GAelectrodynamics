%
% Copyright � 2016 Peeter Joot.  All Rights Reserved.
% Licenced as described in the file LICENSE under the root directory of this GIT repository.
%
%{
\input{../latex/blogpost.tex}
\renewcommand{\basename}{junk}
%\renewcommand{\dirname}{notes/phy1520/}
\renewcommand{\dirname}{notes/ece1228-electromagnetic-theory/}
%\newcommand{\dateintitle}{}
%\newcommand{\keywords}{}

\input{../latex/peeter_prologue_print2.tex}

\usepackage{peeters_layout_exercise}
\usepackage{peeters_braket}
\usepackage{peeters_figures}
%\usepackage{siunitx}
%\usepackage{macros_qed}
%\usepackage{mhchem} % \ce{}
\usepackage{macros_bm} % \bcM
\usepackage{txfonts} % \ointclockwise

\newcommand{\boldTextAndIndex}[1]{%
{\color{Maroon}{#1}}%
\index{#1}%
}

\beginArtNoToc

\generatetitle{Junk}

\section{Vector products in various algebras.}
   %
% Copyright © 2016 Peeter Joot.  All Rights Reserved.
% Licenced as described in the file LICENSE under the root directory of this GIT repository.
%
A new student of vector algebra will first learn
%first see vectors in two or three dimensions as sets of coordinates
%
%\begin{equation}\label{eqn:GAmotivation:20}
%\Ba =
%\begin{bmatrix}
%a_1 \\
%a_2 \\
%a_3 \\
%\end{bmatrix}, \qquad
%\Bb =
%\begin{bmatrix}
%b_1 \\
%b_2 \\
%b_3 \\
%\end{bmatrix},
%\end{equation}
%
%or perhaps explicitly in terms of a basis \( \setlr{ \Be_1, \Be_2 \Be_3 } \)
%
%\begin{dmath}\label{eqn:GAmotivation:40}
%\begin{aligned}
%\Ba &= a_1 \Be_1 + a_2 \Be_2 + a_3 \Be_3  \\
%\Bb &= b_1 \Be_1 + b_2 \Be_2 + b_3 \Be_3
%\end{aligned}.
%\end{dmath}
%
%You will learn the
rules for addition and subtraction of such vectors.
%, and then how to operate on them with rotation matrices or other representations of linear transformations.
This demonstrates to the student that the vector is an algebraic object that generalize numbers, and the question of how to
multiply vectors soon follows.

Given the toolbox of traditional vector algebra, the best answer that a new student will obtain from such a line of questioning is to learn of the dot and cross products, the multiplication like operations that we are all so familiar with

%\begin{itemize}
%\item ``You can not multiply vectors.'', or
%\item ``Vector multiplication is not well defined.'', or
%\item ``We will get to that.'', or
%\item ``There are multiplication like operations.''  The
%\end{itemize}

\begin{dmath}\label{eqn:GAmotivation:60}
\begin{aligned}
\Ba \cdot \Bb &= a_1 b_1 + a_2 b_2 + a_3 b_3 = \Norm{\Ba} \Norm{\Bb} \cos \theta_{ab} \\
\Ba \cross \Bb &=
\begin{vmatrix}
\Be_1 & \Be_2 & \Be_3 \\
a_1 & a_2 & a_3 \\
b_1 & b_2 & b_3 \\
\end{vmatrix}
= \ncap_{ab} \Norm{\Ba} \Norm{\Bb} \sin\theta_{ab}.
\end{aligned}
\end{dmath}

Both of these multiplication like operations live in very different spaces, one scalar, and the other a vector that lies outside of the span of its two vector factors.  Observe that the magnitudes of these two product operations are related to the product of the vectors in a Pythagorean sense

\begin{dmath}\label{eqn:GAmotivation:180}
\lr{ \Ba \cdot \Bb }^2 + \Norm{ \Ba \cross \Bb }^2
=
\Norm{\Ba}^2 \Norm{\Bb}^2 \cos^2 \theta_{ab}
+\Norm{\Ba}^2 \Norm{\Bb}^2 \sin^2 \theta_{ab}
=
\Norm{\Ba}^2 \Norm{\Bb}^2.
\end{dmath}

This can be seen as a hint that the dot and cross products might be components of a single vector product operation, but the precise form of that product is not obvious.

Vector products that have the same form as the scalar magnitudes of the dot and cross products can be found in other algebraic systems.  Given a complex number representation of two vectors in a 2D space

\begin{dmath}\label{eqn:GAmotivation:200}
\begin{aligned}
z &= r e^{i \theta} \leftrightarrow (a, b) \\
w &= \rho e^{i \alpha} \leftrightarrow (a', b'),
\end{aligned}
\end{dmath}

the inner product of such a complex vector representation can be seen to have the same structure as the dot and cross products

\begin{equation}\label{eqn:GAmotivation:100}
\begin{aligned}
\Real( z w^\conj ) &= r \rho \cos(\theta - \alpha) \\
\Imag( z w^\conj ) &= r \rho \sin(\theta - \alpha).
\end{aligned}
\end{equation}

It can be shown
(\cref{problem:introGAproblems:ComplexInnerProductVsDotAndCrossProduct})
that this inner product has the following vector isomorphism

\begin{dmath}\label{eqn:GAmotivation:220}
z w^\conj \leftrightarrow ( a a' + b b', a' b - a b' ).
\end{dmath}

One component is completely symmetric, whereas the other component of this product has a component that is completely antisymmetric.
The 3D cross product components also have this antisymmetry, and that antisymmetry will be seen later to be the key to the generalization of the cross product.  In this particular case, one can view this antisymmetric sum \( a' b - a b' \) as one
answer of how the cross product ``generalizes'' from 3D to 2D without requiring the introduction of a normal dimension.

The answer to questions of exactly how the vector products, in particular the cross product, should generalize to higher dimensional spaces are still outstanding.  It should be expected that this cross product generalization will involve antisymmetry, just as the dot product generalization in higher dimensional spaces is completely symmetric.

Many current students of science never see the exact structure of this generalization.  Should studies happen to include
enough of the right esoteric physics and mathematics (quantum mechanics, QED, calculus on manifolds, ...) then
some answers to those questions may be found.  Unfortunately, there are many such answers, and many of them each only provide
one part of the picture.

For example, a student of non-relativistic quantum mechanics will learn of Pauli matrices, when studying spin operators.  The dot and cross products will be seen to be components of a more general vector multiplication operation

\begin{equation}\label{eqn:GAmotivation:120}
\lr{\Bsigma \cdot \Bx }
\lr{\Bsigma \cdot \By }
=
I \lr{ \Bx \cdot \By } + i \Bsigma \cdot \lr{ \Bx \cross \By }.
\end{equation}

In quaternion algebra, a generalization of complex algebra, when a quaternion is represented as a scalar vector pair \( q = (r, \Bv) \), the quaternion product of two vectors also shows that the dot and cross products are respective components of a product of vectors

\begin{dmath}\label{eqn:GAmotivation:240}
(0, \Bx)
(0, \By) = (-\Bx \cdot \By, \Bx \cross \By).
\end{dmath}

A student of quantum field theory will encounter Dirac matrices, a algebraic structure that allows for the multiplication of four-vectors

\begin{dmath}\label{eqn:GAmotivation:140}
\aslash \bslash
=
\inv{2} \symmetric{ \aslash}{ \bslash }
+
\inv{2} \antisymmetric{ \aslash}{ \bslash }
=
a^\mu b_\mu + \inv{2} a^\mu b^\nu \antisymmetric{\gamma_\mu}{\gamma_\nu}
=
a^\mu b_\mu + \inv{2} a^\mu b^\nu \lr{
\gamma_\mu \gamma_\nu
-
\gamma_\nu \gamma_\mu
}.
\end{dmath}

A product of ``Dirac'' vectors has symmetric and antisymmetric components that generalize the dot and cross products.
Unfortunately, this algebra comes with still another different notation.
One interesting take away from this particular vector product is the fact that one component is a scalar, and other other
involves products of vectors, something that will require further interpretation.  Since the Dirac basis typically has a matrix representation, such a product can be dismissed as just being another matrix.  The products of mutually orthonormal vectors will show up again later in a context where there is no requirement to assume a matrix representation of the underlying basis.

Another common and important context that contains generalizations of the dot and cross products is the subject of differential forms.
A student of differential forms will learn how to compute the wedge products of forms, and of duality operations, which can be used to construct generalized multiplication operations that have the structure of the 3D dot and cross products

\begin{equation}\label{eqn:GAmotivation:160}
\begin{aligned}
df \wedge * dg &= \lr{ \sum_{i=1}^3 \PD{x_i}{f} \PD{x_i}{g} } dx_1 \wedge dx_2 \wedge dx_3 \\
df \wedge dg &= \sum_{1 \le i < j \le 3} \lr{
\PD{x_i}{f} \PD{x_j}{g}
-\PD{x_j}{f} \PD{x_i}{g}
}
dx_i \wedge dx_j.
\end{aligned}
\end{equation}

It is possible to express vectors as a differential form, and some advocate for this \citep{flanders1989dfa}, but this can also seem unnatural.  Regardless, differential forms do highlight the existence of more general concepts of vector multiplication.  %In this particular case, this generality comes with the cost of using yet another notation, one that is considerably different than the vector notation that we are comfortable with.

It should not be surprising that all of these ideas are special cases of a more general algebraic system.

The aim of the material to follow is to provide the instruction manual for an enhanced toolbox of vector algebra techniques that can be used to gain an integrated view of many seemingly disparate mathematical methods.  These are tools that can be learned without having to first study the esoteric arts of quantum mechanics or differential forms, and have many applications once learned.  These notes will focus on applications to the study of electromagnetism.


   \subsection{Problems}
      %
% Copyright © 2016 Peeter Joot.  All Rights Reserved.
% Licenced as described in the file LICENSE under the root directory of this GIT repository.
%

\makeproblem{Complex inner product vs. dot and cross product.}{problem:introGAproblems:ComplexInnerProductVsDotAndCrossProduct}{
Given two 2D vectors \( (a,b) \) and \( (a', b') \), and a complex number representation of these vectors \( z = a + ib, w = a' + i b' \), show that the components of the complex inner product have the representation
given by \cref{eqn:GAmotivation:220}.
} % problem

\makeanswer{problem:introGAproblems:ComplexInnerProductVsDotAndCrossProduct}{
\begin{dmath}\label{eqn:introGAproblems:20}
z w^\conj
=
(a + ib)(a' - ib')
=
a a' + b b'
+ i \lr{ a' b - a b' }
\leftrightarrow
( a a' + b b', a' b - a b' ).
\end{dmath}
} % answer

   \subsection{Problem solutions}
      \shipoutAnswer
\section{SimpleProducts.tex}
   
\subsection{Operational rules}
For Euclidean vector spaces, spaces for which the length of a vector is always positive, a couple operational vector multiplication rules are sufficient to some basic familiarity with the algebra and derive some important conquences.
Assuming an orthonormal basis \( \setlr{ \Be_1, \Be_2, \cdots } \), those rules are

\makerule{Square of a unit vector.}{rule:simple:square}{
The square of a unit vector, such as \( \Be_1 \) is 1, a scalar.  Stated more generally

\begin{equation*}
\Be_i \Be_i = 1, \quad \forall i \in 1, 2, \cdots
\end{equation*}
}

\makerule{Orthogonal unit vectors anticommute.}{rule:simple:anticommute}{
The product of two orthognal unit vectors, such as \( \Be_2 \Be_1 \) anticommutes.  That is \( \Be_2 \Be_1 = - \Be_1 \Be_2 \).  Stated more generally
\begin{equation*}
\Be_i \Be_j = -\Be_j \Be_i, \quad \forall i \ne j \in 1, 2, \cdots
\end{equation*}
}

In addition to these rules, one must assume that linear combinations of products of unit vectors are well formed.

\section{SimpleProducts0.tex}
   %
% Copyright © 2017 Peeter Joot.  All Rights Reserved.
% Licenced as described in the file LICENSE under the root directory of this GIT repository.
%
For example, the \boldTextAndIndex{multivector} \( M \) below is well formed
(cut)

It must also be assumed that products of multivectors are distributive with respect to the chosen basis, and that vector products are associative with respect to multiplication.
These rules and assumptions could be used as the axioms of Geometric Algebra, but it will be desirable to express \ref{rule:simple:square} in a slightly more general form, a form that has both \ref{rule:simple:anticommute} and \ref{rule:simple:square} as consequences.

\subsection{Irreducible products and grade}

Using the rules above, some of the terms in the multivector \( M \) above can be simplified.
The first such simplification follows immediately, since by \ref{rule:simple:square}, that term is

(cut)

demonstrating that multivectors are allowed to contain \boldTextAndIndex{scalars}.
The scalar part of a multivector is said to have a \boldTextAndIndex{grade} of zero, or be of grade-0.
The second term can be reduced by grouping a pair of products and anticommutation

(cut)

which shows that multivectors are allowed to contain \boldTextAndIndex{vectors}.
The next term \( \Be_2 \Be_3 \) cannot be reduced using any of the rules in the toolbox.
An irredicible product of two unit vectors will be referred to as a \boldTextAndIndex{bivector}, and will be said to have grade-2.
An interpretation of such a product will be required, but can be thought of for now as an oriented unit area, just as a vector can be thought of as an oriented line.
(cut)
The next term, a scaled product of four unit vectors can be reduced by a similar process of grouping, anticommutation, and application of rule 1.

(cut)
There is freedom to write this as \( - 5 \Be_2 \Be_1 \) if desired, but regardless, it is a
scaled irredicible product of two orthonormal vectors, so we say it has grade-2, and can call it a bivector like \( \Be_2 \Be_3 \) above.
(cut)

This could be written in other forms, such as \( \Be_2 \Be_1 \Be_3, \Be_1 \Be_3 \Be_2 \), or \( -\Be_3 \Be_1 \Be_2 \), but it is clearly a scaled irredicible product of three orthonormal vectors.
Such a product is said to have grade-3, and will be called a \boldTextAndIndex{trivector}.
This can be thought of as an oriented volume.
%Like an oriented area, the geometrical interpretations of this trivector can be deformed into other shapes.
(cut)

(cut)
Unlike bi- and tri- vectors, such a product is not generally given a special name in higher degree Euclidean vector spaces.
%Having called grade-1, grade-2, and grade-3 multivectors components vectors, bivectors and trivectors respectively, one might be inclined to refer to this as a four-vector.
%Such a label is not generally used, likely because of the existing meaning of four-vector in relativity.

With these various reductions calculated, the multivector \cref{eqn:SimpleProducts:20} is simplified to

(cut)

%Observe that this shorthand makes it a bit easier to pick out the grades
\subsection{Grade selection}

Being able to identify the grades of a multivector of fundamenal importance and utility.
The grade selection operator is defined for this purpose.
By example, using the multivector of \cref{eqn:SimpleProducts:20}, one writes

\begin{dmath}\label{eqn:SimpleProducts:140}
\begin{aligned}
   \gpgrade{M}{0} &= 1 \\
   \gpgradeone{M} &= - 2 \Be_2 \\
   \gpgradetwo{M} &= \Be_2 \Be_3 + 5 \Be_1 \Be_2 \\
   \gpgradethree{M} &= - \Be_1 \Be_2 \Be_3 \\
   \gpgrade{M}{4} &= -\Be_2 \Be_3 \Be_4 \Be_5.
\end{aligned}
\end{dmath}

The scalar (or grade-0) selection is particularly useful, and is given the special notation

\begin{dmath}\label{eqn:SimpleProducts:180}
\gpgradezero{M} = \gpgrade{M}{0}.
\end{dmath}

Given a decomposition of a multivector into its respective grades, it can be recovered from the sum of all its grades, up to the dimension \( N \) of the underlying vector space

\begin{dmath}\label{eqn:SimpleProducts:160}
   M = \sum_{k = 0}^N \gpgrade{M}{k}.
\end{dmath}

\section{SimpleProducts2.tex}
   The scalar portion of this product is the dot product, whereas the bivector portion is something else.


We will show that this bivector is related to the \R{3} cross product.
Low
We have found that it is possible to represent the dot product of two

What is the product of two vectors that

\begin{dmath}\label{eqn:SimpleProducts2:1180}
\Ba \cdot \Bb
=
\gpgradezero{ \Ba \Bb }
=
\gpgradezero{ \lr{ \sum_i a_i \Be_i }
\lr{ \sum_j b_j \Be_j }
}
=
\sum_{ij} a_i b_j \gpgradezero{ \Be_i \Be_j }.
\end{dmath}


In order to discuss the product of two vectors, we have to introduce two additional operators.  The first of these is a generalization of the dot product to multivectors

By this definition, the dot product of two vectors \( \Ba, \Bb \), is

\boxedEquation{eqn:SimpleProducts2:1140}{
\Ba \cdot \Bb
=
\gpgradezero{ \Ba \Bb }.
}

Before anything else, let's check that this definition agrees with our expectations for a vector dot product.  Let
so the generalized dot product is

\begin{dmath}\label{eqn:SimpleProducts2:1560}
\Ba \cdot \Bb
=
\gpgradezero{ \Ba \Bb }
=
\gpgradezero{ \lr{ \sum_i a_i \Be_i }
\lr{ \sum_j b_j \Be_j }
}
=
\sum_{ij} a_i b_j \gpgradezero{ \Be_i \Be_j }.
\end{dmath}

Here a change of index variables was used for \( \Bb \) and the sums and scalars were all factored out of the grade selection operator.  The multivectors \( \Be_i \Be_j \) are unit scalars whenever \( i = j \) and are bivectors whenever \( i \ne j \), so the scalar selection is a Kronecker delta function, leaving

\begin{dmath}\label{eqn:SimpleProducts2:1200}
\Ba \cdot \Bb
=
\sum_{ij} a_i b_j \delta_{ij}
=
\sum_{i} a_i b_i.
\end{dmath}

As claimed this is the familiar (Euclidean) dot product.
Some examples of vector-bivector, bivector-bivector, and vector-trivector dot products will be considered later.
Since immediate goal is examination of the geometric product of two vectors, we need our second special case grade selection operator, the wedge product


In particular, the wedge product of two vectors \( \Ba, \Bb \) is

\boxedEquation{eqn:SimpleProducts2:1220}{
\Ba \wedge \Bb = \gpgradetwo{ \Ba \Bb }.
}

Using the coordinate expansion \cref{eqn:SimpleProducts2:1160}, the wedge product of two vectors is

\begin{dmath}\label{eqn:SimpleProducts2:1240}
\Ba \wedge \Bb
=
\gpgradetwo{ \Ba \Bb }
=
\gpgradetwo{ \lr{ \sum_i a_i \Be_i }
\lr{ \sum_j b_j \Be_j }
}
=
\sum_{ij} a_i b_j \gpgradetwo{ \Be_i \Be_j }
\end{dmath}

As pointed out \( \Be_i \Be_j \) is a (unit) scalar whenever \( i = j \), and is a bivector (grade 2) whenever \( i \ne j \), so

\begin{dmath}\label{eqn:SimpleProducts2:1260}
\Ba \wedge \Bb
=
\sum_{i \ne j} a_i b_j \Be_i \Be_j
=
\inv{2}
\sum_{i \ne j} a_i b_j
\Be_i \Be_j
+
\inv{2}
\sum_{r \ne t} a_r b_t
\Be_r \Be_t.
\end{dmath}

The sum has been halved and doubled using a second set of indexes.  This is a sneaky trick often used in physics with indexed (tensor) equations, and worth knowing.  That second sum can now be manipulated

\begin{dmath}\label{eqn:SimpleProducts2:1280}
\inv{2}
\sum_{r \ne t} a_r b_t
\Be_r \Be_t
=
\inv{2}
\sum_{r \ne t} a_r b_t
(-\Be_t \Be_r)
=
-\inv{2}
\sum_{j \ne i} a_j b_i
\Be_i \Be_j.
\end{dmath}

The normal vectors \( \Be_r, \Be_t \) were first anticommuted (swap the order and the sign), and then a final substitution of summation indexes was made.  Adding the two halves together again gives

\begin{dmath}\label{eqn:SimpleProducts2:1300}
\Ba \wedge \Bb
=
\inv{2} \sum_{i \ne j} (a_i b_j - a_j b_i) \Be_i \Be_j,
\end{dmath}

or, in terms of determinants

The \R{2} expansion of the wedge product is just

\begin{dmath}\label{eqn:SimpleProducts2:1340}
\Ba \wedge \Bb
=
\begin{vmatrix}
a_1 & a_2 \\
b_1 & b_2
\end{vmatrix}
\Be_1 \Be_2.
\end{dmath}
\paragraph{Computing the normal 2D}

Similar to multiplication by the complex number \( i \),
left and right multiplication by the pseudoscalar also rotate vectors by \( \pi/2 \) radians, however, the sign of the rotation depends on whether the multiplication is from the left or from the right.
With

and multiplication from the right

the vector is rotated clockwise by \( \pi/2 \).
% (\cref{problem:left2dimaginarymultiplication:1}).
%Just as the imaginary rotates complex numbers, the 2D pseudoscalar rotates vectors, with the cavaet that one must be careful about the order that this multiplication is performed.
%
\paragraph{Complex numbers and rotations}

The 2D pseudoscalar has been seen to have the characteristics of the complex imaginary.
This analogy can be extended by noting that the following multivector

\begin{dmath}\label{eqn:SimpleProducts2:320}
z = x + \Be_1 \Be_2 y,
\end{dmath}

provides an isomorphic representation of a complex number.
This can be made obvious by introduces the lable \( i = \Be_1 \Be_2 \) for the pseudoscalar, since it has the desired characteristics of the imaginary.

Such a complex exponential only rotates vectors that lie in the plane of the bivector, and commute with vector (or vector components) that lie outside of the rotational plane (\cref{problem:normalMult:1}).
This means that conjugate complex exponentials applied from both sides,
rotate the components of the vector that lie in the rotational plane, but leave the normal components unaltered.
The desired expression of a \R{N} rotation
has the ``sandwich'' structure
\footnote{Dual sided rotations of this form are also found in Pauli (matrix) algebra and quaternion formalisms, both of which, like complex numbers, can be considered special cases of geometric algebras.}

\boxedEquation{eqn:SimpleProducts2:840}{
\Bx' =
e^{-i\theta/2} \Bx e^{i \theta/2},
}

where \( i \) is the unit bivector for the plane of rotation.
When the vector has no components outside of the plane, single sided rotations can be used, which is often convienient

\begin{equation}\label{eqn:SimpleProducts2:980}
\Bx' = \Bx e^{i \theta} = e^{-i\theta} \Bx.
\end{equation}

\paragraph{Vector product.}

The product of two 2D vectors is illustrative.
Let

\begin{dmath}\label{eqn:SimpleProducts2:500}
\begin{aligned}
   \Bx &= x_1 \Be_1 + x_2 \Be_2 \\
   \By &= y_1 \Be_1 + y_2 \Be_2
\end{aligned},
\end{dmath}

and consider their product
\begin{dmath}\label{eqn:SimpleProducts2:520}
\Bx \By
=
\lr{ x_1 \Be_1 + x_2 \Be_2 }
\lr{ y_1 \Be_1 + y_2 \Be_2 }
=
x_1 y_1 \Be_1 \Be_1 + x_2 y_1 \Be_2 \Be_1
+
x_1 y_2 \Be_1 \Be_2 + x_2 y_2 \Be_2 \Be_2
=
x_1 y_1
- x_2 y_1 \Be_1 \Be_2
+ x_1 y_2 \Be_1 \Be_2
+ x_2 y_2
=
x_1 y_1 + x_2 y_2
+ \lr{ x_1 y_2 - x_2 y_1 } \Be_1 \Be_2.
\end{dmath}

The vector product has a scalar and a bivector component, where the scalar component is completely symmetric, whereas the bivector component is completely antisymmetric.
Observe that the scalar selection from the vector product is the dot product.
A name is required for the bivector (grade-2) selection from this product, and it is called the wedge product \( \Bx \wedge \By \), an operation that has to be examined in more depth.
With such a definition, the vector product can be written as

%\begin{dmath}\label{eqn:SimpleProducts2:540}
\boxedEquation{eqn:SimpleProducts2:540}{
\Bx \By
= \Bx \cdot \By + \Bx \wedge \By,
}
%\end{dmath}

where

\begin{dmath}\label{eqn:SimpleProducts2:560}
\begin{aligned}
\Bx \cdot \By &\equiv \gpgradezero{\Bx\By} = x_1 y_1 + x_2 y_2 \\
\Bx \wedge \By &\equiv \gpgradetwo{\Bx\By} =
\begin{vmatrix}
   x_1 & x_2 \\
   y_1 & y_2
\end{vmatrix}
   \Be_1 \Be_2. \\
\end{aligned}
\end{dmath}


\section{SimpleProducts3.tex}
   %
% Copyright © 2017 Peeter Joot.  All Rights Reserved.
% Licenced as described in the file LICENSE under the root directory of this GIT repository.
%
\section{MESS}
The wedge product is related to the cross product, but can generalizes the cross product to two dimensions where there is no normal direction, and can generalize the cross product to greater than three dimensions, where any plane has too many normal directions.
The cross product is a (pseudo)vector that has a magnitude equal to the area of the parallelogram spanned by the crossed vectors, but is pointed normal to the plane of those vectors.
It will be possible to interpret the wedge product as the oriented (signed) area of that parallelogram itself without reference to any normal direction.
In the same sense that a vector is a representation of an oriented line segment, we will see that the wedge product of two vectors can be thought of as a representation of a oriented plane segment.
(cut)

Recall (\cref{problem:SimpleProducts2:areaofparallelogram}) that the absolute value of this determinant is precisely the area of the parallelogram formed by the vectors \( \Ba \) and \( \Bb \).  The wedge product, a bivector, can therefore be interpretted as an oriented (signed) area.  This is illustrated in \cref{fig:orientedParallelogram:orientedParallelogramFig1}.

\imageFigure{../figures/GAelectrodynamics/orientedParallelogramFig1}{Oriented area interpretation of \( \Bv_1 \wedge \Bv_2 \) and \( \Bv_2 \wedge \Bv_1 \).}{fig:orientedParallelogram:orientedParallelogramFig1}{0.3}

(cut)

Observe that in \R{2} the product of any basis vector with a pseudoscalar is normal to the original vector, which is also generally true for any vector in a 2D space,
(snip)
%\cref{fig:rotationOfe1:rotationOfe1Fig1}.
%\imageFigure{../figures/GAelectrodynamics/rotationOfe1Fig1}{CAPTION: rotationOfe1Fig1}{fig:rotationOfe1:rotationOfe1Fig1}{0.3}
%\cref{fig:rotationOfe2:rotationOfe2Fig1}.
%\imageFigure{../figures/GAelectrodynamics/rotationOfe2Fig1}{CAPTION: rotationOfe2Fig1}{fig:rotationOfe2:rotationOfe2Fig1}{0.3}
Such a multiplication induces a \( \pi/2 \) rotation, the direction of which depends on the orientation of pseudoscalar, and upon whether the multiplication is performed from the left or the right.

This unit bivector is seen to square to minus one like the imaginary in complex algebra.
The reader can confirm easily that this is generally true for any unit bivector \( \Be_i \Be_j, \, i \ne j \).
This is a very convienient fact, and allows ad-hoc construction of complex number like coordinate systems in any given planar subspace.

(cut)

It was previously claimed that the pseudoscalar \( \Be_1 \Be_2 \) could be interpretted as an oriented (signed) area.
Because \( \Be_1 \cdot \Be_2 = 0 \), this vector product is also equal to the wedge

\begin{dmath}\label{eqn:SimpleProducts2:640}
\Be_1 \Be_2 = \Be_1 \cdot \Be_2 +
\Be_1 \wedge \Be_2
=
\Be_1 \wedge \Be_2.
\end{dmath}

Recall that the area of the parallopiped spanned by two vectors in a two dimensional space is given by the absolute value of

\begin{dmath}\label{eqn:SimpleProducts2:660}
\begin{vmatrix}
   x_1 & x_2 \\
   y_1 & y_2
\end{vmatrix},
\end{dmath}

a factor that was also found in the coordinate expansion of the wedge product (\cref{eqn:SimpleProducts2:560}).
It is therefore natural to interpret the wedge product as an oriented area.
(cut)
In higher dimensonal spaces, the bivector factor not only encodes a sign for this area, but also its orientation in space.
The wedge product will be seen to encode that orientation without introducing a normal direction for the spanning plane, a nice feature in higher dimensional spaces where a single normal direction is ambiguous.

Because there are many possible pairs of generating vectors for any given bivector, any oriented area in a given plane with a specified area are all equally valid interpretations of a bivector.
(cut)

\section{areaOfParallelogram.tex}
   \makeproblem{Area of a parallelogram.}{problem:SimpleProducts2:areaofparallelogram}{
Show that the area of a parallelogram formed by the \R{2} vectors \( \Ba = (a_1, a_2) \) and \( \Bb = (b_1, b_2) \) is the absolute value of

\begin{equation*}
\begin{vmatrix}
a_1 & a_2 \\
b_1 & b_2
\end{vmatrix}.
\end{equation*}
} % problem

%\makeanswer{problem:SimpleProducts2:areaofparallelogram}{
%} % answer



\section{definitions.tex}
   %
% Copyright � 2016 Peeter Joot.  All Rights Reserved.
% Licenced as described in the file LICENSE under the root directory of this GIT repository.
%
A few new GA terms have been introduced in an ad-hoc fashion as required.  Here is a systematic exposition of some of the key definitions used to refer to the types of the geometric objects that will be encountered.

The grade of a scalar, vector, bivector, and trivector are 0, 1, 2, and 3 respectively.
(cut)

\section{gradeselectionProblems.tex}
   %
% Copyright © 2016 Peeter Joot.  All Rights Reserved.
% Licenced as described in the file LICENSE under the root directory of this GIT repository.
%














%\makeanswer{problem:vectorproduct:cyclicpermutationII}{
%FIXME: todo.
%} % answer
%





\section{introGAproblems.tex}
   %
% Copyright � 2016 Peeter Joot.  All Rights Reserved.
% Licenced as described in the file LICENSE under the root directory of this GIT repository.
%
%%{
%\input{../blogpost.tex}
%\renewcommand{\basename}{introGAproblems.tex}
%%\renewcommand{\dirname}{notes/phy1520/}
%\renewcommand{\dirname}{notes/ece1228-electromagnetic-theory/}
%%\newcommand{\dateintitle}{}
%%\newcommand{\keywords}{}
%
%\input{../peeter_prologue_print2.tex}
%
%\usepackage{peeters_layout_exercise}
%\usepackage{peeters_braket}
%\usepackage{peeters_figures}
%\usepackage{siunitx}
%%\usepackage{mhchem} % \ce{}
%%\usepackage{macros_bm} % \bcM
%%\usepackage{txfonts} % \ointclockwise
%
%\beginArtNoToc
%
%\generatetitle{XXX}
%%\chapter{XXX}
%%\label{chap:introGAproblems.tex}
%% \citep{sakurai2014modern} pr X.Y
%% \citep{pozar2009microwave}
%% \citep{qftLectureNotes}
%% \citep{doran2003gap}
%% \citep{jackson1975cew}
%% \citep{griffiths1999introduction}
%




%%}
%\EndArticle
%%\EndNoBibArticle

\section{multiplication.tex}
   
%%\input{../blogpost.tex}
%%\renewcommand{\basename}{multiplication}
%%%\renewcommand{\dirname}{notes/phy1520/}
%%\renewcommand{\dirname}{notes/ece1228-electromagnetic-theory/}
%%%\newcommand{\dateintitle}{}
%%%\newcommand{\keywords}{}
%%
%%\input{../peeter_prologue_print2.tex}
%%
%%\usepackage{peeters_layout_exercise}
%%\usepackage{peeters_braket}
%%\usepackage{peeters_figures}
%%\usepackage{siunitx}
%%%\usepackage{mhchem} % \ce{}
%%%\usepackage{macros_bm} % \bcM
%%%\usepackage{txfonts} % \ointclockwise
%%
%%\beginArtNoToc
%%
%%\generatetitle{Vector multiplication}
%%%\chapter{Vector multiplication}
%%%\label{chap:multiplication}
%%
Geometric Algebra defines a multiplication operation for vectors, forming a vector space spanned by all the possible vector products.  This algebra is described by the following small set of axioms

\makeaxiom{Associative multiplication.}{axiom:multiplication:associative}{

The product of any three vectors \(\Ba,\Bb,\Bc\) is associative.

\begin{equation*}\label{eqn:multiplication:160}
\Ba (\Bb \Bc)
= (\Ba \Bb) \Bc
= \Ba \Bb \Bc.
\end{equation*}
}

\makeaxiom{Linearity.}{axiom:multiplication:linear}{
Vector products are linear with respect to addition and subtraction.

\begin{dmath*}\label{eqn:multiplication:180}
\begin{aligned}
(\Ba + 3 \Bb \Bd) \Bc &= \Ba \Bb + 3 \Bb \Bd \Bc \\
\Ba (\Bb \Bd - 2 \Bc) &= \Ba \Bb \Bd - 2 \Ba \Bc.
\end{aligned}
\end{dmath*}
}

\makeaxiom{Contraction.}{axiom:multiplication:contraction}{

The square of a vector is the squared length of the vector.

\begin{dmath*}\label{eqn:multiplication:200}
\Ba^2 = \Norm{\Ba}^2.
\end{dmath*}

The notion of length here is metric dependent.  For the problems considered in these notes
it can be assumed that there is an orthonormal Euclidean basis, where the vector length is always positive.
For special relativistic calculations, also of interest in electrodynamics, but not the focus of these notes, the length of a (four-)vector may generally be negative or positive.
}

...

This contraction axiom, justified or not, has additional implications

\begin{dmath}\label{eqn:multiplication:80}
x^2
= \Bx^2
= (x \Be)(x \Be)
= x^2 \Be^2.
\end{dmath}



\section{multivector.tex}
   

%\makedefinition{multivector}{dfn:multivector:multivector}{
%A multivector is a sum of k-vectors.  The grades of any such summands may differ.
%} % definition

This last object, the multivector, assumes an addition operation that allows different k-vectors to be added, even if their grades differ.

Observe first that since a scalar multiple of the square of a vector is as scalar by the definition above,
any scalar is also a multivector.
For example, if \( \Be_1 \) is the unit vector along the x-axis and \( s \) is a scalar, then

\begin{equation}\label{eqn:multivector:20}
   x = s \Be_1^2 = s,
\end{equation}

is a multivector.
Since vectors (a product of one vector, or a scalar multiple thereof) is also a multivector, this
means that vectors are multivectors, and that ``wierd'' sums of scalars and vectors, such as

\begin{dmath}\label{eqn:multivector:40}
   x = 1 + \Be_1,
\end{dmath}

are also multivectors!  A quantity like

\begin{dmath}\label{eqn:multivector:45}
   x = 1 + \Be_1 + \Be_1 \Be_2 - \Be_1 \Be_2 \Be_3,
\end{dmath}

where \( \Be_k \) are the standard orthonormal basis vectors for \R{3} (unit vectors that are mutually perperpendicular), is also a multivector.
The product \( \Be_1 \Be_2 \) is a bivector, and represents a positively oriented unit magnitude area in the x-y plane, whereas \( - \Be_1 \Be_2 \Be_3 \) is a trivector, representing a negatively oriented unit volume (inwards normals).

%%\makedefinition{Scalar}{def:multiplication:scalar}{
%%   A (real) number with no implied direction.
%%}
%%
%%Examples of scalars are \( \pi, 3, -4 \), and \( 0 \).
%%
%%\makedefinition{Vector}{def:multiplication:vector}{
%%%\href{https://www.youtube.com/watch?v=bOIe0DIMbI8}{A quantity with direction and magnitude.}
%%\href{https://youtu.be/bOIe0DIMbI8?t=19}{A quantity with direction and magnitude.}
%%}
%%
%%In this book,
%%In order to express
%%\begin{dmath}\label{eqn:multivector:60}
%%\Bx = c_1 \Be_1 + c_2 \Be_2,
%%\end{dmath}
%%
%%where \( \Be_1 \) and \( \Be_2 \) are a pair of perpendicular vectors of length one along the x and y axis respectively, as illustrated in
%%
%%FIXME: figure.
%%These
%%
%%, as represented pictorially as an arrow
%%
%%
%%
%%\section{Vector space}
%%\section{Vector multiplication}
%%\section{Multivector}
%%
%%Geometric Algebra, or \boldTextAndIndex{GA} defines a multiplication operation for vectors.
%%GA also
%%generalizes the concept of a vector, introducing a new type of mathematical object, the multivector.
%%
%%
%%
%%In traditional vector algebra, a sum of a scalar and a vector, such as
%%
%%\begin{dmath}\label{eqn:multivector:80}
%%M = 1 + 2 \Be_1,
%%\end{dmath}
%%
%%is not considered meaningful.  This is
%%


\section{Junk}
Geometric Algebra (\textAndIndex{GA}) generalizes the concept of vector and a normed vector space.  This is done by introducing a vector multiplication operation into the mix, and a vector generalization called a \textAndIndex{multivector}.

The multivector is a hybrid object that may contain any sum of all or some of

\begin{enumerate}
   \item scalars, numeric quantities with magnitude and no direction,
   \item vectors (1-vectors), quantities with magnitude and direction,
   \item k-vectors, which are generalizatized line, area, volume and hypervolume elements, represent subspaces with orientation and magnitude.
\end{enumerate}

Scalars and vectors are assumed to be familiar, however, a sum of a scalar and vector is a new and arguably strange idea.  The k-vectors with \( k=2 \) and \( k = 3 \) are called bivectors and trivectors, and represent oriented planes and volumes in space respectively.
Bivectors, trivectors, and k-vectors will be defined later in a more precise fashion, as will orientation.  For now, orientation can be thought of algebraically as a sign, but physically may have an interpretation of sidedness, direction of a normal to the surface\footnote{In three dimensional spaces where a normal can be defined.}, or a rotational sense.

FIXME: orientation pictures here.

The vector multiplication operation is a new type of vector product.  The vector product is distinct from, but relatied to, the familiar dot or cross products in a way that will become clear.



\section{multivector2.tex}
   %
% Copyright © 2017 Peeter Joot.  All Rights Reserved.
% Licenced as described in the file LICENSE under the root directory of this GIT repository.
%
\subsection{Irreducible products}

Armed with the contraction axiom and \cref{eqn:multiplication:140} it is now possible to show how to put a multivector into an irreducible form.  As an example, consider

\begin{equation}\label{eqn:SimpleProducts:20}
M = \Be_3 \Be_3 + 2 \Be_1 \Be_2 \Be_1 + \Be_2 \Be_3 - 5 \Be_3 \Be_1 \Be_3 \Be_2 + \Be_4 \Be_1 \Be_4 \Be_2 \Be_3 + \Be_1 \Be_2 \Be_1 \Be_3 \Be_4 \Be_5.
\end{equation}

Application of the contraction axiom shows that the first term is a scalar

\begin{equation}\label{eqn:SimpleProducts:40}
\Be_3 \Be_3 = 1.
\end{equation}

The second term is a vector, as it is possible to reorder normal products (changing sign each time) and regroup terms to apply the contraction axiom, as follows

\begin{dmath}\label{eqn:SimpleProducts:60}
2 \Be_1 \Be_2 \Be_1
=
2 \Be_1 \lr{ \Be_2 \Be_1 }
=
2 \Be_1 \lr{ - \Be_1 \Be_2 }
=
-2 \Be_1 \Be_1 \Be_2
=
-2 \lr{ \Be_1 \Be_1 } \Be_2
=
-2 \Be_2.
\end{dmath}

The third term is a bivector and cannot be reduced further.  The fourth term is also a bivector

\begin{dmath}\label{eqn:SimpleProducts:80}
- 5 \Be_3 \Be_1 \Be_3 \Be_2
=
- 5 \lr{ \Be_3 \Be_1 } \Be_3 \Be_2
=
+ 5 \lr{ \Be_1 \Be_3 } \Be_3 \Be_2
=
+ 5 \Be_1 \lr{ \Be_3 \Be_3 } \Be_2
=
+ 5 \Be_1 \Be_2.
\end{dmath}

As the fifth term has repeated indexes, is is also reducible too

\begin{dmath}\label{eqn:SimpleProducts:100}
\Be_4 \Be_1 \Be_4 \Be_2 \Be_3
=
\lr{ \Be_4 \Be_1} \Be_4 \Be_2 \Be_3
=
-\lr{ \Be_1 \Be_4} \Be_4 \Be_2 \Be_3
=
- \Be_1 \lr{ \Be_4 \Be_4 } \Be_2 \Be_3
=
- \Be_1 \Be_2 \Be_3.
\end{dmath}

The reader should demonstrate that the final term has grade four, and can be reduced to \( -\Be_2 \Be_3 \Be_4 \Be_5 \).

\begin{dmath}\label{eqn:SimpleProducts:120}
M = 1 - 2 \Be_2  + \Be_2 \Be_3 + 5 \Be_1 \Be_2 - \Be_1 \Be_2 \Be_3 -\Be_2 \Be_3 \Be_4 \Be_5.
  = 1 - 2 \Be_2  + \Be_{23} + 5 \Be_{12} - \Be_{123} -\Be_{2345}.
\end{dmath}

\subsection{Mixed grade sums}
In traditional vector algebra, the
``weird'' sum of a scalar and vector is forbidden and undefined, but is explicitly allowed in GA.  For example,

\begin{dmath}\label{eqn:multivector:240}
1 + \Be_1,
\end{dmath}

is a simple mixed grade multivector.
Such mixed grade mathematical objects are not only well defined in GA, but are required to represent some vector products.  One of the simplest examples is the following vector product

\begin{dmath}\label{eqn:multivector:260}
\Be_1 ( \Be_1 + \Be_2 )
=
\Be_1 \Be_1 + \Be_1 \Be_2
=
\Be_1 \cdot \Be_1 + \Be_1 \Be_2
=
1 + \Be_1 \Be_2,
\end{dmath}

where the last step assumes the vector space is Euclidean.


\section{prerequisites.tex}
   
\section{Junk?}

\subsection{Problems}
%\makeproblem{Explicit squared norm}{problem:multivector:60}{
%   Given a coordinate representation of a vector with respect to a standard basis
%\begin{dmath}\label{eqn:multivector:240}
%   \Bx = \sum_{i = 1}^N x_i \Be_i,
%\end{dmath}
%
%show that the squared norm is
%\begin{dmath}\label{eqn:multivector:260}
%   \Norm{\Bx}^2 = \Bx \cdot \Bx = \sum_{i = 1}^N x_i^2 (\Be_i \cdot \Be_i).
%\end{dmath}
%
%Observe that for a Euclidean vector space this is the squared length in the Pythagorean sense.
%}
%
\makeproblem{Null vector}{problem:multivector:80}{
Given a two dimensional non-Euclidean vector space with basis elements satisfying
\( \gamma_0 \cdot \gamma_0 = 1 = -\gamma_1 \cdot \gamma_1 \), construct a vector that has a squared
norm of 0.  Such a vector is called a null vector.
%   \Bx = \gamma_0 + \gamma_1,
}


\subsection{basis, norm, ...}

%We will use a representation such as \( \Bv = x \Be_1 + y \Be_2 + z \Be_3 \) for such vectors, where the
%coordinates are always paired with their respective direction vectors, and will not use
%column vector of coordinates or tuples such as \( \Bv = (x, y, z)\).
%, \Bv = x \xcap + y \ycap + z \zcap, or \Bv = x \ahat_x + y \ahat_y + z \ahat_z.
\makedefinition{Coordinates.}{dfn:prerequisites:coordinates}{
%Given a basis \( B =
FIXME: define
} % definition

%\makedefinition{Basis and coordinates}{dfn:multivector:basis}{
%   If \( N \) is the dimension of a vector space \( V \), a set of \( N \) vectors \( B = \setlr{ \Ba_1, \Ba_2, \cdots , \Ba_N } \) is a basis for that vector space, if it is possible to form any vector \( \Bx \in V \) as a linear combination of those vectors \( \Ba_k \).  That is, there exists scalars \( c_k \) such that for any \( \Bx \in V \)
%
%\begin{equation*}
%   \Bx = \sum_{k = 1}^N c_k \Ba_k.
%\end{equation*}
%
%The numbers \( (c_1, c_2, \cdots, c_N ) \) are referred to as the coordinates of the vector \( \Bx \) with respect to the basis \( B \).
%}

\makedefinition{Standard basis, and dot product properties.}{dfn:multivector:standardbasis}{
   Any vector space \( V \) used in this book will be assumed to have been generated from a basis \( \setlr{ \Be_1, \Be_2, \cdots, \Be_N } \), associated with a dot product that has the properties

\begin{enumerate}
   \item \( \Be_i \cdot \Be_i = \pm 1 \).
   \item \( \Be_i \cdot \Be_j = 0 \) for any \( i \ne j \).
\end{enumerate}

Such a basis will called a standard basis.  When these dot products are always positive, the vector space is referred to as a Euclidean vector space.
}

\paragraph{FIXME: remove?}
There are many possible standard bases sets.  In \R{3}, it is conventional to refer to \( \Be_1, \Be_2, \Be_3 \) as the standard bases elements if these represent the directions of the x, y, and z directions respectively.  Unless otherwise noted \( \Be_k \) refers to the direction vector for the k-th direction in a standard basis for that space.
The only non-Euclidean vector space of interest in this book (for relativistic material), has a Minkowski dot product.  For such a space, the standard basis elements will be labelled \( \setlr{ \gamma_0, \gamma_1, \gamma_2, \gamma_3 } \), where for \( i \in [1,3] \), \( \gamma_0 \cdot \gamma_0 = \pm 1 = -\gamma_i \cdot \gamma_i \).  The positive sign convention will be used.

%GA requires the vector space to have an associated
%dot product \( \Bx \cdot \By \) that
%defines the notion of perpendicularity for the space.  We will want to extend the scalar multiplication operation of the vector
%space to complex numbers, but
%will not require a (complex) order dependent inner product \( \innerprod{\Bx}{\By} \) for our vector space.
%

\paragraph{The metric, length and normality.}

An abstract vector need not have an associated notion of length, nor a notion of perpendicularity (normality).
In abstract vector algebra, length and normality are provided by defining an associated dot product \(\Bx \cdot \By\), or inner product \(\innerprod{\Bx}{\By}\).
In GA, length and normality of two vectors are provided by a metric \(g(\Bx, \By)\).
Like the dot product where \( \Bx \cdot \By = \By \cdot \Bx\), this metric is independent of order, a property that is not generally required of the inner product.
However, unlike both the dot and inner products of abstract vector algebra, where \( \Bx \cdot \Bx \ge 0\), and \( \innerprod{\Bx}{\Bx} \ge 0\), the metric \(g(\Bx, \Bx)\) may be negative (i.e. for spacetime vectors).
If \(c \) is any real or complex number, the metric in GA is \( g(c \Bx, c \Bx) = c^2 g(\Bx, \Bx)\), unlike the inner product in complex spaces, where \( \innerprod{c \Bx}{c \Bx} = \Abs{c}^2 \innerprod{c \Bx}{c \Bx} \).
Effectively, this means that our underlying direction vectors are always real.

\subsection{Orientation}
We are familiar with the idea of an oriented line segment (a vector), a quantity that can be visualized as an arrow with direction and magnitude.
The idea of an oriented plane, volume, or hypervolume is probably less familiar.
An oriented plane segment, in addition to having a specific area and a direction in space, can be visualized as having a
circulation direction, or handedness.
In a three dimensional space, this circulation direction can be associated with one of the two possible normal directions for the plane.
An oriented volume, in addition to having a given magnitude, is considered to have an associated circulation direction along its surface.
In a three dimensional space, an oriented volume can be conceptualized as a volume with either an inwards or outwards normal.

\subsection{dot and metric original text}

Vectors are often represented with an implied basis, with tuples like \( \Bx = (x,y,z) \), or with column (or row) vectors like
\(
   \Bx =
\begin{bmatrix}
x \\
y \\
z
\end{bmatrix}
\).
The values \( x, y, z \) in these representations are called the coordinates of the vectors, but only have specific meaning once a direction and magnitude is associated with each coordinate (i.e. a basis is chosen).
In three dimensions, the simplest such basis choice (the standard basis), associates the respective coordinates with a set of mutually perpendicular (normal) directions.
This is conventionally a right handed triple of direction vectors of unit length, perhaps designated \( \xcap, \ycap, \zcap \) or \( \Be_1, \Be_2, \Be_3 \).

In GA, when working with coordinates, we generally prefer to make the basis explicit, so instead of writing a vector as a set of coordinates, these coordinates
will be explicitly paired with their associated basis vectors.
For example in \R{3} a vector with coordinates \( x, y, z \) will be written as

\begin{dmath}\label{eqn:prerequisites:280}
x \Be_1 + y \Be_2 + z \Be_3.
\end{dmath}

By convention, we understand that \( \Be_1, \Be_2, \Be_3 \) in \cref{eqn:prerequisites:280} are unit length vectors, and are all mutually perpendicular (orthonormal).
The vector space must be augmented with a dot product (or inner product) to provide a measure of length and normality.  

%\makedefinition{Inner product.}{dfn:prerequisites:innerproduct}{
%The inner product 
%} % definition

For \R{3}, the dot product satisfies the following conditions

\begin{equation}\label{eqn:prerequisites:320}
\Be_i \cdot \Be_j = \delta_{ij} \, \forall i, j \in [1,3],
\end{equation}

where \( \delta_{ij} \) is the Kronecker delta \( \delta_{ij} = 1 \) for \( i = j \) and \( \delta_{ij} = 0 \) for \( i \ne j \).
Specifying the action of the dot product on all the unit vectors, completely specifies the action of the dot product on any two vectors, provided one assumes that the dot product is a bilinear operator.
For example, given

\begin{dmath}\label{eqn:prerequisites:340}
\begin{aligned}
\Ba &= a_1 \Be_1 + a_2 \Be_2 + a_3 \Be_3 \\
\Bb &= b_1 \Be_1 + b_2 \Be_2 + b_3 \Be_3,
\end{aligned}
\end{dmath}

or \( \Ba = \sum_i a_i \Be_i, \Bb = \sum_j b_j \Be_j \), we recover the familiar coordinate description of the dot product

\begin{dmath}\label{eqn:prerequisites:360}
\Ba \cdot \Bb
=
\lr{ \sum_i a_i \Be_i } \cdot \lr{ \sum_j b_j \Be_j }
=
\sum_{i,j} a_i b_j \lr{ \Be_i \cdot \Be_j }
=
\sum_{i,j} a_i b_j \delta_{ij}
=
\sum_{i} a_i b_i.
\end{dmath}

Electromagnetism is intrinsically relativistic, and there will be circumstances where vectors with both space and time components are required.
In physics, these are called four-vectors, but we will call them spacetime vectors here to avoid confusion with \( k = 4 \) k-vectors.
Following \citep{doran2003gap}, the Dirac (matrix) notation will be used as the relativistic basis, so a spacetime vector might be written like

\begin{dmath}\label{eqn:prerequisites:300}
A = c t \gamma_0 + x \gamma_1 + y \gamma_2 + z \gamma_3.
\end{dmath}

It will be seen later that our spacetime vector representation has similar properties to Dirac matrices, but we need not refer to any specific matrix representation.

For spacetime vectors, we can also assume a dot product operation between the basis vectors.  For example, given two spacetime vectors

\begin{dmath}\label{eqn:prerequisites:380}
\begin{aligned}
A &= c t \gamma_0 + x \gamma_1 + y \gamma_2 + z \gamma_3 \\
B &= c t' \gamma_0 + x' \gamma_1 + y' \gamma_2 + z' \gamma_3,
\end{aligned}
\end{dmath}

if the action of a ``dot-product'' is known between all basis vectors \( \gamma_\mu, \mu \in [0,3] \), then it will be possible to compute the dot-product of any pair of four vectors as done above for the \R{3} example.  Special relativity constrains the properties of four-vector dot products, requiring the following of the four-vector basis

\begin{dmath}\label{eqn:prerequisites:400}
\left\{
\begin{array}{l l}
\gamma_\mu \cdot \gamma_\nu = 0 & \quad \mbox{ \( \mu \ne \nu ; \mu, \nu \in [0,3] \) } \\
\gamma_0 \cdot \gamma_0 = -\gamma_i \cdot \gamma_i = \pm 1 & \quad \mbox{ \( i \in [1,3] \) }
\end{array}
\right.
\end{dmath}

Strictly speaking, this is a specification of a metric, not a dot product, since this four vector dot product specification does not satisfy the positive definite property required by most dot product definitions (i.e. \( A \cdot A \ge 0 \)).
There is a sign ambiguity in the metric specification above.  The physics of relativity is independent of the sign convention used, but we will use the positive sign convention, consistent with field theory and most matrix representations of the Dirac matrices.
\footnote{In general relativitity, many authors will use the opposite sign convention.}

Stated explicitly, we use a metric where the basis vectors satisfy the following properties

\begin{dmath}\label{eqn:prerequisites:420}
\left\{
\begin{array}{l l}
\gamma_\mu \cdot \gamma_\nu = 0 & \quad \mbox{ \( \mu \ne \nu ; \mu, \nu \in [0,3] \) } \\
\gamma_i \cdot \gamma_i = -1& \quad \mbox{ \( i \in [1,3] \) } \\
\gamma_0 \cdot \gamma_0 = 1. &\\
\end{array}
\right.
\end{dmath}


%%%\makeproblem{}{problem:multivector:50}{
%%%The most general definition of an Euclidean norm satisfies all of the properties
%%%
%%%\begin{enumerate}
%%%   \item \( \Norm{\Bx} \ge 0 \), and \( \Norm{\Bx} = 0 \iff \Bx = 0 \).
%%%   \item \( \Norm{a \Bx} = \Abs{a} \Norm{\Bx} \).
%%%   \item \( \Norm{\Bx + \By} \le \Norm{\Bx} + \Norm{\By} \).
%%%\end{enumerate}
%%%
%%%If the coordinates of a vector with respect to the standard basis are \( x_i \) then show that the Euclidean norm defined in
%%%that the Pythagorean norm
%%%\begin{equation*}
%%%\Norm{\Bx}^2 = \sum_{i = 1}^N x_i^2,
%%%\end{equation*}
%%%
%%%satisfies these properties.
%%%} % problem
%%%

\section{projectionAndRejection.tex}
   
(cut)
The pythagorean property of these two vector components can also be checked.
Computing the squared length using \( \Norm{\By}^2 = \By \cdot \By = \By^2 \), the squared length of the projective component is

\begin{dmath}\label{eqn:SimpleProducts2:740}
\lr{ \lr{\Bx \cdot \ucap } \ucap }^2
=
\lr{\Bx \cdot \ucap }^2
=
(x_1 u_1 + x_2 u_2)^2
=
x_1^2 u_1^2 + x_2^2 u_2^2 + 2 x_1 x_2 u_1 u_2.
\end{dmath}

The squared length of the rejective component is
\begin{dmath}\label{eqn:SimpleProducts2:760}
\lr{ \lr{\Bx \wedge \ucap } \ucap }^2
=
-(\Bx \wedge \ucap) \ucap^2 (\Bx \wedge \ucap)
=
-
\lr{\begin{vmatrix}
   x_1 & x_2 \\
   u_1 & u_2
\end{vmatrix}}^2
(\Be_1 \Be_2)^2
=
x_1^2 u_2^2 + x_2^2 u_1^2 - 2 x_1 x_2 u_1 u_2.
\end{dmath}

Adding these together gives

\begin{dmath}\label{eqn:SimpleProducts2:780}
\lr{ \lr{\Bx \cdot \ucap } \ucap }^2 + \lr{ \lr{\Bx \wedge \ucap } \ucap }^2
=
x_1^2 u_1^2 + x_2^2 u_2^2
+x_1^2 u_2^2 + x_2^2 u_1^2
=
x_1^2 ( u_1^2 + u_2^2 )
+
x_2^2 ( u_1^2 + u_2^2 )
=
\Bx^2,
\end{dmath}

recovering the squared length of the vector as expected.
It is generally true in higher dimensions that the projection and rejection can be written as

\begin{dmath}\label{eqn:SimpleProducts2:800}
\begin{aligned}
\Proj_\ucap(\Bx) &= (\Bx \cdot \ucap) \ucap \\
\RejName_\ucap(\Bx) &= (\Bx \wedge \ucap) \ucap.
\end{aligned}
\end{dmath}

The Pythagorean aspect of this statement in higher degree spaces
will be demonstrated later in a coordinate free fashion after some additional identities have been derived.

The unit vector restriction defining the direction of projection and rejection can be relaxed in a compact fashion by introducing the vector \boldTextAndIndex{inverse}, which is always well defined and unique in a Euclidean space

\boxedEquation{eqn:SimpleProducts2:860}{
\inv{\Bu} \equiv \frac{\Bu}{\Bu^2}.
}

Now the projection and rejection onto the direction of \( \Bu \) are

\boxedEquation{eqn:SimpleProducts2:880}{
\begin{aligned}
\Proj_\Bu(\Bx) &= (\Bx \cdot \Bu) \inv{\Bu} \\
\RejName_\Bu(\Bx) &= (\Bx \wedge \Bu) \inv{\Bu}.
\end{aligned}
}

%\makelemma{\R{3} pseudoscalar commutation.}{dfn:projectionAndRejection:r3pcommutation}{
%The \R{3} pseudoscalar \( I = \Be_1 \Be_2 \Be_3 \) commutes with all \R{3} multivectors.
%} % lemma
%
%To prove this, it is sufficient to consider the commutation of \( I \) with each of the standard basis vectors \( \Be_1, \Be_2, \Be_3 \) (\cref{problem:projectionAndRejection:1160}).
%
Now the cross product form of the rejection equation can be determined

\section{syllabusAndContext.tex}
   \paragraph{Context for the project}

An ECE professor from Università di Perugia, Prof. Mauro Mongiardo, has reached out to me to collaborate on a book and papers related to applications of geometric algebra in electromagnetism, particularly focused on engineering applications in the frequency domain.  As discussed, I am interested in persuing this work for two reasons:

\begin{enumerate}
\item
It is intrinsicaly interesting to me, and I have a strong impression that there is a lot of potential for interesting applications.
\item
Doing this work in the context of an M.Eng project will help satisify the ECE graduation requirements, especially since the electromagnetics group course offerings in recent years have been very limited.
\end{enumerate}

I've read and written considerably about applications of geometric algebra outside of a university context.  That writing is scattered throughout the following notes compilations (and probably other locations)

\begin{itemize}
\item Exploring Physics with Geometric Algebra, Part I \citep{gabookI}
\item Exploring Physics with Geometric Algebra, Part II \citep{gabookII}
\item Classical Mechanics \citep{classicalmechanics}
\item Continuum Mechanics \citep{phy454}
\item Advanced Antenna Theorey \citep{ece1229}
\end{itemize}

Much of the research into geometric algebra applications to electromagnetism has been in the context of relativistic electromagnetism, where Maxwell's equations take a particularly simple form

\begin{dmath}\label{eqn:syllabusAndContext:n}
\grad F = \inv{\epsilon_0} J.
\end{dmath}

This consolidates the two relativisitic tensor relations for Maxwell's equations

\begin{dmath}\label{eqn:syllabusAndContext:n}
\begin{aligned}
\partial_\mu F^{\mu\nu} &= \inv{\epsilon_0} J^\nu \\
\epsilon^{\alpha\beta\gamma\kappa} \partial_\beta F_{\gamma\kappa} &= 0.
\end{aligned}
\end{dmath}

into a single multivector equation, where

\begin{dmath}\label{eqn:syllabusAndContext:n}
\begin{aligned}
F &= \BE + I c \BB \\
I &= \gamma_0 \gamma_1 \gamma_2 \gamma_3 \\
\BE &= \sum_{k = 1}^3 \gamma_k \gamma_0 E^k \\
\BB &= \sum_{k = 1}^3 \gamma_k \gamma_0 B^k \\
\BJ &= \sum_{k = 1}^3 \gamma_k \gamma_0 J^k \\
\grad &= \gamma^\mu \partial_\mu = \gamma^\mu \PD{x^\mu}{} \\
x^0 &= c t \\
\partial_0 &= \inv{c} \PD{t}{} \\
c &= 1/\sqrt{\mu\epsilon} \\
J &= \gamma_0 \lr{ c \rho - \BJ },
\end{aligned}
\end{dmath}

where \( \setlr{ \gamma_\mu } \) is a relativisitic four-vector basis satisfying \( (\gamma_0)^2 = 1 \), \( (\gamma^k)^2 = -1 \), and \( \gamma^\mu \gamma_\mu = 1 \).  In this context (referred to as the Space Time Algebra, or STA) all spatial vectors \((\BE, \BB, \BJ)\) are actually bivectors, as is the electromagnetic field \( F \).  This abstraction allows problems to be formulated 
without any explicit reference to either electric or magnetic fields, quantities that are observer dependent.  Lorentz boosts that translate from an observe frame to can be formulated as easily as rotations, which is especially powerful given that rotations in geometric algebra have such a compact representation.
This power comes with a level of abstraction that makes the subject impalatible for applications in engineering.
There is also a considerable learning curve for geometric algebra, and that learning curve is made still steeper by requiring the electromagnetic practitioner to also deal with the relativisitic abstractions.

Without introducing the abstractions of four vectors and non-Euclidean geometries required for the STA form of Maxwell's equation 

Part of what I'd like to accomplish with this project is to 
Engineering applications of electromagnetism keep time (or frequency) as a distinct quantity of interest, so some thought would be required to figure out how to most naturally express some of the electromagnetic concepts that have concise representation in relativistic geometric algebra where time is just one coordinate of a four vector.  
 (i.e. for personal non-academic research), however, it will still take a considerable amount of work to assemble these ideas into a coherent form

  This would also require a fair amount of literature review since some of the relevant material has probably also been presented in a piecemeal fashion.  

An example is the electromagnetic stress tensor, which in (relativistic) geometric algebra is just T(a) = -F a F/(2 epsilon_0), where a is a four vector.  Picking a timelike four vector 'a', produces the Poynting theorem relations, whereas picking spacelike four vectors 'a' produce the rest of the (lesser known) Poynting like conservation relations that bring in all 16 elements of the stress-energy tensor into the mix.  It is not obvious if there is corresponding natural (yet still concise) representation of the complete stress energy tensor representation when using the usual three-spatial+one-time coordinate geometric algebra representation of Maxwell's equations.


\paragraph{syllabus}

M.Eng project on Engineering applications of Geometric Algebra to engineering electromagnetism.

\section{vectorproduct.tex}
   %
% Copyright � 2016 Peeter Joot.  All Rights Reserved.
% Licenced as described in the file LICENSE under the root directory of this GIT repository.
%
%{
%\input{../blogpost.tex}
%\renewcommand{\basename}{vectorproduct}
%%\renewcommand{\dirname}{notes/phy1520/}
%\renewcommand{\dirname}{notes/ece1228-electromagnetic-theory/}
%%\newcommand{\dateintitle}{}
%%\newcommand{\keywords}{}
%
%\input{../peeter_prologue_print2.tex}
%
%\usepackage{peeters_layout_exercise}
%\usepackage{peeters_braket}
%\usepackage{peeters_figures}
%\usepackage{siunitx}
%%\usepackage{mhchem} % \ce{}
%%\usepackage{macros_bm} % \bcM
%%\usepackage{macros_qed} % \qedmarker
%%\usepackage{txfonts} % \ointclockwise
%
%\beginArtNoToc
%
%\generatetitle{XXX}
%%\chapter{XXX}
%%\label{chap:vectorproduct}
%
Given two vectors \( \Bx, \By \) the scalar grade of the vector product \( \Bx \By \) was shown (\cref{problem:gradeselection:RnDotProduct}) to be
\begin{equation}\label{eqn:vectorproduct:20}
\gpgradezero{ \Bx \By }
=
\sum_{i = 1}^N x_i y_i
=
\Bx \cdot \By.
\end{equation}

The grade two selection of this product was found (\cref{problem:gradeselection:vectorwedge}) to be

\begin{equation}\label{eqn:vectorproduct:40}
\gpgradetwo{ \Bx \By }
=
\sum_{i < j}
%(x_i y_j - x_j y_i)
\begin{vmatrix}
x_i & x_j \\
y_i & y_j
\end{vmatrix}
\Be_i \Be_j
=
\Bx \wedge \By
=
-\By \wedge \Bx.
\end{equation}

The reader should convince themself that the vector product \( \Bx \By \) has only even grades (0,2), and can therefore be expanded as

\begin{dmath}\label{eqn:vectorproduct:60}
\Bx \By
=
\gpgradezero{ \Bx \By }
+
\gpgradetwo{ \Bx \By },
\end{dmath}

or
\boxedEquation{eqn:vectorproduct:80}{
\Bx \By
=
\Bx \cdot \By
+
\Bx \wedge \By.
}

This is a fundamental and very useful relationship.  In these notes this is a consequence of the axioms and the generalized definitions of the dot and wedge products.  Some authors will use this to define the geometric product of two vectors.

Using \cref{problem:gradeselection:dotprod} and \cref{eqn:vectorproduct:80} it can be shown that the wedge product is an explicit antisymmetrized sum of vector products, just as the dot product is the symmetrized vector product sum

\boxedEquation{eqn:vectorproduct:300}{
\begin{aligned}
\Bx \cdot \By &= \inv{2} \lr{ \Bx \By + \By \Bx } \\
\Bx \wedge \By &= \inv{2} \lr{ \Bx \By - \By \Bx }
\end{aligned}
}

Some authors will use these as the respective definitions of the dot and wedge products.

The non-commutative nature of the vector product was one of the first observed consequences of the axioms.  The vector product is also not generally anticommutative, as was the case for normal vectors.  Rearranging \cref{eqn:vectorproduct:300} provides the general commutation identity for two vectors

%\begin{dmath}\label{eqn:vectorproduct:320}
\boxedEquation{eqn:vectorproduct:320}{
\By \Bx = 2 \Bx \cdot \By - \Bx \By.
}
%\end{dmath}

Observe that when the vectors are perpendicular, the strict anticommutation result follows.
This can be a handy tool for abstract multivector expression manipulation.

An additional, and incredibly useful, relationship follows from \cref{eqn:vectorproduct:80} for \R{3} (\cref{problem:gradeselection:WedgeRelationshipToCrossProduct})

\boxedEquation{eqn:vectorproduct:100}{
\Bx \By
=
\Bx \cdot \By
+
I
(\Bx \cross \By).
}

This is the GA equivalent of the Pauli relationship \cref{eqn:GAmotivation:120} that will be familiar to a student of quantum spin states.

The ability to combine dot and cross product relationships into a single multivector equation is not just a theoretical nicety.  This is also one of the primary reasons that GA is so applicable to the study of electromagnetism.   To illustrate this, and provide a hint of things to come, consider the GA formulation of the electrostatic and magnetostatic Maxwell equations.

(cut: maxwell's statics example)

The dot plus wedge product components of the vector product have a geometrical interpretation.  To understand this, consider the components of a vector \( \By \) onto the direction of \( \Bx \) and the perpendicular.  The projection component is

\begin{dmath}\label{eqn:vectorproduct:420}
\Proj_\Bx \By = \xcap \lr{ \xcap \cdot \By },
\end{dmath}

and the rejection (the component of \( \By \) perpendicular to \( \Bx \)), is
\begin{dmath}\label{eqn:vectorproduct:440}
\RejName_\Bx \By
=
\By - \xcap \lr{ \xcap \cdot \By }
=
\Norm{\By} \lr{ \ycap - \xcap \lr{ \xcap \cdot \By } }
=
\Norm{\By} \xcap \lr{ \xcap \ycap - \xcap \cdot \By }
=
\Norm{\By} \xcap \lr{ \xcap \wedge \ycap }
=
\xcap \lr{ \xcap \wedge \By }.
\end{dmath}

FIXME: review this and see what portions if any to keep, now that this is treated in the 2D intro section.
%An example is plotted in \cref{fig:projectionAndRejection:projectionAndRejectionFig1}.
%
%\imageFigure{../figures/GAelectrodynamics/projectionAndRejectionFig1}{Projection and rejection illustrated.}{fig:projectionAndRejection:projectionAndRejectionFig1}{0.45}
%

The magnitudes of \( \xcap \lr{ \xcap \cdot \ycap } \), and \( \xcap \lr{ \xcap \wedge \ycap } \) are neccessarily the cosine and sines of the angle between \( \Bx \) and \( \By \), regardless of the dimension of the underlying vector space.  Those respective magnitudes are

\begin{dmath}\label{eqn:vectorproduct:460}
\begin{aligned}
\Norm{ \xcap \lr{ \xcap \cdot \ycap } }^2 &= \lr{ \xcap \cdot \ycap }^2 \\
\Norm{ \xcap \lr{ \xcap \wedge \ycap } }^2 &= -\lr{ \xcap \wedge \ycap }^2,
\end{aligned}
\end{dmath}

which allows an identification
\begin{dmath}\label{eqn:vectorproduct:500}
\begin{aligned}
\cos\theta &= \xcap \cdot \ycap \\
\sin\theta &= \Norm{\xcap \wedge \ycap},
\end{aligned}
\end{dmath}

where \( \Norm{\xcap \wedge \ycap} = \sqrt{ -\lr{\xcap \wedge \ycap}^2 } \).

It is now possible to express the product of vectors in a trigonometric or exponential form

\begin{dmath}\label{eqn:vectorproduct:480}
\xcap \ycap
= \xcap \cdot \ycap
+ \xcap \wedge \ycap
=
\xcap \cdot \ycap
+ \frac{\xcap \wedge \ycap}{\Norm{\xcap \wedge \ycap}} \Norm{\xcap \wedge \ycap}
=
\cos\theta
+ \frac{\xcap \wedge \ycap}{\Norm{\xcap \wedge \ycap}} \sin\theta,
\end{dmath}

or

\boxedEquation{eqn:vectorproduct:520}{
\Bx \By
=
\Norm{\Bx}
\Norm{\By}
\exp\lr{ \frac{\xcap \wedge \ycap}{\Norm{\xcap \wedge \ycap}} \theta }.
}

The interpretation of this is that the product of two vectors produces a rotation operator that acts in the plane spanned by these vectors, but also scales any such rotated vector from this plane by the product of the magnitudes of the vector product factors.  When those vectors are unit vectors, the vector product is a non-scaling rotation operator

\begin{dmath}\label{eqn:vectorproduct:560}
\xcap \ycap
=
\exp\lr{ \frac{\xcap \wedge \ycap}{\Norm{\xcap \wedge \ycap}} \theta },
\end{dmath}

that (when applied from the right) rotates any vector in \( \Span{\xcap, \ycap} \) by \( \theta \) radians in the direction of shortest rotation from \( \xcap \) to \( \ycap \), and when applied from the left rotates by \( -\theta \).

In particular, if the unit vectors are perpendicular, the rotation operator is
\begin{dmath}\label{eqn:vectorproduct:580}
R(\theta)
=
\exp\lr{ \xcap \ycap \theta }.
\end{dmath}

For \R{3} the wedge product in \cref{eqn:vectorproduct:520} can be expressed as a cross product

\begin{equation}\label{eqn:vectorproduct:540}
\frac{\xcap \wedge \ycap}{\Norm{\xcap \wedge \ycap}}
=
I \frac{\xcap \cross \ycap}{\Norm{\xcap \cross \ycap}}
=
I \ncap,
\end{equation}

This allows the \R{3} vector product to be written as

\begin{equation}\label{eqn:vectorproduct:600}
\Bx \By
=
\Norm{\Bx}
\Norm{\By}
\exp\lr{ I \ncap \theta }.
\end{equation}

In this form it is particularly easy to verify that the factor \( I \ncap \),
the dual of the normal representing the plane of rotation from \( \Bx \) to \( \By \), acts as an imaginary

\begin{dmath}\label{eqn:vectorproduct:400}
(I \ncap)^2
=
(I \ncap) (I \ncap)
=
I^2 \ncap^2
=
(-1)(1)
=
-1.
\end{dmath}

Observe the similarity between this and the complex inner product \( z w^\conj = r \rho e^{i(\theta-\alpha)} \) for the complex numbers of \cref{eqn:GAmotivation:200}.  The primary difference is that the GA imaginary factor also has a spatial orientation that the complex imaginary does not.

%}
%\EndNoBibArticle

\section{vectorproductProblems.tex}
   

%\makeproblem{Cyclic permutation}{problem:vectorproduct:cyclicpermutationII}{
%Any cyclic permutation of the vectors within a grade zero selection leaves the result unchanged, such as
%
%\begin{dmath}\label{eqn:vectorproduct:280}
%\gpgradezero{ \Bx \By \Bz } = \gpgradezero{ \Bz \Bx \By }.
%\end{dmath}
%
%Show that this is the case.
%} % problem

\section{wedgeProductArea.tex}
   
A simple, but inelegant, way to prove this is to specify a coordinate system for which \( \Ba, \Bb \) both lie in the \( x,y \) plane.  Then \( \Ba \wedge \Bb = \alpha i \) for some \( \alpha \), and the anticommutation part of the theorem follows from
%the \R{2} result
\cref{eqn:SimpleProducts2:1760}.  For the normal commutation part of the theorem, pick any vector normal to the \(x, y\) plane, say \( \Be_3\), for which we have

\begin{dmath}\label{eqn:SimpleProducts2:1780}
\Be_3  (\Ba \wedge \Bb)
=
\Be_3 \alpha i
=
\alpha \Be_3 \Be_1 \Be_2
=
\alpha (-\Be_1 \Be_3) \Be_2
=
-\alpha \Be_1 (\Be_3 \Be_2)
=
-\alpha \Be_1 (-\Be_2 \Be_3)
= (\Ba \wedge \Bb) \Be_3.
\end{dmath}

In dimensions with more normals, say \( \Be_4, \cdots \), the steps of \cref{eqn:SimpleProducts2:1780} can be repeated.  The general normal commuation result follows by superposition.

\section{left2dimaginarymultiplication.tex}
   %
% Copyright © 2017 Peeter Joot.  All Rights Reserved.
% Licenced as described in the file LICENSE under the root directory of this GIT repository.
%
\makeproblem{2D left pseudoscalar multiplication}{problem:left2dimaginarymultiplication:1}{

Compute the coordinate representation of an arbitrary 2D vector, and its clockwise rotation, and show that left multiplication by the 2D pseudoscalar produces the same result.
} % problem

\makeanswer{problem:left2dimaginarymultiplication:1}{

The rotated coordinate vector is
\begin{dmath}\label{eqn:left2dimaginarymultiplication:20}
\Bx'
=
\begin{bmatrix}
   0 & 1 \\
   -1 & 0 \\
\end{bmatrix}
\begin{bmatrix}
   \cos\theta \\
   \sin\theta \\
\end{bmatrix}
=
\rho
\begin{bmatrix}
   \sin\theta \\
   -\cos\theta
\end{bmatrix}.
\end{dmath}

This
compares identically to left pseudoscalar product with the standard basis representation of the same vector

\begin{dmath}\label{eqn:left2dimaginarymultiplication:40}
\Bx'
= \Be_1 \Be_2 \rho \lr{ \Be_1 \cos\theta + \Be_2 \sin\theta } \Be_1 \Be_2
= \rho \lr{ -\Be_2 \cos\theta + \Be_1 \sin\theta },
\end{dmath}

} % answer

\section{orientation.tex}
   %
% Copyright © 2017 Peeter Joot.  All Rights Reserved.
% Licenced as described in the file LICENSE under the root directory of this GIT repository.
%
\subsection{Orientation.  figure out where to put this}
Geometric algebra provides a mathematical representation for geometrical objects of each dimension in the space.
In a three dimensional space, there are representations for all of

\begin{itemize}
\item
oriented (signed) points with magnitude
\item
oriented line segments,
\item
oriented planes,
\item
oriented volumes,
\end{itemize}

and in higher dimensional spaces, it will be possible to represent higher dimensonal oriented hypervolumes.


\section{projectnotes.tex}
   \paragraph{Context for the project}

An ECE professor from Universit� di Perugia, Prof. Mauro Mongiardo, has reached out to me to collaborate on a book and papers related to applications of geometric algebra (GA) in electromagnetism, particularly focused on engineering applications in the frequency domain.  As discussed, I am interested in perusing this work for two reasons:

\begin{enumerate}
\item
It is intrinsically interesting to me, and I have a strong impression that there is a great deal of potentially interesting engineering applications.
\item
Doing this work in the context of an M.Eng project will help satisfy the ECE graduation requirements for the M.Eng degree program I am enrolled in.  This is especially true given that the electromagnetics group course offerings in recent years has been particularly limited.
\end{enumerate}

I've read and written considerably about applications of geometric algebra outside of a university context.  That writing is scattered throughout the following notes compilations (and probably other locations)

\begin{itemize}
\item Exploring Physics with Geometric Algebra, Part I \citep{gabookI}
\item Exploring Physics with Geometric Algebra, Part II \citep{gabookII}
\item Classical Mechanics \citep{classicalmechanics}
\item Continuum Mechanics \citep{phy454}
\item Advanced Antenna Theory \citep{ece1229}
\end{itemize}

Much of the research into geometric algebra applications to electromagnetism has been in the context of relativistic electromagnetism, where Maxwell's equations take a particularly simple form

\begin{dmath}\label{eqn:projectnotes:20}
\grad F = \inv{\epsilon} J.
\end{dmath}

It can be shown that this consolidates the standard pair of relativistic tensor relations for Maxwell's equations

\begin{dmath}\label{eqn:projectnotes:40}
\begin{aligned}
\partial_\mu F^{\mu\nu} &= \inv{\epsilon} J^\nu \\
\epsilon^{\alpha\beta\gamma\kappa} \partial_\beta F_{\gamma\kappa} &= 0.
\end{aligned}
\end{dmath}

into a single multivector equation.  There is a lot that is left unspecified in the relation above.  A more complete statement has to also define the operators, fields and sources, which are

\begin{dmath}\label{eqn:projectnotes:60}
\begin{aligned}
F &= \BE + I c \BB \\
I &= \gamma_0 \gamma_1 \gamma_2 \gamma_3 \\
\BE &= \sum_{k = 1}^3 \gamma_k \gamma_0 E^k \\
\BB &= \sum_{k = 1}^3 \gamma_k \gamma_0 B^k \\
\BJ &= \sum_{k = 1}^3 \gamma_k \gamma_0 J^k \\
\grad &= \gamma^\mu \partial_\mu = \gamma^\mu \PD{x^\mu}{} \\
x^0 &= c t \\
\partial_0 &= \inv{c} \PD{t}{} \\
c &= 1/\sqrt{\mu\epsilon} \\
J &= \gamma_0 \lr{ c \rho - \BJ },
\end{aligned}
\end{dmath}

where \( \setlr{ \gamma_\mu } \) is a relativistic four-vector basis satisfying \( (\gamma_0)^2 = 1 \), \( (\gamma^k)^2 = -1 \), and \( \gamma^\mu \gamma_\mu = 1 \).  The geometric algebra over this Minkowski (or Dirac) basis is referred to as the Space Time Algebra, or STA \citep{doran2003gap}.  In the STA representation, all spatial vectors \((\BE, \BB, \BJ)\) are represented as bivectors, as is the electromagnetic field \( F \).  The STA form of Maxwell's equation allows problems to be formulated without any explicit reference to either electric or magnetic fields, quantities that are observer dependent.  Lorentz boosts that translate from an observe frame to can be formulated as easily as rotations, which is especially powerful given that rotations in geometric algebra have such a compact representation.
This power comes with a level of abstraction that makes the subject impalpable for applications in engineering.

There is also a considerable learning curve for geometric algebra, and that learning curve is made still steeper by requiring the electromagnetic practitioner to also deal with the relativistic abstractions.

\paragraph{Maxwell's equation in a Euclidean basis}

Maxwell's equation can also be expressed in a compact geometric algebra multivector equation, without the use of
four vectors and non-Euclidean geometries, namely

\begin{dmath}\label{eqn:projectnotes:80}
\lr{ \inv{c} \PD{t}{} + \spacegrad } F = \inv{\epsilon} J
\end{dmath}

where
\begin{dmath}\label{eqn:projectnotes:120}
\begin{aligned}
F &= \BE + I c \BB \\
I &= \Be_1 \Be_2 \Be_3 \\
c &= 1/\sqrt{\mu\epsilon} \\
J &= c \rho - \BJ.
\end{aligned}
\end{dmath}

A choice of a fixed observer frame fixes the representation of the four-gradient, expressed here as a multivector operator, but result is otherwise identical to the representation of Maxwell's equation \cref{eqn:projectnotes:20}.
It is straightforward to show that this representation is equivalent to the normal vectoral form of Maxwell's equations

\begin{dmath}\label{eqn:projectnotes:100}
\begin{aligned}
\spacegrad \cdot \BE &= \inv{\epsilon} \rho \\
\spacegrad \cdot \BB &= 0 \\
\spacegrad \cross \BE &= -\PD{t}{\BB} \\
\spacegrad \cross \BB &= \mu \lr{ \BJ + \epsilon \PD{t}{\BE} }.
\end{aligned}
\end{dmath}

While the fixed observer frame GA representation of Maxwell's equation \cref{eqn:projectnotes:80} is only subtly different from the STA representation, I believe that this representation is preferable to the study of electromagnetism with respect to engineering applications.  Not having to deal with non-Euclidean geometries and four vectors should considerably reduce the learning curve required to exploit the compact GA representation in real world applications.  A compact representation alone is clearly not the only desirable attribute, for if that were the case, engineers would work exclusively with the tensor form of Maxwell's equations \cref{eqn:projectnotes:40}.


As engineers, having time as an independent variable, and an assumption that the geometry we have to deal with is Euclidean, are definite prerequisites!

\paragraph{Suggested Syllabus topics}

M.Eng project on Engineering applications of Geometric Algebra to engineering electromagnetism.

\begin{itemize}
\item Literature search and summary of non-relativistic treatments of electromagnetism in the formalism of geometric algebra.
\item Express the fundamentals of electromagnetism in a fashion that is natural using GA, while also attempting to present that material in a way that does not overwhelm the student with excessive GA theorems.
\item Attempt to determine what the minimal amount of GA theory that must be presented to a new student that will allow the student to focus to be on applications of electromagnetism.
\item Explore topics that have natural expression in relativistic GA, and determine where possible the most natural expression of those relationships in a Euclidean and explicit time formalism.  For example the energy momentum relationship in STA form is
that includes the continuum Lorentz force and Poynting relationships as special cases has an STA form

\begin{dmath}\label{eqn:projectnotes:140}
\begin{aligned}
\grad \cdot T(a) &= \inv{\epsilon c} \gpgradezero{ F a J } \\
T(a) &= -\inv{2} F a F,
\end{aligned}
\end{dmath}

where \( a = a^\mu \gamma_\mu \) is a four vector.  The stress-energy tensor, is usually expressed as the considerably more complex form

\begin{dmath}\label{eqn:projectnotes:160}
T^{\mu\nu} = \frac{1}{\mu_0} \lr{
F^{\mu \alpha}F^\nu{}_{\alpha} - \frac{1}{4} \eta^{\mu\nu}F_{\alpha\beta} F^{\alpha\beta}
}.
\end{dmath}

Note that the Poynting theorem can be recovered from \cref{eqn:projectnotes:140} using a timelike vector such as \( a = \gamma_0 \), whereas the continuum Lorentz force relationships follow by selecting values of \( a \) that are spacelike.

Is there a natural representation Euclidean representation of the energy momentum relationships that is useful for engineering applications, while still highlighting the underlying connections between the Poynting and Lorentz relationships as aspects of a single relativistic concept?
\item Investigate electromagnetic applications of geometric algebra to topics that have a natural geometric bias that is hard to formulate in traditional vector algebra.  One possible example of such an application could be to exploit the ability to simply express rotations in GA for constructions such as Bessel beams.
\item
Because many computational tools exist for standard vector techniques, engineering adoption and exploitation of GA would be facilitated by also enhancing existing CAS systems with support for the underlying product and selection operations.
Pauli matrices allow a compact representation of GA objects and operators (as do Dirac matrices for STA).  This could provide a mechanism for implementing symbolic GA computation engines with existing tools, with potentially less effort then required to implement symbolic GA CAS systems for the general N dimensional non-Euclidean spaces of litte interest to engineering applications.  It would be worthwhile to explore or implement symbolic GA CAS packages for some subset of the languages that may find use in engineering applications:
\begin{itemize}
\item Mathematica.
\item SymPy using a Python or Julia front end.
\item Maxima.
\end{itemize}
\end{itemize}

\section{r3vectorspaceproblem.tex}
   %
% Copyright © 2017 Peeter Joot.  All Rights Reserved.
% Licenced as described in the file LICENSE under the root directory of this GIT repository.
%
\makeproblem{\R{3}}{problem:prerequisites:R3}{
Define \R{3} as the set of triples \( \setlr{ (x_1, x_2, x_3) | x_i \in \bbR } \).  Given
\( \Bx = (x_1, x_2, x_3) \in \bbR^3 \),
\( \By = (y_1, y_2, y_3) \in \bbR^3 \), and \( a \in \bbR \), then addition and multiplication operations can be defined respectively as

\begin{equation*}
\begin{aligned}
\Bx + \By &\equiv (x_1 + y_1, x_2 + y_2, x_3 + y_3) \\
a \Bx &\equiv (a x_1 , a x_2 , a x_3 ).
\end{aligned}
\end{equation*}

Show that \R{3} is a vector space.
} % problem



\section{paulivectorspace.tex}
   %
% Copyright © 2017 Peeter Joot.  All Rights Reserved.
% Licenced as described in the file LICENSE under the root directory of this GIT repository.
%
\makeproblem{Pauli matrices.}{problem:prerequisites:20}{
The Pauli matrices are defined as

\begin{equation}\label{eqn:prerequisites:160}
   \sigma_1 = \PauliX,\quad
   \sigma_2 = \PauliY,\quad
   \sigma_3 = \PauliZ.
\end{equation}

Given any scalars \( a, b, c \in \bbR \), show that the set \( V = \setlr{ a \sigma_1 + b \sigma_2 + c \sigma_3 } \) is a vector space with respect to the operations of matrix addition and multiplication, and
determine the form of the zero and identity elements.

% FIXME: make this a problem at at location where it makes sense:
%\makesubproblem{}{problem:prerequisites:20:b}
%
%Show that \( \sigma_k^2 = I \), where \( I \) is the 2x2 identity matrix, and that \( \sigma_k \sigma_j = -\sigma_k \sigma_j \) for all \( k \ne j \).
%
%\makesubproblem{}{problem:prerequisites:20:c}
%
%Using the results of \partref{problem:prerequisites:20:b}, show that
%\( \lr{ a \sigma_x + b \sigma_y + c \sigma_z }^2 = (a^2 + b^2 + c^2) I \), where \( I \) is the 2x2 identity matrix.
%%This shows that the Pauli matrices are an example \R{3} basis for which the contraction axiom is built right into the representation.
} % problem


\section{functionvectorspace.tex}
   %
% Copyright © 2017 Peeter Joot.  All Rights Reserved.
% Licenced as described in the file LICENSE under the root directory of this GIT repository.
%
%\makeproblem{Function space.}{problem:prerequisites:30}{
%   Given real functions \( f(x) = e^x \) and \( g(x) = x^2 \), and scalars \( a,b \in \bbR \) determine whether the set
%   \( V = \setlr{ a f(x) + b g(x) } \) is a vector space, and if so,
%determine the form of the zero and identity elements.
%} % problem

\makeproblem{Polynomial vector space.}{problem:prerequisites:35}{
  Show that the set of N'th degree polynomials \( V = \setlr{ \sum_{k=0}^N a_k x^k \mid a_k \in \bbR } \) is a vector space.
} % problem

\section{poynting.tex}
   %
% Copyright © 2016 Peeter Joot.  All Rights Reserved.
% Licenced as described in the file LICENSE under the root directory of this GIT repository.
%
\subsection{Poynting theorem}
\index{Poynting theorem}

Poynting's theorem describes the relationship between the flux of energy through a surface bounding a volume.
The theorem follows from computing the divergence of the Poynting vector \( \BS = \BE \cross \BH \).
In terms of \( \BE \) and \( \BH \) the Poynting vector can be written in dual form as a dot product

\begin{equation}\label{eqn:maxwellsEquations:780}
\BE \cross \BH
=
\gpgradeone{ I (\BH \wedge \BE) }
=
\gpgradeone{ I \BH \BE }
=
(I \BH) \cdot \BE.
\end{equation}

Similarly, the Poynting divergence is most compactly expressed as a scalar selection operation

\begin{equation}\label{eqn:maxwellsEquations:640}
\spacegrad \cdot \lr{ \BE \cross \BH }
=
\gpgradezero{ \spacegrad I \lr{ \BH \wedge \BE } }
=
\gpgradezero{ \spacegrad I \BH \BE }.
\end{equation}

Here the gradient is acting on everything to the right, however, allowing the gradient to act bidirectionally, and employing the
the flexibility to use cyclic permutation within a scalar selection
(i.e. \(\gpgradezero{ABC} = \gpgradezero{CAB}\))
, allows for the easy application of the chain rule

\begin{dmath}\label{eqn:maxwellsEquations:760}
\gpgradezero{ \spacegrad I \BH \BE }
=
\gpgradezero{ \BE \lrspacegrad I \BH }
=
\gpgradezero{ (\BE \lspacegrad) I \BH }
+\gpgradezero{ \BE (\rspacegrad I \BH) }
\end{dmath}

Explicit left and right acting gradients are required because the gradient operator does not commute with the vector fields.

The gradient action on \( I \BH \) (from the left) is given by
\cref{eqn:maxwellsEquations:380}.
The right acting gradient action on \( \BE \) is given by reversing all the products in
%\spacegrad \BE = \inv{\epsilon} \rho - I \BM - \mu \PD{t}{(I\BH)}
\cref{eqn:maxwellsEquations:360} (in particular noting that \( I^\dagger = -I \) )

\begin{dmath}\label{eqn:maxwellsEquations:660}
%I \BH \lspacegrad = \frac{I \rho_\txtm}{\mu} + \BJ + \epsilon \PD{t}{\BE}.
\BE \lspacegrad = \inv{\epsilon} \rho + I \BM + \mu \PD{t}{(I\BH)}.
\end{dmath}

This gives
\begin{dmath}\label{eqn:maxwellsEquations:680}
\spacegrad \cdot \BS
=
\gpgradezero
{
\lr{ \inv{\epsilon} \rho + I \BM + \mu \PD{t}{(I\BH)} } I \BH
+
\BE
\lr
{
\frac{I \rho_\txtm}{\mu} - \BJ - \epsilon \PD{t}{\BE}
}
}
,
\end{dmath}

or
%\begin{dmath}\label{eqn:maxwellsEquations:700}
\boxedEquation{eqn:maxwellsEquations:720}{
0 =
\spacegrad \cdot \BS
+
\BH \cdot \BM + \BJ \cdot \BE
+ \PD{t}{\BB} \cdot \BH
+ \PD{t}{\BD} \cdot \BE.
}
%\end{dmath}

The sum of the last two terms is the time rate of change of the energy density.
To illustrate this consider the change of energy density through a volume with neither electric nor magnetic current sources in that region of space

\begin{dmath}\label{eqn:maxwellsEquations:740}
\PD{t}{} \int_V
\inv{2} dV \lr{
\BB \cdot \BH
+ \BD \cdot \BE
}
=
-\int_{\partial V} dA \ncap \cdot \BS.
\end{dmath}

Here \( \ncap \) is the outward normal, so if the energy contained in the volume is decreasing, then \( \BS \) must represent the energy per unit area that leaves the volume.
The direction of the Poynting vector is the direction that the energy is leaving the volume.
Only the components of the Poynting vector that are colinear with the surface normal will result in energy leaving or entering the volume.


\subsection{poyntingComplexPower.tex}
   %
% Copyright © 2016 Peeter Joot.  All Rights Reserved.
% Licenced as described in the file LICENSE under the root directory of this GIT repository.
%
\subsection{Complex power}
\index{complex power}

Let \( \BE \) and \( \BH \) be complex phasor (time harmonic) representations of the fields so that

\begin{dmath}\label{eqn:maxwellsEquations:900}
\begin{aligned}
\BE(\Bx, t) &= \Real\lr{ \BE e^{j \omega t} } \\
\BH(\Bx, t) &= \Real\lr{ \BH e^{j \omega t} }
\end{aligned}
\end{dmath}

The time harmonic expansion of the Poynting vector is

\begin{dmath}\label{eqn:maxwellsEquations:820}
\BS
= \lr{ I \BH(\Bx, t)} \cdot \BE(\Bx, t)
=
\lr{ I \Real \lr{ \BH e^{j \omega t}} } \cdot
\Real \lr{ \BE e^{j \omega t} }
= \inv{4} \gpgradeone{
I \lr{
\BH e^{j \omega t}
+ \BH^\conj e^{-j \omega t}
}
\lr{ \BE e^{j \omega t} + \BE^\conj e^{-j \omega t} }
}
=
\inv{4}
\lr{
(I \BH^\conj) \cdot \BE
+(I \BH) \cdot \BE^\conj
+
(I \BH) \cdot \BE e^{ 2 j \omega t }
+
(I \BH^\conj) \cdot \BE^\conj e^{ 2 j \omega t }
}.
\end{dmath}

This shows that the time harmonic representation of the Poynting vector is

\begin{dmath}\label{eqn:maxwellsEquations:840}
\BS = \inv{2} \Real \lr{ (I \BH^\conj) \cdot \BE + (I \BH) \cdot \BE e^{ 2 j \omega t } }.
\end{dmath}

This motivates the definition of a complex Poynting vector

\begin{dmath}\label{eqn:maxwellsEquations:860}
\bcS = \inv{2} (I \BH^\conj) \cdot \BE.
\end{dmath}

Note that this is conventionally written as \( \BS = (\ifrac{1}{2}) \BE \cross \BH^\conj \).
Observe that on average (over a given period), the \( e^{2 j \omega t} \) component of the real Poynting vector has no contribution to the power flux

\begin{dmath}\label{eqn:maxwellsEquations:880}
\inv{T} \int_0^T dt \BS = \Real \bcS.
\end{dmath}

% orphaned:
\subsection{greensFunctionLaplacian.tex}
   %
% Copyright � 2016 Peeter Joot.  All Rights Reserved.
% Licenced as described in the file LICENSE under the root directory of this GIT repository.
%
%{
\label{chap:greensFunctionLaplacian}

The first thing to show is that \( \spacegrad^2 1/r = 0 \) everywhere that \( r \ne 0 \).  Note that

\begin{dmath}\label{eqn:greensFunctionLaplacian:40}
\spacegrad r^{-n}
=
\spacegrad \lr{\Br^2}^{-n/2}
=
-\sum_{k,m} \frac{n}{2} \frac{\Be_k \partial_k x_m x_m}{r^{n + 2}}
=
-\sum_{k,m} \frac{n}{2} \frac{2 \Be_k \delta_{km} x_m}{r^{n + 2}}
=
-n \frac{\Br}{r^{n+2}}.
\end{dmath}

% switch to R{3} here!
The Laplacian is
\begin{dmath}\label{eqn:greensFunctionLaplacian:60}
\spacegrad^2 \inv{r}
=
\spacegrad \cdot \spacegrad \inv{r}
=
-\spacegrad \cdot
\frac{\Br}{r^{3}}
=
-\frac{\spacegrad \cdot \Br}{r^3}
-
\lr{ \spacegrad \inv{r^3} } \cdot \Br
=
-\frac{3}{r^3}
-
\lr{ -3 \frac{\Br}{r^5}} \cdot \Br
=
-\frac{3}{r^3}
+
3 \frac{ r^2}{r^5}
=
0,
\end{dmath}

provided \( \Br \ne 0 \), an expected property of the delta function.

To complete the proof, we have to show that this Laplacian has the desired filtering effect under convolution

\begin{dmath}\label{eqn:greensFunctionLaplacian:80}
\int dV' f(\Br') \delta^3(\Br' -\Br) = f(\Br).
\end{dmath}

\index{delta function}
Inserting the assumed delta function representation we have

\begin{dmath}\label{eqn:greensFunctionLaplacian:100}
\int dV' f(\Br') \delta^3(\Br' - \Br)
=
-\inv{4\pi} \int dV' f(\Br') {\spacegrad'}^2 \inv{\Norm{\Br' - \Br}}
=
-\lim_{\epsilon \rightarrow 0}
\inv{4\pi} \int_{\Norm{\Br - \Br'} < \epsilon} dV' f(\Br') \spacegrad' \cdot \spacegrad' \inv{\Norm{\Br' - \Br}}
=
\lim_{\epsilon \rightarrow 0}
\inv{4\pi} \int_{\Norm{\Br - \Br'} < \epsilon} dV' f(\Br') \spacegrad' \cdot \frac{\Br' - \Br}{\Norm{\Br' - \Br}^3}
=
\lim_{\epsilon \rightarrow 0}
\inv{4\pi} \int_{\Norm{\Br - \Br'} = \epsilon} dV' f(\Br') \ncap \cdot \frac{\Br' - \Br}{\Norm{\Br' - \Br}^3}.
\end{dmath}

Because the Laplacian has been shown to be zero everywhere where \( \Br \ne \Br' \) the volume integral over all space has been restricted to a small spherical volume surrounding the point \( \Br \).  The divergence theorem is then used to transform this integral into a surface integral over that spherical volume.  However, the exterior normal to this surface is \( \ncap = \lr{\Br' - \Br}/\Norm{\Br' - \Br} \), leaving

\begin{dmath}\label{eqn:greensFunctionLaplacian:120}
\int dV' f(\Br') \delta^3(\Br' - \Br)
=
\lim_{\epsilon \rightarrow 0}
\inv{4\pi} \int_{\Norm{\Br - \Br'} = \epsilon} dV' f(\Br') \inv{\Norm{\Br' - \Br}^2}
=
\lim_{\epsilon \rightarrow 0}
\inv{4\pi} \int_{\Norm{\Br - \Br'} = \epsilon} dV' f(\Br') \inv{\epsilon^2}
=
\lim_{\epsilon \rightarrow 0}
\inv{4\pi} f(\Br) \frac{4 \pi \epsilon^2}{\epsilon^2}
=
f(\Br).
\end{dmath}

In the second last step, the it is assumed that the function \( f(\Br') \) is well behaved enough in the near the point \( \Br \) that it can be pulled out of the integral, and replaced with it's mean value in the neighbourhood of \( \Br \), which then tends to \( f(\Br)\) as \( \epsilon \rightarrow 0 \).
%}

% also orphaned?
\subsection{cylinderField.tex}
   %
% Copyright © 2017 Peeter Joot.  All Rights Reserved.
% Licenced as described in the file LICENSE under the root directory of this GIT repository.
%
%{

\paragraph{Example.  field of a cylindrical distribution.}

Consider an electric multivector current distribution for static surface charge density \( \sigma \) and a current moving with constant (magnitude) current \( \Bv \) on a cylinder of radius \( R \)

\begin{dmath}\label{eqn:cylinderField:20}
J = \inv{\epsilon_0} \sigma \delta( r - R ) \lr{ 1 - \frac{\Bv}{c} }.
\end{dmath}

Using cylindrical coordinates for the field observation point \( \Bx \) and the volume integration point \( \Bx' \) have parameterization

\begin{dmath}\label{eqn:cylinderField:40}
\begin{aligned}
\Bx &= z \Be_3 + r \rhocap \\
\Bx' &= z' \Be_3 + r' \rhocap',
\end{aligned}
\end{dmath}
where \( \rhocap = \Be_1 e^{i \phi} \), and \( \rhocap' = \Be_1 e^{i \phi'} \), and \( i = \Be_{12} \).
The field is

\begin{dmath}\label{eqn:cylinderField:60}
F = \frac{\sigma}{4 \pi \epsilon_0}
\int_{-\infty}^\infty dz' \int_0^\infty r' dr' \int d\phi' \frac{
   \gpgrade{ \lr{ (z-z') \Be_3 + \Delta_r }\lr{ 1 - \Bv/c }  }{1,2}
}{ \lr{ (z -z')^2 + \Delta_r^2 }^{3/2} },
\end{dmath}
where \( \Delta_r = r \rhocap - r' \rhocap' \).
The z-axis integrals are of the form

\begin{dmath}\label{eqn:cylinderField:80}
\int_{-\infty}^\infty dz' \frac{ (1, z-z') }{ \lr{(z-z')^2 + a^2 }^{3/2}}
=
\lr{ \frac{2}{a^2} \sgn(z), 0 },
\end{dmath}
so the field is

\begin{dmath}\label{eqn:cylinderField:100}
F
=
\frac{\sigma R \sgn(z)}{2 \pi \epsilon_0}
\int_0^{2 \pi} d\phi' \frac{
   \gpgrade{ \lr{ r\rhocap - R \rhocap' } \lr{ 1 - \Bv/c } }{1,2}
}
{ (r\rhocap - R \rhocap')^2 }
=
\frac{\sigma \sgn(z)}{2 \pi \epsilon_0}
\int_0^{2 \pi} d\phi' \frac{
   \gpgrade{ \lr{ \tilde{r}\rhocap - \rhocap' } \lr{ 1 - \Bv/c } }{1,2}
}
{ (\tilde{r}\rhocap - \rhocap')^2 },
\end{dmath}
where \( \tilde{r} = r/R \).
Let's consider two cases \( \Bv = v \Be_3 \), and \( \Bv(\phi') = v \phicap' = v \Be_2 e^{i\phi'} \).  This means there are three integrals of interest

\begin{dmath}\label{eqn:cylinderField:120}
\int_0^{2 \pi} d\phi' \frac{(1, e^{i\phi'}, e^{2 i\phi'})}{1 + a^2 - 2 a \cos(\phi-\phi')}.
\end{dmath}
%}

% now orphaned.
\subsection{curvilinearGradient.tex}
   %
% Copyright � 2017 Peeter Joot.  All Rights Reserved.
% Licenced as described in the file LICENSE under the root directory of this GIT repository.
%
While the reciprocal frame can be computed explicitly, it can also be computed very simply by computing the gradient of the parameters themselves.
Two theorems relate the gradient and the reciprocal frame vectors.

\maketheorem{Gradient definition of reciprocal frame vectors}{thm:curvilinearGradient:1}{

Given a curvilinear basis \( \setlr{ \Bx_k } \), the reciprocal frame vectors are

\begin{dmath*}
\Bx^i = \spacegrad u_i.
\end{dmath*}
} % theorem

\maketheorem{Curvilinear representation of the gradient}{thm:curvilinearGradient:2}{

Given an n-parameter representation of a vector that spans an n-dimensional space

\begin{dmath*}
\Bx = \Bx(u_1, \cdots, u_n),
\end{dmath*}

the curvilinear representation of the gradient is

\begin{dmath*}
\spacegrad = \sum_i \Bx^i \PD{u_i}{}.
\end{dmath*}

It is often convenient to write this as

\begin{dmath*}
\spacegrad = \sum_{i=1}^n \Bx^i \partial_i,
\end{dmath*}

or the same with sums over mixed indexes implied.

} % theorem

We have only to apply the chain rule to prove both theorems.
Assuming \cref{thm:curvilinearGradient:1} to be true, then the dot products of the reciprocal frame vectors with the curvilinear basis vectors are

\begin{dmath}\label{eqn:curvilinearGradient:20}
\Bx^i \cdot \Bx_j
=
(\spacegrad u_i) \cdot \PD{u_j}{\Bx}
=
\sum_{r,s=1}^n
\lr{ \Be_r \PD{x_r}{u_i} } \cdot \lr{ \Be_s \PD{u_j}{x_s} }
=
\sum_{r,s=1}^n (\Be_r \cdot \Be_s)
\PD{x_r}{u_i} \PD{u_j}{x_s}
=
\sum_{r,s=1}^n \delta_{rs}
\PD{x_r}{u_i} \PD{u_j}{x_s}
=
\sum_{r=1}^n
\PD{x_r}{u_i} \PD{u_j}{x_r}
=
\PD{u_i}{u_j}
=
\delta_{ij}.
\end{dmath}

This shows that \( \Bx^i = \spacegrad u_i \) has the properties required of the reciprocal frame, proving the theorem.

The curvilinear representation of the gradient follows from the gradient representation of the reciprocal frame, and the chain rule.
The sum in \cref{thm:curvilinearGradient:2} expands as

\begin{dmath}\label{eqn:curvilinearGradient:40}
\sum_{i=1}^n
\Bx^i \PD{u_i}{F}
=
\sum_{i=1}^n
(\spacegrad u_i) \PD{u_i}{F}
=
\sum_{i,j=1}^n
\Be_j \PD{x_j}{u_i}
\PD{u_i}{F}
=
\sum_{j=1}^n
\Be_j
\PD{x_j}{F}
=
\spacegrad F,
\end{dmath}

which proves the result.

Note that the gradient representation of the reciprocal frame is mainly useful for theoretical reasons (i.e. the proof of the curvilinear representation of the gradient).
In many cases it will likely be more difficult to compute the reciprocal frame vectors using the gradient of the parameters than by other methods.

An excellent (and more detailed) discussion of the relationships of the reciprocal frame and the gradient can be found in \citep{aMacdonaldVAGC}.

\subsection{greensTheorem.tex}
   %
% Copyright © 2013 Peeter Joot.  All Rights Reserved.
% Licenced as described in the file LICENSE under the root directory of this GIT repository.
%
Given a two parameter (\(u,v\)) surface parameterization, the curvilinear coordinate representation of a vector \(\Bf\) has the form

\begin{dmath}\label{eqn:greensTheorem:1640}
\Bf = f_u \Bx^u + f_v \Bx^v + f_\perp \Bx^\perp.
\end{dmath}

We assume that the vector space is of dimension two or greater but otherwise unrestricted, and need not have an Euclidean basis.
Here \(f_\perp \Bx^\perp\) denotes the rejection of \(\Bf\) relative to the tangent space at the point of evaluation.
Green's theorem relates the integral around a closed curve to an ``area'' integral on that surface

\maketheorem{Green's Theorem}{thm:stokesTheoremGeometricAlgebra:1660}{
\index{Green's theorem}
\begin{equation*}
\ointctrclockwise \Bf \cdot d\Bl
=
\iint \lr{
-\PD{v}{f_u}
+\PD{u}{f_v}
}
du dv
\end{equation*}
}

Following the arguments used in \citep{schwartz1987pe} for Stokes' theorem in three dimensions, we first evaluate the loop integral along the differential element of the surface at the point \(\Bx(u_0, v_0)\) evaluated over the range \((du, dv)\), as shown in the infinitesimal loop of \cref{fig:loopIntegralInfinitesimal:loopIntegralInfinitesimalFig1}.

\imageFigure{../figures/gabook/loopIntegralInfinitesimalFig1}{Infinitesimal loop integral.}{fig:loopIntegralInfinitesimal:loopIntegralInfinitesimalFig1}{0.35}

Over the infinitesimal area, the loop integral decomposes into

\begin{dmath}\label{eqn:greensTheorem:1700}
\ointctrclockwise \Bf \cdot d\Bl
=
\int \Bf \cdot d\Bx_1
+\int \Bf \cdot d\Bx_2
+\int \Bf \cdot d\Bx_3
+\int \Bf \cdot d\Bx_4,
\end{dmath}

where the differentials along the curve are

\begin{dmath}\label{eqn:greensTheorem:1600}
\begin{aligned}
d\Bx_1 &= \evalbar{ \PD{u}{\Bx} }{v = v_0} du \\
d\Bx_2 &= \evalbar{ \PD{v}{\Bx} }{u = u_0 + du} dv \\
d\Bx_3 &= -\evalbar{ \PD{u}{\Bx} }{v = v_0 + dv} du \\
d\Bx_4 &= -\evalbar{ \PD{v}{\Bx} }{u = u_0} dv.
\end{aligned}
\end{dmath}

It is assumed that the parameterization change \((du, dv)\) is small enough that this loop integral can be considered planar (regardless of the dimension of the vector space).
Making use of the fact that \(\Bx^\perp \cdot \Bx_\alpha = 0\) for \(\alpha \in \setlr{u,v}\), the loop integral is

\begin{dmath}\label{eqn:greensTheorem:1620}
\ointctrclockwise \Bf \cdot d\Bl
=
\int
\lr{
f_u \Bx^u + f_v \Bx^v + f_\perp \Bx^\perp
}
\cdot
\Bigl(
\Bx_u(u, v_0) du - \Bx_u(u, v_0 + dv) du
+\Bx_v(u_0 + du, v) dv - \Bx_v(u_0, v) dv
\Bigr)
=
\int
f_u(u, v_0) du - f_u(u, v_0 + dv) du
+
f_v(u_0 + du, v) dv - f_v(u_0, v) dv
\end{dmath}

With the distances being infinitesimal, these differences can be rewritten as partial differentials

\begin{dmath}\label{eqn:greensTheorem:1860}
\ointctrclockwise \Bf \cdot d\Bl
=
\iint \lr{
-\PD{v}{f_u}
+\PD{u}{f_v}
}
du dv.
\end{dmath}

We can now sum over a larger area as in \cref{fig:loopIntegralInfinitesimalSum:loopIntegralInfinitesimalSumFig2}

\imageFigure{../figures/gabook/loopIntegralInfinitesimalSumFig2}{Sum of infinitesimal loops.}{fig:loopIntegralInfinitesimalSum:loopIntegralInfinitesimalSumFig2}{0.35}

All the opposing oriented loop elements cancel, so the integral around the complete boundary of the surface \(\Bx(u, v)\) is given by the \(u,v\) area integral of the partials difference.

We will see that Green's theorem is a special case of the Stokes' theorem.
This observation will also provide a geometric interpretation of the right hand side area integral of \cref{thm:stokesTheoremGeometricAlgebra:1660}, and allow for a coordinate free representation.

\paragraph{Special case:}

An important special case of Green's theorem is for a Euclidean two dimensional space where the vector function is

\begin{dmath}\label{eqn:greensTheorem:1720}
\Bf = P \Be_1 + Q \Be_2.
\end{dmath}

Here Green's theorem takes the form

\boxedEquation{eqn:stokesTheoremGeometricAlgebra:1710}{
\ointctrclockwise P dx + Q dy
=
\iint \lr{
\PD{x}{Q}
-\PD{y}{P}
}
dx dy.
}

\subsection{oneparameter.tex}
   %
% Copyright © 2016 Peeter Joot.  All Rights Reserved.
% Licenced as described in the file LICENSE under the root directory of this GIT repository.
%
\index{differential form}
An example parameterization with one parameter, and the corresponding differential with respect to that parameter, is plotted in
\cref{fig:oneParameterDifferential:oneParameterDifferentialFig1}.
%, for a parameterization over \( [a, b] \in [0,1]\otimes[0,1] \).

\imageFigure{../figures/GAelectrodynamics/oneParameterDifferentialFig1}{One parameter manifold.}{fig:oneParameterDifferential:oneParameterDifferentialFig1}{0.3}

The differential with respect to the parameter \( a \) is

\begin{equation}\label{eqn:oneparameter:20}
d\Bx_a = \PD{a}{\Bx} da = \Bx_a da.
\end{equation}

On this curve the projection of the gradient (the vector derivative) has just one component

\begin{dmath}\label{eqn:oneparameter:40}
\boldpartial
=
\sum_i \Bx^i (\Bx_i \cdot \spacegrad)
=
\Bx^a \PD{a}{}
\equiv
\Bx^a \partial_a.
\end{dmath}

Stokes' theorem for a one parameter manifold can only be expressed for scalar fields.
That is

\begin{dmath}\label{eqn:oneparameter:60}
\int d\Bx \cdot (\boldpartial \wedge \psi)
=
\int d\Bx \cdot \boldpartial \psi
=
\int da \PD{a}{ \psi }
= \evalbar{\psi}{\Delta a}.
\end{dmath}

Observe that the vector derivative can be replaced by the gradient since \( d\Bx \cdot \boldpartial = d\Bx \cdot \spacegrad \).
This is because only components of the gradient that are tangent to the curve defined by the differential element \( d\Bx \) make any contribution in the dot product.

\index{Stokes' theorem}
That means that Stokes' theorem for a one parameter curve is exactly the fundamental theorem of calculus for line integrals

%\begin{dmath}\label{eqn:oneparameter:80}
\boxedEquation{eqn:oneparameter:80}{
\int_{\Ba}^{\Bb} d\Bx \cdot \spacegrad \psi = \psi(\Bb) - \psi(\Ba).
}
%\end{dmath}

\subsection{twoparameter.tex}
   %
% Copyright © 2016 Peeter Joot.  All Rights Reserved.
% Licenced as described in the file LICENSE under the root directory of this GIT repository.
%
\index{area element}
\index{differential form}
An example parameterization with two parameters, and the corresponding differentials with respect to those parameters, is plotted in
\cref{fig:twoParameterDifferential:twoParameterDifferentialFig1}.

\imageFigure{../figures/GAelectrodynamics/twoParameterDifferentialFig1}{Two parameter manifold differentials.}{fig:twoParameterDifferential:twoParameterDifferentialFig1}{0.4}

Given parameters \( a, b \), the differentials along each of the parameterization directions are

\begin{dmath}\label{eqn:twoparameter:100}
\begin{aligned}
d\Bx_a &= \PD{a}{\Bx} da = \Bx_a da \\
d\Bx_b &= \PD{b}{\Bx} db = \Bx_b db.
\end{aligned}
\end{dmath}

The ``volume'' element for this parameterization (a surface area element) is

\begin{equation}\label{eqn:twoparameter:120}
d^2 \Bx
=
d\Bx_a \wedge
d\Bx_b
=
da db (\Bx_a \wedge \Bx_b).
\end{equation}

The vector derivative, the projection of the gradient onto the surface at the point of integration (also called the tangent space), now has two components

\begin{dmath}\label{eqn:twoparameter:200}
\boldpartial
=
\sum_\mu \Bx^\mu (\Bx_\mu \cdot \spacegrad)
=
\Bx^a \PD{a}{}
+
\Bx^b \PD{b}{}
\equiv
\Bx^a \partial_a
+
\Bx^b \partial_b.
\end{dmath}

The Stokes integral can be evaluated over this volume element for either scalar fields \( \psi \) or vector fields \( \Bf \), and takes the form

\begin{subequations}
\label{eqn:twoparameter:140}
\begin{equation}\label{eqn:twoparameter:160}
\int_A d^2 \Bx \cdot (\boldpartial \wedge \psi) =
\int_A (d^2 \Bx \cdot \boldpartial) \psi
=
\int_{\partial A} d^1 \Bx \psi
\end{equation}
\begin{equation}\label{eqn:twoparameter:180}
\int_A d^2 \Bx \cdot (\boldpartial \wedge \Bf) =
\int_A (d^2 \Bx \cdot \boldpartial) \cdot \Bf
=
\int_{\partial A} d^1 \Bx \cdot \Bf.
\end{equation}
\end{subequations}
FIXME: Wolfgang: ``SHOULD these GET A FRAMING RECTANGLE?''

To extract the full meaning of this the boundary differential \( d^1 \Bx \) must be computed.
This has the same structure for a vector or scalar field

\begin{dmath}\label{eqn:twoparameter:220}
\begin{aligned}
\int_A d^2 \Bx \cdot (\boldpartial \wedge \Bf)
&=
\int_A (d^2 \Bx \cdot \boldpartial) \cdot \Bf \\
&=
\sum_\mu \int_A (d^2 \Bx \cdot \Bx^\mu) \cdot \partial_\mu \Bf \\
&=
\sum_\mu \int_A da db  \lr{ \Bx_a \wedge \Bx_b ) \cdot \Bx^\mu } \cdot \partial_\mu \Bf \\
&=
\sum_\mu \int_A da db  \lr{ \Bx_a {\delta_b}^\mu - \Bx_b {\delta_a}^\mu } \cdot \partial_\mu \Bf \\
&=
\int_A da db  \lr{ \Bx_a \cdot \PD{b}{ \Bf} - \Bx_b \cdot \PD{a}{\Bf} }
\end{aligned}
\end{dmath}

While \( \Bx_a, \Bx_b \) both depend on both parameters \( a, b \), the differential form immediately above is still a perfect integral in the variables of the partials since \( \Bx_a \) is computed with \( b \) held fixed, and \( \Bx_b \) is computed with \( a \) held fixed.
Proceeding with the integrals that match the respective partials, this gives

\begin{dmath}\label{eqn:twoparameter:240}
\int_A d^2 \Bx \cdot (\boldpartial \wedge \Bf)
=
\int
da \Bx_a \cdot \evalbar{\Bf}{\Delta b}
-\int
db \Bx_b \cdot \evalbar{\Bf}{\Delta a}
=
\int
d\Bx_a \cdot \evalbar{\Bf}{\Delta b}
-\int
d\Bx_b \cdot \evalbar{\Bf}{\Delta a}.
\end{dmath}

This shows that the boundary differential \( d^1 \Bx \) in \cref{eqn:twoparameter:140} is given by

\begin{dmath}\label{eqn:twoparameter:260}
d^1 \Bx = d\Bx_a - d\Bx_b,
\end{dmath}

where it is implied that the field in question is evaluated at the boundaries of the parameter that has been eliminated by this first integration.
This boundary integral can be interpreted as the integral around a contour, as indicated in
\cref{fig:twoParameterDifferentialBoundary:twoParameterDifferentialBoundaryFig2}.

\imageFigure{../figures/GAelectrodynamics/twoParameterDifferentialBoundaryFig2}{Contour for two parameter surface boundary.}{fig:twoParameterDifferentialBoundary:twoParameterDifferentialBoundaryFig2}{0.4}

Additionally, as with the single parameter case, a substitution of the gradient does not change the result, since any component of the gradient that lies outside of the tangent space on the surface has a zero dot product with the surface volume element \( d^2 \Bx \).
This allows the two parameter Stokes integrals to be written as

%\begin{dmath}\label{eqn:twoparameter:280}
\boxedEquation{eqn:twoparameter:280}{
\begin{aligned}
\int_A d^2 \Bx \cdot \spacegrad \psi &= \ointclockwise d\Bx \psi \\
\int_A d^2 \Bx \cdot (\spacegrad \wedge \Bf) &= \ointclockwise d\Bx \cdot \Bf.
\end{aligned}
}
%\end{dmath}

It can be shown that this two parameter Stokes integral is equivalent to Green's theorem.

\subsection{stokesAndGreens.tex}
   %
% Copyright � CCYY Peeter Joot.  All Rights Reserved.
% Licenced as described in the file LICENSE under the root directory of this GIT repository.
%
\makeproblem{Stokes' theorem relation to Green's theorem}{problem:stokesAndGreens:1}{
Show that Stokes' theorem, in its two parameter form, applied to a vector field recovers Green's theorem.
\index{Green's theorem}
\index{Stokes' theorem}
} % problem

\makeanswer{problem:stokesAndGreens:1}{

To demonstrate this, expand the LHS of the Stokes identity

\begin{dmath}\label{eqn:stokesAndGreens:20}
\int_A d^2 \Bx \cdot (\boldpartial \wedge \Bf) = \ointclockwise d\Bx \cdot \Bf.
\end{dmath}

Assuming \( u, v\) parameterization

\begin{dmath}\label{eqn:stokesAndGreens:40}
\int_A d^2 \Bx \cdot (\boldpartial \wedge \Bf)
=
\int_A (d\Bx_u \wedge d\Bx_v) \cdot (\boldpartial \wedge \Bf)
=
\int_A ((d\Bx_u \wedge d\Bx_v) \cdot \Bx^u) \cdot \partial_u \Bf
+
\int_A ((d\Bx_u \wedge d\Bx_v) \cdot \Bx^v) \cdot \partial_v \Bf
=
-\int_A du dv \Bx_v \cdot \partial_u \Bf
+
\int_A du dv \Bx_u \cdot \partial_v \Bf
=
-\int_A du dv \Bx_v \cdot \partial_u \Bf
+
\int_A du dv \lr{
-\Bx_v \cdot \partial_u \Bf
+
\Bx_u \cdot \partial_v \Bf
}.
\end{dmath}

The coordinate expansion of \( \Bf \) with respect to the tangent space coordinates is

\begin{dmath}\label{eqn:stokesAndGreens:60}
\Bf = \Bx^u f_u + \Bx^v f_v + \Bf_\perp
\end{dmath}

where \( \Bf_\perp \) lies in normal to the tangent space at the point in question.
Because \( \Bx_v \) is computed with \( u \) held fixed and \( \Bx_u \) computed with \( v \) held fixed, the area integrand can be written

\begin{dmath}\label{eqn:stokesAndGreens:80}
-\Bx_v \cdot \partial_u \Bf
+
\Bx_u \cdot \partial_v \Bf
=
-\PD{u}{}\lr{ \Bx_v \cdot \Bf }
+\PD{v}{}\lr{ \Bx_u \cdot \Bf }
=
-\PD{u}{f_v}
+\PD{v}{f_u},
\end{dmath}

which gives
\begin{dmath}\label{eqn:stokesAndGreens:100}
\int_A du dv \lr{ -\PD{u}{f_v}
+\PD{v}{f_u}
}
=
\ointclockwise d\Bx \cdot \Bf,
\end{dmath}

which recovers \cref{thm:stokesTheoremGeometricAlgebra:1660} as desired.
} % answer

\subsection{Justifying_the_contraction_axiom.tex}
   %
% Copyright � 2016 Peeter Joot.  All Rights Reserved.
% Licenced as described in the file LICENSE under the root directory of this GIT repository.
%
%{
\index{number line}
%\footnote{Similar to Feynman on gravitation \citep{feynman1963flp} ``... have shall said everything required, for a sufficiently talented mathematician could then deduce all the consequences of these principles.  However, since you are not assumed to be sufficiently talented yet, we shall discuss the consequences in more detail''.}.

The contraction axiom is arguably the most important of the multivector space axioms.
The use of this axiom was not justified or motivated in any way.
It could be argued that the subsequent use of the axiom provides justification, but that may be an unsatisfactory argument.

Here I make an argument that the contraction axiom is consistent with the rules for multiplying numbers, in particular, with the rule for squaring a number.
It that is the case, and the rules for multiplying numbers should be consistent with the rules for multiplying vectors in a one dimensional vector space, or with the rules for multiplying vectors in any one dimensional vector subspace, then we have a justification for the contraction axiom for multiplication of vectors from arbitrary vector spaces.

Consider \R{1}, a one dimensional vector space, spanned by a single unit vector \( \setlr{ \Be_1 } \).
A point \( x \) on a real number line can be associated with each vector \( x \Be_1 \) in this space.
This establishes an isomorphism \( x \leftrightarrow x \Be_1 \) between the real number line \( \setlr{x}, x \in \bbR \), and this one dimensional vector space \( \setlr{
\Bx = x \Be_1, x \in \bbR} \).

To illustrate this isomorphism, the vectors \( -3\Be_1 \) and \( 7 \Be_1 \) are plotted in
\cref{fig:1Darrows:1DarrowsFig2} for both \R{1} and the real number line.
\imageTwoFigures
{../figures/GAelectrodynamics/1DarrowsFig2}
{../figures/GAelectrodynamics/1DnumberlineFig1}
{Equivalent vectors in \R{1} and on a number line.}{fig:1Darrows:1DarrowsFig2}{scale=0.5}
%\imageFigure{../figures/GAelectrodynamics/1DarrowsFig2}{Vectors in 1D space.}{fig:1Darrows:1DarrowsFig2}{0.03}
%\imageFigure{../figures/GAelectrodynamics/1DnumberlineFig1}{Points on a number line.}{fig:1Dnumberline:1DnumberlineFig1}{0.045}
%\cref{fig:1Dnumberline:1DnumberlineFig1}.

We know how to multiply real numbers, so can we use the same implicit rules to determine how we should multiply one dimensional vectors?  In particular,
the rules for real numbers require that for any point \( x \) distant from the origin, we have
\begin{equation}\label{eqn:multiplication:60}
(\pm x)^2 = \Abs{x}^2 = x^2.
\end{equation}

This is the familiar rule for real number multiplication, the square of a number (positive or negative) equals the squared distance of that number from zero (i.e. numbers squared are positive).

An equivalent statement for the square of a vector in \R{1} is
\begin{equation}\label{eqn:multiplication:40}
(\pm \Bx)^2 = \Abs{x}^2 = x^2.
\end{equation}

Observe that this is identical to the contraction axiom for a one dimensional Euclidean vector space.
If this the desired behaviour of a vector square in \R{1}, then it should also be the rule for squaring any vector lying in a one dimensional vector subspace, therefore providing a justification of the contraction axiom in general.

In this sense the contraction axiom can be conceptualized as the vector space equivalent of the numeric multiplication rule ``the square of a number is positive''.

%%}
%%%\EndArticle
%%\EndNoBibArticle

\subsection{statement.tex}
   %
% Copyright © 2016 Peeter Joot.  All Rights Reserved.
% Licenced as described in the file LICENSE under the root directory of this GIT repository.
%
\index{Stokes' theorem}
Stokes' theorem is fairly easy to state, but takes a fair amount of work to understand and unpack its implications.

%
% Copyright © 2013 Peeter Joot.  All Rights Reserved.
% Licenced as described in the file LICENSE under the root directory of this GIT repository.
%
\maketheorem{Stokes' Theorem}{thm:stokesTheoremGeometricAlgebra:1740}{

For blades \(F \in \bigwedge^{s}\), and \(m\) volume element \(d^k \Bx, s < k\),

\begin{equation*}%\label{eqn:stokesTheoremTheStatement:120}
\int_V d^k \Bx \cdot (\boldpartial \wedge F) = \int_{\partial V} d^{k-1} \Bx \cdot F.
\end{equation*}

Here the volume integral is over a \(m\) dimensional surface (manifold).  The derivative operator \(\boldpartial\) is called the vector derviative and is the projection of the gradient onto the tangent space of the manifold.  Integration over the boundary of \(V\) is indicated by \( \partial V \).
}

The vector derivative is defined by

\begin{equation}\label{eqn:stokesTheoremTheStatement:1400}
\boldpartial = \Bx^i \partial_i = \sum_i \Bx_i \PD{u^i}{}.
\end{equation}

where \( \Bx^i \) are reciprocal frame vectors dual to the tangent vector basis \( \Bx_i \) associated with the parameters \( u^1, u^2, \cdots \).
%These will be defined in more detail in the next section.
Once the volume element, vector product and the other concepts are defined, the proof of
Stokes theorem is really just a statement that

\boxedEquation{eqn:stokesTheoremGeometricAlgebra:2840}{
\int_V d^k \Bx \cdot (\Bx^i \partial_i \wedge F) =
\int_V \lr{ d^k \Bx \cdot \Bx^i } \cdot \partial_i F.
}

This dot product expansion applies to any degree blade and volume element provided the degree of the blade is less than that of the volume element (i.e. \(s < k\)).  That magic follows directly from \cref{thm:stokesTheoremGeometricAlgebra:1420}.


\index{oriented volume element}
This dot product defines the oriented surface ``area'' elements associated with the ``volume'' element \( d^k \Bx \).
That area element can be obtained from the mnemonic
\begin{dmath}\label{eqn:statement:1561}
\sum_i d^k \Bx \cdot \Bx^i,
\end{dmath}
with each of the i-th differentials evaluated.
This will be made clear by example.

\section{(REWRITE)Stokes' theorem.}
   \subsection{Three parameter specialization of Stokes' theorem.}
      %
% Copyright © 2016 Peeter Joot.  All Rights Reserved.
% Licenced as described in the file LICENSE under the root directory of this GIT repository.
%

\index{volume parameterization}
\index{area element}
\index{differential form}
An example parameterization with three parameters, and the corresponding differentials with respect to those parameters, and the outwards normals, are sketched in
\cref{fig:normalsOnVolumeAreaElement:normalsOnVolumeAreaElementFig11}.

\imageFigure{../figures/gabook/normalsOnVolumeAreaElementFig11}{Three parameter volume element.}{fig:normalsOnVolumeAreaElement:normalsOnVolumeAreaElementFig11}{0.4}

Given parameters \( a, b, c \), the differentials along each of the parameterization directions are
\begin{dmath}\label{eqn:threeparameter:1421}
\begin{aligned}
d\Bx_a &= \PD{a}{\Bx} da = \Bx_a da \\
d\Bx_b &= \PD{b}{\Bx} db = \Bx_b db \\
d\Bx_c &= \PD{c}{\Bx} dc = \Bx_c dc.
\end{aligned}
\end{dmath}

The ``volume'' element for this parameterization (a surface area element) is
\begin{equation}\label{eqn:threeparameter:1441}
d^3 \Bx
=
d\Bx_a
\wedge
d\Bx_b
\wedge
d\Bx_c
=
da db dc (\Bx_a \wedge \Bx_b \wedge \Bx_c).
\end{equation}

The vector derivative, the projection of the gradient onto the surface at the point of integration (also called the tangent space), now has three components
\begin{dmath}\label{eqn:threeparameter:1461}
\boldpartial
=
\sum_\mu \Bx^\mu (\Bx_\mu \cdot \spacegrad)
=
\Bx^a \PD{a}{}
+
\Bx^b \PD{b}{}
+
\Bx^c \PD{c}{}
\equiv
\Bx^a \partial_a
+
\Bx^b \partial_b
+
\Bx^c \partial_c.
\end{dmath}

The Stokes integral can be evaluated over this volume element for either scalar fields \( \psi \), vector fields \( \Bf \), or bivector fields \( B \) and takes the form

\begin{subequations}
\label{eqn:threeparameter:1481}
\begin{equation}\label{eqn:threeparameter:1501}
\int_V d^3 \Bx \cdot (\boldpartial \wedge \psi) =
\int_V (d^3 \Bx \cdot \boldpartial) \psi
=
\int_{\partial V} d^2 \Bx \psi
\end{equation}
\begin{equation}\label{eqn:threeparameter:1521}
\int_V d^3 \Bx \cdot (\boldpartial \wedge \Bf) =
\int_V (d^3 \Bx \cdot \boldpartial) \cdot \Bf
=
\int_{\partial V} d^2 \Bx \cdot \Bf
\end{equation}
\begin{equation}\label{eqn:threeparameter:1541}
\int_V d^3 \Bx \cdot (\boldpartial \wedge B) =
\int_V (d^3 \Bx \cdot \boldpartial) \cdot B
=
\int_{\partial V} d^2 \Bx \cdot B.
\end{equation}
\end{subequations}

When working with \R{3} vector spaces, \( \boldpartial = \spacegrad \), but in higher dimensional spaces, the gradient can also be substituted above due using the same arguments about projection onto the tangent space.

An explicit value for the differential form of the boundary integral is desired and can be obtained from the mnemonic \cref{eqn:statement:1561}
\begin{dmath}\label{eqn:threeparameter:1581}
\sum_i d^3 \Bx \cdot \Bx^i
=
\sum_i da db dc \lr{ \Bx_a \wedge \Bx_b \wedge \Bx_c } \cdot \Bx^i
=
\sum_i da db dc \lr{
\Bx_a \wedge \Bx_b +
\Bx_b \wedge \Bx_c +
\Bx_c \wedge \Bx_a }.
\end{dmath}

The bounding form for the three parameter volume is therefore
\begin{dmath}\label{eqn:threeparameter:1601}
d^2 \Bx
=
d\Bx_a \wedge d\Bx_b +
d\Bx_b \wedge d\Bx_c +
d\Bx_c \wedge d\Bx_a.
\end{dmath}

   \subsection{Using scalar volume elements}
      %
% Copyright © 2016 Peeter Joot.  All Rights Reserved.
% Licenced as described in the file LICENSE under the root directory of this GIT repository.
%

FIXME: remove most of this and introduce inline with the oriented area and volume results.  This is already done for the \( d^2 \Bx \) integrals.

In \R{3} the area elements of
(FIXME: equation reference dead with rewrite)
%\cref{eqn:twoparameter:140}
, and volume elements of 
\cref{eqn:threeparameter:1481}
can be reexpressed as scalars, recovering a number of the integral calculus identities that are more familiar than the wedge product variants above.

The pseudoscalar volume element can be written
\begin{dmath}\label{eqn:scalarVolumeElement:1621}
d^3 \Bx = I dV,
\end{dmath}
and the (oriented) area elements can be written as
\begin{dmath}\label{eqn:scalarVolumeElement:1641}
d^2 \Bx \ncap = I dA,
\end{dmath}
or
\begin{dmath}\label{eqn:scalarVolumeElement:1661}
d^2 \Bx = I \ncap dA.
\end{dmath}

For \( \psi \in \bigwedge^0, \Bf \in \bigwedge^1, B \in \bigwedge^2 \), this gives

\begin{subequations}
\label{eqn:scalarVolumeElement:1681}
\begin{equation}\label{eqn:scalarVolumeElement:1701}
I \int_A dA \ncap \wedge \spacegrad \psi = \ointclockwise d\Bx \psi
\end{equation}
\begin{equation}\label{eqn:scalarVolumeElement:1721}
I \int_A dA \ncap \wedge \spacegrad \wedge \Bf = \ointclockwise d\Bx \cdot \Bf
\end{equation}
\begin{equation}\label{eqn:scalarVolumeElement:1741}
\int_V dV \spacegrad \psi = \int_{\partial V} dA \ncap \psi
\end{equation}
\begin{equation}\label{eqn:scalarVolumeElement:1761}
\int_V dV \spacegrad \wedge \Bf = \int_{\partial V} dA \ncap \wedge \Bf
\end{equation}
\begin{equation}\label{eqn:scalarVolumeElement:1781}
\int dV \spacegrad \wedge B = \int_{\partial V} dA \ncap \wedge B
\end{equation}
\end{subequations}

It is straightforward to re-express all the wedge products above in their dual forms.
With \( B = I \Bf \), that is

\begin{subequations}
\label{eqn:scalarVolumeElement:1801}
\begin{equation}\label{eqn:scalarVolumeElement:1821}
\int_A dA \ncap \cross \spacegrad \psi = \ointctrclockwise d\Bx \psi
\end{equation}
\begin{equation}\label{eqn:scalarVolumeElement:1841}
\int_A dA \ncap \cdot (\spacegrad \cross \Bf) = \ointctrclockwise d\Bx \cdot \Bf
\end{equation}
\begin{equation}\label{eqn:scalarVolumeElement:1861}
\int_V dV \spacegrad \psi = \int_{\partial V} dA \ncap \psi
\end{equation}
\begin{equation}\label{eqn:scalarVolumeElement:1881}
\int_V dV \spacegrad \cross \Bf = \int_{\partial V} dA \ncap \cross \Bf
\end{equation}
\begin{equation}\label{eqn:scalarVolumeElement:1901}
\int dV \spacegrad \cdot \Bf = \int_{\partial V} dA \ncap \cdot \Bf.
\end{equation}
\end{subequations}

Each of the cross product terms above can also be put into dual forms, giving

\begin{subequations}
\label{eqn:scalarVolumeElement:1801c}
\begin{equation}\label{eqn:scalarVolumeElement:1821c}
\int_A dA \ncap \cdot \lr{ I \spacegrad \psi } = \ointclockwise d\Bx \psi
\end{equation}
\begin{equation}\label{eqn:scalarVolumeElement:1841c}
\int_A dA \ncap \cdot (\spacegrad \cdot B) = \ointctrclockwise d\Bx \cdot (I B)
\end{equation}
\begin{equation}\label{eqn:scalarVolumeElement:1881c}
\int_V dV \spacegrad \cdot B = \int_{\partial V} dA \ncap \cdot B.
\end{equation}
\end{subequations}

Note that all of
\cref{eqn:scalarVolumeElement:1861}, \cref{eqn:scalarVolumeElement:1901}, and \cref{eqn:scalarVolumeElement:1881c} all have the same form
%\begin{equation}\label{eqn:scalarVolumeElement:1881d}
\boxedEquation{eqn:scalarVolumeElement:1881d}{
\int_V dV \spacegrad \cdot A = \int_{\partial V} dA \ncap \cdot A.
}
%\end{equation}
\index{divergence theorem}

This is also true for pseudoscalar grades, which can be demonstrated by multiplying both sides of \cref{eqn:scalarVolumeElement:1741} with \( I \).
This implies that \cref{eqn:scalarVolumeElement:1881d} is valid for any \R{3} multivector, generalizing the conventional divergence theorem over a 3D volume to all spatial grades.

   \subsection{Problems}
      %
% Copyright © 2016 Peeter Joot.  All Rights Reserved.
% Licenced as described in the file LICENSE under the root directory of this GIT repository.
%

\makeproblem{\R{3} dual forms of Stokes' theorem.}{problem:stokesTheoremCoreProblems:1}{
Prove
\makesubproblem{}{problem:stokesTheoremCoreProblems:1:a}
\cref{eqn:scalarVolumeElement:1681},
\makesubproblem{}{problem:stokesTheoremCoreProblems:1:b}
\cref{eqn:scalarVolumeElement:1801},
\makesubproblem{}{problem:stokesTheoremCoreProblems:1:c}
and \cref{eqn:scalarVolumeElement:1801c}.
} % problem

\makeanswer{problem:stokesTheoremCoreProblems:1}{

The volume elements are
\makeSubAnswer{}{problem:stokesTheoremCoreProblems:1:a}
\begin{subequations}
\label{eqn:stokesTheoremCoreProblems:20}
\begin{dmath}\label{eqn:stokesTheoremCoreProblems:40}
d^2 \Bx \cdot \spacegrad
=
dA \gpgradeone{ I \ncap \spacegrad }
=
dA I \ncap \wedge \spacegrad
\end{dmath}
\begin{dmath}\label{eqn:stokesTheoremCoreProblems:60}
d^2 \Bx \cdot (\spacegrad \wedge \BA)
=
dA \gpgradezero{ I \ncap \spacegrad \BA }
=
dA I \ncap \wedge \spacegrad \wedge \BA
\end{dmath}
\begin{dmath}\label{eqn:stokesTheoremCoreProblems:80}
d^3 \Bx \cdot \spacegrad \phi
=
dV \gpgradetwo{ I \spacegrad \phi }
=
dV I \spacegrad \phi
\end{dmath}
\begin{dmath}\label{eqn:stokesTheoremCoreProblems:100}
d^3 \Bx \cdot (\spacegrad \wedge \BA)
=
dV \gpgradeone{ I (\spacegrad \wedge \BA) }
=
dV I \spacegrad \wedge \BA
\end{dmath}
\begin{dmath}\label{eqn:stokesTheoremCoreProblems:120}
d^3 \Bx \cdot (\spacegrad \wedge B)
=
dV \gpgradezero{ I (\spacegrad \wedge B) }
=
dV I \spacegrad \wedge B.
\end{dmath}
\end{subequations}

The corresponding boundary forms are
\begin{subequations}
\label{eqn:stokesTheoremCoreProblems:140}
\begin{equation}\label{eqn:stokesTheoremCoreProblems:160}
d\Bx \psi
\end{equation}
\begin{dmath}\label{eqn:stokesTheoremCoreProblems:180}
d\Bx \cdot \BA
\end{dmath}
\begin{dmath}\label{eqn:stokesTheoremCoreProblems:200}
d^2 \Bx \psi
=
dA I \ncap \psi
\end{dmath}
\begin{dmath}\label{eqn:stokesTheoremCoreProblems:220}
d^2 \Bx \cdot \BA
=
dA \gpgradeone{ I \ncap \BA }
=
dA I \ncap \wedge \BA
\end{dmath}
\begin{dmath}\label{eqn:stokesTheoremCoreProblems:240}
d^2 \Bx \cdot B
=
dA \gpgradezero{ I \ncap B }
=
dA I \ncap \wedge B.
\end{dmath}
\end{subequations}

Assembling these pieces back into the integrals proves the relationships.

\makeSubAnswer{}{problem:stokesTheoremCoreProblems:1:b}

To show \cref{eqn:scalarVolumeElement:1841} note that
\begin{dmath}\label{eqn:stokesTheoremCoreProblems:260}
I (\Ba \wedge \Bb \wedge \Bc)
=
\gpgradezero{ I \Ba \wedge \Bb \wedge \Bc }
=
\gpgradezero{ I \Ba (\Bb \wedge \Bc) -
I \Ba \cdot (\Bb \wedge \Bc)
}
=
\gpgradezero{ I \Ba I(\Bb \cross \Bc) }
=
- \Ba \cdot (\Bb \cross \Bc).
\end{dmath}

To show \cref{eqn:scalarVolumeElement:1901} note that
\begin{dmath}\label{eqn:stokesTheoremCoreProblems:280}
\Ba \wedge (I \BA)
=
\Ba \wedge (I \BA)
=
\gpgradethree{ \Ba I \BA }
=
\gpgradethree{ I \Ba \cdot \BA }
=
I (\Ba \cdot \BA).
\end{dmath}

\makeSubAnswer{}{problem:stokesTheoremCoreProblems:1:c}

For vector \( \Ba \), these transformations all follow from
\begin{dmath}\label{eqn:stokesTheoremCoreProblems:300}
\Ba \cross \Bf
=
\gpgradeone{ -I \Ba \wedge \Bf}
=
\gpgradeone{ -I \Ba \Bf}
=
-\gpgradeone{ \Ba I \Bf}
=
-\Ba \cdot (I \Bf)
=
\Ba \cdot B.
\end{dmath}

} % answer

\section{Problems}
   %
% Copyright © 2017 Peeter Joot.  All Rights Reserved.
% Licenced as described in the file LICENSE under the root directory of this GIT repository.
%
\makeproblem{One dimensional multivector space.}{problem:multivector:40}{
   Verify that for \( c, d \in \bbR \) the set \( M = \setlr{ c + d \Be_1 } \) satisfies all the multivector axioms.
} % problem

\makeproblem{Normal anticommutation.}{problem:multiplication:anticommutationNormal}{
Prove \cref{thm:multiplication:anticommutationNormal}.
}

   %
% Copyright © 2017 Peeter Joot.  All Rights Reserved.
% Licenced as described in the file LICENSE under the root directory of this GIT repository.
%
\makeproblem{Ninety degree rotation in coordinates.}{problem:coordinateNinetyDegreeRotation:1}{

Rotate the vector

\begin{dmath}\label{eqn:coordinateNinetyDegreeRotation:240}
   \Bx = \rho
\begin{bmatrix}
   \cos\theta \\
   \sin\theta \\
\end{bmatrix},
\end{dmath}

in the \( x,y \) plane by \(\pm \pi/2\) using the 2D rotation matrix
\begin{dmath}\label{eqn:coordinateNinetyDegreeRotation:280}
R_\theta =
\begin{bmatrix}
\cos\theta & -\sin\theta \\
\sin\theta & \cos\theta
\end{bmatrix},
\end{dmath}

and compare the result with \cref{eqn:SimpleProducts2:300} and \cref{eqn:SimpleProducts2:310}.
} % problem

\makeanswer{problem:coordinateNinetyDegreeRotation:1}{
The counterclockwise (positive) rotation is

\begin{dmath}\label{eqn:coordinateNinetyDegreeRotation:260}
\Bx'
=
\begin{bmatrix}
   0 & -1 \\
   1 & 0 \\
\end{bmatrix}
\begin{bmatrix}
   \cos\theta \\
   \sin\theta \\
\end{bmatrix}
=
\rho
\begin{bmatrix}
   -\sin\theta \\
   \cos\theta
\end{bmatrix},
\end{dmath}

consistent with \cref{eqn:SimpleProducts2:300}.  The clockwise rotation is

\begin{dmath}\label{eqn:coordinateNinetyDegreeRotation:300}
\Bx''
=
\begin{bmatrix}
   0 & 1 \\
   -1 & 0 \\
\end{bmatrix}
\begin{bmatrix}
   \cos\theta \\
   \sin\theta \\
\end{bmatrix}
=
\rho
\begin{bmatrix}
   \sin\theta \\
   -\cos\theta
\end{bmatrix},
\end{dmath}

which matches \cref{eqn:SimpleProducts2:310}.
} % answer

   %
% Copyright © 2017 Peeter Joot.  All Rights Reserved.
% Licenced as described in the file LICENSE under the root directory of this GIT repository.
%
\makeproblem{Right complex exponential multiplication}{problem:rightComplexMult:1}{

Show that the following product encodes a counterclockwise rotation by \( \theta \)
\begin{equation*}
\Bx'
= e^{-i\theta} \lr{ x \Be_1 + y \Be_2 }.
\end{equation*}
} % problem

\makeanswer{problem:rightComplexMult:1}{
\begin{dmath}\label{eqn:rightComplexMult:20}
\Bx'
= e^{-i\theta} \lr{ x \Be_1 + y \Be_2 }
= \lr{ \cos\theta + \Be_1 \Be_2 \sin\theta } \lr{ x \Be_1 + y \Be_2 }
=
x \cos\theta \Be_1
- x \sin\theta \Be_2
+ y \cos\theta \Be_2
+ y \sin\theta \Be_1
=
\lr{ x \cos\theta
+ y \sin\theta } \Be_1
+ \lr{
- x \sin\theta
+ y \cos\theta } \Be_2.
\end{dmath}

Observe that this matches the result obtained by application of a rotation matrix to the coordinates

\begin{dmath}\label{eqn:rightComplexMult:40}
\begin{bmatrix}
\cos\theta & -\sin\theta \\
\sin\theta & \cos\theta
\end{bmatrix}
\begin{bmatrix}
x \\
y
\end{bmatrix}
=
\begin{bmatrix}
x \cos\theta - y\sin\theta \\
x \sin\theta + y\cos\theta
\end{bmatrix}.
\end{dmath}
} % answer

   %
% Copyright © 2017 Peeter Joot.  All Rights Reserved.
% Licenced as described in the file LICENSE under the root directory of this GIT repository.
%
\makeproblem{Exponential commutation in a plane}{problem:sandwichMult:1}{
Prove that

\begin{dmath}\label{eqn:sandwichMult:10}
\begin{aligned}
   \Be_1 e^{i\theta} &= e^{-i\theta} \Be_1 \\
   \Be_2 e^{i\theta} &= e^{-i\theta} \Be_2 \\
\end{aligned}.
\end{dmath}
} % problem

\makeanswer{problem:sandwichMult:1}{
\begin{dmath}\label{eqn:sandwichMult:20}
\Be_1 e^{i\theta}
=
\Be_1 \lr{ \cos\theta + \Be_1 \Be_2 \sin\theta }
=
\Be_1 \cos\theta + \Be_2 \sin\theta
=
\lr{ \cos\theta + \Be_2 \Be_1 \sin\theta } \Be_1
=
\lr{ \cos\theta - \Be_1 \Be_2 \sin\theta } \Be_1
=
e^{-i\theta} \Be_1.
\end{dmath}

\begin{dmath}\label{eqn:sandwichMult:40}
\Be_2 e^{i\theta}
=
\Be_2 \lr{ \cos\theta + \Be_1 \Be_2 \sin\theta }
=
\Be_2 \cos\theta - \Be_1 \sin\theta
=
\lr{ \cos\theta - \Be_1 \Be_2 \sin\theta } \Be_2
=
e^{-i\theta} \Be_2.
\end{dmath}
} % answer

   %
% Copyright © 2017 Peeter Joot.  All Rights Reserved.
% Licenced as described in the file LICENSE under the root directory of this GIT repository.
%
\makeproblem{Normal multiplication with complex exponential.}{problem:normalMult:1}{
Show that any component of a vector normal to the x-y plane commutes with the complex exponential rotation operator for that plane.
} % problem

\makeanswer{problem:normalMult:1}{
This can be demonstrated by considering any vector \( \Be_k \), \( k \ne 1,2 \), such as \( \Be_3 \)

\begin{dmath}\label{eqn:normalMult:20}
\Be_3 e^{i\theta}
=
\Be_3 \lr{ \cos\theta + \Be_1 \Be_2 \sin\theta }
=
\Be_3 \cos\theta + \Be_3 \Be_1 \Be_2 \sin\theta
=
\Be_3 \cos\theta - \Be_1 \Be_3 \Be_2 \sin\theta
=
\Be_3 \cos\theta + \Be_1 \Be_2 \Be_3 \sin\theta
=
\lr{ \cos\theta + \Be_1 \Be_2 \sin\theta  } \Be_3
=
e^{i\theta} \Be_3.
\end{dmath}
} % answer

   %
% Copyright © 2017 Peeter Joot.  All Rights Reserved.
% Licenced as described in the file LICENSE under the root directory of this GIT repository.
%
\makeproblem{Linear intersection in \R{3}}{problem:crossProductLinearIntersectionProblem:1}{
Solve \cref{eqn:SimpleProducts2:1060} in \R{3} using the cross product.
} % problem

\makeanswer{problem:crossProductLinearIntersectionProblem:1}{

Because vector division in \R{3} is not traditionally available, the solution would have to take the form

\begin{dmath}\label{eqn:crossProductLinearIntersectionProblem:10}
\begin{aligned}
s &= -\frac{(\Bd \cross \Bu_2) \cdot (\Bu_1 \cross \Bu_2)}{\Norm{\Bu_1 \cross \Bu_2}^2} \\
t &= -\frac{(\Bd \cross \Bu_1) \cdot (\Bu_1 \cross \Bu_2)}{\Norm{\Bu_1 \cross \Bu_2}^2},
\end{aligned}
\end{dmath}

where the restriction required for solution validity would be \( ( \Bd \cross \Bu_2 ) \cross (\Bu_1 \cross \Bu_2 ) = 0 \).
} % answer

   %
% Copyright © 2016 Peeter Joot.  All Rights Reserved.
% Licenced as described in the file LICENSE under the root directory of this GIT repository.
%
\makeproblem{}{problem:multiplication:2dvectorsquare}{
Generalize the calculation of \cref{eqn:gaTutorial:80} to calculate the square of an \R{n} vector.

\begin{dmath}\label{eqn:multiplication:100}
\Bx = \sum_i x_i \Be_i
\end{dmath}
} % problem

\makeanswer{problem:multiplication:2dvectorsquare}{
Consider the 2D case to start with

\begin{dmath}\label{eqn:multiplication:120}
\Bx^2
=
\lr{ x \Be_1 + y \Be_2}
\lr{ x \Be_1 + y \Be_2}
=
\lr{ x \Be_1 } \lr{ x \Be_1 }
+
\lr{ y \Be_2 } \lr{ y \Be_2 }
+
\lr{ x \Be_1 } \lr{ y \Be_2 }
+
\lr{ y \Be_2 } \lr{ x \Be_1 }
=
x^2 \Be_1^2
+
y^2 \Be_2^2
+
x y \lr{ \Be_1 \Be_2 + \Be_2 \Be_1 }
=
x^2 + y^2
+
x y \lr{ \Be_1 \Be_2 + \Be_2 \Be_1 }.
\end{dmath}

The contraction axiom requires the bivector terms to sum to zero, as also demonstrated previously for the specific example \( \Bx = \Be_1 + \Be_2 \).

More generally for \R{N}

\begin{dmath}\label{eqn:multiplication:121}
\Bx^2
=
\lr{ \sum_i x_i \Be_i }
\lr{ \sum_j x_j \Be_j }
=
\sum_{ij} x_i x_j \Be_i \Be_j
=
\sum_{i = j} x_i x_j \Be_i \Be_j
+
\sum_{i \ne j} x_i x_j \Be_i \Be_j
=
\sum_{i} x_i^2
+
\sum_{i \ne j} x_i x_j \Be_i \Be_j
=
\sum_{i} x_i^2
+
\sum_{i < j} x_i x_j (\Be_i \Be_j + \Be_j \Be_i).
\end{dmath}

The contraction axiom requires all the bivector pairs to sum to zero.  That is, for each \( i \ne j \)

\begin{dmath}\label{eqn:introGAproblems:140}
\Be_i \Be_j = -\Be_j \Be_i.
\end{dmath}
} % answer

   %
% Copyright © 2016 Peeter Joot.  All Rights Reserved.
% Licenced as described in the file LICENSE under the root directory of this GIT repository.
%

\makeproblem{Normal anticommutation}{problem:multiplication:unitsquare}{
Prove \cref{thm:multiplication:anticommutationNormal}.
} % problem

\makeanswer{problem:multiplication:unitsquare}{
Consider the square of a vector \( \Bx = u \Bu + v \Bv \) with respect to a basis of unit vectors \( \setlr{ \Bu, \Bv }\).  That is

\begin{dmath}\label{eqn:introGAproblems:160}
\Bx^2
=
\lr{ u \Bu + v \Bv }
\lr{ u \Bu + v \Bv }
=
u^2 \Bu^2
+ v^2 \Bv^2
+ u v \lr{ \Bu \Bv + \Bv \Bu }
=
u^2
+ v^2
+ u v \lr{ \Bu \Bv + \Bv \Bu }.
\end{dmath}

If these vectors are normal \( \Bx^2 = u^2 + v^2 \), which means
\begin{dmath}\label{eqn:introGAproblems:180}
\Bu \Bv = -\Bv \Bu.
\end{dmath}

Observe that a side effect of this computation shows that the traditional vector dot product of two unit vectors can also be written as a symmetric bivector sum

\begin{dmath}\label{eqn:introGAproblems:200}
\Bu \cdot \Bv = \inv{2} \lr{ \Bu \Bv + \Bv \Bu }.
\end{dmath}
} % answer

   %
% Copyright © 2016 Peeter Joot.  All Rights Reserved.
% Licenced as described in the file LICENSE under the root directory of this GIT repository.
%

\makeproblem{\R{n} dot product.}{problem:gradeselection:RnDotProduct}{
Show that \ref{dfn:gradeselection:100} when applied to two vectors
is equivalent to the traditional \R{n} dot product.
} % problem

\makeanswer{problem:gradeselection:RnDotProduct}{
Let
\begin{dmath}\label{eqn:gradeselectionProblems:180}
\begin{aligned}
\Bx &= \sum_{i=1}^N x_i \Be_i \\
\By &= \sum_{i=1}^N y_i \Be_i.
\end{aligned}
\end{dmath}

The dot product of these two vectors is
\begin{dmath}\label{eqn:gradeselectionProblems:200}
\Bx \cdot \By
\equiv
\gpgradezero{ \Bx \By }
=
\gpgradezero{
\lr{ \sum_{i=1}^N x_i \Be_i}
\lr{ \sum_{j=1}^N y_j \Be_j}
}
=
\sum_{1 \le i = j \le N}
x_i y_j
\gpgradezero{ \Be_i \Be_j }
+
\sum_{1 \le i \ne j \le N}
x_i y_j
\gpgradezero{ \Be_i \Be_j }
\end{dmath}

In the \( i = j \) sum, the term \( \Be_i \Be_j = \Be_i^2 = 1 \), so the scalar grade selection of that multivector product is just 1.  In the \( i = j \) term, each of the \( \Be_i \Be_j \) products is a bivector, so each of those scalar grade selections is zero.

That leaves

\begin{dmath}\label{eqn:gradeselectionProblems:220}
\Bx \cdot \By
=
\sum_{i =1}^N x_i y_i. \qedmarker
\end{dmath}
} % answer

   %
% Copyright © 2016 Peeter Joot.  All Rights Reserved.
% Licenced as described in the file LICENSE under the root directory of this GIT repository.
%
\makeproblem{Dot product of vectors as symmetric sum}{problem:gradeselection:dotprod}{
Show that the dot product of two vectors can be written as a symmetric sum

\begin{dmath}\label{eqn:gradeselection:600}
\Bx \cdot \By = \inv{2} \lr{ \Bx \By + \By \Bx }.
\end{dmath}
} % problem

\makeanswer{problem:gradeselection:dotprod}{
There are a few ways that this can be demonstrated.  The first relies on the classical definition of the dot product.  Expanding the square of a vector sum gives

\begin{dmath}\label{eqn:gradeselectionProblems:700}
(\Bx + \By)^2 = \Bx^2 + \By^2 + \Bx \By + \By \Bx.
\end{dmath}

By comparison this must also be equal to

\begin{dmath}\label{eqn:gradeselectionProblems:720}
\Norm{\Bx + \By}^2 = \Bx^2 + \By^2 + 2 \Bx \cdot \By,
\end{dmath}

so
\begin{dmath}\label{eqn:gradeselectionProblems:740}
\Bx \By + \By \Bx = 2 \Bx \cdot \By.
\end{dmath}

This might be viewed as a cheat, since it is not using the dot product as defined by grade zero selection according to \cref{dfn:gradeselection:100}.  Using that definition will produce the same result

\begin{dmath}\label{eqn:gradeselectionProblems:760}
\gpgradezero{ (\Bx + \By)^2 }
=
\gpgradezero{ \Bx^2 + \By^2 + \Bx \By + \By \Bx }
=
\Bx^2 + \By^2
+
\gpgradezero{
\Bx \By }
+ \gpgradezero{ \By \Bx }.
\end{dmath}

It was shown in \cref{problem:gradeselection:cyclicpermutationtwo} that \( \gpgradezero{ \Bx \By } = \gpgradezero{ \By \Bx } \) so
\begin{dmath}\label{eqn:gradeselectionProblems:780}
2 \gpgradezero{ \Bx \By } = \Bx \By + \By \Bx.
\end{dmath}

Using \cref{dfn:gradeselection:100}, this completes the problem.
} % answer

   %
% Copyright © 2016 Peeter Joot.  All Rights Reserved.
% Licenced as described in the file LICENSE under the root directory of this GIT repository.
%
\makeproblem{Plane rotations.}{problem:gradeselection:PlaneRotations}{

With \( i = \Be_1 \Be_2 \) for the pseudoscalar of the \( x,y \) plane,

\makesubproblem{}{problem:gradeselection:3:b}
justify the assertion that \( e^{i \theta} = \cos\theta + i \sin\theta \), where \( theta \) is a scalar angle.

\makesubproblem{}{problem:gradeselection:3:c}
Show that right multiplication of a 2D vector by \( e^{i\theta} \) rotates that vector by \( \theta \) radians.

\makesubproblem{}{problem:gradeselection:3:d}
Does the rotation multivector \( e^{i\theta} \) commute with the 2D basis vectors?

\makesubproblem{}{problem:gradeselection:3:e}
What is the action of multiplication of a vector by \( e^{i\theta} \) from the left?
} % problem

\makeanswer{problem:gradeselection:PlaneRotations}{
\makeSubAnswer{}{problem:gradeselection:3:b}

Assume that the exponential of a multivector argument is represented by a Taylor series

\begin{dmath}\label{eqn:gradeselectionProblems:280}
e^X = \sum_{k = 0}^\infty \frac{X^k}{k!},
\end{dmath}

and note that the pseudoscalar commutes with scalar rotation angles \( \theta \), so
\begin{dmath}\label{eqn:gradeselectionProblems:300}
e^{i\theta}
= \sum_{k = 0}^\infty \frac{(i\theta)^k}{k!}
= \sum_{k = 0}^\infty \frac{i^k\theta^k}{k!}
=
\sum_{k = 0}^\infty \frac{i^{2k}\theta^{2k}}{(2k)!}
+
\sum_{k = 0}^\infty \frac{i^{2k + 1}\theta^{2k +1}}{(2k + 1)!}
=
\sum_{k = 0}^\infty \frac{(-1)^{k}\theta^{2k}}{(2k)!}
+
i \sum_{k = 0}^\infty \frac{(-1)^{k}\theta^{2k +1}}{(2k + 1)!}
= \cos \theta + i \sin\theta.
\end{dmath}
\makeSubAnswer{}{problem:gradeselection:3:c}

Consider the action of the exponential on each of the unit vectors.  For \( \Be_1 \) that is

\begin{dmath}\label{eqn:gradeselectionProblems:320}
\Be_1 e^{i \theta}
=
\Be_1 \lr{ \cos\theta + i \sin\theta }
=
\Be_1 \cos\theta + \Be_1 (\Be_1 \Be_2 )\sin\theta
=
\Be_1 \cos\theta + \Be_2 \sin\theta.
\end{dmath}

This shows that the vector \( \Be_1 \) is rotated counterclockwise by \( \theta \) radians.  Similarly for \( \Be_2 \)

\begin{dmath}\label{eqn:gradeselectionProblems:340}
\Be_2 e^{i \theta}
=
\Be_2 \lr{ \cos\theta + i \sin\theta }
=
\Be_2 \cos\theta + \Be_1 (\Be_1 \Be_2 )\sin\theta
=
\Be_2 \cos\theta + \Be_1 (-\Be_2 \Be_1) \sin\theta.
=
\Be_2 \cos\theta - \Be_1 \sin\theta.
\end{dmath}

This is also a rotation by \( \theta \) radians.  Given a vector \( \Bx = x \Be_1 + y \Be_2 \), this gives

\begin{dmath}\label{eqn:gradeselectionProblems:360}
\Bx'
= \Bx e^{i\theta}
=
x \lr{ \Be_1 \cos\theta + \Be_2 \sin\theta } + y \lr{ \Be_2 \cos\theta - \Be_1 \sin\theta }.
\end{dmath}

In particular

\begin{dmath}\label{eqn:gradeselectionProblems:380}
\begin{bmatrix}
\Bx' \cdot \Be_1 \\
\Bx' \cdot \Be_2 \\
\end{bmatrix}
=
\begin{bmatrix}
x \cos\theta - y \sin\theta \\
x \sin\theta + y \cos\theta
\end{bmatrix}
=
\begin{bmatrix}
\cos\theta &- \sin\theta \\
\sin\theta &+ \cos\theta
\end{bmatrix}
\begin{bmatrix}
x \\
y
\end{bmatrix}.
\end{dmath}

Observe that this is the rotation matrix that takes the points \((x, y)\) to their position \((x', y')\) rotated by \( \theta \) radians.
\makeSubAnswer{}{problem:gradeselection:3:d}

The action from the left on \( \Be_1 \) is

\begin{dmath}\label{eqn:gradeselectionProblems:400}
e^{i\theta} \Be_1
=
\lr{ \cos\theta + \Be_1 \Be_2 \sin\theta} \Be_1
=
\Be_1 \cos\theta + \Be_1 \Be_2 \Be_1 \sin\theta
=
\Be_1 \cos\theta + \Be_1 (-\Be_1 \Be_2) \sin\theta
=
\Be_1 \lr{ \cos\theta - i \sin\theta }
=
\Be_1 e^{-i\theta},
\end{dmath}

and the action from the left on \( \Be_2 \) is

\begin{dmath}\label{eqn:gradeselectionProblems:420}
e^{i\theta} \Be_2
=
\lr{ \cos\theta + \Be_1 \Be_2 \sin\theta} \Be_2
=
\Be_2 \cos\theta + \Be_1 \sin\theta
=
\Be_2 \cos\theta + (\Be_2 \Be_2) \Be_1 \sin\theta
=
\Be_2 \lr{ \cos\theta - i \sin\theta }
=
\Be_2 e^{-i\theta}.
\end{dmath}

This change of sign is due to the fact that the pseudoscalar anticommutes with each of the basis vectors in the plane.

\makeSubAnswer{}{problem:gradeselection:3:e}

Since the exponential toggles sign on commutation with both of the vectors of the plane, the rotation operation can be applied from either left or right, with sufficient care to get the direction right

\begin{equation}\label{eqn:gradeselectionProblems:440}
\Bx e^{i\theta} = e^{-i\theta} \Bx.
\end{equation}

It is also possible to split the rotation operation into half angle rotation operators that act from both the left and right

\begin{dmath}\label{eqn:gradeselectionProblems:460}
\Bx' = e^{-i\theta/2} \Bx e^{i\theta/2}.
\end{dmath}

A student who has studied computer graphics rotation theory may have seen quaternion rotation operators with this form, and a student of quantum mechanics will have seen Pauli matrix rotation operations of this form.  This is, in fact, the form that is generally desirable for 3D or higher order rotations, since it rotates the portions of a vector that lie in the rotation plane, leaving the normal components untouched.
} % answer

   %
% Copyright © 2016 Peeter Joot.  All Rights Reserved.
% Licenced as described in the file LICENSE under the root directory of this GIT repository.
%
\makeproblem{Complex numbers}{problem:gradeselection:ComplexNumbers}{
Show that complex numbers can be represented as even grade multivectors \( z = a + \Be_1 \Be_2 b \).
} % problem

\makeanswer{problem:gradeselection:ComplexNumbers}{
Let \( i = \Be_1 \Be_2 \), for which we find

\begin{dmath}\label{eqn:gradeselectionProblems:260}
i^2
=
\lr{ \Be_1 \Be_2 }
\lr{ \Be_1 \Be_2 }
=
\Be_1 (\Be_2 \Be_1) \Be_2
=
\Be_1 (-\Be_1 \Be_2) \Be_2
=
-(\Be_1^2) (\Be_2^2)
=
-1.
\end{dmath}

The even grade multivector \( z = a + i b \) is thus seen to have all the properties required of complex numbers.
} % answer

   %
% Copyright © 2016 Peeter Joot.  All Rights Reserved.
% Licenced as described in the file LICENSE under the root directory of this GIT repository.
%
\makeproblem{\R{3} pseudoscalar commutation.}{problem:gradeselection:R3PseudoscalarCommutation}{
Show that \( I \) given by \cref{eqn:definitions:340}
commutes with any grade \R{3} multivector.
} % problem

\makeanswer{problem:gradeselection:R3PseudoscalarCommutation}{

Showing that \( I \) commutes with each of the basis vectors is sufficient

\begin{dmath}\label{eqn:gradeselectionProblems:620}
\Be_1 I
=
\Be_1 (\Be_1 \Be_2 \Be_3)
=
\Be_1 (-\Be_2 \Be_1) \Be_3
=
-\Be_1 \Be_2 (-\Be_3 \Be_1)
=
I \Be_1
\end{dmath}
\begin{dmath}\label{eqn:gradeselectionProblems:640}
\Be_2 I
=
\Be_2 (\Be_1 \Be_2 \Be_3)
=
\Be_2 \Be_1 (-\Be_3 \Be_2)
=
-(-\Be_1 \Be_2) \Be_3 \Be_2
=
I \Be_2.
\end{dmath}
\begin{dmath}\label{eqn:gradeselectionProblems:660}
\Be_3 I
=
\Be_3 (\Be_1 \Be_2 \Be_3)
=
(\Be_3 \Be_1 \Be_2) \Be_3
=
-(\Be_1 \Be_3) \Be_2 \Be_3
=
-\Be_1 (-\Be_2 \Be_3) \Be_3
=
I \Be_3. \qedmarker
\end{dmath}
} % answer

   %
% Copyright © 2016 Peeter Joot.  All Rights Reserved.
% Licenced as described in the file LICENSE under the root directory of this GIT repository.
%
\makeproblem{Vector wedge coordinate expansion and antisymmetry}{problem:gradeselection:vectorwedge}{
Show that
\begin{dmath}\label{eqn:gradeselection:620}
\Bx \wedge \By
=
%\sum_{i < j} (x_i y_j - x_j y_i) \Be_i \Be_j.
\sum_{i < j}
\begin{vmatrix}
x_i & x_j \\
y_i & y_j
\end{vmatrix}
\Be_i \Be_j.
\end{dmath}

Observe from this coordinate expansion that the wedge product of two vectors is antisymmetric
\boxedEquation{eqn:gradeselection:640}{
\Bx \wedge \By = -\By \wedge \Bx.
}
} % problem

\makeanswer{problem:gradeselection:vectorwedge}{
Given \( \Bx = \sum_i x_i \Be_i \), and \( \By = \sum_i y_i \Be_i \), the wedge of these two vectors is a grade two selection that picks out only products that differ in index

\begin{dmath}\label{eqn:gradeselectionProblems:580}
\Bx \wedge \By
=
\gpgradetwo{ \Bx \By }
=
\sum_{i,j} \gpgradetwo{ x_i \Be_i y_j \Be_j }
=
\sum_{i \ne j} x_i y_j \gpgradetwo{ \Be_i \Be_j }
=
\sum_{i \ne j} x_i y_j \Be_i \Be_j
=
\sum_{i < j} (x_i y_j - x_j y_i) \Be_i \Be_j
=
\sum_{i < j}
\begin{vmatrix}
x_i & x_j \\
y_i & y_j
\end{vmatrix}
\Be_i \Be_j.
\end{dmath}

When \( \Bx, \By \), the rows in each of the above determinants will swap, negating the sign of each.  This implies \( \By \wedge \Bx = -\Bx \wedge \By \).
} % answer

   %
% Copyright © 2016 Peeter Joot.  All Rights Reserved.
% Licenced as described in the file LICENSE under the root directory of this GIT repository.
%

\makeproblem{Wedge relationship to the cross product.}{problem:gradeselection:WedgeRelationshipToCrossProduct}{
For a pair of \R{3} vectors \( \Bx, \By \), show that the wedge and cross products are related by
\begin{dmath}\label{eqn:gradeselectionProblems:560}
\Bx \wedge \By = I (\Bx \cross \By),
\end{dmath}

where \( I = \Be_1 \Be_2 \Be_3 \) is the \R{3} pseudoscalar.
} % problem

\makeanswer{problem:gradeselection:WedgeRelationshipToCrossProduct}{
Writing out \cref{eqn:gradeselectionProblems:580} explicitly gives

\begin{dmath}\label{eqn:gradeselectionProblems:600}
\Bx \wedge \By
=
\begin{vmatrix}
x_1 & x_2 \\
y_1 & y_2
\end{vmatrix}
\Be_1 \Be_2
+
\begin{vmatrix}
x_1 & x_3 \\
y_1 & y_3
\end{vmatrix}
\Be_1 \Be_3
+
\begin{vmatrix}
x_2 & x_3 \\
y_2 & y_3
\end{vmatrix}
\Be_2 \Be_3
=
\begin{vmatrix}
  \Be_2 \Be_3
& \Be_3 \Be_1
& \Be_1 \Be_2 \\
x_1 & x_2 & x_3 \\
y_1 & y_2 & y_3
\end{vmatrix}
=
\begin{vmatrix}
  (\Be_1 \Be_1) \Be_2 \Be_3
& \Be_3 \Be_1 (\Be_2 \Be_2)
& \Be_1 \Be_2 (\Be_3 \Be_3) \\
x_1 & x_2 & x_3 \\
y_1 & y_2 & y_3
\end{vmatrix}
=
\Be_1 \Be_2 \Be_3
\begin{vmatrix}
\Be_1 & \Be_2 & \Be_3 \\
x_1 & x_2 & x_3 \\
y_1 & y_2 & y_3
\end{vmatrix}
= I (\Bx \cross \By).
\end{dmath}
%\begin{aligned}
%(x_1 y_2 - x_2 y_1) \Be_1 \Be_2 \\
%&\quad+
%(x_1 y_3 - x_3 y_1) \Be_1 \Be_3 \\
%&\quad+
%(x_2 y_3 - x_3 y_2) \Be_2 \Be_3 \\
%&=
%(x_1 y_2 - x_2 y_1) \Be_1 \Be_2 (\Be_3 \Be_3) \\
%&\quad+
%(x_1 y_3 - x_3 y_1) \Be_1 \Be_3 (\Be_2 \Be_2) \\
%&\quad+
%(x_2 y_3 - x_3 y_2) \Be_2 \Be_3 (\Be_1 \Be_1) \\
%&=
%(x_1 y_2 - x_2 y_1) I \Be_3 \\
%&\quad+
%(x_1 y_3 - x_3 y_1) (-I) \Be_2 \\
%&\quad+
%(x_2 y_3 - x_3 y_2) I \Be_1 \\
%\end{aligned}
} % answer

   %
% Copyright © 2016 Peeter Joot.  All Rights Reserved.
% Licenced as described in the file LICENSE under the root directory of this GIT repository.
%
\makeproblem{Commutation within grade zero selection}{problem:vectorproduct:cyclicpermutation}{

It was previously shown using coordinates that

\begin{dmath}\label{eqn:vectorproduct:260}
\gpgradezero{ \Bx \By } = \gpgradezero{ \By \Bx }.
\end{dmath}

Repeat this proof using \cref{eqn:vectorproduct:80}.
} % problem

\makeanswer{problem:vectorproduct:cyclicpermutation}{
\begin{dmath}\label{eqn:vectorproduct:301}
\gpgradezero{ \By \Bx }
=
\gpgradezero{ \By \cdot \Bx  + \By \wedge \Bx }
=
\gpgradezero{ \By \cdot \Bx }
=
\gpgradezero{ \Bx \cdot \By }
=
\gpgradezero{ \Bx \By }
\end{dmath}
} % answer

   %
% Copyright © 2016 Peeter Joot.  All Rights Reserved.
% Licenced as described in the file LICENSE under the root directory of this GIT repository.
%

\makeproblem{Vector wedge antisymmetric structure}{problem:vectorproduct:wedgeantisym}{
Prove the wedge product relationship of \cref{eqn:vectorproduct:300}.
} % problem

\makeanswer{problem:vectorproduct:wedgeantisym}{

Rearranging \cref{eqn:vectorproduct:80} for the wedge product and substitution of the dot product symmetric sum from \cref{problem:gradeselection:dotprod} gives

\begin{dmath}\label{eqn:gradeselectionProblems:800}
\Bx \wedge \By
= \Bx \By - \Bx \cdot \By
= \Bx \By - \inv{2} \lr{ \Bx \By + \By \Bx }
= \inv{2} \lr{ 2 \Bx \By - \Bx \By - \By \Bx }
= \inv{2} \lr{ \Bx \By - \By \Bx }.
\end{dmath}
} % answer

\section{Problem solutions}
   \shipoutAnswer
\section{more}
\subsection{manyfaces.tex}
   %
% Copyright © 2018 Peeter Joot.  All Rights Reserved.
% Licenced as described in the file LICENSE under the root directory of this GIT repository.
%
%{
\subsubsection{The many faces of electromagnetism.}

There is a long history of attempts to find more elegant, compact and powerful ways of encoding and working with Maxwell's equations.

\paragraph{Maxwell's formulation.}
Maxwell
\citep{maxwell1881treatiseVolumeI}
employs some differential operators, including the
gradient \( \spacegrad \) and Laplacian \( \spacegrad^2 \), but the divergence and gradient
are always written out in full using coordinates, usually in integral form.
Reading the original Treatise highlights how important notation can be, as
most modern engineering or physics practitioners would find his original work incomprehensible.
A nice translation from Maxwell's notation to the modern Heaviside-Gibbs notation can be found in
\citep{chappell2014geometricAppendixA}.

\paragraph{Quaterion representation.}
In his second volume
\citep{maxwell1881treatiseVolumeII} the equations of electromagnetism are stated
using quaternions (an extension of complex numbers to three dimensions), but quaternions are not used in the work.
The modern form of Maxwell's equations in quaternion form is
\begin{dmath}\label{eqn:manyfaces:220}
\begin{aligned}
\inv{2} \antisymmetric{ \frac{d}{dr} }{ \BH } - \inv{2} \symmetric{ \frac{d}{dr} } { c \BD } &= c \rho + \BJ \\
\inv{2} \antisymmetric{ \frac{d}{dr} }{ \BE } + \inv{2} \symmetric{ \frac{d}{dr} }{ c \BB } &= 0,
\end{aligned}
\end{dmath}
where \( \ifrac{d}{dr} = (1/c) \PDi{t}{} + \Bi \PDi{x}{} + \Bj \PDi{y}{} + \Bk \PDi{z}{} \) \citep{jack2003physical} acts bidirectionally, and vectors are expressed in terms of the quaternion basis \( \setlr{ \Bi, \Bj, \Bk } \), subject to the relations \(
\Bi^2 = \Bj^2 = \Bk^2 = -1, \quad
\Bi \Bj = \Bk = -\Bj \Bi, \quad
\Bj \Bk = \Bi = -\Bk \Bj, \quad
\Bk \Bi = \Bj = -\Bi \Bk \).
There is clearly more structure to these equations than the traditional Heaviside-Gibbs representation that we are used to, which says something for the quaternion model.
However, this structure requires notation that is arguably non-intuitive.
The
fact that the quaternion representation was abandoned long ago by most electromagnetism researchers and engineers supports such an argument.

\paragraph{Minkowski tensor representation.}
Minkowski introduced the concept of a complex time coordinate \( x_4 = i c t \) for special relativity \citep{einstein2015relativityMinkowski}.
Such a four-vector representation can be used for many of the relativistic four-vector pairs of electromagnetism, such as the current \((c\rho, \BJ)\), and the energy-momentum Lorentz force relations, and can also be applied to Maxwell's equations
\begin{equation}\label{eqn:manyfaces:140}
\sum_{\mu= 1}^4 \PD{x_\mu}{F_{\mu\nu}} = - 4 \pi j_\nu.
\qquad
\sum_{\lambda\rho\mu=1}^4
\epsilon_{\mu\nu\lambda\rho}
\PD{x_\mu}{F_{\lambda\rho}} = 0,
\end{equation}
where
\begin{dmath}\label{eqn:manyfaces:160}
F
=
\begin{bmatrix}
0 & B_z & -B_y & -i E_x \\
-B_z & 0 & B_x & -i E_y \\
B_y & -B_x & 0 & -i E_z \\
i E_x & i E_y & i E_z & 0
\end{bmatrix}.
\end{dmath}
A rank-2 complex (Hermitian) tensor contains all six of the field components.
Transformation of coordinates for this representation of the field may be performed exactly
like the transformation for any other four-vector.
This formalism is described nicely in \citep{schwartz1987pe}, where the structure used is motivated by transformational requirements.
One of the costs of this tensor representation is that we loose the clear separation of the electric and magnetic fields that we are so comfortable with.
Another cost is that we loose the
distinction between space and time,
as separate space and time coordinates have to be projected out of a larger four vector.
Both of these costs have theoretical benefits in some applications, particularly for high energy
problems where relativity is important, but for the low velocity problems
near and dear to electrical engineers who can freely treat space and time independently,
the advantages are not clear.

\paragraph{Modern tensor formalism.}
The Minkowski representation fell out of favour in theoretical physics, which settled on a real tensor representation that utilizes an explicit metric tensor \( g_{\mu\nu} = \pm \diag(1, -1, -1, -1) \) to represent the complex inner products of special relativity.
In this tensor formalism, Maxwell's equations are also reduced to a set of two tensor relationships (\citep{landau1980classical},
\citep{jackson1975cew},
\citep{griffiths1999introduction}).
\begin{dmath}\label{eqn:manyfaces:40}
\begin{aligned}
\partial_\mu F^{\mu \nu} &= \mu_0 J^\nu \\
\epsilon^{\alpha \beta \mu \nu} \partial_\beta F_{\mu \nu} &= 0,
\end{aligned}
\end{dmath}
where \( F^{\mu\nu} \) is a \textit{real} rank-2 antisymmetric tensor that contains all six electric and magnetic field components, and \( J^\nu \) is a four-vector current containing both charge density and current density components.
\Cref{eqn:manyfaces:40} provides a unified and simpler theoretical framework for electromagnetism, and is used extensively in physics but not engineering.

\paragraph{Differential forms.}
It has been argued that a differential forms treatment of
electromagnetism provides some of the same theoretical advantages as the tensor formalism, without the disadvantages
of introducing a hellish mess of index manipulation into the mix.
With differential forms it is also possible to express Maxwell's equations as two equations.
The free-space differential forms equivalent
\citep{flanders1989dfa}
to \cref{eqn:manyfaces:40} is
\begin{dmath}\label{eqn:manyfaces:60}
\begin{aligned}
d \alpha &= 0 \\
d *\alpha &= 0,
\end{aligned}
\end{dmath}
where
\begin{dmath}\label{eqn:manyfaces:180}
\alpha = \lr{ E_1 dx^1 + E_2 dx^2 + E_3 dx^3 }(c dt) + H_1 dx^2 dx^3 + H_2 dx^3 dx^1 + H_3 dx^1 dx^2.
\end{dmath}
One of the advantages of this representation is that it is valid even for curvilinear coordinate representations, which are handled naturally in differential forms.
However, this formalism also comes with a number of costs.
One cost (or benefit), like that of the tensor formalism, is that this is implicitly a relativistic
approach subject to non-Euclidean orthonormality conditions \( (dx^i, dx^j) = \delta^{ij}, (dx^i, c dt) = 0, (c dt, c dt) = -1 \).
Most grievous of the costs is the requirement to use differentials \( dx^1, dx^2, dx^3, c dt \), instead of a more familiar set of basis vectors, even for non-curvilinear coordinates.
This requirement is easily viewed as unnatural, and likely one of the reasons that electromagnetism with differential forms has never become popular.

\paragraph{Vector formalism.}
Euclidean vector algebra, in particular the vector algebra and calculus of \R{3},
is the de-facto language of electrical engineering for electromagnetism.
Maxwell's equations in the Heaviside-Gibbs vector formalism are
\begin{dmath}\label{eqn:manyfaces:20}
\begin{aligned}
%\spacegrad \cross \BE &= - \BM - \PD{t}{\BB} \\
\spacegrad \cross \BE &= - \PD{t}{\BB} \\
\spacegrad \cross \BH &= \BJ + \PD{t}{\BD} \\
\spacegrad \cdot \BD &= \rho \\
%\spacegrad \cdot \BB &= \rho_\txtm.
\spacegrad \cdot \BB &= 0.
\end{aligned}
\end{dmath}
We are all intimately familiar with these equations, with the dot and the cross products, and with gradient, divergence and curl operations that are used to express them.
Given how comfortable we are with this mathematical formalism, there has to be a really good reason to switch to something else.

\paragraph{Space time algebra (geometric algebra).}
An alternative to any of the electrodynamics formalisms described above is
STA, the \textit{Space Time Algebra}.
STA is a relativistic \textit{geometric algebra} that allows Maxwell's equations \cref{eqn:manyfaces:40} to be combined into one equation (\citep{doran2003gap}, \citep{hestenes1966space})
\begin{dmath}\label{eqn:manyfaces:80}
\grad F = J,
\end{dmath}
where
\begin{equation}\label{eqn:manyfaces:200}
F = \BE + I c \BB \qquad (= \BE + I \eta \BH)
\end{equation}
is a bivector
field containing both the electric and magnetic field ``vectors'', \( \grad = \gamma^\mu \partial_\mu \) is the spacetime gradient, \( J \) is a four vector containing electric charge and current components, and \( I = \gamma_0 \gamma_1 \gamma_2 \gamma_3 \) is the spacetime pseudoscalar, the ordered product of the basis vectors \( \setlr{ \gamma_\mu } \).
The STA representation is explicitly relativistic with a non-Euclidean relationships between the basis vectors \( \gamma_0 \cdot \gamma_0 = 1 = -\gamma_k \cdot \gamma_k, \forall k > 0 \).
In this formalism ``spatial'' vectors \( \Bx = \sum_{k>0} \gamma_k \gamma_0 x^k \) are represented as spacetime bivectors, requiring a small slight of hand when switching between STA notation and conventional vector representation.
Uncoincidentally \( F \) has exactly the same structure as the 2-form \(\alpha\) in \cref{eqn:manyfaces:180}, provided the differential 1-forms \( dx^\mu \) are replaced by the basis vectors \( \gamma_\mu \).
However, there is a simple complex structure inherent in \cref{eqn:manyfaces:200} that is not obvious in the 2-form equivalent.
The bivector representation of the field \( F \) directly encodes the antisymmetric nature of \( F^{\mu\nu} \) from the tensor formalism, and the tensor equivalents of most STA results can be calculated easily.

Having a single PDE for all of Maxwell's equations allows for direct Green's function solution of the field, and has a number of other advantages.
There is extensive literature exploring selected applications of STA to electrodynamics.
Many theoretical results have been derived using this formalism that require significantly more complex approaches using conventional vector or tensor analysis.
Unfortunately, much of the STA literature is inaccessible to the engineering student, practising engineers, or engineering instructors.
To even start reading the literature, one must learn geometric algebra, aspects of special relativity and non-Euclidean geometry, generalized integration theory, and even some tensor analysis.

\paragraph{Paravector formalism (geometric algebra).}
In the geometric algebra literature, there are a few authors who have endorsed the use of Euclidean geometric algebras for relativistic applications (\citep{baylis2004electrodynamics},
\citep{chappell2010simplified})
These authors use an Euclidean basis ``vector'' \( \Be_0 = 1 \) for the timelike direction, along with a standard Euclidean basis \( \setlr{ \Be_i } \) for the spatial directions.
A hybrid scalar plus vector representation of four vectors, called paravectors is employed.
Maxwell's equation is written as a multivector equation
\begin{dmath}\label{eqn:manyfaces:120}
\lr{ \spacegrad + \inv{c} \PD{t}{} } F = J,
\end{dmath}
where \( J \) is a multivector source containing both the electric charge and currents, and \( c \) is the group velocity for the medium (assumed uniform and isometric).
\( J \) may optionally include the (fictitious) magnetic charge and currents useful in antenna theory.
Like STA the paravector formalism uses a the hybrid electromagnetic field representation of \cref{eqn:manyfaces:200},
however,
\( I = \Be_1 \Be_2 \Be_3 \) is interpreted as the \R{3} pseudoscalar, the ordered product of the basis vectors \( \setlr{ \Be_i } \), and \( F \) represents a multivector with vector and bivector
components.
Unlike STA where \( \BE \) and \( \BB \) (or \( \BH \)) are interpreted as spacetime bivectors, here
they are plain old Euclidean vectors in \R{3}, entirely consistent with conventional Heaviside-Gibbs notation.
Like \cref{eqn:manyfaces:80}, \cref{eqn:manyfaces:120} is directly invertible using Green's function techniques, without requiring the solution of equivalent second order potential problems, nor any requirement to take the derivatives of those potentials to determine the fields.

Lorentz transformation and manipulation of paravectors requires a
variety of conjugation, real and imaginary operators, unlike STA where such operations have the same complex exponential structure as any 3D rotation expressed in geometric algebra.
The advocates of the paravector representation argue that this provides an effective pedagogical bridge from Euclidean geometry to the Minkowski geometry of special relativity.
This author agrees that this form of Maxwell's equations is the natural choice for an introduction to electromagnetism using geometric algebra, but
for relativistic operations, STA is a much more natural and less confusing choice.

\subsubsection{Results.}
The end product of this project was a fairly small self contained book, titled ``Geometric Algebra for Electrical Engineers''.
This book includes
an introduction to Euclidean geometric algebra focused on \R{2} and \R{3} (64 pages), an introduction to geometric calculus and multivector Green's functions (64 pages), and applications to electromagnetism (75 pages).
This report summarizes results from this book, omitting most derivations, and attempts to provide an overview that may be used as a road map for the book for further exploration.
Many of the fundamental results of electromagnetism are derived directly from the geometric algebra form of Maxwell's equation in a streamlined and compact fashion.
This includes some new results, and many of the existing non-relativistic results from the geometric algebra STA and paravector literature.
It will be clear to the reader that it is often simpler to have the electric and magnetic on equal footing, and demonstrates this by deriving most results in terms of the
total electromagnetic field \( F \).
Many examples of how to extract the conventional electric and magnetic fields from the geometric algebra results expressed in terms of \( F \) are given as a bridge between the multivector and vector representations.

The aim of this work was to remove some of the prerequisite conceptual roadblocks that make electromagnetism using geometric algebra inaccessible.
In particular, this project explored non-relativistic applications of geometric algebra to electromagnetism.
After derivation from the conventional Heaviside-Gibbs representation of Maxwell's equations, the
paravector representation
\cref{eqn:manyfaces:120} is used as the starting point for of all subsequent analysis.
However, the paravector literature includes a confusing set of conjugation and real and imaginary selection operations that are tailored for relativistic applications.
These are not necessary for low velocity applications, and have been
avoided completely with the aim of making the subject more accessibility to the engineer.

In the book an attempt has been made to avoid introducing as little new notation as possible.
For example, some authors use special notation for the bivector valued magnetic field \( I \BB \), such as \( \bcap \) or \( \Bcap \).
Given the inconsistencies in the literature, \( I \BB \) (or \( I \BH \)) will be used explicitly for the bivector (magnetic) components of the total electromagnetic field \( F \).
In the geometric algebra literature, there are conflicting conventions for the operator \( \spacegrad + (1/c) \PDi{t}{} \),
\footnote{Examples of different notations can be found in: \citep{jancewicz1988multivectors}, \citep{baylis2004electrodynamics}, and \citep{chappell2014geometric}.}
which we will call the spacetime gradient after the STA equivalent.
In the book the spacetime gradient is always written out in full to avoid picking from or explaining some of the subtleties of the competing notations.

Some researchers will find it distasteful that STA and relativity have been avoided completely in this book.
Maxwell's equations are inherently relativistic, and
STA expresses the relativistic aspects of electromagnetism in an exceptional and beautiful fashion.
However, a student of this book will have learned the geometric algebra and calculus prerequisites of STA.
This makes the STA literature much more accessible, especially since most of the
results in the book can be trivially translated into STA notation.
%}

\subsection{waveequation.tex}
   %
% Copyright © 2016 Peeter Joot.  All Rights Reserved.
% Licenced as described in the file LICENSE under the root directory of this GIT repository.
%
%\section{Wave equation.}
\index{wave equation}
%Having assembled all of Maxwell's equations into \cref{dfn:isotropicMaxwells:680}, some results now follow almost trivially.
%One such result is the wave equation in space free of sources.
%In such a region, Maxwell's equation is just
%\begin{dmath}\label{eqn:waveequation:480}
%\lr{ \spacegrad + \inv{c} \PD{t}{} } F = 0.
%\end{dmath}
%
%This can be multiplied from the left with the multivector operator \( \spacegrad - \inv{c} \PD{t}{} \), to give
%\begin{dmath}\label{eqn:waveequation:500}
%0 =
%\lr{ \spacegrad - \inv{c} \PD{t}{} }
%\lr{ \spacegrad + \inv{c} \PD{t}{} } F
%=
%\lr{ \spacegrad^2 - \inv{c^2} \PDSq{t}{} } F,
%\end{dmath}
%or
In source free conditions
\begin{dmath}\label{eqn:waveequation:520}
\spacegrad^2 F = \inv{c^2} \PDSq{t}{F}.
\end{dmath}

Since \( \spacegrad^2 \) is a scalar operator, selection of the vector and bivector components of \cref{eqn:waveequation:520} gives
\begin{dmath}\label{eqn:waveequation:540}
\begin{aligned}
\spacegrad^2 \BE &= \inv{c^2} \PDSq{t}{\BE} \\
\spacegrad^2 (I \BH) &= \inv{c^2} \PDSq{t}{(I \BH)}.
\end{aligned}
\end{dmath}

These equations can be solved independently, provided the solutions are also constrained by Maxwell's equation \cref{eqn:waveequation:480}.

\subsection{electrostatics.tex}
   %
% Copyright © 2017 Peeter Joot.  All Rights Reserved.
% Licenced as described in the file LICENSE under the root directory of this GIT repository.
%
(cut)

\begin{subequations}
\label{eqn:electrostatics:99}
\begin{dmath}\label{eqn:electrostatics:100}
\spacegrad \cross \BE = 0
\end{dmath}
\begin{dmath}\label{eqn:electrostatics:120}
\spacegrad \cross \BB = 0
\end{dmath}
\begin{dmath}\label{eqn:electrostatics:140}
\spacegrad \cdot \BE = \frac{\rho}{\epsilon}
\end{dmath}
\begin{dmath}\label{eqn:electrostatics:160}
\spacegrad \cdot \BB = 0.
\end{dmath}
\end{subequations}

All the complicated coupling of the electric and magnetic fields is eliminated, and the only source term remaining is a time independent charge density \( \rho = \rho(\Bx) \).

Utilizing \cref{eqn:SimpleProducts2:1640}, the geometric product of the gradient \( \spacegrad \) with a vector \( \Ba = \Bx(\Bx) \) is
\begin{dmath}\label{eqn:electrostatics:240}
\spacegrad \Ba = \spacegrad \cdot \Ba + I(\spacegrad \cross \Ba).
\end{dmath}

\Cref{eqn:electrostatics:240} can be used to rewrite the electrostatic Maxwell equations (\cref{eqn:electrostatics:99}), as a pair of multivector gradient equations
%\begin{subequations}
%\label{eqn:electrostatics:360}
%\begin{equation}\label{eqn:electrostatics:380}
%\begin{equation}\label{eqn:electrostatics:400}
\boxedEquation{eqn:electrostatics:380}{
\begin{aligned}
\spacegrad \BE &= \frac{\rho}{\epsilon} \\
\spacegrad \BB &= 0.
\end{aligned}
}
%\end{subequations}


\subsection{Inverting the gradient equations.}
   %
% Copyright © 2017 Peeter Joot.  All Rights Reserved.
% Licenced as described in the file LICENSE under the root directory of this GIT repository.
%
\index{Green's function}
\index{convolution}
From \cref{thm:gradientGreensFunctionEuclidean:720}
%From \cref{thm:gradientGreensFunctionEuclidean:1}, the
\R{3} Green's function for the gradient (on an infinite spherical bounding surface) is
\begin{dmath}\label{eqn:electrostatics_invertingGradient:260}
G(\Bx, \Bx') = \inv{4 \pi} \frac{\Bx - \Bx'}{\Norm{\Bx - \Bx'}^3},
\end{dmath}
so the convolution that inverts the electric field gradient equation is
\begin{dmath}\label{eqn:electrostatics_invertingGradient:621}
\BE(\Bx)
= \int_V dV' G(\Bx, \Bx') \spacegrad' \BE(\Bx')
= \int_V dV' G(\Bx, \Bx') \lr{ \inv{\epsilon}\rho(\Bx') }
= \inv{4\pi} \int_V dV' \frac{\Bx - \Bx'}{ \Abs{\Bx - \Bx'}^3 } \lr{ \inv{\epsilon}\rho(\Bx') },
\end{dmath}
or
%\begin{dmath}\label{eqn:electrostatics_invertingGradient:340}
\boxedEquation{eqn:electrostatics_invertingGradient:340}{
\BE(\Bx) =
\inv{4 \pi \epsilon} \int dV' \rho(\Bx') \frac{\Bx - \Bx'}{\Norm{\Bx - \Bx'}^3},
}
%\end{dmath}
which is Coulomb's law.

The convolution for the magnetic field is trivial
\begin{dmath}\label{eqn:electrostatics_invertingGradient:641}
\BB(\Bx)
= \int_V dV' G(\Bx, \Bx') \spacegrad' \BB(\Bx')
= \int_V dV' G(\Bx, \Bx') (0),
\end{dmath}
so the magnetic field is zero everywhere
\begin{dmath}\label{eqn:electrostatics_invertingGradient:320}
\BB(\Bx) = 0.
\end{dmath}

%Question: would a non-zero magnetic field solution be possible if a Green's function for a finite bounded surface were to be used instead?


\subsection{magnetostatics.tex}
   %
% Copyright © 2017 Peeter Joot.  All Rights Reserved.
% Licenced as described in the file LICENSE under the root directory of this GIT repository.
%
\index{magnetostatics}
\index{time independence}
Magnetostatics is the study of Maxwell's equations where
a time independent restriction of the fields is imposed, and
it assumed that there are no static charge distributions.
For such constraints (and no magnetic sources) the free space Maxwell's equations are simply

\begin{subequations}
\label{eqn:magnetostatics:99}
\begin{dmath}\label{eqn:magnetostatics:100}
\spacegrad \cross \BE = 0
\end{dmath}
\begin{dmath}\label{eqn:magnetostatics:120}
\spacegrad \cross \BB = \mu \BJ
\end{dmath}
\begin{dmath}\label{eqn:magnetostatics:140}
\spacegrad \cdot \BE = 0
\end{dmath}
\begin{dmath}\label{eqn:magnetostatics:160}
\spacegrad \cdot \BB = 0.
\end{dmath}
\end{subequations}

\Cref{eqn:electrostatics:240} can be used to rewrite the magnetostatic Maxwell equations (\cref{eqn:magnetostatics:99}), as a pair of multivector gradient equations.
The electric field equation is just
\begin{equation}\label{eqn:magnetostatics:400}
\spacegrad \BE = 0,
\end{equation}
and for the magnetic field, we have
\begin{dmath}\label{eqn:magnetostatics:420}
\spacegrad \BB
=
\spacegrad \cdot \BB
+
I (\spacegrad \cross \BB).
\end{dmath}

As was the case in electrostatics, Maxwell's equations can be reduced to a pair of multivector gradient equations
%\begin{dmath}\label{eqn:magnetostatics:380}
\boxedEquation{eqn:magnetostatics:380}{
\begin{aligned}
\spacegrad \BB &= I \mu \BJ \\
\spacegrad \BE &= 0.
\end{aligned}
}
%\end{dmath}

Constraints must be imposed on the current density for \cref{eqn:magnetostatics:380} to be satisfied.
This can be seen by left multiplying with the gradient
\begin{dmath}\label{eqn:magnetostatics:440}
\spacegrad^2 \BB
= \mu I \spacegrad \BJ
= \mu I \lr{ \spacegrad \cdot \BJ + \spacegrad \wedge \BJ }
= \mu \lr{ I (\spacegrad \cdot \BJ) - \spacegrad \cross \BJ }.
\end{dmath}

The left hand side is a vector, whereas the right hand side has vector and pseudoscalar grades.
This means that magnetostatics conditions require the divergence of the current density to be zero
\begin{dmath}\label{eqn:magnetostatics:460}
\spacegrad \cdot \BJ = 0.
\end{dmath}

\subsection{electrostatics_enclosedCharge.tex}
   %
% Copyright © 2017 Peeter Joot.  All Rights Reserved.
% Licenced as described in the file LICENSE under the root directory of this GIT repository.
%
\index{electrostatics}
\index{time independence}
%The study of
%field and charge distributions that are independent of time is called electrostatics, a special case of the statics conditions already covered.
%In such a configuration Maxwell's equation is just
%\begin{dmath}\label{eqn:electrostatics_enclosedCharge:n}
%\spacegrad F = \frac{\rho}{\epsilon}.
%\end{dmath}
With such conditions, the free space representation of Maxwell's equations \cref{eqn:freespace:3399} with no magnetic sources is simply

In the electrostatics 

The charge in a volume can be related to the electric field by integrating \cref{eqn:electrostatics:380}
\begin{dmath}\label{eqn:electrostatics_enclosedCharge:420}
\int_V d^3 \Bx \spacegrad \BE = \inv{\epsilon} \int_V d^3 \Bx \rho(\Bx).
\end{dmath}

This is an oriented integral, where \( d^3 \Bx \) is a pseudoscalar volume element, such as
\( d^3 \Bx = (\Be_1 dx) \wedge (\Be_2 dy) \wedge (\Be_3 dz) = I dx dy dz \).

The LHS integral can be evaluated using
%the fundamental theorem of geometric calculus
\cref{thm:fundamentalTheoremOfCalculus:1}
\begin{dmath}\label{eqn:electrostatics_enclosedCharge:461}
\int_{\partial V} d^2 \Bx \BE = \frac{I}{\epsilon} \int_V dV \rho(\Bx).
\end{dmath}

An outward normal \( \ncap \) can be used to
parameterize the bivector surface area element \( d^2 \Bx = I \ncap dA \), which allows the pseudoscalar factors on both
sides to be cancelled
%\begin{dmath}\label{eqn:electrostatics_enclosedCharge:460}
\boxedEquation{eqn:electrostatics_enclosedCharge:460}{
\int_{\partial V} dA \ncap \BE = \frac{1}{\epsilon} \int_V dV \rho(\Bx).
}
%\end{dmath}

This is a multivector equation which must be simultaneously satisfied by its scalar and bivector components

\begin{subequations}
\label{eqn:electrostatics_enclosedCharge:481}
\begin{dmath}\label{eqn:electrostatics_enclosedCharge:501}
\int_{\partial V} dA \ncap \cdot \BE = \frac{1}{\epsilon} \int_V dV \rho(\Bx)
\end{dmath}
\begin{dmath}\label{eqn:electrostatics_enclosedCharge:521}
\int_{\partial V} dA \ncap \wedge \BE = 0.
\end{dmath}
\end{subequations}

\index{enclosed charge}
The first equation is the familiar relationship between the divergence and the enclosed charge, which could have been derived from \cref{eqn:electrostatics:140} directly.
The second provides a constraint on the tangential components of the field with respect to the enclosed volume, and could have been derived from
\cref{eqn:electrostatics:100} directly.
The multivector equation \cref{eqn:electrostatics_enclosedCharge:460} encodes both of these relationships, simultaneously incorporating the contributions of the Maxwell divergence and curl equations for the electric field, relating both to the enclosed charge.


\subsection{Biot-Savart law.}
   %
% Copyright © 2018 Peeter Joot.  All Rights Reserved.
% Licenced as described in the file LICENSE under the root directory of this GIT repository.
%
%{
\index{Biot-Savart law}
\index{Green's function!gradient}
The magnetostatic Maxwell equation \cref{eqn:magnetostatics:380} can be inverted directly using the Green's function for the gradient
\begin{dmath}\label{eqn:magnetostatics_biotSavart:40}
I \BB(\Bx)
= \int_V dV' G(\Bx, \Bx') \spacegrad' I \BB(\Bx').
\end{dmath}

This expansion can be simplified by inserting a no-op grade selection operation
\begin{dmath}\label{eqn:magnetostatics_biotSavart:680}
I \BB(\Bx)
= \gpgradetwo{ \int_V dV' G(\Bx, \Bx') \spacegrad' I \BB(\Bx') }
= \int_V dV' \gpgradetwo{ G(\Bx, \Bx') (-\mu \BJ(\Bx')) }
= \inv{4\pi} \int_V dV' \frac{\Bx - \Bx'}{ \Abs{\Bx - \Bx'}^3 } \wedge (-\mu \BJ(\Bx')),
\end{dmath}
or
\boxedEquation{eqn:magnetostatics:760}{
I \BB(\Bx)
= \frac{\mu}{4\pi} \int_V dV' \BJ(\Bx') \wedge \frac{\Bx - \Bx'}{ \Abs{\Bx - \Bx'}^3 }.
}

This is the Biot-Savart law in its GA form.
The traditional expression requires only a duality transformation \( \BJ \wedge \Bf = I ( \BJ \cross \Bf) \), or
\begin{dmath}\label{eqn:magnetostatics_biotSavart:700}
\BB(\Bx)
= \frac{\mu}{4\pi} \int_V dV' \BJ(\Bx') \cross \frac{\Bx - \Bx'}{ \Abs{\Bx - \Bx'}^3 }.
\end{dmath}

The freedom to insert a no-op bivector grade selection may seem like a sneaky move.
To remove doubt about the validity of this move, here is a demonstration that
the scalar grade discarded by the grade selection operation on the integrand of \cref{eqn:magnetostatics_biotSavart:680} is explicitly zero,
provided the current density vanishes faster than \( r \) on the infinite sphere.
\begin{dmath}\label{eqn:magnetostatics_biotSavart:60}
 \int_V dV' \frac{\Bx - \Bx'}{ \Abs{\Bx - \Bx'}^3 } \cdot \BJ(\Bx')
= \frac{\mu}{4\pi} \int_V dV' \lr{ \spacegrad \inv{ \Abs{\Bx - \Bx'} }} \cdot \BJ(\Bx')
=  \int_V dV' \lr{ \spacegrad' \inv{ \Abs{\Bx - \Bx'} }} \cdot \BJ(\Bx')
=  \int_V dV' \lr{
\spacegrad' \cdot \frac{\BJ(\Bx')}{ \Abs{\Bx - \Bx'} }
-
\frac{\spacegrad' \cdot \BJ(\Bx')}{ \Abs{\Bx - \Bx'} }
}.
\end{dmath}

By \cref{eqn:magnetostatics:460}, the divergence of the current density is zero, which kills the second term.
The divergence theorem can be used to express the remaining integral as a surface integral, so
\begin{dmath}\label{eqn:magnetostatics_biotSavart:100}
 \int_V dV' \frac{\Bx - \Bx'}{ \Abs{\Bx - \Bx'}^3 } \cdot \BJ(\Bx')
=  \int_V dV' \spacegrad' \cdot \frac{\BJ(\Bx')}{ \Abs{\Bx - \Bx'} }
=  \int_{\partial V} dA' \ncap \cdot \frac{\BJ(\Bx')}{ \Abs{\Bx - \Bx'} }.
\end{dmath}

Provided the normal component of \( \BJ(\Bx')/\Abs{\Bx - \Bx'} \) vanishes on the boundary of an infinite sphere, we see that the
the scalar selection of the convolution integral is zero, justifying the (sneaky) bivector selection operation.

%}

\subsection{Enclosed current density.}
   %
% Copyright © 2018 Peeter Joot.  All Rights Reserved.
% Licenced as described in the file LICENSE under the root directory of this GIT repository.
%
%{
\index{enclosed current density}

A volume integral of \cref{eqn:magnetostatics:380} provides a relationship between the total enclosed current density and the magnetic field.
The fundamental theorem gives
\begin{dmath}\label{eqn:magnetostatics_enclosedCurrentDensity:580}
-\mu I
\int_V d^3 \Bx \BJ =
\int_V d^3 \Bx \spacegrad \BB =
\int_{\partial V} d^2 \Bx \BB.
\end{dmath}

With a normal parameterization of the oriented surface area element \( d^2 \Bx = I \ncap dA \), and \( d^3 \Bx = I dV \),
\cref{eqn:magnetostatics_enclosedCurrentDensity:580} is reduced to
%\begin{dmath}\label{eqn:magnetostatics_enclosedCurrentDensity:600}
\boxedEquation{eqn:magnetostatics_enclosedCurrentDensity:600}{
\int_{\partial V} dA I \ncap \BB = \mu  \int_V dV \BJ.
}
%\end{dmath}

This can be split into two grades

\begin{subequations}
\label{eqn:magnetostatics_enclosedCurrentDensity:620}
\begin{dmath}\label{eqn:magnetostatics_enclosedCurrentDensity:640}
I \int_{\partial V} dA \ncap \cdot \BB = 0
\end{dmath}
\begin{dmath}\label{eqn:magnetostatics_enclosedCurrentDensity:660}
\int_{\partial V} dA \ncap \cross \BB = -\mu  \int_V dV \BJ.
\end{dmath}
\end{subequations}

\Cref{eqn:magnetostatics_enclosedCurrentDensity:640} states that the magnetic flux through a closed surface is zero, which is not be a surprise since it is a direct consequence of Gauss's law \( \spacegrad \cdot \BB = 0 \).
\Cref{eqn:magnetostatics_enclosedCurrentDensity:660} provides a relationship between the tangential components of the magnetic field and the total enclosed current density.

%}

\subsection{Enclosed current.}
   %
% Copyright © 2018 Peeter Joot.  All Rights Reserved.
% Licenced as described in the file LICENSE under the root directory of this GIT repository.
%
%{

\index{enclosed current}

Starting with
\cref{eqn:magnetostatics:380}, the
integral of the
charge density through an open surface can be calculated
\begin{dmath}\label{eqn:magnetostatics_enclosedCurrent:720}
\mu \int_S d^2 \Bx \cdot (I \BJ)
=
\int_S d^2 \Bx \cdot (\spacegrad \BB)
=
\int_S d^2 \Bx \cdot (\spacegrad \wedge \BB)
=
\int_{\partial S} d^1 \Bx \cdot \BB.
\end{dmath}

The specific orientation of the path around the surface for a two parameter expansion of Stokes' theorem 
(\cref{thm:surfaceintegral:500}),
%(\cref{eqn:twoparameter:280}),
was a clockwise loop.
Let
\( d^2 \Bx = I \ncap dA \), \( d^1 \Bx = d\Bl \), giving
\begin{dmath}\label{eqn:magnetostatics_enclosedCurrent:740}
\ointclockwise_{\partial S} d\Bl \cdot \BB
= \mu \int_S dA (\ncap I) \cdot (I \BJ)
= -\mu \int_S dA \ncap \cdot \BJ.
\end{dmath}

The circulation of the magnetic field around the boundary of a surface that the current density is flowing through is the enclosed surface.
This can be written as the dot product of an oriented area element with the dual of the current density (a bivector), or in the traditional form as a dot product of the current density with the outwards normal to the surface
%\begin{equation}\label{eqn:magnetostatics_enclosedCurrent:780}
\boxedEquation{eqn:magnetostatics:800}{
\begin{aligned}
\ointclockwise_{\partial S} d\Bl \cdot \BB &= \mu \int_S d^2 \Bx \cdot (I \BJ) \\
\ointctrclockwise_{\partial S} d\Bl \cdot \BB &= \mu \int_S dA \ncap \cdot \BJ.
\end{aligned}
}
%\end{equation}

%The bivector form of the dot product makes some geometrical sense, as we take a per unit area quantity (\(-I \BJ\)) with bivector grade and multiply it with a bivector area element, producing exactly the enclosed current flowing through the surface.

%}

\subsection{Poynting theorem.}
   %
% Copyright © 2017 Peeter Joot.  All Rights Reserved.
% Licenced as described in the file LICENSE under the root directory of this GIT repository.
%

Poynting's theorem, the conservation relationship between energy and momentum density (or more generally, the energy-momentum tensor) and the sources, can be stated in terms of the multivector field \( F \) and the multivector current \( J \).
To derive this relationship we can act on (all terms of) \( F a F^\dagger \) with the space+time derivative operator \( \spacegrad + (1/c) \partial_t \).
We do so within a scalar selection operation, which simplifies things, and allows for cyclic permutation of the multivector factors (i.e. \(\gpgradezero{ABC} = \gpgradezero{CAB}\)).
\begin{dmath}\label{eqn:poyntingTheorem:100}
\frac{\epsilon}{2} \gpgradezero{ \lr{ c \spacegrad + \PD{t}{} } F a F^\dagger }
=
\frac{\epsilon}{2} \gpgradezero{ \lr{ c \spacegrad + \PD{t}{} } \dot{F} a F^\dagger }
+
\frac{\epsilon}{2} \gpgradezero{ \lr{ c \spacegrad + \PD{t}{} } F a \dot{F}^\dagger }
=
\frac{\epsilon}{2} \gpgradezero{ c J a F^\dagger }
+
\frac{\epsilon}{2} \gpgradezero{ \dot{F}^\dagger \lr{ c \spacegrad + \PD{t}{} } F a },
\end{dmath}
where
the over-dot notation of
\citep{hestenes1999nfc} was used to indicate the desired action of the derivative operators in the
chain rule expansion of
\cref{eqn:poyntingTheorem:100}
, as the gradient may not commute with \( F \).  Another application of Maxwell's equation reduces this further
\begin{dmath}\label{eqn:poyntingTheorem:960}
\frac{\epsilon}{2} \gpgradezero{ \lr{ c \spacegrad + \PD{t}{} } F a F^\dagger }
=
\frac{\epsilon}{2} \gpgradezero{ c J F^\dagger a }
+
\frac{\epsilon}{2} \gpgradezero{ \lr{ \lr{ c \spacegrad + \PD{t}{} } F }^\dagger F a }
=
c \frac{\epsilon}{2} \gpgradezero{ F^\dagger J a + J^\dagger F a },
\end{dmath}
or
\boxedEquation{eqn:poyntingF:980}{
c \spacegrad \cdot \gpgradeone{ \frac{\epsilon}{2} F a F^\dagger }
+ \PD{t}{} \gpgradezero{ \frac{\epsilon}{2} F a F^\dagger }
=
\frac{1}{2 \eta} \gpgradezero{ a \lr{ F^\dagger J + J^\dagger F} }.
}

For \( a = 1 \), since scalars are reversion invariant (\(\alpha^\dagger = \alpha\) for any scalars \( \alpha \))
\begin{equation}\label{eqn:poyntingTheorem:1000}
\gpgradezero{ F^\dagger J }
=
\gpgradezero{ F^\dagger J }^\dagger
=
\gpgradezero{ J^\dagger F },
\end{equation}
so the
multivector form of Poynting's theorem with respect to the time variation of the energy of the field is
\index{Poynting theorem}
\boxedEquation{eqn:poyntingF:220}{
c \spacegrad \cdot \gpgradeone{ \frac{\epsilon}{2} F F^\dagger }
+ \PD{t}{} \gpgradezero{ \frac{\epsilon}{2} F F^\dagger }
=
\inv{\eta} \gpgradezero{ J^\dagger F }.
}

The conventional statement of this theorem in terms of \( \BD, \BE, \BB, \BH, \BJ, \BM \) follows by direct substitution.
The multivector current \( J \) and its reverse are
\begin{dmath}\label{eqn:poyntingTheorem:160}
\begin{aligned}
J &= \eta \lr{ c \rho - \BJ } + I \lr{ c \rho_m - \BM } \\
J^\dagger &= \eta \lr{ c \rho - \BJ } - I \lr{ c \rho_m - \BM },
\end{aligned}
\end{dmath}
so
\begin{dmath}\label{eqn:poyntingTheorem:180}
0 =
\spacegrad \cdot \BS
-
\inv{\eta}
\lr{
- \eta \BJ \cdot \BE
- \eta \BM \cdot \BH
}
+ \PD{t}{\calE},
\end{dmath}
or
\boxedEquation{eqn:poyntingF:200}{
\spacegrad \cdot \BS + \BJ \cdot \BE + \BM \cdot \BH
%+ \PD{t}{\calE} = 0.
+ \PD{t}{\BB} \cdot \BH
+ \PD{t}{\BD} \cdot \BE = 0.
}

The sum of the last two terms is the time rate of change of the energy density.
In particular,
with neither electric nor magnetic current sources in a region of space,
the change of energy density through a volume is matched by a corresponding flux through the bounding surface
\begin{dmath}\label{eqn:poyntingTheorem:740}
\PD{t}{} \int_V
\inv{2} dV \lr{
\BB \cdot \BH
+ \BD \cdot \BE
}
=
-\int_{\partial V} dA \ncap \cdot \BS.
\end{dmath}

Here \( \ncap \) is the outward normal, so if the energy contained in the volume is decreasing, then \( \BS \) must represent the energy per unit area that leaves the volume.
The direction of the Poynting vector is the direction that the energy is leaving the volume.
Only the components of the Poynting vector that are colinear with the surface normal will result in energy leaving or entering the volume.


   %
% Copyright � 2018 Peeter Joot.  All Rights Reserved.
% Licenced as described in the file LICENSE under the root directory of this GIT repository.
%
%{
In this section we show that the
conservation laws \cref{eqn:poyntingF:980} associated with the tensor components \( T(\Be_k) \) relate the time rate of change of the Poynting vector to the continuum Lorentz force equation.
The Poynting and Lorentz force relations can be shown by summing \cref{eqn:poyntingF:980} over all the unit vector directions
\begin{dmath}\label{eqn:poyntingLorentzForce:20}
\sum_{k = 1}^3
c \lr{ \spacegrad \cdot \BT(\Be_k) } \Be_k
+
\sum_{k = 1}^3
\PD{t}{} \gpgradezero{ T(\Be_k) } \Be_k
=
\frac{1}{2 \eta}
\sum_{k = 1}^3
\gpgradezero{ \Be_k \lr{ F^\dagger J + J^\dagger F} } \Be_k.
\end{dmath}

From \cref{eqn:poyntingF:800} we see that
\begin{dmath}\label{eqn:poyntingLorentzForce:40}
\gpgradezero{ T(\Be_k) }
=
-\inv{c} \BS \cdot \Be_k.
\end{dmath}

The gradient terms submit to a bit of manipulation and notational sugar
\begin{dmath}\label{eqn:poyntingLorentzForce:260}
\sum_{k = 1}^3
\lr{ \spacegrad \cdot \BT(\Be_k) } \Be_k
=
\sum_{k,m = 1}^3
\Be_k \partial_m \BT(\Be_k) \cdot \Be_m
=
\sum_{k,m = 1}^3
\Be_k \partial_m T_{km}
=
\sum_{k,m = 1}^3
\partial_m T_{mk} \Be_k
=
\sum_{k,m = 1}^3
\partial_m \BT(\Be_m).
\end{dmath}

If we allow the partials to act on \( \BT \),
this shows that we can parameterize the Maxwell stress tensor by the gradient itself
\begin{dmath}\label{eqn:poyntingLorentzForce:280}
\sum_{k = 1}^3
\lr{ \spacegrad \cdot \BT(\Be_k) } \Be_k
=
\BT(\spacegrad),
\end{dmath}
which really means
\begin{dmath}\label{eqn:poyntingLorentzForce:300}
\BT(\spacegrad)
=
\frac{\epsilon}{2} \gpgradeone{ F \spacegrad F^\dagger },
\end{dmath}
where \( \spacegrad \) acts bidirectionally using the chain rule on both \( F \) and \( F^\dagger \).

To reduce the current terms, we will consider electric and magnetic sources in sequence.
Starting with an electric current density
\( \BJ = \rho \Bv \), for which the multivector current is
\begin{equation}\label{eqn:poyntingLorentzForce:60}
J = \frac{\rho}{\epsilon}\lr{ 1 - \Bv_\txte/c } = J^\dagger,
\end{equation}
the grade selection of the current term is
\begin{dmath}\label{eqn:poyntingLorentzForce:80}
\inv{2 \eta} \gpgradezero{ \Be_k \lr{ F^\dagger J + J^\dagger F} }
=
\frac{\rho}{2 \epsilon \eta} \gpgradezero{ \Be_k \lr{ (\BE - I \eta \BH)(1 - \Bv_\txte/c)  + (1 - \Bv_\txte/c)( \BE + I \eta \BH )} }
=
\frac{\rho}{2 \epsilon \eta} \Be_k \cdot \gpgradeone{ (\BE - I \eta \BH)(1 - \Bv_\txte/c)  + (1 - \Bv_\txte/c)( \BE + I \eta \BH )}
=
\frac{\rho}{\epsilon \eta} \Be_k \cdot \lr{ \BE + I \eta \BH \wedge \Bv_\txte/c }
=
\frac{\rho}{\epsilon \eta} \Be_k \cdot \lr{ \BE + \Bv_\txte \cross \BB }
=
c\rho \Be_k \cdot \lr{ \BE + \Bv_\txte \cross \BB }.
\end{dmath}

Given a magnetic current density \( \BM = \Bv_\txtm \rho_\txtm \) the multivector current is
\begin{equation}\label{eqn:poyntingLorentzForce:120}
J = I \rho_\txtm \lr{ 1 - \Bv_\txtm/c } = -J^\dagger.
\end{equation}

This time we have
\begin{dmath}\label{eqn:poyntingLorentzForce:180}
\inv{2 \eta} \gpgradezero{ \Be_k \lr{ F^\dagger J + J^\dagger F} }
=
\frac{\rho_\txtm}{2  \eta} \gpgradezero{ \Be_k \lr{ (\BE - I c \BB) I (1 - \Bv_\txtm/c)  - I (1 - \Bv_\txtm/c)( \BE + I c \BB )} }
=
\frac{\rho_\txtm}{2  \eta} \Be_k \cdot \gpgradeone{ (I \BE + c \BB)(1 - \Bv_\txtm/c)  + (1 - \Bv_\txtm/c)( -I \BE + c \BB )}
=
\frac{\rho_\txtm}{ \eta} \Be_k \cdot \lr{ c \BB - I \BE \wedge \Bv_\txtm/c }
=
c \epsilon \rho_\txtm \Be_k \cdot \lr{ c \BB - \frac{\Bv_\txtm}{c} \cross \BE }.
\end{dmath}

Putting the pieces together we have
\begin{dmath}\label{eqn:poyntingLorentzForce:100}
\rho \lr{ \BE + \Bv_\txte \cross \BB }
+ \epsilon \rho_\txtm \lr{ c \BB - \frac{\Bv_\txtm}{c} \cross \BE }
=
\BT(\spacegrad)
-
\inv{c^2}
\PD{t}{ \BS }.
\end{dmath}

When there are only electric sources, we have the continuum equivalent of the Lorentz force on the RHS, and can make the identification of \( \rho \lr{ \BE + \Bv_\txte \cross \BB } \) with the force density acting on the charge and current distribution.
Incidentally, as a side effect, this shows the desired form of the dual Lorentz force equation for magnetic sources must be
\begin{dmath}\label{eqn:poyntingLorentzForce:140}
\frac{d\Bp}{dt} = \epsilon q_\txtm \lr{ c \BB - \frac{\Bv_\txtm}{c} \cross \BE },
\end{dmath}
so for (fictious) magnetic sources the electric field does no work, whereas for electric sources the magnetic field does no work.
If we identify
\begin{dmath}\label{eqn:poyntingLorentzForce:200}
\bcP_{\textrm{em}} = \inv{c^2} \BS/c^2,
\end{dmath}
as the momentum density of the field, and identify
\begin{dmath}\label{eqn:poyntingLorentzForce:220}
\bcP_{\textrm{mech}}
=
\rho \lr{ \BE + \Bv_\txte \cross \BB }
+ \rho_\txtm \lr{ \BB - \Bv_\txtm \cross \BE },
\end{dmath}
as the mechanical momentum density associated with the particles in the volume, then the total momentum density is
%\begin{dmath}\label{eqn:poyntingLorentzForce:240}
\boxedEquation{eqn:poyntingLorentzForce:240}{
\PD{t}{} \lr{
\bcP_{\textrm{mech}}
+
\bcP_{\textrm{em}}
}
=
\BT(\spacegrad).
}
%\end{dmath}

We'd like the integrate \cref{eqn:poyntingLorentzForce:240} over a volume to determine the forces on the charge and current distributions in that volume.
Before doing so, we have to first step back and consider what the volume integral of \( \BT(\spacegrad) \) means.
Using coordinates as an intermediate, we have
\begin{dmath}\label{eqn:poyntingLorentzForce:320}
\int_V dV \BT(\spacegrad)
=
\sum_k \int_V dV \Be_k \spacegrad \cdot \BT(\Be_k)
=
\sum_k \int_{\partial V} dA \Be_k \ncap \cdot \BT(\Be_k)
=
\sum_{k,m} \int_{\partial V} dA \Be_k n_m {\BT(\Be_k) \cdot \Be_m}
=
\sum_{k,m} \int_{\partial V} dA \Be_k n_m T_{km}
=
\int_{\partial V} dA n_m T_{mk} \Be_k
=
\int_{\partial V} dA n_m \BT(\Be_m)
=
\int_{\partial V} dA \BT(\ncap).
\end{dmath}

As \( \BT(\spacegrad) \) had the structure of a set of divergence operators, it is perhaps not surprising that this should be the result.

Now we can integrate over the volume to find the force on the charged particles in that region
%\begin{dmath}\label{eqn:poyntingLorentzForce:160}
\boxedEquation{eqn:poyntingLorentzForce:160}{
\BF =
\int_V dV \rho \lr{ \BE + \Bv_\txte \cross \BB }
+ \int_V dV \rho_\txtm \lr{ \BB - \Bv_\txtm \cross \BE }
=
\int_{\partial V} dA \BT(\ncap)
-
\inv{c^2}
\int_V dV
\PD{t}{ \BS }.
}
%\end{dmath}

This way of expressing the force equation is quite nice as the Maxwell stress tensor need only be evaluated in the normal direction.
%}

\section{Electrostatics.}
   %
% Copyright © 2017 Peeter Joot.  All Rights Reserved.
% Licenced as described in the file LICENSE under the root directory of this GIT repository.
%

If a gradient representation of the electric field is assumed
\begin{dmath}\label{eqn:electrostatics_electricPotential:561}
\BE(\Bx) = -\spacegrad \phi(\Bx),
\end{dmath}
then
inserting
this assumed representation into \cref{eqn:electrostatics:380} provides the
Poisson equation directly
%\begin{dmath}\label{eqn:electrostatics_electricPotential:581}
\boxedEquation{eqn:electrostatics_electricPotential:581}{
\spacegrad^2 \phi = -\frac{\rho}{\epsilon}.
}
%\end{dmath}
%This has solution \cref{eqn:electrostatics_poissonEquationSolution:601} but can also be solved using

   %
% Copyright © 2017 Peeter Joot.  All Rights Reserved.
% Licenced as described in the file LICENSE under the root directory of this GIT repository.
%
\index{Poisson equation}
Provided \( \Bx \ne \Bx' \), it is simple to show that (\cref{problem:electrostatics:gradrelation})
\begin{dmath}\label{eqn:electrostatics_poissonEquationSolution:541}
\spacegrad \Norm{\Bx - \Bx'}^k = k (\Bx - \Bx') \Norm{ \Bx - \Bx' }^{k-2}.
\end{dmath}

In particular
\begin{dmath}\label{eqn:electrostatics_poissonEquationSolution:542}
\spacegrad \inv{\Norm{\Bx - \Bx'}} = - \frac{(\Bx - \Bx')}{\Norm{ \Bx - \Bx' }^{3}}.
\end{dmath}

Inserting \cref{eqn:electrostatics_poissonEquationSolution:542} into Coulomb's law \cref{eqn:electrostatics_invertingGradient:340} gives
\begin{dmath}\label{eqn:electrostatics_poissonEquationSolution:661}
\BE(\Bx)
=
-\inv{4 \pi \epsilon} \int dV' \rho(\Bx') \spacegrad \inv{\Norm{\Bx - \Bx'}}
=
- \spacegrad \inv{4 \pi \epsilon} \int dV' \frac{\rho(\Bx')}{\Norm{\Bx - \Bx'}},
\end{dmath}
which implicitly provides the solution to the Poisson equation \cref{eqn:electrostatics_electricPotential:581}
%\begin{dmath}\label{eqn:electrostatics_poissonEquationSolution:601}
\boxedEquation{eqn:electrostatics_poissonEquationSolution:601}{
\phi(\Bx) = \inv{4 \pi \epsilon} \int dV' \frac{ \rho(\Bx') }{\Norm{\Bx - \Bx'}}.
}
%\end{dmath}

\index{Laplacian}
The solution \cref{eqn:electrostatics_poissonEquationSolution:601} could also have been obtained from the Green's function for the Laplacian
\cref{eqn:helmholtzDerviationMultivectorSolution:100}.

   %
% Copyright © 2017 Peeter Joot.  All Rights Reserved.
% Licenced as described in the file LICENSE under the root directory of this GIT repository.
%
\makeproblem{Radial gradient.}{problem:electrostatics:gradrelation}{
Prove \cref{eqn:electrostatics_poissonEquationSolution:541}.
} % problem


\section{Magnetostatics.}
   %
% Copyright © 2018 Peeter Joot.  All Rights Reserved.
% Licenced as described in the file LICENSE under the root directory of this GIT repository.
%
%{
\index{vector potential}

Similar to electrostatics where it was assumed that the electric field could be expressed as the gradient of a scalar potential,
a vector potential \( \BA \) solution for the dual of the magnetic field can be assumed
\begin{dmath}\label{eqn:magnetostatics_vectorPotential:480}
\spacegrad \BA = I \BB.
\end{dmath}

As the right hand side is a bivector, we must have \( \spacegrad \cdot \BA = 0 \) for this presumed solution to be valid.
Assuming (for now) a zero divergence constraint for the vector potential, then \cref{eqn:magnetostatics:380} is reduced to
%\begin{dmath}\label{eqn:magnetostatics_vectorPotential:540}
\boxedEquation{eqn:magnetostatics_vectorPotential:540}{
\spacegrad^2 \BA = -\mu \BJ,
}
%\end{dmath}
which can be solved immediately
%\begin{dmath}\label{eqn:magnetostatics_vectorPotential:560}
\boxedEquation{eqn:magnetostatics_vectorPotential:560}{
\BA(\Bx) = \frac{\mu}{4\pi} \int dV' \frac{ \BJ(\Bx') }{\Norm{\Bx - \Bx'}}.
}
%\end{dmath}

The zero divergence constraint for the vector potential is easily dealt with by adding a gradient to the vector potential with the
transformation \( \BA \rightarrow \overbar{\BA} + \spacegrad \chi \).
This gives
\begin{dmath}\label{eqn:magnetostatics_vectorPotential:500}
\spacegrad \BA
=
\spacegrad \overbar{\BA} + \spacegrad^2 \chi
=
\spacegrad \cdot \overbar{\BA} + \spacegrad \wedge \overbar{\BA} + \spacegrad^2 \chi,
\end{dmath}
which has the required bivector grade when \( \spacegrad \cdot \overbar{\BA} = -\spacegrad^2 \chi \), or
\begin{dmath}\label{eqn:magnetostatics_vectorPotential:520}
\chi(\Bx) = \inv{4\pi} \int dV' \frac{ \spacegrad' \cdot \overbar{\BA}(\Bx') }{\Norm{\Bx - \Bx'}}.
\end{dmath}
%}

\section{Potentials.}
      %
% Copyright � 2017 Peeter Joot.  All Rights Reserved.
% Licenced as described in the file LICENSE under the root directory of this GIT repository.
%
%{
\index{potential}
\index{multivector potential}

For both electrostatics and magnetostatics, where Maxwell's equations are both a pair of gradients, we were able to require that the respective scalar and vector potentials were both gradients.
For electrodynamics where Maxwell's equation is
\begin{dmath}\label{eqn:potentialSection:1800}
\lr{ \spacegrad + \inv{c} \PD{t}{} } F = J,
\end{dmath}
it seems more reasonable to demand a different structure of the potential, say
\begin{dmath}\label{eqn:potentialSection:1820}
F = \lr{ \spacegrad - \inv{c} \PD{t}{} } A,
\end{dmath}
where \( A \) is a multivector potential that may contain all grades, with structure to be determined.
If such a multivector potential can be found, then Maxwell's equation is reduced to a single wave equation
\begin{dmath}\label{eqn:potentialSection:1840}
\lr{ \spacegrad^2 - \inv{c^2} \PDSq{t}{} } A = J,
\end{dmath}
which can be thought of as one wave equation for each multivector grade of the multivector source \( J \).

Some thought shows that the guess \cref{eqn:potentialSection:1820} is not quite right, as it allows for the invalid possibility that \( F \) has scalar or pseudoscalar grades.
While it is possible to impose constraints (a gauge choice) on potential \( A \) that ensure
\( F \) has only the vector and bivector grades,
in general,
a grade selection filter must be imposed
\boxedEquation{eqn:potentialSection:1860}{
F
=
\gpgrade{ \lr{ \spacegrad - \inv{c} \PD{t}{} } A }{1,2}.
}

(cut)


   \subsection{Electric sources.}
      %
% Copyright © 2018 Peeter Joot.  All Rights Reserved.
% Licenced as described in the file LICENSE under the root directory of this GIT repository.
%
%{
For a multivector current with only electric sources
\begin{dmath}\label{eqn:potentialSection_electric:1880}
J = \eta \lr{ c \rho - \BJ },
\end{dmath}
we can construct a multivector potential with only scalar and vector grades
\begin{dmath}\label{eqn:potentialSection_electric:1900}
A = - \phi + c \BA.
\end{dmath}

The resulting field is
\begin{dmath}\label{eqn:potentialSection_electric:80}
F
=
\BE + I \eta \BH
=
\gpgrade{ \lr{ \spacegrad - \inv{c} \PD{t}{} }
\lr{
      - \phi
      + c \BA
}
}{1,2},
\end{dmath}
which expands to
\boxedEquation{eqn:potentialSection:2240}{
F =
-\spacegrad \phi
-\PD{t}{\BA}
+ c \spacegrad \wedge \BA.
}

The respective electric and magnetic fields can be extracted using a duality transformation for the bivector curl
\begin{dmath}\label{eqn:potentialSection_electric:1920}
F
=
-\spacegrad \phi
-\PD{t}{\BA}
+ I c \spacegrad \cross \BA,
\end{dmath}
from which we can read off the field components
\begin{dmath}\label{eqn:potentialSection_electric:100}
\begin{aligned}
\BE &= -\spacegrad \phi -\PD{t}{\BA} \\
\mu \BH &= \spacegrad \cross \BA.
\end{aligned}
\end{dmath}

Observe that the grade selection encodes the precise recipe required to produce the desired combination of gradients, curls and time partials.

The potential representation of the field \cref{eqn:potentialSection_electric:80} is only a solution if Maxwell's equation is also satisfied, or
\begin{dmath}\label{eqn:potentialSection_electric:1960}
\lr{ \spacegrad^2 - \inv{c^2} \PDSq{t}{} } \lr{ -\phi + c\BA }
= \eta \lr{ c \rho - \BJ } +
\lr{ \spacegrad + \inv{c} \PD{t}{} } \gpgrade{ \lr{ \spacegrad - \inv{c} \PD{t}{} } \lr{ -\phi + c\BA } }{0,3}
= \eta \lr{ c \rho - \BJ } +
\lr{ \spacegrad + \inv{c} \PD{t}{} } \lr{ c \spacegrad \cdot \BA + \inv{c} \PD{t}{\phi} }.
\end{dmath}

Imposing a constraint on the potential grades
\begin{dmath}\label{eqn:potentialSection_electric:2020}
\spacegrad \cdot \BA + \inv{c^2} \PD{t}{\phi} = 0,
\end{dmath}
the Lorenz gauge condition, is clearly an expedient way to simplify this relationship.
In particular,
in the frequency domain \( \PDi{t}{} \leftrightarrow j \omega = j k c \), this gauge choice allows the scalar potential to be entirely eliminated, since
\begin{dmath}\label{eqn:potentialSection_electric:2040}
\phi = \frac{j c^2}{\omega} \spacegrad \cdot \BA.
\end{dmath}

so the multivector potential is completely determined by a single vector potential
\begin{dmath}\label{eqn:potentialSection_electric:2060}
A =
-\frac{j c^2}{\omega} \spacegrad \cdot \BA + c \BA.
\end{dmath}

Maxwell's equation is reduced to a Helmholtz equation
\begin{dmath}\label{eqn:potentialSection_electric:2080}
\lr{ \spacegrad^2 + k^2} A = J,
\end{dmath}
and the field is simply
\begin{dmath}\label{eqn:potentialSection_electric:2100}
F = \lr{ \spacegrad - j k } A.
\end{dmath}

The conventional electric and magnetic field expressions can be found by substituting \cref{eqn:potentialSection_electric:2040} into
\cref{eqn:potentialSection_electric:1920} and switching to the frequency domain
\begin{dmath}\label{eqn:potentialSection_electric:2380}
F
=
-j \frac{c^2}{\omega} \spacegrad \lr{ \spacegrad \cdot \BA }
-j \omega \BA
+ I c \spacegrad \cross \BA,
\end{dmath}
%To compare this to conventional results, let's substitute \cref{eqn:potentialSection_electric:2060} into \cref{eqn:potentialSection_electric:2100},
%
%\begin{dmath}\label{eqn:potentialSection_electric:2300}
%F
%=
%\lr{ \spacegrad - j k }
%\lr{
%-\frac{j c^2}{\omega} \spacegrad \cdot \BA + c \BA
%}
%=
%-\frac{k c^2}{\omega} \spacegrad \cdot \BA -j k c \BA
%-\frac{j c^2}{\omega} \spacegrad \lr{ \spacegrad \cdot \BA } + c \spacegrad \BA
%=
%- \cancel{ c \spacegrad \cdot \BA }
%-j \omega \BA
%-\frac{j c}{k} \spacegrad \lr{ \spacegrad \cdot \BA }
%+ \cancel{ c \spacegrad \cdot \BA }
%+ c \spacegrad \wedge \BA
%=
%-j \omega \BA -\frac{j c}{k} \spacegrad \lr{ \spacegrad \cdot \BA }
%+ c \spacegrad \wedge \BA.
%\end{dmath}
%
%From this the electric and magnetic fields can be read off
so
\begin{dmath}\label{eqn:potentialSection_electric:2320}
\begin{aligned}
\BE &= -j \omega \BA -\frac{j c}{k} \spacegrad \lr{ \spacegrad \cdot \BA } \\
\mu \BH &= \spacegrad \cross \BA.
\end{aligned}
\end{dmath}
%}

   \subsection{Magnetic sources.}
      %
% Copyright © 2018 Peeter Joot.  All Rights Reserved.
% Licenced as described in the file LICENSE under the root directory of this GIT repository.
%
%{

For a multivector current with only magnetic sources
\begin{dmath}\label{eqn:potentialSection_magnetic:2140}
J = I \lr{ c \rho_m - \BM },
\end{dmath}
we can construct a multivector potential with only pseudoscalar and bivector grades
\begin{dmath}\label{eqn:potentialSection_magnetic:2160}
A = \eta I\lr{ - \phi_m + c \BF}.
\end{dmath}

The resulting field is
\begin{dmath}\label{eqn:potentialSection_magnetic:120}
F
=
\BE + I \eta \BH
=
\gpgrade{ \lr{ \spacegrad - \inv{c}\PD{t}{} }
\lr{
      - I \eta \phi_m
      + I \eta c \BF
}
}{1,2},
\end{dmath}
which simplifies to
\boxedEquation{eqn:potentialSection:2260}{
F
=
I \eta \lr{ c \spacegrad \wedge \BF
-\PD{t}{\BF}
- \spacegrad \phi_m
}.
}

The separate electric and magnetic field contributions can be read off from
\begin{dmath}\label{eqn:potentialSection_magnetic:2280}
F
=
- \eta c \spacegrad \cross \BF
+ \eta I \lr
{
-\spacegrad \phi_m
- \PD{t}{\BF}
},
\end{dmath}
yielding
\begin{dmath}\label{eqn:potentialSection_magnetic:140}
\begin{aligned}
\BE &= -\inv{\epsilon} \spacegrad \cross \BF \\
\BH &= -\spacegrad \phi_m - \PD{t}{\BF}.
\end{aligned}
\end{dmath}

The potential representation of the field \cref{eqn:potentialSection_magnetic:140} is only a solution if Maxwell's equation is also satisfied, or
\begin{dmath}\label{eqn:potentialSection_magnetic:2120}
\lr{ \spacegrad^2 - \inv{c^2} \PDSq{t}{} }
\eta I \lr{ - \phi_m + c \BF}
=
I \lr{ c \rho_m - \BM }
+
\lr{ \spacegrad + \inv{c} \PD{t}{} } \gpgrade{ \lr{ \spacegrad - \inv{c} \PD{t}{} } \eta I \lr{ -\phi_m + c\BF } }{0,3}
=
I \lr{ c \rho_m - \BM }
+
\lr{ \spacegrad + \inv{c} \PD{t}{} }
\lr{
\frac{\eta I}{c} \PD{t}{\phi_m} + \eta c I \spacegrad \cdot \BF
}.
\end{dmath}

Again, imposing a constraint on the potential grades
\begin{dmath}\label{eqn:potentialSection_magnetic:2180}
\spacegrad \cdot \BF
+ \inv{c^2}
\PD{t}{\phi_m}
= 0,
\end{dmath}
the Lorenz gauge condition for the magnetic potentials, is clearly an expedient way to simplify this relationship.
As before, in the frequency domain the scalar potential can be entirely eliminated
\begin{dmath}\label{eqn:potentialSection_magnetic:2200}
\phi_m = \frac{j c^2}{\omega} \spacegrad \cdot \BF.
\end{dmath}

In this case the
multivector potential is
\begin{dmath}\label{eqn:potentialSection_magnetic:2220}
A =
\eta I \lr{
-\frac{j c^2}{\omega} \spacegrad \cdot \BF + c \BF
},
\end{dmath}
and Maxwell's equation and the field are given by
\cref{eqn:potentialSection_electric:2080} and
\cref{eqn:potentialSection_electric:2100} respectively.

In the frequency domain, the electric and magnetic fields can be found from
\cref{eqn:potentialSection_magnetic:2200} and \cref{eqn:potentialSection_magnetic:140}
\begin{dmath}\label{eqn:potentialSection_magnetic:2540}
\begin{aligned}
\BE &= -\inv{\epsilon} \spacegrad \cross \BF \\
\BH &=
- j \omega \BF
-\frac{j c}{k} \spacegrad \lr{ \spacegrad \cdot \BF }
.
\end{aligned}
\end{dmath}

%\begin{dmath}\label{eqn:potentialSection_magnetic:2340}
%F
%=
%\lr{ \spacegrad - j k }
%\eta I \lr{
%-\frac{j c^2}{\omega} \spacegrad \cdot \BF + c \BF
%}
%=
%\eta I
%\lr{
%- c \spacegrad \cdot \BF -j \omega \BF
%-\frac{j c}{k} \spacegrad \lr{ \spacegrad \cdot \BF } + c \spacegrad \BF
%}
%=
%\eta I
%\lr{
%-j \omega \BF
%-\frac{j c}{k} \spacegrad \lr{ \spacegrad \cdot \BF } + c \spacegrad \wedge \BF
%}
%=
%-\inv{\epsilon} \spacegrad \cross \BF
%- j \eta I \lr{
%\omega \BF + \frac{c}{k} \spacegrad \lr{ \spacegrad \cdot \BF }
%},
%\end{dmath}
%
%so
%
%\begin{dmath}\label{eqn:potentialSection_magnetic:2360}
%\begin{aligned}
%\BE &= -\inv{\epsilon} \spacegrad \cross \BF  \\
%\BH &= -j \omega \BF -j \frac{c}{k} \spacegrad \lr{ \spacegrad \cdot \BF }.
%\end{aligned}
%\end{dmath}
%}

         \subsection{Lorenz gauge.}
            %
% Copyright © 2018 Peeter Joot.  All Rights Reserved.
% Licenced as described in the file LICENSE under the root directory of this GIT repository.
%
%{
With the flexibility to alter make a gauge transformation of the potential, it is useful to examine the conditions for which it is possible to express the electromagnetic field without any grade selection operation.
That is
\begin{dmath}\label{eqn:potentialSection_LorenzGauge:1720}
F
=
\conjstgrad
\lr{
      - \phi
      + c \BA
      + \eta I \lr{ -\phi_m + c \BF }
}.
\end{dmath}

There should be no a priori assumption that such a field representation has no scalar, nor no pseudoscalar components, which can be seen by the explicit expansion in grades
\begin{dmath}\label{eqn:potentialSection_LorenzGauge:1640}
\begin{aligned}
F
&=
\conjstgrad A \\
&=
\conjstgrad \lr{ -\phi + c \BA + \eta I \lr{ -\phi_m + c \BF } } \\
&=
\inv{c} \partial_t \phi
+ c \spacegrad \cdot \BA  \\
&-\spacegrad \phi
+ I \eta c \spacegrad \wedge \BF
- \partial_t \BA  \\
&+ c \spacegrad \wedge \BA
- \eta I \spacegrad \phi_m
- I \eta \partial_t \BF \\
&+ \eta I \inv{c} \partial_t \phi_m
+ I \eta c \spacegrad \cdot \BF,
\end{aligned}
\end{dmath}
so if this potential representation has only vector and bivector grades, it must be true that
\begin{dmath}\label{eqn:potentialSection_LorenzGauge:1660}
\begin{aligned}
\inv{c} \partial_t \phi + c \spacegrad \cdot \BA &= 0 \\
\inv{c} \partial_t \phi_m + c \spacegrad \cdot \BF &= 0.
\end{aligned}
\end{dmath}

The first is the well known Lorenz gauge condition, whereas the second is the dual of that condition for magnetic sources.

Should one of these conditions, say the Lorenz condition for the electric source potentials, be non-zero, then it is possible to make a potential transformation for which this condition is zero
\begin{dmath}\label{eqn:potentialSection_LorenzGauge:1680}
0 \ne
\inv{c} \partial_t \phi + c \spacegrad \cdot \BA
=
\inv{c} \partial_t (\phi' - \partial_t \psi) + c \spacegrad \cdot (\BA' + \spacegrad \psi)
=
\inv{c} \partial_t \phi' + c \spacegrad \BA'
+ c \lr{ \spacegrad^2 - \inv{c^2} \partial_{tt} } \psi,
\end{dmath}
so if \( \inv{c} \partial_t \phi' + c \spacegrad \BA' \) is zero, \( \psi \) must be found such that
\begin{dmath}\label{eqn:potentialSection_LorenzGauge:1700}
\inv{c} \partial_t \phi + c \spacegrad \cdot \BA
= c \lr{ \spacegrad^2 - \inv{c^2} \partial_{tt} } \psi.
\end{dmath}

Such a gauge transformation requires a non-homogeneous wave equation solution, or equivalently in the frequency domain requires the solution of a Helmholtz equation
\begin{dmath}\label{eqn:potentialSection_LorenzGauge:1740}
\inv{c} j \omega \phi + c \spacegrad \cdot \BA
= c \lr{ \spacegrad^2 + k^2 } \psi.
\end{dmath}

A similar transformation is also clearly possible to eliminate any pseudoscalar grades in \cref{eqn:potentialSection_LorenzGauge:1720}.
Such a potential representation is desirable since
Maxwell's equations for such a potential are completely decoupled
\begin{dmath}\label{eqn:potentialSection_LorenzGauge:1760}
%\lr{ \spacegrad^2 - \inv{c^2} \PDSq{t}{} } 
\dLambertian
A = J,
\end{dmath}
which is equivalent to precisely one non-homogeneous wave equation for each grade source and potential
\begin{dmath}\label{eqn:potentialSection_LorenzGauge:1600}
\begin{aligned}
\dLambertian
%\lr{ \spacegrad^2 - \inv{c^2} \PDSq{t}{} } 
\phi &= - \inv{\epsilon} \rho \\
\dLambertian
%\lr{ \spacegrad^2 - \inv{c^2} \PDSq{t}{} } 
\BA &= - \mu \BJ \\
\dLambertian
%\lr{ \spacegrad^2 - \inv{c^2} \PDSq{t}{} } 
\phi_m &= - \frac{I}{\mu} \rho_m \\
\dLambertian
%\lr{ \spacegrad^2 - \inv{c^2} \PDSq{t}{} } 
\BF &= - I \epsilon \BM,
\end{aligned}
\end{dmath}
or equivalently, in the frequency domain, a forced Helmholtz equation for each grade
\begin{dmath}\label{eqn:potentialSection_LorenzGauge:1780}
\begin{aligned}
\lr{ \spacegrad^2 + k^2 } \phi &= - \inv{\epsilon} \rho \\
\lr{ \spacegrad^2 + k^2 } \BA &= - \mu \BJ \\
\lr{ \spacegrad^2 + k^2 } \phi_m &= - \frac{1}{\mu} \rho_m \\
\lr{ \spacegrad^2 + k^2 } \BF &= - \epsilon \BM.
\end{aligned}
\end{dmath}
%}

\section{Potentials. rewrite I.}
   \subsection{Definition (statics.)}
      %
% Copyright � 2018 Peeter Joot.  All Rights Reserved.
% Licenced as described in the file LICENSE under the root directory of this GIT repository.
%
%{
%\input{../latex/blogpost.tex}
%\renewcommand{\basename}{unpackStaticPotential}
%%\renewcommand{\dirname}{notes/phy1520/}
%\renewcommand{\dirname}{notes/ece1228-electromagnetic-theory/}
%%\newcommand{\dateintitle}{}
%%\newcommand{\keywords}{}
%
%\input{../latex/peeter_prologue_print2.tex}
%
%\usepackage{peeters_layout_exercise}
%\usepackage{peeters_braket}
%\usepackage{peeters_figures}
%\usepackage{siunitx}
%%\usepackage{mhchem} % \ce{}
%%\usepackage{macros_bm} % \bcM
%%\usepackage{macros_qed} % \qedmarker
%%\usepackage{txfonts} % \ointclockwise
%
%\beginArtNoToc
%
%\generatetitle{Unpacking the potential equation.}
%%\chapter{Unpacking the potential equation.}
\label{chap:unpackStaticPotential}

In this section we will explore potential (second order) solutions of
\cref{eqn:statics:20}, the Maxwell statics equation.
While we can solve this for the field directly using
the Green's function for the gradient \cref{eqn:greensFunctionFirstOrderHelmholtz:900}, a second order solution can use the scalar valued Green's function \cref{eqn:greensFunctionHelmholtz:80} which may be simpler to evaluate in some circumstances.

\makedefinition{Multivector potential (statics.)}{thm:staticPotentials:380}{
We call \( A \) the \textit{multivector potential} for the (static) field if
\begin{equation*}
F = \gpgrade{\spacegrad A}{1,2},
\end{equation*}
and
\begin{equation*}
\spacegrad \gpgrade{\spacegrad A}{1,2} = J.
\end{equation*}
} % definition

Without any a-priori knowledge, we should assume that the multivector potential contains all possible grades.  If those grades are indicated with subscripts, as in
\begin{dmath}\label{eqn:unpackStaticPotential:20}
A = A_0 + A_1 + A_2 + A_3,
\end{dmath}
then
\begin{dmath}\label{eqn:unpackStaticPotential:40}
F =F =  \gpgrade{\spacegrad A}{1,2}
=
\mathLabelBox[ labelstyle={below of=m\themathLableNode, below of=m\themathLableNode} ]
{
\spacegrad A_0 + \spacegrad \cdot A_2
}
{
\(\BE\)
}
+
%\mathLabelBox[ labelstyle={yshift=0.2em}, linestyle={} ]
\mathLabelBox[ labelstyle={below of=m\themathLableNode, below of=m\themathLableNode} ]
{
\spacegrad \wedge A_1 + \spacegrad A_3
}
{
\(I \eta \BH\)
}
.
\end{dmath}
Application of the gradient to \cref{eqn:unpackStaticPotential:40} provides one equation for each grade
\begin{dmath}\label{eqn:unpackStaticPotential:60}
\begin{aligned}
\spacegrad \cdot \lr{ \spacegrad A_0 + \cancel{\spacegrad \cdot A_2} } &= \frac{\rho}{\epsilon} \\
\spacegrad \cdot \lr{ \spacegrad \wedge A_1 + \cancel{\spacegrad A_3} } &= -\eta \BJ \\
\spacegrad \wedge \lr{ \cancel{\spacegrad A_0} + \spacegrad \cdot A_2 } &= -I \BM \\
\spacegrad \wedge \lr{ \cancel{\spacegrad \wedge A_1} + \spacegrad A_3 } &= I c \rho_\txtm,
\end{aligned}
\end{dmath}
where the cancelled terms follow from \( \spacegrad \wedge \spacegrad \wedge f = 0 \), after noting that \( \spacegrad \cdot (\spacegrad \cdot A_2) = I \spacegrad \wedge (\spacegrad \wedge (-I A_2)) \).
The Maxwell statics equation uncouples completely, providing one equation for each grade, where each grade of the multivector potential is related to precisely one source.

The conventional representation of these equations may be obtained with the following representation of the multivector potential.
\makedefinition{Multivector potential representation.}{dfn:unpackStaticPotential:80}{
Let
%\label{eqn:potentialSection:40}
\begin{equation*}
A =
      - \phi
      + c \BA
      + \eta I \lr{ -\phi_m + c \BF },
\end{equation*}
where
\begin{enumerate}
\item \( \phi \) is the scalar potential \si{V} (Volts).
\item \( \BA \) is the vector potential \si{W/m} (Webers/meter).
\item \( \phi_m \) is the scalar potential for (fictitious) magnetic sources \si{A} (Amperes).
\item \( \BF \) is the vector potential for (fictitious) magnetic sources \si{C} (Coulombs).
\end{enumerate}
} % definition

This specific breakdown of \( A \) into scalar and vector potentials, and dual (pseudoscalar and bivector) potentials has been chosen to match SI conventions, specifically those of \citep{balanis2005antenna}.

With substitution of \cref{dfn:unpackStaticPotential:80} into \cref{eqn:unpackStaticPotential:60}, and a bit of manipulation (\cref{problem:unpackStaticPotential:120}), the multivector potential equation \( \spacegrad \gpgrade{\spacegrad A}{1,2} = J \) is seen to expand to
\begin{dmath}\label{eqn:unpackStaticPotential:100}
\begin{aligned}
\spacegrad^2 \phi &= -\frac{\rho}{\epsilon} \\
\spacegrad \cdot \lr{ \spacegrad \wedge \BA } = \spacegrad^2 \BA - \spacegrad (\spacegrad \cdot \BA) &= - \mu \BJ \\
\spacegrad \cdot \lr{ \spacegrad \wedge \BF } = \spacegrad^2 \BF - \spacegrad (\spacegrad \cdot \BF) &= - \epsilon \BM \\
\spacegrad^2 \phi_\txtm &= -\frac{\rho_\txtm}{\mu} \\
\end{aligned}
\end{dmath}

For statics problems, it is clearly desirable if we can find \( \BA, \BF \) such that \( \spacegrad \cdot \BA = 0, \spacegrad \cdot \BF = 0 \), as the potentials all follow by solution of a set of independent Laplacian equations.
This can arranged by a suitable choice of gauge.  We will look next at gauge transformations from a multivector point of view.

\makeproblem{Grade split of the potential equations.}{problem:unpackStaticPotential:120}{
Verify \cref{eqn:unpackStaticPotential:100}.
} % problem

%}
%\EndArticle

   \subsection{Gauge transformations (statics.)}
      %
% Copyright � 2018 Peeter Joot.  All Rights Reserved.
% Licenced as described in the file LICENSE under the root directory of this GIT repository.
%
%{
%\input{../latex/blogpost.tex}
%\renewcommand{\basename}{staticPotentials}
%%\renewcommand{\dirname}{notes/phy1520/}
%\renewcommand{\dirname}{notes/ece1228-electromagnetic-theory/}
%%\newcommand{\dateintitle}{}
%%\newcommand{\keywords}{}
%
%\input{../latex/peeter_prologue_print2.tex}
%
%\usepackage{peeters_layout_exercise}
%\usepackage{peeters_braket}
%\usepackage{peeters_figures}
%\usepackage{siunitx}
%%\usepackage{mhchem} % \ce{}
%%\usepackage{macros_bm} % \bcM
%%\usepackage{macros_qed} % \qedmarker
%%\usepackage{txfonts} % \ointclockwise
%
%\beginArtNoToc
%
%\generatetitle{Potential solutions to the static Maxwell's equation using geometric algebra}
%\chapter{Potential solutions to the static Maxwell's equation using geometric algebra}
\label{chap:staticPotentials}
%\subsubsection{Gauge transformations.}

Finding the potential requires solving the equation
\begin{dmath}\label{eqn:staticPotentials:140}
\spacegrad^2 A - \spacegrad \gpgrade{\spacegrad A}{0,3} = J.
\end{dmath}
It would clearly be desirable if we could find a potential for which the field is a strict gradient
\begin{dmath}\label{eqn:staticPotentials:60}
F = \spacegrad A,
\end{dmath}
since that would reduce Maxwell's equation to simply
\begin{dmath}\label{eqn:staticPotentials:80}
\spacegrad^2 A = J,
\end{dmath}
which we know how to solve.
Given the potential specified by \cref{eqn:unpackStaticPotential:20}, the
split of the field
given by
\cref{eqn:staticPotentials:60}
into
vector+bivector grades and scalar+pseudoscalar grades is:
\begin{dmath}\label{eqn:staticPotentials:420}
\begin{aligned}
F
&= \spacegrad A \\
&=
 \spacegrad A_0 +
 \spacegrad \cdot A_2 +
 \spacegrad \wedge A_1 +
 \spacegrad \cdot A_3 \\
&\quad + \spacegrad \cdot A_1 + \spacegrad \wedge A_2.
\end{aligned}
\end{dmath}
Only the vector and bivector grades of the potential can contribute scalar and pseudoscalar grades, so if
gradient representation of the field as in \cref{eqn:staticPotentials:60} can be found,
we require the imposition of a constraint on the vector and bivector grades of a multivector potential
\begin{dmath}\label{eqn:staticPotentials:440}
0 = \spacegrad \cdot A_1 + \spacegrad \wedge A_2.
\end{dmath}
Such constraints are conspiring to make life difficult, but
luckily, a transformation of potentials (a gauge transformation) can be used to impose these constraints.
As a side effect such transformation
reduced the problem to that of a Laplacian.

\maketheorem{Potential solution of Maxwell static's equation.}{thm:staticPotentials:420}{
If \( A \) is a multivector solution to \( \spacegrad^2 A = J \),
for which \( \gpgrade{\spacegrad A}{0,3} \ne 0 \),
%, that is
%\begin{equation*}
%A(\Bx)
%= \int dV' G(\Bx, \Bx') J(\Bx')
%= -\int dV' \frac{J(\Bx')}{\Norm{\Bx - \Bx'} },
%\end{equation*}
then
\( F = \spacegrad A' \) is a solution to Maxwell's equation, where \( A' = A - \tilde{A} \), and \( \tilde{A} \)
%is a solution to the non-homogeneous Laplacian equation
%\begin{equation*}
%\spacegrad^2 \tilde{A} = \spacegrad \gpgrade{\spacegrad A}{0,3},
%\end{equation*}
of the non-homogeneous gradient equation
\begin{equation*}
\spacegrad \tilde{A} = \gpgrade{\spacegrad A}{0,3}.
\end{equation*}
} % theorem

We can prove \cref{thm:staticPotentials:420}
by assuming that it is possible to find a solution of the Laplacian equation that has the desired grade restrictions.
That is
\begin{dmath}\label{eqn:staticPotentials:160}
\begin{aligned}
\spacegrad^2 A' &= J \\
\gpgrade{\spacegrad A'}{0,3} &= 0,
\end{aligned}
\end{dmath}
for which \( F = \spacegrad A' \) is a grade 1,2 solution to \( \spacegrad F = J \).
Suppose that \( A \) is any formal solution, free of any grade restrictions, to \( \spacegrad^2 A = J \), and \( F = \gpgrade{\spacegrad A}{1,2} \).
Can we find a function \( \tilde{A} \) for which \( A = A' + \tilde{A} \)?

Maxwell's equation in terms of \( A \) is
\begin{dmath}\label{eqn:staticPotentials:180}
J
= \spacegrad \gpgrade{\spacegrad A}{1,2}
= \spacegrad^2 A
- \spacegrad \gpgrade{\spacegrad A}{0,3}
= \spacegrad^2 (A' + \tilde{A})
- \spacegrad \gpgrade{\spacegrad A}{0,3}
\end{dmath}
or
\begin{dmath}\label{eqn:staticPotentials:200}
\spacegrad^2 \tilde{A} = \spacegrad \gpgrade{\spacegrad A}{0,3}.
\end{dmath}
%This non-homogeneous Laplacian equation that can be solved as is for \( \tilde{A} \) using the Green's function for the Laplacian.
Using the Green's function for the gradient, we can
find a solution to the non-homogeneous gradient equation
\begin{dmath}\label{eqn:staticPotentials:220}
\spacegrad \tilde{A} = \gpgrade{\spacegrad A}{0,3},
\end{dmath}
so that \( A' = A - \tilde{A} \) has the required constraints on its vector and bivector components.
If \( \spacegrad \gpgrade{\spacegrad A}{0,3} \) is zero, then the constraints are automatically satisfied, but we are
free to add any \( \tilde{A} \) to the potential, provided that it is a solution of the homogeneous Laplacian equation \( \spacegrad^2 \tilde{A} = 0\).

\subsubsection{Integral form of the gauge transformation.}

Additional insight is possible by considering the gauge transformation in integral form.
\maketheorem{Integral form of potential gauge transformation.}{thm:staticPotentials:460}{
The solution of the Laplacian \cref{eqn:staticPotentials:80} is
\begin{equation*}
A(\Bx) = -\int_V dV' \frac{J(\Bx')}{\Norm{\Bx - \Bx'} } - \tilde{A}(\Bx),
\end{equation*}
where \( \tilde{A} \) is any function for which \( \spacegrad^2 \tilde{A} = 0 \).
For \( F = \spacegrad A \) to be a solution of
\cref{eqn:statics:20}, the Maxwell statics equation, \( \tilde{A} \) must also be a solution of the gradient equation
\begin{equation*}
\spacegrad \tilde{A}(\Bx)
= \int_{\partial V} dA' \frac{ \gpgrade{\ncap' J(\Bx')}{0,3} }{\Norm{\Bx - \Bx'} }
=
-\int_{\partial V} dA' \frac{ \eta \ncap' \cdot \BJ(\Bx') + I \ncap' \cdot \BM(\Bx')}{\Norm{\Bx - \Bx'} },
\end{equation*}
where
\( \ncap' = (\Bx' - \Bx)/\Norm{\Bx' - \Bx} \) is the unit normal pointing away from the point \( \Bx \).
} % theorem

To prove \cref{thm:staticPotentials:460} we will look
at the constraints on \( \tilde{A} \) that must be imposed for \( F = \spacegrad A \) to be a valid (i.e. grade 1,2) solution of Maxwell's equation.
\begin{dmath}\label{eqn:staticPotentials:300}
F
= \spacegrad A
=
-\int_V dV' \lr{ \spacegrad \inv{\Norm{\Bx - \Bx'} } } J(\Bx')
- \spacegrad \tilde{A}(\Bx)
=
\int_V dV' \lr{ \spacegrad' \inv{\Norm{\Bx - \Bx'} } } J(\Bx')
- \spacegrad \tilde{A}(\Bx)
=
\int_V dV' \spacegrad' \frac{J(\Bx')}{\Norm{\Bx - \Bx'} } - \int_V dV' \frac{\spacegrad' J(\Bx')}{\Norm{\Bx - \Bx'} }
- \spacegrad \tilde{A}(\Bx)
=
\int_{\partial V} dA' \ncap' \frac{J(\Bx')}{\Norm{\Bx - \Bx'} } - \int_V \frac{\spacegrad' J(\Bx')}{\Norm{\Bx - \Bx'} }
- \spacegrad \tilde{A}(\Bx),
\end{dmath}
where the fundamental theorem of geometric calculus
has been used to transform the gradient volume integral into an integral over the bounding surface.
Operating on Maxwell's equation with the gradient gives \( \spacegrad^2 F = \spacegrad J \), which has only grades 1,2 on the left hand side, meaning that \( J \) is constrained in a way that requires \( \spacegrad J \) to have only grades 1,2.
% reference to section that discussed this.
This means that \( F \) has grades 1,2 if
\begin{dmath}\label{eqn:staticPotentials:320}
\spacegrad \tilde{A}(\Bx)
= \int_{\partial V} dA' \frac{ \gpgrade{\ncap' J(\Bx')}{0,3} }{\Norm{\Bx - \Bx'} }.
\end{dmath}
The product \( \ncap' J \) expands to
\begin{dmath}\label{eqn:staticPotentials:340}
\ncap' J
=
\gpgradezero{\ncap' J_1} + \gpgradethree{\ncap' J_2}
=
\ncap' \cdot (-\eta \BJ) + \gpgradethree{\ncap' (-I \BM)}
=- \eta \ncap' \cdot \BJ -I \ncap' \cdot \BM,
\end{dmath}
so
\begin{dmath}\label{eqn:staticPotentials:360}
\spacegrad \tilde{A}(\Bx)
=
-\int_{\partial V} dA' \frac{ \eta \ncap' \cdot \BJ(\Bx') + I \ncap' \cdot \BM(\Bx')}{\Norm{\Bx - \Bx'} }.
\end{dmath}
Observe that if there is no flux of current density \( \BJ \) and (fictitious) magnetic current density \( \BM \) through the surface (i.e. the currents form a closed loop in this volume), then \( F = \spacegrad A \) is a solution to Maxwell's equation without any gauge transformation.
%Alternatively \( F = \spacegrad A \) is also a solution if \( \lim_{\Bx' \rightarrow \infty} \BJ(\Bx')/\Norm{\Bx - \Bx'} = \lim_{\Bx' \rightarrow \infty} \BM(\Bx')/\Norm{\Bx - \Bx'} = 0 \) and the bounding volume is taken to infinity.

%In the case where there are non-zero normal current components out of the volume,
%it should be noted that the question of existance has been ignored.
%In particular, while we can find a solution \( \tilde{A} \) of \cref{eqn:staticPotentials:360} using the Green's function for the gradient \cref{eqn:greensFunctionFirstOrderHelmholtz:900}, is such a solution also a solution of the homogeneous Laplacian equation?

%}
%\EndNoBibArticle

   \subsection{Definition (general.)}
      %
% Copyright � 2018 Peeter Joot.  All Rights Reserved.
% Licenced as described in the file LICENSE under the root directory of this GIT repository.
%
%{
%\input{../latex/blogpost.tex}
%\renewcommand{\basename}{generalPotential}
%%\renewcommand{\dirname}{notes/phy1520/}
%\renewcommand{\dirname}{notes/ece1228-electromagnetic-theory/}
%%\newcommand{\dateintitle}{}
%%\newcommand{\keywords}{}
%
%\input{../latex/peeter_prologue_print2.tex}
%
%\usepackage{peeters_layout_exercise}
%\usepackage{peeters_braket}
%\usepackage{peeters_figures}
%\usepackage{siunitx}
%%\usepackage{mhchem} % \ce{}
%%\usepackage{macros_bm} % \bcM
%%\usepackage{macros_qed} % \qedmarker
%%\usepackage{txfonts} % \ointclockwise
%
%\beginArtNoToc
%
%\generatetitle{Time domain multivector potentials}
%%\chapter{Time domain multivector potentials}
\label{chap:generalPotential}

For time dependent fields and sources we need a time derivative in the derivative operator that produces the field from the potential.
\makedefinition{Multivector potential.}{thm:generalPotential:80}{
We call \( A \) the \textit{multivector potential} for the field if
\begin{equation*}
F = \gpgrade{\conjstgrad A}{1,2},
\end{equation*}
and \( (\spacegrad + (1/c)\partial_t) F = J \).
%  That is
%\begin{equation*}
%\stgrad \gpgrade{\conjstgrad A}{1,2} = J.
%\end{equation*}
} % definition

We may write Maxwell's equation in terms of the multivector potential as a modified wave equation
\begin{dmath}\label{eqn:generalPotential:20}
%\lr{ \spacegrad^2 - \inv{c^2} \PDSq{t}{} }
\dLambertian
A = J +
\stgrad \gpgrade{\conjstgrad A}{0,3}.
\end{dmath}
As we saw in statics the pesky grade selection term on the right can be gauge transformed away.  Before formulating that
gauge transformation,
it's worthwhile to unpack
\cref{thm:staticPotentials:380} to show how the electric and magnetic fields are related to the multivector potential, and
get some intuition about the quantity to be transformed away.

\maketheorem{Fields and the potential wave equations.}{thm:generalPotential:40}{
In terms of the potential components, the electric field vector and the magnetic field bivector are
\begin{equation*}
\begin{aligned}
\BE &=
\gpgrade{\conjstgrad A}{1}
=
   - \spacegrad \phi
   - \PD{t}{\BA}
   - \inv{\epsilon} \spacegrad \cross \BF \\
I \eta \BH &=
\gpgrade{\conjstgrad A}{2}
=
   I \eta
   \lr{
      - \spacegrad \phi_\txtm
      - \PD{t}{\BF}
      + \inv{\mu} \spacegrad \cross \BA
   }
.
\end{aligned}
\end{equation*}
The potentials are related to the sources by
%\begin{equation*}
%\begin{aligned}
%- \lr{ \spacegrad^2 - \inv{c^2} \PDSq{t}{} } \phi &= \frac{\rho}{\epsilon} + \inv{c} \PD{t}{} \lr{ c \spacegrad \cdot \BA + \inv{c} \PD{t}{\phi} } \\
%c \lr{ \spacegrad^2 - \inv{c^2} \PDSq{t}{} } \BA &= -\eta \BJ + \spacegrad \lr{ c \spacegrad \cdot \BA + \inv{c} \PD{t}{\phi} } \\
%\eta c I \lr{ \spacegrad^2 - \inv{c^2} \PDSq{t}{} } \BF &= - I \BM + \spacegrad \cdot \lr{ I \eta\lr{ c \spacegrad \cdot \BF + \inv{c} \PD{t}{\phi_\txtm} } } \\
%-I \eta \lr{ \spacegrad^2 - \inv{c^2} \PDSq{t}{} } \phi_\txtm &= I c \rho_\txtm + \inv{c} \PD{t}{} I \eta\lr{ c \spacegrad \cdot \BF + \inv{c} \PD{t}{\phi_\txtm} }
%\end{aligned}
%\end{equation*}
\begin{equation*}
\begin{aligned}
%\lr{ \spacegrad^2 - \inv{c^2} \PDSq{t}{} }
\dLambertian
\phi &= -\frac{\rho}{\epsilon} - \PD{t}{} \lr{ \spacegrad \cdot \BA + \inv{c^2} \PD{t}{\phi} } \\
%\lr{ \spacegrad^2 - \inv{c^2} \PDSq{t}{} }
\dLambertian
\BA &= -\mu \BJ + \spacegrad \lr{ \spacegrad \cdot \BA + \inv{c^2} \PD{t}{\phi} } \\
%\lr{ \spacegrad^2 - \inv{c^2} \PDSq{t}{} }
\dLambertian
\BF &= - \epsilon \BM + \spacegrad \lr{ \spacegrad \cdot \BF + \inv{c^2} \PD{t}{\phi_\txtm} } \\
%\lr{ \spacegrad^2 - \inv{c^2} \PDSq{t}{} }
\dLambertian
\phi_\txtm &= -\frac{\rho_\txtm}{\mu} - \PD{t}{} \lr{ \spacegrad \cdot \BF + \inv{c^2} \PD{t}{\phi_\txtm} }
\end{aligned}
\end{equation*}
} % theorem

To prove \cref{thm:generalPotential:40} we start by expanding \( (\spacegrad - (1/c)\partial_t) A \) using
\cref{dfn:unpackStaticPotential:80} and then group by grade to find
\begin{dmath}\label{eqn:generalPotential:60}
\begin{aligned}
\conjstgrad A
&=
\conjstgrad \lr{  - \phi
      + c \BA
      + \eta I \lr{ -\phi_m + c \BF } } \\
&=
- \spacegrad \phi + c \spacegrad \cdot \BA + c \spacegrad \wedge \BA + \inv{c} \PD{t}{\phi} - \PD{t}{\BA} \\
&\quad + I \eta
\lr{
- \spacegrad \phi_\txtm + c \spacegrad \cdot \BF + c \spacegrad \wedge \BF + \inv{c} \PD{t}{\phi_\txtm} - \PD{t}{\BF}
} \\
&=
c \spacegrad \cdot \BA
+ \inv{c} \PD{t}{\phi}
\\
&
+
\mathLabelBox[ labelstyle={below of=m\themathLableNode, below of=m\themathLableNode} ]
{
   - \spacegrad \phi
   - \PD{t}{\BA}
   - \inv{\epsilon} \spacegrad \cross \BF
}
{
\(\BE\)
}
+
\mathLabelBox[ labelstyle={below of=m\themathLableNode, below of=m\themathLableNode} ]
{
   I \eta
   \lr{
      - \spacegrad \phi_\txtm
      - \PD{t}{\BF}
      + \inv{\mu} \spacegrad \cross \BA
   }
}
{\(I \eta \BH\)
} \\
&
+ I \eta\lr{
  c \spacegrad \cdot \BF
+ \inv{c} \PD{t}{\phi_\txtm}
},
\end{aligned}
\end{dmath}
which shows the claimed field split.
Unpacking Maxwell's equation by grade selection gives
\begin{equation*}
\begin{aligned}
-
%\lr{ \spacegrad^2 - \inv{c^2} \PDSq{t}{} }
\dLambertian
\phi &= \frac{\rho}{\epsilon} + \inv{c} \PD{t}{} \lr{ c \spacegrad \cdot \BA + \inv{c} \PD{t}{\phi} } \\
c
%\lr{ \spacegrad^2 - \inv{c^2} \PDSq{t}{} }
\dLambertian
\BA &= -\eta \BJ + \spacegrad \lr{ c \spacegrad \cdot \BA + \inv{c} \PD{t}{\phi} } \\
\eta c I
%\lr{ \spacegrad^2 - \inv{c^2} \PDSq{t}{} }
\dLambertian
\BF &= - I \BM + \spacegrad \cdot \lr{ I \eta\lr{ c \spacegrad \cdot \BF + \inv{c} \PD{t}{\phi_\txtm} } } \\
-I \eta
%\lr{ \spacegrad^2 - \inv{c^2} \PDSq{t}{} }
\dLambertian
\phi_\txtm &= I c \rho_\txtm + \inv{c} \PD{t}{} I \eta\lr{ c \spacegrad \cdot \BF + \inv{c} \PD{t}{\phi_\txtm} }
\end{aligned}
\end{equation*}
Using \( \eta = \mu c, \eta c \epsilon = 1 \), and
\( \spacegrad \cdot (I \psi) = I \spacegrad \psi \) for scalar \(\psi\), a bit
of rearrangement completes the proof.

%}
%\EndNoBibArticle

   \subsection{Gauge transformations (general.)}
      %
% Copyright � 2018 Peeter Joot.  All Rights Reserved.
% Licenced as described in the file LICENSE under the root directory of this GIT repository.
%
%{
%\input{../latex/blogpost.tex}
%\renewcommand{\basename}{gaugeGeneral}
%%\renewcommand{\dirname}{notes/phy1520/}
%\renewcommand{\dirname}{notes/ece1228-electromagnetic-theory/}
%%\newcommand{\dateintitle}{}
%%\newcommand{\keywords}{}
%
%\input{../latex/peeter_prologue_print2.tex}
%
%\usepackage{peeters_layout_exercise}
%\usepackage{peeters_braket}
%\usepackage{peeters_figures}
%\usepackage{siunitx}
%\RequirePackage{amssymb} % relwave.tex uses this?
%%\usepackage{mhchem} % \ce{}
%%\usepackage{macros_bm} % \bcM
%%\usepackage{macros_qed} % \qedmarker
%%\usepackage{txfonts} % \ointclockwise
%
%%\newcommand{\Box}[0]{\square}
%\newcommand{\dLambertian}[0]{\square}
%\newcommand{\stgrad}[0]{\lr{ \spacegrad + \inv{c} \PD{t}{}}}
%\newcommand{\conjstgrad}[0]{\lr{ \spacegrad - \inv{c} \PD{t}{}}}
%
%\beginArtNoToc
%
%\generatetitle{Gauge transformation}
%%\chapter{Gauge transformation}
\label{chap:gaugeGeneral}

\maketheorem{Potential solution of Maxwell's equation.}{thm:gaugeGeneral:20}{
If \( A \) is a multivector solution to \( \dLambertian A = J \),
for which \( \gpgrade{\conjstgrad A}{0,3} \ne 0 \),
%, that is
%\begin{equation*}
%A(\Bx)
%= \int dV' G(\Bx, \Bx') J(\Bx')
%= -\int dV' \frac{J(\Bx')}{\Norm{\Bx - \Bx'} },
%\end{equation*}
then
\( F = \conjstgrad A' \) is a grade 1,2 solution to Maxwell's equation \(\stgrad F = J\), where \( A' = A - \tilde{A} \), and \( \tilde{A} \)
is a solution of the non-homogeneous wave equation
\begin{equation*}
\dLambertian \tilde{A} = \stgrad \gpgrade{\conjstgrad A}{0,3}.
\end{equation*}
} % theorem

To prove \cref{thm:gaugeGeneral:20}, let's first assume that there exists an \( A' \) for which \( \gpgrade{\conjstgrad A}{0,3} = 0 \) that satisifes the wave equation
\begin{dmath}\label{eqn:gaugeGeneral:60}
\dLambertian A' = J.
\end{dmath}
For such a field \( F = \conjstgrad A' \) is a grade 1,2 solution of Maxwell's equation.
Given an arbitrary solution \( A \) of \( \dLambertian A = J \), then \( F = \gpgrade{\conjstgrad A}{1,2} \) is a also solution
of Maxwell's equation.  Let \( A = A' + \tilde{A} \) and insert this into Maxwell's equation
\begin{dmath}\label{eqn:gaugeGeneral:20}
J
= \stgrad F
= \stgrad \gpgrade{\conjstgrad A}{1,2}
= \dLambertian A - \stgrad \gpgrade{\conjstgrad A}{0,3}
= \dLambertian \lr{ A' + \tilde{A} } - \stgrad \gpgrade{\conjstgrad A}{0,3}.
\end{dmath}
Subtracting \cref{eqn:gaugeGeneral:60} from both sides of \cref{eqn:gaugeGeneral:20} and rearranging, we are left with
\begin{dmath}\label{eqn:gaugeGeneral:40}
\dLambertian \tilde{A} = \stgrad \gpgrade{\conjstgrad A}{0,3},
\end{dmath}
which completes the proof.

From \cref{eqn:generalPotential:60}, we see that \cref{eqn:gaugeGeneral:40} means we seek a solution to
\begin{dmath}\label{eqn:gaugeGeneral:80}
\dLambertian \tilde{A} = \stgrad
\lr{
c \spacegrad \cdot \BA + \inv{c} \PD{t}{\phi}
+ I \eta
   \lr{
     c \spacegrad \cdot \BF + \inv{c} \PD{t}{\phi_\txtm}
   }
}
.
\end{dmath}

When the potentials are constrained by
\begin{dmath}\label{eqn:gaugeGeneral:100}
\begin{aligned}
0 &= \spacegrad \cdot \BA + \inv{c^2} \PD{t}{\phi}
0 &= \spacegrad \cdot \BF + \inv{c^2} \PD{t}{\phi_\txtm},
\end{aligned}
\end{dmath}
\index{Lorenz gauge}
then
we are said to be working in the Lorenz gauge.

%}
%\EndNoBibArticle

   \subsection{Gauge transformations (OLD. PICK A COUPLE BITS FROM)}
      %
% Copyright © 2018 Peeter Joot.  All Rights Reserved.
% Licenced as described in the file LICENSE under the root directory of this GIT repository.
%
%{

\index{gauge transformation}

Because the potential representation of the field is expressed as a grade 1,2 selection, the addition of scalar or pseudoscalar components to the grade selection will not alter the field.
In particular, it is possible to alter the multivector potential
\begin{dmath}\label{eqn:potentialSection_gauge:160}
A \rightarrow A + \stgrad \psi,
\end{dmath}
where \( \psi \) is any multivector field with scalar and pseudoscalar grades, without changing the field
\begin{dmath}\label{eqn:potentialSection_gauge:180}
F
\rightarrow
\gpgrade{
   \conjstgrad
   \lr{ A + \stgrad \psi }
}{1,2}
=
F +
\gpgrade{
   \lr{ \spacegrad^2 - \inv{c^2} \PDSq{t}{}} 
%\dLambertian
\psi
}{1,2}
.
\end{dmath}

That last grade selection is zero, since \( \psi \) has no vector or bivector grades, demonstrating that the electromagnetic field is invariant with respect to this multivector potential transformation.

It is worth looking how such a transformation impacts each grade of the potential.
Let \( \psi = c \psi^\e + \eta c I \psi^\m \), where \( \psi^\e \) and \( \psi^\m \) are both scalar fields.
The gauge transformation provides the mapping

\begin{subequations}
\label{eqn:potentialSection_gauge:220}
\begin{dmath}\label{eqn:potentialSection_gauge:200}
- \phi \rightarrow - \phi + \PD{t}{} \psi^\e
\end{dmath}
\begin{dmath}\label{eqn:potentialSection_gauge:240}
c \BA \rightarrow c \BA + c \spacegrad \psi^\e
\end{dmath}
\begin{dmath}\label{eqn:potentialSection_gauge:260}
I c \BF \rightarrow I c \BF + I c \spacegrad \psi^\m
\end{dmath}
\begin{dmath}\label{eqn:potentialSection_gauge:280}
- I \eta \phi_m \rightarrow -I \eta \phi_m + I \eta \PD{t}{} \psi^\m,
\end{dmath}
\end{subequations}

or

\begin{subequations}
\label{eqn:potentialSection_gauge:400}
\begin{dmath}\label{eqn:potentialSection_gauge:420}
\phi \rightarrow \phi - \PD{t}{} \psi^\e
\end{dmath}
\begin{dmath}\label{eqn:potentialSection_gauge:440}
\BA \rightarrow \BA + \spacegrad \psi^\e
\end{dmath}
\begin{dmath}\label{eqn:potentialSection_gauge:460}
\BF \rightarrow \BF + \spacegrad \psi^\m
\end{dmath}
\begin{dmath}\label{eqn:potentialSection_gauge:480}
\phi_m \rightarrow \phi_m - \PD{t}{} \psi^\m.
\end{dmath}
\end{subequations}

These have the alternation of sign that is found in the usual recipe for gauge transformation of the scalar and vector potentials.
In conventional electromagnetism, the first two relations are usually found by observing it is possible to add any gradient to the vector potential, and then finding the transformation consequences that that choice imposes on the electric field.
With the grade selection formulation of the electromagnetic field, this special coupling of the field potentials comes for free without having to consider the curl of a specific field component.

Note that the latter two dual transformation relationships are for magnetic sources, and are usually expressed in the frequency domain, where the gauge transformations take the form

\begin{subequations}
\label{eqn:potentialSection_gauge:300}
\begin{dmath}\label{eqn:potentialSection_gauge:320}
\phi \rightarrow \phi - j \omega \psi^\e
\end{dmath}
\begin{dmath}\label{eqn:potentialSection_gauge:340}
\BA \rightarrow \BA + \spacegrad \psi^\e
\end{dmath}
\begin{dmath}\label{eqn:potentialSection_gauge:360}
\BF \rightarrow \BF + \spacegrad \psi^\m
\end{dmath}
\begin{dmath}\label{eqn:potentialSection_gauge:380}
\phi_m \rightarrow \phi_m -j \omega \psi^\m.
\end{dmath}
\end{subequations}

%}

   \subsection{inMatter.tex}
      %
% Copyright © 2017 Peeter Joot.  All Rights Reserved.
% Licenced as described in the file LICENSE under the root directory of this GIT repository.
%
So far, we've considered only media where the linear constitutive relationships \cref{eqn:freespace:301} hold.
Without such assumptions the GA formalism for Maxwell's equations cannot be written as a single equation with one multivector field, but requires two equations and two multivector fields.

The two multivector fields are
\begin{dmath}\label{eqn:inMatter:40}
\begin{aligned}
G &= \BE + I c \BB \\
F &= \BD + \frac{I}{c} \BH,
\end{aligned}
\end{dmath}
for which Maxwell's equations are
\begin{dmath}\label{eqn:inMatter:60}
\begin{aligned}
\gpgrade{ \stgrad F }{0,1} &= \rho - \frac{\BJ}{c} \\
\gpgrade{ \stgrad G }{2,3} &= I \lr{ c \rho_m - \BM }.
\end{aligned}
\end{dmath}

Here \( c \) is a non-dimensionalizing constant with dimensions [L/T], but is otherwise unspecified.
Direct expansion can be used to show that \cref{eqn:inMatter:60} is equivalent to Maxwell's equations.
Doing so for each of the grades in turn, we have

\begin{subequations}
\label{eqn:inMatter:80}
\begin{dmath}\label{eqn:inMatter:100}
\rho
=
\gpgradezero{ \stgrad F }
=
\gpgradezero{ \stgrad \lr{ \BD + \frac{I}{c} \BH } }
=
\spacegrad \cdot \BD
\end{dmath}
\begin{dmath}\label{eqn:inMatter:120}
- \frac{\BJ}{c}
=
\gpgradeone{ \stgrad F }
=
\gpgradeone{ \stgrad \lr{ \BD + \frac{I}{c} \BH } }
=
\inv{c} \PD{t}{\BD} + \frac{I}{c} \spacegrad \wedge \BH
=
\inv{c} \PD{t}{\BD} - \frac{1}{c} \spacegrad \cross \BH
\end{dmath}
\begin{dmath}\label{eqn:inMatter:140}
- I \BM
=
\gpgrade{ \stgrad G }{2}
=
\gpgrade{ \stgrad \lr{ \BE + I c \BB} }{2}
=
\spacegrad \wedge \BE + I \PD{t}{\BB}
\end{dmath}
\begin{dmath}\label{eqn:inMatter:160}
I c \rho_m
=
\gpgrade{ \stgrad G }{3}
=
\gpgrade{ \stgrad \lr{ \BE + I c \BB} }{3}
=
c I \spacegrad \cdot \BB.
\end{dmath}
\end{subequations}

After rearranging and cancelling common factors of \( c, I \) Maxwell's equations are recovered
\begin{dmath}\label{eqn:inMatter:180}
\begin{aligned}
\spacegrad \cdot \BD &= \rho \\
\spacegrad \cross \BH &= \BJ + \PD{t}{\BD}  \\
\spacegrad \cross \BE &= -\BM - \PD{t}{\BB} \\
\spacegrad \cdot \BB &= \rho_m.
\end{aligned}
\end{dmath}

One possible strategy for solving these equations is to impose an additional set of constraints on the grades in question
\begin{dmath}\label{eqn:inMatter:200}
\begin{aligned}
\gpgrade{ \stgrad F }{2,3} &= 0 \\
\gpgrade{ \stgrad G }{0,1} &= 0,
\end{aligned}
\end{dmath}
so that all the grade selection filters can be cleared
\begin{dmath}\label{eqn:inMatter:220}
\begin{aligned}
\stgrad F &= \rho - \frac{\BJ}{c} \\
\stgrad G &= I \lr{ c \rho_m - \BM }.
\end{aligned}
\end{dmath}

Each of these now separately has the form of Maxwell's equation, and could be solved separately, subject to the constraint equations.
Only if \( G, F \) can be related by a constant factor, say \( \epsilon G = F \), can these be summed directly (after non-dimensional scaling) to form Maxwell's equation.
Other non-constraint strategies for solving \cref{eqn:inMatter:60} would require additional thought and study.


%}
%\EndArticle
\EndNoBibArticle
