%
% Copyright � 2017 Peeter Joot.  All Rights Reserved.
% Licenced as described in the file LICENSE under the root directory of this GIT repository.
%
\index{curvilinear coordinates}
\index{spherical coordinates}
The spherical vector parameterization admits a compact GA representation.
From the coordinate representation, some factoring gives

\begin{dmath}\label{eqn:curvilinearspherical:20}
\Bx
= r \lr{ \Be_1 \sin\theta \cos\phi + \Be_2 \sin\theta \sin\phi + \Be_3 \cos\theta }
= r \lr{ \sin\theta \Be_1 (\cos\phi + \Be_{12} \sin\phi ) + \Be_3 \cos\theta }
= r \lr{ \sin\theta \Be_1 e^{\Be_{12} \phi } + \Be_3 \cos\theta }
= r \Be_3 \lr{ \cos\theta + \sin\theta \Be_3 \Be_1 e^{\Be_{12} \phi } }.
\end{dmath}

With
\begin{dmath}\label{eqn:curvilinearspherical:40}
\begin{aligned}
i &= \Be_{12} \\
j &= \Be_{31} e^{i \phi},
\end{aligned}
\end{dmath}

this is

\begin{dmath}\label{eqn:curvilinearspherical:60}
\Bx = r \Be_3 e^{j \theta}.
\end{dmath}

The curvilinear basis vectors can now be computed

\begin{subequations}
\label{eqn:curvilinearspherical:80}
\begin{dmath}\label{eqn:curvilinearspherical:100}
\Bx_r = \Be_3 e^{j \theta}
\end{dmath}
\begin{dmath}\label{eqn:curvilinearspherical:120}
\Bx_\theta
= r \Be_3 j e^{j \theta}
= r \Be_3 \Be_{31} e^{i\phi} e^{j \theta}
= r \Be_1 e^{i\phi} e^{j \theta}
\end{dmath}
\begin{dmath}\label{eqn:curvilinearspherical:140}
\Bx_\phi
=
\PD{\phi}{} \lr{
r \Be_3 \sin\theta \Be_{31} e^{i \phi}
}
=
r \sin\theta \Be_1 \Be_{12} e^{i \phi}
=
r \sin\theta \Be_2 e^{i \phi}.
\end{dmath}
\end{subequations}

These are all mutually normal, which can be verified by computing dot products.
With that asserted, orthonormalization of the curvilinear basis is now possible by inspection

\begin{dmath}\label{eqn:curvilinearspherical:240}
\begin{aligned}
\rcap &= \Bx_r = \Be_3 e^{j \theta} \\
\thetacap &= \inv{r} \Bx_\theta = \Be_1 e^{i\phi} e^{j \theta} \\
\phicap &= \inv{r \sin\theta} \Bx_\phi = \Be_2 e^{i \phi},
\end{aligned}
\end{dmath}

so

\begin{dmath}\label{eqn:curvilinearspherical:260}
\begin{aligned}
\Bx^r &= \rcap = \Be_3 e^{j \theta} \\
\Bx^\theta &= \inv{r} \thetacap = \inv{r} \Be_1 e^{i\phi} e^{j \theta} \\
\Bx^\phi &= \inv{r \sin\theta} \phicap = \inv{r \sin\theta} \Be_2 e^{i \phi}.
\end{aligned}
\end{dmath}

\index{gradient!spherical}
In particular, this shows that the spherical representation of the gradient is
\begin{dmath}\label{eqn:curvilinearspherical:280}
\spacegrad
=
\Bx^r \PD{r}{}
+ \Bx^\theta \PD{\theta}{}
+ \Bx^\phi \PD{\phi}{}
=
\rcap \PD{r}{}
+\inv{r} \thetacap \PD{\theta}{}
+\inv{r \sin\theta} \PD{\phi}{}.
\end{dmath}

The spherical (oriented) volume element can also be computed in a compact fashion, without having to evaluate a very messy Jacobian determinant

\begin{dmath}\label{eqn:curvilinearspherical:300}
\Bx_r \wedge \Bx_\theta \wedge \Bx_\phi
=
\gpgradethree{
\Bx_r \Bx_\theta \Bx_\phi
}
=
\gpgradethree{
\Be_3 e^{j \theta}
r \Be_1 e^{i\phi} e^{j \theta}
r \sin\theta \Be_2 e^{i \phi}
}
=
r^2 \sin\theta
\gpgradethree{
\Be_3 e^{j \theta}
\Be_1 e^{i\phi} e^{j \theta}
\Be_2 e^{i \phi}
}
=
r^2 \sin\theta \Be_{123}
.
\end{dmath}

The final reduction is left as a problem for the student.
It is left to the student to evaluate whether this method is easier or more difficult than the conventional volume element Jacobean determinant expansion

\begin{dmath}\label{eqn:curvilinearspherical:320}
dV =
dr d\theta d\phi\,
\frac{\partial( x_1, x_2, x_3)}{\partial(r, \theta, \phi)}
=
dr d\theta d\phi\,
\begin{vmatrix}
\sin\theta \cos\phi & \sin\theta \sin\phi & \cos\theta \\
r \cos\theta \cos\phi & r \cos\theta \sin\phi & -r \sin\theta \\
-r \sin\theta \sin\phi & r \sin\theta \cos\phi & 0 \\
\end{vmatrix}.
\end{dmath}

It is easily argued that both volume element calculation methods are best performed by a computer algebra system.
