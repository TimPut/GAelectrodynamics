%
% Copyright © 2017 Peeter Joot.  All Rights Reserved.
% Licenced as described in the file LICENSE under the root directory of this GIT repository.
%
A general bivector has the form

\begin{dmath}\label{eqn:SimpleProducts2:1800}
B = \sum_{i \ne j} a_{ij} \Be_i \Be_j,
\end{dmath}

and is not necessarily a blade.  For example the bivector \( \Be_1 \Be_2 + \Be_3 \Be_4 \) cannot be factored into any product of normal vectors.
On the other hand,

\index{wedge factorization}
\maketheorem{Wedge product normal factorization}{thm:SimpleProducts2:wnormalfactorize}{
The wedge product of any two non-colinear vectors \( \Ba, \Bb \) always has a normal (2-blade) factorization
\begin{equation*}
\Ba \wedge \Bb = \Bu \Bv, \quad \Bu \cdot \Bv = 0.
\end{equation*}
} % theorem

This can be proven by construction.  Pick \( \Bu = \Ba \) and \( \Bv = \Rej{\Ba}{\Bb} \), then

\begin{dmath}\label{eqn:SimpleProducts2:1840}
\Ba \Rej{\Ba}{\Bb}
=
\cancel{\Ba \cdot \Rej{\Ba}{\Bb}}
+
\Ba \wedge \Rej{\Ba}{\Bb}
=
\Ba \wedge \lr{ \Bb - \frac{\Bb \cdot \Ba}{\Norm{\Ba}^2} \Ba }
=
\Ba \wedge \Bb,
\end{dmath}

since \( \Ba \wedge (\alpha \Ba) = 0 \) for any scalar \( \alpha \).

The significance of \cref{thm:SimpleProducts2:wnormalfactorize} is that the square of any wedge product is negative

\begin{dmath}\label{eqn:SimpleProducts2:1820}
(\Bu \Bv)^2
=
(\Bu \Bv) (-\Bv \Bu)
=
-\Bu (\Bv^2) \Bu
=
- \Abs{\Bu}^2 \Abs{\Bv}^2,
\end{dmath}

which in turn means that exponentials with wedge product arguments can be used as rotation operators.


