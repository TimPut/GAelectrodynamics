%
% Copyright © 2017 Peeter Joot.  All Rights Reserved.
% Licenced as described in the file LICENSE under the root directory of this GIT repository.
%
%{

In general the product of two k-vectors is a multivector, with a selection of different grades.
For example, the product of two bivectors may have grades 0, 2, or 4
\begin{dmath}\label{eqn:generalizedDot:601}
\Be_{12} \lr{ \Be_{21} + \Be_{23} + \Be_{34} }
=
1 + \Be_{13} + \Be_{1234}.
\end{dmath}

Similarly,
the product of a vector and bivector generally has grades 1 and 3
\begin{dmath}\label{eqn:generalizedDot:621}
\Be_1 \lr{ \Be_{12} + \Be_{23} }
=
\Be_2 + \Be_{123}.
\end{dmath}

The dot product was identified with scalar grade selection, which picks out the lowest grade of their product.
This motivates the definition of a general multivector dot product

\index{multivector dot product}
\index{\(A \cdot B\)}
\makedefinition{Multivector dot product}{dfn:generalizedDot:100}{
The dot (or inner) product of two multivectors
\( A = \sum_{i = 0}^N \gpgrade{A}{i}, B = \sum_{i = 0}^N \gpgrade{B}{i} \)
is defined as
\begin{equation*}
A \cdot B \equiv
\sum_{i,j = 0}^N \gpgrade{ A_i B_j }{\Abs{i - j}}.
\end{equation*}
}

Consider some examples.  First, if \( A, B \) are k-vectors with equal grade, then the dot product is just the scalar selection of their product
\begin{dmath}\label{eqn:generalizedDot:580}
A \cdot B = \gpgradezero{ A B }.
\end{dmath}
Specific examples of this are
\begin{dmath}\label{eqn:generalizedDot:741}
\begin{aligned}
\Be_{12} \cdot \Be_{23} &= \gpgradezero{ \Be_{12} \Be_{23} } = 0 = \gpgradezero{ \Be_{1} \Be_{3} } = 0 \\
\Be_{32} \cdot \Be_{23} &= \gpgradezero{ \Be_{32} \Be_{23} } = 1 \\
\Be_{123} \cdot \Be_{312} &= \gpgradezero{ \Be_{123} \Be_{312} } = \gpgradezero{ \Be_{12} \Be_{12} } = -1.
\end{aligned}
\end{dmath}
If \( A, B \) are a k-vectors with grades \( r \ne s \) respectively, then their dot product is a single grade selection
\begin{dmath}\label{eqn:generalizedDot:581}
A \cdot B = \gpgrade{ A B }{\Abs{r - s}}.
\end{dmath}
\Cref{eqn:generalizedDot:621} provides a trivial multivector example, where we see that
\begin{dmath}\label{eqn:generalizedDot:761}
\begin{aligned}
\Be_1 \cdot \lr{ \Be_{12} + \Be_{23} }
&=
\Be_2
 + \Be_{123} \\
\Be_1 \wedge \lr{ \Be_{12} + \Be_{23} }
&=
 \Be_{123}.
\end{aligned}
\end{dmath}

\makeexample{Multivector dot products.}{example:generalizedDot:661}{

The most common and useful multivector dot products are for pairs of multivectors that are each entirely a specific grade, such as a vector-bivector dot product
\begin{dmath}\label{eqn:generalizedDot:681}
\lr{ \Be_1 + 2 \Be_2 } \cdot \lr{ \Be_{12} + \Be_{23} }
=
\gpgradeone{
   \lr{ \Be_1 + 2 \Be_2 } \lr{ \Be_{12} + \Be_{23} }
}
=
\Be_2 - 2 \Be_1 + 2 \Be_3,
\end{dmath}
or a bivector-trivector dot product
\begin{dmath}\label{eqn:generalizedDot:701}
\lr{ \Be_1 + 2 \Be_2 } \cdot \Be_{123}
=
\gpgradetwo{
\lr{ \Be_1 + 2 \Be_2 } \Be_{123}
}
=
\Be_{23} + 2 \Be_{31}.
\end{dmath}

Should the products be of mixed grade, then we sum all the individual dot products
\begin{dmath}\label{eqn:generalizedDot:721}
\lr{ 1 + \Be_1 + 2 \Be_{23 } } \cdot \lr{ \Be_2 - \Be_{31} }
=
\gpgradeone{
1 \Be_2
}
+
\gpgradezero{
\Be_{1} \Be_2
}
+
2
\gpgradeone{
\Be_{23} \Be_2
}
-\gpgradetwo{
1 \Be_{31}
}
-
\gpgradeone{
\Be_{1} \Be_{31}
}
-
2
\gpgradezero{
\Be_{23} \Be_{31}
}
=
\Be_2
-
2 \Be_3
+
\Be_{13}
+ \Be_3.
\end{dmath}
} % example

Unfortunately, the notation for the multivector dot product is not standardized.
In particular, some authors \citep{dorst2007gac} prefer left and right contraction operations that omit the absolute value in the grade selections.  A
dot product like operator for scalar selection is also common.

\index{\(A \rfloor B\)}
\index{\(A \lfloor B\)}
\index{\(A \conj B\)}
\makedefinition{Alternate dot products.}{dfn:generalizedDot:641}{

The left and right contraction operations are respectively defined as
\begin{equation*}
\begin{aligned}
A \rfloor B &= \sum_{i,j = 0}^N \gpgrade{ A_i B_j }{j - i} \\
A \lfloor B &= \sum_{i,j = 0}^N \gpgrade{ A_i B_j }{i - j},
\end{aligned}
\end{equation*}
where any selection of a negative grade is taken to be zero.  The scalar product is defined as
\begin{equation*}
A \conj B = \sum_{i,j = 0}^N \gpgradezero{ A_i B_j }
\end{equation*}
} % definition

In an attempt to avoid inundating the reader with too many new operators, this book will stick to the dot, wedge and grade selection operators.  However, these are common enough that they deserve mentioning.
%}
