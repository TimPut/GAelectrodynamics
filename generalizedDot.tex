%
% Copyright © 2017 Peeter Joot.  All Rights Reserved.
% Licenced as described in the file LICENSE under the root directory of this GIT repository.
%

In general the product of two k-vectors is a multivector, with a selection of different grades.
For example, the product of two bivectors may have grades 0, 2, or 4

\begin{dmath}\label{eqn:generalizedDot:601}
\Be_{12} \lr{ \Be_{21} + \Be_{23} + \Be_{34} }
=
1 + \Be_{13} + \Be_{1234}.
\end{dmath}

Similarly,
the product of a vector and bivector generally has grades 1 and 3

\begin{dmath}\label{eqn:generalizedDot:621}
\Be_1 \lr{ \Be_{12} + \Be_{23} }
=
\Be_2 + \Be_{123}.
\end{dmath}

We've identified the vector dot product with scalar grade selection of their vector product, the selection of the lowest grade of their product.
This motivates the definition of a general multivector dot product

\index{multivector dot product}
\makedefinition{Multivector dot product}{dfn:generalizedDot:100}{
The dot (or inner) product of two multivectors
\( A = \sum_{i = 0}^N \gpgrade{A}{i}, B = \sum_{i = 0}^N \gpgrade{B}{i} \)
is defined as
\begin{equation*}
A \cdot B \equiv
\sum_{i,j = 0}^N \gpgrade{ A_i B_j }{\Abs{i - j}}.
\end{equation*}
}

If \( A, B \) are k-vectors with equal grade, then the dot product is just the scalar selection of their product

\begin{dmath}\label{eqn:generalizedDot:580}
A \cdot B = \gpgradezero{ A B },
\end{dmath}

and if \( A, B \) are a k-vectors with grades \( r \ne s \) respectively, then their dot product is a single grade selection

\begin{dmath}\label{eqn:generalizedDot:581}
A \cdot B = \gpgrade{ A B }{\Abs{r - s}}.
\end{dmath}

