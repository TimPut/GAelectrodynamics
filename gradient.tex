%
% Copyright � 2018 Peeter Joot.  All Rights Reserved.
% Licenced as described in the file LICENSE under the root directory of this GIT repository.
%
%{
%%%\input{../latex/blogpost.tex}
%%%\renewcommand{\basename}{gradient}
%%%%\renewcommand{\dirname}{notes/phy1520/}
%%%\renewcommand{\dirname}{notes/ece1228-electromagnetic-theory/}
%%%%\newcommand{\dateintitle}{}
%%%%\newcommand{\keywords}{}
%%%
%%%\input{../latex/peeter_prologue_print2.tex}
%%%
%%%\usepackage{peeters_layout_exercise}
%%%\usepackage{peeters_braket}
%%%\usepackage{peeters_figures}
%%%\usepackage{siunitx}
%%%%\usepackage{mhchem} % \ce{}
%%%%\usepackage{macros_bm} % \bcM
%%%%\usepackage{macros_qed} % \qedmarker
%%%%\usepackage{txfonts} % \ointclockwise
%%%
%%%\beginArtNoToc
%%%
%%%\generatetitle{Gradient and vector derivative.}
%%%%\chapter{Gradient.}
%%%\label{chap:gradient}
%%%
%%%\paragraph{definition}
%%%%
% Copyright © 2018 Peeter Joot.  All Rights Reserved.
% Licenced as described in the file LICENSE under the root directory of this GIT repository.
%
%{

We can extend the previous two parameter subspace ideas to higher dimensional (or one dimensional) subspaces associated with a parameterization

\index{tangent space}
\index{curvilinear coordinates}
\index{oriented volume element}
\index{volume element}
\index{\(d^k \Bx\)}
\index{\(\Bx_i\)}
\makedefinition{Curvilinear coordinates and volume element}{dfn:curvilinearThree:280}{
Given a parameterization \( \Bx(u_1, u_2, \cdots, u_k) \) with \( k \) degrees of freedom, we define the curvilinear basis elements \( \Bx_i \) by the partials
\begin{equation*}
\Bx_{i} = \PD{u_i}{\Bx}.
\end{equation*}
The span of \( \setlr{ \Bx_{i} } \) at the point of evaluation is called the tangent space.
A subspace associated with a parameterization of this sort is also called a manifold.
The volume element for the subspace is
\begin{equation*}
d^k \Bx = du_1 du_2 \cdots du_k\,
\Bx_{1} \wedge
\Bx_{2} \wedge \cdots \wedge
\Bx_{k}.
\end{equation*}
Such a volume element is a k-vector.  The volume of the (hyper-) parallelepiped bounded by \( \setlr{ \Bx_{i} } \)  is \( \sqrt{\Abs{(d^k \Bx)^2}} \).
} % definition

We will assume that the parameterization is non-generate.
This means that the
volume element \( d^k \Bx \) is non-zero in the region of interest.
Note that a zero volume element implies a linear dependency in the curvilinear basis elements \( \Bx_i \).

Given a parameterization \( \Bx = \Bx(u,v,\cdots, w) \), write
\( \Bx_u, \Bx_v, \cdots, \Bx_w \) for the curvilinear basis elements, and
\( \Bx^u, \Bx^v, \cdots, \Bx^w \) for the reciprocal frame.
When doing so, sums over numeric indexes like \( \sum_i \Bx^i \Bx_i \) should be interpreted as a sum over all the parameter labels, i.e. \( \Bx^u \Bx_u + \Bx^v \Bx_v + \cdots \).

%}

%%%
%%%\paragraph{Gradient.}
%%%
With the introduction of the ideas of reciprocal frame and curvilinear coordinates, we are getting closer to be able to formulate the geometric algebra generalizations of vector calculus.

The next step in the required mathematical preliminaries for geometric calculus is to determine the form of the gradient with respect to curvilinear coordinates and the
parameters associated with those coordinates.

Suppose we have a vector parameterization of \R{N}
\begin{dmath}\label{eqn:gradient:60}
\Bx = \Bx(u_1, u_2, \cdots, u_N).
\end{dmath}

We can employ the chain rule to express the gradient in terms of derivatives with respect to \( u_i \)
\begin{dmath}\label{eqn:gradient:80}
\spacegrad
=
\sum_i \Be_i \PD{x_i}{}
=
\sum_{i,j} \Be_i
\PD{x_i}{u_j}
\PD{u_j}{}
=
\sum_j \lr{ \sum_i \Be_i \PD{x_i}{u_j} } \PD{u_j}{}
=
\sum_j \lr{ \spacegrad u_j } \PD{u_j}{}.
\end{dmath}

It turns out that the gradients of the parameters are in fact the reciprocal frame vectors

\maketheorem{Reciprocal frame vectors}{thm:curvilinearGradient:1}{
Given a curvilinear basis with elements \( \Bx_i = \PDi{u_i}{\Bx} \), the reciprocal frame vectors are given by
\begin{dmath*}
\Bx^i = \spacegrad u_i.
\end{dmath*}
} % theorem

This can be proven by direct computation
\begin{dmath}\label{eqn:gradient:20}
\Bx^i \cdot \Bx_j
=
(\spacegrad u_i) \cdot \PD{u_j}{\Bx}
=
\sum_{r,s=1}^n
\lr{ \Be_r \PD{x_r}{u_i} } \cdot \lr{ \Be_s \PD{u_j}{x_s} }
=
\sum_{r,s=1}^n (\Be_r \cdot \Be_s)
\PD{x_r}{u_i} \PD{u_j}{x_s}
=
\sum_{r,s=1}^n \delta_{rs}
\PD{x_r}{u_i} \PD{u_j}{x_s}
=
\sum_{r=1}^n
\PD{x_r}{u_i} \PD{u_j}{x_r}
=
\PD{u_i}{u_j}
=
\delta_{ij}.
\end{dmath}

This shows that \( \Bx^i = \spacegrad u_i \) has the properties required of the reciprocal frame, proving the theorem.  We are now able to define the gradient with respect to an arbitrary set of parameters

\maketheorem{Curvilinear representation of the gradient}{thm:curvilinearGradient:2}{
Given an N-parameter vector parameterization
\( \Bx = \Bx(u_1, u_2, \cdots, u_N) \)
of \R{N},
with curvilinear basis elements \( \Bx_i = \PDi{u_i}{\Bx} \), the gradient can be expressed as
\begin{dmath*}
\spacegrad = \sum_i \Bx^i \PD{u_i}{}.
\end{dmath*}
It is often convenient to define \( \partial_i \equiv \PDi{u_i}{} \), so that the gradient can be expressed in mixed index representation
\begin{dmath*}
\spacegrad = \sum_i \Bx^i \partial_i.
\end{dmath*}
%or the same with sums over mixed indexes implied.
} % theorem


%}
%\EndArticle
