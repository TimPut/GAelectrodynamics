%
% Copyright © 2017 Peeter Joot.  All Rights Reserved.
% Licenced as described in the file LICENSE under the root directory of this GIT repository.
%
The GA formulation of Maxwell's equation has only been applied in media where it has been assumed throughout that linear constitutive relationships

\begin{dmath}\label{eqn:inMatter:20}
\begin{aligned}
\BD &= \epsilon \BE \\
\BB &= \mu \BH,
\end{aligned}
\end{dmath}

have been available.  Without such assumptions the GA formalism for Maxwell's equations cannot be written as a single equation with one multivector field, but requires two equations and two multivector fields.  The two multivector fields are

\begin{dmath}\label{eqn:inMatter:40}
\begin{aligned}
F &= \BE + I c \BB \\
G &= \BD + \frac{I}{c} \BH
\end{aligned}
\end{dmath}

for which Maxwell's equations are

\begin{dmath}\label{eqn:inMatter:60}
\begin{aligned}
\gpgrade{ \lr{ \spacegrad + \inv{c} \PD{t}{} } G }{0,1} &= \rho - \frac{\BJ}{c} \\
\gpgrade{ \lr{ \spacegrad + \inv{c} \PD{t}{} } F }{2,3} &= I \lr{ c \rho_m - \BM }.
\end{aligned}
\end{dmath}

Here \( c \) is a non-dimensionalizing constant with dimensions [L/T], but is otherwise unspecified.
Direct expansion can be used to show that \cref{eqn:inMatter:60} is equivalent to Maxwell's equations.
Doing so for each of the grades in turn, we have

\begin{subequations}
\label{eqn:inMatter:80}
\begin{dmath}\label{eqn:inMatter:100}
\rho
=
\gpgradezero{ \lr{ \spacegrad + \inv{c} \PD{t}{} } G }
=
\gpgradezero{ \lr{ \spacegrad + \inv{c} \PD{t}{} } \lr{ \BD + \frac{I}{c} \BH } }
=
\spacegrad \cdot \BD
\end{dmath}
\begin{dmath}\label{eqn:inMatter:120}
- \frac{\BJ}{c}
=
\gpgradeone{ \lr{ \spacegrad + \inv{c} \PD{t}{} } G }
=
\gpgradeone{ \lr{ \spacegrad + \inv{c} \PD{t}{} } \lr{ \BD + \frac{I}{c} \BH } }
=
\inv{c} \PD{t}{\BD} + \frac{I}{c} \spacegrad \wedge \BH
=
\inv{c} \PD{t}{\BD} - \frac{1}{c} \spacegrad \cross \BH
\end{dmath}
\begin{dmath}\label{eqn:inMatter:140}
- I \BM
=
\gpgrade{ \lr{ \spacegrad + \inv{c} \PD{t}{} } F }{2}
=
\gpgrade{ \lr{ \spacegrad + \inv{c} \PD{t}{} } \lr{ \BE + I c \BB} }{2}
=
\spacegrad \wedge \BE + I \PD{t}{\BB}
\end{dmath}
\begin{dmath}\label{eqn:inMatter:160}
I c \rho_m
=
\gpgrade{ \lr{ \spacegrad + \inv{c} \PD{t}{} } F }{3}
=
\gpgrade{ \lr{ \spacegrad + \inv{c} \PD{t}{} } \lr{ \BE + I c \BB} }{3}
=
c I \spacegrad \cdot \BB.
\end{dmath}
\end{subequations}

After rearranging and cancelling common factors of \( c, I \) Maxwell's equations are recovered

\begin{dmath}\label{eqn:inMatter:180}
\begin{aligned}
\spacegrad \cdot \BD &= \rho \\
\spacegrad \cross \BH &= \BJ + \PD{t}{\BD}  \\
\spacegrad \cross \BE &= -\BM - \PD{t}{\BB} \\
\spacegrad \cdot \BB &= \rho_m.
\end{aligned}
\end{dmath}

One possible strategy for solving these equations is to impose an additional set of constraints on the grades in question

\begin{dmath}\label{eqn:inMatter:200}
\begin{aligned}
\gpgrade{ \lr{ \spacegrad + \inv{c} \PD{t}{} } G }{2,3} &= 0 \\
\gpgrade{ \lr{ \spacegrad + \inv{c} \PD{t}{} } F }{0,1} &= 0,
\end{aligned}
\end{dmath}

so that all the grade selection filters can be cleared

\begin{dmath}\label{eqn:inMatter:220}
\begin{aligned}
\lr{ \spacegrad + \inv{c} \PD{t}{} } G &= \rho - \frac{\BJ}{c} \\
\lr{ \spacegrad + \inv{c} \PD{t}{} } F &= I \lr{ c \rho_m - \BM }.
\end{aligned}
\end{dmath}

Each of these now separately has the form of Maxwell's equation, and could be solved separately, subject to the constraint equations.
Only if \( F, G \) can be related by a constant factor, say \( \epsilon F = G \), can these be summed directly (after non-dimensional scaling) to form Maxwell's equation.
Other non-constraint strategies for solving \cref{eqn:inMatter:60} would require additional thought and study.
