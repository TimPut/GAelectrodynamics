%
% Copyright © 2017 Peeter Joot.  All Rights Reserved.
% Licenced as described in the file LICENSE under the root directory of this GIT repository.
%
\index{boundary values}
\maketheorem{Boundary value relations.}{thm:boundarySurfaceSources:480}{
The difference in the normal and tangential components of the electromagnetic field spanning a surface on which there are
a surface current or surface charge or current densities can be related to those surface sources as follows
%\label{eqn:boundarySurfaceSources:420}
\begin{equation*}
\begin{aligned}
\gpgradezero{\ncap (\epsilon_2 F_2 - \epsilon_1 F_1) } &= \rho_s \\
\gpgradeone{\ncap \lr{\inv{\eta_2} F_2 - \inv{\eta_1} F_1 } } &= - \BJ_s \\
\gpgradezero{\ncap I \lr{ \inv{v_2}F_2 - \inv{v_1} F_1 } } &= - \rho_{ms} \\
\gpgradeone{\ncap I (F_2 - F_1)} &= \BM_s.
\end{aligned}
\end{equation*}
In terms of the conventional constituent fields, these may be written
%\label{eqn:boundarySurfaceSources:460}
\begin{equation*}
\begin{aligned}
\ncap \cdot \lr{ \BD_2 - \BD_1 } &= \rho_s \\
\ncap \cross \lr{ \BH_2 - \BH_1 } &= \BJ_s \\
\ncap \cdot \lr{ \BB_2 - \BB_1 } &= \rho_{ms} \\
\ncap \cross \lr{ \BE_2 - \BE_1 } &= -\BM_s.
\end{aligned}
\end{equation*}
} % theorem

These relationships can be determined by integrating Maxwell's equation over the
pillbox configuration illustrated in \cref{fig:ps3Problem1Pillbox:ps3Problem1PillboxFig1}.

%\imageFigure{../figures/ece1228-electromagnetic-theory/ps3Problem1PillboxFig1}{Pillbox integration volume.}{fig:ps3Problem1Pillbox:ps3Problem1PillboxFig1}{0.2}
\imageFigure{../figures/GAelectrodynamics/pillboxIntegrationVolumeFig1}{Pillbox integration volume.}{fig:ps3Problem1Pillbox:ps3Problem1PillboxFig1}{0.3}

An assumption that the sources are primarily constrained to the surface can be written as
\begin{dmath}\label{eqn:boundarySurfaceSources:20}
J = J_s \delta(y),
\end{dmath}
where the \( y \) coordinate is locally normal to the surface at any given point.
In terms of the scalar and vector potentials, such a surface source model is
\begin{dmath}\label{eqn:boundarySurfaceSources:40}
J = \lr{ \eta\lr{ c \rho_s - \BJ_s } + I \lr{ c \rho_{ms} - \BM_s } }
\delta(y).
\end{dmath}

It will be
simplest to demonstrate the boundary relationships in the frequency domain, where Maxwell's equation can be written as either

\begin{subequations}
\label{eqn:boundarySurfaceSources:60}
\begin{dmath}\label{eqn:boundarySurfaceSources:80}
\spacegrad F = -j k F + J,
\end{dmath}
or
\begin{dmath}\label{eqn:boundarySurfaceSources:100}
\spacegrad I F = -j k I F + I J.
\end{dmath}
\end{subequations}

Application of contraction operations gives

\begin{subequations}
\label{eqn:boundarySurfaceSources:120}
\begin{dmath}\label{eqn:boundarySurfaceSources:140}
\spacegrad \cdot F
= \gpgrade{-j k F + J}{0,1}
= -j k \BE + \eta( c \rho_s - \BJ_s ) \delta(y)
\end{dmath}
\begin{dmath}\label{eqn:boundarySurfaceSources:160}
\spacegrad \cdot (I F)
= \gpgrade{-j k I F + I J}{0,1}
= j k \eta \BH - ( c \rho_{ms} - \BM_s ) \delta(y).
\end{dmath}
\end{subequations}

Each of these contraction operations can be evaluated over the pillbox volume above using the divergence theorem, however, the delta function integrals are problematic.
Those integrals dependent on \( \eta \) and \( c \) which vary across the surface, but are also dependent on the delta function surface contribution, which is valid at only the surface.
Consider the vector potential term for electric sources as an example, where the volume integral of that term is
\begin{dmath}\label{eqn:boundarySurfaceSources:180}
-\int dV \eta \BJ_s \delta(y)
=
-\int_{y=0}^{h/2} \int dA \eta_2 \BJ_s \delta(y)
-\int_{y=-h/2}^0 \int dA \eta_1 \BJ_s \delta(y).
\end{dmath}

The delta function is only well defined when integrated across the \( y = 0 \) point.
This problem can be overcome by applying grade selection operations to each of the components of \cref{eqn:boundarySurfaceSources:120}, and then rearranging so that all the medium specific contributions to the integrals are factored away from the delta functions

\begin{subequations}
\label{eqn:boundarySurfaceSources:200}
\begin{dmath}\label{eqn:boundarySurfaceSources:220}
\gpgradezero{\spacegrad \cdot \lr{ \epsilon F}} = \rho_s \delta(y)
\end{dmath}
\begin{dmath}\label{eqn:boundarySurfaceSources:240}
\gpgradeone{\spacegrad \cdot \lr{ \inv{\eta} F}} = -j \frac{k}{\eta} \BE - \BJ_s \delta(y)
\end{dmath}
\begin{dmath}\label{eqn:boundarySurfaceSources:260}
\gpgradezero{\spacegrad \cdot \lr{ I \inv{c}F}} = - \rho_{ms} \delta(y)
\end{dmath}
\begin{dmath}\label{eqn:boundarySurfaceSources:280}
\gpgradeone{\spacegrad \cdot \lr{ I F}} = j k \eta \BH + \BM_s \delta(y).
\end{dmath}
\end{subequations}

Each of the grade selections picks off one of \( \BD, \BB, \BH \) or \( \BE \), so this could have been obtained directly from the conventional set of individual Maxwell equations, however, it is instructional to see how to work with the complete electromagnetic field \( F \).
This also provides a method of evaluating the boundary conditions that is both coordinate free, and uses the same integral form for all the boundary conditions.

Applying \cref{thm:volumeintegral:2661}, the multivector divergence theorem, gives

\begin{subequations}
\label{eqn:boundarySurfaceSources:300}
\begin{dmath}\label{eqn:boundarySurfaceSources:320}
\gpgradezero{\int dV \ncap \cdot \lr{ \epsilon F}} = \Delta A \rho_s
\end{dmath}
\begin{dmath}\label{eqn:boundarySurfaceSources:340}
\gpgradeone{\int dV \ncap \cdot \lr{ \inv{\eta} F}} = -j \omega \int_y dy \int dA \BD - \Delta A \BJ_s
\end{dmath}
\begin{dmath}\label{eqn:boundarySurfaceSources:360}
\gpgradezero{\int dV \ncap \cdot \lr{ I \inv{c}F}} = - \Delta A \rho_{ms}
\end{dmath}
\begin{dmath}\label{eqn:boundarySurfaceSources:380}
\gpgradeone{\int dV \ncap \cdot \lr{ I F}} = j \omega \int_y dy \int dA \BB + \Delta A \BM_s
\end{dmath}
\end{subequations}

The \( y \) (normal) integral components of the volume integrals are all assumed to vanish as \( \Delta y \rightarrow 0 \), providing the first relations
stated in \cref{thm:boundarySurfaceSources:480}.
Expanding the grade selection operations, we find
\begin{dmath}\label{eqn:boundarySurfaceSources:440}
\begin{aligned}
\ncap \cdot (\BD_2 - \BD_1) &= \rho_s \\
I \ncap \wedge \lr{ \BH_2 - \BH_1 } &= - \BJ_s \\
-\ncap \cdot (\BB_2 - \BB_1) &= - \rho_{ms} \\
I \ncap \wedge (\BE_2 - \BE_1) &= \BM_s,
\end{aligned}
\end{dmath}
and expansion of the wedge's using \cref{eqn:SimpleProducts2:1620} completes the proof.

The crazy jumble of dot products, cross products and field components in this conventional statement of the boundary conditions is seen to follow systematically from Maxwell's equation \cref{eqn:boundarySurfaceSources:80}, and reflects the fact that the components of Maxwell's equation have to be treated individually by grade when evaluating the boundary integrals.

In the special case where there are surface charge and current densities along the interface surface, but the media is uniform (\(\epsilon_1 = \epsilon_2, \mu_1 = \mu_2\)), then the field and current relationship has a particularily simple form \citep{chappell2014geometric}
\begin{dmath}\label{eqn:boundarySurfaceSources:421}
\ncap (F_2 - F_1) = J_s.
\end{dmath}

\makeproblem{Uniform media with currents and densities.}{problem:boundarySurfaceSources:1}{
Prove that \cref{eqn:boundarySurfaceSources:421} is equivalent to {eqn:boundarySurfaceSources:420} if \( \epsilon_1 = \epsilon_2, \mu_1 = \mu_2 \).
} % problem
