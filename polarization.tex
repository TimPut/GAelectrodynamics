%
% Copyright © 2017 Peeter Joot.  All Rights Reserved.
% Licenced as described in the file LICENSE under the root directory of this GIT repository.
%
\subsection{Polarization.}
\index{plane wave}
\index{polarization}
In a discussion of polarization, it is convenient to align the propagation direction along a fixed direction, usually the z-axis.
Setting \( \kcap = \Be_3, \beta z = \Bk \cdot \Bx \) in \cref{eqn:frequencydomainCore:200} the plane wave representation of the field is

\begin{dmath}\label{eqn:polarization:20}
\begin{aligned}
F(\Bx, \omega) &= (1 + \Be_3) \BE e^{-j \beta z} \\
F(\Bx, t) &= \Real\lr{ F(\Bx, \omega) e^{j \omega t} }.
\end{aligned}
\end{dmath}

Here the imaginary \( j \) has no intrinsic geometrical interpretation, \( \BE = \BE_\txtr + j \BE_\txti \) is allowed to have complex values, and all components of \( \BE \) is perpendicular to the propagation direction (\( \Be_\txtr \cdot \Be_3 = \BE_\txti \cdot \Be_3 = 0 \)).
\index{Jones vector}
A common representation of the electric field components is the Jones vector \( (c_1, c_2) \), which specifies complex coefficients for the electric field phasor in each of the possible directions

\begin{dmath}\label{eqn:polarization:120}
\BE = c_1 \Be_1 + c_2 \Be_2,
\end{dmath}

where \( c_1, c_2 \) are complex valued, say

\begin{dmath}\label{eqn:polarization:140}
\begin{aligned}
c_1 &= \alpha_1 + j \beta_1 \\
c_2 &= \alpha_2 + j \beta_2.
\end{aligned}
\end{dmath}

The tuple \( (c_1, c_2) \) is called the Jones vector, and compactly encodes the geometry of the pattern that the electric field traces out in the transverse plane.

\subsection{Circular polarization basis.}
\index{circular polarization}
\index{left circular polarization}
\index{right circular polarization}

The time domain field when written out explicitly in terms of the Jones vector components is

\begin{dmath}\label{eqn:polarization:160}
F(\Bx, t) = (1 + \Be_3) \lr{
\lr{ \alpha_1 \Be_1 + \alpha_2 \Be_2 } \cos\lr{ \omega t - \beta z }
-\lr{ \beta_1 \Be_1 + \beta_2 \Be_2 } \sin\lr{ \omega t - \beta z }
}.
\end{dmath}

Linear, circular, and elliptical polarization patterns can be obtained using specific values for \( \alpha_1, \alpha_2, \beta_1, \beta_2 \).
In particular,
a field for which the
change in phase \( \phi = \omega t - \beta z \) results in the electric field tracing out a (counterclockwise,clockwise) circle

\begin{dmath}\label{eqn:polarization:180}
\begin{aligned}
\BE_\txtR &= \Abs{\BE} \lr{ \Be_1 \cos\phi + \Be_2 \sin\phi } = \Abs{\BE} \Be_1 \exp\lr{  \Be_{12} \phi } \\
\BE_\txtL &= \Abs{\BE} \lr{ \Be_1 \cos\phi - \Be_2 \sin\phi } = \Abs{\BE} \Be_1 \exp\lr{ -\Be_{12} \phi },
\end{aligned}
\end{dmath}

is referred to as having
(left,right) circular polarization.
There are different conventions for the polarization orientation, and here the IEEE antenna convention discussed in \citep{balanis1989advanced} are used.

Fixme: check that I have this orientation right.  Noticed after the fact that the figures in Balanis use an orientation with x-axis up and y-axis right!

The bivector exponential representation of the circularly polarized electric fields in \cref{eqn:polarization:180} indicates that it is possible to represent arbitrary field polarization in a GA form that does not require any real part operation, as follows

\begin{dmath}\label{eqn:polarization:200}
F = \lr{ 1 + \Be_3 } \Be_1 \lr{ \alpha_\txtR e^{i\phi} + \alpha_\txtL e^{-i\phi} },
\end{dmath}

where the constants \( \alpha_\txtR, \alpha_\txtL \) are both complex with respect to the unit bivector imaginary \( i = \Be_{12} \) representing the plane transverse to the propagation direction

\begin{dmath}\label{eqn:polarization:220}
\begin{aligned}
\alpha_\txtR &= \alpha_{\txtR 1} + i \alpha_{\txtR 2} \\
\alpha_\txtL &= \alpha_{\txtL 1} + i \alpha_{\txtL 2}.
\end{aligned}
\end{dmath}

If a transformation from scalar to bivector imaginary \( j \rightarrow \Be_{12} = i \) is made in the Jones vector component representation of \cref{eqn:polarization:140},
then
the coefficients \cref{eqn:polarization:220} of the circular polarization states are related to the Jones vector by (\cref{problem:polarization:1})

\begin{dmath}\label{eqn:polarization:260}
\begin{aligned}
\alpha_\txtR &= \inv{2}\lr{ c_1 - i c_2 } \\
\alpha_\txtL &= \inv{2}\lr{ c_1 + i c_2 }^\dagger.
\end{aligned}
\end{dmath}

\subsection{Linear polarization.}

Linear polarization is described by

\begin{dmath}\label{eqn:polarization:280}
\begin{aligned}
\alpha_\txtR &= \inv{2}\Abs{\BE} \Be_1 e^{i(\psi + \theta)} \\
\alpha_\txtL &= \inv{2}\Abs{\BE} \Be_1 e^{i(\psi - \theta)},
\end{aligned}
\end{dmath}

or
\begin{dmath}\label{eqn:polarization:300}
F = \lr{ 1 + \Be_3 } \Abs{\BE} \Be_1 e^{i\psi} \cos( \omega t - \beta z + \theta ).
\end{dmath}

The electric field \( \BE \) traces out all the points along the line spanning the points between \( \pm \Be_1 e^{i\psi} \Abs{\BE} \), whereas the magnetic field \( \BH \) traces
out all the points along \( \pm \Be_2 e^{i\psi} \Abs{\BE}/\eta \) as illustrated (with \( \eta = 1 \)) in
\cref{fig:linearPolarization:linearPolarizationFig1}.

\imageFigure{../figures/GAelectrodynamics/linearPolarizationFig1}{Linear polarization.}{fig:linearPolarization:linearPolarizationFig1}{0.3}

\subsection{Elliptical parameterization.}

An ellipical polarized electric field can be parameterized as
\begin{dmath}\label{eqn:ellipticalWaves:340}
\BE
=
E_a \Be_1 \cos\theta + E_b \Be_2 \sin\theta,
\end{dmath}

which corresponds to a Jones vector \( (E_a, -i E_b) \), or circular polarization coefficients with values

\begin{dmath}\label{eqn:polarization:400}
\begin{aligned}
\alpha_\txtR &= \inv{2}\lr{ E_a - E_b } \\
\alpha_\txtL &= \inv{2}\lr{ E_a + E_b }.
\end{aligned}
\end{dmath}

Therefore an elliptically polarized field can be represented as

\begin{dmath}\label{eqn:polarization:420}
F = \inv{2} (1 + \Be_3) \Be_1 \lr{ (E_a + E_b) e^{i\phi} + (E_a - E_b) e^{-i\phi} }.
\end{dmath}

An interesting variation of the elliptical polarization uses a hyperbolic parameterization.
If \( a, b \) are the semi-major/minor axes of the ellipse (i.e. \( a > b \)),
and \( \Ba = a \Be_1 e^{i\psi} \) is the vectoral representation of the semimajor axis (not necessarily placed along \( \Be_1 \)),
and \( e = \sqrt{1 - (b/a)^2} \) is the eccentricity of the ellipse,
then it can be shown (\citep{hestenes1999nfc})
that an elliptic parameterization can be written
in the compact form

\begin{dmath}\label{eqn:ellipticalWaves:360}
\Br(\phi)
=
e \Ba \cosh( \Atanh(b/a) + i \phi).
\end{dmath}

When the bivector imaginary \( i = \Be_{12} \) is used then
this parameterization is real and has only vector grades, so the electromagnetic field for a general elliptic wave has the form

\begin{dmath}\label{eqn:ellipticalWaves:380}
\begin{aligned}
F &= e E_a \lr{ 1 + \Be_3 } \Be_1 e^{ i \psi } \cosh\lr{ \mu + i \phi} \\
\mu &= \Atanh\lr{ E_b/E_a } \\
e &= \sqrt{1 - {(E_b/E_a)}^2 },
\end{aligned}
\end{dmath}

where \( E_a(E_b) \) are the magnitudes of the electric field components lying along the semi-major(minor) axes, and the propagation direction \( \Be_3 \) is normal to both the major and minor axis directions, as illustrated in \cref{fig:ellipticalPolarization:ellipticalPolarizationFig1}.
Observe that setting \( E_b = 0 \) results in the linearly polarized field of \cref{eqn:polarization:300}.
\imageFigure{../figures/GAelectrodynamics/ellipticalPolarizationFig1}{Electric field with elliptical polarization.}{fig:ellipticalPolarization:ellipticalPolarizationFig1}{0.3}

\subsection{Pseudoscalar imaginary.}

...

\subsection{Problems.}

\makeproblem{Circular polarization coefficients relationship to the Jones vector.}{problem:polarization:1}{
By substituting \cref{eqn:polarization:220} into \cref{eqn:polarization:200}, and comparing to \cref{eqn:polarization:160},
show that the circular state coefficients have the following relationship to the Jones vector coordinates
\begin{equation*}
\begin{aligned}
\alpha_\txtR &= \lr{ \alpha_1 + \beta_2 }/2 + i \lr{ -\alpha_2 + \beta_1 }/2 \\
\alpha_\txtL &= \lr{ \alpha_1 - \beta_2 }/2 + i \lr{ -\alpha_2 - \beta_1 }/2,
\end{aligned}
\end{equation*}
and use this to prove \cref{eqn:polarization:260}.
} % problem
