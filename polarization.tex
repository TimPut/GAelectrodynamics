%
% Copyright © 2017 Peeter Joot.  All Rights Reserved.
% Licenced as described in the file LICENSE under the root directory of this GIT repository.
%
%{
\subsection{Plane wave.}
\index{plane wave}
\index{polarization}
In a discussion of polarization, it is convenient to align the propagation direction along a fixed direction, usually the z-axis.
Setting \( \kcap = \Be_3, \beta z = \Bk \cdot \Bx \) in \cref{eqn:frequencydomainCore:200} the plane wave representation of the field is

\begin{dmath}\label{eqn:polarization:20}
\begin{aligned}
F(\Bx, \omega) &= (1 + \Be_3) \BE e^{-j \beta z} \\
F(\Bx, t) &= \Real\lr{ F(\Bx, \omega) e^{j \omega t} }.
\end{aligned}
\end{dmath}

Here the imaginary \( j \) has no intrinsic geometrical interpretation, \( \BE = \BE_\txtr + j \BE_\txti \) is allowed to have complex values, and all components of \( \BE \) is perpendicular to the propagation direction (\( \Be_\txtr \cdot \Be_3 = \BE_\txti \cdot \Be_3 = 0 \)).
\index{Jones vector}
A common representation of the electric field components is the Jones vector \( (c_1, c_2) \), which specifies complex coefficients for the electric field phasor in each of the possible directions

\begin{dmath}\label{eqn:polarization:120}
\BE = c_1 \Be_1 + c_2 \Be_2,
\end{dmath}

where \( c_1, c_2 \) are complex valued, say

\begin{dmath}\label{eqn:polarization:140}
\begin{aligned}
c_1 &= \alpha_1 + j \beta_1 \\
c_2 &= \alpha_2 + j \beta_2.
\end{aligned}
\end{dmath}

The tuple \( (c_1, c_2) \) is called the Jones vector, and compactly encodes the geometry of the pattern that the electric field traces out in the transverse plane.

\subsection{Circular polarization basis.}
\index{circular polarization}
\index{left circular polarization}
\index{right circular polarization}

The time domain field when written out explicitly in terms of the Jones vector components is

\begin{dmath}\label{eqn:polarization:160}
F(\Bx, t) = (1 + \Be_3) \lr{
\lr{ \alpha_1 \Be_1 + \alpha_2 \Be_2 } \cos\lr{ \omega t - \beta z }
-\lr{ \beta_1 \Be_1 + \beta_2 \Be_2 } \sin\lr{ \omega t - \beta z }
}.
\end{dmath}

Linear, circular, and elliptical polarization patterns can be obtained by selecting specific values of the Jones vector, or equivalently by selecting specific values for the \( \alpha_1, \alpha_2, \beta_1, \beta_2 \) constants in the time domain representation of \cref{eqn:polarization:160}.
In particular,
a field for which the
change in phase

\begin{dmath}\label{eqn:polarization:520}
\phi = \omega t - \beta z
\end{dmath}

results in the electric field tracing out a (clockwise,counterclockwise) circle

\begin{dmath}\label{eqn:polarization:180}
\begin{aligned}
\BE_\txtL &= \Abs{\BE} \lr{ \Be_1 \cos\phi + \Be_2 \sin\phi } = \Abs{\BE} \Be_1 \exp\lr{  \Be_{12} \phi } \\
\BE_\txtR &= \Abs{\BE} \lr{ \Be_1 \cos\phi - \Be_2 \sin\phi } = \Abs{\BE} \Be_1 \exp\lr{ -\Be_{12} \phi },
\end{aligned}
\end{dmath}

is referred to as having
(right,left) circular polarization.
There are different conventions for the polarization orientation, and here the IEEE antenna convention discussed in \citep{balanis1989advanced} are used.

%Fixme: flipped the orientation I was using, as I noticed after the fact that the figures in Balanis use an orientation with x-axis up and y-axis right!
% check that everything still looks correct.
% ( I didn't notice the inscribed X's in the polarization figures: https://en.wikibooks.org/wiki/Physics_Study_Guide/Vectors_and_scalars#How_to_draw_vectors_that_are_in_or_out_of_the_plane_of_the_page_.28or_board.29 )

The bivector exponential representation of the circularly polarized electric fields in \cref{eqn:polarization:180} indicates that it is possible to represent arbitrary field polarization in a GA form that does not require any real part operation, as follows

\begin{dmath}\label{eqn:polarization:200}
F = \lr{ 1 + \Be_3 } \Be_1 \lr{ \alpha_\txtL e^{i\phi} + \alpha_\txtR e^{-i\phi} },
\end{dmath}

where the constants \( \alpha_\txtL, \alpha_\txtR \) are both complex with respect to the unit bivector imaginary \( i = \Be_{12} \) representing the plane transverse to the propagation direction

\begin{dmath}\label{eqn:polarization:220}
\begin{aligned}
\alpha_\txtL &= \alpha_{\txtL 1} + i \alpha_{\txtL 2} \\
\alpha_\txtR &= \alpha_{\txtR 1} + i \alpha_{\txtR 2}.
\end{aligned}
\end{dmath}

If a transformation from scalar to bivector imaginary \( j \rightarrow \Be_{12} = i \) is made in the Jones vector component representation of \cref{eqn:polarization:140},
then
the coefficients \cref{eqn:polarization:220} of the circular polarization states are related to the Jones vector by (\cref{problem:polarization:1})

\begin{dmath}\label{eqn:polarization:260}
\begin{aligned}
\alpha_\txtL &= \inv{2}\lr{ c_1 - i c_2 } \\
\alpha_\txtR &= \inv{2}\lr{ c_1 + i c_2 }^\dagger.
\end{aligned}
\end{dmath}

\subsection{Linear polarization.}

Linear polarization is described by

\begin{dmath}\label{eqn:polarization:280}
\begin{aligned}
\alpha_\txtL &= \inv{2}\Abs{\BE} \Be_1 e^{i(\psi + \theta)} \\
\alpha_\txtR &= \inv{2}\Abs{\BE} \Be_1 e^{i(\psi - \theta)},
\end{aligned}
\end{dmath}

or
\begin{dmath}\label{eqn:polarization:300}
F = \lr{ 1 + \Be_3 } \Abs{\BE} \Be_1 e^{i\psi} \cos( \omega t - \beta z + \theta ),
\end{dmath}

where \( \theta \) is an arbitrary initial phase.  The electric field \( \BE \) traces out all the points along the line spanning the points between \( \pm \Be_1 e^{i\psi} \Abs{\BE} \), whereas the magnetic field \( \BH \) traces
out all the points along \( \pm \Be_2 e^{i\psi} \Abs{\BE}/\eta \) as illustrated (with \( \eta = 1 \)) in
\cref{fig:linearPolarization:linearPolarizationFig1}.
\imageFigure{../figures/GAelectrodynamics/linearPolarizationFig1}{Linear polarization.}{fig:linearPolarization:linearPolarizationFig1}{0.3}

\subsection{Other phase dependence and energy momentum.}

The linear polarization of \cref{eqn:polarization:300} can be generalized from sinosoidal functions of the phase angle \cref{eqn:polarization:520}, to arbitrary functions, as in

\begin{dmath}\label{eqn:polarization:540}
F = \lr{ 1 + \Be_3 } \Abs{\BE} \Be_1 e^{i\psi} f(\phi).
\end{dmath}

For example, \( f(\phi) = e^{i\phi} \) would result in a circularly polarized state, and
a Gaussian modulation could be added into the mix with \( f(\phi) = e^{i \phi - (\phi/\sigma)^2/2 } \).

If the phase dependence of \cref{eqn:polarization:540} is a scalar function, then
the energy momentum multivector for the field can be calculated simply

\begin{dmath}\label{eqn:polarization:560}
\calE + \frac{\BS}{v}
=
\inv{2} \epsilon
F F^\dagger
=
\inv{2} \epsilon
\lr{ 1 + \Be_3 } \Abs{\BE}^2 \Be_1 \cancel{e^{i\psi}} f^2(\phi)
\cancel{e^{-i\psi} }
\Be_1
\lr{ 1 + \Be_3 }
=
\inv{2} \epsilon
\lr{ 1 + \Be_3 } \Abs{\BE}^2 \cancel{\Be_1} f^2(\phi)
\cancel{\Be_1 }
\lr{ 1 + \Be_3 }
=
\epsilon \lr{ 1 + \Be_3 } \Abs{\BE}^2 f^2(\phi),
\end{dmath}

where the projective property \( \lr{ 1 + \Be_3 }^2 = 2 \lr{ 1 + \Be_3 } \) was used in the final simplification.
The energy, and Poynting vectors are
\begin{dmath}\label{eqn:polarization:580}
\begin{aligned}
\calE &= \epsilon \Abs{\BE}^2 f^2(\phi) \\
\BS &= \inv{\eta} \Be_3 \Abs{\BE}^2 f^2(\phi).
\end{aligned}
\end{dmath}
% v epsilon = sqrt( epsilon^2/ (epsilon mu) ) = 1/eta

More care for this calculation is required if the phase function \( f(\phi) \) is multivector valued, since it may not commute with the \( \Be_1 \) and \( e^{i\psi} \) factors of \( F \).

\subsection{Elliptical parameterization.}

An elliptical polarized electric field can be parameterized as
\begin{dmath}\label{eqn:ellipticalWaves:340}
\BE
=
E_a \Be_1 \cos\theta + E_b \Be_2 \sin\theta,
\end{dmath}

which corresponds to a Jones vector \( (E_a, -i E_b) \), or circular polarization coefficients with values

\begin{dmath}\label{eqn:polarization:400}
\begin{aligned}
\alpha_\txtL &= \inv{2}\lr{ E_a - E_b } \\
\alpha_\txtR &= \inv{2}\lr{ E_a + E_b }.
\end{aligned}
\end{dmath}

Therefore an elliptically polarized field can be represented as

\begin{dmath}\label{eqn:polarization:420}
F = \inv{2} (1 + \Be_3) \Be_1 \lr{ (E_a + E_b) e^{i\phi} + (E_a - E_b) e^{-i\phi} }.
\end{dmath}

An interesting variation of the elliptical polarization uses a hyperbolic parameterization.
If \( a, b \) are the semi-major/minor axes of the ellipse (i.e. \( a > b \)),
and \( \Ba = a \Be_1 e^{i\psi} \) is the vectoral representation of the semi-major axis (not necessarily placed along \( \Be_1 \)),
and \( e = \sqrt{1 - (b/a)^2} \) is the eccentricity of the ellipse,
then it can be shown (\citep{hestenes1999nfc})
that an elliptic parameterization can be written
in the compact form

\begin{dmath}\label{eqn:ellipticalWaves:360}
\Br(\phi)
=
e \Ba \cosh( \tanh^{-1}(b/a) + i \phi).
\end{dmath}

When the bivector imaginary \( i = \Be_{12} \) is used then
this parameterization is real and has only vector grades, so the electromagnetic field for a general elliptic wave has the form

\begin{dmath}\label{eqn:ellipticalWaves:380}
\begin{aligned}
F &= e E_a \lr{ 1 + \Be_3 } \Be_1 e^{ i \psi } \cosh\lr{ m + i \phi} \\
m &= \tanh^{-1}\lr{ E_b/E_a } \\
e &= \sqrt{1 - {(E_b/E_a)}^2 },
\end{aligned}
\end{dmath}

where \( E_a(E_b) \) are the magnitudes of the electric field components lying along the semi-major(minor) axes, and the propagation direction \( \Be_3 \) is normal to both the major and minor axis directions.
An elliptic electric field polarization is illustrated in \cref{fig:ellipticalPolarization:ellipticalPolarizationFig1}, where the vectors representing the major and minor axes are \( \BE_a = E_a \Be_1 e^{i\psi}, \BE_b = E_b \Be_1 e^{i\psi} \).
Observe that setting \( E_b = 0 \) results in the linearly polarized field of \cref{eqn:polarization:300}.
\imageFigure{../figures/GAelectrodynamics/ellipticalPolarizationFig1}{Electric field with elliptical polarization.}{fig:ellipticalPolarization:ellipticalPolarizationFig1}{0.3}

Following the procedure of \cref{eqn:polarization:560}, the energy-momentum of an elliptically polarized field is
\begin{dmath}\label{eqn:polarization:600}
\calE + \frac{\BS}{v}
=
\inv{2} \epsilon
F F^\dagger
=
\inv{2} \epsilon
e^2 E_a^2 \lr{ 1 + \Be_3 } \Be_1 \cancel{e^{ i \psi }} \cosh\lr{ m + i \phi}
\cosh\lr{ m - i \phi}
\cancel{e^{ -i \psi } }
\Be_1
\lr{ 1 + \Be_3 }
=
\inv{2} \epsilon
e^2 E_a^2 \lr{ 1 + \Be_3 }
\lr{ \cosh(2m) + \cos(2 \phi) }
=
\inv{2} \epsilon
\lr{ 1 + \Be_3 }
\lr{ E_b^2 + 2 \lr{
E_a^2 - E_b^2
 } \cos^2 \phi }
.
\end{dmath}

The simplification above made use of the identity \( (1 - (b/a)^2) \cosh(2 \Atanh(b/a)) = 1 + (b/a)^2 \).
%
% $Assumptions = b > 0 && b < 1 && a > 0 && a > b;
% (1 - (b/a)^2) Cosh[2 ArcTanh[b/a]] // FullSimplify

\subsection{Pseudoscalar imaginary.}

The multivector \( 1 + \Be_3 \) acts as a projector, consuming any factors of \( \Be_3 \)

\begin{dmath}\label{eqn:polarization:440}
(1 + \Be_3) \Be_3
=
\Be_3 + \Be_3^2
=
1 + \Be_3.
\end{dmath}

This property allows all the bivector imaginaries \( i = \Be_{12} = \Be_3 I \) in \cref{eqn:polarization:200} to be re-expressed in terms of the \R{3} pseudoscalar \( I = \Be_{123} \).  To illustrate this consider just the left circular polarized wave

\begin{dmath}\label{eqn:polarization:460}
F_\txtL
=
\lr{ 1 + \Be_3 } \Be_1 \alpha_\txtL e^{i\phi}
=
\lr{ 1 + \Be_3 } \Be_1 \alpha_\txtL \lr{ \cos\phi + \Be_3 I \sin\phi }
=
\lr{ 1 + \Be_3 } \Be_1 \alpha_\txtL \cos\phi
-\lr{ 1 + \Be_3 } \Be_3 \Be_1 \alpha_\txtL I \sin\phi
=
\lr{ 1 + \Be_3 } \Be_1 \alpha_\txtL e^{-I\phi}
=
\lr{ 1 + \Be_3 } \Be_1 \lr{ \alpha_{\txtL 1} + \Be_3 I\alpha_{\txtL 2}  } e^{-I\phi}
=
\lr{ 1 + \Be_3 } \Be_1 \lr{ \alpha_{\txtL 1} - I \alpha_{\txtL 2} } e^{-I\phi}.
\end{dmath}

This shows that the coefficients for the circular polarized states can be redefined using the pseudoscalar as an imaginary (in contrast to the bivector imaginary used in \cref{eqn:polarization:220})
\begin{dmath}\label{eqn:polarization:480}
\begin{aligned}
\alpha_\txtL' &= \alpha_{\txtL 1} - I \alpha_{\txtL 2} \\
\alpha_\txtR' &= \alpha_{\txtR 1} - I \alpha_{\txtR 2},
\end{aligned}
\end{dmath}

so that the plane wave is
\begin{dmath}\label{eqn:polarization:500}
F = \lr{ 1 + \Be_3 } \Be_1 \lr{ \alpha_\txtL' e^{-I\phi} + \alpha_\txtR' e^{I\phi} }.
\end{dmath}

Like \cref{eqn:polarization:200} this plane wave representation does not require taking any real parts.  The transverse plane in which the electric and magnetic fields lie is defined by the duality relation \( i = I \Be_3 \).

The energy momentum multivector for a wave described in terms of the pseudoscalar circular polarization states of \cref{eqn:polarization:500} is just

\begin{dmath}\label{eqn:polarization:n}
\calE + \frac{\BS}{v} =
\epsilon \lr{ 1 + \Be_3 } \lr{ \Abs{\alpha_\txtL'}^2 + \Abs{\alpha_\txtR'}^2 },
\end{dmath}

where the absolute value is computed using the reverse as the conjugation operation \( \Abs{z}^2 = z z^\dagger \).

\subsection{Problems.}

\makeproblem{Circular polarization coefficients relationship to the Jones vector.}{problem:polarization:1}{
By substituting \cref{eqn:polarization:220} into \cref{eqn:polarization:200}, and comparing to \cref{eqn:polarization:160},
show that the circular state coefficients have the following relationship to the Jones vector coordinates
\begin{equation*}
\begin{aligned}
\alpha_\txtL &= \lr{ \alpha_1 + \beta_2 }/2 + i \lr{ -\alpha_2 + \beta_1 }/2 \\
\alpha_\txtR &= \lr{ \alpha_1 - \beta_2 }/2 + i \lr{ -\alpha_2 - \beta_1 }/2,
\end{aligned}
\end{equation*}
and use this to prove \cref{eqn:polarization:260}.
} % problem

\makeproblem{Pseudoscalar Jones vector.}{problem:polarization:2}{
With the Jones vector defined in terms of the \R{3} pseudoscalar
\begin{equation*}
\begin{aligned}
c_1 &= \alpha_1 + I \beta_1 \\
c_2 &= \alpha_2 + I \beta_2,
\end{aligned}
\end{equation*}
calculate the values \( \alpha_\txtL', \alpha_\txtR' \) of \cref{eqn:polarization:480} in terms of this Jones vector.
} % problem
%}
