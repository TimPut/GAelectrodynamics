%
% Copyright © 2018 Peeter Joot.  All Rights Reserved.
% Licenced as described in the file LICENSE under the root directory of this GIT repository.
%
%{

\index{enclosed current}

Starting with
\cref{eqn:magnetostatics:380}, the
integral of the
charge density through an open surface can be calculated

\begin{dmath}\label{eqn:magnetostatics_enclosedCurrent:720}
\mu \int_S d^2 \Bx \cdot (I \BJ)
=
\int_S d^2 \Bx \cdot (\spacegrad \BB)
=
\int_S d^2 \Bx \cdot (\spacegrad \wedge \BB)
=
\int_{\partial S} d^1 \Bx \cdot \BB.
\end{dmath}

The specific orientation of the path around the surface for a two parameter expansion of Stokes' theorem (\cref{eqn:twoparameter:280}), was a clockwise loop.
Let
\( d^2 \Bx = I \ncap dA \), \( d^1 \Bx = d\Bl \), giving

\begin{dmath}\label{eqn:magnetostatics_enclosedCurrent:740}
\ointclockwise_{\partial S} d\Bl \cdot \BB
= \mu \int_S dA (\ncap I) \cdot (I \BJ)
= -\mu \int_S dA \ncap \cdot \BJ.
\end{dmath}

The circulation of the magnetic field around the boundary of a surface that the current density is flowing through is the enclosed surface.
This can be written as the dot product of an oriented area element with the dual of the current density (a bivector), or in the traditional form as a dot product of the current density with the outwards normal to the surface

%\begin{equation}\label{eqn:magnetostatics_enclosedCurrent:780}
\boxedEquation{eqn:magnetostatics:800}{
\begin{aligned}
\ointclockwise_{\partial S} d\Bl \cdot \BB &= \mu \int_S d^2 \Bx \cdot (I \BJ) \\
\ointctrclockwise_{\partial S} d\Bl \cdot \BB &= \mu \int_S dA \ncap \cdot \BJ.
\end{aligned}
}
%\end{equation}

%The bivector form of the dot product makes some geometrical sense, as we take a per unit area quantity (\(-I \BJ\)) with bivector grade and multiply it with a bivector area element, producing exactly the enclosed current flowing through the surface.

%}
