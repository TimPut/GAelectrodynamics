%\section{Two dimensions}

\subsection{\R{2} multiplication table and rotations}

In \R{2} many of the interesting vector products involve the unit bivector \( i = \Be_1 \Be_2 \), the \R{2} unit pseudoscalar.
It is not a coincidence that the symbol for the complex imaginary \( i \) is used for this bivector.
The square of this bivector

\begin{dmath}\label{eqn:SimpleProducts2:220}
   \lr{ \Be_1 \Be_2 }^2
   =
   \Be_1 \Be_2
   \Be_1 \Be_2
   =
   \Be_1 \lr{ \Be_2 \Be_1 } \Be_2
   =
   \Be_1 \lr{ -\Be_1 \Be_2 } \Be_2
   =
   -\lr{ \Be_1 \Be_1 }
   \lr{ \Be_2 \Be_2 }
   = -1,
\end{dmath}

like the complex imaginary, also squares to \( -1 \).

In complex algebra, multiplication by \( \pm i \) will rotate a complex number \( z = x + i y \) by \( \pm \pi/2 \) radians.
Multiplying an \R{2} vector by \( i = \Be_1 \Be_2 \) also produces \( \pi/2 \) rotations, however the rotation direction depends on whether left of right multiplication is used.
Computing the left and right products of \( i \) with the \R{2} basis vectors provides a
simple illustratation of these rotational effects

\begin{dmath}\label{eqn:SimpleProducts2:180}
\begin{aligned}
   \Be_1 i &= \Be_1 \lr{ \Be_1 \Be_2 } \\
           &= \lr{ \Be_1 \Be_1 } \Be_2 \\
           &= \Be_2 \\
   i \Be_1 &= \lr{ \Be_1 \Be_2 } \Be_1 \\
           &= \lr{ -\Be_2 \Be_1 } \Be_1 \\
           &= -\Be_2 \lr{ \Be_1 \Be_1 } \\
           &= -\Be_2 \\
   \Be_2 i &= \Be_2 \lr{ \Be_1 \Be_2 } \\
           &= \Be_2 \lr{ -\Be_2 \Be_1 } \\
           &= -\lr{ \Be_2 \Be_2 }\Be_1 \\
           &= -\Be_1 \\
   i \Be_2 &= \lr{ \Be_1 \Be_2 } \Be_2 \\
           &= \Be_1 \lr{ \Be_2 \Be_2 } \\
           &= \Be_1.
\end{aligned}
\end{dmath}

There are a number of noteworthy aspects of these calculations.

\begin{itemize}
\item The pseudoscalar \( i \) does not commute with either basis vector, but anticommutes with both, since \( i \Be_1 = - \Be_1 i \), and \( i \Be_2 = - \Be_2 i \).  By superposition \( i \) anticommutes with any vector in the plane.
\item The \( i \) products do rotate the basis vectors as claimed, which is
illustrated in \cref{fig:rotationOfe1:rotationOfe1Fig1}.
\item The products of
\cref{eqn:SimpleProducts2:220}
\cref{eqn:SimpleProducts2:180} can now be tabulated, constructing the geometric algebra multiplication table associated with the \R{2} standard basis.
\end{itemize}

\imageTwoFigures
{../figures/GAelectrodynamics/rotationOfe1Fig1}
{../figures/GAelectrodynamics/rotationOfe2Fig1}
{Multiplication by \( \Be_1 \Be_2 \).}{fig:rotationOfe1:rotationOfe1Fig1}{scale=0.5}

%\ref{tab:SimpleProducts2:10}.
%FIXME: how to reference a tcolorbox table?
% examples in http://ctan.mirrors.hoobly.com/macros/latex/contrib/tcolorbox/tcolorbox.pdf section 5.1
% requires setting up a counter variable like some of the others (theorem environments)

% various options for prettier than default table:
% https://tex.stackexchange.com/a/135421/15
% https://tex.stackexchange.com/a/298109/15
% https://tex.stackexchange.com/a/112359/15
%\captionedTable{2D Multiplication table.}{tab:SimpleProducts2:10}{
%\begin{tabular}{|l||l|l|l|l|}
%\hline
%        & \( 1 \) & \( \Be_1 \) & \( \Be_2 \) & \( \Be_1 \Be_2 \) \\ \hline
%\( 1 \) & \( 1 \) & \( \Be_1 \) & \( \Be_2 \) & \( \Be_1 \Be_2 \) \\ \hline
%\( \Be_1\) & \( \Be_1 \) & \( 1 \) & \( \Be_1 \Be_2 \) & \( \Be_2 \)\\ \hline
%\( \Be_2\) & \( \Be_2 \) & \( -\Be_1 \Be_2 \) & \( 1 \) & \( -\Be_1 \)\\ \hline
%\( \Be_1 \Be_2\) & \( \Be_1 \Be_2 \) & \( -\Be_2 \) & \( \Be_1 \) & \( -1 \) \\ \hline
%\end{tabular}
%}

%\label{tab:SimpleProducts2:10}
\begin{tcolorbox}[tab2,tabularx={X||Y|Y|Y|Y},title=2D Multiplication table.,boxrule=0.5pt]
        & \( 1 \) & \( \Be_1 \) & \( \Be_2 \) & \( \Be_1 \Be_2 \) \\ \hline
\( 1 \) & \( 1 \) & \( \Be_1 \) & \( \Be_2 \) & \( \Be_1 \Be_2 \) \\ \hline
\( \Be_1\) & \( \Be_1 \) & \( 1 \) & \( \Be_1 \Be_2 \) & \( \Be_2 \)\\ \hline
\( \Be_2\) & \( \Be_2 \) & \( -\Be_1 \Be_2 \) & \( 1 \) & \( -\Be_1 \)\\ \hline
\( \Be_1 \Be_2\) & \( \Be_1 \Be_2 \) & \( -\Be_2 \) & \( \Be_1 \) & \( -1 \) \\ \hline
\end{tcolorbox}

Given an arbitrary vector in a polar representation

\begin{dmath}\label{eqn:SimpleProducts2:280}
   \Bx = \rho \lr{ \Be_1 \cos\theta + \Be_2 \sin\theta },
\end{dmath}

left and right multiplication by the unit pseudoscalar gives

\begin{dmath}\label{eqn:SimpleProducts2:300}
\begin{aligned}
\Bx i
&= \Bx \Be_1 \Be_2 \\
&= \rho \lr{ \Be_1 \cos\theta + \Be_2 \sin\theta } \Be_1 \Be_2 \\
&= \rho \lr{ \Be_2 \cos\theta - \Be_1 \sin\theta } \\
i \Bx &= \Be_1 \Be_2 \Bx \\
&= \rho \Be_1 \Be_2 \lr{ \Be_1 \cos\theta + \Be_2 \sin\theta } \Be_1 \Be_2 \\
&= \rho \lr{ -\Be_2 \cos\theta + \Be_1 \sin\theta }.
\end{aligned}
\end{dmath}

It is left as a problem for the reader to show (using familiar methods, such as rotation matrices)
that \cref{eqn:SimpleProducts2:300} are the \( \pi/2 \) counterclockwise and clockwise rotations of \cref{eqn:SimpleProducts2:280} respectively.  These rotations are illustrated in \cref{fig:rotationOfV:rotationOfVFig1}.

\imageFigure{../figures/GAelectrodynamics/rotationOfVFig1}{\( \pi/2\) rotation using pseudoscalar multiplication.}{fig:rotationOfV:rotationOfVFig1}{0.3}

We can use Euler's formula with the \R{2} pseudoscalar representation of the complex imaginary

\begin{dmath}\label{eqn:SimpleProducts2:340}
e^{i \theta} = \cos\theta + i \sin\theta.
\end{dmath}

This can be justified by the fact that \( i = \Be_1 \Be_2 \) commutes with itself.

It is somewhat remarkable that \( \Be_1 \) can be directly factored from the
polar vector representation \cref{eqn:SimpleProducts2:280}, leaving a complex exponential.
This factorization relies on the trick mentioned earlier, utilizing a unit vector factorization of unity
\( 1 = \Be_1 \Be_1 \).  First factoring \( \Be_1 \) to the left,

\begin{dmath}\label{eqn:SimpleProducts2:940}
\Bx
=
\rho \lr{ \Be_1 \cos\theta + \Be_2 \sin\theta }
=
\rho \lr{ \Be_1 \cos\theta + (\Be_1 \Be_1) \Be_2 \sin\theta }
=
\rho \Be_1 \lr{ \cos\theta + \Be_1 \Be_2 \sin\theta }
=
\rho \Be_1 \lr{ \cos\theta + i \sin\theta }
=
\rho \Be_1 e^{i\theta},
\end{dmath}

a complex exponential (a multivector with grades 0,2) is left as a right factor.

Alternatively, by factoring \( \Be_1 \) to the right

\begin{dmath}\label{eqn:SimpleProducts2:960}
\Bx
=
\rho \lr{ \Be_1 \cos\theta + \Be_2 \sin\theta }
=
\rho \lr{ \Be_1 \cos\theta + \Be_2 (\Be_1 \Be_1) \sin\theta }
=
\rho \lr{ \cos\theta - \Be_1 \Be_2 \sin\theta } \Be_1
=
\rho \lr{ \cos\theta - i \sin\theta } \Be_1
=
\rho e^{-i\theta} \Be_1,
\end{dmath}

a complex exponential (with negative sign) is left factor.
The polar representation can therefore be expressed as either left or right complex exponential rotation of the vector \( \rho \Be_1 \).

\begin{equation}\label{eqn:SimpleProducts2:1120}
\rho \lr{ \Be_1 \cos\theta + \Be_2 \sin\theta }
= \rho e^{-i\theta} \Be_1 = \rho \Be_1 e^{i\theta}
\end{equation}

In general a positive right complex exponential multiplication (of any vector) rotates that vector counterclockwise (i.e. from \( \Be_1 \) to \( \Be_2 \)), whereas a positive left complex exponential multiplication would rotate that vector clockwise.  This is
illustrated in \cref{fig:rotationOfX:rotationOfXFig1}.
\imageFigure{../figures/GAelectrodynamics/rotationOfXFig1}{Rotation in a plane.}{fig:rotationOfX:rotationOfXFig1}{0.3}

\makedigression{Orientation}{
This is the first hint that a bivector can be thought of having a rotational sense, or orientation.  This is very similar to the orientation change that a vector undergoes by changing its sign.  As we think of vectors as oriented line segments, we will eventually come to think of bivectors as oriented plane segments, trivectors as oriented volume elements, and k-vectors as oriented hypervolumes.
}

\subsection{Vector product, dot product and wedge product.}

The product of two colinear vectors is a scalar, and the product of two normal vectors is a bivector.
To understand the form for a product of two unrestricted vectors, consider their product expressed as a coordinate expansion.  Let

\begin{dmath}\label{eqn:SimpleProducts2:1160}
\begin{aligned}
\Ba &= \sum_i a_i \Be_i \\
\Bb &= \sum_i b_i \Be_i,
\end{aligned}
\end{dmath}

The product of these vectors is

\begin{dmath}\label{eqn:SimpleProducts2:1360}
\Ba \Bb
=
\lr{ \sum_i a_i \Be_i } \lr{ \sum_j b_j \Be_j }
=
\sum_{ij} a_i b_j \Be_i \Be_j
=
\sum_{i = j} a_i b_j \Be_i \Be_j
+
\sum_{i \ne j} a_i b_j \Be_i \Be_j
\end{dmath}

The first sum over \( i = j \) is just the dot product since \( \Be_i \Be_i = 1 \), so the general product of two vectors is

\begin{dmath}\label{eqn:SimpleProducts2:1480}
\Ba \Bb
=
\Ba \cdot \Bb
+
\sum_{i \ne j} a_i b_j \Be_i \Be_j.
\end{dmath}

The product of two vectors is a multivector with a scalar (grade 0) component, and a bivector (grade 2) component.  This can be written symbolically as

\boxedEquation{eqn:SimpleProducts2:1380}{
\Ba \Bb = \gpgradezero{ \Ba \Bb } + \gpgradetwo{ \Ba \Bb }.
}

As a side effect of having performed this expansion, we see that it is possible to compute the dot product of two vectors by scalar grade selection

\boxedEquation{eqn:SimpleProducts2:1400}{
\Ba \cdot \Bb = \gpgradezero{ \Ba \Bb }.
}

This form of dot product motivates a more general definition of dot product for multivectors, which is

\makedefinition{Multivector dot product}{dfn:gradeselection:100}{
The dot (or inner) product of two multivectors

\begin{equation*}
\begin{aligned}
A &= \sum_{i = 0}^N A_i = \sum_{i = 0}^N \gpgrade{A}{i}, \\
B &= \sum_{i = 0}^N B_i = \sum_{i = 0}^N \gpgrade{B}{i},
\end{aligned}
\end{equation*}

is defined as
\begin{equation*}
A \cdot B \equiv
\sum_{i,j = 0}^N \gpgrade{ A_i B_j }{\Abs{i - j}}
\end{equation*}
} % definition

The bivector term of the vector product is called the wedge product, and written as

\boxedEquation{eqn:SimpleProducts2:1420}{
\Ba \wedge \Bb \equiv \gpgradetwo{ \Ba \Bb }.
}

Similar to the multivector dot product, the multivector wedge product that generalizes
\cref{eqn:SimpleProducts2:1420} is defined as

\makedefinition{Multivector wedge product.}{dfn:gradeselection:480}{
For the multivectors \( A, B \) defined in \cref{dfn:gradeselection:100}, the wedge (or outer) product is defined as

\begin{equation*}
A \wedge B
\equiv
\sum_{i,j = 0}^N \gpgrade{ A_i B_j }{i + j}.
\end{equation*}
} % definition

Instead of using grade selection, \cref{eqn:SimpleProducts2:1400} can now be expressed as a sum of dot and wedge products

\boxedEquation{eqn:SimpleProducts2:1440}{
\Ba \Bb = \Ba \cdot \Bb + \Ba \wedge \Bb.
}

This is a very important identity, and will have a number of applications.
It is premature to consider applications since
the properties and geometry of the wedge product have not been explored.

To start exploring that geometry let's consider the polar form of the vector, dot and wedge products for two vectors \( \Ba \) and \( \Bb \), with respective magnitudes \( a, b \).  
Let \( \ucap \) and \( \vcap \) be an orthonormal pair of vectors in the plane of \( \Ba \) and \( \Bb \), oriented in a positive rotational sense as illustrated in 
\cref{fig:Parallelogram:ParallelogramFig1}.
\imageFigure{../figures/GAelectrodynamics/ParallelogramFig1}{Two vectors in a plane.}{fig:Parallelogram:ParallelogramFig1}{0.3}
If \( i_{ab} = \ucap \vcap \) is the unit pseudoscalar for the plane containing these vectors, then the polar forms are

\begin{dmath}\label{eqn:SimpleProducts2:1660}
\begin{aligned}
\Ba &= a \ucap e^{ i_{ab} \theta_a } = a e^{ -i_{ab} \theta_a } \ucap \\
\Bb &= b \ucap e^{ i_{ab} \theta_b } = b e^{ -i_{ab} \theta_b } \ucap,
\end{aligned}
\end{dmath}

The vector product of these two vectors is

\begin{dmath}\label{eqn:SimpleProducts2:1680}
\Ba \Bb
=
\lr{ a e^{ -i_{ab} \theta_a } \ucap } \lr{ b \ucap e^{ i_{ab} \theta_b } }
=
a b
e^{ -i_{ab} \theta_a } ( \ucap \ucap ) e^{ i_{ab} \theta_b }
=
a b
e^{ i_{ab} (\theta_b - \theta_a)}.
\end{dmath}

The vector, dot and wedge products are therefore

\boxedEquation{eqn:SimpleProducts2:1700}{
\begin{aligned}
\Ba \Bb &= \Norm{\Ba} \Norm{\Bb} \exp\lr{ i_{ab} (\theta_b - \theta_a) } \\
\Ba \cdot \Bb &= \Norm{\Ba} \Norm{\Bb} \cos( \theta_b - \theta_a ) \\
\Ba \wedge \Bb &= i_{ab} \Norm{\Ba} \Norm{\Bb} \sin( \theta_b - \theta_a ).
\end{aligned}
}

The product of two vectors is the product of their magnitudes, multiplied by a ``unit'' complex exponential multivector with grades 0,2.

Since the cross product can be written as \( \Ba \cross \Bb = \ncap_{ab} \sin(\theta_b - \theta_a) \),
\cref{eqn:SimpleProducts2:1700} provides a strong hint that the
wedge and cross products are related.

One property of the wedge product follows by setting \( \Bb = \Ba \) in
\cref{eqn:SimpleProducts2:1440}, which gives

\begin{dmath}\label{eqn:SimpleProducts2:1500}
\Ba \Ba = \Ba \cdot \Ba + \Ba \wedge \Ba,
\end{dmath}

but since \( \Ba \Ba = \Ba \cdot \Ba \), we conclude that

\boxedEquation{eqn:SimpleProducts2:1520}{
\Ba \wedge \Ba = 0.
}

Like the cross product, the
wedge product of any colinear vectors is also zero, which should be clear from the polar form
of the wedge product in
\cref{eqn:SimpleProducts2:1700}.

Let's compare the coordinate expansion of the wedge product to that of the cross product, by
eliminating the redundant terms in the bivector term of \cref{eqn:SimpleProducts2:1480}

\begin{dmath}\label{eqn:SimpleProducts2:1460}
\Ba \wedge \Bb
=
\sum_{i \ne j} a_i b_j \Be_i \Be_j
=
\sum_{i < j} a_i b_j \Be_i \Be_j
+
\sum_{j < i} a_i b_j \Be_i \Be_j
=
\sum_{i < j} a_i b_j \Be_i \Be_j
+
\sum_{i < j} a_j b_i \Be_j \Be_i
=
\sum_{i < j} a_i b_j \Be_i \Be_j
+
\sum_{i < j} a_j b_i (-\Be_i \Be_j)
=
\sum_{i < j} (a_i b_j - a_j b_i) \Be_i \Be_j.
\end{dmath}

The scalar factors can be written as a determinants, yielding a tidy coordinate expansion of the wedge product of two vectors

\boxedEquation{eqn:SimpleProducts2:1320}{
\Ba \wedge \Bb
=
\sum_{i < j}
\begin{vmatrix}
a_i & a_j \\
b_i & b_j
\end{vmatrix}
\Be_i \Be_j.
}

This coordinate expansion can also be used to show that the wedge product of any colinear vectors is zero.
We also see that the wedge product is antisymmetric (exchanging the vectors toggles the sign), or

\boxedEquation{eqn:SimpleProducts2:1540}{
\Ba \wedge \Bb
=
-\Bb \wedge \Ba,
}

It's helpful to write out the coordinate expansion of
\cref{eqn:SimpleProducts2:1320} explicitly for \R{2} and \R{3}.
For \R{2} there is only one term in this sum

\begin{dmath}\label{eqn:SimpleProducts2:1720}
\Ba \wedge \Bb
=
\begin{vmatrix}
a_1 & a_2 \\
b_1 & b_2
\end{vmatrix}
\Be_1 \Be_2.
\end{dmath}

For \R{3} the wedge product has three terms, which can be combined using a cross product like determinant mnemonic

\begin{dmath}\label{eqn:SimpleProducts2:1740}
\Ba \wedge \Bb
=
\begin{vmatrix}
\Be_2 \Be_3 & \Be_3 \Be_1 & \Be_1 \Be_2 \\
a_1 & a_2 & a_3 \\
b_1 & b_2 & b_3 \\
\end{vmatrix}.
\end{dmath}

Let's summarize the wedge product properties and relations we have found so far, and compare those to the cross product

\begin{tcolorbox}[tab2,tabularx={X||Y|Y},title=Cross product and \R{3} wedge product comparison.,boxrule=0.5pt]
Property & Cross product & Wedge product
\\ \hline
Same vectors & \( \Ba \cross \Ba = 0 \) & \( \Ba \wedge \Ba = 0 \)
\\ \hline
Antisymmetry & \( \Bb \cross \Ba = -\Ba \cross \Bb \) & \( \Bb \wedge \Ba = -\Ba \wedge \Bb \)
\\ \hline
Determinant expansion
&
\(
\Ba \cross \Bb
=
\begin{vmatrix}
\Be_1 & \Be_2 & \Be_3 \\
a_1 & a_2 & a_3 \\
b_1 & b_2 & b_3 \\
\end{vmatrix}
\)
&
\(
\Ba \wedge \Bb
=
\begin{vmatrix}
\Be_2 \Be_3 & \Be_3 \Be_1 & \Be_1 \Be_2 \\
a_1 & a_2 & a_3 \\
b_1 & b_2 & b_3 \\
\end{vmatrix}
\)
\\ \hline
Polar form &
\( \ncap_{ab} \Norm{\Ba} \Norm{\Bb} \sin( \theta_b - \theta_a )  \) &
\( i_{ab} \Norm{\Ba} \Norm{\Bb} \sin( \theta_b - \theta_a )  \)
\\ \hline
\end{tcolorbox}

All the wedge properties except the determinant expansion above are valid in any dimension.
It is reasonable to guess that the \R{3} wedge product is related to the cross product by some constant multivector factor \( i_{ab} = A \ncap_{ab} \).  In coordinate form, this requires a simultaneous solution to

\begin{dmath}\label{eqn:SimpleProducts2:1580}
\begin{aligned}
\Be_2 \Be_3 &= A \Be_1 \\
\Be_3 \Be_1 &= A \Be_2 \\
\Be_1 \Be_2 &= A \Be_3.
\end{aligned}
\end{dmath}

Multiplying on the right by \( \Be_1, \Be_2, \Be_3 \) respectively, this factor seems to be

\begin{equation}\label{eqn:SimpleProducts2:1600}
A = \Be_2 \Be_3 \Be_1 = \Be_3 \Be_1 \Be_2 = \Be_1 \Be_2 \Be_3,
\end{equation}

which are all permutations of the \R{3} unit pseudoscalar \( I = \Be_1 \Be_2 \Be_3 \).
This indicates that the cyclic permutations of the \R{3} pseudoscalar must all be identical (\cref{problem:SimpleProducts2:permutationspseudoscalar}).

We now have a coordinate free relationship for the \R{3} wedge product and the cross product

\boxedEquation{eqn:SimpleProducts2:1620}{
\Ba \wedge \Bb = I ( \Ba \cross \Bb ),
}

and can also express the
\R{3} vector product as a multivector combination of the dot and cross products

\boxedEquation{eqn:SimpleProducts2:1640}{
\Ba \Bb = \Ba \cdot \Bb + I(\Ba \cross \Bb).
}

Like
\cref{eqn:SimpleProducts2:1440}, this is also a very important relationship.
In particular, this identity will be what we use to assemble all the separate scalar and vector Maxwell's equations into a single multivector equation.

\makeproblem{Wedge product of colinear vectors.}{problem:SimpleProducts2:wedgecolinear}{
Given \( \Bb = \alpha \Ba \), use
\cref{eqn:SimpleProducts2:1320} to show that the wedge product of any pair of colinear vectors is zero.
} % problem

\makeproblem{Wedge product antisymmetry.}{problem:SimpleProducts2:1560}{
Prove \cref{eqn:SimpleProducts2:1540} using \cref{eqn:SimpleProducts2:1320}.
} % problem

\makeproblem{Permutations of the \R{3} pseudoscalar}{problem:SimpleProducts2:permutationspseudoscalar}{
Show that each of the permutations of
\cref{eqn:SimpleProducts2:1600} are all equal.
} % problem

\subsection{The wedge product as an area element.}
\subsection{Symmetric and antisymmetric vector sums.}
\subsection{Duality}
\subsection{Cyclic permutation in scalar selection.}
\subsection{Dot products of other blades.}
\subsection{Orientation.}
\subsection{Projection and rejection.}
\subsection{Wedge of multiple vectors.}
\subsection{Reflection.}
\subsection{Linear systems}

\section{END MARKER}

The wedge product is related to the cross product, but can generalizes the cross product to two dimensions where there is no normal direction, and can generalize the cross product to greater than three dimensions, where any plane has too many normal directions.
The cross product is a (pseudo)vector that has a magnitude equal to the area of the parallelogram spanned by the crossed vectors, but is pointed normal to the plane of those vectors.
It will be possible to interpret the wedge product as the oriented (signed) area of that parallelogram itself without reference to any normal direction.
In the same sense that a vector is a representation of an oriented line segment, we will see that the wedge product of two vectors can be thought of as a representation of a oriented plane segment.
(cut)

Recall (\cref{problem:SimpleProducts2:areaofparallelogram}) that the absolute value of this determinant is precisely the area of the parallelogram formed by the vectors \( \Ba \) and \( \Bb \).  The wedge product, a bivector, can therefore be interpretted as an oriented (signed) area.  This is illustrated in \cref{fig:orientedParallelogram:orientedParallelogramFig1}.

\imageFigure{../figures/GAelectrodynamics/orientedParallelogramFig1}{Oriented area interpretation of \( \Bv_1 \wedge \Bv_2 \) and \( \Bv_2 \wedge \Bv_1 \).}{fig:orientedParallelogram:orientedParallelogramFig1}{0.3}

\makeproblem{Area of a parallelogram.}{problem:SimpleProducts2:areaofparallelogram}{
Show that the area of a parallelogram formed by the \R{2} vectors \( \Ba = (a_1, a_2) \) and \( \Bb = (b_1, b_2) \) is the absolute value of

\begin{equation*}
\begin{vmatrix}
a_1 & a_2 \\
b_1 & b_2
\end{vmatrix}.
\end{equation*}
} % problem

%\makeanswer{problem:SimpleProducts2:areaofparallelogram}{
%} % answer

\section{REWRITING MARKER END.}

Observe that in \R{2} the product of any basis vector with a pseudoscalar is normal to the original vector, which is also generally true for any vector in a 2D space,
(snip)
%\cref{fig:rotationOfe1:rotationOfe1Fig1}.
%\imageFigure{../figures/GAelectrodynamics/rotationOfe1Fig1}{CAPTION: rotationOfe1Fig1}{fig:rotationOfe1:rotationOfe1Fig1}{0.3}
%\cref{fig:rotationOfe2:rotationOfe2Fig1}.
%\imageFigure{../figures/GAelectrodynamics/rotationOfe2Fig1}{CAPTION: rotationOfe2Fig1}{fig:rotationOfe2:rotationOfe2Fig1}{0.3}
Such a multiplication induces a \( \pi/2 \) rotation, the direction of which depends on the orientation of pseudoscalar, and upon whether the multiplication is performed from the left or the right.

This unit bivector is seen to square to minus one like the imaginary in complex algebra.
The reader can confirm easily that this is generally true for any unit bivector \( \Be_i \Be_j, \, i \ne j \).
This is a very convienient fact, and allows ad-hoc construction of complex number like coordinate systems in any given planar subspace.

\section{Rework}
(cut)
Because the wedge product is completely antisymmetric, it must be true that

\begin{dmath}\label{eqn:SimpleProducts2:580}
\By \wedge \Bx = -\Bx \wedge \By,
\end{dmath}

so
\begin{dmath}\label{eqn:SimpleProducts2:600}
\By \Bx
= \By \cdot \Bx + \By \wedge \Bx
= \Bx \cdot \By - \Bx \wedge \By.
\end{dmath}

Taken together \cref{eqn:SimpleProducts2:540} and \cref{eqn:SimpleProducts2:600} allow for a construction of a coordinate free form of both the dot and wedge products

%\begin{dmath}\label{eqn:SimpleProducts2:620}
\boxedEquation{eqn:SimpleProducts2:620}{
\begin{aligned}
\Bx \cdot \By   &= \inv{2}\lr{ \Bx \By + \By \Bx } \\
\Bx \wedge \By  &= \inv{2}\lr{ \Bx \By - \By \Bx }.
\end{aligned}
}
%\end{dmath}

These highlight the symmetric and antisymmetric nature of the respective dot and wedge products.
Some authors will use \cref{eqn:SimpleProducts2:620} as the definitions of the dot and wedge products instead of defining them in terms of grade selection.
Grade selection is preferred here since it allows for a generalization of the wedge product to multiple vectors in higher degree spaces in a particularily simple way, and also allows for the generalization of the dot and wedge products with higher order geometric structures to be discussed.

\paragraph{Area}

It was previously claimed that the pseudoscalar \( \Be_1 \Be_2 \) could be interpretted as an oriented (signed) area.
Because \( \Be_1 \cdot \Be_2 = 0 \), this vector product is also equal to the wedge

\begin{dmath}\label{eqn:SimpleProducts2:640}
\Be_1 \Be_2 = \Be_1 \cdot \Be_2 +
\Be_1 \wedge \Be_2
=
\Be_1 \wedge \Be_2.
\end{dmath}

Recall that the area of the parallopiped spanned by two vectors in a two dimensional space is given by the absolute value of

\begin{dmath}\label{eqn:SimpleProducts2:660}
\begin{vmatrix}
   x_1 & x_2 \\
   y_1 & y_2
\end{vmatrix},
\end{dmath}

a factor that was also found in the coordinate expansion of the wedge product (\cref{eqn:SimpleProducts2:560}).
It is therefore natural to interpret the wedge product as an oriented area.
(cut)
In higher dimensonal spaces, the bivector factor not only encodes a sign for this area, but also its orientation in space.
The wedge product will be seen to encode that orientation without introducing a normal direction for the spanning plane, a nice feature in higher dimensional spaces where a single normal direction is ambiguous.

Because there are many possible pairs of generating vectors for any given bivector, any oriented area in a given plane with a specified area are all equally valid interpretations of a bivector.
This is illustrated in \cref{fig:orientedAreasVariety:orientedAreasVarietyFig1}.
\imageFigure{../figures/GAelectrodynamics/orientedAreasVarietyFig1}{Different shape representations of a given bivector.}{fig:orientedAreasVariety:orientedAreasVarietyFig1}{0.2}

\paragraph{Projection and rejection}

An immediate application of the vector product is the computation of the projective and rejective components of a vector with respect to another.
This follows by a unit vector factoring of unity \( \ucap^2 = 1 \),

\begin{dmath}\label{eqn:SimpleProducts2:680}
\Bx =
\Bx \ucap \ucap
=
\lr{ \Bx \ucap } \ucap
=
\lr{ \Bx \cdot \ucap + \Bx \wedge \ucap } \ucap
=
\lr{ \Bx \cdot \ucap } \ucap + \lr{ \Bx \wedge \ucap } \ucap.
\end{dmath}

The first term \( \lr{ \Bx \cdot \ucap } \ucap \) is the familiar projection along the direction \( \ucap \).
Since both sides of the equation must be a vector, this means that the multivector

\begin{dmath}\label{eqn:SimpleProducts2:700}
\lr{ \Bx \wedge \ucap } \ucap
=\Bx - \lr{ \Bx \cdot \ucap } \ucap,
\end{dmath}

is also a vector.
Subtracting the projection from the vector itself is the perpendicular projection, called the \boldTextAndIndex{rejection} in GA literature.
A few of the expected algebraic properties of this multivector rejection expression can be demonstrated.
Recall that the dot product can be computed using scalar grade selection, so

\begin{dmath}\label{eqn:SimpleProducts2:720}
\lr{ \lr{ \Bx \wedge \ucap } \ucap } \cdot \lr{ \ucap \lr{ \Bx \cdot \ucap } }
=
\gpgradezero{ \lr{ \Bx \wedge \ucap } \ucap \ucap \lr{ \Bx \cdot \ucap } }
=
\lr{ \Bx \cdot \ucap }
\gpgradezero{ \Bx \wedge \ucap }
= 0.
\end{dmath}

Here the scalar \( \ucap \ucap \lr{ \Bx \cdot \ucap } = \Bx \cdot \ucap \) was brought outside of the grade selection operator, leaving a scalar selection of a bivector, which is zero by definition.
This shows explicitly that the projection and the rejection are perpendicular.
The pythagorean property of these two vector components can also be checked.
Computing the squared length using \( \Norm{\By}^2 = \By \cdot \By = \By^2 \), the squared length of the projective component is

\begin{dmath}\label{eqn:SimpleProducts2:740}
\lr{ \lr{\Bx \cdot \ucap } \ucap }^2
=
\lr{\Bx \cdot \ucap }^2
=
(x_1 u_1 + x_2 u_2)^2
=
x_1^2 u_1^2 + x_2^2 u_2^2 + 2 x_1 x_2 u_1 u_2.
\end{dmath}

The squared length of the rejective component is
\begin{dmath}\label{eqn:SimpleProducts2:760}
\lr{ \lr{\Bx \wedge \ucap } \ucap }^2
=
-(\Bx \wedge \ucap) \ucap^2 (\Bx \wedge \ucap)
=
-
\lr{\begin{vmatrix}
   x_1 & x_2 \\
   u_1 & u_2
\end{vmatrix}}^2
(\Be_1 \Be_2)^2
=
x_1^2 u_2^2 + x_2^2 u_1^2 - 2 x_1 x_2 u_1 u_2.
\end{dmath}

Adding these together gives

\begin{dmath}\label{eqn:SimpleProducts2:780}
\lr{ \lr{\Bx \cdot \ucap } \ucap }^2 + \lr{ \lr{\Bx \wedge \ucap } \ucap }^2
=
x_1^2 u_1^2 + x_2^2 u_2^2
+x_1^2 u_2^2 + x_2^2 u_1^2
=
x_1^2 ( u_1^2 + u_2^2 )
+
x_2^2 ( u_1^2 + u_2^2 )
=
\Bx^2,
\end{dmath}

recovering the squared length of the vector as expected.
It is generally true in higher dimensions that the projection and rejection can be written as

\begin{dmath}\label{eqn:SimpleProducts2:800}
\begin{aligned}
\Proj_\ucap(\Bx) &= (\Bx \cdot \ucap) \ucap \\
\RejName_\ucap(\Bx) &= (\Bx \wedge \ucap) \ucap.
\end{aligned}
\end{dmath}

The Pythagorean aspect of this statement in higher degree spaces
will be demonstrated later in a coordinate free fashion after some additional identities have been derived.

The unit vector restriction defining the direction of projection and rejection can be relaxed in a compact fashion by introducing the vector \boldTextAndIndex{inverse}, which is always well defined and unique in a Euclidean space

\boxedEquation{eqn:SimpleProducts2:860}{
\inv{\Bu} \equiv \frac{\Bu}{\Bu^2}.
}

Now the projection and rejection onto the direction of \( \Bu \) are

\boxedEquation{eqn:SimpleProducts2:880}{
\begin{aligned}
\Proj_\Bu(\Bx) &= (\Bx \cdot \Bu) \inv{\Bu} \\
\RejName_\Bu(\Bx) &= (\Bx \wedge \Bu) \inv{\Bu}.
\end{aligned}
}
\index{projection}
\index{rejection}

An illustrative example of both projection and rejection is plotted in \cref{fig:projectionAndRejection:projectionAndRejectionFig1}.

\imageFigure{../figures/GAelectrodynamics/projectionAndRejectionFig1}{Projection and rejection illustrated.}{fig:projectionAndRejection:projectionAndRejectionFig1}{0.3}

\makedigression{Wedge and cross product relationships}{
Given that the wedge product has the ``cross product like'' properties \( \Bx \wedge \Bx = 0 \), and \( \Bx \wedge \By = -\By \wedge \Bx \),
and because the \R{3} expression of the rejection is

\begin{equation}\label{eqn:SimpleProducts2:820}
\RejName_\ucap(\Bx) = \Bx - (\Bx \cdot \ucap) \ucap = \ucap \cross (\Bx \cross \ucap),
\end{equation}

which is clearly similar to that of \cref{eqn:SimpleProducts2:800}, the observant reader may see from this expression that the wedge product in \R{3} seems to be related to the cross product in some fashion.
The precise nature of that relationship will be detailed later.
} % digression

\paragraph{Reflection}
\index{reflection}

Computation of the reflection of a vector through the origin, across the direction given by the vector \( \Bu \) (also passing through the origin) can now be expressed compactly.
That is

\begin{dmath}\label{eqn:SimpleProducts2:900}
\Bx'
= \lr{ \Bx \cdot \Bu }\Bu - \lr{ \Bx \wedge \Bu } \inv{\Bu }
= \lr{ \Bx \cdot \Bu - \Bx \wedge \Bu } \inv{\Bu }
= \inv{2} \lr{ \Bx \Bu + \Bu \Bx - \Bx \Bu + \Bu \Bx } \inv{\Bu },
\end{dmath}

or
\boxedEquation{eqn:SimpleProducts2:920}{
\Bx' = \Bu \Bx \inv{\Bu}.
}

An illustration of the geometry of reflection is provided in \cref{fig:reflection:reflectionFig1}.

\imageFigure{../figures/GAelectrodynamics/reflectionFig1}{Reflection}{fig:reflection:reflectionFig1}{0.3}

\paragraph{Solution of linear systems}

Various types of linear systems can be solved using the wedge product.
An illustrative example is that of the intersection of two lines as illustrated in \cref{fig:intersectionOfLines:intersectionOfLinesFig1}.

\imageFigure{../figures/GAelectrodynamics/intersectionOfLinesFig1}{Intersection of two lines.}{fig:intersectionOfLines:intersectionOfLinesFig1}{0.3}

In parametric form, the lines in this problem are

\begin{dmath}\label{eqn:SimpleProducts2:1000}
\begin{aligned}
\Br_1(s) &= \Ba_0 + s( \Ba_1 - \Ba_2 ) \\
\Br_2(t) &= \Bb_0 + t( \Bb_1 - \Bb_2 ),
\end{aligned}
\end{dmath}

so the solution, if it exists, is found at the point satisfying the equality

\begin{dmath}\label{eqn:SimpleProducts2:1020}
\Ba_0 + s( \Ba_1 - \Ba_2 ) = \Bb_0 + t( \Bb_1 - \Bb_2 ).
\end{dmath}

With
\begin{dmath}\label{eqn:SimpleProducts2:1040}
\begin{aligned}
\Bu_1 &= \Ba_1 - \Ba_2 \\
\Bu_2 &= \Bb_1 - \Bb_2 \\
\Bd &= \Ba_0 - \Bb_0,
\end{aligned}
\end{dmath}

so the desired equation to solve is

\begin{dmath}\label{eqn:SimpleProducts2:1060}
\Bd + s \Bu_1 = t \Bu_2.
\end{dmath}

In \R{3} this problem can solved using the cross product (\cref{problem:crossProductLinearIntersectionProblem:1}), however, this can be solved more generally as a
bivector equation by wedging both sides with either \( \Bu_1 \) or \( \Bu_2 \)

\begin{dmath}\label{eqn:SimpleProducts2:1080}
\begin{aligned}
\Bd \wedge \Bu_1 &= t \Bu_2 \wedge \Bu_1 \\
\Bd \wedge \Bu_2 + s \Bu_1 \wedge \Bu_2 &= 0,
\end{aligned}
\end{dmath}

In \R{2} these equations have a solution if \( \Bu_1 \wedge \Bu_2 \ne 0 \), and in \R{N} these have solutions if the bivectors on each sides of the equations describe the same plane.
Put another way, these have solutions when \( s \) and \( t \) are scalars and

\begin{dmath}\label{eqn:SimpleProducts2:1100}
\begin{aligned}
s &= \frac{\Bu_2 \wedge \Bd}{\Bu_1 \wedge \Bu_2} \\
t &= \frac{\Bu_1 \wedge \Bd}{\Bu_1 \wedge \Bu_2}.
\end{aligned}
\end{dmath}

For \R{2},
where the wedge product can be expressed as a (unit bivector scaled) determinant, this is precisely the Cramer's rule solution of the equivalent matrix equation.
