%\section{Two dimensions}

%\subsection{\R{2} multiplication table and rotations}
\subsection{2D rotations}

In \R{2} many of the interesting vector products involve the unit bivector \( i = \Be_1 \Be_2 \), the \R{2} unit pseudoscalar.
It is not a coincidence that the symbol for the complex imaginary \( i \) is used for this bivector.
The square of this bivector

\begin{dmath}\label{eqn:SimpleProducts2:220}
   \lr{ \Be_1 \Be_2 }^2
   =
   \Be_1 \Be_2
   \Be_1 \Be_2
   =
   \Be_1 \lr{ \Be_2 \Be_1 } \Be_2
   =
   \Be_1 \lr{ -\Be_1 \Be_2 } \Be_2
   =
   -\lr{ \Be_1 \Be_1 }
   \lr{ \Be_2 \Be_2 }
   = -1,
\end{dmath}

like the complex imaginary, also squares to \( -1 \).

In complex algebra, multiplication by \( \pm i \) will rotate a complex number \( z = x + i y \) by \( \pm \pi/2 \) radians.
Multiplying an \R{2} vector by \( i = \Be_1 \Be_2 \) also produces \( \pi/2 \) rotations, however the rotation direction depends on whether left of right multiplication is used.
Computing the left and right products of \( i \) with the \R{2} basis vectors provides a
simple illustratation of these rotational effects

\begin{dmath}\label{eqn:SimpleProducts2:180}
\begin{aligned}
   \Be_1 i &= \Be_1 \lr{ \Be_1 \Be_2 } \\
           &= \lr{ \Be_1 \Be_1 } \Be_2 \\
           &= \Be_2 \\
   i \Be_1 &= \lr{ \Be_1 \Be_2 } \Be_1 \\
           &= \lr{ -\Be_2 \Be_1 } \Be_1 \\
           &= -\Be_2 \lr{ \Be_1 \Be_1 } \\
           &= -\Be_2 \\
   \Be_2 i &= \Be_2 \lr{ \Be_1 \Be_2 } \\
           &= \Be_2 \lr{ -\Be_2 \Be_1 } \\
           &= -\lr{ \Be_2 \Be_2 }\Be_1 \\
           &= -\Be_1 \\
   i \Be_2 &= \lr{ \Be_1 \Be_2 } \Be_2 \\
           &= \Be_1 \lr{ \Be_2 \Be_2 } \\
           &= \Be_1.
\end{aligned}
\end{dmath}

There are a number of noteworthy aspects of these calculations.

\begin{itemize}
\item The pseudoscalar \( i \) does not commute with either basis vector, but anticommutes with both, since \( i \Be_1 = - \Be_1 i \), and \( i \Be_2 = - \Be_2 i \).  By superposition \( i \) anticommutes with any vector in the plane.
\item The \( i \) products do rotate the basis vectors as claimed, which is
illustrated in \cref{fig:rotationOfe1:rotationOfe1Fig1}.
\item The products of
\cref{eqn:SimpleProducts2:220}
\cref{eqn:SimpleProducts2:180} can now be tabulated, constructing the geometric algebra multiplication table associated with the \R{2} standard basis.
\end{itemize}

\imageTwoFigures
{../figures/GAelectrodynamics/rotationOfe1Fig1}
{../figures/GAelectrodynamics/rotationOfe2Fig1}
{Multiplication by \( \Be_1 \Be_2 \).}{fig:rotationOfe1:rotationOfe1Fig1}{scale=0.5}

%\ref{tab:SimpleProducts2:10}.
%FIXME: how to reference a tcolorbox table?
% examples in http://ctan.mirrors.hoobly.com/macros/latex/contrib/tcolorbox/tcolorbox.pdf section 5.1
% requires setting up a counter variable like some of the others (theorem environments)

% various options for prettier than default table:
% https://tex.stackexchange.com/a/135421/15
% https://tex.stackexchange.com/a/298109/15
% https://tex.stackexchange.com/a/112359/15
%\captionedTable{2D Multiplication table.}{tab:SimpleProducts2:10}{
%\begin{tabular}{|l||l|l|l|l|}
%\hline
%        & \( 1 \) & \( \Be_1 \) & \( \Be_2 \) & \( \Be_1 \Be_2 \) \\ \hline
%\( 1 \) & \( 1 \) & \( \Be_1 \) & \( \Be_2 \) & \( \Be_1 \Be_2 \) \\ \hline
%\( \Be_1\) & \( \Be_1 \) & \( 1 \) & \( \Be_1 \Be_2 \) & \( \Be_2 \)\\ \hline
%\( \Be_2\) & \( \Be_2 \) & \( -\Be_1 \Be_2 \) & \( 1 \) & \( -\Be_1 \)\\ \hline
%\( \Be_1 \Be_2\) & \( \Be_1 \Be_2 \) & \( -\Be_2 \) & \( \Be_1 \) & \( -1 \) \\ \hline
%\end{tabular}
%}

%\label{tab:SimpleProducts2:10}
\begin{tcolorbox}[tab2,tabularx={X||Y|Y|Y|Y},title=2D Multiplication table.,boxrule=0.5pt]
        & \( 1 \) & \( \Be_1 \) & \( \Be_2 \) & \( \Be_1 \Be_2 \) \\ \hline
\( 1 \) & \( 1 \) & \( \Be_1 \) & \( \Be_2 \) & \( \Be_1 \Be_2 \) \\ \hline
\( \Be_1\) & \( \Be_1 \) & \( 1 \) & \( \Be_1 \Be_2 \) & \( \Be_2 \)\\ \hline
\( \Be_2\) & \( \Be_2 \) & \( -\Be_1 \Be_2 \) & \( 1 \) & \( -\Be_1 \)\\ \hline
\( \Be_1 \Be_2\) & \( \Be_1 \Be_2 \) & \( -\Be_2 \) & \( \Be_1 \) & \( -1 \) \\ \hline
\end{tcolorbox}

Given an arbitrary vector in a polar representation

\begin{dmath}\label{eqn:SimpleProducts2:280}
   \Bx = \rho \lr{ \Be_1 \cos\theta + \Be_2 \sin\theta },
\end{dmath}

left and right multiplication by the unit pseudoscalar gives

\begin{dmath}\label{eqn:SimpleProducts2:300}
\begin{aligned}
\Bx i
&= \Bx \Be_1 \Be_2 \\
&= \rho \lr{ \Be_1 \cos\theta + \Be_2 \sin\theta } \Be_1 \Be_2 \\
&= \rho \lr{ \Be_2 \cos\theta - \Be_1 \sin\theta } \\
i \Bx &= \Be_1 \Be_2 \Bx \\
&= \rho \Be_1 \Be_2 \lr{ \Be_1 \cos\theta + \Be_2 \sin\theta } \Be_1 \Be_2 \\
&= \rho \lr{ -\Be_2 \cos\theta + \Be_1 \sin\theta }.
\end{aligned}
\end{dmath}

It is left as a problem for the reader to show (using familiar methods, such as rotation matrices)
that \cref{eqn:SimpleProducts2:300} are the \( \pi/2 \) counterclockwise and clockwise rotations of \cref{eqn:SimpleProducts2:280} respectively.  These rotations are illustrated in \cref{fig:rotationOfV:rotationOfVFig1}.

\imageFigure{../figures/GAelectrodynamics/rotationOfVFig1}{\( \pi/2\) rotation using pseudoscalar multiplication.}{fig:rotationOfV:rotationOfVFig1}{0.3}

We can use Euler's formula with the \R{2} pseudoscalar representation of the complex imaginary

\begin{dmath}\label{eqn:SimpleProducts2:340}
e^{i \theta} = \cos\theta + i \sin\theta.
\end{dmath}

This can be justified by the fact that \( i = \Be_1 \Be_2 \) commutes with itself.

It is somewhat remarkable that \( \Be_1 \) can be directly factored from the
polar vector representation \cref{eqn:SimpleProducts2:280}, leaving a complex exponential.
This factorization relies on the trick mentioned earlier, utilizing a unit vector factorization of unity
\( 1 = \Be_1 \Be_1 \).  First factoring \( \Be_1 \) to the left,

\begin{dmath}\label{eqn:SimpleProducts2:940}
\Bx
=
\rho \lr{ \Be_1 \cos\theta + \Be_2 \sin\theta }
=
\rho \lr{ \Be_1 \cos\theta + (\Be_1 \Be_1) \Be_2 \sin\theta }
=
\rho \Be_1 \lr{ \cos\theta + \Be_1 \Be_2 \sin\theta }
=
\rho \Be_1 \lr{ \cos\theta + i \sin\theta }
=
\rho \Be_1 e^{i\theta},
\end{dmath}

a complex exponential (a multivector with grades 0,2) is left as a right factor.

Alternatively, by factoring \( \Be_1 \) to the right

\begin{dmath}\label{eqn:SimpleProducts2:960}
\Bx
=
\rho \lr{ \Be_1 \cos\theta + \Be_2 \sin\theta }
=
\rho \lr{ \Be_1 \cos\theta + \Be_2 (\Be_1 \Be_1) \sin\theta }
=
\rho \lr{ \cos\theta - \Be_1 \Be_2 \sin\theta } \Be_1
=
\rho \lr{ \cos\theta - i \sin\theta } \Be_1
=
\rho e^{-i\theta} \Be_1,
\end{dmath}

a complex exponential (with negative sign) is left factor.
The polar representation can therefore be expressed as either left or right complex exponential rotation of the vector \( \rho \Be_1 \).

\begin{equation}\label{eqn:SimpleProducts2:1120}
\rho \lr{ \Be_1 \cos\theta + \Be_2 \sin\theta }
= \rho e^{-i\theta} \Be_1 = \rho \Be_1 e^{i\theta}
\end{equation}

In general a positive right complex exponential multiplication (of any vector) rotates that vector counterclockwise (i.e. from \( \Be_1 \) to \( \Be_2 \)), whereas a positive left complex exponential multiplication would rotate that vector clockwise.  This is
illustrated in \cref{fig:rotationOfX:rotationOfXFig1}.
\imageFigure{../figures/GAelectrodynamics/rotationOfXFig1}{Rotation in a plane.}{fig:rotationOfX:rotationOfXFig1}{0.3}

\makedigression{Orientation}{
This is the first hint that a bivector can be thought of having a rotational sense, or orientation.  This is very similar to the orientation change that a vector undergoes by changing its sign.  As we think of vectors as oriented line segments, we will eventually come to think of bivectors as oriented plane segments, trivectors as oriented volume elements, and k-vectors as oriented hypervolumes.
}

\subsection{Vector product, dot product and wedge product.}

The product of two colinear vectors is a scalar, and the product of two normal vectors is a bivector.
To understand the form for a product of two unrestricted vectors, consider their product expressed as a coordinate expansion.  Let

\begin{dmath}\label{eqn:SimpleProducts2:1160}
\begin{aligned}
\Ba &= \sum_i a_i \Be_i \\
\Bb &= \sum_i b_i \Be_i,
\end{aligned}
\end{dmath}

The product of these vectors is

\begin{dmath}\label{eqn:SimpleProducts2:1360}
\Ba \Bb
=
\lr{ \sum_i a_i \Be_i } \lr{ \sum_j b_j \Be_j }
=
\sum_{ij} a_i b_j \Be_i \Be_j
=
\sum_{i = j} a_i b_j \Be_i \Be_j
+
\sum_{i \ne j} a_i b_j \Be_i \Be_j
\end{dmath}

The first sum over \( i = j \) is just the dot product since \( \Be_i \Be_i = 1 \), so the general product of two vectors is

\begin{dmath}\label{eqn:SimpleProducts2:1480}
\Ba \Bb
=
\Ba \cdot \Bb
+
\sum_{i \ne j} a_i b_j \Be_i \Be_j.
\end{dmath}

The product of two vectors is a multivector with a scalar (grade 0) component, and a bivector (grade 2) component.  This can be written symbolically as

\boxedEquation{eqn:SimpleProducts2:1380}{
\Ba \Bb = \gpgradezero{ \Ba \Bb } + \gpgradetwo{ \Ba \Bb }.
}

As a side effect of having performed this expansion, we see that it is possible to compute the dot product of two vectors by scalar grade selection

\boxedEquation{eqn:SimpleProducts2:1400}{
\Ba \cdot \Bb = \gpgradezero{ \Ba \Bb }.
}

This form of dot product motivates a more general definition of dot product for multivectors, which is

\makedefinition{Multivector dot product}{dfn:gradeselection:100}{
The dot (or inner) product of two multivectors

\begin{equation*}
\begin{aligned}
A &= \sum_{i = 0}^N A_i = \sum_{i = 0}^N \gpgrade{A}{i}, \\
B &= \sum_{i = 0}^N B_i = \sum_{i = 0}^N \gpgrade{B}{i},
\end{aligned}
\end{equation*}

is defined as
\begin{equation*}
A \cdot B \equiv
\sum_{i,j = 0}^N \gpgrade{ A_i B_j }{\Abs{i - j}}
\end{equation*}
} % definition

The bivector term of the vector product is called the wedge product, and written as

\boxedEquation{eqn:SimpleProducts2:1420}{
\Ba \wedge \Bb \equiv \gpgradetwo{ \Ba \Bb }.
}

Similar to the multivector dot product, the multivector wedge product that generalizes
\cref{eqn:SimpleProducts2:1420} is defined as

\makedefinition{Multivector wedge product.}{dfn:gradeselection:480}{
For the multivectors \( A, B \) defined in \cref{dfn:gradeselection:100}, the wedge (or outer) product is defined as

\begin{equation*}
A \wedge B
\equiv
\sum_{i,j = 0}^N \gpgrade{ A_i B_j }{i + j}.
\end{equation*}
} % definition

Instead of using grade selection, \cref{eqn:SimpleProducts2:1400} can now be expressed as a sum of dot and wedge products

\boxedEquation{eqn:SimpleProducts2:1440}{
\Ba \Bb = \Ba \cdot \Bb + \Ba \wedge \Bb.
}

This is a very important identity, and will have a number of applications.
It is premature to consider applications since
the properties and geometry of the wedge product have not been explored.

To start exploring that geometry let's consider the polar form of the vector, dot and wedge products for two vectors \( \Ba \) and \( \Bb \), with respective magnitudes \( a, b \).
Let \( \ucap \) and \( \vcap \) be an orthonormal pair of vectors in the plane of \( \Ba \) and \( \Bb \), oriented in a positive rotational sense as illustrated in
\cref{fig:Parallelogram:ParallelogramFig1}.
\imageFigure{../figures/GAelectrodynamics/ParallelogramFig1}{Two vectors in a plane.}{fig:Parallelogram:ParallelogramFig1}{0.3}
If \( i_{ab} = \ucap \vcap \) is the unit pseudoscalar for the plane containing these vectors, then the polar forms are

\begin{dmath}\label{eqn:SimpleProducts2:1660}
\begin{aligned}
\Ba &= a \ucap e^{ i_{ab} \theta_a } = a e^{ -i_{ab} \theta_a } \ucap \\
\Bb &= b \ucap e^{ i_{ab} \theta_b } = b e^{ -i_{ab} \theta_b } \ucap,
\end{aligned}
\end{dmath}

The vector product of these two vectors is

\begin{dmath}\label{eqn:SimpleProducts2:1680}
\Ba \Bb
=
\lr{ a e^{ -i_{ab} \theta_a } \ucap } \lr{ b \ucap e^{ i_{ab} \theta_b } }
=
a b
e^{ -i_{ab} \theta_a } ( \ucap \ucap ) e^{ i_{ab} \theta_b }
=
a b
e^{ i_{ab} (\theta_b - \theta_a)}.
\end{dmath}

The vector, dot and wedge products are therefore

\boxedEquation{eqn:SimpleProducts2:1700}{
\begin{aligned}
\Ba \Bb &= \Norm{\Ba} \Norm{\Bb} \exp\lr{ i_{ab} (\theta_b - \theta_a) } \\
\Ba \cdot \Bb &= \Norm{\Ba} \Norm{\Bb} \cos( \theta_b - \theta_a ) \\
\Ba \wedge \Bb &= i_{ab} \Norm{\Ba} \Norm{\Bb} \sin( \theta_b - \theta_a ).
\end{aligned}
}

The product of two vectors is the product of their magnitudes, multiplied by a ``unit'' complex exponential multivector with grades 0,2.

Since the cross product can be written as \( \Ba \cross \Bb = \ncap_{ab} \sin(\theta_b - \theta_a) \),
\cref{eqn:SimpleProducts2:1700} provides a strong hint that the
wedge and cross products are related.

One property of the wedge product follows by setting \( \Bb = \Ba \) in
\cref{eqn:SimpleProducts2:1440}, which gives

\begin{dmath}\label{eqn:SimpleProducts2:1500}
\Ba \Ba = \Ba \cdot \Ba + \Ba \wedge \Ba,
\end{dmath}

but since \( \Ba \Ba = \Ba \cdot \Ba \), we conclude that

\boxedEquation{eqn:SimpleProducts2:1520}{
\Ba \wedge \Ba = 0.
}

Like the cross product, the
wedge product of any colinear vectors is also zero, which should be clear from the polar form
of the wedge product in
\cref{eqn:SimpleProducts2:1700}.

Let's compare the coordinate expansion of the wedge product to that of the cross product, by
eliminating the redundant terms in the bivector term of \cref{eqn:SimpleProducts2:1480}

\begin{dmath}\label{eqn:SimpleProducts2:1460}
\Ba \wedge \Bb
=
\sum_{i \ne j} a_i b_j \Be_i \Be_j
=
\sum_{i < j} a_i b_j \Be_i \Be_j
+
\sum_{j < i} a_i b_j \Be_i \Be_j
=
\sum_{i < j} a_i b_j \Be_i \Be_j
+
\sum_{i < j} a_j b_i \Be_j \Be_i
=
\sum_{i < j} a_i b_j \Be_i \Be_j
+
\sum_{i < j} a_j b_i (-\Be_i \Be_j)
=
\sum_{i < j} (a_i b_j - a_j b_i) \Be_i \Be_j.
\end{dmath}

The scalar factors can be written as a determinants, yielding a tidy coordinate expansion of the wedge product of two vectors

\boxedEquation{eqn:SimpleProducts2:1320}{
\Ba \wedge \Bb
=
\sum_{i < j}
\begin{vmatrix}
a_i & a_j \\
b_i & b_j
\end{vmatrix}
\Be_i \Be_j.
}

This coordinate expansion can also be used to show that the wedge product of any colinear vectors is zero.
We also see that the wedge product is antisymmetric (exchanging the vectors toggles the sign), or

\boxedEquation{eqn:SimpleProducts2:1540}{
\Ba \wedge \Bb
=
-\Bb \wedge \Ba,
}

It's helpful to write out the coordinate expansion of
\cref{eqn:SimpleProducts2:1320} explicitly for \R{2} and \R{3}.
For \R{2} there is only one term in this sum

\begin{dmath}\label{eqn:SimpleProducts2:1720}
\Ba \wedge \Bb
=
\begin{vmatrix}
a_1 & a_2 \\
b_1 & b_2
\end{vmatrix}
\Be_1 \Be_2.
\end{dmath}

For \R{3} the wedge product has three terms, which can be combined using a cross product like determinant mnemonic

\begin{dmath}\label{eqn:SimpleProducts2:1740}
\Ba \wedge \Bb
=
\begin{vmatrix}
\Be_2 \Be_3 & \Be_3 \Be_1 & \Be_1 \Be_2 \\
a_1 & a_2 & a_3 \\
b_1 & b_2 & b_3 \\
\end{vmatrix}.
\end{dmath}

Let's summarize the wedge product properties and relations we have found so far, and compare those to the cross product

\begin{tcolorbox}[tab2,tabularx={X||Y|Y},title=Cross product and \R{3} wedge product comparison.,boxrule=0.5pt]
Property & Cross product & Wedge product
\\ \hline
Same vectors & \( \Ba \cross \Ba = 0 \) & \( \Ba \wedge \Ba = 0 \)
\\ \hline
Antisymmetry & \( \Bb \cross \Ba = -\Ba \cross \Bb \) & \( \Bb \wedge \Ba = -\Ba \wedge \Bb \)
\\ \hline
Determinant expansion
&
\(
\Ba \cross \Bb
=
\begin{vmatrix}
\Be_1 & \Be_2 & \Be_3 \\
a_1 & a_2 & a_3 \\
b_1 & b_2 & b_3 \\
\end{vmatrix}
\)
&
\(
\Ba \wedge \Bb
=
\begin{vmatrix}
\Be_2 \Be_3 & \Be_3 \Be_1 & \Be_1 \Be_2 \\
a_1 & a_2 & a_3 \\
b_1 & b_2 & b_3 \\
\end{vmatrix}
\)
\\ \hline
Polar form &
\( \ncap_{ab} \Norm{\Ba} \Norm{\Bb} \sin( \theta_b - \theta_a )  \) &
\( i_{ab} \Norm{\Ba} \Norm{\Bb} \sin( \theta_b - \theta_a )  \)
\\ \hline
\end{tcolorbox}

All the wedge properties except the determinant expansion above are valid in any dimension.
It is reasonable to guess that the \R{3} wedge product is related to the cross product by some constant multivector factor \( i_{ab} = A \ncap_{ab} \).  In coordinate form, this requires a simultaneous solution to

\begin{dmath}\label{eqn:SimpleProducts2:1580}
\begin{aligned}
\Be_2 \Be_3 &= A \Be_1 \\
\Be_3 \Be_1 &= A \Be_2 \\
\Be_1 \Be_2 &= A \Be_3.
\end{aligned}
\end{dmath}

Multiplying on the right by \( \Be_1, \Be_2, \Be_3 \) respectively, this factor seems to be

\begin{equation}\label{eqn:SimpleProducts2:1600}
A = \Be_2 \Be_3 \Be_1 = \Be_3 \Be_1 \Be_2 = \Be_1 \Be_2 \Be_3,
\end{equation}

which are all permutations of the \R{3} unit pseudoscalar \( I = \Be_1 \Be_2 \Be_3 \).
This indicates that the cyclic permutations of the \R{3} pseudoscalar must all be identical (\cref{problem:SimpleProducts2:permutationspseudoscalar}).

We now have a coordinate free relationship for the \R{3} wedge product and the cross product

\boxedEquation{eqn:SimpleProducts2:1620}{
\Ba \wedge \Bb = I ( \Ba \cross \Bb ),
}

and can also express the
\R{3} vector product as a multivector combination of the dot and cross products

\boxedEquation{eqn:SimpleProducts2:1640}{
\Ba \Bb = \Ba \cdot \Bb + I(\Ba \cross \Bb).
}

Like
\cref{eqn:SimpleProducts2:1440}, this is also a very important relationship.
In particular, this identity will be what we use to assemble all the separate scalar and vector Maxwell's equations into a single multivector equation.

\makeproblem{Wedge product of colinear vectors.}{problem:SimpleProducts2:wedgecolinear}{
Given \( \Bb = \alpha \Ba \), use
\cref{eqn:SimpleProducts2:1320} to show that the wedge product of any pair of colinear vectors is zero.
} % problem

\makeproblem{Wedge product antisymmetry.}{problem:SimpleProducts2:1560}{
Prove \cref{eqn:SimpleProducts2:1540} using \cref{eqn:SimpleProducts2:1320}.
} % problem

\makeproblem{Permutations of the \R{3} pseudoscalar}{problem:SimpleProducts2:permutationspseudoscalar}{
Show that each of the permutations of
\cref{eqn:SimpleProducts2:1600} are all equal.
} % problem

\subsection{Imaginary nature of the \R{3} pseudoscalar.}

%
% Copyright © 2016 Peeter Joot.  All Rights Reserved.
% Licenced as described in the file LICENSE under the root directory of this GIT repository.
%

\index{complex imaginary}
\index{pseudoscalar}

Using the reversion operation it is simple to show that the \R{3} pseudoscalar
behaves like a complex imaginary with \( I^2 = -1 \)
\begin{dmath}\label{eqn:R3PseudoscalarSquare:3310}
I^2
=
I (-I^\dagger)
=
-
(\Be_1 \Be_2 \Be_3)(\Be_3 \Be_2 \Be_1)
=
-
\Be_1 \Be_2 \Be_2 \Be_1
=
-
\Be_1 \Be_1
=
-1.
\end{dmath}


\subsection{Projection and rejection}


(cut)
The pythagorean property of these two vector components can also be checked.
Computing the squared length using \( \Norm{\By}^2 = \By \cdot \By = \By^2 \), the squared length of the projective component is

\begin{dmath}\label{eqn:SimpleProducts2:740}
\lr{ \lr{\Bx \cdot \ucap } \ucap }^2
=
\lr{\Bx \cdot \ucap }^2
=
(x_1 u_1 + x_2 u_2)^2
=
x_1^2 u_1^2 + x_2^2 u_2^2 + 2 x_1 x_2 u_1 u_2.
\end{dmath}

The squared length of the rejective component is
\begin{dmath}\label{eqn:SimpleProducts2:760}
\lr{ \lr{\Bx \wedge \ucap } \ucap }^2
=
-(\Bx \wedge \ucap) \ucap^2 (\Bx \wedge \ucap)
=
-
\lr{\begin{vmatrix}
   x_1 & x_2 \\
   u_1 & u_2
\end{vmatrix}}^2
(\Be_1 \Be_2)^2
=
x_1^2 u_2^2 + x_2^2 u_1^2 - 2 x_1 x_2 u_1 u_2.
\end{dmath}

Adding these together gives

\begin{dmath}\label{eqn:SimpleProducts2:780}
\lr{ \lr{\Bx \cdot \ucap } \ucap }^2 + \lr{ \lr{\Bx \wedge \ucap } \ucap }^2
=
x_1^2 u_1^2 + x_2^2 u_2^2
+x_1^2 u_2^2 + x_2^2 u_1^2
=
x_1^2 ( u_1^2 + u_2^2 )
+
x_2^2 ( u_1^2 + u_2^2 )
=
\Bx^2,
\end{dmath}

recovering the squared length of the vector as expected.
It is generally true in higher dimensions that the projection and rejection can be written as

\begin{dmath}\label{eqn:SimpleProducts2:800}
\begin{aligned}
\Proj_\ucap(\Bx) &= (\Bx \cdot \ucap) \ucap \\
\RejName_\ucap(\Bx) &= (\Bx \wedge \ucap) \ucap.
\end{aligned}
\end{dmath}

The Pythagorean aspect of this statement in higher degree spaces
will be demonstrated later in a coordinate free fashion after some additional identities have been derived.

The unit vector restriction defining the direction of projection and rejection can be relaxed in a compact fashion by introducing the vector \boldTextAndIndex{inverse}, which is always well defined and unique in a Euclidean space

\boxedEquation{eqn:SimpleProducts2:860}{
\inv{\Bu} \equiv \frac{\Bu}{\Bu^2}.
}

Now the projection and rejection onto the direction of \( \Bu \) are

\boxedEquation{eqn:SimpleProducts2:880}{
\begin{aligned}
\Proj_\Bu(\Bx) &= (\Bx \cdot \Bu) \inv{\Bu} \\
\RejName_\Bu(\Bx) &= (\Bx \wedge \Bu) \inv{\Bu}.
\end{aligned}
}

%\makelemma{\R{3} pseudoscalar commutation.}{dfn:projectionAndRejection:r3pcommutation}{
%The \R{3} pseudoscalar \( I = \Be_1 \Be_2 \Be_3 \) commutes with all \R{3} multivectors.
%} % lemma
%
%To prove this, it is sufficient to consider the commutation of \( I \) with each of the standard basis vectors \( \Be_1, \Be_2, \Be_3 \) (\cref{problem:projectionAndRejection:1160}).
%
Now the cross product form of the rejection equation can be determined


\subsection{Normal factorization of the wedge product.}

A general bivector has the form

\begin{dmath}\label{eqn:SimpleProducts2:1800}
B = \sum_{i \ne j} a_{ij} \Be_i \Be_j,
\end{dmath}

and is not neccessarily a blade.  For example the bivector \( \Be_1 \Be_2 + \Be_3 \Be_4 \) cannot be factored into any product of normal vectors.
On the other hand,

\maketheorem{Wedge product normal factorization}{thm:SimpleProducts2:wnormalfactorize}{
The wedge product of any two non-colinear vectors \( \Ba, \Bb \) always has a normal factorization
\begin{equation*}
\Ba \wedge \Bb = \Bu \Bv, \quad \Bu \cdot \Bv = 0.
\end{equation*}
} % theorem

The significant of \cref{thm:SimpleProducts2:wnormalfactorize} means that the square of any wedge product is negative

\begin{dmath}\label{eqn:SimpleProducts2:1820}
(\Bu \Bv)^2
=
(\Bu \Bv) (-\Bv \Bu)
=
-\Bu (\Bv^2) \Bu
=
- \Abs{\Bu}^2 \Abs{\Bv}^2,
\end{dmath}

which in turn means that exponentials with wedge product arguments can be used as rotation operators.

To prove \cref{thm:SimpleProducts2:wnormalfactorize}, first assume that there is an orthonormal basis \( \setlr{\ucap, \vcap} \) for the planar subspace \( P = \Span\setlr{ \Ba, \Bb } \), for which

\begin{dmath}\label{eqn:SimpleProducts2:1840}
\begin{aligned}
\Ba &= (\Ba \cdot \ucap) \ucap + (\Ba \cdot \vcap) \vcap \\
\Bb &= (\Bb \cdot \ucap) \ucap + (\Bb \cdot \vcap) \vcap.
\end{aligned}
\end{dmath}

The wedge of \( \Ba, \Bb \) in terms of this basis is

\begin{dmath}\label{eqn:SimpleProducts2:1860}
\Ba \wedge \Bb
=
\gpgradetwo{
   \lr{
   (\Ba \cdot \ucap) \ucap + (\Ba \cdot \vcap) \vcap
   }
   \lr{
   (\Bb \cdot \ucap) \ucap + (\Bb \cdot \vcap) \vcap
   }
}
=
\gpgradetwo{
\cancel{
   (\Ba \cdot \ucap) (\Bb \cdot \ucap) \ucap^2
}
+
\cancel{
   (\Ba \cdot \vcap) (\Bb \cdot \vcap) \vcap^2
}
+
\lr{
      (\Ba \cdot \ucap)
   (\Bb \cdot \vcap)
   -
   (\Ba \cdot \vcap) (\Bb \cdot \ucap)
}
\ucap \vcap
}
=
\lr{
      (\Ba \cdot \ucap)
   (\Bb \cdot \vcap)
   -
   (\Ba \cdot \vcap) (\Bb \cdot \ucap)
}
\ucap \vcap.
\end{dmath}

Such a basis allows for the most compact (single term) coordinate representation of the wedge product

\begin{dmath}\label{eqn:SimpleProducts2:1880}
\Ba \wedge \Bb
=
\begin{vmatrix}
   \Ba \cdot \ucap & \Ba \cdot \vcap \\
   \Bb \cdot \ucap & \Bb \cdot \vcap
\end{vmatrix}
\ucap \vcap.
\end{dmath}

The wedge product is therefore the (possibly signed) area of the parallelopiped formed by the vectors \( \Ba, \Bb \), multiplied by a unit pseudoscalar for the subspace of the plane \( P \).  Provided the area of this parallelopiped is non-zero, which is always the case for non-colinear vectors, there are clearly many possible normal factorizations for the wedge product.

\subsection{General rotation.}

\Cref{eqn:SimpleProducts2:180} showed that the \R{2} pseudoscalar anticommutes with any vector \( \Bx \in \bbR^{2} \),

\begin{dmath}\label{eqn:SimpleProducts2:1760}
\Bx i = -i \Bx.
\end{dmath}

The higher dimensional generalization of this result is

\maketheorem{Commutation rules for wedge products.}{thm:SimpleProducts2:1780}{
Given a planar subspace formed by the span of two non-colinear vectors \( S = \Span \setlr{ \Ba, \Bb } \), any vector \( \Bx \in S \) anticommutes with the wedge product \( \Ba \wedge \Bb \)

\begin{equation*}
\Bx (\Ba \wedge \Bb) = - (\Ba \wedge \Bb) \Bx.
\end{equation*}

Moreover, any vector \( \Bn \) normal to this plane (\( \Bn \cdot \Ba = \Bn \cdot \Bb = 0 \)) commutes with this wedge product
\begin{equation*}
\Bn (\Ba \wedge \Bb) = (\Ba \wedge \Bb) \Bn.
\end{equation*}
} % theorem

A simple inelegant way to prove this is to specify a coordinate system for which \( \Ba, \Bb \) both lie in the \( x,y \) plane.  Then \( \Ba \wedge \Bb = \alpha i \) for some \( \alpha \), and the anticommutation part of the theorem follows from
%the \R{2} result
\cref{eqn:SimpleProducts2:1760}.  For the normal commutation part of the theorem, pick any vector normal to the \(x, y\) plane, say \( \Be_3\), for which we have

\begin{dmath}\label{eqn:SimpleProducts2:1780}
\Be_3  (\Ba \wedge \Bb)
=
\Be_3 \alpha i
=
\alpha \Be_3 \Be_1 \Be_2
=
\alpha (-\Be_1 \Be_3) \Be_2
=
-\alpha \Be_1 (\Be_3 \Be_2)
=
-\alpha \Be_1 (-\Be_2 \Be_3)
= (\Ba \wedge \Bb) \Be_3.
\end{dmath}

In dimensions with more normals, say \( \Be_4, \cdots \), the steps of \cref{eqn:SimpleProducts2:1780} can be repeated.  The general normal commuation result follows by superposition.

\subsection{The wedge product as an area element.}
\subsection{Symmetric and antisymmetric vector sums.}
\subsection{Duality}
\subsection{Cyclic permutation in scalar selection.}
\subsection{Dot products of other blades.}
\subsection{Orientation.}
\subsection{Wedge of multiple vectors.}
\subsection{Reflection.}
\subsection{Linear systems}

