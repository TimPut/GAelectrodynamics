%
% Copyright © 2016 Peeter Joot.  All Rights Reserved.
% Licenced as described in the file LICENSE under the root directory of this GIT repository.
%
\index{Maxwell's equation}
Assuming
isotropic constituency relationships \cref{eqn:freespace:300}, and
eliminating cross products in favor of wedge products,
Maxwell's equations \cref{eqn:freespace:3399} become

\begin{subequations}
\label{eqn:isotropicMaxwells:61}
\begin{dmath}\label{eqn:isotropicMaxwells:60}
\spacegrad \wedge \BE = - I \BM - \mu \PD{t}{(I\BH)}
\end{dmath}
\begin{dmath}\label{eqn:isotropicMaxwells:80}
\spacegrad \wedge \BH = I \BJ + I \epsilon \PD{t}{\BE}
\end{dmath}
\begin{dmath}\label{eqn:isotropicMaxwells:100}
\spacegrad \cdot \BE = \rho/\epsilon
\end{dmath}
\begin{dmath}\label{eqn:isotropicMaxwells:120}
\spacegrad \cdot \BH = \rho_\txtm/\mu.
\end{dmath}
\end{subequations}

With a bit of rescaling, the dot and wedge products of \cref{eqn:isotropicMaxwells:60}, \cref{eqn:isotropicMaxwells:100} can be added, as can those of \cref{eqn:isotropicMaxwells:80}, \cref{eqn:isotropicMaxwells:120}.
This reduces Maxwell's equations to a pair of first order coupled gradient equations

\begin{subequations}
\label{eqn:isotropicMaxwells:361}
\begin{dmath}\label{eqn:isotropicMaxwells:360}
\spacegrad \BE = \inv{\epsilon} \rho - I \BM - \mu \PD{t}{(I\BH)}
\end{dmath}
\begin{dmath}\label{eqn:isotropicMaxwells:380}
\spacegrad (I \BH) = \frac{I \rho_\txtm}{\mu} - \BJ - \epsilon \PD{t}{\BE}.
\end{dmath}
\end{subequations}

There is a symmetry in \cref{eqn:isotropicMaxwells:361} that an be made more apparent by
making a transformation to fields with the same dimensions \( \sqrt{\epsilon} \BE, \sqrt{\mu} \BH \)

\begin{subequations}
\begin{dmath}\label{eqn:isotropicMaxwells:400}
\spacegrad (\sqrt{\epsilon} \BE) = \inv{\sqrt{\epsilon}} \rho - I \sqrt{\epsilon} \BM - \sqrt{\epsilon \mu} \PD{t}{(I\sqrt{\mu} \BH)}
\end{dmath}
\begin{dmath}\label{eqn:isotropicMaxwells:420}
\spacegrad (I \sqrt{\mu} \BH) = \frac{I \rho_\txtm}{\sqrt{\mu}} - \sqrt{\mu} \BJ - \sqrt{\epsilon\mu} \PD{t}{\sqrt{\epsilon} \BE}.
\end{dmath}
\end{subequations}

Observations
\begin{itemize}
\item
The electric charge density of \cref{eqn:isotropicMaxwells:400} has a corresponding dual magnetic charge density in \cref{eqn:isotropicMaxwells:420}.
\item
There is a similar dual relationship between the electric and magnetic current densities and the fields themselves.
\item
The multivector equation \cref{eqn:isotropicMaxwells:400} has grades 0,2 (scalar and bivector), whereas the multivector equation \cref{eqn:isotropicMaxwells:420} has grades 1,3 (vector, pseudoscalar).
\item
The
dimensions of the space and time differential operators \( [\spacegrad] = [\sqrt{\epsilon\mu} \PDi{t}{}] = 1/L = [\sqrt{\epsilon\mu}\PDi{t}{}]\) are equal.
now equal,
\end{itemize}

Without any overlap of grades, it is possible to add (or subtract) these equations without any loss of information, as
the original equations could always be recovered by grade selection.
Adding \cref{eqn:isotropicMaxwells:400} gives

\begin{dmath}\label{eqn:isotropicMaxwells:440}
\lr{ \spacegrad + \sqrt{\epsilon\mu} \PD{t}{} }
\lr{ \sqrt{\epsilon} \BE
+
I \sqrt{\mu} \BH
}
=
\inv{\sqrt{\epsilon}} \rho
- I \sqrt{\epsilon} \BM
+\lr{
\frac{I \rho_\txtm}{\sqrt{\mu}}
- \sqrt{\mu} \BJ
}
.
\end{dmath}

After rescaling so that we are using
the engineering convention where \( \BE \) and \( \BH \) are the primary fields, we obtain the geometric algebra multivector form of Maxwell's equation

\boxedEquation{eqn:maxwellsEquations:460}{
\begin{aligned}
F &= \BE + \eta I \BH \\
J &=
\eta
\lr{ c \rho - \BJ }
+ I \lr{ c \rho_\txtm - \BM } \\
\lr{ \spacegrad + \inv{c} \PD{t}{} } F
&= J.
\end{aligned}
}

Here

\begin{itemize}
\item \( F = \BE + \eta I \BH \) [\si{V/m}] (Volts/meter), is the electromagnetic field strength.
This is a multivector field with vector and bivector components.  Some authors call this the Faraday.
\item \( J = \eta \lr{ c \rho - \BJ } + I \lr{ c \rho_\txtm - \BM } \) ([\si{A/m^2}] (Amperes/square meter)),
is a multivector source containing all the electric and magnetic charge and current densities.  This is called the current.
When the fictious magnetic source terms are included this source multivector has one grade for each possible source (scalar, vector, bivector, trivector).  When the current has no fictious magnetic sources, this is still a multivector, but contains only scalar and vector grades.
\item \( \eta = \sqrt{\mu/\epsilon} \) (\( [\Omega] \) Ohms), is the impedance of the media.
\item \( c = 1/\sqrt{\epsilon\mu} \) ([\si{m/s}] meters/second), is the group velocity of a wave in the media, and when \( \epsilon = \epsilon_0, \mu = \mu_0 \) is the speed of light.  
A justification for calling this the group velocity will follow shortly.
\end{itemize}

Maxwell's equation \cref{eqn:maxwellsEquations:460} is a
single multivector equation that relates a combined electromagnetic field to all the sources, including fictional magnetic sources if desired.
%With electric and magnetic fields being observer frame dependent, we

There will be a number of problems where it will be simpler to solve for the electromagnetic field than to solve separately for the electric and magnetic fields.
Given such a field solution, the electric and magnetic fields can be recovered by grade selection

\begin{dmath}\label{eqn:isotropicMaxwells:480}
\begin{aligned}
\BE &= \gpgradeone{F} \\
I \BH &= \inv{\eta} \gpgradetwo{F}.
\end{aligned}
\end{dmath}
