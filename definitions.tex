%
% Copyright � 2016 Peeter Joot.  All Rights Reserved.
% Licenced as described in the file LICENSE under the root directory of this GIT repository.
%
%{
%%\input{../blogpost.tex}
%%\renewcommand{\basename}{multiplication}
%%%\renewcommand{\dirname}{notes/phy1520/}
%%\renewcommand{\dirname}{notes/ece1228-electromagnetic-theory/}
%%%\newcommand{\dateintitle}{}
%%%\newcommand{\keywords}{}
%%
%%\input{../peeter_prologue_print2.tex}
%%
%%\usepackage{peeters_layout_exercise}
%%\usepackage{peeters_braket}
%%\usepackage{peeters_figures}
%%\usepackage{siunitx}
%%%\usepackage{mhchem} % \ce{}
%%%\usepackage{macros_bm} % \bcM
%%%\usepackage{txfonts} % \ointclockwise
%%
%%\beginArtNoToc
%%
%%\generatetitle{Vector multiplication}
%%%\chapter{Vector multiplication}
%%%\label{chap:multiplication}
%%

A few new GA terms have been introduced in an ad-hoc fashion as required.  Here is a systematic exposition of some of the key definitions used to refer to the types of the geometric objects that will be encountered.

\makedefinition{Scalar}{def:multiplication:scalar}{
A real number with no implied direction.
}

\makedefinition{Vector}{def:multiplication:vector}{
%\href{https://www.youtube.com/watch?v=bOIe0DIMbI8}{A quantity with direction and magnitude.}
\href{https://youtu.be/bOIe0DIMbI8?t=19}{A quantity with direction and magnitude.}
}

\makedefinition{Bivector}{def:multiplication:bivector}{
A product of two normal vectors, or a sum thereof.
}

The product \( \Be_1 \Be_2 \) is a bivector, as is \( \Be_2 \Be_3 + 3 \Be_4 \Be_1 \)

\makedefinition{Trivector}{def:multiplication:trivector}{
A product of three mutually normal vectors, or a sum thereof.
}

The quantity \( \Be_3 \Be_1 \Be_2 \) is a trivector, as is \( \Be_1 \Be_2 \Be_3 + 3 \Be_5 \Be_4 \Be_1 \).

\makedefinition{Blade}{def:multiplication:blade}{
A scalar, vector, bivector, or a trivector (or higher degree analogue), that can be constructed by multiplication of a number of vectors, but not an unfactorable sum of thereof.
}

The factorable quantity
\begin{dmath}\label{eqn:multiplication:220}
\Be_1 \Be_2 + 3 \Be_1 \Be_3
=
\Be_1 (\Be_2 + 3 \Be_3)
\end{dmath}

is a blade, whereas

\begin{dmath}\label{eqn:multiplication:240}
\Be_1 \Be_2 + 3 \Be_3 \Be_4,
\end{dmath}

and
\begin{dmath}\label{eqn:multiplication:260}
\Be_1 \Be_2 + \Be_2 \Be_3 + \Be_3 \Be_1
\end{dmath}

are not.

Scalar, vector, bivector, and trivectors are also referred to as sums of 0-blades, 1-blades, 2-blades, and 3-blades respectively.

\makedefinition{Grade.}{def:multiplication:grade}{
The minimum number of vector products required to form a given blade.
}

The grade of a scalar, vector, bivector, and trivector are 0, 1, 2, and 3 respectively.

The quantities
\begin{dmath}\label{eqn:multiplication:300}
\begin{aligned}
\Be_2 + \Be_1 \Be_2 \Be_2 &= \Be_1 + \Be_2 \\
\Be_1 \Be_2 \Be_2 \Be_2 \Be_3 &= \Be_1 \Be_2 \Be_3 \\
\end{aligned}
\end{dmath}

have grades 1 and 3 respectively.

Quantities with higher grades than 3 are not generally given explicit names, but can be referred to having grade-k.  When an object of grade-k is
also a blade, it can be referred to as a k-blade.

In a three dimensional space the highest grade possible is 3.  Blades can have grades higher than 3 in higher dimensional vector spaces.

\makedefinition{Pseudoscalar.}{def:multiplication:pseudoscalar}{
A blade with grade that matches the dimension of the space.
}

In a two dimensional space \( \Be_2 \Be_1 \) is a pseudoscalar.  In a three dimensional space
\( \Be_3 \Be_1 \Be_2 \) is a pseudoscalar, as is \( \Be_3 \Be_1 (\Be_2 + \Be_3 ) \).  A pseudoscalar has an implied orientation, which can be
associated with the handedness of the underlying basis.  It is conventional to refer to

\begin{dmath}\label{eqn:definitions:320}
i = \Be_1 \Be_2,
\end{dmath}

as ``the pseudoscalar'' for a two dimensional space, and to

\begin{dmath}\label{eqn:definitions:340}
I = \Be_1 \Be_2 \Be_3,
\end{dmath}

as ``the pseudoscalar'' for a three dimensional space.

\makedefinition{Multivector.}{def:multiplication:multivector}{
A sum of zero or more blades.
}

Examples include
\begin{dmath}\label{eqn:multiplication:280}
\begin{aligned}
&3 \\
& 1 + \Be_1 \Be_2 \\
& 2 - \Be_1 \Be_2 \Be_3 \\
& \Be_1 + 2 \Be_1 \Be_2 + \Be_2 \Be_3 - 3 \Be_3 \Be_1 + \Be_1 \Be_2 \Be_4
\end{aligned}
\end{dmath}

\makedefinition{Dual}{dfn:definitions:dual}{
The dual of a multivector is the product of that multivector with a pseudoscalar for a subspace that contains the multivector.  Such multiplication is referred to as a duality transformation.
} % definition

For example, given \( M = 1 + \Be_1 \Be_2 \), multiplication by \( i = \Be_1 \Be_2 \) is a duality transformation with respect to the x-y plane, and multiplication by \( I = \Be_1 \Be_2 \Be_3 \) is a duality transformation with \R{3}.

With the following shorthand notation is convienent for sucessive products of orthonormal basis vectors

\begin{dmath}\label{eqn:definitions:420}
\Be_{ij\cdots k}  = \Be_i \Be_j \cdots \Be_k,
\end{dmath}

consider some concrete examples of duality transformations of blades.
The dual vectors to the basis vectors of a 2D space are those same vectors rotated by \( \pi/2 \)

\begin{dmath}\label{eqn:definitions:360}
\begin{aligned}
\Be_1 \Be_{12} &= \Be_2 \\
\Be_2 \Be_{12} &= -\Be_1,
\end{aligned}
\end{dmath}

with an inverse duality transformation given by the multiplication with \( \Be_{12}^{-1} = \Be_{21} \)

\begin{dmath}\label{eqn:definitions:440}
\begin{aligned}
\Be_2 \Be_{21} &= \Be_1 \\
-\Be_1 \Be_{21} &= \Be_2.
\end{aligned}
\end{dmath}

The \R{3} duals to the basis vectors are bivectors

\begin{dmath}\label{eqn:definitions:380}
\begin{aligned}
\Be_1 \Be_{123} &= \Be_{23} \\
\Be_2 \Be_{123} &= \Be_{31} \\
\Be_3 \Be_{123} &= \Be_{12},
\end{aligned}
\end{dmath}

whereas the duals to those bivectors with respect to the pseudoscalar \( I^{-1} = \Be_{321} \) are the original basis vectors

\begin{dmath}\label{eqn:definitions:400}
\begin{aligned}
\Be_{23} \Be_{321} &= \Be_1 \\
\Be_{31} \Be_{321} &= \Be_2 \\
\Be_{12} \Be_{321} &= \Be_3.
\end{aligned}
\end{dmath}

In a sense that can be defined more precisely once the general dot product operator is defined, the dual to a given blade represents an object that is normal to the original blade.

The dual of any scalar is a pseudoscalar, whereas the dual of a pseudoscalar is a scalar.

%When working with multivector integrals it will be useful to consider the differential volume element a volume weighted pseudoscalar.

%%%}
%%%\EndArticle
%%\EndNoBibArticle
