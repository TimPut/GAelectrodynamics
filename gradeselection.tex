%
% Copyright � 2016 Peeter Joot.  All Rights Reserved.
% Licenced as described in the file LICENSE under the root directory of this GIT repository.
%
%{
%\input{../blogpost.tex}
%\renewcommand{\basename}{gradeselection}
%%\renewcommand{\dirname}{notes/phy1520/}
%\renewcommand{\dirname}{notes/ece1228-electromagnetic-theory/}
%%\newcommand{\dateintitle}{}
%%\newcommand{\keywords}{}
%
%\input{../peeter_prologue_print2.tex}
%
%\usepackage{peeters_layout_exercise}
%\usepackage{peeters_braket}
%\usepackage{peeters_figures}
%\usepackage{siunitx}
%%\usepackage{mhchem} % \ce{}
%%\usepackage{macros_bm} % \bcM
%\usepackage{macros_qed} % \qedmarker
%%\usepackage{txfonts} % \ointclockwise
%
%\beginArtNoToc
%
%\generatetitle{XXX}
%%\chapter{XXX}
%%\label{chap:gradeselection}
%
Having defined the axioms and definitions of Geometric Algebra, it desirable to define the grade selection operator, the dot product operator and the wedge product operator, and consider some simple examples of each.

(cut)
As an example, consider two vectors in a 2D space

\begin{dmath}\label{eqn:gradeselection:140}
\begin{aligned}
\Ba  &= \lr{ x \Be_1 + y \Be_2 } \\
\Ba' &= \lr{ x' \Be_1 + y' \Be_2 },
\end{aligned}
\end{dmath}

for which this definition of the dot product gives

\begin{dmath}\label{eqn:gradeselection:160}
\Ba \cdot \Ba'
=
\gpgrade{ \Ba \Ba' }{\Abs{1 - 1}}
=
\gpgradezero{ \Ba \Ba' }
=
\gpgradezero{ \lr{ x \Be_1 + y \Be_2 } \lr{ x' \Be_1 + y' \Be_2 } }
=
\gpgradezero{ x x' \Be_1^2 + y y' \Be_2^2 + (x y' - y x') \Be_1 \Be_2 }
=
x x' + y y'.
\end{dmath}

It is left to the reader (\cref{problem:gradeselection:RnDotProduct}) to show that this definition also reduces to the traditional \R{n} dot product.

As a second example, consider the dot product of a vector with a bivector.  With \( \Ba \) as defined in \cref{eqn:gradeselection:140} and \( i = \Be_1 \Be_2 \)

\begin{dmath}\label{eqn:gradeselection:240}
\Ba \cdot i
=
\gpgrade{ \Ba i }{1}
=
\gpgrade{ \lr{ x \Be_1 + y \Be_2 } \Be_1 \Be_2 }{1}
=
\gpgrade{ x \Be_1^2 \Be_2 + y \Be_2 (-\Be_2 \Be_1) }{1}
=
\gpgrade{ x \Be_2 - y \Be_1 }{1}
=
x \Be_2 - y \Be_1.
\end{dmath}

This particular dot product is trivial, since the product \( \Ba i \) has only a vector component.
In this example \( i \) is the pseudoscalar for the two dimensional space, and it can be observed that multiplication of a vector from the right serves to rotate the vector by 90 degrees.  It is not a coincidence that this is strikingly similar to the action of the imaginary from complex algebra.  It can be shown (\cref{problem:gradeselection:PlaneRotations})
that \( e^{i\theta} \) acts as a rotation operator as it does in complex algebra, and that a GA representation of complex numbers is possible (\cref{problem:gradeselection:ComplexNumbers}).

For a non-trivial vector-bivector dot product, consider

\begin{dmath}\label{eqn:gradeselection:560}
\lr{ \Be_1 + \Be_2 } \cdot \lr{ \Be_1 \Be_2 + 3 \Be_2 \Be_3 }
=
\gpgradeone{
\lr{ \Be_1 + \Be_2 } \lr{ \Be_1 \Be_2 + 3 \Be_2 \Be_3 }
}
=
\gpgradeone{
\Be_1^2 \Be_2 + 3 \Be_1 \Be_2 \Be_3
+
\Be_2 \Be_1 \Be_2 + 3 \Be_2^2 \Be_3
}
=
\gpgradeone{
\Be_2 + 3 \cancel{\Be_1 \Be_2 \Be_3}
-
\Be_1 + 3 \Be_3
}
=
\Be_2 - \Be_1 + 3 \Be_3.
\end{dmath}

The vector-bivector dot product filters out products that no common factors, since such products result in trivector components.

(cut)

For example, the wedge product of the 2D vectors of \cref{eqn:gradeselection:140} is

\begin{dmath}\label{eqn:gradeselection:500}
\Ba \wedge \Bb
=
\gpgradetwo{
\lr{ x \Be_1 + y \Be_2 }
\lr{ x' \Be_1 + y' \Be_2 }
}
=
\gpgradetwo{
(x x' + y y') + (x y' - x' y) \Be_1 \Be_2
}
=
(x y' - x' y) \Be_1 \Be_2.
\end{dmath}

The wedge product of two vectors in a plane contains an antisymmetrized sum of the vector coefficients, but is weighted by a ``unit'' bivector, the pseudoscalar for the plane.

As another example consider

\begin{dmath}\label{eqn:gradeselection:520}
\Be_1 \wedge \lr{ 2\Be_1 + 3 \Be_2 }
=
\gpgradetwo{
\Be_1 \lr{ 2\Be_1 + 3 \Be_2 }
}
=
\gpgradetwo{
2 \Be_1^2 + 3 \Be_1 \Be_2
}
=
3 \Be_1 \Be_2.
\end{dmath}

Components of the vectors are that colinear are filtered out.  In this case that is the \( \Be_1 \) component of the second vector \( \Be_1 + 3 \Be_2 \).  It is not coincidence that this is also a property of the cross product.  That relationship will be explored in (\cref{problem:gradeselection:WedgeRelationshipToCrossProduct}).

As a final example, consider the wedge product of a vector with a bivector

\begin{dmath}\label{eqn:gradeselection:540}
\Be_1 \wedge \lr{ \Be_1 \Be_2 - 7 \Be_2 \Be_3 }
=
\gpgradethree{
\Be_1 \lr{ \Be_1 \Be_2 - 7 \Be_2 \Be_3 }
}
=
\gpgradethree{
\Be_1^2 \Be_2 - 7 \Be_1 \Be_2 \Be_3
}
=
- 7 \Be_1 \Be_2 \Be_3.
\end{dmath}

Because \( \Be_1 \Be_2 \) has a common factor with \( \Be_1 \) it is filtered out of the resulting wedge product.  The end result, in this case, is a 3D pseudoscalar.

The wedge product of two bivectors in \R{3}, by this definition, is always zero, since there can be no grade 4 term in such a product.  It is also the case that the components of any \R{3} bivectors wedged together will also have a common factor, which nessesarily kills the wedge product of any two \R{3} bivectors.  This is not the case for arbitrary \R{N} bivectors, an example of which is \( \Be_1 \Be_2 + \Be_3 \Be_4 \).  There is no common factor in this bivector, so it can be wedged with itself and still produce a non-zero result (i.e. this bivector is not a blade).

%}
%\EndNoBibArticle
