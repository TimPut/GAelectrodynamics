%
% Copyright � 2018 Peeter Joot.  All Rights Reserved.
% Licenced as described in the file LICENSE under the root directory of this GIT repository.
%
%{
%%%\input{../latex/blogpost.tex}
%%%\renewcommand{\basename}{greensFunctionSpacetimeGradient}
%%%%\renewcommand{\dirname}{notes/phy1520/}
%%%\renewcommand{\dirname}{notes/ece1228-electromagnetic-theory/}
%%%%\newcommand{\dateintitle}{}
%%%%\newcommand{\keywords}{}
%%%
%%%\input{../latex/peeter_prologue_print2.tex}
%%%
%%%\usepackage{peeters_layout_exercise}
%%%\usepackage{peeters_braket}
%%%\usepackage{peeters_figures}
%%%\usepackage{siunitx}
%%%%\usepackage{mhchem} % \ce{}
%%%%\usepackage{macros_bm} % \bcM
%%%%\usepackage{macros_qed} % \qedmarker
%%%%\usepackage{txfonts} % \ointclockwise
%%%
%%%\beginArtNoToc
%%%
%%%\generatetitle{Green's function for the spacetime gradient}
%%%%\chapter{Green's function for the spacetime gradient}
%%%%\label{chap:greensFunctionSpacetimeGradient}
%%%% \citep{griffiths1999introduction}

We want to find the Green's function that solves spacetime gradient equations of the form \cref{eqn:greensFunctionOverview:220}.
For the wave equation operator, it is helpful to introduce a d'Lambertian operator, defined as follows.

\index{\(\dAlembertian\)}
%
% Copyright � 2018 Peeter Joot.  All Rights Reserved.
% Licenced as described in the file LICENSE under the root directory of this GIT repository.
%
\makedefinition{d'Lambertian (wave equation) operator.}{dfn:continuity:120}{
Let
\begin{equation*}
\dAlembertian =
\conjstgrad
\stgrad
=
\spacegrad^2 - \inv{c^2} \PDSq{t}{}.
\end{equation*}
} % definition


We will be able to derive the Green's function for the spacetime gradient from the Green's function for the d'Lambertian.  The Green's function for the spacetime gradient is multivector valued and given by the following.
%
% Copyright � 2018 Peeter Joot.  All Rights Reserved.
% Licenced as described in the file LICENSE under the root directory of this GIT repository.
%
\maketheorem{Green's function for the spacetime gradient.}{thm:greensFunctionSpacetimeGradient:120}{
The \textit{Green's function for the spacetime gradient} \( \spacegrad + (1/c) \partial_t \) satisfies
\begin{equation*}
\stgrad G(\Bx - \Bx', t - t') = \delta(\Bx - \Bx') \delta(t - t'),
\end{equation*}
and has the value
\begin{equation*}
G(\Bx - \Bx', t - t')
=
\inv{4\pi} \lr{
- \frac{\rcap}{r^2} \PD{r}{}
+ \frac{\rcap}{r}
+ \inv{c r} \PD{t}{}
}
\delta( -r/c + t - t' ),
\end{equation*}
where \( \Br = \Bx - \Bx', r = \Norm{\Br} \) and \( \rcap = \Br/r \).
} % theorem


With the help of \cref{eqn:derivativeOfDeltaFunction:140}
it is possible to further evaluate the delta function derivatives, however, we will defer doing so until we are ready to apply this Green's
function in a convolution integral to solve Maxwell's equation.
\begin{proof}
To prove this result, let \( \phi(\Bx - \Bx', t - t') \) be the retarded time (causal)
Green's function for the wave equation, satisfying
\begin{dmath}\label{eqn:greensFunctionSpacetimeGradient:40}
\dAlembertian
\phi(\Bx - \Bx', t - t')
=
\stgrad
\conjstgrad
\phi(\Bx - \Bx', t - t')
= \delta(\Bx - \Bx') \delta(t - t').
\end{dmath}

This function has the value
\begin{dmath}\label{eqn:greensFunctionSpacetimeGradient:60}
\phi(\Br, t - t')
=
-\inv{4 \pi r} \delta( -r/c + t - t' ),
\end{dmath}
where \( \Br = \Bx - \Bx', r = \Norm{\Br} \).  Derivations of this Green's function, and it's acausal advanced time friend, can be found in
\citep{schwinger1998classical}, \citep{jackson1975cew}, and use the usual Fourier transform and contour integration tricks.

Comparing \cref{eqn:greensFunctionSpacetimeGradient:40} to the defining statement of \cref{thm:greensFunctionSpacetimeGradient:120}, we see that the spacetime gradient Green's function is given by
\begin{dmath}\label{eqn:greensFunctionSpacetimeGradient:80}
G(\Bx - \Bx', t - t')
=
\conjstgrad \phi(\Br, t - t')
=
\lr{ \rcap \PD{r}{} - \inv{c} \PD{t}{} } \phi(\Br, t - t'),
\end{dmath}
where \( \rcap = \Br/r \).  Evaluating the derivatives gives
\begin{dmath}\label{eqn:greensFunctionSpacetimeGradient:100}
G(\Br, t - t')
=
-\inv{4\pi} \lr{ \rcap \PD{r}{} - \inv{c} \PD{t}{} } \frac{ \delta( -r/c + t - t' ) }{r}
=
-\inv{4\pi} \lr{
\frac{\rcap}{r} \PD{r}{} \delta( -r/c + t - t' )
- \frac{\rcap}{r^2} \delta( -r/c + t - t' )
- \inv{c r} \PD{t}{} \delta( -r/c + t - t' )
},
\end{dmath}
which completes the proof after some sign cancellation and minor rearrangement.
\end{proof}
%}
%%%\EndArticle
%%%%\EndNoBibArticle
