%
% Copyright © 2018 Peeter Joot.  All Rights Reserved.
% Licenced as described in the file LICENSE under the root directory of this GIT repository.
%
%{

\index{circular polarization}
\index{left circular polarization}
\index{right circular polarization}

The time domain field when written out explicitly in terms of the Jones vector components is
\begin{dmath}\label{eqn:polarization_circular:160}
F(\Bx, t) = (1 + \Be_3) \lr{
\lr{ \alpha_1 \Be_1 + \alpha_2 \Be_2 } \cos\lr{ \omega t - \beta z }
-\lr{ \beta_1 \Be_1 + \beta_2 \Be_2 } \sin\lr{ \omega t - \beta z }
}.
\end{dmath}

Linear, circular, and elliptical polarization patterns can be obtained by selecting specific values of the Jones vector, or equivalently by selecting specific values for the \( \alpha_1, \alpha_2, \beta_1, \beta_2 \) constants in the time domain representation of \cref{eqn:polarization_circular:160}.
In particular,
a field for which the
change in phase
\begin{dmath}\label{eqn:polarization_circular:520}
\phi = \omega t - \beta z,
\end{dmath}
results in the electric field tracing out a (clockwise,counterclockwise) circle
\begin{dmath}\label{eqn:polarization_circular:180}
\begin{aligned}
\BE_\txtL &= \Abs{\BE} \lr{ \Be_1 \cos\phi + \Be_2 \sin\phi } = \Abs{\BE} \Be_1 \exp\lr{  \Be_{12} \phi } \\
\BE_\txtR &= \Abs{\BE} \lr{ \Be_1 \cos\phi - \Be_2 \sin\phi } = \Abs{\BE} \Be_1 \exp\lr{ -\Be_{12} \phi },
\end{aligned}
\end{dmath}
is referred to as having
(right,left) circular polarization.
There are different conventions for the polarization orientation, and here the IEEE antenna convention discussed in \citep{balanis1989advanced} are used.

%Fixme: flipped the orientation I was using, as I noticed after the fact that the figures in Balanis use an orientation with x-axis up and y-axis right!
% check that everything still looks correct.
% ( I didn't notice the inscribed X's in the polarization figures: https://en.wikibooks.org/wiki/Physics_Study_Guide/Vectors_and_scalars#How_to_draw_vectors_that_are_in_or_out_of_the_plane_of_the_page_.28or_board.29 )

The bivector exponential representation of the circularly polarized electric fields in \cref{eqn:polarization_circular:180} indicates that it is possible to represent arbitrary field polarization in a GA form that does not require any real part operation, as follows
\begin{dmath}\label{eqn:polarization_circular:200}
F = \lr{ 1 + \Be_3 } \Be_1 \lr{ \alpha_\txtL e^{i\phi} + \alpha_\txtR e^{-i\phi} },
\end{dmath}
where the constants \( \alpha_\txtL, \alpha_\txtR \) are both complex with respect to the unit bivector imaginary \( i = \Be_{12} \) representing the plane transverse to the propagation direction
\begin{dmath}\label{eqn:polarization_circular:220}
\begin{aligned}
\alpha_\txtL &= \alpha_{\txtL 1} + i \alpha_{\txtL 2} \\
\alpha_\txtR &= \alpha_{\txtR 1} + i \alpha_{\txtR 2}.
\end{aligned}
\end{dmath}

If a transformation from scalar to bivector imaginary \( j \rightarrow \Be_{12} = i \) is made in the Jones vector component representation of \cref{eqn:polarization:140},
then
the coefficients \cref{eqn:polarization_circular:220} of the circular polarization states are related to the Jones vector by (\cref{problem:polarization:1})
\begin{dmath}\label{eqn:polarization_circular:260}
\begin{aligned}
\alpha_\txtL &= \inv{2}\lr{ c_1 - i c_2 } \\
\alpha_\txtR &= \inv{2}\lr{ c_1 + i c_2 }^\dagger.
\end{aligned}
\end{dmath}
%}
