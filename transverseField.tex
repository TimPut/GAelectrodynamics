%
% Copyright © 2017 Peeter Joot.  All Rights Reserved.
% Licenced as described in the file LICENSE under the root directory of this GIT repository.
%
%original ideas from gabookII/electrodynamics/transverseField.tex:
\section{Transverse fields in a waveguide.}

Under source free conditions, Maxwell's equation in GA form is

\begin{equation}\label{eqn:transverseField:2}
\begin{aligned}
F &= \BE + I \eta \BH \\
0 &= \lr{ \spacegrad + \inv{t} \partial_t } F
\end{aligned}
\end{equation}

Maxwell's equation allows for components of the electric and magnetic field along the propagation direction and the transverse plane, however, it is possible to relate the transverse and propagating field components.  Assume that the propagation direction is along the z-axis (either forward or backwards), with angular frequency \( \omega \), with the field represented by the real part of

\begin{dmath}\label{eqn:transverseField:28}
F(x, y, z, t) = F(x, y) e^{j \omega t \mp j k z}.
\end{dmath}

Breaking with convention, we will use \( \zcap \) instead of \( \Be_3 \) for the unit vector along the z-axis, since that matches the explicit z-suffixes that we will use, such as the following split of the
gradient into transverse and \(z\)-axis components

\begin{dmath}\label{eqn:transverseField:88}
\spacegrad = \spacegrad_t + \zcap \partial_z.
\end{dmath}

Maxwell's equation becomes

\begin{equation}\label{eqn:transverseField:summaryMax2}
\lr{\spacegrad_t + j k \lr{ 1 \mp \zcap } } F(x,y) = 0,
\end{equation}

or with \( F = F(x, y) \)

\begin{dmath}\label{eqn:transverseField:148}
\spacegrad_t F = - j k \lr{ 1 \mp \zcap } F
\end{dmath}

We require a way to expressing the components of the field that lie in the propagation direction and transverse planes.
Let the propagation component be designated \( F_z \) so that

\begin{dmath}\label{eqn:transverseField:108}
F_z
=
\lr{ \BE \cdot \zcap }
 \zcap
+ I \eta \lr{ \BH \cdot \zcap } \zcap
=
\inv{2}
\lr{ \BE \zcap + \zcap \BE }
 \zcap
+ \inv{2} I \eta \lr{ \BH \zcap + \zcap \BH } \zcap
=
\inv{2}
\lr{ \BE + \zcap \BE \zcap }
+ \inv{2} I \eta \lr{ \BH + \zcap \BH \zcap },
\end{dmath}

showing that the propagation component \( F_z \) and transverse components \( F_t = F - F_z \) of the total field are

\begin{dmath}\label{eqn:transverseField:128}
\begin{aligned}
F_z &= \inv{2} \lr{ F + \zcap F \zcap } \\
F_t &= \inv{2} \lr{ F - \zcap F \zcap } \\
\end{aligned}
\end{dmath}

Since \( \spacegrad_t \) has no \( \xcap, \ycap \) components, \( \zcap \) anticommutes with the transverse gradient

\begin{dmath}\label{eqn:transverseField:168}
\zcap \spacegrad_t = - \spacegrad_t \zcap,
\end{dmath}

but commutes with \( 1 \mp \zcap \).
%In \cref{eqn:transverseField:168} it is implied that the action of \( \spacegrad_t \) is on everything to its right.
This means that

\begin{dmath}\label{eqn:transverseField:188}
\inv{2} \lr{ \spacegrad_t F \pm \zcap \lr{ \spacegrad_t F } \zcap }
=
\inv{2} \lr{ \spacegrad_t F \mp \spacegrad_t \zcap F \zcap }
=
\spacegrad_t
\inv{2} \lr{ F \mp \zcap F \zcap },
\end{dmath}

or
\begin{dmath}\label{eqn:transverseField:208}
\begin{aligned}
\inv{2} \lr{ \spacegrad_t F + \zcap \lr{ \spacegrad_t F } \zcap } &= \spacegrad_t F_t \\
\inv{2} \lr{ \spacegrad_t F - \zcap \lr{ \spacegrad_t F } \zcap } &= \spacegrad_t F_z,
\end{aligned}
\end{dmath}

so Maxwell's equation \cref{eqn:transverseField:148} becomes

\begin{dmath}\label{eqn:transverseField:228}
\begin{aligned}
\spacegrad_t F_t &= - j k \lr{ 1 \mp \zcap } F_z \\
\spacegrad_t F_z &= - j k \lr{ 1 \mp \zcap } F_t.
\end{aligned}
\end{dmath}

This has the appearance of being invertable, like so

\boxedEquation{eqn:transverseField:248}{
\begin{aligned}
F_z &= j \inv{ k \lr{ 1 \mp \zcap }} \spacegrad_t F_t \\
F_t &= j \inv{ k \lr{ 1 \mp \zcap }} \spacegrad_t F_z \\
\end{aligned}
}

so that if one of \( F_z \) or \( F_t \) is known, then the other can be computed by applying the transverse gradient.  However, this is WRONG, since a multivector like \( 1 + \zcap \) is not invertible!

\section{original content to rewrite.}

With an application of the transverse gradient to the \(z\)-axis field we easily found the relation between the two field components

\begin{equation}\label{eqn:transverseField:6}
\begin{aligned}
\spacegrad_t F_z &= i \left( \pm k \zcap - \sqrt{\mu\epsilon}\frac{\omega}{c}\right) F_t
\end{aligned}
\end{equation}

A left division by the multivector factor gives the total transverse field

\begin{equation}\label{eqn:transverseField:7}
\begin{aligned}
F_t &= \inv{i \left( \pm k \zcap - \sqrt{\mu\epsilon}\frac{\omega}{c}\right) } \spacegrad_t F_z
\end{aligned}
\end{equation}

Multiplication of both the numerator and denominator by the conjugate normalizes this

\begin{equation}\label{eqn:transverseField:48}
\begin{aligned}
F_t &= \frac{i}{k^2 - \mu\epsilon\frac{\omega^2}{c^2}} \left( \pm k \zcap + \sqrt{\mu\epsilon}\frac{\omega}{c}\right) \spacegrad_t F_z
\end{aligned}
\end{equation}

The electric and magnetic fields can be picked off by selecting the grade one (vector) components

\begin{equation}\label{eqn:transverseField:8}
\begin{aligned}
\BE &= \gpgradeone{F} \\
\BB &= \sqrt{\mu\epsilon} \gpgradeone{-I F}
\end{aligned}
\end{equation}


From this the transverse electric and magnetic fields may be picked off using the projective grade selection operations of \eqnref{eqn:transverseField:8}, and are

\begin{equation}\label{eqn:transverseField:68}
\begin{aligned}
\BE_t &= \frac{i}{\mu\epsilon\frac{\omega^2}{c^2} -k^2} \left( \pm k \spacegrad_t E_z - \frac{\omega}{c} \zcap \cross \spacegrad_t B_z \right) \\
\BB_t &= \frac{i}{\mu\epsilon\frac{\omega^2}{c^2} -k^2} \left( {\mu\epsilon}\frac{\omega}{c} \zcap \cross \spacegrad_t E_z \pm k \spacegrad_t B_z \right)
\end{aligned}
\end{equation}
