%
% Copyright © 2016 Peeter Joot.  All Rights Reserved.
% Licenced as described in the file LICENSE under the root directory of this GIT repository.
%
The proof is straightforward, but also mechanical.
Start by expanding the wedge and dot products within a grade selection operator
\begin{dmath}\label{eqn:wedgeDistributionIdentityProblemsWedgeDistributionIdentityProof:1460}
A_s \cdot \lr{ \Bb \wedge B_r }
=
\gpgrade{A_s (\Bb \wedge B_r)}{s - (r + 1)}
=
\inv{2} \gpgrade{A_s \lr{\Bb B_r + (-1)^{r} B_r \Bb} }{s - (r + 1)}.
\end{dmath}

Solving for \(B_r \Bb\) in
\begin{dmath}\label{eqn:wedgeDistributionIdentityProblemsWedgeDistributionIdentityProof:1480}
2 \Bb \cdot B_r = \Bb B_r - (-1)^{r} B_r \Bb,
\end{dmath}
we have
\begin{dmath}\label{eqn:wedgeDistributionIdentityProblemsWedgeDistributionIdentityProof:1500}
A_s \cdot \lr{ \Bb \wedge B_r }
=
\inv{2} \gpgrade{ A_s \Bb B_r + A_s \lr{ \Bb B_r - 2 \Bb \cdot B_r } }{s - (r + 1)}
=
\gpgrade{ A_s \Bb B_r }{s - (r + 1)}
-
\cancel{\gpgrade{ A_s \lr{ \Bb \cdot B_r } }{s - (r + 1)}}.
\end{dmath}

The last term above is zero since we are selecting the \(s - r - 1\) grade element of a multivector with grades \(s - r + 1\) and \(s + r - 1\), which has no terms for \(r > 0\).
Now we can expand the \(A_s \Bb\) multivector product, for
\begin{dmath}\label{eqn:wedgeDistributionIdentityProblemsWedgeDistributionIdentityProof:1520}
A_s \cdot \lr{ \Bb \wedge B_r }
=
\gpgrade{ \lr{ A_s \cdot \Bb + A_s \wedge \Bb} B_r }{s - (r + 1)}.
\end{dmath}

The latter multivector (with the wedge product factor) above has grades \(s + 1 - r\) and \(s + 1 + r\), so this selection operator finds nothing.
This leaves
\begin{dmath}\label{eqn:wedgeDistributionIdentityProblemsWedgeDistributionIdentityProof:1540}
A_s \cdot \lr{ \Bb \wedge B_r }
=
\gpgrade{
\lr{ A_s \cdot \Bb } \cdot B_r
+ \lr{ A_s \cdot \Bb } \wedge B_r
}{s - (r + 1)}.
\end{dmath}

The first dot products term has grade \(s - 1 - r\) and is selected, whereas the wedge term has grade \(s - 1 + r \ne s - r - 1\) (for \(r > 0\)), which completes the proof.
%.  \(\qedmarker\)

%Next consider an expansion that we cannot do above, but require
