%
% Copyright © 2016 Peeter Joot.  All Rights Reserved.
% Licenced as described in the file LICENSE under the root directory of this GIT repository.
%
\makeproblem{Wedge of three vectors}{problem:gradethreeselectionWedge:wedgeThree}{
Show that
\begin{dmath}\label{eqn:gradethreeselectionWedge:700}
\gpgradethree{ \Ba \Bb \Bc }
=
\Ba \wedge ( \Bb \wedge \Bc )
=
(\Ba \wedge \Bb) \wedge \Bc
=
-
\Bb \wedge (\Ba \wedge \Bc).
\end{dmath}

Observe that is antisymmetric in any two vectors, and thus completely antisymmetric (i.e. associative).  This allows the grade three selection of any three vectors to be written more simply as

\boxedEquation{eqn:gradethreeselectionWedge:720}{
\gpgradethree{ \Ba \Bb \Bc }
=
\Ba \wedge \Bb \wedge \Bc.
}

This can be considered the definition of \( \Ba \wedge \Bb \wedge \Bc \).

Some authors will define the wedge product of \( m \) vectors as

\begin{equation}\label{eqn:gradethreeselectionWedge:820}
\Bx_1 \wedge \Bx_2 \wedge \cdots \wedge \Bx_m
= \inv{m!} \sum \Bx_{i_1} \Bx_{i_2} \cdots \Bx_{i_m} \sgn(\pi(i_1 i_2 \cdots i_m)),
\end{equation}

where \(\sgn(\pi(\cdots))\) is the sign of the permutation of the indices.  With focus on \R{3}, such a definition is not required.  A reader interested in the \R{N} case should demonstrate from the axioms and definitions that
\( \gpgrade{ \Bx_1 \Bx_2 \cdots \Bx_m}{m} \) expands as specified in \cref{eqn:gradethreeselectionWedge:820}.
} % problem

\makeanswer{problem:gradethreeselectionWedge:wedgeThree}{
Consider an expansion first in products of \( \Ba, \Bb \)

\begin{dmath}\label{eqn:gradethreeselectionWedge:740}
\gpgradethree{ \Ba \Bb \Bc }
=
\gpgradethree{ (\cancel{\Ba \cdot \Bb} + \Ba \wedge \Bb) \Bc }
=
\gpgradethree{ (\Ba \wedge \Bb) \Bc }.
\end{dmath}

The dot product was killed since it leaves only a vector product within the grade selection operator.  Since a vector bivector product can have only grade 1 and grade three terms (example: \( \Be_1 (\Be_1 \wedge \Be_2) = \Be_2, \Be_1 (\Be_2 \wedge \Be_3) = \Be_1 \Be_2 \Be_3 \), this leaves just

\begin{dmath}\label{eqn:gradethreeselectionWedge:760}
\gpgradethree{ \Ba \Bb \Bc }
=
(\Ba \wedge \Bb) \wedge \Bc.
\end{dmath}

Similarly, expanding the \( \Bb \Bc \) product gives
\begin{dmath}\label{eqn:gradethreeselectionWedge:780}
\gpgradethree{ \Ba \Bb \Bc }
=
\gpgradethree{ \Ba (\cancel{\Bb \cdot \Bc} + \Ba \wedge \Bc) }
=
\gpgradethree{ \Ba (\Bb \wedge \Bc) }
=
\Ba \wedge (\Bb \wedge \Bc),
\end{dmath}

and finally, expanding products of \( \Ba \Bc \) after commutation

\begin{dmath}\label{eqn:gradethreeselectionWedge:800}
\gpgradethree{ \Ba \Bb \Bc }
=
\gpgradethree{ (\cancel{2 \Bb \cdot \Ba} - \Bb \Ba)
\Bc}
=
-\gpgradethree{ \Bb \Ba \Bc }
=
-\gpgradethree{ \Bb (\cancel{\Ba \cdot \Bc} + \Ba \wedge \Bc) }
=
- \Bb \wedge (\Ba \wedge \Bc).
\end{dmath}
} % answer
