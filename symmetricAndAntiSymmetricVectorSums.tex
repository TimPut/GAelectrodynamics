Because the wedge product is completely antisymmetric, it must be true that

\begin{dmath}\label{eqn:SimpleProducts2:580}
\By \wedge \Bx = -\Bx \wedge \By,
\end{dmath}

so
\begin{dmath}\label{eqn:SimpleProducts2:600}
\By \Bx
= \By \cdot \Bx + \By \wedge \Bx
= \Bx \cdot \By - \Bx \wedge \By.
\end{dmath}

Taken together \cref{eqn:SimpleProducts2:540} and \cref{eqn:SimpleProducts2:600} allow for a construction of a coordinate free form of both the dot and wedge products

%\begin{dmath}\label{eqn:SimpleProducts2:620}
\boxedEquation{eqn:SimpleProducts2:620}{
\begin{aligned}
\Bx \cdot \By   &= \inv{2}\lr{ \Bx \By + \By \Bx } \\
\Bx \wedge \By  &= \inv{2}\lr{ \Bx \By - \By \Bx }.
\end{aligned}
}
%\end{dmath}

These highlight the symmetric and antisymmetric nature of the respective dot and wedge products.
Some authors will use \cref{eqn:SimpleProducts2:620} as the definitions of the dot and wedge products instead of defining them in terms of grade selection.
Grade selection is preferred here since it allows for a generalization of the wedge product to multiple vectors in higher degree spaces in a particularily simple way, and also allows for the generalization of the dot and wedge products with higher order geometric structures to be discussed.


