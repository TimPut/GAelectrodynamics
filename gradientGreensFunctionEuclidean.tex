%
% Copyright � 2016 Peeter Joot.  All Rights Reserved.
% Licenced as described in the file LICENSE under the root directory of this GIT repository.
%
\index{Green's function}

\subsection{Definition.}
Green's functions allow for convolution solution of linear differential operators.  Given an operator

\begin{equation}\label{eqn:gradientGreensFunctionEuclidean:20}
L' F(\Bx') = M(\Bx'),
\end{equation}

the operator \( L \) can be inverted, yielding a solution
\begin{equation}\label{eqn:gradientGreensFunctionEuclidean:40}
F(\Bx) = \int_V dV' G(\Bx, \Bx') M(\Bx'),
\end{equation}

where \( G(\Bx, \Bx') \) is called the Green's function.

\subsection{Green's functions for electrodynamics.}

There are a few Green's functions of interest for electrodynamics, the Green's function for the Laplacian\( \spacegrad^2 \) , the gradient \( \spacegrad \) , the Helmholtz operator \(\spacegrad^2 + k^2 \), and a for the (linear) factors of the Helmholtz operator \( \spacegrad \pm j k \).  The Green's functions for the operators that are linear in the gradient are multivector functions and will be applied to multivector functions.

\index{Laplacian!Green's function}
\index{Green's function!Laplacian representation}

\maketheorem{Laplacian Green's function.}{thm:gradientGreensFunctionEuclidean:60}{

The Green's function for the \R{3} Laplacian is
\begin{equation*}
G(\Bx, \Bx') =
-\inv{4 \pi} \inv{\Norm{\Bx - \Bx'}}.
\end{equation*}
} % theorem

This representation is proven in \cref{chap:greensFunctionLaplacian}.

\index{gradient!Green's function}
\index{Green's function!gradient representation}
%The Green's function for the Laplacian of the Green's function is well known, but there are advantages to utilizing the Green's function for first order gradient equations directly.

\maketheorem{Green's function for the gradient}{thm:gradientGreensFunctionEuclidean:1}{

%A sufficient convergence constraint for \( F \) over the infinite sphere is
For a
multivector function \( F \) in a Euclidean space that tends to zero at infinity at least as fast as
\begin{equation*}
\lim_{\Bx' \rightarrow \infty} \frac{\Norm{F(\Bx')}}{\Norm{\Bx - \Bx'}^{n-1}} \rightarrow 0,
\end{equation*}

the
first order multivector gradient equation
\begin{equation*}
\spacegrad' F(\Bx') = M(\Bx'),
\end{equation*}

has an inverse
\begin{equation*}
F(\Bx) = \int_V dV' G(\Bx,\Bx') M(\Bx').
\end{equation*}

Here \( G \) is the Green's function for the gradient over an infinite spherical boundary, satisfying
\begin{equation*}
   \spacegrad G = \spacegrad \cdot G = \delta(\Bx - \Bx'),
\end{equation*}

and is given by
\begin{equation*}
   G(\Bx, \Bx') = \inv{S_n} \frac{\Bx - \Bx'}{\Norm{\Bx-\Bx'}^n}.
\end{equation*}

Here \( n \) is the dimension of the space, \( S_n \) is the area of the unit sphere.
} % theorem

Conventions for the sign of Green's functions or the parameters in the convolution integral often vary.
The sign convention used here is that of \citep{doran2003gap}.

%This result applies not only to gradient equations in Euclidean spaces, but also to multivector (or even just vector) fields \( F \), instead of the usual scalar functions that we usually apply Green's functions to.
Note that the statement that
\( \spacegrad G = \spacegrad \cdot G \), or \( \rspacegrad G = G \lspacegrad \), implies that the curl of the Green's function must be zero.

The proof of this theorem will be left to \cref{chap:gradientGreensFunctionProof}.

\maketheorem{Green's function for the Helmholtz operator.}{thm:gradientGreensFunctionEuclidean:3}{
The Green's function for the Helmholze operator \( \spacegrad^2 + k^2 \) is
\begin{equation*}
G(\Bx, \Bx') = -\frac{e^{\pm j k \Norm{ \Bx - \Bx' } }}{ 4 \pi \Norm{\Bx - \Bx'}}.
\end{equation*}
With a time harmonic dependence \( e^{j \omega t} \) we want the negative sign variation for causal solutions, or
%\( r = \Bx - \Bx' \), that is
\begin{equation*}
G(\Bx, \Bx') = -\frac{e^{-j k \Norm{\Bx - \Bx'}}}{ 4 \pi \Norm{\Bx - \Bx'}}.
\end{equation*}
} % theorem

%}
