%
% Copyright � 2016 Peeter Joot.  All Rights Reserved.
% Licenced as described in the file LICENSE under the root directory of this GIT repository.
%

The Green's function for the Laplacian of the Green's function is well known, but there are advantages to utilizing the Green's function for first order gradient equations directly.

\maketheorem{Green's function for the gradient}{thm:gradientGreensFunctionEuclidean:1}{

%A sufficient convergence constraint for \( F \) over the infinite sphere is
For a
multivector function \( F \) in a Euclidean space that tends to zero at infinity at least as fast as

\begin{equation*}
\lim_{\Bx' \rightarrow \infty} \frac{\Norm{F(\Bx')}}{\Norm{\Bx - \Bx'}^{n-1}} \rightarrow 0,
\end{equation*}

the
first order multivector gradient equation

\begin{equation*}
\spacegrad' F(\Bx') = M(\Bx'),
\end{equation*}

has an inverse

\begin{equation*}
F(\Bx) = \int_V dV' G(\Bx,\Bx') M(\Bx').
\end{equation*}

Here \( G \) is the Green's function for the gradient over an infinite spherical boundary, satisfying

\begin{equation*}
   \spacegrad G = \spacegrad \cdot G = \delta(\Bx - \Bx'),
\end{equation*}

and is given by

\begin{equation*}
   G(\Bx, \Bx') = \inv{S_n} \frac{\Bx - \Bx'}{\Norm{\Bx-\Bx'}^n}.
\end{equation*}

Here \( n \) is the dimension of the space, \( S_n \) is the area of the unit sphere.
} % theorem

Conventions for the sign of Green's functions or the parameters in the convolution integral often vary.  The sign convention used here is that of \citep{doran2003gap}.

%This result applies not only to gradient equations in Euclidean spaces, but also to multivector (or even just vector) fields \( F \), instead of the usual scalar functions that we usually apply Green's functions to.
The statement that
\( \spacegrad G = \spacegrad \cdot G \), or \( \rspacegrad G = G \lspacegrad \), implies that the curl of the Green's function must be zero, which is straightforward to check for all points \( \Bx \ne \Bx' \)

\begin{dmath}\label{eqn:gradientGreensFunctionEuclidean:140}
S_n (\spacegrad \wedge G)
=
\frac{\spacegrad \wedge (\Bx-\Bx')}{\Norm{\Bx - \Bx'}^{n}}
+
\spacegrad \inv{\Norm{\Bx - \Bx'}^{n}} \wedge (\Bx-\Bx')
=
\frac{\spacegrad \wedge (\Bx-\Bx')}{\Norm{\Bx - \Bx'}^{n}}
- \cancel{n
\frac{\Bx - \Bx'}{\Norm{\Bx - \Bx'}^{n}} \wedge (\Bx-\Bx')}.
\end{dmath}

However,

\begin{dmath}\label{eqn:gradientGreensFunctionEuclidean:160}
\spacegrad \wedge (\Bx-\Bx')
=
\spacegrad \wedge \Bx
=
\sum_{k,j} e_k \wedge e_j \partial_k \Bx_j
=
\sum_{k,j} e_k \wedge e_j \delta_{k j}
=
\sum_{k} e_k \wedge e_k
=
0.
\end{dmath}

Next consider the divergence of the Green's function for points \( \Bx \ne \Bx' \).

\begin{dmath}\label{eqn:gradientGreensFunctionEuclidean:120}
S_n \spacegrad \cdot G
=
\spacegrad \cdot \frac{\Bx - \Bx'}{\Norm{\Bx - \Bx'}^n}
=
\lr{ \frac{\spacegrad \cdot \lr{\Bx - \Bx'}}{\Norm{\Bx - \Bx'}^{n}} + \lr{ \spacegrad \inv{\Norm{\Bx - \Bx'}^{n}} } \cdot \lr{ \Bx - \Bx'} }.
\end{dmath}

However, the gradient of \( 1/\Norm{\Bx - \Bx'}^m \), for points \( \Bx \ne \Bx' \) is

\begin{dmath}\label{eqn:gradientGreensFunctionEuclidean:100}
\spacegrad \inv{\Norm{\Bx - \Bx'}^m}
=
-\frac{m}{2} \sum_{k,j} \frac{e_k \partial_k (\Bx_j - \Bx_j')(\Bx_j - {\Bx_j}')}{\Norm{\Bx - \Bx'}^{m+2}}
=
-\frac{m}{2} 2 \frac{e_k (\Bx_j - \Bx_j') \delta_{k j} }{\Norm{\Bx - \Bx'}^{m+2}}
=
-m \frac{ \Bx - \Bx'}{\Norm{\Bx - \Bx'}^{m+2}}.
\end{dmath}

so everywhere that \( \Bx \ne \Bx' \)

\begin{dmath}\label{eqn:gradientGreensFunctionEuclidean:340}
S_n \spacegrad \cdot G
=
\frac{n}{\Norm{\Bx - \Bx'}^{n}} + \lr{ -n \frac{\Bx - \Bx'}{\Norm{\Bx - \Bx'}^{n+2} } \cdot \lr{ \Bx - \Bx'} }
=
0.
\end{dmath}

Finally, the Fundamental Theorem of (Geometric) Calculus, is applied to the
weighted convolution of the Green's function with the
(bidirectionally acting) gradient

\begin{dmath}\label{eqn:gradientGreensFunctionEuclidean:60}
   \int_V G(\Bx,\Bx') d^n \Bx' \lrgrad' F(\Bx')
=
\int_V dV' \lr{G(\Bx,\Bx') I \lgrad'} F(\Bx')
+
\int_V dV' G(\Bx,\Bx') I \lr{ \rgrad' F(\Bx') }
= \int_{\partial V} G(\Bx,\Bx') d^{n-1} \Bx' F(\Bx').
\end{dmath}

The boundary integral tends to zero.  To show this,
let \( d^n \Bx' = dV' I \), \( d^{n-1} \Bx' \Bn = dA' I \), where \( \Bn = \Bn(\Bx') \) is the outward unit normal to the area element \( d^{n-1} \Bx' \).
The
pseudoscalar \( I \) either commutes or anticommutes with vectors according to the dimension of the space.  In particular

\begin{dmath}\label{eqn:gradientGreensFunctionEuclidean:360}
\begin{aligned}
\Bn I &= (-1)^{n-1} I \Bn \\
G I &= (-1)^{n-1} I G,
\end{aligned}
\end{dmath}

so

\begin{dmath}\label{eqn:gradientGreensFunctionEuclidean:80}
\int_{\partial V} G(\Bx,\Bx') d^{n-1} \Bx' F(\Bx')
= \int_{\partial V} G(\Bx,\Bx') I \inv{I} d^{n-1} \Bx' F(\Bx')
= \int_{\partial V} G(\Bx,\Bx') I \inv{I} \Bn dA' I F(\Bx')
= (-1)^{2(n-1)} I \int_{\partial V} dA' G(\Bx,\Bx') \Bn F(\Bx')
= I \int_{\partial V} dA' G(\Bx,\Bx') \Bn F(\Bx').
\end{dmath}

The absolute limit of this boundary integral is therefore bounded by

\begin{dmath}\label{eqn:gradientGreensFunctionEuclidean:380}
S_n \Norm{ \int_{\partial V} G(\Bx,\Bx') d^{n-1} \Bx' F(\Bx') }
= S_n \Norm{ \int_{\partial V} dA' G(\Bx,\Bx') \Bn F(\Bx') }
\le S_n \int_{\partial V} dA' \Norm{ G(\Bx,\Bx') \Bn F(\Bx') }
= \int_{\partial V} dA' \Norm{ \frac{\Bx -\Bx'}{\Norm{\Bx - \Bx'}^n} \Bn F(\Bx') }
\le \int_{\partial V} dA' \Norm{ \frac{\Bx -\Bx'}{\Norm{\Bx - \Bx'}^n} F(\Bx') }
\le \int_{\partial V} dA' \frac{ \Norm{F(\Bx')}}{\Norm{\Bx - \Bx'}^{n-1} },
\end{dmath}

so a sufficient limit on \( F \) to kill off this surface integral is given by

\begin{dmath}\label{eqn:gradientGreensFunctionEuclidean:200}
\lim_{\Bx' \rightarrow \infty}
\frac{ \Norm{F(\Bx')}}{\Norm{\Bx - \Bx'}^{n-1} } \rightarrow 0.
\end{dmath}

This leaves

\begin{dmath}\label{eqn:gradientGreensFunctionEuclidean:220}
\int_V G(\Bx,\Bx') d^n \Bx' \lr{ \rgrad' F(\Bx') }
=
-\int_V \lr{G(\Bx,\Bx') d^n \Bx' \lgrad'} F(\Bx').
\end{dmath}

The RHS is zero everywhere that \( \Bx \ne \Bx' \) so it can be restricted to a spherical ball around \( \Bx \).

\begin{dmath}\label{eqn:gradientGreensFunctionEuclidean:240}
\int_V \lr{G(\Bx,\Bx') d^n \Bx' \lgrad'} F(\Bx')
=
\lim_{\epsilon \rightarrow 0}
\int_{\Norm{\Bx - \Bx'} < \epsilon} \lr{G(\Bx,\Bx') d^n \Bx' \lgrad'} F(\Bx')
=
\lr{ \lim_{\epsilon \rightarrow 0}
\lr{ 
\int_{\Norm{\Bx - \Bx'} < \epsilon} \lr{G(\Bx,\Bx') d^n \Bx' \lgrad'}
F(\overbar{\Bx})
}
} 
\rightarrow 
\lr{ \lim_{\epsilon \rightarrow 0}
\lr{ 
\int_{\Norm{\Bx - \Bx'} < \epsilon} \lr{G(\Bx,\Bx') d^n \Bx' \lgrad'}
}
} 
F(\Bx)
\end{dmath}

where it was assumed that \( G \) has sufficient continuity for the mean value theorem to allow for a point \( \overbar{\Bx} \) in this infinitesimal hyperspherical region to be selected.
With an additional assumption that \( F \) is sufficiently well behaved in this region to be pulled out of the limit, 
the Fundamental theorem can be applied over the infinitesimal hypersphere centred on \( \Bx \)

\begin{dmath}\label{eqn:gradientGreensFunctionEuclidean:400}
\int_V \lr{G(\Bx,\Bx') d^n \Bx' \lgrad'} F(\Bx')
=
\lr{ \lim_{\epsilon \rightarrow 0}
\int_{\Norm{\Bx - \Bx'} = \epsilon} G(\Bx,\Bx') d^{n-1} \Bx'
} F(\Bx)
=
(-1)^{n-1} I
\lr{ \lim_{\epsilon \rightarrow 0}
\int_{\Norm{\Bx - \Bx'} = \epsilon} G(\Bx,\Bx') dA' I \Bn
} F(\Bx)
=
(-1)^{n-1} I
\lr{ \lim_{\epsilon \rightarrow 0}
\int_{\Norm{\Bx - \Bx'} = \epsilon} dA' G(\Bx,\Bx') \Bn
} F(\Bx)
=
(-1)^{n-1} I
\lr{ \lim_{\epsilon \rightarrow 0}
\int_{\Norm{\Bx - \Bx'} = \epsilon} dA' \inv{S_n} \frac{ -\Bn }{ \epsilon^n} \Bn
} F(\Bx)
=
(-1)^{n} I
\lim_{\epsilon \rightarrow 0}
\frac{F(\Bx)}{S_n \epsilon^{n-1}}
\int_{\Norm{\Bx - \Bx'} = \epsilon} dA'
=
(-1)^{n} I
\lim_{\epsilon \rightarrow 0}
\frac{F(\Bx)}{S_n \epsilon^{n-1}}
S_n \epsilon^{n-1}
=
(-1)^{n} I
F(\Bx).
\end{dmath}

Rearranging for \( F \) provides the desired result
%This is essentially a divergence integral without assuming that the gradient commutes with the Green's function in this infinitesimal region.  

\begin{dmath}\label{eqn:gradientGreensFunctionEuclidean:280}
F(\Bx) 
= 
(-1)^{n-1} \inv{I}
\int_V G(\Bx,\Bx') \lr{ \spacegrad' d^n \Bx' F(\Bx') }
= 
\int_V dV' G(\Bx,\Bx') \lr{ \spacegrad' F(\Bx') }
= 
\int_V dV' G(\Bx,\Bx') M(\Bx').
\end{dmath}

%}
