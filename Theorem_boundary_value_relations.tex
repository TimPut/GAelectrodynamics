%
% Copyright � 2018 Peeter Joot.  All Rights Reserved.
% Licenced as described in the file LICENSE under the root directory of this GIT repository.
%
\maketheorem{Boundary value relations.}{thm:boundarySurfaceSources:480}{
The difference in the normal and tangential components of the electromagnetic field spanning a surface on which there are
a surface current or surface charge or current densities \( J_\txte = J_{\textrm{es}} \delta(n), J_\txtm = J_{\textrm{ms}} \delta(n) \)
can be related to those surface sources as follows
%\label{eqn:boundarySurfaceSources:420}
\begin{equation*}
\begin{aligned}
\gpgrade{\ncap (F_2 - F_1) }{0,1} &= J_{\textrm{es}} \\
\gpgrade{\ncap (G_2 - G_1) }{2,3} &= I J_{\textrm{ms}},
\end{aligned}
\end{equation*}
where \( F_k = \BD_k + I \BH_k/c, G_k = \BE_k + I c \BB_k, k = 1,2 \) are the fields in the
where \( \ncap = \ncap_2 = -\ncap_1 \) is the outwards facing normal in the second medium.
In terms of the conventional constituent fields, these may be written
%\label{eqn:boundarySurfaceSources:460}
\begin{equation*}
\begin{aligned}
\ncap \cdot \lr{ \BD_2 - \BD_1 } &= \rho_\txts \\
\ncap \cross \lr{ \BH_2 - \BH_1 } &= \BJ_\txts \\
\ncap \cdot \lr{ \BB_2 - \BB_1 } &= \rho_{\textrm{ms}} \\
\ncap \cross \lr{ \BE_2 - \BE_1 } &= -\BM_\txts.
\end{aligned}
\end{equation*}
} % theorem
