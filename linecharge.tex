%
% Copyright © 2017 Peeter Joot.  All Rights Reserved.
% Licenced as described in the file LICENSE under the root directory of this GIT repository.
%
%\makeexample{Line charge.}{example:linecharge:linecharge}{
In this example the electric field is calculated at a point on the z-axis, due to a line charge density of \( \lambda \) along a segment \( [a,b] \) of the x-axis.  This is illustrated in \cref{fig:linecharge:linechargeFig1}.
\imageFigure{../figures/GAelectrodynamics/linechargeFig1}{Line charge density.}{fig:linecharge:linechargeFig1}{0.3}

Introducing a unit imaginary \( i = \Be_{13} \) for the rotation from the x-axis to the z-axis, the field point observation point is

\begin{dmath}\label{eqn:linecharge:120}
\Bx = r \Be_1 e^{i \theta}.
\end{dmath}

The charge element point is \( \Bx' = x \Be_1 \), so the difference can now be written with \( \Be_1 \) factored to the left or to the right

\begin{equation}\label{eqn:linecharge:20}
\Bx - \Bx'
= \Be_1\lr{ r e^{i\theta} - x }
= \lr{ r e^{-i\theta} - x } \Be_1.
\end{equation}

This allows the squared vector length to be calculated as a product of complex conjugates

\begin{dmath}\label{eqn:linecharge:40}
\lr{ \Bx - \Bx' }^2
= \lr{ r e^{-i\theta} - x } \Be_1 \Be_1\lr{ r e^{i\theta} - x }
= \lr{ r e^{-i\theta} - x } \lr{ r e^{i\theta} - x }
= r^2 + x^2 - r x \lr{ e^{i\theta} + e^{-i\theta} }
= r^2 + x^2 - 2 r x \cos\theta.
\end{dmath}

The total electric field is therefore
\begin{dmath}\label{eqn:linecharge:60}
\BE
= \frac{\lambda}{4 \pi \epsilon_0} \int_a^b dx \frac{ r \Be_1 e^{i\theta} - x \Be_1 }{ \lr{ r^2 + x^2 - 2 x r \cos\theta }^{3/2} }
= \frac{\lambda \Be_1}{4 \pi \epsilon_0 r} \int_{a/r}^{b/r} du \frac{ e^{i\theta} - u }{ \lr{ 1 + u^2 - 2 u \cos\theta }^{3/2} }.
\end{dmath}

This integral can be evaluated by table lookup or using tools like Mathematica.  For \( \theta = \pi/2 \)

\begin{dmath}\label{eqn:linecharge:80}
\int
du \frac{ e^{i\theta} - u }{ \lr{ 1 + u^2 - 2 u \cos\theta }^{3/2} }
= \frac{1 + i u}{\sqrt{1 + u^2}},
\end{dmath}

and for other angles \( \theta \neq n \pi/2 \)

\begin{dmath}\label{eqn:linecharge:100}
\int
du \frac{ e^{i\theta} - u }{ \lr{ 1 + u^2 - 2 u \cos\theta }^{3/2} }
= \frac{(1 -u e^{-i\theta}) \sqrt{1 + u^2 - 2 u \cos\theta}}{(1 + u^2) \sin(2\theta)}.
\end{dmath}

The interesting takeaway is not the form of the solution, but the fact that GA allows an introduction of a ``complex plane'' for many problems that have polar representations in a plane.  When such a complex plane is introduced,
existing Computer Algebra Systems (CAS), like Mathematica, can be utilized for the grunt work of the evaluation.

Observe that the numerator factors like \( \Be_1 (1 + i u) \) and \( \Be_1(1 - u e^{-i\theta}) \)
compactly describe the direction of the vector field at the observation point.  Either of these can be expanded explicitly in sines and cosines if desired

\begin{dmath}\label{eqn:linecharge:140}
\begin{aligned}
\Be_1 (1 + i u) &= \Be_1 + u \Be_3 \\
\Be_1(1 - u e^{-i\theta}) &= \Be_1(1 - u \cos\theta) + u \Be_3 \sin\theta.
\end{aligned}
\end{dmath}

%} % example
