%
% Copyright © 2018 Peeter Joot.  All Rights Reserved.
% Licenced as described in the file LICENSE under the root directory of this GIT repository.
%
%{
   \chapter{Electromagnetism.}
      \section{Conventional formulation.}
         %
% Copyright © 2017 Peeter Joot.  All Rights Reserved.
% Licenced as described in the file LICENSE under the root directory of this GIT repository.
%
Maxwell's equations provide an abstraction, the field, that aggregates the effects of an arbitrary electric charge and current
distribution on a ``test'' charge distribution.
The test charge is assumed to be small and isolated enough that it does not also appreciably change the fields themselves.
Once the fields are determined, the Lorentz force equation can be used to determine the dynamics of the
test particle.
These dynamics can be determined without having to
compute all the interactions of that charge with all the charges and currents in space, nor having to continually account for
the interactions of those charge with each other.

We will use vector differential form of Maxwell's equations with antenna theory extensions (fictious magnetic sources) as our starting point

\index{magnetic charge density}
\index{electric charge density}
\index{magnetic current density}
\index{electric current density}

\begin{subequations}
\label{eqn:freespace:3399}
\begin{dmath}\label{eqn:freespace:3100}
%\spacegrad \cross \BE = - \PD{t}{\BB}
\spacegrad \cross \BE = - \BM - \PD{t}{\BB}
\end{dmath}
\begin{dmath}\label{eqn:freespace:3120}
%\spacegrad \cross \BB = \mu_0 \lr{ \BJ + \epsilon_0 \PD{t}{\BE} }
\spacegrad \cross \BH = \BJ + \PD{t}{\BD}
\end{dmath}
\begin{dmath}\label{eqn:freespace:3140}
%\spacegrad \cdot \BE = \frac{\rho}{\epsilon_0}
\spacegrad \cdot \BD = \rho
\end{dmath}
\begin{dmath}\label{eqn:freespace:3160}
%\spacegrad \cdot \BB = 0.
\spacegrad \cdot \BB = \rho_\txtm.
\end{dmath}
\end{subequations}

These equations relate the primary electric and magnetic fields

\begin{itemize}
	\item \( \BE(\Bx, t) \) : Electric field intensity [\si{V/m}] (Volts/meter)
	\item \( \BH(\Bx, t) \) : Magnetic field intensity [\si{A/m}] (Amperes/meter),
\end{itemize}

and the induced electric and magnetic fields

\begin{itemize}
	\item \( \BD(\Bx, t) \) : Electric flux density (or displacement vector) [\si{C/m}] (Coulombs/meter)
	\item \( \BB(\Bx, t) \) : Magnetic flux density [\si{W/m^2}] (Webers/square meter),
\end{itemize}

to the charge densities

\begin{itemize}
	\item \( \rho(\Bx, t) \) : Electric charge density [\si{C/m^3}] (Coulombs/cubic meter)
   \item \( \rho_\txtm(\Bx, t) \) : Magnetic charge density [\si{W/m^3}] (Webers/cubic meter),
\end{itemize}

and the current densities

\begin{itemize}
	\item \( \BJ(\Bx, t) \) : Electric current density [\si{A/m^2}] (Amperes/square meter),
   \item \( \BM(\Bx, t) \) : Magnetic current density [\si{V/m^2}] (Volts/square meter).
\end{itemize}

All of the fields and sources can vary in space and time, and are specified here in SI units.
The sources \( \BM, \rho_\txtm \) can be considered fictional, representing physical phenomena such as infinitesimal current loops.

In general, the relationship between the electric and magnetic fields (constitutivity relationships) may be complicated
non-isotropic tensor operators, functions of all of \( \BE, \BD, \BB \) and \( \BH \).
It will often be useful to assume that these constitutive relationships between the electric and magnetic fields are independent

\begin{subequations}
\label{eqn:freespace:300}
\begin{dmath}\label{eqn:freespace:320}
\BB = \mu \BH
\end{dmath}
\begin{dmath}\label{eqn:freespace:340}
\BD = \epsilon \BE,
\end{dmath}
\end{subequations}

where \( \epsilon = \epsilon_r \epsilon_0 \) is the permittivity of the medium [\si{F/m}] (Farads/meter), and \( \mu = \mu_r \mu_0 \) is the permeability of the medium [\si{H/m}] (Henries/meter).
The permittivity and permeability may be functions of both time and position, and model the materials that the fields are propagating through.
In free space \( \mu_r = 1 \) and \( \epsilon_r = 1\) so these relationships are simply

\begin{subequations}
\label{eqn:freespace:301}
\begin{dmath}\label{eqn:freespace:321}
\BB = \mu_0 \BH
\end{dmath}
\begin{dmath}\label{eqn:freespace:341}
\BD = \epsilon_0 \BE,
\end{dmath}
\end{subequations}

where

\begin{itemize}
\item \( \epsilon_0 = 8.85 \times 10^{-12} \si{C^2/N/m^2}\) : Permittivity of free space (Coulombs squared/Newton/square meter)
\item \( \mu_0 = 4 \pi \times 10^{-7} \si{N/A^2}\) : Permeability of free space (Newtons/Ampere-squared).
\end{itemize}

These constants are related to the speed of light, \( c = 3.00 \times 10^8 \si{m/s} \) by \( \mu_0 \epsilon_0 = 1/c^2 \).

Antenna theory extends Maxwell's equations with fictional magnetic charge and current densities that are useful to model
real phenomena such as infinitesimal current loops.

The fields and sources are all real valued functions of both space and time.
In many situations it will be desirable to work with a time harmonic (frequency-domain phasor) form of Maxwell's equations.
Two conventions exist for the frequency dependence
\index{time harmonic}

\begin{subequations}
\label{eqn:freespace:900}
\begin{dmath}\label{eqn:freespace:20}
\BY(\Bx, t) = \Real( \BY(\Bx, \omega) e^{j\omega t} )
\end{dmath}
\begin{dmath}\label{eqn:freespace:40}
\BY(\Bx, t) = \Real( \BY(\Bx, \omega) e^{-i\omega t} ).
\end{dmath}
\end{subequations}

\Cref{eqn:freespace:20} is the engineering convention (used herein), and \cref{eqn:freespace:40} is the physics convention.
Care is required by the reader to understand which form of frequency dependence has been assumed.
In either representation the field (or source) \( \BY(\Bx, \Bomega) \) is allowed to be complex valued, but we still require the basis to be Euclidean, and do not require complex inner product spaces.  In such a representation, the square of a vector is still a scalar, but may be complex valued.
%%%Given this frequency dependence Maxwell's equations take the form
%%%
%%%%\input{../ece1229-antenna/MaxwellsTimeHarmonic.tex}
%%%\begin{subequations}
%%%\label{eqn:freespace:99}
%%%\begin{dmath}\label{eqn:freespace:100}
%%%\spacegrad \cross \BE = - \BM - j \omega \BB
%%%\end{dmath}
%%%\begin{dmath}\label{eqn:freespace:120}
%%%\spacegrad \cross \BH = \BJ + j \omega \BD
%%%\end{dmath}
%%%\begin{dmath}\label{eqn:freespace:140}
%%%\spacegrad \cdot \BD = \rho
%%%\end{dmath}
%%%\begin{dmath}\label{eqn:freespace:160}
%%%\spacegrad \cdot \BB = \rho_\txtm.
%%%\end{dmath}
%%%\end{subequations}

Continuous models for charge and current distributions are used in Maxwell's equations, despite the
fact that charges (i.e. electrons) are particles, and are not distributed in space.
The discrete nature of electronic charge can be modelled using a delta function representation of the charge and current densities

\begin{dmath}\label{eqn:freespace:240}
\begin{aligned}
\rho(\Bx, t) &= \sum_a q_a \delta( \Bx - \Bx_a( t) ) \\
\BJ(\Bx, t) &= \sum_a q_a \Bv_a( \Bx, \Bt ).
\end{aligned}
\end{dmath}

This model is inherently non-quantum mechanical, as it assumes that it is possible to
simultaneous measure the position and velocity of an electron.
%Additionally, irrespective of an electron's wave function distribution, this model requires that all electron interactions occur at fixed points in space and time.

The dynamics of particle interaction with the fields are provided by the
Lorentz force and power equations

\begin{subequations}
\label{eqn:freespace:180}
\begin{dmath}\label{eqn:freespace:200}
\ddt{\Bp} = q \lr{ \BE + \Bv \cross \BB }
\end{dmath}
\begin{dmath}\label{eqn:freespace:220}
\ddt{\calE} = q \BE \cdot \Bv.
\end{dmath}
\end{subequations}

Both the energy and the momentum relations of \cref{eqn:freespace:180} are stated, since
the simplest (relativistic) form of the Lorentz force equation directly encodes both.
For readers unfamiliar with \cref{eqn:freespace:220}, \cref{problem:freespace:LorentzPower} provides a derivation method.
%energy and momentum are intimate partners in the context of special relativity, and Maxwell's equations themselves are inherently relativistic.

The quantities involved in the Lorentz equations are

\begin{itemize}
	\item \( \Bp(\Bx, t) \) : Test particle momentum [\si{kg\, m/s}] (Kilogram meters/second)
	\item \( \calE(\Bx, t) \) : Test particle kinetic energy [\si{J}] (Joules, kilogram meter^2/second^2)
	\item \( q \) : Test particle charge [\si{C}] (Coulombs)
	\item \( \Bv \) : Test particle velocity [\si{m/s}] (Meters/second)
\end{itemize}

The task of extracting all the physical meaning from the Maxwell and Lorentz equations is a difficult one.
Our attempt to do so will use the formalism of geometric algebra.


         \subsection{Problems.}
            %
% Copyright © 2017 Peeter Joot.  All Rights Reserved.
% Licenced as described in the file LICENSE under the root directory of this GIT repository.
%
\makeoproblem{Lorentz power and force relationship.}{problem:freespace:LorentzPower}{\S 17 \citep{landau1980classical}}{
Using the relativistic definitions of momentum and energy
\begin{equation*}
\begin{aligned}
	\Bp(\Bx, t) &= \frac{m \Bv}{\sqrt{1-\Bv^2/c^2}} \\
	\calE(\Bx, t) &= \frac{m c^2}{\sqrt{1-\Bv^2/c^2}},
\end{aligned}
\end{equation*}
show that \( d\calE/dt = \Bv \cdot d\Bp/dt \), and use this to derive
\cref{eqn:freespace:220} from \cref{eqn:freespace:200}.
} % problem

      \section{Maxwell's equation.}
         %
% Copyright © 2016 Peeter Joot.  All Rights Reserved.
% Licenced as described in the file LICENSE under the root directory of this GIT repository.
%
\index{Maxwell's equation}
We will work with a multivector representation of the fields in isotropic media satisfying the
constituency relationships from \cref{eqn:freespace:300}, and define a multivector field that includes both electric and magnetic components

\makedefinition{Electromagnetic field strength.}{dfn:isotropicMaxwells:640}{
The \textit{electromagnetic field strength} ([\si{V/m}] (Volts/meter)) is defined as
\begin{equation*}
F = \BE + I \eta \BH \quad(= \BE + I c \BB),
\end{equation*}
where
\begin{itemize}
\item \( \eta = \sqrt{\mu/\epsilon} \) (\( [\Omega] \) Ohms), is the impedance of the media.
\item \( c = 1/\sqrt{\epsilon\mu} \) ([\si{m/s}] meters/second), is the group velocity of a wave in the media.  When \( \epsilon = \epsilon_0, \mu = \mu_0 \), \( c \) is the speed of light.
\end{itemize}
\( F \) is called the \textit{F}araday by some authors.
} % definition

The factors of \( \eta \) (or \( c \)) that multiply the magnetic fields are for dimensional consistency, since \( [\sqrt{\epsilon} \BE] = [\sqrt{\mu} \BH] = [\BB/\sqrt{\mu}]\).
The justification for imposing a dual (or complex) structure on the electromagnetic field strength can be found in the historical development of
Maxwell's equations, but we will also see such a structure arise naturally in short order.

No information is lost by imposing the complex structure of
\cref{dfn:isotropicMaxwells:640}, since we can always obtain the
electric field vector \( \BE \) and the magnetic field bivector \( I \BH \) by grade selection
from the electromagnetic field strength when desired
\begin{dmath}\label{eqn:isotropicMaxwells:620}
\begin{aligned}
\BE &= \gpgradeone{ F } \\
I \BH &= \inv{\eta} \gpgradetwo{ F }.
\end{aligned}
\end{dmath}

We will also
define a multivector current containing all charge densities and current densities
\makedefinition{Multivector current.}{dfn:isotropicMaxwells:660}{
The \textit{current} ([\si{A/m^2}] (Amperes/square meter)) is defined as
\begin{equation*}
J = \eta \lr{ c \rho - \BJ } + I \lr{ c \rho_\txtm - \BM }.
\end{equation*}
} % definition
When the fictitious magnetic source terms \((\rho_\txtm, \BM)\) are included, the current has one grade for each possible source (scalar, vector, bivector, trivector).  With only conventional electric sources, the current is still a multivector, but contains only scalar and vector grades.

Given the multivector field and current, it is now possible to state Maxwell's equation (singular) in its geometric algebra form
\maketheorem{Maxwell's equation.}{dfn:isotropicMaxwells:680}{
Maxwell's equation is a multivector equation relating the change in the electromagnetic field strength to charge and current densities and is written as
\begin{equation*}
\stgrad F = J.
\end{equation*}
} % theorem
Maxwell's equation in this form will be the starting place for all the subsequent analysis in this book.
As mentioned in \cref{chap:GreensFunctions}, the operator \( \spacegrad + (1/c) \partial_t \) will be called the \textit{spacetime gradient}\footnote{This form of spacetime gradient is given a special symbol by a number of authors, but there is no general agreement on what to use.
Instead of entering the fight, it will be written it out in full in this book.}.

To prove \cref{dfn:isotropicMaxwells:680} we
first insert the
isotropic
constituency relationships from \cref{eqn:freespace:300} into
\cref{eqn:freespace:3399}, so that we are working with two field variables instead of four
\begin{dmath}\label{eqn:isotropicMaxwells:500}
\begin{aligned}
\spacegrad \cdot \BE &= \inv{\epsilon} \rho \\
\spacegrad \cross \BE &= - \BM - \mu \PD{t}{\BH} \\
\spacegrad \cdot \BH &= \inv{\mu} \rho_\txtm \\
\spacegrad \cross \BH &= \BJ + \epsilon \PD{t}{\BE}
\end{aligned}
\end{dmath}
Inserting \( \Ba = \spacegrad \) into \cref{eqn:SimpleProducts2:1640} the vector product of the gradient with another vector
\begin{dmath}\label{eqn:isotropicMaxwells:520}
\spacegrad \Bb = \spacegrad \cdot \Bb + I \spacegrad \cross \Bb.
\end{dmath}
The respective dot and cross products for \( \BE \) and \( \BH \) in
\cref{eqn:isotropicMaxwells:500}
can be grouped using \cref{eqn:isotropicMaxwells:520} into multivector gradient equations
\begin{dmath}\label{eqn:isotropicMaxwells:540}
\begin{aligned}
\spacegrad \BE &= \inv{\epsilon} \rho + I \lr{ - \BM - \mu \PD{t}{\BH} } \\
\spacegrad \BH &= \inv{\mu} \rho_\txtm + I \lr{ \BJ + \epsilon \PD{t}{\BE} }.
\end{aligned}
\end{dmath}
Multiplying the gradient equation for the magnetic field by \( \eta I \) so that both equations have the same dimensions, and so that the electric field appears in both equations as \( \BE \) and not \( I \BE \), we find
\begin{dmath}\label{eqn:isotropicMaxwells:560}
\begin{aligned}
\spacegrad \BE        + \inv{c} \PD{t}{} (I \eta \BH) &= \inv{\epsilon}\rho - I \BM  \\
\spacegrad I \eta \BH + \inv{c} \PD{t}{\BE}           &= I c \rho_\txtm - \eta \BJ,
\end{aligned}
\end{dmath}
where \( \mu/\eta = \eta \epsilon = 1/c \) was used to simplify things slightly, and all the field contributions have been moved to the left hand side.
The first multivector equation has only scalar and bivector grades, whereas the second has only vector and trivector grades.  This means that if we add these equations, we can recover each by grade selection, and no information is lost.  That sum is
\begin{dmath}\label{eqn:isotropicMaxwells:580}
\stgrad \lr{ \BE + I \eta \BH } = \eta\lr{ c \rho - \BJ } + I \lr{ c \rho_\txtm - \BM }.
\end{dmath}
Application of \cref{dfn:isotropicMaxwells:640} and \cref{dfn:isotropicMaxwells:660} to
\cref{eqn:isotropicMaxwells:580} proves the theorem, verifying the
assertion that Maxwell's equations can be consolidated into a single multivector equation.
There is a lot of information packed into this single equation.
Where possible, we want to work with the multivector form of Maxwell's equation, either in the
compact form of \cref{dfn:isotropicMaxwells:680} or the explicit form of \cref{eqn:isotropicMaxwells:580},
and not decompose Maxwell's equation into the conventional representation by grade selection operations.

\subsubsection{Problems.}
\makeproblem{Dot and cross product relation to vector product.}{problem:isotropicMaxwells:700}{
Using coordinate expansion, convince yourself of the validity of \cref{eqn:isotropicMaxwells:520}.
} % problem
\makeproblem{Extracting the conventional Maxwell's equations.}{problem:isotropicMaxwells:720}{
Apply grade 0,1,2, and 3 selection operations to \cref{eqn:isotropicMaxwells:580}.  Determine the multiplicative (scalar or trivector) constants required to obtain \cref{eqn:isotropicMaxwells:500} from the equations that result from such grade selection operations.
} % problem

      \section{Wave equation and continuity.}
         %
% Copyright � 2018 Peeter Joot.  All Rights Reserved.
% Licenced as described in the file LICENSE under the root directory of this GIT repository.
%
%{
\label{chap:continuity}
Some would argue that the conventional form \cref{eqn:freespace:3100} of Maxwell's equations have built in redundancy since continuity equations on the charge and current densities couple some of these equations.
We will take an opposing view, and show that such continuity equations are necessary consequences of Maxwell's equation in its wave equation form, and derive those conditions.
This amounts to a statement that the multivector current \( J \) is not completely unconstrained.

%It's helpful to introduce a d'Lambertian operator.  The sign convention that we will use for this operator is given by the following.
%\index{\(\dAlembertian\)}
%\makedefinition{d'Lambertian (wave equation) operator.}{dfn:continuity:190}{
%Let
%\begin{equation*}
%\dAlembertian =
%\conjstgrad
%\stgrad
%=
%\spacegrad^2 - \inv{c^2} \PDSq{t}{}.
%\end{equation*}
%} % definition
%
%
% Copyright � 2018 Peeter Joot.  All Rights Reserved.
% Licenced as described in the file LICENSE under the root directory of this GIT repository.
%
\maketheorem{Electromagnetic wave equation and continuity conditions.}{thm:continuity:600}{
The electromagnetic field is a solution to the non-homogeneous wave equation
\begin{equation*}
%\lr{ \spacegrad^2 - \inv{c^2} \PDSq{t}{} }
\dAlembertian
F =
\conjstgrad J.
\end{equation*}
In source free conditions, this reduces to a homogeneous wave equation, with group velocity \( c \), the speed of the wave in the media.
When expanded explicitly in terms of electric and magnetic fields, and charge and current densities, this single equation resolves to a
non-homogeneous wave equation for each of the electric and magnetic fields
\begin{equation*}
\begin{aligned}
%\lr{ \spacegrad^2 - \inv{c^2} \PDSq{t}{} }
\dAlembertian
\BE
%&= \gpgrade{\conjstgrad J}{1}
&= \inv{\epsilon} \spacegrad \rho + \mu \PD{t}{\BJ} + \spacegrad \cross \BM \\
%\lr{ \spacegrad^2 - \inv{c^2} \PDSq{t}{} }
\dAlembertian
\BH
%&= \inv{I \eta} \gpgrade{\conjstgrad J}{2}
&= \inv{\mu} \spacegrad \rho_\txtm + \epsilon \PD{t}{\BM} - \spacegrad \cross \BJ,
\end{aligned}
\end{equation*}
as well as a pair of continuity equations coupling the respective charge and current densities
\begin{equation*}
\begin{aligned}
\spacegrad \cdot \BJ + \PD{t}{\rho} &= 0 \\
\spacegrad \cdot \BM + \PD{t}{\rho_\txtm} &= 0.
\end{aligned}
\end{equation*}
} % theorem


To prove, we operate on \cref{dfn:isotropicMaxwells:680} with \( \spacegrad - (1/c) \partial_t \), one of the factors, along with the
spacetime gradient, of the
d'Lambertian (wave equation) operator, which gives
\begin{dmath}\label{eqn:continuity:110}
%\lr{ \spacegrad^2 - \inv{c^2} \PDSq{t}{} }
\dAlembertian
F = \conjstgrad J.
\end{dmath}
Since the left hand side has only grades 1 and 2, \cref{eqn:continuity:110} splits naturally into two equations, one for grades 1,2 and one for grades 0,3
\begin{dmath}\label{eqn:continuity:130}
\begin{aligned}
%\lr{ \spacegrad^2 - \inv{c^2} \PDSq{t}{} }
\dAlembertian
F &= \gpgrade{ \conjstgrad J }{1,2} \\
                                           0 &= \gpgrade{ \conjstgrad J }{0,3}.
\end{aligned}
\end{dmath}
Unpacking these further, we find that there is information
carried in the requirement that the grade 0,3 selection of \cref{eqn:continuity:130} is zero.
In particular, grade 0 selection gives
\begin{dmath}\label{eqn:continuity:40}
0
=
\gpgradezero{ ( \spacegrad - (1/c) \partial_t ) J }
=
\gpgradezero{
\Biglr{ \spacegrad - \inv{c} \PD{t}{} }
\biglr{
   \eta \lr{ c \rho - \BJ } + I \lr{ c \rho_\txtm - \BM }
}
}
=
-\eta
\lr{ \spacegrad \cdot \BJ + \PD{t}{\rho} }
,
\end{dmath}
which demonstrates the continuity condition on the electric sources.
Similarly, grade three selection gives
\begin{dmath}\label{eqn:continuity:60}
0
=
\gpgradethree{  (\spacegrad - (1/c) \partial_t ) J }
=
\gpgradethree{
\Biglr{ \spacegrad - \inv{c} \PD{t}{} }
\lr{
   \eta \lr{ c \rho - \BJ } + I \lr{ c \rho_\txtm - \BM }
}
}
=
-I \lr{
   \spacegrad \cdot \BM + \PD{t}{\rho_\txtm}
},
\end{dmath}
which demonstrates the continuity condition on the (fictitious) magnetic sources if included in the current.

For the non-homogeneous wave equation of \cref{thm:continuity:600}, the current derivatives may be expanded explicitly.
For the wave equation for the electric field, this is
\begin{dmath}\label{eqn:continuity:150}
%\lr{ \spacegrad^2 - \inv{c^2} \PDSq{t}{} } \BE
\dAlembertian \BE
=
\gpgradeone{\conjstgrad J}
=
\gpgradeone{
   \conjstgrad
   \lr{
      \frac{\rho}{\epsilon} - \eta \BJ + I \lr{ c \rho_\txtm - \BM }
   }
}
=
\inv{\epsilon} \spacegrad \rho -I \lr{ \spacegrad \wedge \BM } + \inv{c} \eta \PD{t}{\BJ}
= \gpgrade{\conjstgrad J}{1} = \inv{\epsilon} \spacegrad \rho + \mu \PD{t}{\BJ} + \spacegrad \cross \BM,
\end{dmath}
as claimed.
The forced magnetic field equation is
\begin{dmath}\label{eqn:continuity:170}
%\lr{ \spacegrad^2 - \inv{c^2} \PDSq{t}{} } \BH
\dAlembertian \BH
=
\inv{\eta I}
\gpgradetwo{\conjstgrad J}
=
\inv{\eta I}
\gpgradetwo{
   \conjstgrad
   \lr{
      \frac{\rho}{\epsilon} - \eta \BJ + I \lr{ c \rho_\txtm - \BM }
   }
}
=
\inv{\eta I}
\lr{
   -\spacegrad \wedge \BJ + I c \spacegrad \rho_\txtm + \frac{I}{c} \PD{t}{\BM}
}
=
\inv{I}
\lr{
   -I \lr{ \spacegrad \cross \BJ} + I \inv{\mu} \spacegrad \rho_\txtm + I \epsilon \PD{t}{\BM}
}
= \inv{\mu} \spacegrad \rho_\txtm + \epsilon \PD{t}{\BM} - \spacegrad \cross \BJ,
\end{dmath}
completing the proof.

%}

      \section{Plane waves.}
         %
% Copyright � 2018 Peeter Joot.  All Rights Reserved.
% Licenced as described in the file LICENSE under the root directory of this GIT repository.
%
%{
%%%\input{../latex/blogpost.tex}
%%%\renewcommand{\basename}{planewavesMultivector}
%%%%\renewcommand{\dirname}{notes/phy1520/}
%%%\renewcommand{\dirname}{notes/ece1228-electromagnetic-theory/}
%%%%\newcommand{\dateintitle}{}
%%%%\newcommand{\keywords}{}
%%%
%%%\input{../latex/peeter_prologue_print2.tex}
%%%
%%%\usepackage{peeters_layout_exercise}
%%%\usepackage{peeters_braket}
%%%\usepackage{peeters_figures}
%%%\usepackage{siunitx}
%%%%\usepackage{mhchem} % \ce{}
%%%%\usepackage{macros_bm} % \bcM
%%%%\usepackage{macros_qed} % \qedmarker
%%%%\usepackage{txfonts} % \ointclockwise
%%%
%%%\beginArtNoToc
%%%
%%%\generatetitle{Multivector plane wave representation}
%\chapter{Multivector plane wave representation}
\label{chap:planewavesMultivector}

With all sources zero,
the free space Maxwell's equation as given by \cref{dfn:isotropicMaxwells:680} for the
electromagnetic field strength reduces to just
\begin{dmath}\label{eqn:planewavesMultivector:300}
\stgrad F(\Bx, t) = 0.
\end{dmath}

Utilizing a phasor representation of the form \cref{dfn:greensFunctionOverview:300},
we will define the
phasor representation of the field as
%
% Copyright � 2018 Peeter Joot.  All Rights Reserved.
% Licenced as described in the file LICENSE under the root directory of this GIT repository.
%
\makedefinition{Plane wave.}{dfn:planewavesMultivector:680}{
We represent the
electromagnetic field strength
\textit{plane wave} solution of Maxwell's equation in phasor form as
\begin{equation*}
F(\Bx, t) = \Real \lr{ F(\Bk) e^{ j \omega t }  },
\end{equation*}
where the complex valued multivector \( F(\Bk) \) also has a presumed exponential dependence
\begin{equation*}
F(\Bk)
=
\tilde{F}
e^{ -j \Bk \cdot \Bx }.
\end{equation*}
} % definition


We will now show that solutions of the electromagnetic field wave equation have the form

\maketheorem{Plane wave solutions to Maxwell's equation.}{thm:planewavesMultivector:620}{
Single frequency plane wave solutions of Maxwell's equation have the form
\begin{equation*}
F(\Bx, t)
=
\Real \lr{
\lr{ 1 + \kcap }
\kcap \wedge \BE\,
e^{-j \Bk \cdot \Bx + j \omega t}
}
,
\end{equation*}
where \( \Norm{\Bk} = \omega/c \), \( \kcap = \Bk/\Norm{\Bk} \) is the unit vector pointing along the propagation direction, and \( \BE \) is any complex-valued vector variable.  When a \( \BE \cdot \Bk = 0 \) constraint is imposed on the vector variable \( \BE \), that variable can be interpreted as the electric field, and the solution reduces to
\begin{equation*}
F(\Bx, t)
=
\Real \lr{
\lr{ 1 + \kcap }
\BE\,
e^{-j \Bk \cdot \Bx + j \omega t}
}
,
\end{equation*}
showing that the field phasor \( F(\Bk) = \BE(\Bk) + I \eta \BH(\Bk) \) splits naturally into electric and magnetic components
\begin{equation*}
\begin{aligned}
\BE(\Bk) &= \BE e^{-j \Bk \cdot \Bx} \\
\eta \BH(\Bk) &= \kcap \cross \BE \, e^{-j \Bk \cdot \Bx},
\end{aligned}
\end{equation*}
where the directions \( \kcap, \BE, \BH \) form a right handed triple.
} % theorem

We wish to act on \( F(\Bk) e^{-j \Bk \cdot \Bx + j \omega t } \) with the spacetime gradient \( \spacegrad + (1/c)\partial_t \), but
must take care of order when applying the gradient to a non-scalar valued function.  In particular, if \( A \) is a multivector, then
\begin{dmath}\label{eqn:planewavesMultivector:660}
\spacegrad A e^{-j \Bk \cdot \Bx}
=
\sum_{m = 1}^3 \Be_m \partial_m
A e^{-j \Bk \cdot \Bx}
=
\sum_{m = 1}^3 \Be_m
A
\lr{ -j k_m }
e^{-j \Bk \cdot \Bx}
=
-j \Bk A.
\end{dmath}
Therefore, insertion of the presumed phasor solution of the field from
\cref{dfn:planewavesMultivector:680} into
\cref{eqn:planewavesMultivector:300} gives
\begin{dmath}\label{eqn:planewavesMultivector:60}
0
=
-j \lr{ \Bk - \frac{\omega}{c} } F(\Bk).
\end{dmath}

If \( F(\Bk) \) has a left multivector factor
\begin{dmath}\label{eqn:planewavesMultivector:80}
F(\Bk) =
\lr{ \Bk + \frac{\omega}{c} } \tilde{F},
\end{dmath}
where \( \tilde{F} \) is a multivector to be determined, then
\begin{dmath}\label{eqn:planewavesMultivector:100}
\lr{ \Bk - \frac{\omega}{c} }
F(\Bk)
=
\lr{ \Bk - \frac{\omega}{c} }
\lr{ \Bk + \frac{\omega}{c} } \tilde{F}
=
\lr{ \Bk^2 - \lr{\frac{\omega}{c}}^2 } \tilde{F},
\end{dmath}
which is zero if
if \( \Norm{\Bk} = \ifrac{\omega}{c} \).
Let \( \Norm{\Bk} \tilde{F} = F_0 + F_1 + F_2 + F_3 \), where
\( F_0, F_1, F_2, \) and \( F_3 \) respectively have grades 0,1,2,3, so that
\begin{dmath}\label{eqn:planewavesMultivector:120}
F(\Bk)
= \lr{ 1 + \kcap } \lr{ F_0 + F_1 + F_2 + F_3 }
=
F_0 + F_1 + F_2 + F_3
+
\kcap F_0 + \kcap F_1 + \kcap F_2 + \kcap F_3
=
F_0 + F_1 + F_2 + F_3
+
\kcap F_0 + \kcap \cdot F_1 + \kcap \cdot F_2 + \kcap \cdot F_3
+
\kcap \wedge F_1 + \kcap \wedge F_2
=
\lr{
   F_0 + \kcap \cdot F_1
}
+
\lr{
   F_1 + \kcap F_0 + \kcap \cdot F_2
}
+
\lr{
   F_2 + \kcap \cdot F_3 + \kcap \wedge F_1
}
+
\lr{
   F_3 + \kcap \wedge F_2
}.
\end{dmath}
Since the field \( F \) has only vector and bivector grades, the grades zero and three components of the expansion above must be zero, or
\begin{dmath}\label{eqn:planewavesMultivector:140}
\begin{aligned}
   F_0 &= - \kcap \cdot F_1 \\
   F_3 &= - \kcap \wedge F_2,
\end{aligned}
\end{dmath}
so
\begin{dmath}\label{eqn:planewavesMultivector:160}
F(\Bk)
=
\lr{ 1 + \kcap } \lr{
   F_1 - \kcap \cdot F_1 +
   F_2 - \kcap \wedge F_2
}
=
\lr{ 1 + \kcap } \lr{
   F_1 - \kcap F_1 + \kcap \wedge F_1 +
   F_2 - \kcap F_2 + \kcap \cdot F_2
}.
\end{dmath}
The multivector \( 1 + \kcap \) has the projective property of gobbling any leading factors of \( \kcap \)
\begin{dmath}\label{eqn:planewavesMultivector:180}
(1 + \kcap)\kcap
= \kcap + 1
= 1 + \kcap,
\end{dmath}
so for \( F_i \in F_1, F_2 \)
\begin{equation}\label{eqn:planewavesMultivector:200}
(1 + \kcap) ( F_i - \kcap F_i )
=
(1 + \kcap) ( F_i - F_i )
= 0,
\end{equation}
leaving
\begin{dmath}\label{eqn:planewavesMultivector:220}
F(\Bk)
=
\lr{ 1 + \kcap } \lr{
   \kcap \cdot F_2 +
   \kcap \wedge F_1
}.
\end{dmath}

For \( \kcap \cdot F_2 \) to be non-zero \( F_2 \) must be a bivector that lies in a plane containing \( \kcap \), and
\( \kcap \cdot F_2 \) is a vector in that plane that is perpendicular to \( \kcap \).
On the other hand \( \kcap \wedge F_1 \) is non-zero only if \( F_1 \) has a non-zero component that does not lie in along the \( \kcap \) direction, but \( \kcap \wedge F_1 \), like \( F_2 \) describes a plane that containing \( \kcap \).
This means that having both bivector and vector free variables \( F_2 \) and \( F_1 \) provide more degrees of freedom than required.
For example, if \( \BE \) is any vector, and \( F_2 = \kcap \wedge \BE \), then
\begin{dmath}\label{eqn:planewavesMultivector:240}
\lr{ 1 + \kcap }
   \kcap \cdot F_2
=
\lr{ 1 + \kcap }
   \kcap \cdot \lr{ \kcap \wedge \BE }
=
\lr{ 1 + \kcap }
\lr{
   \BE
-
\kcap \lr{ \kcap \cdot \BE }
}
=
\lr{ 1 + \kcap }
\kcap \lr{ \kcap \wedge \BE }
=
\lr{ 1 + \kcap }
\kcap \wedge \BE,
\end{dmath}
which has the form \( \lr{ 1 + \kcap } \lr{ \kcap \wedge F_1 } \), so the electromagnetic field strength phasor may be generally written
\begin{dmath}\label{eqn:planewavesMultivector:280}
F(\Bk)
=
\lr{ 1 + \kcap }
\kcap \wedge \BE \, e^{-j \Bk \cdot \Bx}
,
\end{dmath}
%\end{boxed}
Expanding the multivector factor \( \lr{ 1 + \kcap } \kcap \wedge \BE \) we find
\begin{dmath}\label{eqn:planewavesMultivector:720}
\lr{ 1 + \kcap }
\kcap \wedge \BE
=
\kcap \cdot \lr{ \kcap \wedge \BE }
+\cancel{\kcap \wedge \lr{ \kcap \wedge \BE }}
+
\kcap \wedge \BE
=
\BE - \kcap \lr{ \kcap \wedge \BE }
+
\kcap \wedge \BE.
\end{dmath}
The vector grade has the component of \( \BE \) along the propagation direction removed (i.e. it is the rejection), so there is no loss of generality should a
\( \BE \cdot \Bk = 0 \) constraint be imposed.  Such as constraint let's us write the bivector as a vector product \( \kcap \wedge \BE = \kcap \BE \), and then use the projective property \cref{eqn:planewavesMultivector:180} to gobble the leading \( \kcap \) factor, leaving
\begin{equation}\label{eqn:planewavesMultivector:700}
F(\Bk)
=
\lr{ 1 + \kcap }
\BE \, e^{-j \Bk \cdot \Bx}
=
\lr{ \BE + I \kcap \cross \BE }
\, e^{-j \Bk \cdot \Bx}.
\end{equation}

It is also noteworthy that
the directions \( \kcap, \Ecap, \Hcap \) form a right handed triple, which can be seen by computing their product
\begin{dmath}\label{eqn:planewavesMultivector:740}
(\kcap \Ecap) \Hcap
=
(-\Ecap \kcap) (-I \kcap \Ecap)
=
+I \Ecap^2 \kcap^2
=
I.
\end{dmath}
These vectors must all be mutually orthonormal for their product to be a pseudoscalar multiple.
Should there be doubt, explicit dot products may be computed with ease using grade selection operations
\begin{equation}\label{eqn:planewavesMultivector:760}
\begin{aligned}
\kcap \cdot \Hcap &= \gpgradezero{ \kcap (-I \kcap \Ecap) } = -\gpgradezero{ I \Ecap } = 0 \\
\Ecap \cdot \Hcap &= \gpgradezero{ \Ecap (-I \kcap \Ecap) } = -\gpgradezero{ I \kcap } = 0,
\end{aligned}
\end{equation}
where the zeros follow by noting that \( I \Ecap, I \kcap \) are both bivectors.
The conventional representation of the right handed triple relationship between the propagation direction and fields is stated as a cross product,
not as a
pseudoscalar relationship as in
\cref{eqn:planewavesMultivector:740}. These are easily seen to be equivalent
\begin{dmath}\label{eqn:planewavesMultivector:340}
\kcap
= I \Hcap \Ecap
= I (\Hcap \wedge \Ecap)
= I^2 (\Hcap \cross \Ecap)
= \Ecap \cross \Hcap.
\end{dmath}

%}
%\EndNoBibArticle

      \section{Statics.}
         \subsection{Inverting the Maxwell statics equation.}
            %
% Copyright © 2017 Peeter Joot.  All Rights Reserved.
% Licenced as described in the file LICENSE under the root directory of this GIT repository.
%
%{

Similar to electrostatics and magnetostatics, we can restrict attention to time invariant fields (\( \partial_t F = 0\)) and time invariant sources (\(\partial_t J = 0\)), but consider both electric and magnetic sources.  In that case Maxwell's equation is reduced to a first order gradient equation
\begin{dmath}\label{eqn:statics:20}
\spacegrad F(\Bx) = J(\Bx).
\end{dmath}

Using the Green's function for the (first order) gradient \cref{eqn:electrostatics_invertingGradient:260}, this can be inverted as
\begin{dmath}\label{eqn:statics:40}
F(\Bx)
= \int_V dV' G(\Bx, \Bx') \spacegrad' J(\Bx')
= \gpgrade{\int_V dV' G(\Bx, \Bx') \spacegrad' J(\Bx')}{1,2}
= \inv{4\pi} \int_V dV' \gpgrade{\frac{(\Bx - \Bx') J(\Bx')}{\Norm{\Bx - \Bx'}^3} }{1,2}.
\end{dmath}

A no-op grade selection has been inserted to simplify subsequent manipulation
\footnote{If this grade selection filter is omitted, it is possible to show that the scalar and pseudoscalar contributions to the \( (\Bx -\Bx') J \) product are zero on the boundary of the Green's integration volume. \citep{jancewicz1988multivectors:appendixI}},
yielding a compact solution to the Maxwell statics equation
\boxedEquation{eqn:statics:80}{
F(\Bx)
= \inv{4\pi} \int_V dV' \frac{\gpgrade{(\Bx - \Bx') J(\Bx')}{1,2}}{\Norm{\Bx - \Bx'}^3} + F_0,
}

where \( F_0 \) is any function for which \( \spacegrad F_0 = 0 \).

It will be preferable and simpler to work with combined fields and densities, however, some insight can be gained by
explicit expansion of \cref{eqn:statics:80} in terms of charge and current densities.
The grade selection, in terms of terms of \( \Bs = \Bx -\Bx' \) expands as
\begin{dmath}\label{eqn:statics:60}
\gpgrade{\Bs J}{1,2}
=
\eta \gpgrade{\Bs (v \rho - \BJ)}{1,2}
+
\gpgrade{\Bs I(v \rho_m - \BM)}{1,2}
=
\inv{\epsilon} \Bs \rho + \eta \BJ \wedge \Bs + \Bs v \rho_m I + \Bs \cross \BM,
\end{dmath}
so the solution \( F = \BE + \eta I \BH \) to the Maxwell statics equation in terms of the charge and current densities, and electric and magnetic fields is
\begin{dmath}\label{eqn:statics:100}
\begin{aligned}
\BE(\Bx)
&=
\inv{4\pi} \int_V dV' \inv{\Norm{\Bx - \Bx'}^3}
\lr{
   \inv{\epsilon}(\Bx - \Bx') \rho(\Bx')
   +
   (\Bx - \Bx') \cross \BM(\Bx')
} \\
\BH(\Bx)
&=
\inv{4\pi} \int_V dV' \inv{\Norm{\Bx - \Bx'}^3}
\lr{
  \BJ(\Bx') \cross (\Bx - \Bx')
+ \inv{\mu} (\Bx - \Bx') \rho_m(\Bx')
}.
\end{aligned}
\end{dmath}

We see that the combined solution \cref{eqn:statics:80} incorporates both a Coulomb's law contribution and a Biot-Savart law contribution.
When desired, any fictitious magnetic sources, also contribute to the electric and magnetic fields.

\makeproblem{}{problem:statics:81}{
Fill in any steps left out of the derivations of \cref{eqn:statics:60} and \cref{eqn:statics:100}.
} % problem


         \subsection{Enclosed charge.}
            %
% Copyright © 2018 Peeter Joot.  All Rights Reserved.
% Licenced as described in the file LICENSE under the root directory of this GIT repository.
%
%{
We have already solved the statics equation using Green's function techniques, but we can also gain additional insight by simply integrating Maxwell's equation using \cref{thm:volumeintegral:100} which gives
\begin{equation}\label{eqn:enclosedCurrent:20}
\int_V dV \spacegrad F = \int_{\partial V} dA \ncap F = \int_V dV J
\end{equation}
The total current in the volume is related to the surface integral of \( \ncap F \) over the boundary of the volume.
This is a multivector relationship, containing a substantial amount of information.
This information can be extracted by
expanding \( \ncap F \)
\begin{dmath}\label{eqn:enclosedCurrent:40}
\ncap F
=
\ncap \lr{ \BE + I \eta \BH }
=
\ncap \cdot \BE + I (\ncap \cross \BE) + I \eta \lr{ \ncap \cdot \BH + I \ncap \cross \BH }
=
\ncap \cdot \BE - \eta ( \ncap \cross \BH ) + I (\ncap \cross \BE) + I \eta (\ncap \cdot \BH),
\end{dmath}
so
\begin{dmath}\label{eqn:enclosedCurrent:60}
\begin{aligned}
\int_{\partial V} dA\, \ncap \cdot \BE        &=  \inv{\epsilon} \int_V dV\, \rho \\
\int_{\partial V} dA\, \ncap \cross \BH       &=                 \int_V dV\, \BJ \\
\int_{\partial V} dA\, \ncap \cross \BE       &=               - \int_V dV\, \BM \\
\int_{\partial V} dA\, \ncap \cdot \BH        &=  \inv{\mu} \int_V dV\, \rho_\txtm.
\end{aligned}
\end{dmath}
Of course \cref{eqn:enclosedCurrent:60}
could have obtained directly from Maxwell's equations in their conventional form \cref{eqn:freespace:3399}.
However, had we integrated the conventional Maxwell's equations, it would not have been obvious that the crazy mix of
fields, dot and cross products in \cref{eqn:enclosedCurrent:60} had a hidden structure as simple as
\( \int_{\partial V} dA \ncap F = \int_V dV J \).
%}

         \subsection{Enclosed current.}
            %
% Copyright � 2018 Peeter Joot.  All Rights Reserved.
% Licenced as described in the file LICENSE under the root directory of this GIT repository.
%
%{
%\input{../latex/blogpost.tex}
%\renewcommand{\basename}{amperes}
%%\renewcommand{\dirname}{notes/phy1520/}
%\renewcommand{\dirname}{notes/ece1228-electromagnetic-theory/}
%%\newcommand{\dateintitle}{}
%%\newcommand{\keywords}{}
%
%\input{../latex/peeter_prologue_print2.tex}
%
%\usepackage{peeters_layout_exercise}
%\usepackage{peeters_braket}
%\usepackage{peeters_figures}
%\usepackage{siunitx}
%%\usepackage{mhchem} % \ce{}
%%\usepackage{macros_bm} % \bcM
%%\usepackage{macros_qed} % \qedmarker
%\usepackage{txfonts} % \ointclockwise
%
%\beginArtNoToc
%
%\generatetitle{Generalizing Ampere's law using geometric algebra.}
%%\chapter{Field flux.}
\label{chap:amperes}

%Ampere's law is a relationship for static configurations between the line integral of the magnetic field to the enclosed current, the flux of the current density through a surface
In this section we will present the generalization of Ampere's law to line integrals of the total electromagnetic field strength.

\maketheorem{Line integral of the field.}{thm:amperes:280}{
The line integral of the electromagnetic field strength is
\begin{equation*}
\ointclockwise_{\partial A} d\Bx\, F
=
I \int_A dA \lr{ \ncap J - \PD{n}{F} },
\end{equation*}
where \( \PDi{n}{F} = \lr{ \ncap \cdot \spacegrad } F \).
Expressed in terms of the conventional consistent fields and sources, this multivector relationship expands to four equations, one for each grade
\begin{equation*}
\begin{aligned}
\ointclockwise_{\partial A} d\Bx \cdot \BE &=  \int_A dA\, \ncap \cdot \BM \\
\ointclockwise_{\partial A} d\Bx \cross \BH
&=
\int_A dA
\lr{
   - \ncap \cross \BJ
   + \frac{ \ncap \rho_\txtm }{\mu}
   - \PD{n}{\BH}
} \\
\ointclockwise_{\partial A} d\Bx \cross \BE &=
\int_A dA
\lr{
     \ncap \cross \BM
   + \frac{\ncap \rho}{\epsilon}
   - \PD{n}{\BE}
} \\
\ointclockwise_{\partial A} d\Bx \cdot \BH &= -\int_A dA\, \ncap \cdot \BJ.
\end{aligned}
\end{equation*}
} % theorem
The last of the scalar equations in
\cref{thm:amperes:280}
is Ampere's law
\begin{equation}\label{eqn:amperes:20}
\ointctrclockwise_{\partial A} d\Bx \cdot \BH = \int_A \ncap \cdot \BJ = I_{\textrm{enc}},
\end{equation}
and the first is the dual of Ampere's law for (fictious) magnetic current density\footnote{Even without the fictitious magnetic sources, neither the name nor applications of the two cross product line integrals with the normal derivatives are familiar to the author.}.
In \cref{eqn:amperes:20} the flux of the electric current density equals the enclosed current flowing through an open surface.  This enclosed current equals the line integral of the magnetic field around the boundary of that surface.

To prove
\cref{thm:amperes:280}
we compute the surface integral of the current \( J = \spacegrad F \)
\begin{dmath}\label{eqn:amperes:240}
\int_A d^2 \Bx\, J
=
\int_A d^2 \Bx \spacegrad F.
\end{dmath}
As we are working in \R{3} not \R{2}, the gradient may not be replaced by the vector derivative in \cref{eqn:amperes:240}.  Instead we
must split the gradient into its vector derivative component, the projection of the gradient onto the tangent plane of the integration surface, and its normal component
\begin{dmath}\label{eqn:amperes:160}
\spacegrad = \boldpartial + \ncap (\ncap \cdot \spacegrad).
\end{dmath}
The surface integral form \cref{eqn:surfaceintegral:300}
of the fundamental theorem of geometric calculus may be
applied to the vector derivative portion of the field integral
\begin{dmath}\label{eqn:amperes:100}
\int_A d^2 \Bx \spacegrad F
=
\int_A d^2 \Bx\, \boldpartial F
+
\int_A d^2 \Bx\, \ncap \lr{ \ncap \cdot \spacegrad } F,
\end{dmath}
so
\begin{dmath}\label{eqn:amperes:120}
\ointclockwise_{\partial A} d\Bx\, F
=
\int_A d^2 \Bx \lr{ J - \ncap \lr{ \ncap \cdot \spacegrad } F }
=
\int_A dA \lr{ I \ncap J - \lr{ \ncap \cdot \spacegrad } I F }
=
\int_A dA \lr{ I \ncap J - I \PD{n}{F} },
\end{dmath}
where the surface area bivector has been written in its dual form \( d^2 \Bx = I \ncap dA \) in terms of a scalar area element, and the directional derivative has been written in scalar form with respect to a parameter \( n \) that represents the length along the normal direction.  This proves the first part of
\cref{thm:amperes:280}.

Observe that the \( d\Bx\, F \) product has all possible grades
\begin{dmath}\label{eqn:amperes:180}
d\Bx\, F
=
d\Bx \lr{ \BE + I \eta \BH }
=
d\Bx \cdot \BE + I \eta d\Bx \cdot \BH
+
d\Bx \wedge \BE + I \eta d\Bx \wedge \BH
=
d\Bx \cdot \BE
- \eta (d\Bx \cross \BH)
+ I (d\Bx \cross \BE )
+ I \eta (d\Bx \cdot \BH),
\end{dmath}
as does the \( I \ncap J \) product (in general)
\begin{dmath}\label{eqn:amperes:200}
I \ncap J
=
I \ncap
\lr{
   \frac{\rho}{\epsilon} - \eta \BJ + I \lr{ c \rho_\txtm - \BM }
}
=
\ncap I \frac{\rho}{\epsilon} - \eta \ncap I \BJ - \ncap c \rho_\txtm + \ncap \BM
=
% 0
\ncap \cdot \BM
% 1
+ \eta (\ncap \cross \BJ)
- \ncap c \rho_\txtm
% 2
+ I (\ncap \cross \BM)
+ \ncap I \frac{\rho}{\epsilon}
% 3
- \eta I (\ncap \cdot \BJ).
\end{dmath}
On the other hand \( I F = I \BE - \eta \BH \) has only grades 1,2, like \( F \) itself.  This allows the line integrals to be split by grade selection into components with and without a normal derivative
\begin{dmath}\label{eqn:amperes:140}
\begin{aligned}
\ointclockwise_{\partial A} \gpgrade{d\Bx\, F}{0,3}
&=
\int_A dA\, \gpgrade{ I \ncap J }{0,3} \\
\ointclockwise_{\partial A} \gpgrade{d\Bx\, F}{1,2}
&=
\int_A dA \lr{ \gpgrade{ I \ncap J }{1,2} - \lr{ \ncap \cdot \spacegrad } I F }.
\end{aligned}
\end{dmath}
The first of \cref{eqn:amperes:140} contains Ampere's law and its dual as one multivector equation, which can be seen more readily by explicit expansion in the constituent fields and sources using \cref{eqn:amperes:180}, \cref{eqn:amperes:200}
\begin{dmath}\label{eqn:amperes:220}
\begin{aligned}
\ointclockwise_{\partial A}
\lr{
   d\Bx \cdot \BE + I \eta (d\Bx \cdot \BH)
}
&=
\int_A dA
\lr{
   \ncap \cdot \BM - \eta I (\ncap \cdot \BJ)
} \\
\ointclockwise_{\partial A}
\lr{
   - \eta (d\Bx \cross \BH)
   + I (d\Bx \cross \BE )
}
&=
\int_A dA
\lr{
   % 1
     \eta (\ncap \cross \BJ)
   - \ncap c \rho_\txtm
   % 2
   + I (\ncap \cross \BM)
   + \ncap I \frac{\rho}{\epsilon}
   -\PD{n}{} \lr{ I \BE - \eta \BH }
}.
\end{aligned}
\end{dmath}

Further grade selection operations, and minor adjustments of the leading constants completes the proof.

It is also worth pointing out that for pure magnetostatics problems where \( J = \eta \BJ, F = I \eta \BH \), that Ampere's law can be written in a trivector form
\begin{equation}\label{eqn:amperes:260}
\ointclockwise_{\partial A} d\Bx \wedge F = I \int_A dA\, \ncap \cdot J = I \eta \int_A dA\, \ncap \cdot \BJ.
\end{equation}
This encodes the fact that the magnetic field component of the total electromagnetic field strength is most naturally expressed in
geometric algebra as a bivector.
% \( I \eta \BH \), and not just the vector \( \BH \).

%}
%\EndArticle

         \subsection{Example field calculations.}
            Having seen a number of theoretical applications of the geometric algebra framework, let's now see how some of our new tools can be used to calculate the fields for specific static electromagnetism charge and current configurations.
         \subsubsection{Line segment.}
            %
% Copyright © 2017 Peeter Joot.  All Rights Reserved.
% Licenced as described in the file LICENSE under the root directory of this GIT repository.
%
%\makeexample{Line charge.}{example:linecharge:linecharge}{
\index{line charge}
In this example the electric field is calculated at a point on the z-axis, due to a line charge density of \( \lambda \) along a segment \( [a,b] \) of the x-axis.
This is illustrated in \cref{fig:linecharge:linechargeFig1}.
\imageFigure{../figures/GAelectrodynamics/linechargeFig1}{Line charge density.}{fig:linecharge:linechargeFig1}{0.3}

Introducing a unit imaginary \( i = \Be_{13} \) for the rotation from the x-axis to the z-axis, the field point observation point is

\begin{dmath}\label{eqn:linecharge:120}
\Bx = r \Be_1 e^{i \theta}.
\end{dmath}

The charge element point is \( \Bx' = x \Be_1 \), so the difference can now be written with \( \Be_1 \) factored to the left or to the right

\begin{equation}\label{eqn:linecharge:20}
\Bx - \Bx'
= \Be_1\lr{ r e^{i\theta} - x }
= \lr{ r e^{-i\theta} - x } \Be_1.
\end{equation}

This allows the squared vector length to be calculated as a product of complex conjugates

\begin{dmath}\label{eqn:linecharge:40}
\lr{ \Bx - \Bx' }^2
= \lr{ r e^{-i\theta} - x } \Be_1 \Be_1\lr{ r e^{i\theta} - x }
= \lr{ r e^{-i\theta} - x } \lr{ r e^{i\theta} - x }
= r^2 + x^2 - r x \lr{ e^{i\theta} + e^{-i\theta} }
= r^2 + x^2 - 2 r x \cos\theta.
\end{dmath}

The total electric field is therefore
\begin{dmath}\label{eqn:linecharge:60}
\BE
= \frac{\lambda}{4 \pi \epsilon_0} \int_a^b dx \frac{ r \Be_1 e^{i\theta} - x \Be_1 }{ \lr{ r^2 + x^2 - 2 x r \cos\theta }^{3/2} }
= \frac{\lambda \Be_1}{4 \pi \epsilon_0 r} \int_{a/r}^{b/r} du \frac{ e^{i\theta} - u }{ \lr{ 1 + u^2 - 2 u \cos\theta }^{3/2} }.
\end{dmath}

This integral can be evaluated by table lookup or using tools like Mathematica.
For \( \theta = \pi/2 \)

\begin{dmath}\label{eqn:linecharge:80}
\int
du \frac{ e^{i\theta} - u }{ \lr{ 1 + u^2 - 2 u \cos\theta }^{3/2} }
= \frac{1 + i u}{\sqrt{1 + u^2}},
\end{dmath}

and for other angles \( \theta \neq n \pi/2 \)

\begin{dmath}\label{eqn:linecharge:100}
\int
du \frac{ e^{i\theta} - u }{ \lr{ 1 + u^2 - 2 u \cos\theta }^{3/2} }
= \frac{(1 -u e^{-i\theta}) \sqrt{1 + u^2 - 2 u \cos\theta}}{(1 + u^2) \sin(2\theta)}.
\end{dmath}

\index{complex plane}
The interesting takeaway is not the form of the solution, but the fact that GA allows an introduction of a ``complex plane'' for many problems that have polar representations in a plane.
When such a complex plane is introduced,
existing Computer Algebra Systems (CAS), like Mathematica, can be utilized for the grunt work of the evaluation.

Observe that the numerator factors like \( \Be_1 (1 + i u) \) and \( \Be_1(1 - u e^{-i\theta}) \)
compactly describe the direction of the vector field at the observation point.
Either of these can be expanded explicitly in sines and cosines if desired

\begin{dmath}\label{eqn:linecharge:140}
\begin{aligned}
\Be_1 (1 + i u) &= \Be_1 + u \Be_3 \\
\Be_1(1 - u e^{-i\theta}) &= \Be_1(1 - u \cos\theta) + u \Be_3 \sin\theta.
\end{aligned}
\end{dmath}

%} % example

         \subsubsection{Infinite line current.}
            

Given a static line charge density and current density along the z-axis

\begin{dmath}\label{eqn:statics:120}
\begin{aligned}
\rho(\Bx) &= \lambda \delta(x) \delta(y) \\
\BJ(\Bx) &= \Bv \rho(\Bx) = v \lambda \Be_3 \delta(x) \delta(y),
\end{aligned}
\end{dmath}

the total multivector current is
\begin{dmath}\label{eqn:statics:140}
J
= \eta ( c \rho - \BJ )
= \eta ( c - v \Be_3 ) \lambda \delta(x) \delta(y)
= \frac{\lambda}{\epsilon} \lr{ 1 - \frac{v}{c} \Be_3 } \delta(x) \delta(y)
\end{dmath}

We can find the field for this current by substitution into \cref{eqn:statics:80}.
To do so, let the field observation point be \( \Bx = \Bx_\perp + z \Be_3 \), so the total field is
\begin{dmath}\label{eqn:statics:160}
F(\Bx)
= \frac{\lambda}{4\pi \epsilon} \int_V dx'dy'dz' \frac{\gpgrade{(\Bx - \Bx') (1 - (v/c) \Be_3 )}{1,2}}{\Norm{\Bx - \Bx'}^3} \delta(x') \delta(y')
= \frac{\lambda}{4\pi \epsilon} \int_{-\infty}^\infty dz' \frac{\gpgrade{(\Bx_\perp + (z - z') \Be_3) (1 - (v/c) \Be_3 )}{1,2}}{\lr{\Bx_\perp^2 + (z-z')^2}^{3/2}}
=
\frac{\lambda \lr{ \Bx_\perp - (v/c) \Bx_\perp \Be_3}}{4\pi \epsilon} \int_{-\infty}^\infty \frac{dz'}{\lr{\Bx_\perp^2 + (z-z')^2}^{3/2}}
+
\frac{\lambda \Be_3}{4\pi \epsilon} \int_{-\infty}^\infty \frac{(z - z') dz'}{\lr{\Bx_\perp^2 + (z-z')^2}^{3/2}}.
\end{dmath}

The first integral is \( 2/\Bx_\perp^2 \), whereas the second is zero (odd function, over even interval).
Using cylindrical coordinates \( \Bx = R \rhocap + z \Be_3 \), and since
\( \Bx_\perp \cdot \Be_3 = 0 \), the wedge can be replaced with the vector product \( \Bx_\perp \Be_3 = R \rhocap \Be_3 \), leaving

\begin{equation}\label{eqn:statics:180}
F(\Bx)
=
\frac{\lambda}{2\pi \epsilon R} \rhocap \lr{ 1 - \Bv/c} = \BE \lr{ 1 - \Bv/c }
= \BE + I \lr{ \frac{\Bv}{c} \cross \BE },
\end{equation}

where \( \Bv = v \Be_3 \).
The vector component of this is the electric field, which is therefore directed radially, whereas the (dual) magnetic field \( \eta I \BH \)
is a set of oriented planes spanning the radial and z-axis directions.
We can also see that there is a constant proportionality factor that relates the electric and magnetic field components, namely

\begin{dmath}\label{eqn:statics:200}
I \eta \BH = -\BE \Bv/c,
\end{dmath}

or

\begin{dmath}\label{eqn:statics:220}
\BH = \Bv \cross \BD.
\end{dmath}

\makeproblem{Statics solution for linear magnetic density and currents.}{problem:statics:240}{
Given magnetic charge density \( \rho_m = \lambda_m \delta(x) \delta(y) \), and current density \( \BM = v \Be_3 \rho_m = \Bv \rho_m \), show that the field is given by
\begin{equation*}
F(\Bx) = \frac{\lambda_m c}{4 \pi R} I \rhocap \lr{ 1 - \frac{\Bv}{c} },
\end{equation*}
or with \( \BB = \lambda_m \rhocap/(4 \pi R) \),
\begin{equation*}
F = \BB \cross \Bv + c I \BB.
\end{equation*}
} % problem


         \subsubsection{Infinite planar current.}
            %
% Copyright © 2018 Peeter Joot.  All Rights Reserved.
% Licenced as described in the file LICENSE under the root directory of this GIT repository.
%
%{
A variation on the above example puts a uniform charge density \( \rho(\Bx) = \sigma \delta(z) \) in a plane, along with an associated current density \( \BJ(\Bx) = v \Be_1 e^{i\theta} \rho(\Bx), \quad i = \Be_{12} \).  Letting \( \Bv = v \Be_1 e^{i\theta} \), the multivector current is
\begin{dmath}\label{eqn:statics_infinitePlane:240}
J(\Bx) = \sigma \eta \lr{ c - \Bv } \delta(z),
\end{dmath}
so the field off the plane is
\begin{dmath}\label{eqn:statics_infinitePlane:260}
F(\Bx)
=
\frac{\sigma}{4 \pi \epsilon } \iiint \frac{dz' dA'}{ \Norm{ \Bx - \Bx' }^3 }
\gpgrade{ (\Bx - \Bx') (1 - \Bv/c) }{1,2} \delta(z').
\end{dmath}

If \( \Bx_\parallel = (\Bx \wedge \Be_3) \Be_3 \), and \( \Bx'_\parallel = (\Bx' \wedge \Be_3) \Be_3 \), are the components of the vectors \( \Bx, \Bx' \) in the x-y plane, then integration over \( z' \) and a change of variables \( \Bx'_\parallel - \Bx_\parallel = r' \Be_1 e^{i \theta'} \) yields
\begin{dmath}\label{eqn:statics_infinitePlane:280}
F(\Bx)
=
\frac{\sigma}{4 \pi \epsilon } \iint \frac{r' dr' d\theta'}{ ( z^2 + {r'}^2 )^{3/2} }
\gpgrade{ \lr{ z \Be_3 - r' \Be_1 e^{i\theta'} } (1 - \Bv/c) }{1,2}.
\end{dmath}

The \( e^{i\theta'} \) integrands are killed, so for \( z \ne 0 \), the field is
\begin{dmath}\label{eqn:statics_infinitePlane:300}
F(\Bx)
=
\frac{\sigma z}{4 \pi \epsilon \Abs{z} } \gpgrade{ \Be_3 (1 - \Bv/c) }{1,2}.
\end{dmath}

Since \( \Bv \in \Span \setlr{ \Be_1, \Be_2 } \) the product \( \Be_3 \Bv \) is a bivector and the grade selection can be dropped, leaving
\begin{dmath}\label{eqn:statics_infinitePlane:320}
F(\Bx)
=
\frac{\sigma \sgn(z)}{4 \pi \epsilon } \Be_3 \lr{ 1 - \frac{\Bv}{c}}.
\end{dmath}

This field toggles sign when crossing the plane, but is constant otherwise.  The electric and magnetic field components are once again related by \cref{eqn:statics_infiniteLineCharge:220}.

\makeproblem{Statics solution for planar magnetic density and currents.}{problem:statics:241}{
Given magnetic charge density \( \rho_m = \sigma_m \delta(z) \), and current density \( \BM = \Bv \rho_m, \Bv = v \Be_1 e^{i\theta}, i = \Be_{12}\), show that the field is given by
\begin{equation*}
F(\Bx) = \frac{\sigma_m c \sgn(z)}{4 \pi } i \lr{ 1 - \frac{\Bv}{c} }.
\end{equation*}
} % problem

%}

         \subsubsection{Arc line charge.}
            %
% Copyright © 2017 Peeter Joot.  All Rights Reserved.
% Licenced as described in the file LICENSE under the root directory of this GIT repository.
%
%\makeexample{Circular line charge.}{example:circularlinecharge:circularlinecharge}{
In this example we will examine the electric field due to a line charge density of \( \lambda \) along a circular arc segment \( \phi' \in [a,b] \), of radius \( r \) in the x-y plane.
The field will be evaluated at the
spherical coordinate point \( (R, \theta, \phi) \), as illustrated in \cref{fig:circularlinecharge:circularlinechargeFig1}.  (FIXME: better figure).

\imageFigure{../figures/GAelectrodynamics/circularlinechargeFig1}{Circular line charge.}{fig:circularlinecharge:circularlinechargeFig1}{0.3}

To setup this problem, first consider a GA factorization of the spherical radial point.  Let
\( i = \Be_{12} \), so that

\begin{dmath}\label{eqn:circularlinecharge:20}
\Bx
= R \lr{ \Be_1 \sin\theta \cos\phi + \Be_2 \sin\theta \sin\phi + \Be_3 \cos\theta }
= R \Be_1 \sin\theta ( \cos\phi + \Be_{12} \sin\phi ) + \Be_3 \cos\theta
= R \Be_1 \sin\theta e^{i \phi} + \Be_3 \cos\theta
= R \Be_3 \lr{ \cos\theta + \Be_{31} e^{i \phi} \sin\theta }.
\end{dmath}

Let the bivector for the rotational plane between \( R \Be_3 \rightarrow \Bx \) be represented by

\begin{dmath}\label{eqn:circularlinecharge:40}
j = \Be_{31} e^{i \phi}.
\end{dmath}

The bivector \( j \) is a function of the azimuthal angle \( \phi \), but encodes all the geometry of the rotation.
The observation point now has the simple representation

\begin{dmath}\label{eqn:circularlinecharge:60}
\Bx = R \Be_3 e^{j \theta },
\end{dmath}

and is the product of a polar directed vector with a complex exponential whos argument is the polar rotation angle.
To sum the contributions of the charge elements we need the distance between the charge element and the observation point.  That vector difference is

\begin{dmath}\label{eqn:circularlinecharge:80}
\Bx - \Bx'
=
R \Be_3 e^{j \theta } - r \Be_1 e^{i \phi'}.
\end{dmath}

Compare this to the tuple representation

\begin{dmath}\label{eqn:circularlinecharge:100}
\Bx - \Bx'
= ( R \sin\theta \cos\phi - r \cos\phi', R \sin\theta \sin\phi - r \cos\phi', \cos\theta ),
\end{dmath}

for which the prospect of working with is considerably less attractive.  The squared length of \cref{eqn:circularlinecharge:80} is
\begin{dmath}\label{eqn:circularlinecharge:120}
(\Bx - \Bx')^2
=
R^2 + r^2 - 2 R r \lr{\Be_3 e^{j \theta} } \cdot \lr{ \Be_1 e^{i \phi'} }.
\end{dmath}

The dot product of unit vectors in \cref{eqn:circularlinecharge:120} can be reduced using scalar grade selection

\begin{dmath}\label{eqn:circularlinecharge:140}
\lr{\Be_3 e^{j \theta }} \cdot \lr{ \Be_1 e^{i \phi'} }
=
\gpgradezero{
\lr{ \Be_1 \sin\theta e^{i \phi} } \lr{ \Be_1 e^{i \phi'}}
}
=
\sin\theta
\gpgradezero{
e^{-i \phi} e^{i \phi'}
}
=
\sin\theta \cos( \phi' - \phi ),
\end{dmath}

so
\begin{dmath}\label{eqn:circularlinecharge:160}
\Norm{ \Bx - \Bx' }
=
\sqrt{
R^2 + r^2 - 2 R r \sin\theta \cos( \phi' - \phi )
}.
\end{dmath}

The electric field is
\begin{dmath}\label{eqn:circularlinecharge:180}
\BE = \frac{1}{4 \pi \epsilon_0} \int_a^b \lambda r d\phi' \frac{ R \Be_3 e^{j \theta } - r \Be_1 e^{i \phi'} }{\lr{ R^2 + r^2 - 2 R r \sin\theta \cos( \phi' - \phi ) }^{3/2} }.
\end{dmath}

Non-dimensionalizing \cref{eqn:circularlinecharge:180} with \( u = r/R \), a change of variables \( \alpha = \phi' - \phi \), and noting that \( i \phicap = \Be_1 e^{i \phi} \), the field is

\begin{dmath}\label{eqn:circularlinecharge:200}
\BE
= \frac{\lambda r}{4 \pi \epsilon_0 R^2} \int_{a-\phi}^{b-\phi} d\alpha \frac{ \Be_3 e^{j \theta } - u \Be_1 e^{i\phi} e^{i \alpha} }{\lr{ 1 + u^2 - 2 u \sin\theta \cos \alpha }^{3/2} }
= \frac{\lambda r}{4 \pi \epsilon_0 R^2} \int_{a-\phi}^{b-\phi} d\alpha \frac{ \rcap + u \phicap i e^{i \alpha } }{\lr{ 1 + u^2 - 2 u \sin\theta \cos \alpha }^{3/2} }.
\end{dmath}

The radial integral \( \int dx \lr{ a^2 - 2 b \cos x }^{-3/2} \) scales the radial electric field component without rotating it.
The azimuthal integral \( \int dx \lr{ a^2 - 2 b \cos x }^{-3/2} \) scales and rotates the azimuthal electric field component within the azimuthal plane.
Evaluating \cref{eqn:circularlinecharge:200} using CAS software with no direct GA support requires care, since the bivectors \( i, j \) do not commute and cannot be assembled into a single complex integrand as was possible for the linear line charge.  These integrals can be computed separately without trouble.

%} % example

         \subsubsection{Field of a ring current.}
            %
% Copyright © 2017 Peeter Joot.  All Rights Reserved.
% Licenced as described in the file LICENSE under the root directory of this GIT repository.
%
%{

\subsection{Example.  Field of a ring of charge or current density.}

Let's now compute the field due to a static charge or current density on a ring of radius \( R \) as illustrated in
\cref{fig:chargeAndCurrentOnRing:chargeAndCurrentOnRingFig1}.

\imageFigure{../figures/GAelectrodynamics/chargeAndCurrentOnRingFig1}{Field due to a circular distribution.}{fig:chargeAndCurrentOnRing:chargeAndCurrentOnRingFig1}{0.3}

A static charge distribution on a ring at \( z = 0 \) has the form

\begin{dmath}\label{eqn:ringField:20}
\rho(\Bx) = \lambda \delta(z) \delta(r - R).
\end{dmath}

As always the current distribution is of the form \( \BJ = \Bv \rho \), and in this case the velocity is azimuthal \( \Bv = \Be_2 e^{i\phi}, i = \Be_{12} \).
The total multivector current is

\begin{dmath}\label{eqn:ringField:40}
J = \inv{\epsilon_0} \lambda \delta(z) \delta(r - R) \lr{ 1 - \frac{\Bv}{c} }.
\end{dmath}

Let the point that we observe the field, and the integration variables be

\begin{dmath}\label{eqn:ringField:60}
\begin{aligned}
\Bx &= z \Be_3 + r \rhocap \\
\Bx' &= z' \Be_3 + r' \rhocap'.
\end{aligned}
\end{dmath}

The field is

\begin{dmath}\label{eqn:ringField:80}
F(\Bx)
= \frac{\lambda}{4 \pi \epsilon_0} \iiint dz' r' dr' d\phi' \delta(z') \delta(r' - R) \frac{\gpgrade{ \lr{ (z - z') \Be_3 + r \rhocap - r' \rhocap' } \lr{ 1 - \frac{v}{c} \Be_2 e^{i\phi'} } }{1,2} } { \lr{ (z-z')^2 + (r \rhocap - r' \rhocap')^2}^{3/2} }
= \frac{\lambda}{4 \pi \epsilon_0} \int R d\phi' \frac{\gpgrade{ \lr{ z \Be_3 + r \rhocap - R \rhocap' } \lr{ 1 - \frac{v}{c} \Be_2 e^{i\phi'} } }{1,2} } { \lr{ z^2 + (r \rhocap - R \rhocap')^2}^{3/2} }
\end{dmath}

Without loss of generality, we can align the axes so that \( \rhocap = \Be_1 \), and
introduce dimensionless variables

\begin{dmath}\label{eqn:ringField:100}
\begin{aligned}
\tilde{z} &= z/R \\
\tilde{r} &= r/R,
\end{aligned}
\end{dmath}

which gives
\begin{dmath}\label{eqn:ringField:120}
F
= \frac{\lambda}{4 \pi \epsilon_0 R} \int_0^{2\pi} d\phi' \frac{\gpgrade{ \lr{ \tilde{z} \Be_3 + \tilde{r} \Be_1 - \Be_1 e^{i\phi'} } \lr{ 1 - \frac{v}{c} \Be_2 e^{i\phi'} } }{1,2} } { \lr{ \tilde{z}^2 + (\tilde{r} \Be_1 - \Be_1 e^{i\phi'})^2}^{3/2} }.
\end{dmath}

In the denominator, the vector square expands as
\begin{dmath}\label{eqn:ringField:140}
(\tilde{r} \Be_1 - \Be_1 e^{i\phi'})^2
=
(\tilde{r} - e^{-i\phi'}) \Be_1^2 (\tilde{r} - e^{i\phi'})
=
\tilde{r}^2 + 1 - 2 \tilde{r} \cos(\phi'),
\end{dmath}

and the grade selection in the numerator is

\begin{dmath}\label{eqn:ringField:160}
\begin{aligned}
\gpgrade{ \lr{ \tilde{z} \Be_3 + \tilde{r} \Be_1 - \Be_1 e^{i\phi'} } \lr{ 1 - \frac{v}{c} \Be_2 e^{i\phi'}}}{1,2}
&=
\tilde{z} \Be_3 + \tilde{r} \Be_1 - \Be_1 e^{i\phi'} \\
&-\frac{v}{c}\lr{ \tilde{z} \Be_{31} e^{i\phi'} + \tilde{r} i \cos(\phi') + i }.
\end{aligned}
\end{dmath}

Any of the exponential integrals terms
are of the form

\begin{dmath}\label{eqn:ringField:180}
\int_0^{2\pi} d\phi' e^{i\phi'} f(\cos(\phi')) = \int_0^{2\pi} d\phi' \cos(\phi') f(\cos(\phi')),
\end{dmath}

since
the \( i \sin\phi' f(\cos(\phi') \) contributions are odd functions around \( \phi' = \pi \).

For general \( z, r \) the integrals above require numeric evaluation or special functions.
Let
\begin{subequations}
\label{eqn:ringField:260}
\begin{dmath}\label{eqn:ringField:200}
A
%= A(\tilde{z}, \tilde{r})
= \int_0^{2\pi} d\phi' \frac{1}{\lr{ 1 + \tilde{z}^2 + \tilde{r}^2 - 2 \tilde{r} \cos(\phi') }^{3/2}}
=
\frac{4 E\left(-\frac{4 \tilde{r}}{(\tilde{r}-1)^2+\tilde{z}^2}\right)}{\sqrt{\tilde{z}^2+(\tilde{r}-1)^2} \left(\tilde{z}^2+(\tilde{r}+1)^2\right)}
\end{dmath}
\begin{dmath}\label{eqn:ringField:280}
B
%= B(\tilde{z}, \tilde{r})
= \int_0^{2\pi} d\phi' \frac{\cos(\phi')}{\lr{ 1 + \tilde{z}^2 + \tilde{r}^2 - 2 \tilde{r} \cos(\phi') }^{3/2}}
=
\frac{2 \left(\left(\tilde{z}^2+\tilde{r}^2+1\right) E\left(-\frac{4 \tilde{r}}{(\tilde{r}-1)^2+\tilde{z}^2}\right)-\left(\tilde{z}^2+(\tilde{r}+1)^2\right) K\left(-\frac{4 \tilde{r}}{(\tilde{r}-1)^2+\tilde{z}^2}\right)\right)}{\tilde{r} \sqrt{\tilde{z}^2+(\tilde{r}-1)^2} \left(\tilde{z}^2+(\tilde{r}+1)^2\right)},
\end{dmath}
\end{subequations}

where
\( K(m), E(m) \)
are complete elliptic integrals of the first and second kind respectively.
As seen in \cref{fig:ringFieldAB}, these functions are similar,
both tailing off quickly with \( z, \rho \), with largest values the ring.
\imageTwoFigures
{../figures/GAelectrodynamics/ringFieldAFig1}
{../figures/GAelectrodynamics/ringFieldBFig1}
{(a) \(A(\tilde{z}, \tilde{\rho})\).  (b) \(B(\tilde{z}, \tilde{\rho})\).}{fig:ringFieldAB}{scale=0.3}

Finally, restoring generality by making the
transformation \( \Be_1 \rightarrow \Be_1 e^{i \phi} = \rhocap, \Be_2 \rightarrow \Be_2 e^{i\phi} = \phicap \), the field is
now fully determined

\begin{dmath}\label{eqn:ringField:220}
F
=
\frac{\lambda}{4 \pi \epsilon_0 R}
\lr{
\lr{ \tilde{z} \Be_3 + \tilde{r} \rhocap -\frac{v i}{c} } A
- \lr{
\rhocap + \frac{v}{c} \lr{ \tilde{z} \Be_{3} \rhocap + \tilde{r} i } } B
}.
\end{dmath}

The field directions are nicely parameterized as multivector expresssions, with the relative weightings in different directions scaled by the position dependent integral coefficients of \cref{eqn:ringField:260}.
The multivector field can be separated into its respective electric and magnetic components by inspection

\begin{dmath}\label{eqn:ringField:240}
\begin{aligned}
\BE &=
\gpgradeone{F}
=
\frac{\lambda}{4 \pi R \epsilon_0} \lr{ \tilde{z} A \Be_3 + \rhocap( \tilde{r} A - B) } \\
\BH &=
\inv{\eta_0} \gpgradeone{-I F}
=
\frac{\lambda v}{4 \pi R } \lr{ -\Be_3 \lr{ A + \tilde{r} B } - \phicap \tilde{z} A },
\end{aligned}
\end{dmath}

which, as expected, shows that the static charge distribution \( \rho \propto \lambda \) only contributes to the electric field, and the static current distribution \( \BJ \propto v \lambda \) only contributes to the magnetic field.
See
\cref{fig:ringFieldE:ringFieldEFig1}, \cref{fig:ringFieldH:ringFieldHFig1}
for plots of the electric and magnetic field directional variation near \( \tilde{z} = 0 \), and \cref{fig:ringFieldH:ringFieldHFig2} for larger \( z \) where the azimuthal component of the field dominates.
\imageFigure{../figures/GAelectrodynamics/ringFieldEFig1}{Electric field direction for circular charge density distribution near \( z = 0 \).}{fig:ringFieldE:ringFieldEFig1}{0.3}
\imageFigure{../figures/GAelectrodynamics/ringFieldHFig1}{Magnetic field direction for circular current density distribution near \( z = 0 \).}{fig:ringFieldH:ringFieldHFig1}{0.3}
\imageFigure{../figures/GAelectrodynamics/ringFieldHFig2}{Magnetic field for larger \( z \).}{fig:ringFieldH:ringFieldHFig2}{0.3}

\makeproblem{Magnetic sources on a ring.}{problem:ringField:300}{
Given a constant (magnitude) multivector current on a ring \( J = I \lambda_m \delta(z) \delta(r - R) ( c - v \Be_2 e^{i\phi}), i = \Be_{12} \), show that the field is
\begin{equation*}
F = \frac{\lambda_m c}{4 \pi R} \lr{
\lr{ \tilde{z} i + \tilde{r} \phicap \Be_{3} + \frac{v}{c} \Be_3 } A
+
\lr{ \Be_{3}\phicap + \frac{v}{c} \lr{ \tilde{z} \rhocap - \tilde{r} \Be_3 } } B
}.
\end{equation*}
} % problem

%}

         \subsubsection{Ampere's law.  Two current sources.}
            %
% Copyright � 2018 Peeter Joot.  All Rights Reserved.
% Licenced as described in the file LICENSE under the root directory of this GIT repository.
%
%{
%\input{../latex/blogpost.tex}
%\renewcommand{\basename}{ampereExample}
%%\renewcommand{\dirname}{notes/phy1520/}
%\renewcommand{\dirname}{notes/ece1228-electromagnetic-theory/}
%%\newcommand{\dateintitle}{}
%%\newcommand{\keywords}{}
%
%\input{../latex/peeter_prologue_print2.tex}
%
%\usepackage{peeters_layout_exercise}
%\usepackage{peeters_braket}
%\usepackage{peeters_figures}
%\usepackage{siunitx}
%%\usepackage{mhchem} % \ce{}
%%\usepackage{macros_bm} % \bcM
%%\usepackage{macros_qed} % \qedmarker
%\usepackage{txfonts} % \ointclockwise
%
%\beginArtNoToc
%
%\generatetitle{Ampere's law example.  Two currents.}
%%\chapter{Ampere's law example}
\label{chap:ampereExample}

Let's try using Ampere's law as stated in \cref{thm:amperes:280} two compute the field at a point in the blue region
illustrated in
\cref{fig:amperesLawBetweenTwoCurrents:amperesLawBetweenTwoCurrentsFig1}.
This represents a pair of z-axis electric currents of magnitude \( I_1, I_2 \) flowing through the \( z = 0 \) points \( \Bp_1, \Bp_2 \) on the x-y plane.

\mathImageFigure{../figures/GAelectrodynamics/amperesLawBetweenTwoCurrentsFig1}{Magnetic field between two current sources.}{fig:amperesLawBetweenTwoCurrents:amperesLawBetweenTwoCurrentsFig1}{0.3}{amperesLawMultiplePoints.nb}

Solving the system with superposition, let's consider first one source flowing through \( \Bp = (p_x, p_y, 0) \) with current \( \BJ = \Be_3 I_\txte \delta( x - p_x) \delta( y - p_y ) \), and evaluate the field due to this source at the point \( \Br \).
With
only magnetic sources in the multivector current, Ampere's law takes the form
\begin{equation}\label{eqn:ampereExample:20}
\ointctrclockwise_{\partial A} d\Bx\, F = -I \int_A dA\, \Be_3 (-\eta \BJ) = I \eta I_\txte.
\end{equation}
The field \( F \) must be a bivector satisfying \( d\Bx \cdot F = 0 \).  The circle is parameterized by
\begin{dmath}\label{eqn:ampereExample:40}
\Br = \Bp + R \Be_1 e^{i\phi},
\end{dmath}
so
\begin{equation}\label{eqn:ampereExample:60}
d\Bx = R \Be_2 e^{i\phi} d\phi = R \phicap d\phi.
\end{equation}
With the line element having only a \( \phicap \) component, \( F \) must be a bivector proportional to \( \Be_3 \rcap \).
Let \( F = F_0 \Be_{31} e^{i\phi} \), where \( F_0 \) is a scalar, so that \( d\Br F \) is a constant multiple of the unit pseudoscalar
\begin{dmath}\label{eqn:ampereExample:80}
\int_0^{2\pi} d\Br F
=
R F_0 \int_0^{2\pi} \Be_2 e^{i\phi} d\phi \Be_{31} e^{i\phi}
=
R F_0 \int_0^{2\pi} \Be_{231} e^{-i\phi} e^{i\phi} d\phi
=
2 \pi I R F_0,
\end{dmath}
so
\begin{dmath}\label{eqn:ampereExample:100}
F_0 =
\inv{I 2 \pi R} I I_\txte
=
 \frac{I_\txte}{2 \pi R}.
\end{dmath}
The field strength relative to the point \( \Bp \) is
\begin{dmath}\label{eqn:ampereExample:120}
F
=
 \frac{\eta I_\txte}{2 \pi R} \Be_3 \rcap
=
 \frac{\eta I_\txte}{2 \pi R} \Be_3 \rcap.
\end{dmath}

Switching to an origin relative coordinate system, removing the \( z = 0 \) restriction for \( \Br \) and \( \Bp_k \), and summing over both currents, the total field at any point \( \Br \) strictly between the currents is
\begin{dmath}\label{eqn:ampereExample:140}
F
=
\sum_{k = 1,2}
 \frac{\eta I_k}{2 \pi} \Be_3 \inv{\Be_3 \lr{ \Be_3 \wedge \lr{ \Br - \Bp_k}}}
=
\sum_{k = 1,2}
 \frac{\eta I_k}{2 \pi} \inv{ \Be_3 \wedge \lr{ \Br - \Bp_k} }.
\end{dmath}
The bivector nature of a field with only electric current density sources is naturally represented by the wedge product \( \Be_3 \wedge \lr{ \Br - \Bp_k} \) which is a vector product of \( \Be_3 \) and the projection of \( \Br - \Bp_k \) onto the x-y plane.

%}
%\EndNoBibArticle

      \section{Dynamics.}
         \subsection{Inverting Maxwell's equation.}
            %
% Copyright � 2018 Peeter Joot.  All Rights Reserved.
% Licenced as described in the file LICENSE under the root directory of this GIT repository.
%
%{
%%%\input{../latex/blogpost.tex}
%%%\renewcommand{\basename}{jefimenkosEquations}
%%%%\renewcommand{\dirname}{notes/phy1520/}
%%%\renewcommand{\dirname}{notes/ece1228-electromagnetic-theory/}
%%%%\newcommand{\dateintitle}{}
%%%%\newcommand{\keywords}{}
%%%
%%%\input{../latex/peeter_prologue_print2.tex}
%%%
%%%\usepackage{peeters_layout_exercise}
%%%\usepackage{peeters_braket}
%%%\usepackage{peeters_figures}
%%%\usepackage{siunitx}
%%%%\usepackage{mhchem} % \ce{}
%%%%\usepackage{macros_bm} % \bcM
%%%%\usepackage{macros_qed} % \qedmarker
%%%%\usepackage{txfonts} % \ointclockwise
%%%
%%%\newcommand{\dotBJ}[0]{\mathbf{\dot{J}}}
%%%
%%%\beginArtNoToc
%%%
%%%\generatetitle{Inverting Maxwell's equation}
%%%%\chapter{Inverting Maxwell's equation}
\label{chap:jefimenkosEquations}

Maxwell's equation (\cref{dfn:isotropicMaxwells:680}) is invertable using the Green's function for the spacetime gradient \cref{thm:greensFunctionSpacetimeGradient:120}.  That solution is

\maketheorem{Jefimenkos solution.}{thm:jefimenkosEquations:n}{
The solution of Maxwell's equation is given by
\begin{equation*}
F(\Bx, t)
=
F_0(\Bx, t)
+
\inv{4 \pi}
\int dV'
\lr{
   \frac{\rcap}{r^2} J(\Bx', t_r)
   +
   \inv{c r} \lr{ 1 + \rcap } \dispdot{J}(\Bx', t_r)
},
\end{equation*}
where \( F_0(\Bx, t) \) is any specific solution of the homogenoous equation \( \lr{ \spacegrad + (1/c) \partial_t } F_0 = 0 \),
time derivatives are denoted by overdots, and all times are evaluated at the retarded time \( t_r = t - r/c \).
When expanded in terms of the electric and magnetic fields (ignoring magnetic sources), the non-homoegenous portion of this solution is known as the
Jefimenkos' equations \citep{griffiths1999introduction}.
\begin{dmath}\label{eqn:jefimenkosEquations:100}
\begin{aligned}
\BE &=
\inv{4 \pi}
\int dV'
\lr{
\frac{\rcap}{\epsilon r} \lr{
\frac{\rho(\Bx', t_r)}{r} + \frac{\dispdot{\rho}(\Bx', t_r) }{c} }
   - \frac{\eta }{ c r } \dotBJ(\Bx', t_r)
} \\
\BH &=
\inv{4 \pi}
\int dV'
\lr{
   \frac{1}{c r} \dotBJ(\Bx', t_r)
+
   \frac{1}{r^2} \BJ(\Bx', t_r)
} \cross \rcap,
\end{aligned}
\end{dmath}
%which checks against Griffiths.
} % theorem

The full solution is
\begin{dmath}\label{eqn:jefimenkosEquations:20}
F(\Bx, t)
= F_0(\Bx, t)
+ \int dV' dt' G(\Bx - \Bx', t - t') J(\Bx', t')
= F_0(\Bx, t)
+
\inv{4\pi}
\int dV' dt'
\lr{
   \lr{
   - \frac{\rcap}{r^2} \PD{r}{}
   + \frac{\rcap}{r}
   + \inv{c r} \PD{t}{}
   }
   \delta( -r/c + t - t' )
}
J(\Bx', t')
\end{dmath}
where \( \Br = \Bx - \Bx', r = \Norm{\Br} \) and \( \rcap = \Br/r \).
With the help of \cref{eqn:derivativeOfDeltaFunction:140}, the derivatives in the Green's function \cref{eqn:jefimenkosEquations:20} can be evaluated, and
the convolution reduces to
\begin{dmath}\label{eqn:jefimenkosEquations:40}
\int dt' G(\Bx - \Bx', t - t') J(\Bx', t')
=
\inv{4 \pi}
\evalbar{
\lr{
\frac{\rcap}{r^2} J(\Bx', t_r)
-
\frac{\rcap}{r} \lr{ -\inv{c} } \frac{d}{dt_r} J(\Bx', t_r)
+
\inv{c r} \frac{d}{dt_r} J(\Bx', t_r)
}
}{t_r = t - r/c}.
\end{dmath}

There have been lots of opportuntites to mess up with signs and factors of \( c \), so let's expand this out explicitly for a non-magnetic current source \( J = \rho/\epsilon - \eta \BJ \).
%, and check the results against Jefimenko's equations found in
%\citep{griffiths1999introduction}.
Neglect the contribution of the homogeneous solution \( F_0 \), and utilizing our freedom to
insert a no-op grade 1,2 selection operation, that
removes any scalar and pseudoscalar terms that are neccessarily killed over the full integration range, we find
\begin{dmath}\label{eqn:jefimenkosEquations:80}
F =
\inv{4 \pi}
\int dV'
\gpgrade{
   \frac{\rcap}{r^2}
\lr{ \frac{\rho}{\epsilon} - \eta \BJ }
   +
   \inv{c r} \lr{ 1 + \rcap } \lr{ \frac{\dispdot{\rho}}{\epsilon} - \eta \dotBJ }
}{1,2}
=
\inv{4 \pi}
\int dV'
\lr{
   \frac{\rcap}{\epsilon r^2} \rho
   - \eta \frac{\rcap}{r^2} \wedge \BJ
   - \frac{\eta}{ c r } \dotBJ
   + \inv{c r} \rcap \frac{\dispdot{\rho}}{\epsilon}
   - \frac{\eta}{c r} \rcap \wedge \dotBJ
}
=
\inv{4 \pi}
\int dV'
\lr{
   \frac{\rcap}{\epsilon r^2} \rho
   + \frac{\rcap \dispdot{\rho}}{\epsilon c r}
   - \frac{\eta \dotBJ}{ c r }
   - I \frac{\eta }{c r} \rcap \cross \dotBJ
   - I \frac{\eta}{r^2} \rcap \cross \BJ
}.
\end{dmath}

As \( F = \BE + I \eta \BH \), the respective electric and magnetic fields by inspection.  After re-inserting the space and time parameters that we suppressed temporarily, the proof is complete.

The disadvantages of separating the field and current components into their constituent components is also made painfully obvious by the complexity of the conventional statement of the solution compared to the equivalent multivector form.

%}
%%%\EndArticle

      \section{Energy and momentum.}
         \subsection{Field energy and momentum density and the energy momentum tensor.}
            %
% Copyright © 2017 Peeter Joot.  All Rights Reserved.
% Licenced as described in the file LICENSE under the root directory of this GIT repository.
%

In conventional electromagnetism the energy and momentum density of the fields are

\index{energy density}
\index{momentum density}
\index{Poynting vector}
\index{energy flux}
\begin{dmath}\label{eqn:poyntingF:20}
\begin{aligned}
\calE &= \inv{2} \lr{ \BD \cdot \BE + \BB \cdot \BH } \\
\bcP c &= \inv{c} \BS = \inv{c} \BE \cross \BH.
\end{aligned}
\end{dmath}

where \( \calE \) is the energy density, \( \BS \) is the Poynting vector representing energy flux through a surface per unit time, and \( \bcP \) is the momentum density of the fields.
In geometric algebra, it is arguably more natural to write the Poynting vector as a bivector-vector dot product

\begin{dmath}\label{eqn:poyntingF:1100}
\BS = \lr{ I \BH } \cdot \BE,
\end{dmath}

however, we can do better, relating both the
energy and momentum densities to a single multivector formed from the products of the electrodynamic field \( F \) with its reverse \( F^\dagger \)

\boxedEquation{eqn:poyntingF:60}{
T(1) \equiv \inv{2} \epsilon F F^\dagger = \calE + \bcP c = \calE + \frac{\BS}{c},
}

so
\begin{dmath}\label{eqn:poyntingF:40}
\begin{aligned}
\calE &= \inv{2} \epsilon \gpgradezero{ F F^\dagger } \\
\bcP c &= \inv{2} \epsilon \gpgradeone{ F F^\dagger } \\
\BS &= \inv{2 \eta} \gpgradeone{ F F^\dagger }.
\end{aligned}
\end{dmath}

This dispenses with any requirement to refer to electric or magnetic field components in isolation.

Expanding \( T(1) \) in terms of \( \BE, \BH \) gives

\begin{dmath}\label{eqn:poyntingF:80}
T(1)
=
\inv{2} \epsilon F F^\dagger
=
\inv{2} \epsilon \lr{ \BE + I \eta \BH } \lr{ \BE - I \eta \BH }
=
\inv{2} \epsilon \lr{ \BE^2 + \eta^2 \BH^2 }
+
\inv{2} I \epsilon \eta \lr{ \BH \BE - \BE \BH }
=
\inv{2} \lr{ \BD \cdot \BE + \BH \cdot \BB }
+
\frac{I}{c} \BH \wedge \BE
=
\inv{2} \lr{ \BD \cdot \BE + \BH \cdot \BB }
+
\frac{1}{c} \BE \cross \BH
=
\calE + \frac{\BS}{c},
\end{dmath}

as claimed in \cref{eqn:poyntingF:20} and \cref{eqn:poyntingF:60}.  \( T(1) \) has one scalar component, and three vector components, and represents four
four of the sixteen components of a larger energy-momentum tensor.  The geometric algebra form of the complete energy-momentum tensor is
\index{energy-momentum tensor}
\index{Maxwell stress tensor}

%\begin{dmath}\label{eqn:poyntingF:760}
\boxedEquation{eqn:poyntingF:760}{
T(a) = \inv{2} \epsilon F a F^\dagger,
}
%\end{dmath}

where \( a \) is one of \( 1, \Be_1, \Be_2 \) or \( \Be_3 \), or any linear combination of these 0,1 grade multivector elements.  Observe that \( T(a) \) is a linear operator with respect to any parameter \( a = \alpha + \Ba, \alpha \in \bbR, \Ba \in \bbR^3 \).  \( T(a) \) only 0 and 1 grade elements, which we have seen for scalar parameters \( a \).  We will see by direct expansion that this is also the case for vector parameters.
Such an expansion of \( T(\Be_k) \) is harder to do algebaicly than \cref{eqn:poyntingF:80}
\footnote{Such an expansion is a worthwhile problem to develop GA manipulation skills.  The reader is encouraged to try this independently first, and to refer to
\cref{chap:stressTensorAlgebraically}
for hints if required.}.
On the other hand, it is easy to expand the energy-momentum tensor \( T(a) \) symbolically by brute force using a GA computer algebra package.
A Mathematica expansion of the elements of \( T(a) \) gives:

\begin{subequations}
\label{eqn:poyntingF:800}
\begin{equation}\label{eqn:poyntingF:1020}
\begin{aligned}
T(1)
&= \frac{\epsilon}{2} \lr{E_1^2 + E_2^2 + E_3^2} + \frac{\epsilon \eta^2}{2} \lr{H_1^2 + H_2^2 + H_3^2} \\
&+ \Be_1 \eta \epsilon \lr{E_2 H_3 - E_3 H_2} \\
&+ \Be_2 \eta \epsilon \lr{E_3 H_1 - E_1 H_3} \\
&+ \Be_3 \eta\epsilon \lr{E_1 H_2 - E_2 H_1}
\end{aligned}
\end{equation}
\begin{equation}\label{eqn:poyntingF:1040}
\begin{aligned}
T(\Be_1)
&= \eta \epsilon \lr{E_3 H_2 - E_2 H_3} \\
& + \frac{1}{2} \Be_1 \epsilon \lr{E_1^2 - E_2^2 - E_3^2} + \frac{\epsilon \eta^2}{2} \lr{ H_1^2 -  H_2^2 -  H_3^2} \\
& + \Be_2 \epsilon \lr{E_1 E_2 + \eta^2 H_1 H_2} \\
& + \Be_3 \epsilon \lr{E_1 E_3 + \eta^2 H_1 H_3}
\end{aligned}
\end{equation}
\begin{equation}\label{eqn:poyntingF:1060}
\begin{aligned}
T(\Be_2)
&= \eta \epsilon \lr{E_1 H_3 - E_3 H_1} \\
& + \Be_1 \epsilon \lr{E_1 E_2 + \eta^2 H_1 H_2} \\
& + \frac{1}{2} \Be_2 \epsilon \lr{-E_1^2 + E_2^2 - E_3^2 } + \frac{\epsilon \eta^2}{2} \lr{-H_1^2 +  H_2^2 -  H_3^2} \\
& + \Be_3 \epsilon \lr{E_2 E_3 + \eta^2 H_2 H_3}
\end{aligned}
\end{equation}
\begin{equation}\label{eqn:poyntingF:1080}
\begin{aligned}
T(\Be_3)
&= \eta \epsilon \lr{E_2 H_1 - E_1 H_2} \\
& + \Be_1 \epsilon \lr{E_1 E_3 + \eta^2 H_1 H_3} \\
& + \Be_2 \epsilon \lr{E_2 E_3 + \eta^2 H_2 H_3} \\
& + \frac{1}{2} \Be_3 \epsilon \lr{-E_1^2 - E_2^2 + E_3^2 } + \frac{\epsilon \eta^2}{2} \lr{ -H_1^2 -  H_2^2 + H_3^2}
\end{aligned}
\end{equation}
\end{subequations}

The components of the multivectors \( T(a) \) that we are calling the energy-momentum tensor, are more conventionally written out
as a symmetric tensor \( \Theta^{ij} \) as follows

\begin{dmath}\label{eqn:poyntingF:840}
\begin{aligned}
\Theta^{00} &= \frac{\epsilon}{2} \lr{ \BE^2 + \eta^2 \BH^2 } \\
\Theta^{0i} &= \inv{c} \lr{ \BE \cross \BH } \cdot \Be_i \\
\Theta^{ij} &= -\epsilon \lr{ E_i E_j + \eta^2 H_i H_j - \inv{2} \delta_{ij} \lr{ \BE^2 + \eta^2 \BH^2 } }.
\end{aligned}
\end{dmath}

The names and notation for this tensor components varies.
\( \Theta^{\alpha\beta}, \alpha, \beta \in \setlr{0,1,2,3} \) as defined in \cref{eqn:poyntingF:840} is called the symmetric
stress tensor \citep{jackson1975cew},
whereas other authors call this the energy-momentum tensor and express it as \( T^{\alpha\beta} \) \citep{landau1980classical}, \citep{doran2003gap}.
The sign conventions and notation for the spatial components \( \Theta^{ij}, i, j \in \setlr{1,2,3} \) vary as well, but all authors appear to call this subset the Maxwell stress tensor.
The Maxwell stress tensor is written as \( \sigma_{ij} (=-\Theta^{ij}) \) \citep{landau1980classical}, or as
\( T_{ij} (=-\Theta^{ij}) \)
\citep{griffiths1999introduction},
\citep{jackson1975cew}.

The symmetric stress tensor components of \cref{eqn:poyntingF:840}
are related to the multivector representation expanded in \cref{eqn:poyntingF:800} by

\begin{dmath}\label{eqn:poyntingF:820}
\begin{aligned}
&\gpgradezero{ T(1) }
%= \calE
=
{\Theta_0}^0 = \Theta^{00} \\
&\gpgradeone{ T(1) } \cdot \Be_i
%= \frac{\BS}{c} \cdot \Be_i
= {\Theta_0}^i = \Theta^{0i} \\
&\gpgradezero{ T(\Be_i) }
%= -\frac{\BS}{c} \cdot \Be_i
= {\Theta_i}^0 = -\Theta^{i0} \\
&\gpgradeone{ T(\Be_i) } \cdot \Be_j = {\Theta_i}^j = -\Theta^{ij}.
\end{aligned}
\end{dmath}

The Maxwell stress tensor components \( \gpgradeone{T(\Be_k)} \) can be found expressed in a dyadic notation (\citep{griffiths1999introduction}, \citep{jackson1975cew}) as follows

\begin{dmath}\label{eqn:poyntingF:1140}
\gpgradeone{ T(\Ba) }
=
\sum_i a_i \gpgradeone{ T(\Be_i) }
=
\sum_{i,j} a_i \lr{ \gpgradeone{ T(\Be_i) } \cdot \Be_j } \Be_j
=
\sum_{i,j} a_i T_{ij} \Be_j
\equiv
\Ba \cdot \lrT,
\end{dmath}

so that
\begin{dmath}\label{eqn:poyntingF:1160}
\lr{ \Ba \cdot \lrT } \cdot \Bb
=
\sum_{i,j} a_i T_{ij} b_j.
\end{dmath}

With the Maxwell stress tensor parameterized by a vector, we don't really have any need for this dyadic notation, but it is worth mentioning to
understand how the two formalisms are related.

The complete specification of the energy-momentum tensor for a parameter \( a = \alpha + \Ba = \alpha + \sum_k a_k \Be_k \) is

%\begin{dmath}\label{eqn:poyntingF:1120}
\boxedEquation{eqn:poyntingF:1120}{
T(\alpha + \Ba)
=
\alpha \lr{
   \calE + \frac{\BS}{c}
}
-
\Ba \cdot \frac{\BS}{c}
+ \BT(\Ba),
%+ \gpgradeone{ T(\Ba) }.
}
%\end{dmath}

where the shorthand \( \BT(\Ba) = \gpgradeone{T(\Ba)} \) has been introduced for the Maxwell stress tensor.


         \subsection{Poynting's theorem (prerequisites.)}
            %
% Copyright © 2017 Peeter Joot.  All Rights Reserved.
% Licenced as described in the file LICENSE under the root directory of this GIT repository.
%
%{
%\subsubsection{Poynting theorem prerequistes.}
Poynting's theorem is a set of conservation relationships between relating space and time change of energy density and momentum density, or more generally between related components of the energy momentum tensor.
The most powerful way of stating Poynting's theorem using geometric algebra requires a few new concepts, differential operator valued linear functions, and the adjoint.

\makedefinition{Differential operator valued multivector functions.}{dfn:poyntingTheoremRewrite:1180}{
Given a multivector valued linear functions of the form \( f(x) = A x B \), where \( A, B, x \) are multivectors, and a linear operator \( D \) such as \( \spacegrad, \partial_t \), or \( \spacegrad + (1/c) \partial_t \), the operator valued linear function \( f(D) \) is defined as
\begin{equation*}
f(D)
= A \lroverarrow{D} B
= (A \loverarrow{D}) B + A (\roverarrow{D} B),
\end{equation*}
where \( \lroverarrow{D} \) indicates that \( D \) is acting bidirectionally to the left and to the right.
} % definition

Perhaps counter intuitively, using operator valued parameters for the energy momentum tensor \( T \) or the Maxwell stress tensor \( \BT \) will be particularly effective to express the derivatives of the tensor.  There are a few cases of interest, all related to evaluation of the tensor with a parameter value of the spacetime gradient.
\index{\(T(\partial_t)\)}
\index{\(T(\spacegrad)\)}
\index{\(\BT(\spacegrad)\)}
\maketheorem{Energy momentum tensor operator parameters.}{thm:poyntingTheoremRewrite:1200}{
\begin{equation*}
\begin{aligned}
T((1/c)\partial_t) &= \inv{c} \PD{t}{T(1)} = \inv{c} \PD{t}{} \lr{ \calE + \frac{\BS}{c} } \\
\gpgradezero{T(\spacegrad)} &= - \spacegrad \cdot \frac{\BS}{c} \\
\gpgradeone{T(\spacegrad)} &= \BT(\spacegrad) = \sum_{k = 1}^3 \lr{ \spacegrad \cdot \BT(\Be_k) } \Be_k.
\end{aligned}
\end{equation*}
} % theorem

We will proceed to prove each of the results of
\cref{thm:poyntingTheoremRewrite:1200} in sequence, starting with the time partial, which is a scalar operator
\begin{dmath}\label{eqn:poyntingTheoremRewrite:1220}
T(\partial_t)
=
\frac{\epsilon}{2}
 F \lroverarrow{\partial_t} F^\dagger
=
\frac{\epsilon}{2}
\lr{
 (\partial_t F) F^\dagger
+
 F (\partial_t F^\dagger)
}
=
\frac{\epsilon}{2}
\partial_t
 F F^\dagger
=
\partial_t T(1).
\end{dmath}

To evaluate the tensor at the gradient we have to take care of order.  This is easiest in a scalar selection where we may cyclically permute any multivector factors
\begin{dmath}\label{eqn:poyntingTheoremRewrite:1240}
\gpgradezero{T(\spacegrad)}
=
\frac{\epsilon}{2}
\gpgradezero{
 F \lrspacegrad F^\dagger
}
=
\frac{\epsilon}{2}
\gpgradezero{
 \spacegrad F^\dagger F
}
=
\frac{\epsilon}{2} \spacegrad \gpgradeone{ F^\dagger F },
\end{dmath}
but
\begin{dmath}\label{eqn:poyntingTheoremRewrite:1260}
F^\dagger F
=
\lr{ \BE - I \eta \BH } \lr{ \BE + I \eta \BH }
=
\BE^2 + \eta^2 \BH^2 + I \eta \lr{ \BE \BH - \BH \BE }
=
\BE^2 + \eta^2 \BH^2 - 2 \eta \BE \cross \BH.
\end{dmath}
Plugging \cref{eqn:poyntingTheoremRewrite:1260} into \cref{eqn:poyntingTheoremRewrite:1240} proves the result.

Finally, we want to evaluate the Maxwell stress tensor of the gradient
\begin{dmath}\label{eqn:poyntingTheoremRewrite:260}
\BT(\spacegrad)
=
\sum_{k = 1}^3 \Be_k \lr{ \BT(\spacegrad) } \cdot \Be_k
=
\sum_{k,m = 1}^3 \Be_k \partial_m \lr{ \BT(\Be_m) \cdot \Be_k }
=
\sum_{k,m = 1}^3 \Be_k \partial_m \lr{ \BT(\Be_k) \cdot \Be_m }
=
\sum_{k = 1}^3 \Be_k \lr{ \spacegrad \cdot \BT(\Be_k) },
\end{dmath}
as claimed.

Will want to integrate \( \BT(\spacegrad) \) over a volume, which is essentially a divergence operation.
\maketheorem{Divergence integral for the Maxwell stress tensor.}{thm:poyntingTheoremRewrite:1281}{
\begin{equation*}
\int_V dV\, \BT(\spacegrad)
=
\int_{\partial V} dA\, \BT(\ncap).
\end{equation*}
} % theorem

To prove \cref{thm:poyntingTheoremRewrite:1281}, we make use of the symmetric property of the Maxwell stress tensor
\begin{dmath}\label{eqn:poyntingTheoremRewrite:320}
\int_V dV\, \BT(\spacegrad)
=
\sum_k \int_V dV\, \Be_k \spacegrad \cdot \BT(\Be_k)
=
\sum_k \int_{\partial V} dA\, \Be_k \ncap \cdot \BT(\Be_k)
=
\sum_{k,m} \int_{\partial V} dA\, \Be_k n_m {\BT(\Be_k) \cdot \Be_m}
=
\sum_{k,m} \int_{\partial V} dA\, \Be_k n_m {\BT(\Be_m) \cdot \Be_k}
=
\sum_{k} \int_{\partial V} dA\, \Be_k {\BT(\ncap) \cdot \Be_k}
=
\int_{\partial V} dA\, \BT(\ncap),
\end{dmath}
as claimed.

Finally, before stating Poynting's theorem, we want to introduce the concept of an adjoint.

%
% Copyright � 2018 Peeter Joot.  All Rights Reserved.
% Licenced as described in the file LICENSE under the root directory of this GIT repository.
%
\index{\(\overbar{A}(x)\)}
\makedefinition{Adjoint.}{dfn:poyntingTheorem:1120}{
The \textit{adjoint} \( \overbar{A}(x) \) of a linear operator \( A(x) \) is defined implicitly by the scalar selection
\begin{equation*}
\gpgradezero{ y \overbar{A}(x) } =
\gpgradezero{ x A(y) }.
\end{equation*}
} % definition

The adjoint of the energy momentum tensor is particularly easy to calculate.
\maketheorem{Adjoint of the energy momentum tensor.}{thm:poyntingTheorem:1140}{
The \textit{adjoint of the energy momentum tensor} is
\begin{equation*}
\overbar{T}(x) =
\frac{\epsilon}{2} F^\dagger x F.
\end{equation*}
The adjoint \( \overbar{T} \) and \( T \) satisfy the following relationships
\begin{equation*}
\begin{aligned}
\gpgradezero{\overbar{T}(1)} &= \gpgradezero{T(1)} = \calE \\
\gpgradeone{\overbar{T}(1)} &= -\gpgradeone{T(1)} = -\frac{\BS}{c} \\
\gpgradezero{\overbar{T}(\Ba)} &= -\gpgradezero{T(\Ba)} = \Ba \cdot \frac{\BS}{c} \\
\gpgradeone{\overbar{T}(\Ba)} &= \gpgradeone{T(\Ba)} = \BT(\Ba).
\end{aligned}
\end{equation*}
} % theorem

Using
the cyclic scalar selection permutation property \(\gpgradezero{ABC} = \gpgradezero{CAB}\) we form
\begin{dmath}\label{eqn:poyntingTheoremRewrite:1160}
\gpgradezero{ x T(y) }
=
\frac{\epsilon}{2} \gpgradezero{ x F y F^\dagger }
=
\frac{\epsilon}{2} \gpgradezero{ y F^\dagger x F }.
\end{dmath}
Referring back to \cref{dfn:poyntingTheorem:1120} we see that the adjoint must have the stated form.
Proving the grade selection relationships of \cref{eqn:poyntingTheoremRewrite:1160} has been left as
an exercise for the reader.  A brute force symbolic algebra proof using Mathematica is also available in \itemRef{GAelectrodynamics}{stressEnergyTensorValues.nb}.
%We can now state the adjoint form of \cref{thm:poyntingTheorem:1040}.

As in \cref{thm:poyntingTheoremRewrite:1200},
the adjoint may also be evaluated with differential operator parameters.

\index{\(\overbar{T}(x)\)}
\maketheorem{Adjoint energy-momentum tensor.}{thm:poyntingTheoremRewrite:1280}{
\begin{equation*}
\begin{aligned}
\gpgradezero{\overbar{T}((1/c)\partial_t)} &= \inv{c} \PD{t}{T(1)} = \inv{c} \PD{t}{\calE} \\
\gpgradeone{\overbar{T}((1/c)\partial_t)} &= -\inv{c^2} \PD{t}{\BS} \\
\gpgradezero{\overbar{T}(\spacegrad)} &= \spacegrad \cdot \frac{\BS}{c} \\
\gpgradeone{\overbar{T}(\spacegrad)} &= \BT(\spacegrad).
\end{aligned}
\end{equation*}
} % theorem

The proofs of \cref{thm:poyntingTheoremRewrite:1280} are all fairly simple
\begin{dmath}\label{eqn:poyntingTheoremRewrite:1280}
\overbar{T}((1/c)\partial_t)
=
\inv{c} \frac{\epsilon}{2} \PD{t}{} \lr{ F^\dagger F }
=
\inv{c} \PD{t}{} \lr{ \calE - \frac{\BS}{c} }.
\end{dmath}
\begin{dmath}\label{eqn:poyntingTheoremRewrite:1300}
\gpgradezero{ \overbar{T}(\spacegrad) }
=
\gpgradezero{ 1 \overbar{T}(\spacegrad) }
=
\gpgradezero{ \spacegrad T(1) }
=
\spacegrad \cdot \gpgradeone{ T(1) }
=
\spacegrad \cdot \frac{\BS}{c}.
\end{dmath}
\begin{dmath}\label{eqn:poyntingTheoremRewrite:1320}
\gpgradeone{ \overbar{T}(\spacegrad) }
=
\sum_k \Be_k \lr{ \Be_k \cdot \gpgradeone{ \overbar{T}(\spacegrad) } }
=
\sum_k \Be_k \gpgradezero{ \Be_k \overbar{T}(\spacegrad) }
=
\sum_k \Be_k \gpgradezero{ \spacegrad T(\Be_k) }
=
\sum_k \Be_k \spacegrad \cdot \gpgradeone{ T(\Be_k) }
=
\sum_k \Be_k \spacegrad \cdot \BT(\Be_k)
=
\BT(\spacegrad).
\end{dmath}

\subsection{Poynting theorem.}

All the prerequisites for stating Poynting's theorem are now finally complete.
%
% Copyright � 2018 Peeter Joot.  All Rights Reserved.
% Licenced as described in the file LICENSE under the root directory of this GIT repository.
%
\maketheorem{Poynting's theorem (differential form.)}{thm:poyntingTheorem:1180}{
The adjoint energy momentum tensor of the spacetime gradient satisfies the following multivector equation
\begin{equation*}
\overbar{T}(\spacegrad + (1/c)\partial_t) = \frac{\epsilon}{2} \lr{ F^\dagger J + J^\dagger F }.
\end{equation*}
The multivector \( F^\dagger J + J^\dagger F \) can only have scalar and vector grades, since it equals its reverse.
This equation can be put into a form that is more obviously a conservation law by stating it as a set of
scalar grade identities
\begin{equation*}
\spacegrad \cdot \gpgradeone{ T(a) } + \inv{c} \PD{t}{} \gpgradezero{ T(a) }
=
\frac{\epsilon}{2} \gpgradezero{ a( F^\dagger J + J \dagger F) },
\end{equation*}
or as a pair of scalar and vector grade conservation relationships
%%which expands to the multivector equation
%\begin{equation*}
%\inv{c} \PD{t}{} \lr{ \calE - \frac{\BS}{c} }
%+ \spacegrad \cdot \frac{\BS}{c}
%+ \BT(\spacegrad)
%=
%-\inv{c} \lr{ \BE \cdot \BJ + \BH \cdot \BM }
%+
%\rho \BE + \epsilon \BE \cross \BM
%+
%\rho_\txtm \BH + \mu \BJ \cross \BH,
%\end{equation*}
%or as separate scalar and vector equations
\begin{equation*}
\begin{aligned}
\inv{c} \PD{t}{\calE} + \spacegrad \cdot \frac{\BS}{c} &= -\inv{c} \lr{ \BE \cdot \BJ + \BH \cdot \BM } \\
-\inv{c^2} \PD{t}{\BS} + \BT(\spacegrad) &= \rho \BE + \epsilon \BE \cross \BM + \rho_\txtm \BH + \mu \BJ \cross \BH.
\end{aligned}
\end{equation*}
Conventionally, only the scalar grade relating the time rate of change of the energy density to the flux of the Poynting vector, is called Poynting's theorem.
Here the more general multivector (adjoint) relationship is called \textit{Poynting's theorem}, which includes conservation laws relating for the field energy and momentum densities and conservation laws relating the Poynting vector components and the Maxwell stress tensor.
} % theorem


The conservation relationship of \cref{thm:poyntingTheorem:1180} follows from
\begin{dmath}\label{eqn:poyntingTheoremRewrite:1340}
F^\dagger \lr{ \lrspacegrad + \inv{c}\lroverarrow{\partial_t} } F
=
\lr{ \lr{ \spacegrad + \inv{c} \partial_t } F }^\dagger F
+
F^\dagger \lr{ \lr{ \spacegrad + \inv{c} \partial_t } F }
=
J^\dagger F + F^\dagger J.
\end{dmath}
The scalar form of
\cref{thm:poyntingTheorem:1180}
follows from
\begin{dmath}\label{eqn:poyntingTheoremRewrite:1360}
\gpgradezero{ a \overbar{T}(\spacegrad + (1/c) \partial_t) }
=
\gpgradezero{ (\spacegrad + (1/c) \partial_t) T(a) }
=
\spacegrad \cdot \gpgradeone{ T(a) } + \inv{c} \PD{t}{} \gpgradezero{ T(a) }.
\end{dmath}

We may use the scalar form of the theorem to extract the scalar grade, by setting \( a = 1 \), for which the right hand side
can be reduced to a single term
since scalars are reversion invariant
\begin{equation}\label{eqn:poyntingTheoremRewrite:1000}
\gpgradezero{ F^\dagger J }
=
\gpgradezero{ F^\dagger J }^\dagger
=
\gpgradezero{ J^\dagger F },
\end{equation}
so
\begin{dmath}\label{eqn:poyntingTheoremRewrite:960}
\spacegrad \cdot \gpgradeone{ T(1) }
+ \inv{c} \PD{t}{} \gpgradezero{ T(1) }
=
\spacegrad \cdot \frac{\BS}{c} + \inv{c} \PD{t}{\calE}
=
\frac{\epsilon}{2} \gpgradezero{ F^\dagger J + J^\dagger F }
=
\epsilon
\gpgradezero{ F^\dagger J }
=
\epsilon
\gpgradezero{
   \lr{ \BE -I \eta \BH }\lr{
      \eta \lr{ c \rho - \BJ } + I \lr{ c \rho_m - \BM }
   }
}
=
\epsilon
\lr{
   -\eta \BE \cdot \BJ -\eta \BH \cdot \BM
}
=
- \inv{c} \BE \cdot \BJ - \inv{c} \BH \cdot \BM,
\end{dmath}
which proves the claimed explicit expansion of the scalar grade selection of Poynting's theorem.

The left hand side of the vector grade selection follows by linearity using \cref{thm:poyntingTheoremRewrite:1280}
\begin{dmath}\label{eqn:poyntingTheoremRewrite:1380}
\gpgradeone{ \overbar{T}(\spacegrad + (1/c)\partial_t) }
=
\gpgradeone{ \overbar{T}(\spacegrad) + \overbar{T}((1/c)\partial_t) }
=
\BT(\spacegrad) - \inv{c^2} \PD{t}{\BS}.
\end{dmath}
The right hand side is a bit messier to simplify.
Let's do this in pieces by superposition, first considering just electric sources
\begin{dmath}\label{eqn:poyntingTheoremRewrite:80}
\frac{\epsilon}{2} \gpgradezero{ \Be_k \lr{ F^\dagger J + J^\dagger F} }
=
\frac{\epsilon \eta}{2}
\gpgradezero{ \Be_k \lr{ (\BE - I \eta \BH)(c \rho - \BJ)  + (c \rho - \BJ)( \BE + I \eta \BH )} }
=
\frac{1}{2c } \Be_k \cdot
\gpgradeone{ (\BE - I \eta \BH)(c \rho - \BJ)  + (c \rho - \BJ)( \BE + I \eta \BH ) }
=
\inv{c} \Be_k \cdot \lr{ c \rho \BE + I \eta \BH \wedge \BJ }
=
\inv{c} \Be_k \cdot \lr{ c \rho \BE - \eta \BH \cross \BJ }
=
\Be_k \cdot \lr{ \rho \BE + \mu \BJ \cross \BH },
\end{dmath}
and then magnetic sources
\begin{dmath}\label{eqn:poyntingTheoremRewrite:180}
\frac{\epsilon}{2 } \gpgradezero{ \Be_k \lr{ F^\dagger J + J^\dagger F} }
=
\frac{\epsilon }{2} \gpgradezero{ \Be_k \lr{ (\BE - I \eta \BH) I (c \rho_\txtm - \BM)  - I (c \rho_\txtm - \BM)( \BE + I \eta \BH )} }
=
\frac{\epsilon }{2} \Be_k \cdot \gpgradeone{ (I \BE + \eta \BH)(c \rho_\txtm - \BM)  + (c \rho_\txtm - \BM)( -I \BE + \eta \BH )}
=
\epsilon \Be_k \cdot \lr{ \eta c \rho_\txtm \BH - I \BE \wedge \BM }
=
\Be_k \cdot \lr{ \rho_\txtm \BH + \epsilon \BE \cross \BM }.
\end{dmath}
Jointly,
\cref{eqn:poyntingTheoremRewrite:1380}, \cref{eqn:poyntingTheoremRewrite:80}, \cref{eqn:poyntingTheoremRewrite:180} complete the proof.

The integral form of \cref{thm:poyntingTheorem:1180} submits nicely to physical interpretation.
%
% Copyright � 2018 Peeter Joot.  All Rights Reserved.
% Licenced as described in the file LICENSE under the root directory of this GIT repository.
%
\maketheorem{Poynting's theorem (integral form.)}{thm:poyntingTheoremRewrite:1420}{
\begin{dmath}\label{eqn:poyntingTheoremRewrite:1400}
\begin{aligned}
&\PD{t}{}
\int_V dV\, \calE 
=
-\int_{\partial V} dA\, \ncap \cdot \BS
-
\int_V dV \lr{
   \BJ \cdot \BE
   +
   \BM \cdot \BH
} \\
&
\int_V dV \lr{ \rho \BE + \BJ \cross \BB }
+ \int_V dV \lr{ \rho_\txtm \BH - \epsilon \BM \cross \BE }
=
-
\PD{t}{ }
\int_V dV \bcP
+
\int_{\partial V} dA\, \BT(\ncap).
\end{aligned}
\end{dmath}
} % theorem


Proof of \cref{thm:poyntingTheoremRewrite:1420} requires only the divergence theorem, \cref{thm:poyntingTheoremRewrite:1281}, and
\cref{dfn:poyntingF:1220}.

The scalar integral in \cref{thm:poyntingTheoremRewrite:1420}
relates the rate of change of total energy in a volume to the flux of the Poynting through the surface bounding the volume.
If the energy in the volume increases(decreases), then in a current free region, there must be Poynting flux into(out-of) the volume.
The direction of the Poynting vector is the direction that the energy is leaving the volume, but only the projection of the Poynting vector along the normal direction contributes to this energy loss.

The right hand side of the vector integral in \cref{thm:poyntingTheoremRewrite:1420} is a continuous representation of the Lorentz force
(or dual Lorentz force for magnetic charges),
the mechanical force on the charges in the volume.  This can be seen by setting \( \BJ = \rho \Bv \) (or \( \BM = \rho_\txtm \BM \))
\begin{dmath}\label{eqn:poyntingTheoremRewrite:1420}
\int_V dV \lr{ \rho \BE + \BJ \cross \BB }
=
\int_V dV\, \rho \lr{ \BE + \Bv \cross \BB }
=
\int_V dq \lr{ \BE + \Bv \cross \BB }.
\end{dmath}

As the field in the volume is carrying the (electromagnetic) momentum \( \Bp_{\textrm{em}} = \int_V dV \bcP \), we can identify the sum of the Maxwell stress tensor normal component over the bounding integral as time rate of change of the mechanical and electromagnetic momentum
\index{\(\Bp_{\textrm{mech}}\)}
\index{\(\Bp_{\textrm{em}}\)}
\boxedEquation{eqn:poyntingTheoremRewrite:1440}{
\frac{d\Bp_{\textrm{mech}}}{dt} + \frac{d\Bp_{\textrm{em}}}{dt} = \int_{\partial V} dA \BT(\ncap).
}

%}

         \subsection{Examples: Some static fields.}
            %
% Copyright © 2017 Peeter Joot.  All Rights Reserved.
% Licenced as described in the file LICENSE under the root directory of this GIT repository.
%

We've found solutions for a number of static charge and current distributions.

\begin{enumerate}[(a)]
\item For constant electric sources along the z-axis
(\cref{eqn:statics:180})
, with current \( \BJ \) moving with velocity \( \Bv = v \Be_3 \), the field had the form \( F = E \rhocap \lr{ 1 - \Bv/c } \).
\item For constant magnetic sources along the z-axis
(\cref{problem:statics:240})
, with current \( \BM \) moving with velocity \( \Bv = v \Be_3 \), the field had the form \( F = \eta H I \rhocap \lr{ 1 - \Bv/c } \).
\item For constant electric sources in the x-y plane
(\cref{eqn:statics:320})
, with current \( \BJ \) moving with velocity \( \Bv = v \Be_1 e^{i\theta}, i = \Be_{12} \), the field had the form \( F = E \Be_3 \lr{ 1 - \Bv/c } \).
\item For constant magnetic sources in the x-y plane
(\cref{problem:statics:241})
, with current \( \BM \) moving with velocity \( \Bv = v \Be_1 e^{i\theta}, i = \Be_{12} \), the field had the form \( F = \eta H i \lr{ 1 - \Bv/c } \).
\end{enumerate}

In all cases the field has the form \( F = A ( 1 - \Bv/c ) \), where \( A \) is either a vector or a bivector that anticommutes with the current velocity \( \Bv \), so the stress energy tensor \( T(1) \) has the form

\begin{dmath}\label{eqn:poyntingF:860}
T(1)
= \frac{\epsilon}{2} A ( 1 - \Bv/c )^2 A^\dagger
= \frac{\epsilon}{2} A A^\dagger ( 1 + \Bv/c )^2
= \frac{\epsilon}{2} A A^\dagger \lr{ 1 + \lr{ \frac{\Bv}{c} }^2 + 2 \frac{\Bv}{c} },
\end{dmath}

For the electric sources this is
\begin{dmath}\label{eqn:poyntingF:880}
\calE + \frac{\BS}{c} = \frac{\epsilon}{2} E^2 \lr{ 1 + \lr{ \frac{\Bv}{c} }^2 + 2 \frac{\Bv}{c} },
\end{dmath}

or
\begin{dmath}\label{eqn:poyntingF:900}
\begin{aligned}
\calE &= \frac{\epsilon}{2} E^2 \lr{ 1 + \lr{ \frac{\Bv}{c} }^2 } \\
\BS &= \epsilon E^2 \Bv.
\end{aligned}
\end{dmath}

For the magnetic sources this is
\begin{dmath}\label{eqn:poyntingF:920}
\calH + \frac{\BS}{c} = \frac{\mu}{2} H^2 \lr{ 1 + \lr{ \frac{\Bv}{c} }^2 + 2 \frac{\Bv}{c} },
\end{dmath}

or
\begin{dmath}\label{eqn:poyntingF:940}
\begin{aligned}
\calH &= \frac{\mu}{2} H^2 \lr{ 1 + \lr{ \frac{\Bv}{c} }^2 } \\
\BS &= \mu H^2 \Bv.
\end{aligned}
\end{dmath}

There are three terms in the multivector \( (1 -\Bv/c)^2 = 1 + \lr{ \ifrac{\Bv}{c} }^2 + 2 \ifrac{\Bv}{c} \).  For electric sources,
the first scalar term is due to the charge distribution, and provides the electric field contribution to the energy density.
The second scalar term is due to the current distribution, and provides the magnetic field contribution to the energy density.
The final vector term, proportional to the current velocity contributes to the Poynting vector, showing that the field momentum travels along the direction of the current in these static configurations.

         \subsection{Complex energy and power.}
            %
% Copyright © 2017 Peeter Joot.  All Rights Reserved.
% Licenced as described in the file LICENSE under the root directory of this GIT repository.
%
\subsection{Complex power.}
TODO.
%\index{complex power}

      \section{Lorentz force.}
         \subsection{Statement (REWRITE.)}
            %
% Copyright © 2017 Peeter Joot.  All Rights Reserved.
% Licenced as described in the file LICENSE under the root directory of this GIT repository.
%
We now wish to express the Lorentz force equation \cref{eqn:freespace:200} in its geometric algebra form.
Like the energy momentum tensor, there is value to introduce a multivector with energy and momentum components

\makedefinition{Energy momentum multivector.}{dfn:lorentzForce:300}{
For a particle with energy \( \calE \) and momentum \( \Bp \), we define the \textit{energy momentum multivector} as
\begin{equation*}
T = \calE + c \Bp.
\end{equation*}
} % definition

\makedefinition{Multivector charge.}{dfn:lorentzForce:280}{
We may define a \textit{multivector charge} that includes both the magnitude and velocity (relative to the speed of light) of the charged particle.
\begin{equation*}
Q = \int_V J dV,
\end{equation*}
where \( \BJ = \rho_\txte \Bv_\txte, \BM = \rho_\txtm \Bv_\txtm \).
For electric charges this is
\begin{equation*}
Q = q_\txte \lr{ 1 + \Bv_\txte/c },
\end{equation*}
and for magnetic charges
\begin{equation*}
Q = I q_\txtm \lr{ 1 + \Bv_\txtm/c },
\end{equation*}
where \( q_\txte = \int_V \rho_\txte dV, q_\txtm = \int_V \rho_\txtm dV \).
} % definition

With a multivector charge defined, the Lorentz force equation can be stated in terms of the total electromagnetic field strength
\maketheorem{Lorentz force and power.}{thm:lorentzForce:300}{
The respective power and force experienced by particles with electric (and/or magnetic) charges is described by
\cref{dfn:lorentzForce:280} is
\begin{equation*}
\inv{c} \frac{dT}{dt} = \gpgrade{ F Q^\dagger }{0,1} = \inv{2} \lr{ F^\dagger Q + F Q^\dagger }.
\end{equation*}
where \( \gpgradezero{dT/dt} = \ifrac{d\calE}{dt} \) is the power and \( \gpgradeone{dT/dt} = c \ifrac{d\Bp}{dt} \) is the force on the particle, and
\( Q^\dagger \) is the electric or magnetic charge/velocity multivector of \cref{dfn:lorentzForce:280}.
The conventional representation of the Lorentz force/power equations
\begin{equation*}
\begin{aligned}
\gpgradeone{ F Q^\dagger } &= \ddt{\Bp} = q \lr{ \BE + \Bv \cross \BB } \\
c \gpgradezero{ F Q^\dagger } &= \ddt{\calE} = q \BE \cdot \Bv.
\end{aligned}
\end{equation*}
%given by \cref{eqn:freespace:180}
may be recovered by grade selection operations.
For magnetic particles, such a grade selection gives
\begin{equation*}
\begin{aligned}
\gpgradeone{ F Q^\dagger } &= \frac{d\Bp}{dt} = q_\txtm \lr{ c \BB - \inv{c} \Bv_\txtm \cross \BE } \\
c \gpgradezero{ F Q^\dagger } &= \frac{d\calE}{dt} = \inv{\eta} q_\txtm \BB \cdot \frac{\Bv_\txtm}{c}.
\end{aligned}
\end{equation*}
} % theorem

To prove
\cref{thm:lorentzForce:300},
we can expand the multivector product
\( F q \lr{ 1 + \ifrac{\Bv}{c} } \) into its constituent grades
\begin{dmath}\label{eqn:lorentzForce:40}
q F \lr{ 1 + \frac{\Bv}{c} }
=
q
\lr{ \BE + I c \BB }
\lr{ 1 + \frac{\Bv}{c} }
=
q \BE
+ q I \BB \Bv
+ \frac{q}{c} \BE \Bv
+ q c I \BB
=
  \frac{q}{c} \BE \cdot \Bv
+ q \lr{ \BE + \Bv \cross \BB }
+ q \lr{ c I \BB + \inv{c} \BE \wedge \Bv }
+ q (I \BB) \wedge \Bv.
\end{dmath}

We see the (c-scaled) particle power relationship
\cref{eqn:freespace:220}
in the grade zero component and the Lorentz force \cref{eqn:freespace:220} in the grade 1 component.
A substitution \( q \rightarrow -I q_\txtm, \Bv \rightarrow \Bv_\txtm \), and subsequent grade 0,1 selection gives
\begin{dmath}\label{eqn:lorentzForce:320}
\gpgrade{
-I q_\txtm F \lr{ 1 + \frac{\Bv_\txtm}{c} }
}{0,1}
=
- I q_\txtm \lr{ c I \BB + \inv{c} \BE \wedge \Bv_\txtm }
- I q_\txtm I \BB \cdot \Bv_\txtm
=
q_\txtm \lr{ c \BB - \inv{c} \Bv_\txtm \cross \BE }
+
q_\txtm \BB \cdot \Bv_\txtm.
\end{dmath}
The grade one component of this multivector has the
required form for the dual Lorentz force equation
from \cref{thm:poyntingTheoremRewrite:1420}.
%, as determined from the conservation relationships for the energy momentum tensor in
%\cref{eqn:poyntingLorentzForce:140}.
Scaling the grade zero component by \( c \) completes the proof.

%FIXME: did I have an energy momentum tensor derivation of the time rate of change of energy for a magnetic charge density?

%As the electric and magnetic field contributions to the force are subsumed by the total electromagnetic field strength \( F \), \cref{thm:lorentzForce:300} puts the electric and magnetic fields on equal footing.

         \subsection{Constant magnetic field.}
            %
% Copyright © 2017 Peeter Joot.  All Rights Reserved.
% Licenced as described in the file LICENSE under the root directory of this GIT repository.
%

The Lorentz force equation that determines the dynamics of a charged particle in an external field \( F \) has been restated as a multivector differential equation, but how to solve such an equation is probably not obvious.
Given a constant external magnetic field, the Lorentz force equation is reduced to

\begin{dmath}\label{eqn:lorentzForce:60}
m \frac{d\Bv}{dt} = q (I \BB) \cdot \Bv,
\end{dmath}

or
\begin{dmath}\label{eqn:lorentzForce:80}
\begin{aligned}
\Omega &= -\frac{q I \BB}{m} \\
\frac{d\Bv}{dt} &= \Bv \cdot \Omega,
\end{aligned}
\end{dmath}

where \( \Omega \) is a bivector containing all the constant factors.

This can be solved by introducing a multivector integration factor \( R \) and its reverse \( R^\dagger \) on the left and right respectively

\begin{dmath}\label{eqn:lorentzForce:100}
R \frac{d\Bv}{dt} R^\dagger
= R \Bv \cdot \Omega R^\dagger
= \inv{2} R \lr{ \Bv \Omega - \Omega \Bv} R^\dagger
= \inv{2} R \Bv \Omega R^\dagger - \inv{2} R \Omega \Bv R^\dagger,
\end{dmath}

or
\begin{dmath}\label{eqn:lorentzForce:120}
0 =
R \frac{d\Bv}{dt} R^\dagger
+ \inv{2} R \Omega \Bv R^\dagger
- \inv{2} R \Bv \Omega R^\dagger
\end{dmath}

Let
\begin{dmath}\label{eqn:lorentzForce:140}
\dot{R} = R \Omega/2.
\end{dmath}

Since \( \Omega \) is a bivector \( \dot{R}^\dagger = -\Omega R^\dagger/2 \), so by chain rule

\begin{dmath}\label{eqn:lorentzForce:160}
0
=
\frac{d}{dt} \lr{
R \Bv R^\dagger
}.
\end{dmath}

The integrating factor has solution

\begin{dmath}\label{eqn:lorentzForce:180}
R = e^{\Omega t/2},
\end{dmath}

a ``complex exponential'', so the solution of \cref{eqn:lorentzForce:60} is

\begin{dmath}\label{eqn:lorentzForce:200}
\Bv(t) = e^{-\Omega t/2} \Bv(0) e^{\Omega t/2}.
\end{dmath}

The velocity of the charged particle traces out a helical path.
Any component of the initial velocity \( \Bv(0)_\perp \) perpendicular to the \( \Omega \) plane is untouched by this rotation operation, whereas components of the initial velocity \( \Bv(0)_\parallel \) that lie in the \( \Omega \) plane will trace out a circular path.
If \( \hat{\Omega} \) is the unit bivector for this plane, that velocity is

\begin{dmath}\label{eqn:lorentzForce:220}
\begin{aligned}
\Bv(0)_\parallel &= \lr{ \Bv(0) \cdot \hat{\Omega} } \hat{\Omega}^{-1} \\
\Bv(0)_\perp &= \lr{ \Bv(0) \wedge \hat{\Omega} } \hat{\Omega}^{-1} \\
\Bv(t) &= \Bv(0)_\parallel e^{\Omega t} + \Bv(0)_\perp.
\end{aligned}
\end{dmath}

A multivector integration factor method for solving the Lorentz force equation in constant external electric and magnetic fields can be found in \citep{hestenes1999nfc}.  Other examples, solved using a relativistic formulation of GA, can be found in \citep{doran2003gap},
\citep{hestenes1974properdynamics}, and
\citep{hestenes1974propermechanics}.

      \section{Polarization (REWRITE).}
         \subsection{Plane wave.}
            %
% Copyright © 2017 Peeter Joot.  All Rights Reserved.
% Licenced as described in the file LICENSE under the root directory of this GIT repository.
%
%{
\index{plane wave}
\index{polarization}
In a discussion of polarization, it is convenient to align the propagation direction along a fixed direction, usually the z-axis.
Setting \( \kcap = \Be_3, \beta z = \Bk \cdot \Bx \) in \cref{eqn:frequencydomainPlaneWaves:200} the plane wave representation of the field is

\begin{dmath}\label{eqn:polarization:20}
\begin{aligned}
F(\Bx, \omega) &= (1 + \Be_3) \BE e^{-j \beta z} \\
F(\Bx, t) &= \Real\lr{ F(\Bx, \omega) e^{j \omega t} }.
\end{aligned}
\end{dmath}

Here the imaginary \( j \) has no intrinsic geometrical interpretation, \( \BE = \BE_\txtr + j \BE_\txti \) is allowed to have complex values, and all components of \( \BE \) is perpendicular to the propagation direction (\( \Be_\txtr \cdot \Be_3 = \BE_\txti \cdot \Be_3 = 0 \)).
\index{Jones vector}
A common representation of the electric field components is the Jones vector \( (c_1, c_2) \), which specifies complex coefficients for the electric field phasor in each of the possible directions

\begin{dmath}\label{eqn:polarization:120}
\BE = c_1 \Be_1 + c_2 \Be_2,
\end{dmath}

where \( c_1, c_2 \) are complex valued, say

\begin{dmath}\label{eqn:polarization:140}
\begin{aligned}
c_1 &= \alpha_1 + j \beta_1 \\
c_2 &= \alpha_2 + j \beta_2.
\end{aligned}
\end{dmath}

The tuple \( (c_1, c_2) \) is called the Jones vector, and compactly encodes the geometry of the pattern that the electric field traces out in the transverse plane.


         \subsection{Circular polarization basis.}
            
\index{circular polarization}
\index{left circular polarization}
\index{right circular polarization}

The time domain field when written out explicitly in terms of the Jones vector components is

\begin{dmath}\label{eqn:polarization_circular:160}
F(\Bx, t) = (1 + \Be_3) \lr{
\lr{ \alpha_1 \Be_1 + \alpha_2 \Be_2 } \cos\lr{ \omega t - \beta z }
-\lr{ \beta_1 \Be_1 + \beta_2 \Be_2 } \sin\lr{ \omega t - \beta z }
}.
\end{dmath}

Linear, circular, and elliptical polarization patterns can be obtained by selecting specific values of the Jones vector, or equivalently by selecting specific values for the \( \alpha_1, \alpha_2, \beta_1, \beta_2 \) constants in the time domain representation of \cref{eqn:polarization_circular:160}.
In particular,
a field for which the
change in phase

\begin{dmath}\label{eqn:polarization_circular:520}
\phi = \omega t - \beta z
\end{dmath}

results in the electric field tracing out a (clockwise,counterclockwise) circle

\begin{dmath}\label{eqn:polarization_circular:180}
\begin{aligned}
\BE_\txtL &= \Abs{\BE} \lr{ \Be_1 \cos\phi + \Be_2 \sin\phi } = \Abs{\BE} \Be_1 \exp\lr{  \Be_{12} \phi } \\
\BE_\txtR &= \Abs{\BE} \lr{ \Be_1 \cos\phi - \Be_2 \sin\phi } = \Abs{\BE} \Be_1 \exp\lr{ -\Be_{12} \phi },
\end{aligned}
\end{dmath}

is referred to as having
(right,left) circular polarization.
There are different conventions for the polarization orientation, and here the IEEE antenna convention discussed in \citep{balanis1989advanced} are used.

%Fixme: flipped the orientation I was using, as I noticed after the fact that the figures in Balanis use an orientation with x-axis up and y-axis right!
% check that everything still looks correct.
% ( I didn't notice the inscribed X's in the polarization figures: https://en.wikibooks.org/wiki/Physics_Study_Guide/Vectors_and_scalars#How_to_draw_vectors_that_are_in_or_out_of_the_plane_of_the_page_.28or_board.29 )

The bivector exponential representation of the circularly polarized electric fields in \cref{eqn:polarization_circular:180} indicates that it is possible to represent arbitrary field polarization in a GA form that does not require any real part operation, as follows

\begin{dmath}\label{eqn:polarization_circular:200}
F = \lr{ 1 + \Be_3 } \Be_1 \lr{ \alpha_\txtL e^{i\phi} + \alpha_\txtR e^{-i\phi} },
\end{dmath}

where the constants \( \alpha_\txtL, \alpha_\txtR \) are both complex with respect to the unit bivector imaginary \( i = \Be_{12} \) representing the plane transverse to the propagation direction

\begin{dmath}\label{eqn:polarization_circular:220}
\begin{aligned}
\alpha_\txtL &= \alpha_{\txtL 1} + i \alpha_{\txtL 2} \\
\alpha_\txtR &= \alpha_{\txtR 1} + i \alpha_{\txtR 2}.
\end{aligned}
\end{dmath}

If a transformation from scalar to bivector imaginary \( j \rightarrow \Be_{12} = i \) is made in the Jones vector component representation of \cref{eqn:polarization:140},
then
the coefficients \cref{eqn:polarization_circular:220} of the circular polarization states are related to the Jones vector by (\cref{problem:polarization:1})

\begin{dmath}\label{eqn:polarization_circular:260}
\begin{aligned}
\alpha_\txtL &= \inv{2}\lr{ c_1 - i c_2 } \\
\alpha_\txtR &= \inv{2}\lr{ c_1 + i c_2 }^\dagger.
\end{aligned}
\end{dmath}


         \subsection{Linear polarization.}
            

Linear polarization is described by

\begin{dmath}\label{eqn:polarization_linearPolarization:280}
\begin{aligned}
\alpha_\txtL &= \inv{2}\Abs{\BE} \Be_1 e^{i(\psi + \theta)} \\
\alpha_\txtR &= \inv{2}\Abs{\BE} \Be_1 e^{i(\psi - \theta)},
\end{aligned}
\end{dmath}

or
\begin{dmath}\label{eqn:polarization_linearPolarization:300}
F = \lr{ 1 + \Be_3 } \Abs{\BE} \Be_1 e^{i\psi} \cos( \omega t - \beta z + \theta ),
\end{dmath}

where \( \theta \) is an arbitrary initial phase.  The electric field \( \BE \) traces out all the points along the line spanning the points between \( \pm \Be_1 e^{i\psi} \Abs{\BE} \), whereas the magnetic field \( \BH \) traces
out all the points along \( \pm \Be_2 e^{i\psi} \Abs{\BE}/\eta \) as illustrated (with \( \eta = 1 \)) in
\cref{fig:linearPolarization:linearPolarizationFig1}.
\imageFigure{../figures/GAelectrodynamics/linearPolarizationFig1}{Linear polarization.}{fig:linearPolarization:linearPolarizationFig1}{0.3}


         \subsection{Other phase dependence and energy momentum.}
            

The linear polarization of \cref{eqn:polarization:300} can be generalized from sinosoidal functions of the phase angle \cref{eqn:polarization:520}, to arbitrary functions, as in

\begin{dmath}\label{eqn:polarization:540}
F = \lr{ 1 + \Be_3 } \Abs{\BE} \Be_1 e^{i\psi} f(\phi).
\end{dmath}

For example, \( f(\phi) = e^{i\phi} \) would result in a circularly polarized state, and
a Gaussian modulation could be added into the mix with \( f(\phi) = e^{i \phi - (\phi/\sigma)^2/2 } \).

If the phase dependence of \cref{eqn:polarization:540} is a scalar function, then
the energy momentum multivector for the field can be calculated simply

\begin{dmath}\label{eqn:polarization:560}
\calE + \frac{\BS}{v}
=
\inv{2} \epsilon
F F^\dagger
=
\inv{2} \epsilon
\lr{ 1 + \Be_3 } \Abs{\BE}^2 \Be_1 \cancel{e^{i\psi}} f^2(\phi)
\cancel{e^{-i\psi} }
\Be_1
\lr{ 1 + \Be_3 }
=
\inv{2} \epsilon
\lr{ 1 + \Be_3 } \Abs{\BE}^2 \cancel{\Be_1} f^2(\phi)
\cancel{\Be_1 }
\lr{ 1 + \Be_3 }
=
\epsilon \lr{ 1 + \Be_3 } \Abs{\BE}^2 f^2(\phi),
\end{dmath}

where the projective property \( \lr{ 1 + \Be_3 }^2 = 2 \lr{ 1 + \Be_3 } \) was used in the final simplification.
The energy, and Poynting vectors are
\begin{dmath}\label{eqn:polarization:580}
\begin{aligned}
\calE &= \epsilon \Abs{\BE}^2 f^2(\phi) \\
\BS &= \inv{\eta} \Be_3 \Abs{\BE}^2 f^2(\phi).
\end{aligned}
\end{dmath}
% v epsilon = sqrt( epsilon^2/ (epsilon mu) ) = 1/eta

More care for this calculation is required if the phase function \( f(\phi) \) is multivector valued, since it may not commute with the \( \Be_1 \) and \( e^{i\psi} \) factors of \( F \).


         \subsection{Elliptical parameterization.}
            %
% Copyright © 2018 Peeter Joot.  All Rights Reserved.
% Licenced as described in the file LICENSE under the root directory of this GIT repository.
%
%{
An elliptical polarized electric field can be parameterized as
\begin{dmath}\label{eqn:polarization_elliptical:340}
\BE
=
E_a \Be_1 \cos\theta + E_b \Be_2 \sin\theta,
\end{dmath}
which corresponds to a Jones vector \( (E_a, -i E_b) \), or circular polarization coefficients with values
\begin{dmath}\label{eqn:polarization_elliptical:400}
\begin{aligned}
\alpha_\txtL &= \inv{2}\lr{ E_a - E_b } \\
\alpha_\txtR &= \inv{2}\lr{ E_a + E_b }.
\end{aligned}
\end{dmath}

Therefore an elliptically polarized field can be represented as
\begin{dmath}\label{eqn:polarization_elliptical:420}
F = \inv{2} (1 + \Be_3) \Be_1 \lr{ (E_a + E_b) e^{i\phi} + (E_a - E_b) e^{-i\phi} }.
\end{dmath}

An interesting variation of the elliptical polarization uses a hyperbolic parameterization.
If \( a, b \) are the semi-major/minor axes of the ellipse (i.e. \( a > b \)),
and \( \Ba = a \Be_1 e^{i\psi} \) is the vectoral representation of the semi-major axis (not necessarily placed along \( \Be_1 \)),
and \( e = \sqrt{1 - (b/a)^2} \) is the eccentricity of the ellipse,
then it can be shown (\citep{hestenes1999nfc})
that an elliptic parameterization can be written
in the compact form
\begin{dmath}\label{eqn:polarization_elliptical:360}
\Br(\phi)
=
e \Ba \cosh( \tanh^{-1}(b/a) + i \phi).
\end{dmath}

When the bivector imaginary \( i = \Be_{12} \) is used then
this parameterization is real and has only vector grades, so the electromagnetic field for a general elliptic wave has the form
\begin{dmath}\label{eqn:polarization_elliptical:380}
\begin{aligned}
F &= e E_a \lr{ 1 + \Be_3 } \Be_1 e^{ i \psi } \cosh\lr{ m + i \phi} \\
m &= \tanh^{-1}\lr{ E_b/E_a } \\
e &= \sqrt{1 - {(E_b/E_a)}^2 },
\end{aligned}
\end{dmath}
where \( E_a(E_b) \) are the magnitudes of the electric field components lying along the semi-major(minor) axes, and the propagation direction \( \Be_3 \) is normal to both the major and minor axis directions.
An elliptic electric field polarization is illustrated in \cref{fig:ellipticalPolarization:ellipticalPolarizationFig1}, where the vectors representing the major and minor axes are \( \BE_a = E_a \Be_1 e^{i\psi}, \BE_b = E_b \Be_1 e^{i\psi} \).
Observe that setting \( E_b = 0 \) results in the linearly polarized field of \cref{eqn:polarization_linearPolarization:300}.
\imageFigure{../figures/GAelectrodynamics/ellipticalPolarizationFig1}{Electric field with elliptical polarization.}{fig:ellipticalPolarization:ellipticalPolarizationFig1}{0.3}

Following the procedure of \cref{eqn:polarization_phaseAndEnergyMomentum:560}, the energy momentum of an elliptically polarized field is
\begin{dmath}\label{eqn:polarization_elliptical:600}
\calE + \frac{\BS}{v}
=
\inv{2} \epsilon
F F^\dagger
=
\inv{2} \epsilon
e^2 E_a^2 \lr{ 1 + \Be_3 } \Be_1 \cancel{e^{ i \psi }} \cosh\lr{ m + i \phi}
\cosh\lr{ m - i \phi}
\cancel{e^{ -i \psi } }
\Be_1
\lr{ 1 + \Be_3 }
=
\inv{2} \epsilon
e^2 E_a^2 \lr{ 1 + \Be_3 }
\lr{ \cosh(2m) + \cos(2 \phi) }
=
\inv{2} \epsilon
\lr{ 1 + \Be_3 }
\lr{ E_b^2 + 2 \lr{
E_a^2 - E_b^2
 } \cos^2 \phi }
.
\end{dmath}

The simplification above made use of the identity \( (1 - (b/a)^2) \cosh(2 \Atanh(b/a)) = 1 + (b/a)^2 \).
%
% $Assumptions = b > 0 && b < 1 && a > 0 && a > b;
% (1 - (b/a)^2) Cosh[2 ArcTanh[b/a]] // FullSimplify

%}

         \subsection{Pseudoscalar imaginary.}
            

The multivector \( 1 + \Be_3 \) acts as a projector, consuming any factors of \( \Be_3 \)

\begin{dmath}\label{eqn:polarization_pseudoscalarImaginary:440}
(1 + \Be_3) \Be_3
=
\Be_3 + \Be_3^2
=
1 + \Be_3.
\end{dmath}

This property allows all the bivector imaginaries \( i = \Be_{12} = \Be_3 I \) in \cref{eqn:polarization_circular:200} to be re-expressed in terms of the \R{3} pseudoscalar \( I = \Be_{123} \).  To illustrate this consider just the left circular polarized wave

\begin{dmath}\label{eqn:polarization_pseudoscalarImaginary:460}
F_\txtL
=
\lr{ 1 + \Be_3 } \Be_1 \alpha_\txtL e^{i\phi}
=
\lr{ 1 + \Be_3 } \Be_1 \alpha_\txtL \lr{ \cos\phi + \Be_3 I \sin\phi }
=
\lr{ 1 + \Be_3 } \Be_1 \alpha_\txtL \cos\phi
-\lr{ 1 + \Be_3 } \Be_3 \Be_1 \alpha_\txtL I \sin\phi
=
\lr{ 1 + \Be_3 } \Be_1 \alpha_\txtL e^{-I\phi}
=
\lr{ 1 + \Be_3 } \Be_1 \lr{ \alpha_{\txtL 1} + \Be_3 I\alpha_{\txtL 2}  } e^{-I\phi}
=
\lr{ 1 + \Be_3 } \Be_1 \lr{ \alpha_{\txtL 1} - I \alpha_{\txtL 2} } e^{-I\phi}.
\end{dmath}

This shows that the coefficients for the circular polarized states can be redefined using the pseudoscalar as an imaginary (in contrast to the bivector imaginary used in \cref{eqn:polarization_circular:220})
\begin{dmath}\label{eqn:polarization_pseudoscalarImaginary:480}
\begin{aligned}
\alpha_\txtL' &= \alpha_{\txtL 1} - I \alpha_{\txtL 2} \\
\alpha_\txtR' &= \alpha_{\txtR 1} - I \alpha_{\txtR 2},
\end{aligned}
\end{dmath}

so that the plane wave is
\begin{dmath}\label{eqn:polarization_pseudoscalarImaginary:500}
F = \lr{ 1 + \Be_3 } \Be_1 \lr{ \alpha_\txtL' e^{-I\phi} + \alpha_\txtR' e^{I\phi} }.
\end{dmath}

Like \cref{eqn:polarization_circular:200} this plane wave representation does not require taking any real parts.  The transverse plane in which the electric and magnetic fields lie is defined by the duality relation \( i = I \Be_3 \).

The energy momentum multivector for a wave described in terms of the pseudoscalar circular polarization states of \cref{eqn:polarization_pseudoscalarImaginary:500} is just

\begin{dmath}\label{eqn:polarization_pseudoscalarImaginary:620}
\calE + \frac{\BS}{v} =
\epsilon \lr{ 1 + \Be_3 } \lr{ \Abs{\alpha_\txtL'}^2 + \Abs{\alpha_\txtR'}^2 },
\end{dmath}

where the absolute value is computed using the reverse as the conjugation operation \( \Abs{z}^2 = z z^\dagger \).


         \subsection{Problems.}
            %
% Copyright © 2018 Peeter Joot.  All Rights Reserved.
% Licenced as described in the file LICENSE under the root directory of this GIT repository.
%
%{

\makeproblem{Circular polarization coefficients relationship to the Jones vector.}{problem:polarization:1}{
By substituting \cref{eqn:polarization_circular:220} into \cref{eqn:polarization_circular:200}, and comparing to \cref{eqn:polarization_circular:160},
show that the circular state coefficients have the following relationship to the Jones vector coordinates
\begin{equation*}
\begin{aligned}
\alpha_\txtL &= \lr{ \alpha_1 + \beta_2 }/2 + i \lr{ -\alpha_2 + \beta_1 }/2 \\
\alpha_\txtR &= \lr{ \alpha_1 - \beta_2 }/2 + i \lr{ -\alpha_2 - \beta_1 }/2,
\end{aligned}
\end{equation*}
and use this to prove \cref{eqn:polarization_circular:260}.
} % problem

\makeproblem{Pseudoscalar Jones vector.}{problem:polarization:2}{
With the Jones vector defined in terms of the \R{3} pseudoscalar
\begin{equation*}
\begin{aligned}
c_1 &= \alpha_1 + I \beta_1 \\
c_2 &= \alpha_2 + I \beta_2,
\end{aligned}
\end{equation*}
calculate the values \( \alpha_\txtL', \alpha_\txtR' \) of \cref{eqn:polarization_pseudoscalarImaginary:480} in terms of this Jones vector.
} % problem
%}

      \section{Transverse fields in a waveguide.}
         %
% Copyright © 2017 Peeter Joot.  All Rights Reserved.
% Licenced as described in the file LICENSE under the root directory of this GIT repository.
%
%original ideas from gabookII/electrodynamics/transverseField.tex:
We now wish to consider more general solutions to the source free Maxwell's equation than the plane wave solutions derived in \cref{chap:planewavesMultivector}.
One way of tackling this problem is to assume the solution exists, but ask how the field components that lie strictly along the propagation direction are related to the transverse components of the field.
Without loss of generality, it can be assumed that the propagation direction is along the z-axis.

\maketheorem{Transverse and propagation field components.}{thm:transverseField:288}{
If \( \Be_3 \) is the
propagation direction, the components of a field \( F \) in the propagation direction and in the transverse plane are respectively
\begin{equation*}
\begin{aligned}
F_z &= \inv{2} \lr{ F + \Be_3 F \Be_3 } \\
F_t &= \inv{2} \lr{ F - \Be_3 F \Be_3 },
\end{aligned}
\end{equation*}
where \( F = F_z + F_t \).
} % theorem

To determine the components of the field that lie in the propagation direction and transverse planes, we state the field in the propagation direction, building it from the electric and magnetic field projections along the z-axis
\begin{dmath}\label{eqn:transverseField:108}
F_z
=
\lr{ \BE \cdot \Be_3 }
 \Be_3
+ I \eta \lr{ \BH \cdot \Be_3 } \Be_3
=
\inv{2}
\lr{ \BE \Be_3 + \Be_3 \BE }
 \Be_3
+ \inv{2} I \eta \lr{ \BH \Be_3 + \Be_3 \BH } \Be_3
=
\inv{2}
\lr{ \BE + \Be_3 \BE \Be_3 }
+ \inv{2} I \eta \lr{ \BH + \Be_3 \BH \Be_3 }
=
\inv{2} \lr{ F + \Be_3 F \Be_3 }.
\end{dmath}
The difference \( F - F_z \) is the transverse component
\begin{dmath}\label{eqn:transverseField:308}
F_t
= F - F_z
=
F -
\inv{2} \lr{ F + \Be_3 F \Be_3 }
=
\inv{2} \lr{ F - \Be_3 F \Be_3 },
\end{dmath}
as claimed.

We wish to split the gradient into transverse and propagation direction components.

\makedefinition{Transverse and propagation direction gradients.}{dfn:transverseField:328}{
Define the \textit{propagation direction gradient} as \( \Be_3 \partial_z \), and
\textit{transverse gradient} by
\begin{equation*}
\spacegrad_t = \spacegrad - \Be_3 \partial_z.
\end{equation*}
} % definition

Given this definition, we seek to show that

%
% Copyright � 2018 Peeter Joot.  All Rights Reserved.
% Licenced as described in the file LICENSE under the root directory of this GIT repository.
%
\maketheorem{Transverse and propagation field solutions.}{thm:transverseField:348}{
Given a field propagating along the z-axis (either forward or backwards), with angular frequency \( \omega \), represented by the real part of
\begin{equation*}
F(x, y, z, t) = F(x, y) e^{j \omega t \mp j k z},
\end{equation*}
the field components that solve the source free Maxwell's equation are related by
\begin{equation*}
\begin{aligned}
F_t &= j \inv{ \frac{\omega}{c} \mp k \Be_3 } \spacegrad_t F_z \\
F_z &= j \inv{ \frac{\omega}{c} \mp k \Be_3 } \spacegrad_t F_t.
\end{aligned}
\end{equation*}
Written out explicitly, the transverse field component expands as
\begin{equation*}
\begin{aligned}
\BE_t &=
\frac{j}{{\frac{\omega}{c}}^2 - k^2}
\lr{
   \pm k \spacegrad_t E_z
   + \frac{\omega \eta}{c} \Be_3 \cross \spacegrad_t H_z
}
\\
\eta \BH_t &=
\frac{j}{{\frac{\omega}{c}}^2 - k^2}
\lr{
   \pm k \eta \spacegrad_t H_z
   -
   \frac{\omega}{c}
   \Be_3 \cross \spacegrad_t E_z
}.
\end{aligned}
\end{equation*}
} % theorem


To prove we first insert the assumed phasor representation into Maxwell's equation, which gives
\begin{equation}\label{eqn:transverseField:summaryMax2}
\lr{\spacegrad_t + j \lr{ \frac{\omega}{c} \mp k \Be_3 } } F(x,y) = 0.
\end{equation}

Dropping the \( x, y \) dependence for now (i.e.  \( F(x, y) \rightarrow F \), we find a relation between the transverse gradient of \( F \) and the propagation direction gradient of \( F \)

\begin{dmath}\label{eqn:transverseField:148}
\spacegrad_t F = - j \lr{ \frac{\omega}{c} \mp k \Be_3 } F.
\end{dmath}
From this we now seek to determine the relationships between \( F_t \) and \( F_z \).

Since \( \spacegrad_t \) has no \( \xcap, \ycap \) components, \( \Be_3 \) anticommutes with the transverse gradient
\begin{dmath}\label{eqn:transverseField:168}
\Be_3 \spacegrad_t = - \spacegrad_t \Be_3,
\end{dmath}
but commutes with \( 1 \mp \Be_3 \).
%In \cref{eqn:transverseField:168} it is implied that the action of \( \spacegrad_t \) is on everything to its right.
This means that
\begin{dmath}\label{eqn:transverseField:188}
\inv{2} \lr{ \spacegrad_t F \pm \Be_3 \lr{ \spacegrad_t F } \Be_3 }
=
\inv{2} \lr{ \spacegrad_t F \mp \spacegrad_t \Be_3 F \Be_3 }
=
\spacegrad_t
\inv{2} \lr{ F \mp \Be_3 F \Be_3 },
\end{dmath}
or
\begin{dmath}\label{eqn:transverseField:208}
\begin{aligned}
\inv{2} \lr{ \spacegrad_t F + \Be_3 \lr{ \spacegrad_t F } \Be_3 } &= \spacegrad_t F_t \\
\inv{2} \lr{ \spacegrad_t F - \Be_3 \lr{ \spacegrad_t F } \Be_3 } &= \spacegrad_t F_z,
\end{aligned}
\end{dmath}
so Maxwell's equation \cref{eqn:transverseField:148} becomes
\begin{dmath}\label{eqn:transverseField:228}
\begin{aligned}
\spacegrad_t F_t &= - j \lr{ \frac{\omega}{c} \mp k \Be_3 } F_z \\
\spacegrad_t F_z &= - j \lr{ \frac{\omega}{c} \mp k \Be_3 } F_t.
\end{aligned}
\end{dmath}

Provided \( \omega^2 \ne (k c)^2 \), these can be inverted.
Such an inversion allows an application of the transverse gradient to whichever one
of \( F_z, F_t \) is known, to compute the other, as stated in
\cref{thm:transverseField:348}.

The relation for \( F_t \) in
\cref{thm:transverseField:348}
is usually stated in terms of the electric and magnetic fields.
To perform that expansion, we must first evaluate the multivector inverse explicitly
\begin{dmath}\label{eqn:transverseField:348}
\begin{aligned}
F_z &= j \frac{ \frac{\omega}{c} \pm k \Be_3 }{ \lr{\frac{\omega}{c}}^2 - k^2 } \spacegrad_t F_t \\
F_t &= j \frac{ \frac{\omega}{c} \pm k \Be_3 }{ \lr{\frac{\omega}{c}}^2 - k^2 } \spacegrad_t F_z.
\end{aligned}
\end{dmath}
so that we are in position to expand most of the terms in the numerator
\begin{dmath}\label{eqn:transverseField:268}
\lr{ \frac{\omega}{c} \pm k \Be_3 } \spacegrad_t F_z
=
-\lr{ \Be_3 \frac{\omega}{c} \pm k } \spacegrad_t \Be_3 F_z
=
\lr{ \pm k - \Be_3 \frac{\omega}{c} } \spacegrad_t \lr{ E_z + I \eta H_z }
=
\lr{
   \pm k \spacegrad_t E_z
   + \frac{\omega \eta}{c} \Be_3 \cross \spacegrad_t H_z
}
+ I \lr{
   \pm k \eta \spacegrad_t H_z
   -
   \frac{\omega}{c}
   \Be_3 \cross \spacegrad_t E_z
},
\end{dmath}
from which the transverse electric and magnetic fields stated in
\cref{thm:transverseField:348} can be read off.
A similar expansion for \( \BE_z, \BH_z \) in terms of \( \BE_t, \BH_t \) is also possible.

%There is considerably more complexity required to express the transverse field in terms of separate electric and magnetic components
%compared to the equivalent total transverse field expression of...

\makeproblem{Transverse electric and magnetic field components.}{problem:transverseField:1}{
Fill in the missing details in the steps of \cref{eqn:transverseField:268}.
} % problem

\makeproblem{Propagation direction components.}{problem:transverseField:2}{
Perform an expansion like \cref{eqn:transverseField:268} to find
\( \BE_z, \BH_z \) in terms of \( \BE_t, \BH_t \).
} % problem

      \section{Boundary value conditions.}
         %
% Copyright © 2017 Peeter Joot.  All Rights Reserved.
% Licenced as described in the file LICENSE under the root directory of this GIT repository.
%
\index{boundary values}
The difference in the normal and tangential components of the electromagnetic field spanning a surface on which there are
a surface current or surface charge density can be related to those surface sources.

These relationships can be determined by integrating Maxwell's equation over the
pillbox configuration illustrated in \cref{fig:ps3Problem1Pillbox:ps3Problem1PillboxFig1}.

%\imageFigure{../figures/ece1228-electromagnetic-theory/ps3Problem1PillboxFig1}{Pillbox integration volume.}{fig:ps3Problem1Pillbox:ps3Problem1PillboxFig1}{0.2}
\imageFigure{../figures/GAelectrodynamics/pillboxIntegrationVolumeFig1}{pillboxIntegrationVolumeFigllbox integration volume.1}{fig:ps3Problem1Pillbox:ps3Problem1PillboxFig1}{0.3}

An assumption that the sources are primarily constrained to the surface can be written as

\begin{dmath}\label{eqn:boundary:20}
J = J_s \delta(y),
\end{dmath}

where the \( y \) coordinate is locally normal to the surface at any given point.
In terms of the scalar and vector potentials, such a surface source model is

\begin{dmath}\label{eqn:boundary:40}
J = \lr{ \eta\lr{ v \rho_s - \BJ_s } + I \lr{ v \rho_{ms} - \BM_s } }
\delta(y).
\end{dmath}

It will be
simplest to demonstrate the boundary relationships in the frequency domain, where Maxwell's equation can be written as either

\begin{subequations}
\label{eqn:boundary:60}
\begin{dmath}\label{eqn:boundary:80}
\spacegrad F = -j k F + J,
\end{dmath}

or

\begin{dmath}\label{eqn:boundary:100}
\spacegrad I F = -j k I F + I J.
\end{dmath}
\end{subequations}

Application of contraction operations gives

\begin{subequations}
\label{eqn:boundary:120}
\begin{dmath}\label{eqn:boundary:140}
\spacegrad \cdot F
= \gpgrade{-j k F + J}{0,1}
= -j k \BE + \eta( v \rho_s - \BJ_s ) \delta(y)
\end{dmath}
\begin{dmath}\label{eqn:boundary:160}
\spacegrad \cdot (I F)
= \gpgrade{-j k I F + I J}{0,1}
= j k \eta \BH - ( v \rho_{ms} - \BM_s ) \delta(y).
\end{dmath}
\end{subequations}

Each of these contraction operations can be evaluated over the pillbox volume above using the divergence theorem, however, the delta function integrals are problematic.
Those integrals dependent on \( \eta \) and \( v \) which vary across the surface, but are also dependent on the delta function surface contribution, which is valid at only the surface.
Consider the vector potential term for electric sources as an example, where the volume integral of that term is

\begin{dmath}\label{eqn:boundary:180}
-\int dV \eta \BJ_s \delta(y)
=
-\int_{y=0}^{h/2} \int dA \eta_2 \BJ_s \delta(y)
-\int_{y=-h/2}^0 \int dA \eta_1 \BJ_s \delta(y).
\end{dmath}

The delta function is only well defined when integrated across the \( y = 0 \) point.
This problem can be overcome by applying grade selection operations to each of the components of \cref{eqn:boundary:120}, and then rearranging so that all the medium specific contributions to the integrals are factored away from the delta functions

\begin{subequations}
\label{eqn:boundary:200}
\begin{dmath}\label{eqn:boundary:220}
\gpgradezero{\spacegrad \cdot \lr{ \epsilon F}} = \rho_s \delta(y)
\end{dmath}
\begin{dmath}\label{eqn:boundary:240}
\gpgradeone{\spacegrad \cdot \lr{ \inv{\eta} F}} = -j \frac{k}{\eta} \BE - \BJ_s \delta(y)
\end{dmath}
\begin{dmath}\label{eqn:boundary:260}
\gpgradezero{\spacegrad \cdot \lr{ I \inv{v}F}} = - \rho_{ms} \delta(y)
\end{dmath}
\begin{dmath}\label{eqn:boundary:280}
\gpgradeone{\spacegrad \cdot \lr{ I F}} = j k \eta \BH + \BM_s \delta(y).
\end{dmath}
\end{subequations}

Each of the grade selections picks off one of \( \BD, \BB, \BH \) or \( \BE \), so this could have been obtained directly from the conventional set of individual Maxwell equations, however, it is instructional to see how to work with the complete electromagnetic field \( F \).
This also provides a method of evaluating the boundary conditions that is both coordinate free, and uses the same integral form for all the boundary conditions.

Application of the multivector \R{3} divergence theorem, as stated informally in \cref{eqn:stokesTheoremCore:1881d} gives

\begin{subequations}
\label{eqn:boundary:300}
\begin{dmath}\label{eqn:boundary:320}
\gpgradezero{\int dV \ncap \cdot \lr{ \epsilon F}} = \Delta A \rho_s
\end{dmath}
\begin{dmath}\label{eqn:boundary:340}
\gpgradeone{\int dV \ncap \cdot \lr{ \inv{\eta} F}} = -j \omega \int_y dy \int dA \BD - \Delta A \BJ_s
\end{dmath}
\begin{dmath}\label{eqn:boundary:360}
\gpgradezero{\int dV \ncap \cdot \lr{ I \inv{v}F}} = - \Delta A \rho_{ms}
\end{dmath}
\begin{dmath}\label{eqn:boundary:380}
\gpgradeone{\int dV \ncap \cdot \lr{ I F}} = j \omega \int_y dy \int dA \BB + \Delta A \BM_s
\end{dmath}
\end{subequations}

The \( y \) (normal) integral components of the volume integrals are all assumed to vanish as \( \Delta y \rightarrow 0 \), leaving

%\begin{dmath}\label{eqn:boundary:420}
\boxedEquation{eqn:boundary:420}{
\begin{aligned}
\gpgradezero{\ncap (\epsilon_2 F_2 - \epsilon_1 F_1) } &= \rho_s \\
\gpgradeone{\ncap \lr{\inv{\eta_2} F_2 - \inv{\eta_1} F_1 } } &= - \BJ_s \\
\gpgradezero{\ncap I \lr{ \inv{v_2}F_2 - \inv{v_1} F_1 } } &= - \rho_{ms} \\
\gpgradeone{\ncap I (F_2 - F_1)} &= \BM_s
\end{aligned}
}
%\end{dmath}

These can, of course, each be written in terms of the constituent fields if desired

%\begin{dmath}\label{eqn:boundary:440}
%\begin{aligned}
%\ncap \cdot (\BD_2 - \BD_1) &= \rho_s \\
%I \ncap \wedge \lr{ \BH_2 - \BH_1 } &= - \BJ_s \\
%-\ncap \cdot (\BB_2 - \BB_1) &= - \rho_{ms} \\
%I \ncap \wedge (\BE_2 - \BE_1)} &= \BM_s,
%\end{aligned}
%\end{dmath}
%
%or
%
\begin{dmath}\label{eqn:boundary:460}
\begin{aligned}
\ncap \cdot \lr{ \BD_2 - \BD_1 } &= \rho_s \\
\ncap \cross \lr{ \BH_2 - \BH_1 } &= \BJ_s \\
\ncap \cdot \lr{ \BB_2 - \BB_1 } &= \rho_{ms} \\
\ncap \cross \lr{ \BE_2 - \BE_1 } &= -\BM_s.
\end{aligned}
\end{dmath}

The crazy jumble of dot products, cross products and field components in this conventional statement of the boundary conditions is seen to follow systematically from Maxwell's equation \cref{eqn:boundary:80}, and reflects the fact that the components of Maxwell's equation have to be treated individually by grade when evaluating the boundary integrals.

      \section{Multivector potential.}
         %
% Copyright � 2018 Peeter Joot.  All Rights Reserved.
% Licenced as described in the file LICENSE under the root directory of this GIT repository.
%
%{
%\input{../latex/blogpost.tex}
%\renewcommand{\basename}{gaugeTransformation}
%%\renewcommand{\dirname}{notes/phy1520/}
%\renewcommand{\dirname}{notes/ece1228-electromagnetic-theory/}
%%\newcommand{\dateintitle}{}
%%\newcommand{\keywords}{}
%
%\input{../latex/peeter_prologue_print2.tex}
%
%\usepackage{peeters_layout_exercise}
%\usepackage{peeters_braket}
%\usepackage{peeters_figures}
%\usepackage{siunitx}
%%\usepackage{mhchem} % \ce{}
%%\usepackage{macros_bm} % \bcM
%\usepackage{macros_qed} % \qedmarker
%%\usepackage{txfonts} % \ointclockwise
%
%%\newcommand{\dLambertian}[0]{\Box}
%\newcommand{\dLambertian}[0]{\square}
%
%\newcommand{\stgrad}[0]{\lr{ \spacegrad + \inv{c} \PD{t}{}}}
%\newcommand{\conjstgrad}[0]{\lr{ \spacegrad - \inv{c} \PD{t}{}}}
%
%\beginArtNoToc
%
%\generatetitle{Multivector potentials.}
%%\chapter{Multivector potentials.}
\label{chap:gaugeTransformation}

Conventional electromagnetism utilizes scalar and vector potentials, so it is reasonable to expect that
the desired multivector representation of the potential is a grade 0,1 multivector.
A potential representation with grades 2,3 works for (fictitous) magnetic sources, so we may generally
allow a multivector potential to have any grades.  Such a potential is related to the field as follows.

\makedefinition{Multivector potential.}{thm:generalPotential:80}{
The electromagnetic field strength \( F \) for a \textit{multivector potential} \( A \) is
\begin{equation*}
F = \gpgrade{\conjstgrad A}{1,2}.
\end{equation*}
} % definition

Before unpacking \( \conjstgrad A \), we want to label the
different grades of the multivector potential, and do so in a way that is consisent with the conventional
potential representation of the electric and magnetic fields.
\makedefinition{Multivector potential representation.}{dfn:unpackStaticPotential:80}{
Let
%\label{eqn:gaugeTransformation:1111}
\begin{equation*}
A =
      - \phi
      + c \BA
      + \eta I \lr{ -\phi_m + c \BF },
\end{equation*}
where
\begin{enumerate}
\item \( \phi \) is the scalar potential \si{V} (Volts).
\item \( \BA \) is the vector potential \si{W/m} (Webers/meter).
\item \( \phi_m \) is the scalar potential for (fictitious) magnetic sources \si{A} (Amperes).
\item \( \BF \) is the vector potential for (fictitious) magnetic sources \si{C} (Coulombs).
\end{enumerate}
} % definition
This specific breakdown of \( A \) into scalar and vector potentials, and dual (pseudoscalar and bivector) potentials has been chosen to match SI conventions, specifically those of \citep{balanis2005antenna} (which includes fictitious magnetic sources.)

We can now express the fields in terms of the potentials.

\maketheorem{Fields and the potential wave equations.}{thm:generalPotential:40}{
In terms of the potential components, the electric field vector and the magnetic field bivector are
\begin{equation*}
\begin{aligned}
\BE &=
\gpgrade{\conjstgrad A}{1}
=
   - \spacegrad \phi
   - \PD{t}{\BA}
   - \inv{\epsilon} \spacegrad \cross \BF \\
I \eta \BH &=
\gpgrade{\conjstgrad A}{2}
=
   I \eta
   \lr{
      - \spacegrad \phi_\txtm
      - \PD{t}{\BF}
      + \inv{\mu} \spacegrad \cross \BA
   }
.
\end{aligned}
\end{equation*}
The potentials are related to the sources by
\begin{equation*}
\begin{aligned}
\dLambertian
\phi &= -\frac{\rho}{\epsilon} - \PD{t}{} \lr{ \spacegrad \cdot \BA + \inv{c^2} \PD{t}{\phi} } \\
\dLambertian
\BA &= -\mu \BJ + \spacegrad \lr{ \spacegrad \cdot \BA + \inv{c^2} \PD{t}{\phi} } \\
\dLambertian
\BF &= - \epsilon \BM + \spacegrad \lr{ \spacegrad \cdot \BF + \inv{c^2} \PD{t}{\phi_\txtm} } \\
\dLambertian
\phi_\txtm &= -\frac{\rho_\txtm}{\mu} - \PD{t}{} \lr{ \spacegrad \cdot \BF + \inv{c^2} \PD{t}{\phi_\txtm} }
\end{aligned}
\end{equation*}
} % theorem

To prove \cref{thm:generalPotential:40} we start by expanding \( (\spacegrad - (1/c)\partial_t) A \) using
\cref{dfn:unpackStaticPotential:80} and then group by grade to find
\begin{dmath}\label{eqn:gaugeTransformation:1111}
\begin{aligned}
\conjstgrad A
&=
\conjstgrad \lr{  - \phi
      + c \BA
      + \eta I \lr{ -\phi_m + c \BF } } \\
&=
- \spacegrad \phi + c \spacegrad \cdot \BA + c \spacegrad \wedge \BA + \inv{c} \PD{t}{\phi} - \PD{t}{\BA} \\
&\quad + I \eta
\lr{
- \spacegrad \phi_\txtm + c \spacegrad \cdot \BF + c \spacegrad \wedge \BF + \inv{c} \PD{t}{\phi_\txtm} - \PD{t}{\BF}
} \\
&=
c \spacegrad \cdot \BA
+ \inv{c} \PD{t}{\phi}
\\
&
+
\mathLabelBox[ labelstyle={below of=m\themathLableNode, below of=m\themathLableNode} ]
{
   - \spacegrad \phi
   - \PD{t}{\BA}
   - \inv{\epsilon} \spacegrad \cross \BF
}
{
\(\BE\)
}
+
\mathLabelBox[ labelstyle={below of=m\themathLableNode, below of=m\themathLableNode} ]
{
   I \eta
   \lr{
      - \spacegrad \phi_\txtm
      - \PD{t}{\BF}
      + \inv{\mu} \spacegrad \cross \BA
   }
}
{\(I \eta \BH\)
} \\
&
+ I \eta\lr{
  c \spacegrad \cdot \BF
+ \inv{c} \PD{t}{\phi_\txtm}
},
\end{aligned}
\end{dmath}
which shows the claimed field split.

In terms of the potentials Maxwell's equation \( \stgrad F = J \) is
\begin{dmath}\label{eqn:gaugeTransformation:20}
\stgrad \gpgrade{\conjstgrad A}{1,2} = J,
\end{dmath}
or
\begin{dmath}\label{eqn:gaugeTransformation:40}
\dLambertian A = J + \stgrad \gpgrade{\conjstgrad A}{0,3}.
\end{dmath}
This is almost a wave equation.
Inserting \cref{eqn:gaugeTransformation:1111} into \cref{eqn:gaugeTransformation:40} and selecting each grade gives four almost-wave equations
\begin{equation*}
\begin{aligned}
-
\dLambertian
\phi &= \frac{\rho}{\epsilon} + \inv{c} \PD{t}{} \lr{ c \spacegrad \cdot \BA + \inv{c} \PD{t}{\phi} } \\
c
\dLambertian
\BA &= -\eta \BJ + \spacegrad \lr{ c \spacegrad \cdot \BA + \inv{c} \PD{t}{\phi} } \\
\eta c I
\dLambertian
\BF &= - I \BM + \spacegrad \cdot \lr{ I \eta\lr{ c \spacegrad \cdot \BF + \inv{c} \PD{t}{\phi_\txtm} } } \\
-I \eta
\dLambertian
\phi_\txtm &= I c \rho_\txtm + \inv{c} \PD{t}{} I \eta\lr{ c \spacegrad \cdot \BF + \inv{c} \PD{t}{\phi_\txtm} }
\end{aligned}
\end{equation*}
Using \( \eta = \mu c, \eta c \epsilon = 1 \), and
\( \spacegrad \cdot (I \psi) = I \spacegrad \psi \) for scalar \(\psi\), a bit
of rearrangement completes the proof.

\subsection{Gauge transformations.}
Clearly it is desirable if potentials can be found for which \( \spacegrad \cdot \BA + (1/c^2) \partial_t \phi = \spacegrad \cdot \BF + (1/c^2) \partial_t \phi_\txtm = 0 \).
Finding such potentials relies on the fact that the potential representation is not unique.
In particular,
we have the freedom to add any spacetime gradient of any scalar or pseudoscalar potential without changing the field.
\index{gauge transformation}
\maketheorem{Gauge invariance.}{thm:gaugeTransformation:60}{
The spacetime gradient of a grade 0,3 multivector \( \Psi \) may be added to a multivector potential
\begin{equation*}
A' = A + \stgrad \Psi,
\end{equation*}
without changing the field.
That is
\begin{equation*}
F
= \gpgrade{\conjstgrad A}{1,2}
= \gpgrade{\conjstgrad A'}{1,2}.
\end{equation*}
} % theorem

To prove \cref{thm:gaugeTransformation:60} let
\begin{dmath}\label{eqn:gaugeTransformation:100}
A' = A + \stgrad (\psi + I \phi),
\end{dmath}
where \( \psi \) and \( \phi \) are scalar functions.
The field for potential \( A' \) is
\begin{dmath}\label{eqn:gaugeTransformation:120}
F'
= \gpgrade{\conjstgrad A'}{1,2}
= \gpgrade{\conjstgrad \lr{A + \stgrad (\psi + I \phi)} }{1,2}
= \gpgrade{\conjstgrad A}{1,2} + \gpgrade{ \conjstgrad \stgrad (\psi + I \phi)} {1,2}
= F + \gpgrade{ \dLambertian (\psi + I \phi)} {1,2},
\end{dmath}
which is just \( F \) since
since the d'Lambertian operator \( \dLambertian \) is a scalar operator and \( \psi + I \phi \) has no vector nor bivector grades.

We say that we are working in the Lorenz gauge, if the 0,3 grades of \( \conjstgrad A \) are zero, or a transformation that kills those grades is made.
\index{Lorenz gauge}
\maketheorem{Lorentz gauge transformation.}{thm:gaugeTransformation:140}{
Given any multivector potential \( A \) solution of Maxwell's equation, the transformation
\begin{equation*}
A' = A - \stgrad \Psi,
\end{equation*}
where
\begin{equation*}
\dLambertian \Psi = \gpgrade{ \conjstgrad A }{0,3},
\end{equation*}
allows Maxwell's equation to be written in wave equation form
\begin{equation*}
\dLambertian A' = J.
\end{equation*}
} % theorem

To prove \cref{thm:gaugeTransformation:140}, let
\begin{dmath}\label{eqn:gaugeTransformation:200}
A = A' + \stgrad \Psi,
\end{dmath}
so Maxwell's equation becomes
\begin{dmath}\label{eqn:gaugeTransformation:220}
J
= \stgrad \gpgrade{ \conjstgrad A }{1,2}
= \dLambertian A - \stgrad \gpgrade{ \conjstgrad A }{0,3}
= \dLambertian A' + \dLambertian \stgrad \Psi - \stgrad \gpgrade{ \conjstgrad A }{0,3}
= \dLambertian A' + \stgrad \lr{ \dLambertian \Psi - \gpgrade{ \conjstgrad A }{0,3} }.
\end{dmath}
Requiring
\begin{dmath}\label{eqn:gaugeTransformation:240}
\dLambertian \Psi = \gpgrade{ \conjstgrad A }{0,3},
\end{dmath}
completes the proof.
Observe that \( \Psi \) has only grades 0,3 as required of a gauge function.

Such a transformation completely decouples Maxwell's equation, providing one scalar wave equation for each grade of \( \dLambertian A' = J \), relating each grade of the potential \(A'\) to exactly one grade of the source multivector current \( J \).
We are free to immediately solve for \( A' \) using the (causal) Green's function for the d'Lambertian
\begin{dmath}\label{eqn:gaugeTransformation:160}
A'(\Bx, t)
= -\int dV' dt' \frac{\delta(\Abs{\Bx - \Bx'} - c(t - t')}{4 \pi \Norm{\Bx - \Bx'} } J(\Bx', t')
= -\inv{4\pi} \int dV' \frac{J(\Bx', t - \inv{c} \Norm{\Bx - \Bx'})}{\Norm{\Bx - \Bx'}},
\end{dmath}
which is the sum of all the current contributions relative to the point \( \Bx \) at the retarded time \( t_\txtr = t - (1/c) \Norm{\Bx - \Bx'}\).
The field follows immediately by differentiation and grade selection
\begin{dmath}\label{eqn:gaugeTransformation:180}
F = \gpgrade{ \conjstgrad A' }{1,2}.
\end{dmath}

Again, using the Green's function for the d'Lambertian, the explicit form of the gauge function \( \Psi \) is
\begin{dmath}\label{eqn:gaugeTransformation:260}
\Psi = -\inv{4\pi} \int dV' \frac{\gpgrade{ \conjstgrad A(\Bx', t_\txtr) }{0,3}}{\Norm{\Bx - \Bx'}},
\end{dmath}
however, we don't actually need to compute this.
Instead, we only have to know we are free to construct a field from any solution \( A' \) of \( \dLambertian A' = J \) using \cref{eqn:gaugeTransformation:180}.

%}
%\EndArticle

         \subsection{Far field (REWRITE.)}
            %
% Copyright © 2018 Peeter Joot.  All Rights Reserved.
% Licenced as described in the file LICENSE under the root directory of this GIT repository.
%
%{
\index{far field}
%
% Copyright � 2018 Peeter Joot.  All Rights Reserved.
% Licenced as described in the file LICENSE under the root directory of this GIT repository.
%
\maketheorem{Far field magnetic vector potential.}{thm:potentialSection_farfield:1}{
Given a spherical wave vector(dual-vector) potentials with representations
%\label{eqn:potentialSection_farfield:2400}
\begin{equation*}
\begin{aligned}
\BA &= \frac{e^{-j k r}}{r} \bcA( \theta, \phi ) \\
\BF &= \frac{e^{-j k r}}{r} \bcF( \theta, \phi ),
\end{aligned}
\end{equation*}
the far field (\(r \gg 1 \)) electromagnetic fields are given respectively
%\label{eqn:potentialSection_farfield:2520}{
\begin{equation*}
\begin{aligned}
F &= -j \omega \lr{ 1 + \rcap } \lr{ \rcap \wedge \BA} \\
F &= -j \omega \eta I \lr{ \rcap + 1 } \lr{ \rcap \wedge \BF }.
\end{aligned}
\end{equation*}
} % theorem


To prove \cref{thm:potentialSection_farfield:1}, we will utilize a
spherical representation of the gradient
\begin{dmath}\label{eqn:potentialSection_farfield:2420}
\begin{aligned}
\spacegrad &= \rcap \partial_r + \spacegrad_\perp \\
\spacegrad_\perp &= \frac{\thetacap}{r} \partial_\theta + \frac{\phicap}{r\sin\theta} \partial_\phi.
\end{aligned}
\end{dmath}

The gradient of the vector potential is
\begin{dmath}\label{eqn:potentialSection_farfield:2440}
\spacegrad \BA
=
\biglr{ \rcap \partial_r + \spacegrad_\perp } \frac{e^{-j k r}}{r} \bcA
=
\rcap \lr{ -j k - \inv{r} } \frac{e^{-j k r}}{r} \bcA
+
\frac{e^{-j k r}}{r}
\spacegrad_\perp
\bcA
= - \lr{ j k + \inv{r} } \rcap \BA + O(1/r^2)
\approx
- j k \rcap \BA.
\end{dmath}

Here, all the \( O(1/r^2) \) terms, including the action of the non-radial component of the gradient on the \( 1/r \) potential, have been neglected.
From \cref{eqn:potentialSection_farfield:2440} the far field divergence and the (bivector) curl of \( \BA \) are
\begin{dmath}\label{eqn:potentialSection_farfield:2460}
\begin{aligned}
\spacegrad \cdot \BA &= - j k \rcap \cdot \BA \\
\spacegrad \wedge \BA &= - j k \rcap \wedge \BA.
\end{aligned}
\end{dmath}

Finally, the far field gradient of the divergence of \( \BA \) is
\begin{dmath}\label{eqn:potentialSection_farfield:2480}
\spacegrad \lr{ \spacegrad \cdot \BA }
=
\biglr{ \rcap \partial_r + \spacegrad_\perp } \lr{ - j k \rcap \cdot \BA }
\approx
-j k \rcap \partial_r \lr{ \rcap \cdot \BA }
=
-j k \rcap \lr{ -j k - \inv{r} } \lr{ \rcap \cdot \BA }
\approx
-k^2 \rcap \lr{ \rcap \cdot \BA },
\end{dmath}
again neglecting any \( O(1/r^2) \) terms.  The field is
\begin{dmath}\label{eqn:potentialSection_farfield:2500}
F
=
- j \omega \BA  -j \frac{c^2}{\omega} \spacegrad \lr{ \spacegrad \cdot \BA } + c \spacegrad \wedge \BA
=
- j \omega \BA  +j \omega \rcap \lr{ \rcap \cdot \BA } - j k c \rcap \wedge \BA
=
- j \omega \lr{ \BA - \rcap \lr{ \rcap \cdot \BA }} - j \omega \rcap \wedge \BA
=
-j \omega \rcap \lr{ \rcap \wedge \BA} - j \omega \rcap \wedge \BA
=
-j \omega \lr{ \rcap + 1 } \lr{ \rcap \wedge \BA},
\end{dmath}
which completes the first part of the proof.  Extraction of the electric and magnetic fields can be done by inspection and is left to the reader to prove.

One interpretation of this is that the (bivector) magnetic field is represented by the plane perpendicular to the direction of propagation, and the electric field by a vector in that plane.

\maketheorem{Far field electric vector potential.}{thm:potentialSection_farfield:2}{
Given a vector potential with a radial spherical wave representation
%\label{eqn:potentialSection_farfield:2400}
\begin{equation*}
\BF = \frac{e^{-j k r}}{r} \bcF( \theta, \phi ),
\end{equation*}
the far field (\(r \gg 1 \)) electromagnetic field is
%\label{eqn:potentialSection_farfield:2520}
\begin{equation*}
F = -j \omega \eta I \lr{ \rcap + 1 } \lr{ \rcap \wedge \BF }.
\end{equation*}
If \( \BF_\perp = \rcap \lr{ \rcap \wedge \BF} \) represents the
non-radial component of the potential, the respective electric and magnetic field components are
%\begin{dmath}\label{eqn:potentialSection_farfield:2560}
\begin{equation*}
\begin{aligned}
\BE &= j \omega \eta \rcap \cross \BF \\
\BH &= -j \omega \BF_\perp.
\end{aligned}
\end{equation*}
%\end{dmath}
} % theorem

The proof of \cref{thm:potentialSection_farfield:2} is left to the reader.

\makeexample{Vertical dipole potential.}{example:potentialSection:1}{
We will calculate the far field along the propagation direction vector \( \kcap \) in the z-y plane
\begin{dmath}\label{eqn:potentialSection_farfield:2620}
\begin{aligned}
\kcap &= \Be_3 e^{i \theta} \\
i &= \Be_{32},
\end{aligned}
\end{dmath}
for the infinitesimal dipole potential
\begin{dmath}\label{eqn:potentialSection_farfield:2640}
\BA = \frac{\Be_3 \mu I_0 l}{4 \pi r} e^{-j k r},
\end{dmath}
as illustrated in \cref{fig:vectorPotential:vectorPotentialFig1}.

\mathImageFigure{../figures/GAelectrodynamics/vectorPotentialFig1}{Vertical infinitesimal dipole and selected propagation direction.}{fig:vectorPotential:vectorPotentialFig1}{0.3}{zcapPotential.nb}

The wedge of \( \kcap \) with \( \BA \) is proportional to
\begin{dmath}\label{eqn:potentialSection_farfield:2660}
\kcap \wedge \Be_3
=
\gpgradetwo{
\kcap \Be_3
}
=
\gpgradetwo{
\Be_3 e^{i \theta}
\Be_3
}
=
\gpgradetwo{
\Be_3^2 e^{-i \theta}
}
=
-i \sin\theta,
\end{dmath}
so from \cref{thm:potentialSection_farfield:2} the field is
\begin{dmath}\label{eqn:potentialSection_farfield:2680}
F = j \omega \lr{ 1 + \Be_3 e^{i\theta} } i \sin\theta \frac{\mu I_0 l}{4 \pi r} e^{-j k r}.
\end{dmath}

The electric and magnetic fields can be found from the respective vector and bivector grades of \cref{eqn:potentialSection_farfield:2680}
\begin{dmath}\label{eqn:potentialSection_farfield:2700}
\BE
=
\frac{j \omega \mu I_0 l}{4 \pi r} e^{-j k r} \Be_3 e^{i\theta} i \sin\theta
=
\frac{j \omega \mu I_0 l}{4 \pi r} e^{-j k r} \Be_2 e^{i\theta} \sin\theta
=
\frac{j k \eta I_0 l \sin\theta}{4 \pi r} e^{-j k r} \lr{ \Be_2 \cos\theta - \Be_3 \sin\theta },
\end{dmath}
and
\begin{dmath}\label{eqn:potentialSection_farfield:2720}
\BH
=
\inv{I \eta}
j \omega i \sin\theta_0 \frac{\mu I_0 l}{4 \pi r} e^{-j k r}
=
\inv{\eta} \Be_{321} \Be_{32}
j \omega \sin\theta_0 \frac{\mu I_0 l}{4 \pi r} e^{-j k r}
=
-\Be_1 \frac{ j k \sin\theta_0 I_0 l}{4 \pi r} e^{-j k r}.
\end{dmath}

The multivector electrodynamic field expression
\cref{eqn:potentialSection_farfield:2680} for
\( F \) is more algebraically compact than the separate electric and magnetic field expressions, but this comes with the complexity of dealing with different types of imaginaries.
There are two explicit unit imaginaries in \cref{eqn:potentialSection_farfield:2680}, the scalar imaginary \( j \) used to encode the time harmonic nature of the field, and \( i = \Be_{32} \) used to represent the plane that the far field propagation direction vector lay in.
Additionally, when the magnetic field component was extracted, the pseudoscalar \( I = \Be_{123} \) entered into the mix.
Care is required to keep these all separate, especially since \( I, j \) commute with all grades, but \( i \) does not.
} % example

%}

      \section{Dielectric and magnetic media (REWRITE.)}
         %
% Copyright © 2017 Peeter Joot.  All Rights Reserved.
% Licenced as described in the file LICENSE under the root directory of this GIT repository.
%
So far, we've considered only media where the linear constitutive relationships \cref{eqn:freespace:301} hold.
Without such assumptions the GA formalism for Maxwell's equations cannot be written as a single equation with one multivector field, but requires two equations and two multivector fields.

The two multivector fields are
\begin{dmath}\label{eqn:inMatter:40}
\begin{aligned}
G &= \BE + I c \BB \\
F &= \BD + \frac{I}{c} \BH,
\end{aligned}
\end{dmath}
for which Maxwell's equations are
\begin{dmath}\label{eqn:inMatter:60}
\begin{aligned}
\gpgrade{ \stgrad F }{0,1} &= \rho - \frac{\BJ}{c} \\
\gpgrade{ \stgrad G }{2,3} &= I \lr{ c \rho_m - \BM }.
\end{aligned}
\end{dmath}

Here \( c \) is a non-dimensionalizing constant with dimensions [L/T], but is otherwise unspecified.
Direct expansion can be used to show that \cref{eqn:inMatter:60} is equivalent to Maxwell's equations.
Doing so for each of the grades in turn, we have

\begin{subequations}
\label{eqn:inMatter:80}
\begin{dmath}\label{eqn:inMatter:100}
\rho
=
\gpgradezero{ \stgrad F }
=
\gpgradezero{ \stgrad \lr{ \BD + \frac{I}{c} \BH } }
=
\spacegrad \cdot \BD
\end{dmath}
\begin{dmath}\label{eqn:inMatter:120}
- \frac{\BJ}{c}
=
\gpgradeone{ \stgrad F }
=
\gpgradeone{ \stgrad \lr{ \BD + \frac{I}{c} \BH } }
=
\inv{c} \PD{t}{\BD} + \frac{I}{c} \spacegrad \wedge \BH
=
\inv{c} \PD{t}{\BD} - \frac{1}{c} \spacegrad \cross \BH
\end{dmath}
\begin{dmath}\label{eqn:inMatter:140}
- I \BM
=
\gpgrade{ \stgrad G }{2}
=
\gpgrade{ \stgrad \lr{ \BE + I c \BB} }{2}
=
\spacegrad \wedge \BE + I \PD{t}{\BB}
\end{dmath}
\begin{dmath}\label{eqn:inMatter:160}
I c \rho_m
=
\gpgrade{ \stgrad G }{3}
=
\gpgrade{ \stgrad \lr{ \BE + I c \BB} }{3}
=
c I \spacegrad \cdot \BB.
\end{dmath}
\end{subequations}

After rearranging and cancelling common factors of \( c, I \) Maxwell's equations are recovered
\begin{dmath}\label{eqn:inMatter:180}
\begin{aligned}
\spacegrad \cdot \BD &= \rho \\
\spacegrad \cross \BH &= \BJ + \PD{t}{\BD}  \\
\spacegrad \cross \BE &= -\BM - \PD{t}{\BB} \\
\spacegrad \cdot \BB &= \rho_m.
\end{aligned}
\end{dmath}

One possible strategy for solving these equations is to impose an additional set of constraints on the grades in question
\begin{dmath}\label{eqn:inMatter:200}
\begin{aligned}
\gpgrade{ \stgrad F }{2,3} &= 0 \\
\gpgrade{ \stgrad G }{0,1} &= 0,
\end{aligned}
\end{dmath}
so that all the grade selection filters can be cleared
\begin{dmath}\label{eqn:inMatter:220}
\begin{aligned}
\stgrad F &= \rho - \frac{\BJ}{c} \\
\stgrad G &= I \lr{ c \rho_m - \BM }.
\end{aligned}
\end{dmath}

Each of these now separately has the form of Maxwell's equation, and could be solved separately, subject to the constraint equations.
Only if \( G, F \) can be related by a constant factor, say \( \epsilon G = F \), can these be summed directly (after non-dimensional scaling) to form Maxwell's equation.
Other non-constraint strategies for solving \cref{eqn:inMatter:60} would require additional thought and study.

%   \section{Radiation and scattering}
%TODO.
   %   \subsection{Problem solutions}
   %      \shipoutAnswer
