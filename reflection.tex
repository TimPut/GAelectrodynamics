\subsection{Reflection}
\index{reflection}

Geometrically the reflection of a vector \( \Bx \) across a line directed along \( \Bu \) is the difference of the projection and rejection

\begin{dmath}\label{eqn:SimpleProducts2:900}
\Bx'
= \lr{ \Bx \cdot \Bu }\Bu - \lr{ \Bx \wedge \Bu } \inv{\Bu }
= \lr{ \Bx \cdot \Bu - \Bx \wedge \Bu } \inv{\Bu }
\end{dmath}

Using the symmetric and antisymmetric sum representations of the dot and wedge products 
cref{FIXME}, the reflection can be expressed as vector products

\begin{dmath}\label{eqn:reflection:n}
\Bx'
= \inv{2} \lr{ \cancel{\Bx \Bu} + \Bu \Bx - \cancel{\Bx \Bu} + \Bu \Bx } \inv{\Bu }.
\end{dmath}

or
\boxedEquation{eqn:SimpleProducts2:920}{
\Bx' = \Bu \Bx \inv{\Bu}.
}

An illustration of the geometry of reflection is provided in \cref{fig:reflection:reflectionFig1}.

\imageFigure{../figures/GAelectrodynamics/reflectionFig1}{Reflection}{fig:reflection:reflectionFig1}{0.3}

