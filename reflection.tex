%
% Copyright © 2017 Peeter Joot.  All Rights Reserved.
% Licenced as described in the file LICENSE under the root directory of this GIT repository.
%
\index{reflection}

Geometrically the reflection of a vector \( \Bx \) across a line directed along \( \Bu \) is the difference of the projection and rejection

\begin{dmath}\label{eqn:reflection:900}
\Bx'
= \lr{ \Bx \cdot \Bu } \inv{\Bu} - \lr{ \Bx \wedge \Bu } \inv{\Bu }
= \lr{ \Bx \cdot \Bu - \Bx \wedge \Bu } \inv{\Bu }
\end{dmath}

Using the symmetric and antisymmetric sum representations of the dot and wedge products from
\cref{thm:symmetricAndAntiSymmetricVectorSums:symmetricAndAnti}
the reflection can be expressed as vector products

\begin{dmath}\label{eqn:reflection:940}
\Bx'
= \inv{2} \lr{ \cancel{\Bx \Bu} + \Bu \Bx - \cancel{\Bx \Bu} + \Bu \Bx } \inv{\Bu },
\end{dmath}

yielding a remarkably simple form in terms of vector products

\boxedEquation{eqn:SimpleProducts2:920}{
\Bx' = \Bu \Bx \inv{\Bu}.
}

As an illustration, here is a sample CliffordBasic reflection computation

%\begin{mmaCell}{Input}
%u = 4 e[1] + 2 e[2];
%x = 3 e[1] + 3 e[2];
%invu = Simplify[N[\mmaFrac{u}{InnerProduct[u,u]}]];
%ux = GeometricProduct[u,x];
%uxinvu = N[GeometricProduct[ux,invu]];
%\end{mmaCell}
\begin{mmaCell}[moredefined={u, e, x, uu, InnerProduct, invu, i, o, OuterProduct, proj, rej, GeometricProduct, ux, uxu}]{Input}
u = 4 e[1] + 2 e[2]; x = 3 e[1] + 3 e[2];
uu = InnerProduct[u,u]; invu = \mmaFrac{u}{uu};
i = InnerProduct[x,u]; o = OuterProduct[x,u];
proj = Simplify[N[i invu]];
rej = Simplify[N[GeometricProduct[o,invu]]];
ux = GeometricProduct[u,x];
uxu = Simplify[N[GeometricProduct[ux,invu]]];
\end{mmaCell}

the results of which are plotted in \cref{fig:reflection:reflectionFig1}.

\imageFigure{../figures/GAelectrodynamics/reflectionFig1}{Reflection.}{fig:reflection:reflectionFig1}{0.3}
