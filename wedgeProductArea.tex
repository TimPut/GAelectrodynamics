%
% Copyright © 2017 Peeter Joot.  All Rights Reserved.
% Licenced as described in the file LICENSE under the root directory of this GIT repository.
%
The coordinate representation of the \R{2} wedge product (\cref{eqn:SimpleProducts2:1720}) had a single \( \Be_{12} \) bivector factor, whereas the expansion in coordinates for the general \R{N} wedge product was considerably messier (\cref{eqn:SimpleProducts2:1320}).
This difference can be eliminated by judicious choice of basis.

A simpler coordinate representation for the \R{N} wedge product follows by choosing an
orthonormal basis
for the planar subspace spanned by the wedge vectors.
Given vectors \( \Ba, \Bb \), let \( \setlr{\ucap, \vcap} \) be an orthonormal basis for the plane subspace
\( P = \Span\setlr{ \Ba, \Bb } \).
The coordinate representations of \( \Ba, \Bb \) in this basis are

\begin{dmath}\label{eqn:wedgeProductArea:1900}
\begin{aligned}
\Ba &= (\Ba \cdot \ucap) \ucap + (\Ba \cdot \vcap) \vcap \\
\Bb &= (\Bb \cdot \ucap) \ucap + (\Bb \cdot \vcap) \vcap.
\end{aligned}
\end{dmath}

The wedge of these vectors is

\begin{dmath}\label{eqn:SimpleProducts2:1860}
\Ba \wedge \Bb
=
   \Biglr{
   (\Ba \cdot \ucap) \ucap + (\Ba \cdot \vcap) \vcap
   }
\wedge
   \Biglr{
   (\Bb \cdot \ucap) \ucap + (\Bb \cdot \vcap) \vcap
   }
=
\Biglr{
      (\Ba \cdot \ucap)
   (\Bb \cdot \vcap)
   -
   (\Ba \cdot \vcap) (\Bb \cdot \ucap)
}
\ucap \vcap
=
\begin{vmatrix}
   \Ba \cdot \ucap & \Ba \cdot \vcap \\
   \Bb \cdot \ucap & \Bb \cdot \vcap
\end{vmatrix}
\ucap \vcap.
\end{dmath}

We see that this basis allows for the most compact (single term) coordinate representation of the wedge product.

If a counterclockwise rotation by \( \pi/2 \) takes \( \ucap \) to \( \vcap \) the determinant will equal the area of the parallelogram spanned by \( \Ba \) and \( \Bb \).
Let that area be designated

\begin{dmath}\label{eqn:wedgeProductArea:1920}
A =
\begin{vmatrix}
   \Ba \cdot \ucap & \Ba \cdot \vcap \\
   \Bb \cdot \ucap & \Bb \cdot \vcap
\end{vmatrix}.
\end{dmath}

Any number of possible wedge (or normal) product representations of the same numeric 2-blade quantity are possible

\begin{dmath}\label{eqn:wedgeProductArea:1940}
\begin{aligned}
\Ba \wedge \Bb
&= (\Ba + \beta \Bb ) \wedge \Bb \\
&= \Ba \wedge ( \Bb + \alpha \Ba ) \\
&= (A \ucap) \wedge \vcap \\
&= \ucap \wedge (A \vcap) \\
&= (\alpha A \ucap) \wedge \frac{\vcap}{\alpha} \\
&= (\beta A \ucap') \wedge \frac{\vcap'}{\beta} \\
\end{aligned}
\end{dmath}

These equivalencies can be thought of as different possible parallelogram possible geometrical representations of the same object. Since the spanned area and relative ordering of the wedged vectors remains constant.
Some different parallelogram representations of a single 2-blade are illustrated in \cref{fig:parrallelograms:parrallelogramsFig1}.

\imageFigure{../figures/GAelectrodynamics/parrallelogramsFig1}{Parallelogram representations of wedge products.}{fig:parrallelograms:parrallelogramsFig1}{0.3}

An arbitrary 2-blade need not be ``factored'' into any pair of normal vectors, nor is there any a priori reason to represent such an object as a wedge product of two specific vectors.
In this sense there is no definitive geometry to a 2-blade, and it can be represented as any fixed area with a given cyclic orientation, as
illustrated in \cref{fig:orientedAreasVariety:orientedAreasVarietyFig1}.

\imageFigure{../figures/GAelectrodynamics/orientedAreasVarietyFig1}{Different shape representations of a given bivector.}{fig:orientedAreasVariety:orientedAreasVarietyFig1}{0.2}

\index{parallelogram}
\makeproblem{Parallelogram area.}{problem:wedgeProductArea:R2parallelogramarea}{
Show that the area \( A \) of the parallelogram spanned by vectors
\begin{equation*}
\begin{aligned}
\Ba &= a_1 \Be_1 + a_2 \Be_2 \\
\Bb &= b_1 \Be_1 + b_2 \Be_2,
\end{aligned}
\end{equation*}

is
\begin{equation*}
A =
\pm
\begin{vmatrix}
   a_1 & a_2 \\
   b_1 & b_2 \\
\end{vmatrix}
,
\end{equation*}

and that the sign is positive if the rotation angle \( \theta \) that takes \( \acap \) to \( \bcap \) is positive, \( \theta \in (0,\pi) \).
} % problem

