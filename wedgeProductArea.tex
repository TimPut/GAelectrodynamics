
To prove \cref{thm:SimpleProducts2:wnormalfactorize}, first assume that there is an orthonormal basis \( \setlr{\ucap, \vcap} \) for the planar subspace \( P = \Span\setlr{ \Ba, \Bb } \), for which

\begin{dmath}\label{eqn:SimpleProducts2:1840}
\begin{aligned}
\Ba &= (\Ba \cdot \ucap) \ucap + (\Ba \cdot \vcap) \vcap \\
\Bb &= (\Bb \cdot \ucap) \ucap + (\Bb \cdot \vcap) \vcap.
\end{aligned}
\end{dmath}

The wedge of \( \Ba, \Bb \) in terms of this basis is

\begin{dmath}\label{eqn:SimpleProducts2:1860}
\Ba \wedge \Bb
=
\gpgradetwo{
   \lr{
   (\Ba \cdot \ucap) \ucap + (\Ba \cdot \vcap) \vcap
   }
   \lr{
   (\Bb \cdot \ucap) \ucap + (\Bb \cdot \vcap) \vcap
   }
}
=
\gpgradetwo{
\cancel{
   (\Ba \cdot \ucap) (\Bb \cdot \ucap) \ucap^2
}
+
\cancel{
   (\Ba \cdot \vcap) (\Bb \cdot \vcap) \vcap^2
}
+
\lr{
      (\Ba \cdot \ucap)
   (\Bb \cdot \vcap)
   -
   (\Ba \cdot \vcap) (\Bb \cdot \ucap)
}
\ucap \vcap
}
=
\lr{
      (\Ba \cdot \ucap)
   (\Bb \cdot \vcap)
   -
   (\Ba \cdot \vcap) (\Bb \cdot \ucap)
}
\ucap \vcap.
\end{dmath}

Such a basis allows for the most compact (single term) coordinate representation of the wedge product

\begin{dmath}\label{eqn:SimpleProducts2:1880}
\Ba \wedge \Bb
=
\begin{vmatrix}
   \Ba \cdot \ucap & \Ba \cdot \vcap \\
   \Bb \cdot \ucap & \Bb \cdot \vcap
\end{vmatrix}
\ucap \vcap.
\end{dmath}

The wedge product is therefore the (possibly signed) area of the parallelopiped formed by the vectors \( \Ba, \Bb \), multiplied by a unit pseudoscalar for the subspace of the plane \( P \).  Provided the area of this parallelopiped is non-zero, which is always the case for non-colinear vectors, there are clearly many possible normal factorizations for the wedge product.

\subsection{General rotation.}

\Cref{eqn:SimpleProducts2:180} showed that the \R{2} pseudoscalar anticommutes with any vector \( \Bx \in \bbR^{2} \),

\begin{dmath}\label{eqn:SimpleProducts2:1760}
\Bx i = -i \Bx.
\end{dmath}

The higher dimensional generalization of this result is

\maketheorem{Commutation rules for wedge products.}{thm:SimpleProducts2:1780}{
Given a planar subspace formed by the span of two non-colinear vectors \( S = \Span \setlr{ \Ba, \Bb } \), any vector \( \Bx \in S \) anticommutes with the wedge product \( \Ba \wedge \Bb \)

\begin{equation*}
\Bx (\Ba \wedge \Bb) = - (\Ba \wedge \Bb) \Bx.
\end{equation*}

Moreover, any vector \( \Bn \) normal to this plane (\( \Bn \cdot \Ba = \Bn \cdot \Bb = 0 \)) commutes with this wedge product
\begin{equation*}
\Bn (\Ba \wedge \Bb) = (\Ba \wedge \Bb) \Bn.
\end{equation*}
} % theorem

A simple inelegant way to prove this is to specify a coordinate system for which \( \Ba, \Bb \) both lie in the \( x,y \) plane.  Then \( \Ba \wedge \Bb = \alpha i \) for some \( \alpha \), and the anticommutation part of the theorem follows from
%the \R{2} result
\cref{eqn:SimpleProducts2:1760}.  For the normal commutation part of the theorem, pick any vector normal to the \(x, y\) plane, say \( \Be_3\), for which we have

\begin{dmath}\label{eqn:SimpleProducts2:1780}
\Be_3  (\Ba \wedge \Bb)
=
\Be_3 \alpha i
=
\alpha \Be_3 \Be_1 \Be_2
=
\alpha (-\Be_1 \Be_3) \Be_2
=
-\alpha \Be_1 (\Be_3 \Be_2)
=
-\alpha \Be_1 (-\Be_2 \Be_3)
= (\Ba \wedge \Bb) \Be_3.
\end{dmath}

In dimensions with more normals, say \( \Be_4, \cdots \), the steps of \cref{eqn:SimpleProducts2:1780} can be repeated.  The general normal commuation result follows by superposition.
