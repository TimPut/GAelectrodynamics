%
% Copyright � 2016 Peeter Joot.  All Rights Reserved.
% Licenced as described in the file LICENSE under the root directory of this GIT repository.
%
%----------------------------------------------------------------------------------------
\part{Geometric Algebra}
   \chapter{Basics}
      \section{Did you ever ask your teacher how to multiply vectors?}
         %
% Copyright © 2016 Peeter Joot.  All Rights Reserved.
% Licenced as described in the file LICENSE under the root directory of this GIT repository.
%
A new student of vector algebra will first learn
%first see vectors in two or three dimensions as sets of coordinates
%
%\begin{equation}\label{eqn:GAmotivation:20}
%\Ba =
%\begin{bmatrix}
%a_1 \\
%a_2 \\
%a_3 \\
%\end{bmatrix}, \qquad
%\Bb =
%\begin{bmatrix}
%b_1 \\
%b_2 \\
%b_3 \\
%\end{bmatrix},
%\end{equation}
%
%or perhaps explicitly in terms of a basis \( \setlr{ \Be_1, \Be_2 \Be_3 } \)
%
%\begin{dmath}\label{eqn:GAmotivation:40}
%\begin{aligned}
%\Ba &= a_1 \Be_1 + a_2 \Be_2 + a_3 \Be_3  \\
%\Bb &= b_1 \Be_1 + b_2 \Be_2 + b_3 \Be_3
%\end{aligned}.
%\end{dmath}
%
%You will learn the
rules for addition and subtraction of such vectors.
%, and then how to operate on them with rotation matrices or other representations of linear transformations.
This demonstrates to the student that the vector is an algebraic object that generalize numbers, and the question of how to
multiply vectors soon follows.

Given the toolbox of traditional vector algebra, the best answer that a new student will obtain from such a line of questioning is to learn of the dot and cross products, the multiplication like operations that we are all so familiar with

%\begin{itemize}
%\item ``You can not multiply vectors.'', or
%\item ``Vector multiplication is not well defined.'', or
%\item ``We will get to that.'', or
%\item ``There are multiplication like operations.''  The
%\end{itemize}

\begin{dmath}\label{eqn:GAmotivation:60}
\begin{aligned}
\Ba \cdot \Bb &= a_1 b_1 + a_2 b_2 + a_3 b_3 = \Abs{\Ba} \Abs{\Bb} \cos \theta_{ab} \\
\Ba \cross \Bb &=
\begin{vmatrix}
\Be_1 & \Be_2 & \Be_3 \\
a_1 & a_2 & a_3 \\
b_1 & b_2 & b_3 \\
\end{vmatrix}
= \ncap_{ab} \Abs{\Ba} \Abs{\Bb} \sin\theta_{ab}.
\end{aligned}
\end{dmath}

Both of these multiplication like operations live in very different spaces, one scalar, and the other a vector that lies outside of the span of its two vector factors.  Observe that the magnitudes of these two product operations are related to the product of the vectors in a Pythagorean sense

\begin{dmath}\label{eqn:GAmotivation:180}
\lr{ \Ba \cdot \Bb }^2 + \lr{ \Ba \cross \Bb }^2
=
\Abs{\Ba}^2 \Abs{\Bb}^2 \cos^2 \theta_{ab}
+\Abs{\Ba}^2 \Abs{\Bb}^2 \sin^2 \theta_{ab}
=
\Abs{\Ba}^2 \Abs{\Bb}^2.
\end{dmath}

This can be seen as a hint that the dot and cross products might be components of a single vector product operation, but the precise form of that product is not obvious.

Vector products that have the same form as the scalar magnitudes of the dot and cross products can be found in other algebraic systems.  Given a complex number representation of two vectors in a 2D space

\begin{dmath}\label{eqn:GAmotivation:200}
\begin{aligned}
z &= r e^{i \theta} \leftrightarrow (a, b) \\
w &= \rho e^{i \alpha} \leftrightarrow (a', b'),
\end{aligned}
\end{dmath}

the inner product of such a complex vector representation can be seen to have the same structure as the dot and cross products

\begin{equation}\label{eqn:GAmotivation:100}
\begin{aligned}
\Real( z w^\conj ) &= r \rho \cos(\theta - \alpha) \\
\Imag( z w^\conj ) &= r \rho \sin(\theta - \alpha).
\end{aligned}
\end{equation}

It can be shown
(\cref{problem:introGAproblems:ComplexInnerProductVsDotAndCrossProduct})
that this inner product has the following vector isomorphism

\begin{dmath}\label{eqn:GAmotivation:220}
z w^\conj \leftrightarrow ( a a' + b b', a' b - a b' ).
\end{dmath}

One component is completely symmetric, whereas the other component of this product has a component that is completely antisymmetric.
The 3D cross product also has this antisymmetry, and that antisymmetry will be seen later to be the key to the generalization of the cross product.  In this particular case, one can view this antisymmetric sum \( a' b - a b' \) as one
answer of how the cross product ``generalizes'' from 3D to 2D without requiring the introduction of a normal dimension.

The answer to questions of exactly how the vector products, in particular the cross product, should generalize to higher dimensional spaces are still outstanding.  It should be expected that this cross product generalization will involve antisymmetry, just as the dot product generalization in higher dimensional spaces is completely symmetric.

Many current students of science never see the exact structure of this generalization.  Should studies happen to include
enough of the right esoteric physics and mathematics (quantum mechanics, QED, calculus on manifolds, ...) then
some answers to those questions may be found.  Unfortunately, there are many such answers, and many of them each only provide
one part of the picture.

For example, a student of non-relativistic quantum mechanics will learn of Pauli matrices, when studying spin operators.  The dot and cross products will be seen to be components of a more general vector multiplication operation

\begin{equation}\label{eqn:GAmotivation:120}
\lr{\Bsigma \cdot \Bx }
\lr{\Bsigma \cdot \By }
=
I \lr{ \Bx \cdot \By } + i \Bsigma \cdot \lr{ \Bx \cross \By }.
\end{equation}

In quaternion algebra, a generalization of complex algebra, when a quaternion is represented as a scalar vector pair \( q = (r, \Bv) \), the quaternion product of two vectors also shows that the dot and cross products are respective components of a product of vectors

\begin{dmath}\label{eqn:GAmotivation:240}
(0, \Bx)
(0, \By) = (-\Bx \cdot \By, \Bx \cross \By).
\end{dmath}

A student of quantum field theory will encounter Dirac matrices, a algebraic structure that allows for the multiplication of four-vectors

\begin{dmath}\label{eqn:GAmotivation:140}
\aslash \bslash
=
\inv{2} \symmetric{ \aslash}{ \bslash }
+
\inv{2} \antisymmetric{ \aslash}{ \bslash }
=
a^\mu b_\mu + \inv{2} a^\mu b^\nu \antisymmetric{\gamma_\mu}{\gamma_\nu}
=
a^\mu b_\mu + \inv{2} a^\mu b^\nu \lr{
\gamma_\mu \gamma_\nu
-
\gamma_\nu \gamma_\mu
}.
\end{dmath}

A product of ``Dirac'' vectors has symmetric and antisymmetric components that generalize the dot and cross products.
Unfortunately, this algebra comes with still another different notation.
One interesting take away from this particular vector product is the fact that one component is a scalar, and other other
involves products of vectors, something that will require further interpretation.  Since the Dirac basis typically has a matrix representation, such a product can be dismissed as just being another matrix.  The products of mutually orthonormal vectors will show up again later in a context where there is no requirement to assume a matrix representation of the underlying basis.

Another common and important context that contains generalizations of the dot and cross products is the subject of differential forms.
A student of differential forms will learn how to compute the wedge products of forms, and of duality operations, which can be used to construct generalized multiplication operations that have the structure of the 3D dot and cross products

\begin{equation}\label{eqn:GAmotivation:160}
\begin{aligned}
df \wedge * dg &= \lr{ \sum_{i=1}^3 \PD{x_i}{f} \PD{x_i}{g} } dx_1 \wedge dx_2 \wedge dx_3 \\
df \wedge dg &= \sum_{1 \le i < j \le 3} \lr{
\PD{x_i}{f} \PD{x_j}{g}
-\PD{x_j}{f} \PD{x_i}{g}
}
dx_i \wedge dx_j.
\end{aligned}
\end{equation}

It is possible to express vectors as a differential form, and some advocate for this \citep{flanders1989dfa}, but this can also seem unnatural.  Regardless, differential forms do highlight the existence of more general concepts of vector multiplication.  %In this particular case, this generality comes with the cost of using yet another notation, one that is considerably different than the vector notation that we are comfortable with.

It should not be surprising that all of these ideas are special cases of a more general algebraic system.

The aim of the material to follow is to provide the instruction manual for an enhanced toolbox of vector algebra techniques that can be used to gain an integrated view of many seemingly disparate mathematical methods.  These are tools that can be learned without having to first study the esoteric arts of quantum mechanics or differential forms, and have many applications once learned.  These notes will focus on applications to the study of electromagnetism.


         \subsection{Problems}
            %
% Copyright © 2016 Peeter Joot.  All Rights Reserved.
% Licenced as described in the file LICENSE under the root directory of this GIT repository.
%

\makeproblem{Complex inner product vs. dot and cross product.}{problem:introGAproblems:ComplexInnerProductVsDotAndCrossProduct}{
Given two 2D vectors \( (a,b) \) and \( (a', b') \), and a complex number representation of these vectors \( z = a + ib, w = a' + i b' \), show that the components of the complex inner product have the representation
given by \cref{eqn:GAmotivation:220}.
} % problem

\makeanswer{problem:introGAproblems:ComplexInnerProductVsDotAndCrossProduct}{
\begin{dmath}\label{eqn:introGAproblems:20}
z w^\conj
=
(a + ib)(a' - ib')
=
a a' + b b'
+ i \lr{ a' b - a b' }
\leftrightarrow
( a a' + b b', a' b - a b' ).
\end{dmath}
} % answer

      \section{Vector multiplication}
         %
% Copyright � 2016 Peeter Joot.  All Rights Reserved.
% Licenced as described in the file LICENSE under the root directory of this GIT repository.
%
%{
%%\input{../blogpost.tex}
%%\renewcommand{\basename}{multiplication}
%%%\renewcommand{\dirname}{notes/phy1520/}
%%\renewcommand{\dirname}{notes/ece1228-electromagnetic-theory/}
%%%\newcommand{\dateintitle}{}
%%%\newcommand{\keywords}{}
%%
%%\input{../peeter_prologue_print2.tex}
%%
%%\usepackage{peeters_layout_exercise}
%%\usepackage{peeters_braket}
%%\usepackage{peeters_figures}
%%\usepackage{siunitx}
%%%\usepackage{mhchem} % \ce{}
%%%\usepackage{macros_bm} % \bcM
%%%\usepackage{txfonts} % \ointclockwise
%%
%%\beginArtNoToc
%%
%%\generatetitle{Vector multiplication}
%%%\chapter{Vector multiplication}
%%%\label{chap:multiplication}
%%
Geometric Algebra defines a multiplication operation for vectors, forming a vector space spanned by all the possible vector products.  This algebra is described by the following small set of axioms

\makeaxiom{Associative multiplication.}{axiom:multiplication:associative}{

The product of any three vectors \(\Ba,\Bb,\Bc\) is associative.

\begin{equation*}\label{eqn:multiplication:160}
\Ba (\Bb \Bc)
= (\Ba \Bb) \Bc
= \Ba \Bb \Bc.
\end{equation*}
}

\makeaxiom{Linearity.}{axiom:multiplication:linear}{
Vector products are linear with respect to addition and subtraction.

\begin{dmath*}\label{eqn:multiplication:180}
\begin{aligned}
(\Ba + 3 \Bb \Bd) \Bc &= \Ba \Bb + 3 \Bb \Bd \Bc \\
\Ba (\Bb \Bd - 2 \Bc) &= \Ba \Bb \Bd - 2 \Ba \Bc.
\end{aligned}
\end{dmath*}
}

\makeaxiom{Contraction.}{axiom:multiplication:contraction}{

The square of a vector is the squared length of the vector.

\begin{dmath*}\label{eqn:multiplication:200}
\Ba^2 = \Abs{\Ba}^2.
\end{dmath*}

The notion of length here is metric dependent.  For the problems considered in these notes
it can be assumed that there is an orthonormal Euclidean basis, where the vector length is always positive.
For special relativistic calculations, also of interest in electrodynamics, but not the focus of these notes, the length of a (four-)vector may generally be negative or positive.
}

These axioms are simple enough, but have a rich set of consequences\footnote{Similar to Feynman on gravitation \citep{feynman1963flp} ``... have shall said everything required, for a sufficiently talented mathematician could then deduce all the consequences of these principles.  However, since you are not assumed to be sufficiently talented yet, we shall discuss the consequences in more detail''.}.

The linearity and associativity axioms need little comment, but the contraction property might be surprising.  For one justification of this rule, consider a one dimensional vector space spanned by a single unit vector \( \setlr{ \Be } \).  That span, for real \( x \) is all the values

\begin{dmath}\label{eqn:multiplication:20}
\Bx = x \Be.
\end{dmath}

FIXME: picture to demonstrate the number line isomorphism.

This vector space is isomorphic with a number line, all the possible real values \( x \).
Given a positive number \( x \), the multiplication rules for real numbers require that \( (\pm x)^2 = x^2 \).
The square of a number provides the (squared) length of the number, its distance from the origin.
The same rule can be imposed for one dimensional vectors,
a requirement that the (squared) distance from the origin equals the square of the vector itself.   Such a rule is consistent with the rules of scalar multiplication, and for the
one dimensional vectors of \cref{eqn:multiplication:20} can be stated as

\begin{equation}\label{eqn:multiplication:40}
\Bx^2 = x^2.
\end{equation}

This contraction axiom, justified or not, has additional implications

\begin{dmath}\label{eqn:multiplication:80}
x^2
= \Bx^2
= (x \Be)(x \Be)
= x^2 \Be^2.
\end{dmath}

This rule requires the square of a unit (Euclidean) vector to be unity

%\begin{equation}\label{eqn:multiplication:60}
\boxedEquation{eqn:multiplication:60}{
\Be^2 = 1.
}
%\end{equation}

With this implication noted, now consider the square of a simple two dimensional vector

\begin{dmath}\label{eqn:gaTutorial:80}
2
=
(\Be_1 + \Be_2)^2
= (\Be_1 + \Be_2)(\Be_1 + \Be_2)
= \Be_1^2 + \Be_2 \Be_1 + \Be_1 \Be_2 + \Be_2^2
= 2 + \Be_2 \Be_1 + \Be_1 \Be_2.
\end{dmath}

The sum above with both scalar terms and terms that are composed of products of vectors is called a multivector.
A product of two perpendicular vectors (or a sum of such products) is called a bivector, and can be used to represent an oriented plane.
Geometric Algebra allows for sums of scalars, vectors, bivectors, and higher degree products.

Observe that for this identity to hold, the bivector terms must sum to zero.  That is

%\begin{dmath}\label{eqn:multiplication:140}
\boxedEquation{eqn:multiplication:140}{
\Be_1 \Be_2 = -\Be_1 \Be_2.
}
%\end{dmath}

This implies that the product of two orthonormal vectors anticommutes.  In general it is also true that

\maketheorem{Normal anticommutation}{thm:multiplication:anticommutationNormal}{
The product of any two normal vectors \(\Bu\), and \(\Bv\) anticommute.
\begin{equation*}
\Bu \Bv = -\Bv \Bu.
\end{equation*}
} % theorem

%%%}
%%%\EndArticle
%%\EndNoBibArticle

         \subsection{Problems}
            %
% Copyright © 2016 Peeter Joot.  All Rights Reserved.
% Licenced as described in the file LICENSE under the root directory of this GIT repository.
%
\makeproblem{}{problem:multiplication:2dvectorsquare}{
Generalize the calculation of \cref{eqn:gaTutorial:80} to calculate the square of an \R{n} vector.

\begin{dmath}\label{eqn:multiplication:100}
\Bx = \sum_i x_i \Be_i
\end{dmath}
} % problem

\makeanswer{problem:multiplication:2dvectorsquare}{
Consider the 2D case to start with

\begin{dmath}\label{eqn:multiplication:120}
\Bx^2
=
\lr{ x \Be_1 + y \Be_2}
\lr{ x \Be_1 + y \Be_2}
=
\lr{ x \Be_1 } \lr{ x \Be_1 }
+
\lr{ y \Be_2 } \lr{ y \Be_2 }
+
\lr{ x \Be_1 } \lr{ y \Be_2 }
+
\lr{ y \Be_2 } \lr{ x \Be_1 }
=
x^2 \Be_1^2
+
y^2 \Be_2^2
+
x y \lr{ \Be_1 \Be_2 + \Be_2 \Be_1 }
=
x^2 + y^2
+
x y \lr{ \Be_1 \Be_2 + \Be_2 \Be_1 }.
\end{dmath}

The contraction axiom requires the bivector terms to sum to zero, as also demonstrated previously for the specific example \( \Bx = \Be_1 + \Be_2 \).

More generally for \R{N}

\begin{dmath}\label{eqn:multiplication:121}
\Bx^2
=
\lr{ \sum_i x_i \Be_i }
\lr{ \sum_j x_j \Be_j }
=
\sum_{ij} x_i x_j \Be_i \Be_j
=
\sum_{i = j} x_i x_j \Be_i \Be_j
+
\sum_{i \ne j} x_i x_j \Be_i \Be_j
=
\sum_{i} x_i^2
+
\sum_{i \ne j} x_i x_j \Be_i \Be_j
=
\sum_{i} x_i^2
+
\sum_{i < j} x_i x_j (\Be_i \Be_j + \Be_j \Be_i).
\end{dmath}

The contraction axiom requires all the bivector pairs to sum to zero.  That is, for each \( i \ne j \)

\begin{dmath}\label{eqn:introGAproblems:140}
\Be_i \Be_j = -\Be_j \Be_i.
\end{dmath}
} % answer

            %
% Copyright © 2016 Peeter Joot.  All Rights Reserved.
% Licenced as described in the file LICENSE under the root directory of this GIT repository.
%

\makeproblem{Normal anticommutation}{problem:multiplication:unitsquare}{
Prove \cref{thm:multiplication:anticommutationNormal}.
} % problem

\makeanswer{problem:multiplication:unitsquare}{
Consider the square of a vector \( \Bx = u \Bu + v \Bv \) with respect to a basis of unit vectors \( \setlr{ \Bu, \Bv }\).  That is

\begin{dmath}\label{eqn:introGAproblems:160}
\Bx^2
=
\lr{ u \Bu + v \Bv }
\lr{ u \Bu + v \Bv }
=
u^2 \Bu^2
+ v^2 \Bv^2
+ u v \lr{ \Bu \Bv + \Bv \Bu }
=
u^2
+ v^2
+ u v \lr{ \Bu \Bv + \Bv \Bu }.
\end{dmath}

If these vectors are normal \( \Bx^2 = u^2 + v^2 \), which means
\begin{dmath}\label{eqn:introGAproblems:180}
\Bu \Bv = -\Bv \Bu.
\end{dmath}

Observe that a side effect of this computation shows that the traditional vector dot product of two unit vectors can also be written as a symmetric bivector sum

\begin{dmath}\label{eqn:introGAproblems:200}
\Bu \cdot \Bv = \inv{2} \lr{ \Bu \Bv + \Bv \Bu }.
\end{dmath}
} % answer

      \section{Definitions}
         %
% Copyright � 2016 Peeter Joot.  All Rights Reserved.
% Licenced as described in the file LICENSE under the root directory of this GIT repository.
%
%{
%%\input{../blogpost.tex}
%%\renewcommand{\basename}{multiplication}
%%%\renewcommand{\dirname}{notes/phy1520/}
%%\renewcommand{\dirname}{notes/ece1228-electromagnetic-theory/}
%%%\newcommand{\dateintitle}{}
%%%\newcommand{\keywords}{}
%%
%%\input{../peeter_prologue_print2.tex}
%%
%%\usepackage{peeters_layout_exercise}
%%\usepackage{peeters_braket}
%%\usepackage{peeters_figures}
%%\usepackage{siunitx}
%%%\usepackage{mhchem} % \ce{}
%%%\usepackage{macros_bm} % \bcM
%%%\usepackage{txfonts} % \ointclockwise
%%
%%\beginArtNoToc
%%
%%\generatetitle{Vector multiplication}
%%%\chapter{Vector multiplication}
%%%\label{chap:multiplication}
%%

A few new GA terms have been introduced in an ad-hoc fashion as required.  Here is a systematic exposition of some of the key definitions used to refer to the types of the geometric objects that will be encountered.

The grade of a scalar, vector, bivector, and trivector are 0, 1, 2, and 3 respectively.

\makedefinition{Pseudoscalar.}{def:multiplication:pseudoscalar}{
A blade with grade that matches the dimension of the space.
}

In a two dimensional space \( \Be_2 \Be_1 \) is a pseudoscalar.  In a three dimensional space
\( \Be_3 \Be_1 \Be_2 \) is a pseudoscalar, as is \( \Be_3 \Be_1 (\Be_2 + \Be_3 ) \).  A pseudoscalar has an implied orientation, which can be
associated with the handedness of the underlying basis.  It is conventional to refer to

\begin{dmath}\label{eqn:definitions:320}
i = \Be_1 \Be_2,
\end{dmath}

as ``the pseudoscalar'' for a two dimensional space, and to

\begin{dmath}\label{eqn:definitions:340}
I = \Be_1 \Be_2 \Be_3,
\end{dmath}

as ``the pseudoscalar'' for a three dimensional space.

\makedefinition{Multivector.}{def:multiplication:multivector}{
A sum of zero or more blades.
}

Examples include
\begin{dmath}\label{eqn:multiplication:280}
\begin{aligned}
&3 \\
& 1 + \Be_1 \Be_2 \\
& 2 - \Be_1 \Be_2 \Be_3 \\
& \Be_1 + 2 \Be_1 \Be_2 + \Be_2 \Be_3 - 3 \Be_3 \Be_1 + \Be_1 \Be_2 \Be_4
\end{aligned}
\end{dmath}

\makedefinition{Dual}{dfn:definitions:dual}{
The dual of a multivector is the product of that multivector with a pseudoscalar for a subspace that contains the multivector.  Such multiplication is referred to as a duality transformation.
} % definition

For example, given \( M = 1 + \Be_1 \Be_2 \), multiplication by \( i = \Be_1 \Be_2 \) is a duality transformation with respect to the x-y plane, and multiplication by \( I = \Be_1 \Be_2 \Be_3 \) is a duality transformation with \R{3}.

With the following shorthand notation is convienent for sucessive products of orthonormal basis vectors

\begin{dmath}\label{eqn:definitions:420}
\Be_{ij\cdots k}  = \Be_i \Be_j \cdots \Be_k,
\end{dmath}

consider some concrete examples of duality transformations of blades.
The dual vectors to the basis vectors of a 2D space are those same vectors rotated by \( \pi/2 \)

\begin{dmath}\label{eqn:definitions:360}
\begin{aligned}
\Be_1 \Be_{12} &= \Be_2 \\
\Be_2 \Be_{12} &= -\Be_1,
\end{aligned}
\end{dmath}

with an inverse duality transformation given by the multiplication with \( \Be_{12}^{-1} = \Be_{21} \)

\begin{dmath}\label{eqn:definitions:440}
\begin{aligned}
\Be_2 \Be_{21} &= \Be_1 \\
-\Be_1 \Be_{21} &= \Be_2.
\end{aligned}
\end{dmath}

The \R{3} duals to the basis vectors are bivectors

\begin{dmath}\label{eqn:definitions:380}
\begin{aligned}
\Be_1 \Be_{123} &= \Be_{23} \\
\Be_2 \Be_{123} &= \Be_{31} \\
\Be_3 \Be_{123} &= \Be_{12},
\end{aligned}
\end{dmath}

whereas the duals to those bivectors with respect to the pseudoscalar \( I^{-1} = \Be_{321} \) are the original basis vectors

\begin{dmath}\label{eqn:definitions:400}
\begin{aligned}
\Be_{23} \Be_{321} &= \Be_1 \\
\Be_{31} \Be_{321} &= \Be_2 \\
\Be_{12} \Be_{321} &= \Be_3.
\end{aligned}
\end{dmath}

In a sense that can be defined more precisely once the general dot product operator is defined, the dual to a given blade represents an object that is normal to the original blade.

The dual of any scalar is a pseudoscalar, whereas the dual of a pseudoscalar is a scalar.

%When working with multivector integrals it will be useful to consider the differential volume element a volume weighted pseudoscalar.

% relative path because this is shared with gabookI
%
% Copyright � 2016 Peeter Joot.  All Rights Reserved.
% Licenced as described in the file LICENSE under the root directory of this GIT repository.
%
\index{reverse}
\makedefinition{Reverse}{dfn:reverse:1}{

Let \( A \) be a multivector with j multivector factors,
\( A = B_1 B_2 \cdots B_j \),
not necessarily normal.
The reverse \( A^\dagger \), or reversion, of this multivector \( A \) is
\begin{equation*}
A^\dagger = B_j^\dagger B_{j-1}^\dagger \cdots B_1^\dagger.
\end{equation*}
Scalars and vectors are their own reverse, and
the reverse of a sum of multivectors is the sum of the reversions of its summands.
} % definition

Examples:
\begin{dmath}\label{eqn:reverseDefined:21}
\begin{aligned}
\lr{ 1 + 2 \Be_{12} + 3 \Be_{321} }^\dagger &= 1 + 2 \Be_{21} + 3 \Be_{123} \\
\lr{ (1 + \Be_1)(\Be_{23} - \Be_{12} }^\dagger &= (\Be_{32} + \Be_{12})(1 + \Be_1).
\end{aligned}
\end{dmath}


Given a k-blade \( A_k = \Ba_1 \Ba_2 \cdots \Ba_k \), then

\begin{dmath}\label{eqn:scalarPermutation:81}
\begin{aligned}
A_k^\dagger
&= \Ba_k \Ba_{k-1} \cdots \Ba_1 \\
&= (-1)^{k-1} \Ba_1 \Ba_k \Ba_{k-1} \cdots \Ba_2 \\
&= (-1)^{k-1} (-1)^{k-2} \Ba_1 \Ba_2 \Ba_k \Ba_{k-1} \cdots \Ba_3 \\
&\vdots \\
&= (-1)^{k-1} (-1)^{k-2} \cdots (-1)^1 \Ba_1 \Ba_2 \cdots \Ba_k,
\end{aligned}
\end{dmath}

or
\begin{dmath}\label{eqn:scalarPermutation:101}
A_k^\dagger = (-1)^{k(k-1)/2} A_k.
\end{dmath}

%%%}
%%%\EndArticle
%%\EndNoBibArticle

         \subsection{Problems}
            %
% Copyright © 2016 Peeter Joot.  All Rights Reserved.
% Licenced as described in the file LICENSE under the root directory of this GIT repository.
%

\index{complex imaginary}
\index{pseudoscalar}

Using the reversion operation it is simple to show that the \R{3} pseudoscalar
behaves like a complex imaginary with \( I^2 = -1 \)
\begin{dmath}\label{eqn:R3PseudoscalarSquare:3310}
I^2
=
I (-I^\dagger)
=
-
(\Be_1 \Be_2 \Be_3)(\Be_3 \Be_2 \Be_1)
=
-
\Be_1 \Be_2 \Be_2 \Be_1
=
-
\Be_1 \Be_1
=
-1.
\end{dmath}

      \section{Grade selection, dot and wedge product operators}
         %
% Copyright � 2016 Peeter Joot.  All Rights Reserved.
% Licenced as described in the file LICENSE under the root directory of this GIT repository.
%
%{
%\input{../blogpost.tex}
%\renewcommand{\basename}{gradeselection}
%%\renewcommand{\dirname}{notes/phy1520/}
%\renewcommand{\dirname}{notes/ece1228-electromagnetic-theory/}
%%\newcommand{\dateintitle}{}
%%\newcommand{\keywords}{}
%
%\input{../peeter_prologue_print2.tex}
%
%\usepackage{peeters_layout_exercise}
%\usepackage{peeters_braket}
%\usepackage{peeters_figures}
%\usepackage{siunitx}
%%\usepackage{mhchem} % \ce{}
%%\usepackage{macros_bm} % \bcM
%\usepackage{macros_qed} % \qedmarker
%%\usepackage{txfonts} % \ointclockwise
%
%\beginArtNoToc
%
%\generatetitle{XXX}
%%\chapter{XXX}
%%\label{chap:gradeselection}
%
Having defined the axioms and definitions of Geometric Algebra, it desirable to define the grade selection operator, the dot product operator and the wedge product operator, and consider some simple examples of each.

\makedefinition{Grade selection operator}{dfn:gradeselection:gradeselection}{
Given a multivector \( M \) containing k-grade components \( M_k \)

\begin{equation*}
M = \sum_{i = 0}^N M_i,
\end{equation*}

the grade selection operator is defined as

\begin{equation*}\label{eqn:gradeselection:40}
\gpgrade{M}{k} \equiv M_k.
\end{equation*}

Selection of the (scalar) zero grade is often written as
\begin{equation*}
\gpgradezero{M} \equiv \gpgrade{M}{0} = M_0.
\end{equation*}
}

For example, if \( M = 3 - \Be_3 + 2 \Be_1 \Be_2 \), then
\begin{equation}\label{eqn:gradeselection:80}
\begin{aligned}
\gpgradezero{M} &= 3 \\
\gpgrade{M}{1} &= - \Be_3 \\
\gpgrade{M}{2} &= 2 \Be_1 \Be_2 \\
\gpgrade{M}{3} &= 0.
\end{aligned}
\end{equation}

\makedefinition{Dot product}{dfn:gradeselection:100}{
The dot (or inner) product of two multivectors

\begin{equation*}
\begin{aligned}
A &= \sum_{i = 0}^N A_i, \\
B &= \sum_{i = 0}^N B_i,
\end{aligned}
\end{equation*}

is defined as
\begin{equation*}
A \cdot B \equiv
\sum_{i,j = 0}^N \gpgrade{ A_i B_j }{\Abs{i - j}}
\end{equation*}
} % definition

As an example, consider two vectors in a 2D space

\begin{dmath}\label{eqn:gradeselection:140}
\begin{aligned}
\Ba  &= \lr{ x \Be_1 + y \Be_2 } \\
\Ba' &= \lr{ x' \Be_1 + y' \Be_2 },
\end{aligned}
\end{dmath}

for which this definition of the dot product gives

\begin{dmath}\label{eqn:gradeselection:160}
\Ba \cdot \Ba'
=
\gpgrade{ \Ba \Ba' }{\Abs{1 - 1}}
=
\gpgradezero{ \Ba \Ba' }
=
\gpgradezero{ \lr{ x \Be_1 + y \Be_2 } \lr{ x' \Be_1 + y' \Be_2 } }
=
\gpgradezero{ x x' \Be_1^2 + y y' \Be_2^2 + (x y' - y x') \Be_1 \Be_2 }
=
x x' + y y'.
\end{dmath}

It is left to the reader (\cref{problem:gradeselection:RnDotProduct}) to show that this definition also reduces to the traditional \R{n} dot product.

As a second example, consider the dot product of a vector with a bivector.  With \( \Ba \) as defined in \cref{eqn:gradeselection:140} and \( i = \Be_1 \Be_2 \)

\begin{dmath}\label{eqn:gradeselection:240}
\Ba \cdot i
=
\gpgrade{ \Ba i }{1}
=
\gpgrade{ \lr{ x \Be_1 + y \Be_2 } \Be_1 \Be_2 }{1}
=
\gpgrade{ x \Be_1^2 \Be_2 + y \Be_2 (-\Be_2 \Be_1) }{1}
=
\gpgrade{ x \Be_2 - y \Be_1 }{1}
=
x \Be_2 - y \Be_1.
\end{dmath}

This particular dot product is trivial, since the product \( \Ba i \) has only a vector component.
In this example \( i \) is the pseudoscalar for the two dimensional space, and it can be observed that multiplication of a vector from the right serves to rotate the vector by 90 degrees.  It is not a coincidence that this is strikingly similar to the action of the imaginary from complex algebra.  It can be shown (\cref{problem:gradeselection:PlaneRotations})
that \( e^{i\theta} \) acts as a rotation operator as it does in complex algebra, and that a GA representation of complex numbers is possible (\cref{problem:gradeselection:ComplexNumbers}).

For a non-trivial vector-bivector dot product, consider

\begin{dmath}\label{eqn:gradeselection:560}
\lr{ \Be_1 + \Be_2 } \cdot \lr{ \Be_1 \Be_2 + 3 \Be_2 \Be_3 }
=
\gpgradeone{
\lr{ \Be_1 + \Be_2 } \lr{ \Be_1 \Be_2 + 3 \Be_2 \Be_3 }
}
=
\gpgradeone{
\Be_1^2 \Be_2 + 3 \Be_1 \Be_2 \Be_3
+
\Be_2 \Be_1 \Be_2 + 3 \Be_2^2 \Be_3
}
=
\gpgradeone{
\Be_2 + 3 \cancel{\Be_1 \Be_2 \Be_3}
-
\Be_1 + 3 \Be_3
}
=
\Be_2 - \Be_1 + 3 \Be_3.
\end{dmath}

The vector-bivector dot product filters out products that no common factors, since such products result in trivector components.

\makedefinition{Wedge product.}{dfn:gradeselection:480}{
For the multivectors \( A, B \) defined in \cref{dfn:gradeselection:100}, the wedge (or outer) product is defined as

\begin{equation*}
A \wedge B
\equiv
\sum_{i,j = 0}^N \gpgrade{ A_i B_j }{i + j}.
\end{equation*}
} % definition

For example, the wedge product of the 2D vectors of \cref{eqn:gradeselection:140} is

\begin{dmath}\label{eqn:gradeselection:500}
\Ba \wedge \Bb
=
\gpgradetwo{
\lr{ x \Be_1 + y \Be_2 }
\lr{ x' \Be_1 + y' \Be_2 }
}
=
\gpgradetwo{
(x x' + y y') + (x y' - x' y) \Be_1 \Be_2
}
=
(x y' - x' y) \Be_1 \Be_2.
\end{dmath}

The wedge product of two vectors in a plane contains an antisymmetrized sum of the vector coefficients, but is weighted by a ``unit'' bivector, the pseudoscalar for the plane.

As another example consider

\begin{dmath}\label{eqn:gradeselection:520}
\Be_1 \wedge \lr{ 2\Be_1 + 3 \Be_2 }
=
\gpgradetwo{
\Be_1 \lr{ 2\Be_1 + 3 \Be_2 }
}
=
\gpgradetwo{
2 \Be_1^2 + 3 \Be_1 \Be_2
}
=
3 \Be_1 \Be_2.
\end{dmath}

Components of the vectors are that colinear are filtered out.  In this case that is the \( \Be_1 \) component of the second vector \( \Be_1 + 3 \Be_2 \).  It is not coincidence that this is also a property of the cross product.  That relationship will be explored in (\cref{problem:gradeselection:WedgeRelationshipToCrossProduct}).

As a final example, consider the wedge product of a vector with a bivector

\begin{dmath}\label{eqn:gradeselection:540}
\Be_1 \wedge \lr{ \Be_1 \Be_2 - 7 \Be_2 \Be_3 }
=
\gpgradethree{
\Be_1 \lr{ \Be_1 \Be_2 - 7 \Be_2 \Be_3 }
}
=
\gpgradethree{
\Be_1^2 \Be_2 - 7 \Be_1 \Be_2 \Be_3
}
=
- 7 \Be_1 \Be_2 \Be_3.
\end{dmath}

Because \( \Be_1 \Be_2 \) has a common factor with \( \Be_1 \) it is filtered out of the resulting wedge product.  The end result, in this case, is a 3D pseudoscalar.

The wedge product of two bivectors in \R{3}, by this definition, is always zero, since there can be no grade 4 term in such a product.  It is also the case that the components of any \R{3} bivectors wedged together will also have a common factor, which nessesarily kills the wedge product of any two \R{3} bivectors.  This is not the case for arbitrary \R{N} bivectors, an example of which is \( \Be_1 \Be_2 + \Be_3 \Be_4 \).  There is no common factor in this bivector, so it can be wedged with itself and still produce a non-zero result (i.e. this bivector is not a blade).

%}
%\EndNoBibArticle

         \subsection{Problems}
            %
% Copyright © 2016 Peeter Joot.  All Rights Reserved.
% Licenced as described in the file LICENSE under the root directory of this GIT repository.
%

\makeproblem{\R{n} dot product.}{problem:gradeselection:RnDotProduct}{
Show that \ref{dfn:gradeselection:100} when applied to two vectors
is equivalent to the traditional \R{n} dot product.
} % problem

\makeanswer{problem:gradeselection:RnDotProduct}{
Let
\begin{dmath}\label{eqn:gradeselectionProblems:180}
\begin{aligned}
\Bx &= \sum_{i=1}^N x_i \Be_i \\
\By &= \sum_{i=1}^N y_i \Be_i.
\end{aligned}
\end{dmath}

The dot product of these two vectors is
\begin{dmath}\label{eqn:gradeselectionProblems:200}
\Bx \cdot \By
\equiv
\gpgradezero{ \Bx \By }
=
\gpgradezero{
\lr{ \sum_{i=1}^N x_i \Be_i}
\lr{ \sum_{j=1}^N y_j \Be_j}
}
=
\sum_{1 \le i = j \le N}
x_i y_j
\gpgradezero{ \Be_i \Be_j }
+
\sum_{1 \le i \ne j \le N}
x_i y_j
\gpgradezero{ \Be_i \Be_j }
\end{dmath}

In the \( i = j \) sum, the term \( \Be_i \Be_j = \Be_i^2 = 1 \), so the scalar grade selection of that multivector product is just 1.  In the \( i = j \) term, each of the \( \Be_i \Be_j \) products is a bivector, so each of those scalar grade selections is zero.

That leaves

\begin{dmath}\label{eqn:gradeselectionProblems:220}
\Bx \cdot \By
=
\sum_{i =1}^N x_i y_i. \qedmarker
\end{dmath}
} % answer

            %
% Copyright © 2016 Peeter Joot.  All Rights Reserved.
% Licenced as described in the file LICENSE under the root directory of this GIT repository.
%

\makeproblem{}{problem:gradeselection:cyclicpermutationtwo}{
Show that
\begin{dmath}\label{eqn:gradeselection:580}
\gpgradezero{ \Bx \By }
=
\gpgradezero{ \By \Bx }.
\end{dmath}
} % problem

\makeanswer{problem:gradeselection:cyclicpermutationtwo}{

Expanding the vector grade zero selection in coordinates gives
\begin{dmath}\label{eqn:gradeselectionProblems:680}
\gpgradezero{ \Bx \By }
=
\sum_{ij} \gpgradezero{ x_i y_j \Be_i \Be_j }
=
\sum_{i = j} \gpgradezero{ x_i y_i \Be_i \Be_i }
=
\sum_{i = j} \gpgradezero{ (y_i \Be_i)(x_i \Be_i) }
=
\sum_{i,j} \gpgradezero{ (y_i \Be_i)(x_i \Be_i) }
=
\gpgradezero{ \By \Bx }.
\end{dmath}
} % answer

            %
% Copyright © 2016 Peeter Joot.  All Rights Reserved.
% Licenced as described in the file LICENSE under the root directory of this GIT repository.
%
\makeproblem{Dot product of vectors as symmetric sum}{problem:gradeselection:dotprod}{
Show that the dot product of two vectors can be written as a symmetric sum

\begin{dmath}\label{eqn:gradeselection:600}
\Bx \cdot \By = \inv{2} \lr{ \Bx \By + \By \Bx }.
\end{dmath}
} % problem

\makeanswer{problem:gradeselection:dotprod}{
There are a few ways that this can be demonstrated.  The first relies on the classical definition of the dot product.  Expanding the square of a vector sum gives

\begin{dmath}\label{eqn:gradeselectionProblems:700}
(\Bx + \By)^2 = \Bx^2 + \By^2 + \Bx \By + \By \Bx.
\end{dmath}

By comparison this must also be equal to

\begin{dmath}\label{eqn:gradeselectionProblems:720}
\Norm{\Bx + \By}^2 = \Bx^2 + \By^2 + 2 \Bx \cdot \By,
\end{dmath}

so
\begin{dmath}\label{eqn:gradeselectionProblems:740}
\Bx \By + \By \Bx = 2 \Bx \cdot \By.
\end{dmath}

This might be viewed as a cheat, since it is not using the dot product as defined by grade zero selection according to \cref{dfn:gradeselection:100}.  Using that definition will produce the same result

\begin{dmath}\label{eqn:gradeselectionProblems:760}
\gpgradezero{ (\Bx + \By)^2 }
=
\gpgradezero{ \Bx^2 + \By^2 + \Bx \By + \By \Bx }
=
\Bx^2 + \By^2
+
\gpgradezero{
\Bx \By }
+ \gpgradezero{ \By \Bx }.
\end{dmath}

It was shown in \cref{problem:gradeselection:cyclicpermutationtwo} that \( \gpgradezero{ \Bx \By } = \gpgradezero{ \By \Bx } \) so
\begin{dmath}\label{eqn:gradeselectionProblems:780}
2 \gpgradezero{ \Bx \By } = \Bx \By + \By \Bx.
\end{dmath}

Using \cref{dfn:gradeselection:100}, this completes the problem.
} % answer

            %
% Copyright © 2016 Peeter Joot.  All Rights Reserved.
% Licenced as described in the file LICENSE under the root directory of this GIT repository.
%
\makeproblem{Plane rotations.}{problem:gradeselection:PlaneRotations}{

With \( i = \Be_1 \Be_2 \) for the pseudoscalar of the \( x,y \) plane,

\makesubproblem{}{problem:gradeselection:3:b}
justify the assertion that \( e^{i \theta} = \cos\theta + i \sin\theta \), where \( theta \) is a scalar angle.

\makesubproblem{}{problem:gradeselection:3:c}
Show that right multiplication of a 2D vector by \( e^{i\theta} \) rotates that vector by \( \theta \) radians.

\makesubproblem{}{problem:gradeselection:3:d}
Does the rotation multivector \( e^{i\theta} \) commute with the 2D basis vectors?

\makesubproblem{}{problem:gradeselection:3:e}
What is the action of multiplication of a vector by \( e^{i\theta} \) from the left?
} % problem

\makeanswer{problem:gradeselection:PlaneRotations}{
\makeSubAnswer{}{problem:gradeselection:3:b}

Assume that the exponential of a multivector argument is represented by a Taylor series

\begin{dmath}\label{eqn:gradeselectionProblems:280}
e^X = \sum_{k = 0}^\infty \frac{X^k}{k!},
\end{dmath}

and note that the pseudoscalar commutes with scalar rotation angles \( \theta \), so
\begin{dmath}\label{eqn:gradeselectionProblems:300}
e^{i\theta}
= \sum_{k = 0}^\infty \frac{(i\theta)^k}{k!}
= \sum_{k = 0}^\infty \frac{i^k\theta^k}{k!}
=
\sum_{k = 0}^\infty \frac{i^{2k}\theta^{2k}}{(2k)!}
+
\sum_{k = 0}^\infty \frac{i^{2k + 1}\theta^{2k +1}}{(2k + 1)!}
=
\sum_{k = 0}^\infty \frac{(-1)^{k}\theta^{2k}}{(2k)!}
+
i \sum_{k = 0}^\infty \frac{(-1)^{k}\theta^{2k +1}}{(2k + 1)!}
= \cos \theta + i \sin\theta.
\end{dmath}
\makeSubAnswer{}{problem:gradeselection:3:c}

Consider the action of the exponential on each of the unit vectors.  For \( \Be_1 \) that is

\begin{dmath}\label{eqn:gradeselectionProblems:320}
\Be_1 e^{i \theta}
=
\Be_1 \lr{ \cos\theta + i \sin\theta }
=
\Be_1 \cos\theta + \Be_1 (\Be_1 \Be_2 )\sin\theta
=
\Be_1 \cos\theta + \Be_2 \sin\theta.
\end{dmath}

This shows that the vector \( \Be_1 \) is rotated counterclockwise by \( \theta \) radians.  Similarly for \( \Be_2 \)

\begin{dmath}\label{eqn:gradeselectionProblems:340}
\Be_2 e^{i \theta}
=
\Be_2 \lr{ \cos\theta + i \sin\theta }
=
\Be_2 \cos\theta + \Be_1 (\Be_1 \Be_2 )\sin\theta
=
\Be_2 \cos\theta + \Be_1 (-\Be_2 \Be_1) \sin\theta.
=
\Be_2 \cos\theta - \Be_1 \sin\theta.
\end{dmath}

This is also a rotation by \( \theta \) radians.  Given a vector \( \Bx = x \Be_1 + y \Be_2 \), this gives

\begin{dmath}\label{eqn:gradeselectionProblems:360}
\Bx'
= \Bx e^{i\theta}
=
x \lr{ \Be_1 \cos\theta + \Be_2 \sin\theta } + y \lr{ \Be_2 \cos\theta - \Be_1 \sin\theta }.
\end{dmath}

In particular

\begin{dmath}\label{eqn:gradeselectionProblems:380}
\begin{bmatrix}
\Bx' \cdot \Be_1 \\
\Bx' \cdot \Be_2 \\
\end{bmatrix}
=
\begin{bmatrix}
x \cos\theta - y \sin\theta \\
x \sin\theta + y \cos\theta
\end{bmatrix}
=
\begin{bmatrix}
\cos\theta &- \sin\theta \\
\sin\theta &+ \cos\theta
\end{bmatrix}
\begin{bmatrix}
x \\
y
\end{bmatrix}.
\end{dmath}

Observe that this is the rotation matrix that takes the points \((x, y)\) to their position \((x', y')\) rotated by \( \theta \) radians.
\makeSubAnswer{}{problem:gradeselection:3:d}

The action from the left on \( \Be_1 \) is

\begin{dmath}\label{eqn:gradeselectionProblems:400}
e^{i\theta} \Be_1
=
\lr{ \cos\theta + \Be_1 \Be_2 \sin\theta} \Be_1
=
\Be_1 \cos\theta + \Be_1 \Be_2 \Be_1 \sin\theta
=
\Be_1 \cos\theta + \Be_1 (-\Be_1 \Be_2) \sin\theta
=
\Be_1 \lr{ \cos\theta - i \sin\theta }
=
\Be_1 e^{-i\theta},
\end{dmath}

and the action from the left on \( \Be_2 \) is

\begin{dmath}\label{eqn:gradeselectionProblems:420}
e^{i\theta} \Be_2
=
\lr{ \cos\theta + \Be_1 \Be_2 \sin\theta} \Be_2
=
\Be_2 \cos\theta + \Be_1 \sin\theta
=
\Be_2 \cos\theta + (\Be_2 \Be_2) \Be_1 \sin\theta
=
\Be_2 \lr{ \cos\theta - i \sin\theta }
=
\Be_2 e^{-i\theta}.
\end{dmath}

This change of sign is due to the fact that the pseudoscalar anticommutes with each of the basis vectors in the plane.

\makeSubAnswer{}{problem:gradeselection:3:e}

Since the exponential toggles sign on commutation with both of the vectors of the plane, the rotation operation can be applied from either left or right, with sufficient care to get the direction right

\begin{equation}\label{eqn:gradeselectionProblems:440}
\Bx e^{i\theta} = e^{-i\theta} \Bx.
\end{equation}

It is also possible to split the rotation operation into half angle rotation operators that act from both the left and right

\begin{dmath}\label{eqn:gradeselectionProblems:460}
\Bx' = e^{-i\theta/2} \Bx e^{i\theta/2}.
\end{dmath}

A student who has studied computer graphics rotation theory may have seen quaternion rotation operators with this form, and a student of quantum mechanics will have seen Pauli matrix rotation operations of this form.  This is, in fact, the form that is generally desirable for 3D or higher order rotations, since it rotates the portions of a vector that lie in the rotation plane, leaving the normal components untouched.
} % answer

            %
% Copyright © 2016 Peeter Joot.  All Rights Reserved.
% Licenced as described in the file LICENSE under the root directory of this GIT repository.
%
\makeproblem{Complex numbers}{problem:gradeselection:ComplexNumbers}{
Show that complex numbers can be represented as even grade multivectors \( z = a + \Be_1 \Be_2 b \).
} % problem

\makeanswer{problem:gradeselection:ComplexNumbers}{
Let \( i = \Be_1 \Be_2 \), for which we find

\begin{dmath}\label{eqn:gradeselectionProblems:260}
i^2
=
\lr{ \Be_1 \Be_2 }
\lr{ \Be_1 \Be_2 }
=
\Be_1 (\Be_2 \Be_1) \Be_2
=
\Be_1 (-\Be_1 \Be_2) \Be_2
=
-(\Be_1^2) (\Be_2^2)
=
-1.
\end{dmath}

The even grade multivector \( z = a + i b \) is thus seen to have all the properties required of complex numbers.
} % answer

            %
% Copyright © 2016 Peeter Joot.  All Rights Reserved.
% Licenced as described in the file LICENSE under the root directory of this GIT repository.
%
\makeproblem{\R{3} pseudoscalar commutation.}{problem:gradeselection:R3PseudoscalarCommutation}{
Show that \( I \) given by \cref{eqn:definitions:340}
commutes with any grade \R{3} multivector.
} % problem

\makeanswer{problem:gradeselection:R3PseudoscalarCommutation}{

Showing that \( I \) commutes with each of the basis vectors is sufficient

\begin{dmath}\label{eqn:gradeselectionProblems:620}
\Be_1 I
=
\Be_1 (\Be_1 \Be_2 \Be_3)
=
\Be_1 (-\Be_2 \Be_1) \Be_3
=
-\Be_1 \Be_2 (-\Be_3 \Be_1)
=
I \Be_1
\end{dmath}
\begin{dmath}\label{eqn:gradeselectionProblems:640}
\Be_2 I
=
\Be_2 (\Be_1 \Be_2 \Be_3)
=
\Be_2 \Be_1 (-\Be_3 \Be_2)
=
-(-\Be_1 \Be_2) \Be_3 \Be_2
=
I \Be_2.
\end{dmath}
\begin{dmath}\label{eqn:gradeselectionProblems:660}
\Be_3 I
=
\Be_3 (\Be_1 \Be_2 \Be_3)
=
(\Be_3 \Be_1 \Be_2) \Be_3
=
-(\Be_1 \Be_3) \Be_2 \Be_3
=
-\Be_1 (-\Be_2 \Be_3) \Be_3
=
I \Be_3. \qedmarker
\end{dmath}
} % answer

            %
% Copyright © 2016 Peeter Joot.  All Rights Reserved.
% Licenced as described in the file LICENSE under the root directory of this GIT repository.
%
\makeproblem{Vector wedge coordinate expansion and antisymmetry}{problem:gradeselection:vectorwedge}{
Show that
\begin{dmath}\label{eqn:gradeselection:620}
\Bx \wedge \By
=
%\sum_{i < j} (x_i y_j - x_j y_i) \Be_i \Be_j.
\sum_{i < j}
\begin{vmatrix}
x_i & x_j \\
y_i & y_j
\end{vmatrix}
\Be_i \Be_j.
\end{dmath}

Observe from this coordinate expansion that the wedge product of two vectors is antisymmetric
\boxedEquation{eqn:gradeselection:640}{
\Bx \wedge \By = -\By \wedge \Bx.
}
} % problem

\makeanswer{problem:gradeselection:vectorwedge}{
Given \( \Bx = \sum_i x_i \Be_i \), and \( \By = \sum_i y_i \Be_i \), the wedge of these two vectors is a grade two selection that picks out only products that differ in index

\begin{dmath}\label{eqn:gradeselectionProblems:580}
\Bx \wedge \By
=
\gpgradetwo{ \Bx \By }
=
\sum_{i,j} \gpgradetwo{ x_i \Be_i y_j \Be_j }
=
\sum_{i \ne j} x_i y_j \gpgradetwo{ \Be_i \Be_j }
=
\sum_{i \ne j} x_i y_j \Be_i \Be_j
=
\sum_{i < j} (x_i y_j - x_j y_i) \Be_i \Be_j
=
\sum_{i < j}
\begin{vmatrix}
x_i & x_j \\
y_i & y_j
\end{vmatrix}
\Be_i \Be_j.
\end{dmath}

When \( \Bx, \By \), the rows in each of the above determinants will swap, negating the sign of each.  This implies \( \By \wedge \Bx = -\Bx \wedge \By \).
} % answer

            %
% Copyright © 2016 Peeter Joot.  All Rights Reserved.
% Licenced as described in the file LICENSE under the root directory of this GIT repository.
%

\makeproblem{Wedge relationship to the cross product.}{problem:gradeselection:WedgeRelationshipToCrossProduct}{
For a pair of \R{3} vectors \( \Bx, \By \), show that the wedge and cross products are related by
\begin{dmath}\label{eqn:gradeselectionProblems:560}
\Bx \wedge \By = I (\Bx \cross \By),
\end{dmath}

where \( I = \Be_1 \Be_2 \Be_3 \) is the \R{3} pseudoscalar.
} % problem

\makeanswer{problem:gradeselection:WedgeRelationshipToCrossProduct}{
Writing out \cref{eqn:gradeselectionProblems:580} explicitly gives

\begin{dmath}\label{eqn:gradeselectionProblems:600}
\Bx \wedge \By
=
\begin{vmatrix}
x_1 & x_2 \\
y_1 & y_2
\end{vmatrix}
\Be_1 \Be_2
+
\begin{vmatrix}
x_1 & x_3 \\
y_1 & y_3
\end{vmatrix}
\Be_1 \Be_3
+
\begin{vmatrix}
x_2 & x_3 \\
y_2 & y_3
\end{vmatrix}
\Be_2 \Be_3
=
\begin{vmatrix}
  \Be_2 \Be_3
& \Be_3 \Be_1
& \Be_1 \Be_2 \\
x_1 & x_2 & x_3 \\
y_1 & y_2 & y_3
\end{vmatrix}
=
\begin{vmatrix}
  (\Be_1 \Be_1) \Be_2 \Be_3
& \Be_3 \Be_1 (\Be_2 \Be_2)
& \Be_1 \Be_2 (\Be_3 \Be_3) \\
x_1 & x_2 & x_3 \\
y_1 & y_2 & y_3
\end{vmatrix}
=
\Be_1 \Be_2 \Be_3
\begin{vmatrix}
\Be_1 & \Be_2 & \Be_3 \\
x_1 & x_2 & x_3 \\
y_1 & y_2 & y_3
\end{vmatrix}
= I (\Bx \cross \By).
\end{dmath}
%\begin{aligned}
%(x_1 y_2 - x_2 y_1) \Be_1 \Be_2 \\
%&\quad+
%(x_1 y_3 - x_3 y_1) \Be_1 \Be_3 \\
%&\quad+
%(x_2 y_3 - x_3 y_2) \Be_2 \Be_3 \\
%&=
%(x_1 y_2 - x_2 y_1) \Be_1 \Be_2 (\Be_3 \Be_3) \\
%&\quad+
%(x_1 y_3 - x_3 y_1) \Be_1 \Be_3 (\Be_2 \Be_2) \\
%&\quad+
%(x_2 y_3 - x_3 y_2) \Be_2 \Be_3 (\Be_1 \Be_1) \\
%&=
%(x_1 y_2 - x_2 y_1) I \Be_3 \\
%&\quad+
%(x_1 y_3 - x_3 y_1) (-I) \Be_2 \\
%&\quad+
%(x_2 y_3 - x_3 y_2) I \Be_1 \\
%\end{aligned}
} % answer

            %
% Copyright © 2016 Peeter Joot.  All Rights Reserved.
% Licenced as described in the file LICENSE under the root directory of this GIT repository.
%

\makeproblem{Vector bivector dot product}{problem:gradeselection:vectorBivectorDot}{
The dot product of a vector and bivector in \R{N} (or in fact any metric) expands as

\boxedEquation{eqn:gradeselection:660}{
\Ba \cdot \lr{ \Bb \wedge \Bc }
=
-\lr{ \Bb \wedge \Bc } \cdot \Ba
=
( \Ba \cdot \Bb ) \Bc
-( \Ba \cdot \Bc ) \Bb.
}

Demonstrate this by coordinate expansion using an orthonormal basis for \R{N}.

The right hand side may look familiar.  Demonstrate, for \R{3} without expansion in coordinates, that

\boxedEquation{eqn:gradeselection:680}{
\Ba \cdot \lr{ \Bb \wedge \Bc }
=
-\Ba \cross \lr{ \Bb \cross \Bc }.
}
} % problem

\makeanswer{problem:gradeselection:vectorBivectorDot}{
Expansion in coordinates is frowned upon in a number of GA references, but can be a quick way to the results of interest.  Consider such an expansion for a \R{N} vector space

\begin{dmath}\label{eqn:gradeselectionProblems:681}
\begin{aligned}
\Ba \cdot \lr{ \Bb \wedge \Bc } &= \sum_{i, j, k} a_i b_j c_k \Be_i \cdot (\Be_j \wedge \Be_k) \\
\lr{ \Bb \wedge \Bc } \cdot \Ba &= \sum_{i, j, k} a_i b_j c_k (\Be_j \wedge \Be_k) \cdot \Be_i
\end{aligned}
\end{dmath}

Observe that these sums can be restricted to indexes \( i \ne j \), since \( \Bx \wedge \Bx = 0 \) for any \(\Bx\).  The dot products are

\begin{dmath}\label{eqn:gradeselectionProblems:820}
\Be_i \cdot (\Be_j \wedge \Be_k)
=
\gpgradeone{ \Be_i (\Be_j \wedge \Be_k) }
=
\gpgradeone{ \Be_i \Be_j \Be_k },
\end{dmath}

and
\begin{dmath}\label{eqn:gradeselectionProblems:840}
(\Be_j \wedge \Be_k) \cdot \Be_i
=
\gpgradeone{ (\Be_j \wedge \Be_k) \Be_i }
=
\gpgradeone{ \Be_j \Be_k \Be_i }.
\end{dmath}

In each expansion, there are three cases, one where \( i,j,k\) are all unique.  In this case, the vector product is a trivector, so the grade one selection is zero.  That leaves only \( i = j \ne k \), and \( i = k \ne j \).

Consider the \( i = j \) case in the first dot product expansion

\begin{dmath}\label{eqn:gradeselectionProblems:860}
\gpgradeone{ \Be_i \Be_j \Be_k }
=
\gpgradeone{ \Be_i \Be_i \Be_k }
=
\gpgradeone{ \Be_k }
=
\Be_k.
\end{dmath}

For the \( i = k \) case, this is

\begin{dmath}\label{eqn:gradeselectionProblems:880}
\gpgradeone{ \Be_i \Be_j \Be_k }
=
\gpgradeone{ \Be_i (-\Be_k \Be_j) }
=
-\gpgradeone{ \Be_i \Be_i \Be_j }
=
-\gpgradeone{ \Be_j }
=
-\Be_j.
\end{dmath}

Inspection shows that the general pattern is
\begin{dmath}\label{eqn:gradeselectionProblems:900}
\Be_i \cdot (\Be_j \wedge \Be_k) =
(\Be_i \cdot \Be_j) \Be_k
-(\Be_i \cdot \Be_k) \Be_j,
\end{dmath}

and
\begin{dmath}\label{eqn:gradeselectionProblems:920}
(\Be_j \wedge \Be_k) \cdot \Be_i =
(\Be_i \cdot \Be_k) \Be_j
-(\Be_i \cdot \Be_j) \Be_k.
\end{dmath}

Substitution back into \cref{eqn:gradeselectionProblems:681} proves the first result for Euclidean spaces.  For the relation to the cross product in the \R{3} case

\begin{dmath}\label{eqn:gradeselectionProblems:940}
\Ba \cdot \lr{ \Bb \wedge \Bc }
=
\gpgradeone{
\Ba \lr{ \Bb \wedge \Bc }
}
=
\gpgradeone{
\Ba I \lr{ \Bb \cross \Bc }
}
=
\gpgradeone{
I \Ba \lr{ \Bb \cross \Bc }
}
=
\gpgradeone{
I \lr{
\Ba \wedge \lr{ \Bb \cross \Bc }
+
\Ba \cdot \lr{ \Bb \cross \Bc }
}
}.
\end{dmath}

The dot product leaves the vector selection of a trivector, which is zero.  Expanding the wedge product as a cross product once again gives
\begin{dmath}\label{eqn:gradeselectionProblems:960}
\Ba \cdot \lr{ \Bb \wedge \Bc }
=
\gpgradeone{
I^2
\Ba \cross \lr{ \Bb \cross \Bc }
}
=
-\Ba \cross \lr{ \Bb \cross \Bc }.
\end{dmath}

} % answer

            %
% Copyright © 2016 Peeter Joot.  All Rights Reserved.
% Licenced as described in the file LICENSE under the root directory of this GIT repository.
%

\makeproblem{Vector trivector dot product}{problem:gradeselection:vectorTrivectorDot}{
Show that

\begin{dmath}\label{eqn:vectorTrivectorDot:20}
\Ba \cdot \lr{ \Bb \wedge \Bc \wedge \Bd}
=
\lr{ \Bb \wedge \Bc \wedge \Bd} \cdot \Ba
=
( \Ba \cdot \Bb ) (\Bc \wedge \Bd)
-( \Ba \cdot \Bc ) (\Bb \wedge \Bd)
+( \Ba \cdot \Bd ) (\Bb \wedge \Bc).
\end{dmath}

Note that this is another specific case of the more general identity

%\begin{dmath}\label{eqn:vectorTrivectorDot:100}
\boxedEquation{eqn:vectorTrivectorDot:100}{
\Bx \cdot \lr{ \By_1 \wedge \By_2 \wedge \cdots \wedge \By_n }
=
\sum_{i = 1}^n (-1)^i (\Bx \cdot \By_i) \lr{ \By_1 \wedge \cdots \wedge \By_{i-1} \wedge \By_{i+1} \wedge \cdots \wedge \By_n }.
}
%\end{dmath}

This dot product is symmetric(antisymmetric) when the grade of the blade the vector is dotted with is odd(even).

See \citep{doran2003gap} for a demonstration that this holds for any metric.
} % problem

\makeanswer{problem:gradeselection:vectorTrivectorDot}{
Expanding in coordinates gives

\begin{dmath}\label{eqn:vectorTrivectorDot:40}
\Ba \cdot \lr{ \Bb \wedge \Bc \wedge \Bd}
= \sum_{j \ne k \ne l} a_i b_j c_k d_l
\gpgradetwo{ \Be_i \Be_j \Be_k \Be_l }.
\end{dmath}

The products within the grade two selection operator can be of either grade two or grade four, so only the terms where one of
\( i = j \), \( i = k \), or \( i = l \) contributes.  Repeated anticommutation of the normal unit vectors can put each such pair adjacent, where they square to unity.  Those are respectively

\begin{dmath}\label{eqn:vectorTrivectorDot:60}
\begin{aligned}
\gpgradetwo{ \Be_i \Be_i \Be_k \Be_l } &= \Be_k \Be_l  \\
\gpgradetwo{ \Be_i \Be_j \Be_i \Be_l } &= -\gpgradetwo{ \Be_j \Be_i \Be_i \Be_l } = - \Be_j \Be_l \\
\gpgradetwo{ \Be_i \Be_j \Be_k \Be_i } &= -\gpgradetwo{ \Be_j \Be_i \Be_k \Be_i } = +\gpgradetwo{ \Be_j \Be_k \Be_i \Be_i } = \Be_j \Be_k
\end{aligned}
\end{dmath}

Substitution back into \cref{eqn:gradeselectionProblems:681} gives

\begin{dmath}\label{eqn:vectorTrivectorDot:80}
\Ba \cdot \lr{ \Bb \wedge \Bc \wedge \Bd}
= \sum_{j \ne k \ne l} a_i b_j c_k d_l
\lr{
\Be_i \cdot \Be_j (\Be_k \Be_l)
-
\Be_i \cdot \Be_k (\Be_j \Be_l)
+
\Be_i \cdot \Be_l (\Be_j \Be_k)
}
=
( \Ba \cdot \Bb ) (\Bc \wedge \Bd)
-( \Ba \cdot \Bc ) (\Bb \wedge \Bd)
+( \Ba \cdot \Bd ) (\Bb \wedge \Bc).
\end{dmath}

Repeating this from the other direction gives the same result.
} % answer

            
Alternatively, this result can be obtained compactly using tensor contraction techniques, first making a duality transformation and then expanding in coordinates

\begin{dmath}\label{eqn:bivectorDot:100}
(\Ba \wedge \Bb) \cdot (\Bc \wedge \Bd)
=
\gpgradezero{
-I (\Ba \cross \Bb) (-I) (\Bc \cross \Bd)
}
=
- (\Ba \cross \Bb) \cdot (\Bc \cross \Bd)
=
-(\epsilon_{ijk} \Be_i a_j b_k) \cdot (\epsilon_{r s t} \Be_r c_s d_t)
=
-\epsilon_{ijk} a_j b_k \epsilon_{i s t} c_s d_t
=
-\delta_{jk}^{[st]}
a_j b_k c_s d_t
=
-a_s b_t c_s d_t
+
a_t b_s c_s d_t
=
-(\Ba \cdot \Bc)(\Bb \cdot \Bd)
+
(\Ba \cdot \Bd)(\Bb \cdot \Bc).
\end{dmath}

A student of physics might consider this a natural alternative approach.
%Some GA authors may find this alternate derivation offensive, as it contains both an expansion by coordinates and requires an alternate toolbox.

            %
% Copyright © 2016 Peeter Joot.  All Rights Reserved.
% Licenced as described in the file LICENSE under the root directory of this GIT repository.
%

\makeproblem{\R{4} wedge of a non-blade with itself.}{problem:gradeselection:r4nonzerobivectorwedgewithself}{
While the wedge product of a blade with itself is always zero, this is not generally true of the wedge products of arbitrary k-vectors in higher dimensional spaces.
To demonstrate this, show that the wedge of the bivector 
\( B = \Be_1 \Be_2 + \Be_3 \Be_4 \) with itself is non-zero.
Why is this bivector not a blade?
%, show that \( B \wedge B \ne 0 \).
} % problem

            %
% Copyright � CCYY Peeter Joot.  All Rights Reserved.
% Licenced as described in the file LICENSE under the root directory of this GIT repository.
%
\makeproblem{Cyclic permutation within scalar selection.}{problem:scalarPermutation:1}{
Show that, for multivectors \( A \), \( B \) a cyclic permutation of the multivectors within a grade zero selection is possible

\begin{equation}\label{eqn:scalarPermutation:1}
\gpgradezero{A B C}
=
\gpgradezero{B C A}.
\end{equation}
} % problem

\makeanswer{problem:scalarPermutation:1}{

It is sufficient to show that

\begin{equation}\label{eqn:scalarPermutation:21}
\gpgradezero{A B}
=
\gpgradezero{B A}.
\end{equation}

If the maximum grade of \( A \) and \( B \) are \( a \) and \( b \) respectively, then

\begin{equation}\label{eqn:scalarPermutation:41}
\gpgradezero{A B}
=
\sum_{r = 0}^a \sum_{s = 0}^b \gpgradezero{A_r B_s}
=
\sum_{k = 0}^{\min(a,b)} \gpgradezero{A_k B_k}
\end{equation}

Because \( \gpgradezero{M}^\dagger = \gpgradezero{M^\dagger} = \gpgradezero{M} \), reversing all the factors in each of these grade zero selections leaves the result unchanged.  That is

\begin{equation}\label{eqn:scalarPermutation:61}
\gpgradezero{A B}
=
\sum_{k = 0}^{\min(a,b)} \gpgradezero{B_k^\dagger A_k^\dagger}.
\end{equation}

Using \cref{eqn:scalarPermutation:101}, this is

\begin{dmath}\label{eqn:scalarPermutation:121}
\gpgradezero{A B}
=
\sum_{k = 0}^{\min(a,b)} \lr{(-1)^{k(k-1)/2} }^2 \gpgradezero{B_k A_k}
=
\sum_{k = 0}^{\min(a,b)} \gpgradezero{B_k A_k}
=
\gpgradezero{ B A }. \qedmarker
\end{dmath}

} % answer

      \section{Product of two vectors}
         %
% Copyright � 2016 Peeter Joot.  All Rights Reserved.
% Licenced as described in the file LICENSE under the root directory of this GIT repository.
%
%{
%\input{../blogpost.tex}
%\renewcommand{\basename}{vectorproduct}
%%\renewcommand{\dirname}{notes/phy1520/}
%\renewcommand{\dirname}{notes/ece1228-electromagnetic-theory/}
%%\newcommand{\dateintitle}{}
%%\newcommand{\keywords}{}
%
%\input{../peeter_prologue_print2.tex}
%
%\usepackage{peeters_layout_exercise}
%\usepackage{peeters_braket}
%\usepackage{peeters_figures}
%\usepackage{siunitx}
%%\usepackage{mhchem} % \ce{}
%%\usepackage{macros_bm} % \bcM
%%\usepackage{macros_qed} % \qedmarker
%%\usepackage{txfonts} % \ointclockwise
%
%\beginArtNoToc
%
%\generatetitle{XXX}
%%\chapter{XXX}
%%\label{chap:vectorproduct}
%
Given two vectors \( \Bx, \By \) the scalar grade of the vector product \( \Bx \By \) was shown (\cref{problem:gradeselection:RnDotProduct}) to be
\begin{equation}\label{eqn:vectorproduct:20}
\gpgradezero{ \Bx \By }
=
\sum_{i = 1}^N x_i y_i
=
\Bx \cdot \By.
\end{equation}

The grade two selection of this product was found (\cref{problem:gradeselection:vectorwedge}) to be

\begin{equation}\label{eqn:vectorproduct:40}
\gpgradetwo{ \Bx \By }
=
\sum_{i < j}
%(x_i y_j - x_j y_i)
\begin{vmatrix}
x_i & x_j \\
y_i & y_j
\end{vmatrix}
\Be_i \Be_j
=
\Bx \wedge \By
=
-\By \wedge \Bx.
\end{equation}

The reader should convince themself that the vector product \( \Bx \By \) has only even grades (0,2), and can therefore be expanded as

\begin{dmath}\label{eqn:vectorproduct:60}
\Bx \By
=
\gpgradezero{ \Bx \By }
+
\gpgradetwo{ \Bx \By },
\end{dmath}

or
\boxedEquation{eqn:vectorproduct:80}{
\Bx \By
=
\Bx \cdot \By
+
\Bx \wedge \By.
}

This is a fundamental and very useful relationship.  In these notes this is a consequence of the axioms and the generalized definitions of the dot and wedge products.  Some authors will use this to define the geometric product of two vectors.

Using \cref{problem:gradeselection:dotprod} and \cref{eqn:vectorproduct:80} it can be shown that the wedge product is an explicit antisymmetrized sum of vector products, just as the dot product is the symmetrized vector product sum

\boxedEquation{eqn:vectorproduct:300}{
\begin{aligned}
\Bx \cdot \By &= \inv{2} \lr{ \Bx \By + \By \Bx } \\
\Bx \wedge \By &= \inv{2} \lr{ \Bx \By - \By \Bx }
\end{aligned}
}

Some authors will use these as the respective definitions of the dot and wedge products.

The non-commutative nature of the vector product was one of the first observed consequences of the axioms.  The vector product is also not generally anticommutative, as was the case for normal vectors.  Rearranging \cref{eqn:vectorproduct:300} provides the general commutation identity for two vectors

%\begin{dmath}\label{eqn:vectorproduct:320}
\boxedEquation{eqn:vectorproduct:320}{
\By \Bx = 2 \Bx \cdot \By - \Bx \By.
}
%\end{dmath}

Observe that when the vectors are perpendicular, the strict anticommutation result follows.
This can be a handy tool for abstract multivector expression manipulation.

An additional, and incredibly useful, relationship follows from \cref{eqn:vectorproduct:80} for \R{3} (\cref{problem:gradeselection:WedgeRelationshipToCrossProduct})

\boxedEquation{eqn:vectorproduct:100}{
\Bx \By
=
\Bx \cdot \By
+
I
(\Bx \cross \By).
}

This is the GA equivalent of the Pauli relationship \cref{eqn:GAmotivation:120} that will be familiar to a student of quantum spin states.

The ability to combine dot and cross product relationships into a single multivector equation is not just a theoretical nicety.  This is also one of the primary reasons that GA is so applicable to the study of electromagnetism.   To illustrate this, and provide a hint of things to come, consider the GA formulation of the electrostatic and magnetostatic Maxwell equations.

\makeexample{Electrostatic and magnetostatics.}{example:vectorproduct:electrostatics}{

With no magnetic current, no magnetic sources, and no time derivatives, Maxwell's equations in simple media take the form

\begin{dmath}\label{eqn:vectorproduct:120}
\begin{aligned}
\spacegrad \cdot \BB &= 0 \\
\spacegrad \cross \BB &= \mu \BJ \\
\spacegrad \cross \BE &= 0 \\
\spacegrad \cdot \BE &= \frac{\rho}{\epsilon}.
\end{aligned}
\end{dmath}

For electrostatic conditions \( \BJ = 0 \), so using \cref{eqn:vectorproduct:100} the first and last equations can be combined into a single first order homogeneous multivector gradient equation

\begin{equation}\label{eqn:vectorproduct:140}
\spacegrad \BB
=
\spacegrad \cdot \BB +I (\spacegrad \cross \BB )
=
0.
\end{equation}

The electric gradient equation is

\begin{equation}\label{eqn:vectorproduct:160}
\spacegrad \BE
=
\spacegrad \cdot \BE +I (\spacegrad \cross \BE )
=
\frac{\rho}{\epsilon}.
\end{equation}

Maxwell's equations are reduced to two multivector equations with this transformation
\begin{dmath}\label{eqn:vectorproduct:180}
\begin{aligned}
\spacegrad \BE &= \frac{\rho}{\epsilon} \\
\spacegrad \BB &= 0.
\end{aligned}
\end{dmath}

For magnetostatics \( \rho = 0 \), and the same assembly of Maxwell's equations gives

\begin{dmath}\label{eqn:vectorproduct:220}
\begin{aligned}
\spacegrad \BB &= I \mu \BJ \\
\spacegrad \BE &= 0.
\end{aligned}
\end{dmath}

It will be seen later that it is actually more natural to express magnetic fields as a bivector \( I \BB \).  Using \( I^2 = -1 \) (\cref{problem:gradeselection:R3PseudoscalarSquare}) the magnetostatic equation takes the form

\begin{dmath}\label{eqn:vectorproduct:240}
\spacegrad (I \BB) = - \mu \BJ.
\end{dmath}

Both the electrostatic and magnetostatic equations can be solved directly using the Green's function for the gradient, producing the Coulomb integral for the electric field and Biot-Savart's law for the magnetic field.
Before demonstrating this, the concepts required to attack multivector integrals must be formulated.
} % example

The dot plus wedge product components of the vector product have a geometrical interpretation.  To understand this, consider the components of a vector \( \By \) onto the direction of \( \Bx \) and the perpendicular.  The projection component is

\begin{dmath}\label{eqn:vectorproduct:420}
\Proj_\Bx \By = \xcap \lr{ \xcap \cdot \By },
\end{dmath}

and the rejection (the component of \( \By \) perpendicular to \( \Bx \)), is
\begin{dmath}\label{eqn:vectorproduct:440}
\RejName_\Bx \By
=
\By - \xcap \lr{ \xcap \cdot \By }
=
\Abs{\By} \lr{ \ycap - \xcap \lr{ \xcap \cdot \By } }
=
\Abs{\By} \xcap \lr{ \xcap \ycap - \xcap \cdot \By }
=
\Abs{\By} \xcap \lr{ \xcap \wedge \ycap }
=
\xcap \lr{ \xcap \wedge \By }.
\end{dmath}

FIXME: review this and see what portions if any to keep, now that this is treated in the 2D intro section.
%An example is plotted in \cref{fig:projectionAndRejection:projectionAndRejectionFig1}.
%
%\imageFigure{../figures/GAelectrodynamics/projectionAndRejectionFig1}{Projection and rejection illustrated.}{fig:projectionAndRejection:projectionAndRejectionFig1}{0.45}
%

The magnitudes of \( \xcap \lr{ \xcap \cdot \ycap } \), and \( \xcap \lr{ \xcap \wedge \ycap } \) are neccessarily the cosine and sines of the angle between \( \Bx \) and \( \By \), regardless of the dimension of the underlying vector space.  Those respective magnitudes are

\begin{dmath}\label{eqn:vectorproduct:460}
\begin{aligned}
\Abs{ \xcap \lr{ \xcap \cdot \ycap } }^2 &= \lr{ \xcap \cdot \ycap }^2 \\
\Abs{ \xcap \lr{ \xcap \wedge \ycap } }^2 &= -\lr{ \xcap \wedge \ycap }^2,
\end{aligned}
\end{dmath}

which allows an identification
\begin{dmath}\label{eqn:vectorproduct:500}
\begin{aligned}
\cos\theta &= \xcap \cdot \ycap \\
\sin\theta &= \Abs{\xcap \wedge \ycap},
\end{aligned}
\end{dmath}

where \( \Abs{\xcap \wedge \ycap} = \sqrt{ -\lr{\xcap \wedge \ycap}^2 } \).

It is now possible to express the product of vectors in a trigonometric or exponential form

\begin{dmath}\label{eqn:vectorproduct:480}
\xcap \ycap
= \xcap \cdot \ycap
+ \xcap \wedge \ycap
=
\xcap \cdot \ycap
+ \frac{\xcap \wedge \ycap}{\Abs{\xcap \wedge \ycap}} \Abs{\xcap \wedge \ycap}
=
\cos\theta
+ \frac{\xcap \wedge \ycap}{\Abs{\xcap \wedge \ycap}} \sin\theta,
\end{dmath}

or

\boxedEquation{eqn:vectorproduct:520}{
\Bx \By
=
\Abs{\Bx}
\Abs{\By}
\exp\lr{ \frac{\xcap \wedge \ycap}{\Abs{\xcap \wedge \ycap}} \theta }.
}

The interpretation of this is that the product of two vectors produces a rotation operator that acts in the plane spanned by these vectors, but also scales any such rotated vector from this plane by the product of the magnitudes of the vector product factors.  When those vectors are unit vectors, the vector product is a non-scaling rotation operator

\begin{dmath}\label{eqn:vectorproduct:560}
\xcap \ycap
=
\exp\lr{ \frac{\xcap \wedge \ycap}{\Abs{\xcap \wedge \ycap}} \theta },
\end{dmath}

that (when applied from the right) rotates any vector in \( \Span{\xcap, \ycap} \) by \( \theta \) radians in the direction of shortest rotation from \( \xcap \) to \( \ycap \), and when applied from the left rotates by \( -\theta \).

In particular, if the unit vectors are perpendicular, the rotation operator is
\begin{dmath}\label{eqn:vectorproduct:580}
R(\theta)
=
\exp\lr{ \xcap \ycap \theta }.
\end{dmath}

For \R{3} the wedge product in \cref{eqn:vectorproduct:520} can be expressed as a cross product

\begin{equation}\label{eqn:vectorproduct:540}
\frac{\xcap \wedge \ycap}{\Abs{\xcap \wedge \ycap}}
=
I \frac{\xcap \cross \ycap}{\Abs{\xcap \cross \ycap}}
=
I \ncap,
\end{equation}

This allows the \R{3} vector product to be written as

\begin{equation}\label{eqn:vectorproduct:600}
\Bx \By
=
\Abs{\Bx}
\Abs{\By}
\exp\lr{ I \ncap \theta }.
\end{equation}

In this form it is particularly easy to verify that the factor \( I \ncap \),
the dual of the normal representing the plane of rotation from \( \Bx \) to \( \By \), acts as an imaginary

\begin{dmath}\label{eqn:vectorproduct:400}
(I \ncap)^2
=
(I \ncap) (I \ncap)
=
I^2 \ncap^2
=
(-1)(1)
=
-1.
\end{dmath}

Observe the similarity between this and the complex inner product \( z w^\conj = r \rho e^{i(\theta-\alpha)} \) for the complex numbers of \cref{eqn:GAmotivation:200}.  The primary difference is that the GA imaginary factor also has a spatial orientation that the complex imaginary does not.

%}
%\EndNoBibArticle

         \subsection{Miscellanious theorems}
            %
% Copyright © 2016 Peeter Joot.  All Rights Reserved.
% Licenced as described in the file LICENSE under the root directory of this GIT repository.
%

\maketheorem{K-vector dot and wedge product relations.}{thm:bladeDotWedgeSymmetryIdentities:180}{
Given a k-vector \( B \) and a vector \( \Ba \), the dot and wedge products have the following commutation relationships
\boxedEquation{eqn:bladeDotWedgeSymmetryIdentities:200}{
\begin{aligned}
B \cdot \Ba  &= (-1)^{k-1} \Ba \cdot B \\
B \wedge \Ba &= (-1)^k \Ba \wedge B,
\end{aligned}
}
and can be expressed as symmetric and antisymmetric sums depending on the grade of the blade
\boxedEquation{eqn:bladeDotWedgeSymmetryIdentities:220}{
\begin{aligned}
\Ba \wedge B &= \inv{2}\lr{ \Ba B + (-1)^k B \Ba } \\
\Ba \cdot B &= \inv{2}\lr{ \Ba B - (-1)^k B \Ba }.
\end{aligned}
}
} % theorem

For example, if \( B \) and \( \Ba \) are both vectors, we recover \cref{thm:symmetricAndAntiSymmetricVectorSums:symmetricAndAnti}.  As an other example,
if \( B \) is a 2-vector, then
\begin{dmath}\label{eqn:bladeDotWedgeSymmetryIdentitiesTheorem:480}
\begin{aligned}
2 ( \Ba \wedge B ) &= \Ba B + B \Ba  \\
2 ( \Ba \cdot B ) &= \Ba B - B \Ba.
\end{aligned}
\end{dmath}
Observe that the dot(wedge) of two vectors is a (anti)symmetric sum of products, whereas the wedge(dot) of a vector and a bivector is an (anti)symmetric sum.

To prove \cref{thm:bladeDotWedgeSymmetryIdentities:180}, split the blade into components that intersect with and are disjoint from \( \Ba \) as follows
\begin{dmath}\label{eqn:bladeDotWedgeSymmetryIdentitiesTheorem:240}
B
=
\inv{\Ba} \Bn_1 \Bn_2 \cdots \Bn_{k-1} + \Bm_1 \Bm_2 \cdots \Bm_k,
\end{dmath}
where \( \Bn_i \) orthogonal to \( \Ba \) and each other, and where \( \Bm_i \) are all orthogonal.  The products of \( B \) with \( \Ba \) are
\begin{dmath}\label{eqn:bladeDotWedgeSymmetryIdentitiesTheorem:340}
\Ba B
=
\Ba \inv{\Ba} \Bn_1 \Bn_2 \cdots \Bn_{k-1} + \Ba \Bm_1 \Bm_2 \cdots \Bm_k
=
\Bn_1 \Bn_2 \cdots \Bn_{k-1} + \Ba \Bm_1 \Bm_2 \cdots \Bm_k,
\end{dmath}
and
\begin{dmath}\label{eqn:bladeDotWedgeSymmetryIdentitiesTheorem:360}
B \Ba
=
\inv{\Ba} \Bn_1 \Bn_2 \cdots \Bn_{k-1} \Ba + \Bm_1 \Bm_2 \cdots \Bm_k \Ba
=
(-1)^{k-1} \Bn_1 \Bn_2 \cdots \Bn_{k-1} + (-1)^k \Ba \Bm_1 \Bm_2 \cdots \Bm_k
=
(-1)^k \lr{ - \Bn_1 \Bn_2 \cdots \Bn_{k-1} + \Ba \Bm_1 \Bm_2 \cdots \Bm_k },
\end{dmath}
or
\begin{dmath}\label{eqn:bladeDotWedgeSymmetryIdentitiesTheorem:380}
(-1)^k B \Ba
=
- \Bn_1 \Bn_2 \cdots \Bn_{k-1} + \Ba \Bm_1 \Bm_2 \cdots \Bm_k.
\end{dmath}

Respective addition and subtraction of \cref{eqn:bladeDotWedgeSymmetryIdentitiesTheorem:340} and \cref{eqn:bladeDotWedgeSymmetryIdentitiesTheorem:380} gives
\begin{dmath}\label{eqn:bladeDotWedgeSymmetryIdentitiesTheorem:400}
\Ba B + (-1)^k B \Ba
= 2 \Ba \Bm_1 \Bm_2 \cdots \Bm_k
= 2 \Ba \wedge B,
\end{dmath}
and
\begin{dmath}\label{eqn:bladeDotWedgeSymmetryIdentitiesTheorem:420}
\Ba B - (-1)^k B \Ba
=
2
\Bn_1 \Bn_2 \cdots \Bn_{k-1}
= 2 \Ba \cdot B,
\end{dmath}
proving \cref{eqn:bladeDotWedgeSymmetryIdentities:220}.  Grade selection from \cref{eqn:bladeDotWedgeSymmetryIdentitiesTheorem:380} gives
\begin{dmath}\label{eqn:bladeDotWedgeSymmetryIdentitiesTheorem:440}
(-1)^k B \cdot \Ba
=
- \Bn_1 \Bn_2 \cdots \Bn_{k-1}
= - \Ba \cdot B,
\end{dmath}
and
\begin{dmath}\label{eqn:bladeDotWedgeSymmetryIdentitiesTheorem:460}
(-1)^k B \wedge \Ba
=
\Ba \Bm_1 \Bm_2 \cdots \Bm_k
= \Ba \wedge B,
\end{dmath}
which proves \cref{eqn:bladeDotWedgeSymmetryIdentities:200}.


            %
% Copyright © 2016 Peeter Joot.  All Rights Reserved.
% Licenced as described in the file LICENSE under the root directory of this GIT repository.
%
\maketheorem{Distribution of inner products}{thm:stokesTheoremGeometricAlgebra:1420}{
Given two blades \(A_s, B_r\) with grades subject to \(s > r > 0\), and a vector \(\Bb\), the inner product distributes according to
\begin{equation*}
A_s \cdot \lr{ \Bb \wedge B_r } = \lr{ A_s \cdot \Bb } \cdot B_r.
\end{equation*}
}

         \subsection{Problems}
            %
% Copyright © 2016 Peeter Joot.  All Rights Reserved.
% Licenced as described in the file LICENSE under the root directory of this GIT repository.
%
\makeproblem{Commutation within grade zero selection}{problem:vectorproduct:cyclicpermutation}{

It was previously shown using coordinates that

\begin{dmath}\label{eqn:vectorproduct:260}
\gpgradezero{ \Bx \By } = \gpgradezero{ \By \Bx }.
\end{dmath}

Repeat this proof using \cref{eqn:vectorproduct:80}.
} % problem

\makeanswer{problem:vectorproduct:cyclicpermutation}{
\begin{dmath}\label{eqn:vectorproduct:301}
\gpgradezero{ \By \Bx }
=
\gpgradezero{ \By \cdot \Bx  + \By \wedge \Bx }
=
\gpgradezero{ \By \cdot \Bx }
=
\gpgradezero{ \Bx \cdot \By }
=
\gpgradezero{ \Bx \By }
\end{dmath}
} % answer

            %
% Copyright © 2016 Peeter Joot.  All Rights Reserved.
% Licenced as described in the file LICENSE under the root directory of this GIT repository.
%

\makeproblem{Vector wedge antisymmetric structure}{problem:vectorproduct:wedgeantisym}{
Prove the wedge product relationship of \cref{eqn:vectorproduct:300}.
} % problem

\makeanswer{problem:vectorproduct:wedgeantisym}{

Rearranging \cref{eqn:vectorproduct:80} for the wedge product and substitution of the dot product symmetric sum from \cref{problem:gradeselection:dotprod} gives

\begin{dmath}\label{eqn:gradeselectionProblems:800}
\Bx \wedge \By
= \Bx \By - \Bx \cdot \By
= \Bx \By - \inv{2} \lr{ \Bx \By + \By \Bx }
= \inv{2} \lr{ 2 \Bx \By - \Bx \By - \By \Bx }
= \inv{2} \lr{ \Bx \By - \By \Bx }.
\end{dmath}
} % answer

            %
% Copyright © 2016 Peeter Joot.  All Rights Reserved.
% Licenced as described in the file LICENSE under the root directory of this GIT repository.
%
\makeproblem{Wedge of three vectors}{problem:gradethreeselectionWedge:wedgeThree}{
Show that
\begin{dmath}\label{eqn:gradethreeselectionWedge:700}
\gpgradethree{ \Ba \Bb \Bc }
=
\Ba \wedge ( \Bb \wedge \Bc )
=
(\Ba \wedge \Bb) \wedge \Bc
=
-
\Bb \wedge (\Ba \wedge \Bc).
\end{dmath}

Observe that is antisymmetric in any two vectors, and thus completely antisymmetric (i.e. associative).  This allows the grade three selection of any three vectors to be written more simply as

\boxedEquation{eqn:gradethreeselectionWedge:720}{
\gpgradethree{ \Ba \Bb \Bc }
=
\Ba \wedge \Bb \wedge \Bc.
}

This can be considered the definition of \( \Ba \wedge \Bb \wedge \Bc \).

Some authors will define the wedge product of \( m \) vectors as

\begin{equation}\label{eqn:gradethreeselectionWedge:820}
\Bx_1 \wedge \Bx_2 \wedge \cdots \wedge \Bx_m
= \inv{m!} \sum \Bx_{i_1} \Bx_{i_2} \cdots \Bx_{i_m} \sgn(\pi(i_1 i_2 \cdots i_m)),
\end{equation}

where \(\sgn(\pi(\cdots))\) is the sign of the permutation of the indices.  With focus on \R{3}, such a definition is not required.  A reader interested in the \R{N} case should demonstrate from the axioms and definitions that
\( \gpgrade{ \Bx_1 \Bx_2 \cdots \Bx_m}{m} \) expands as specified in \cref{eqn:gradethreeselectionWedge:820}.
} % problem

\makeanswer{problem:gradethreeselectionWedge:wedgeThree}{
Consider an expansion first in products of \( \Ba, \Bb \)

\begin{dmath}\label{eqn:gradethreeselectionWedge:740}
\gpgradethree{ \Ba \Bb \Bc }
=
\gpgradethree{ (\cancel{\Ba \cdot \Bb} + \Ba \wedge \Bb) \Bc }
=
\gpgradethree{ (\Ba \wedge \Bb) \Bc }.
\end{dmath}

The dot product was killed since it leaves only a vector product within the grade selection operator.  Since a vector bivector product can have only grade 1 and grade three terms (example: \( \Be_1 (\Be_1 \wedge \Be_2) = \Be_2, \Be_1 (\Be_2 \wedge \Be_3) = \Be_1 \Be_2 \Be_3 \), this leaves just

\begin{dmath}\label{eqn:gradethreeselectionWedge:760}
\gpgradethree{ \Ba \Bb \Bc }
=
(\Ba \wedge \Bb) \wedge \Bc.
\end{dmath}

Similarly, expanding the \( \Bb \Bc \) product gives
\begin{dmath}\label{eqn:gradethreeselectionWedge:780}
\gpgradethree{ \Ba \Bb \Bc }
=
\gpgradethree{ \Ba (\cancel{\Bb \cdot \Bc} + \Ba \wedge \Bc) }
=
\gpgradethree{ \Ba (\Bb \wedge \Bc) }
=
\Ba \wedge (\Bb \wedge \Bc),
\end{dmath}

and finally, expanding products of \( \Ba \Bc \) after commutation

\begin{dmath}\label{eqn:gradethreeselectionWedge:800}
\gpgradethree{ \Ba \Bb \Bc }
=
\gpgradethree{ (\cancel{2 \Bb \cdot \Ba} - \Bb \Ba)
\Bc}
=
-\gpgradethree{ \Bb \Ba \Bc }
=
-\gpgradethree{ \Bb (\cancel{\Ba \cdot \Bc} + \Ba \wedge \Bc) }
=
- \Bb \wedge (\Ba \wedge \Bc).
\end{dmath}
} % answer

            \input{../stokesTheorem/bladeDotWedgeSymmetryIdentities.tex}
            \input{../gabookI/appendix/wedgeDistributionIdentityProblems.tex}
      \section{Problem solutions}
         \shipoutAnswer
   \chapter{Geometry}
      \section{Bivectors}
gabookI: 3.9
      \section{Problems}
      \section{Trivectors}
      \subsection{Problems}
      %\section{Projection and rejection}
      \subsection{Problems}
      \section{Rotations}
gabookI: 2.5 rotations. esp fig 2.1, fig 2.2.
gabookI: 10.4.3 bivector generator of rotations.
gabookI: 29.1
      \subsection{Problems}
      \section{Equivalent identities}
gabookI: 4.1+
      \section{Cramer's rule}
gabookI: 5.  Generalize examples to higher dimensions.
         \subsection{Problems}
      \section{Problem solutions}
         \shipoutAnswer
   \chapter{Vector calculus}
      \section{Reciprocal frames}
         %
% Copyright � 2016 Peeter Joot.  All Rights Reserved.
% Licenced as described in the file LICENSE under the root directory of this GIT repository.
%
%{
%\input{../blogpost.tex}
%\renewcommand{\basename}{reciprocal}
%%\renewcommand{\dirname}{notes/phy1520/}
%\renewcommand{\dirname}{notes/ece1228-electromagnetic-theory/}
%%\newcommand{\dateintitle}{}
%%\newcommand{\keywords}{}
%
%\input{../peeter_prologue_print2.tex}
%
%\usepackage{peeters_layout_exercise}
%\usepackage{peeters_braket}
%\usepackage{peeters_figures}
%\usepackage{siunitx}
%%\usepackage{mhchem} % \ce{}
%%\usepackage{macros_bm} % \bcM
%%\usepackage{macros_qed} % \qedmarker
%%\usepackage{txfonts} % \ointclockwise
%
%\beginArtNoToc
%
%\generatetitle{Reciprocal frame vectors}
%%\chapter{reciprocal frame vectors}
%%\label{chap:reciprocal}
%
The end goal of this chapter is to be able to integrate multivector functions along curves and surfaces, known collectively as manifolds.
For our purposes, a manifold is defined by a parameterization, such as the vector valued function \( \Bx(a,b) \) where \( a, b\) are scalar parameters.  With one parameter the vector traces out a curve, with two a surface, three a volume, and so forth.
The respective partial derivatives of such a parameterized vector define a local basis for the surface at the point at which the partials are evaluated.
The span of such a basis is called the tangent space, and the partials that constitute it are not necessarily orthonormal, or even normal.

Unfortunately, in order to work with the curvilinear non-orthonormal bases that will be encountered in general integration theory, some
additional tools are required.

\begin{itemize}
\item
We introduce a reciprocal frame which partially generalizes the notion of normal to non-orthonormal bases.
\item
We will borrow the upper and lower index (tensor) notation from relativistic physics that is useful for the intrinsically non-orthonormal spaces encountered in that study, as this notation works well to define the reciprocal frame.
\end{itemize}

\index{reciprocal frame}
\makedefinition{Reciprocal frame}{dfn:reciprocal:frame}{
Given a basis for a subspace \( \setlr{ \Bx_1, \Bx_2, \cdots \Bx_n } \), where the vectors \( \Bx_i \) are not necessarily orthonormal, the reciprocal frame is defined as the set of vectors \( \setlr{ \Bx^1, \Bx^2, \cdots \Bx^n } \) satisfying

\begin{dmath*}
\Bx_i \cdot \Bx^j = {\delta_i}^j,
\end{dmath*}

where the vector \( \Bx^j \) is not the j-th power of \( \Bx \), but is a superscript index, the conventional way of denoting a reciprocal frame vector, and \( {\delta_i}^j \) is the Kronecker delta.
} % definition

This definition introduces mixed index variables for the first time in this text, which may be unfamiliar.  These are most often used in tensor algebra, where any expression that has pairs of upper and lower indexes implies a sum, and is called the summation (or Einstein) convention.  For example:

\begin{dmath}\label{eqn:reciprocal:400}
\begin{aligned}
a_i b^i &\equiv \sum_i a_i b^i \\
{A^{i}}_j B_i C^j &\equiv \sum_{i,j} {A^{i}}_j B_i C^j.
\end{aligned}
\end{dmath}

Summation convention will not be used explicitly in this text, as it deviates from normal practises in electrical engineering\footnote{Generally, when summation convention is used, explicit summation is only used explicitly when upper and lower indexes are not perfectly matched, but summation is still implied.  Readers of texts that use summation convention can check for proper matching of upper and lower indexes to ensure that the expressions make sense.  Such matching is the reason a mixed index Kronecker delta has been used in the definition of the reciprocal frame.}.

The most important application of a reciprocal frame is for the computation of the coordinates of a vector with respect to a non-orthonormal frame.
Let a vector \( \Ba \) have coordinates \( a^i \) with respect to a basis \( \setlr{ \Bx_i } \)

\begin{dmath}\label{eqn:reciprocal:20}
\Ba = \sum_j a^j \Bx_j,
\end{dmath}

where \( j \) is an index not a power\footnote{In tensor algebra, any index that is found in matched upper and lower index pairs, is known as a dummy summation index, whereas an index that is unmatched is known as a free index.  For example, in \( a^j b_{ij} \) (summation implied) \( j \) is a summation index, and \( i \) is a free index.  We are free to make a change of variables of any summation index, so for the same example we can write
\( a^k b_{ik} \).  These index tracking conventions are obvious when summation symbols are included explicitly, as we will do.}.

Dotting with the reciprocal frame vectors \( \Bx^i \) provides these coordinates \( a^i \)

\begin{dmath}\label{eqn:reciprocal:40}
\Ba \cdot \Bx^i
= \lr{\sum_j a^j \Bx_j} \cdot \Bx^i
= \sum_j a^j {\delta_j}^i
= a^i.
\end{dmath}

The vector can also be expressed with coordinates taken with respect to the reciprocal frame.  Let those coordinates be \( a_i \), so that

\begin{dmath}\label{eqn:reciprocal:60}
\Ba = \sum_i a_i \Bx^i.
\end{dmath}

Dotting with the basis vectors \( \Bx_i \) provides the reciprocal frame relative coordinates \( a_i \)

\begin{dmath}\label{eqn:reciprocal:80}
\Ba \cdot \Bx_i
= \lr{\sum_j a_j \Bx^j} \cdot \Bx_i
= \sum_j a_j {\delta^j}_i
= a_i.
\end{dmath}

We can summarize \cref{eqn:reciprocal:40} and \cref{eqn:reciprocal:80} by stating that a vector can be expressed in terms of coordinates relative to either the original or reciprocal basis as follows

\begin{equation}\label{eqn:reciprocal:420}
\Ba
= \sum_j \lr{ \Ba \cdot \Bx^j } \Bx_j
= \sum_j \lr{ \Ba \cdot \Bx_j } \Bx^j.
\end{equation}

In tensor algebra the basis is generally implied\footnote{
In tensor algebra, a vector, identified by the coordinates \( a^i \) is called a contravariant vector.
When that vector is identified by the coordinates \( a_i \) it is called a covariant vector.  These labels relate to how the coordinates transform with respect to norm preserving transformations.
We have no need of this nomenclature, since we never transform coordinates in isolation, but will always transform the coordinates along with their associated basis vectors.}.

%When doing tensor algebra manipulations, you'll generally have the freedom to swap any pairs of upper and lower indexes as done above.

An example of a 2D oblique Euclidean basis and a corresponding reciprocal basis is plotted in \cref{fig:obliqueReciprocal:obliqueReciprocalFig2}.
Also plotted are the superposition of the projections required to arrive at a given point \( (4,2) \)) along the \( \Be_1, \Be_2 \) directions or the \( \Be^1, \Be^2 \) directions.
In this plot, neither of the reciprocal frame vectors \( \Be^i \) are normal to the corresponding basis vectors \( \Be_i \).
When one of \( \Be_i \) is increased(decreased) in magnitude, there will be a corresponding decrease(increase) in the magnitude of \( \Be^i \), but if the orientation is remained fixed, the corresponding direction of the reciprocal frame vector stays the same.

\imageFigure{../figures/GAelectrodynamics/obliqueReciprocalFig2}{Oblique and reciprocal bases.}{fig:obliqueReciprocal:obliqueReciprocalFig2}{0.5}

How are the reciprocal frame vectors computed?  While these vectors have a natural GA representation, this is not intrinsically a GA problem, and can be solved with standard linear algebra, using a matrix inversion.
For example, given a 2D basis \( \setlr{ \Bx_1, \Bx_2 } \), the reciprocal basis can be assumed to have a coordinate representation in the original basis

\begin{dmath}\label{eqn:reciprocal:100}
\begin{aligned}
\Bx^1 &= a \Bx_1 + b \Bx_2 \\
\Bx^2 &= c \Bx_1 + d \Bx_2.
\end{aligned}
\end{dmath}

Imposing the constraints of \cref{dfn:reciprocal:frame} leads to a pair of 2x2 linear systems that are easily solved to find
\begin{dmath}\label{eqn:reciprocal:120}
\begin{aligned}
\Bx^1 &= \inv{ (\Bx_1)^2 (\Bx_2)^2 - \lr{ \Bx_1 \cdot \Bx_2}^2 } \lr{ (\Bx_2)^2 \Bx_1 - \lr{ \Bx_1 \cdot \Bx_2 } \Bx_2 } \\
\Bx^2 &= \inv{ (\Bx_1)^2 (\Bx_2)^2 - \lr{ \Bx_1 \cdot \Bx_2}^2 } \lr{ (\Bx_1)^2 \Bx_2 - \lr{ \Bx_1 \cdot \Bx_2 } \Bx_1 } \\
\end{aligned}
\end{dmath}

The reader may notice that for \R{3} the denominator is related to the norm of the cross product \( \Bx_1 \cross \Bx_2 \).
More generally, this can be expressed as the square of the bivector \( \Bx_1 \wedge \Bx_2 \)

\begin{dmath}\label{eqn:reciprocal:140}
-\lr{\Bx_1 \wedge \Bx_2 }^2
=
-\lr{\Bx_1 \wedge \Bx_2 } \cdot \lr{\Bx_1 \wedge \Bx_2 }
=
-\lr{ \lr{\Bx_1 \wedge \Bx_2 } \cdot \Bx_1 } \cdot \Bx_2
=
(\Bx_1)^2 (\Bx_2)^2 - \lr{\Bx_1 \cdot \Bx_2}^2.
\end{dmath}

Additionally, the numerators are each dot products of \( \Bx_1, \Bx_2 \) with that same bivector

\begin{dmath}\label{eqn:reciprocal:160}
\begin{aligned}
\Bx^1 &= \frac{\Bx_2 \cdot \lr{ \Bx_1 \wedge \Bx_2 } }{ \lr{ \Bx_1 \wedge \Bx_2}^2 } \\
\Bx^2 &= \frac{\Bx_1 \cdot \lr{ \Bx_2 \wedge \Bx_1 } }{ \lr{ \Bx_1 \wedge \Bx_2}^2 },
\end{aligned}
\end{dmath}

or

%\begin{dmath}\label{eqn:reciprocal:180}
\boxedEquation{eqn:reciprocal:180}{
\begin{aligned}
\Bx^1 &= \Bx_2 \cdot \inv{ \Bx_1 \wedge \Bx_2 } \\
\Bx^2 &= \Bx_1 \cdot \inv{ \Bx_2 \wedge \Bx_1 }.
\end{aligned}
}
%\end{dmath}

Geometrically, dotting with the bivector of the plane is a hybrid rotation and scaling operation.
For example, for \R{2} with \( \Bx_1 = a_1 \Be_1 + a_2 \Be_2, \Bx_2 = b_1 \Be_1 + b_2 \Be_2 \), that pseudoscalar for this basis is

\begin{dmath}\label{eqn:reciprocal:260}
\Bx_1 \wedge \Bx_2
=
\lr{ a_1 \Be_1 + a_2 \Be_2 } \wedge \lr{ b_1 \Be_1 + b_2 \Be_2 }
=
\lr{ a_1 b_2 - a_2 b_1 } \Be_{12}.
\end{dmath}

This has inverse
\begin{dmath}\label{eqn:reciprocal:280}
\inv{\Bx_1 \wedge \Bx_2 }
=
\inv{ a_1 b_2 - a_2 b_1 } \Be_{21}.
\end{dmath}

So for the \R{2} the reciprocal frame is just

\begin{dmath}\label{eqn:reciprocal:300}
\begin{aligned}
\Bx^1 &= \Bx_2 \frac{\Be_{21}}{ a_1 b_2 - a_2 b_1 } \\
\Bx^2 &= \Bx_1 \frac{\Be_{12}}{ a_1 b_2 - a_2 b_1 }
\end{aligned}
\end{dmath}

The vector \( \Bx^1 \) is obtained by rotating \( \Bx_2 \) by \( -\pi/2 \), and rescaling it.
The vector \( \Bx^2 \) is similarly obtained by a scaling and a rotation of \( \Bx_1 \) by \( \pi/2 \).

Generalizing \cref{eqn:reciprocal:180} is almost possible by inspection.
Given
a subspace spanned by a three vector basis \( \setlr{ \Bx_1, \Bx_2, \Bx_3 } \) the reciprocal frame vectors can be written as dot products

\begin{dmath}\label{eqn:reciprocal:320}
\begin{aligned}
\Bx^1 &= \lr{ \Bx_2 \wedge \Bx_3 } \cdot \lr{ \Bx^3 \wedge \Bx^2 \wedge \Bx^1 } \\
\Bx^2 &= \lr{ \Bx_3 \wedge \Bx_1 } \cdot \lr{ \Bx^1 \wedge \Bx^3 \wedge \Bx^2 } \\
\Bx^3 &= \lr{ \Bx_1 \wedge \Bx_2 } \cdot \lr{ \Bx^2 \wedge \Bx^1 \wedge \Bx^3 } \\
\end{aligned}
\end{dmath}

Each of those trivector terms equals \( - \Bx^1 \wedge \Bx^2 \wedge \Bx^3 \) and can be related to the (known) pseudoscalar \( \Bx_1 \wedge \Bx_2 \wedge \Bx_3 \) by observing that

\begin{dmath}\label{eqn:reciprocal:340}
\lr{ \Bx^1 \wedge \Bx^2 \wedge \Bx^3 } \cdot \lr{ \Bx_3 \wedge \Bx_2 \wedge \Bx_1 }
=
\Bx^1 \cdot \lr{ \Bx^2 \cdot \lr{ \Bx^3 \cdot \lr{ \Bx_3 \wedge \Bx_2 \wedge \Bx_1 } }}
=
\Bx^1 \cdot \lr{ \Bx^2 \cdot \lr{ \Bx_2 \wedge \Bx_1 } }
=
\Bx^1 \cdot \Bx_1
=
1,
\end{dmath}

which means that

\begin{dmath}\label{eqn:reciprocal:360}
-\Bx^1 \wedge \Bx^2 \wedge \Bx^3
= -\inv{ \Bx_3 \wedge \Bx_2 \wedge \Bx_1 }
= \inv{ \Bx_1 \wedge \Bx_2 \wedge \Bx_3 },
\end{dmath}

and

\boxedEquation{eqn:reciprocal:380}{
\begin{aligned}
\Bx^1 &= \lr{ \Bx_2 \wedge \Bx_3 } \cdot \inv{ \Bx_1 \wedge \Bx_2 \wedge \Bx_3 } \\
\Bx^2 &= \lr{ \Bx_3 \wedge \Bx_1 } \cdot \inv{ \Bx_1 \wedge \Bx_2 \wedge \Bx_3 } \\
\Bx^3 &= \lr{ \Bx_1 \wedge \Bx_2 } \cdot \inv{ \Bx_1 \wedge \Bx_2 \wedge \Bx_3 }
\end{aligned}
}

Geometrically, this trivector division is a duality transformation within the subspace spanned by the three vectors \( \Bx_1, \Bx_2, \Bx_3 \), also scaling the result so that the \( \Bx_i \cdot \Bx^j = {\delta_i}^j \) condition is satisfied.

It should be clear how to generalize the reciprocal basis calculation formulas of
\cref{eqn:reciprocal:180} and \cref{eqn:reciprocal:380} to higher dimensions if desired.
%}
%\EndNoBibArticle

         \subsection{Problems}
            %
% Copyright © 2016 Peeter Joot.  All Rights Reserved.
% Licenced as described in the file LICENSE under the root directory of this GIT repository.
%
\makeproblem{Reciprocal frame for two dimensional subspace.}{problem:reciprocal:2dsubspaceRecip}{
Prove \cref{eqn:reciprocal:120}.
} % problem

\makeanswer{problem:reciprocal:2dsubspaceRecip}{

Assuming the representation of \cref{eqn:reciprocal:100}, the dot products are

\begin{dmath}\label{eqn:2dreciprocalMatrixCalculation:200}
\begin{aligned}
1 &= \Bx_1 \cdot \Bx^1 = a \Bx_1^2 + b \Bx_1 \cdot \Bx_2 \\
0 &= \Bx_2 \cdot \Bx^1 = a \Bx_2 \cdot \Bx_1 + b \Bx_2^2 \\
0 &= \Bx_1 \cdot \Bx^2 = c \Bx_1^2 + d \Bx_1 \cdot \Bx_2 \\
1 &= \Bx_2 \cdot \Bx^2 = c \Bx_2 \cdot \Bx_1 + d \Bx_2^2
\end{aligned}
\end{dmath}

This can be written out as a pair of matrix equations

\begin{dmath}\label{eqn:2dreciprocalMatrixCalculation:220}
\begin{aligned}
\begin{bmatrix}
1 \\
0
\end{bmatrix}
&=
\begin{bmatrix}
\Bx_1^2 & \Bx_1 \cdot \Bx_2 \\
\Bx_2 \cdot \Bx_1 & \Bx_2^2 \\
\end{bmatrix}
\begin{bmatrix}
a \\
b
\end{bmatrix} \\
\begin{bmatrix}
0 \\
1
\end{bmatrix}
&=
\begin{bmatrix}
\Bx_1^2 & \Bx_1 \cdot \Bx_2 \\
\Bx_2 \cdot \Bx_1 & \Bx_2^2 \\
\end{bmatrix}
\begin{bmatrix}
c \\
d
\end{bmatrix}.
\end{aligned}
\end{dmath}

The matrix inverse is
\begin{dmath}\label{eqn:2dreciprocalMatrixCalculation:240}
{
\begin{bmatrix}
\Bx_1^2 & \Bx_1 \cdot \Bx_2 \\
\Bx_2 \cdot \Bx_1 & \Bx_2^2 \\
\end{bmatrix}
}^{-1}
=
\inv{ \Bx_1^2 \Bx_2^2 - \lr{\Bx_1 \cdot \Bx_2}^2 }
\begin{bmatrix}
\Bx_2^2 & -\Bx_1 \cdot \Bx_2 \\
-\Bx_2 \cdot \Bx_1 & \Bx_1^2 \\
\end{bmatrix}
\end{dmath}

Multiplying by the \( (1,0) \), and \( (0,1) \) vectors picks out the respective columns, and gives \cref{eqn:reciprocal:120}.
} % answer

            %
% Copyright © 2016 Peeter Joot.  All Rights Reserved.
% Licenced as described in the file LICENSE under the root directory of this GIT repository.
%

\index{reciprocal frame}
\makeproblem{Two vector reciprocal frame}{problem:2subspaceR3reciprocalExample:2subspaceR3reciprocalExample}{
Calculate the reciprocal frame for the \R{3} subspace spanned by \( \setlr{ \Bx_1, \Bx_2 } \) where

\begin{dmath}\label{eqn:2subspaceR3reciprocalExample:20}
\begin{aligned}
\Bx_1 &= \Be_1 + 2 \Be_2 \\
\Bx_2 &= \Be_2 - \Be_3.
\end{aligned}
\end{dmath}
} % problem

\makeanswer{problem:2subspaceR3reciprocalExample:2subspaceR3reciprocalExample}{
The bivector for the plane spanned by this basis is

\begin{dmath}\label{eqn:2subspaceR3reciprocalExample:40}
\Bx_1 \wedge \Bx_2
=
\lr{ \Be_1 + 2 \Be_2 } \wedge
\lr{ \Be_2 - \Be_3 }
=
\Be_{12} - \Be_{13} - 2 \Be_{23}
=
\Be_{12} + \Be_{31} + 2 \Be_{32}.
\end{dmath}

This has the square
\begin{dmath}\label{eqn:2subspaceR3reciprocalExample:60}
\lr{ \Bx_1 \wedge \Bx_2 }^2
=
\lr{ \Be_{12} + \Be_{31} + 2 \Be_{32} }
\cdot
\lr{ \Be_{12} + \Be_{31} + 2 \Be_{32} }
=
-1 -1 -4
=
-6.
\end{dmath}

Dotting \( -\Bx_1 \) with the bivector is
\begin{dmath}\label{eqn:2subspaceR3reciprocalExample:80}
\Bx_1 \cdot \lr{ \Bx_2 \wedge \Bx_1 }
=
-\lr{ \Be_1 + 2 \Be_2 } \cdot \lr{\Be_{12} + \Be_{31} + 2 \Be_{32} }
=
-\lr{ \Be_2 - \Be_3 - 2 \Be_1 - 4 \Be_3 }
= 2 \Be_1 - \Be_2 + 5 \Be_3.
\end{dmath}

For \( \Bx_2 \) the dot product with the bivector is

\begin{dmath}\label{eqn:2subspaceR3reciprocalExample:100}
\Bx_2 \cdot \lr{ \Bx_1 \wedge \Bx_2 }
=
\lr{ \Be_2 - \Be_3 } \cdot \lr{\Be_{12} + \Be_{31} + 2 \Be_{32} }
=
- \Be_1 - 2 \Be_3 - \Be_1 - 2 \Be_2
=
- 2 \Be_1 - 2 \Be_2 - 2 \Be_3,
\end{dmath}
so
\begin{dmath}\label{eqn:2subspaceR3reciprocalExample:120}
\begin{aligned}
\Bx^1 &= \inv{3} \lr{ \Be_1 + \Be_2 + \Be_3 } \\
\Bx^2 &= \inv{6} \lr{ -2 \Be_1 + \Be_2 - 5 \Be_3 }.
\end{aligned}
\end{dmath}
It is easy to verify that this has the desired semantics.
} % answer

      \section{Curvilinear coordinates}
gabook: 31.1
      \section{Green's theorem}
         %
% Copyright © 2013 Peeter Joot.  All Rights Reserved.
% Licenced as described in the file LICENSE under the root directory of this GIT repository.
%
Given a two parameter (\(u,v\)) surface parameterization, the curvilinear coordinate representation of a vector \(\Bf\) has the form

\begin{dmath}\label{eqn:stokesTheoremGeometricAlgebra:1640}
\Bf = f_u \Bx^u + f_v \Bx^v + f_\perp \Bx^\perp.
\end{dmath}

We assume that the vector space is of dimension two or greater but otherwise unrestricted, and need not have an Euclidean basis.
Here \(f_\perp \Bx^\perp\) denotes the rejection of \(\Bf\) from the tangent space at the point of evaluation.
Green's theorem relates the integral around a closed curve to an ``area'' integral on that surface

\maketheorem{Green's Theorem}{thm:stokesTheoremGeometricAlgebra:1660}{
\index{Green's theorem}
\begin{equation*}
\ointctrclockwise \Bf \cdot d\Bl
=
\iint \lr{
-\PD{v}{f_u}
+\PD{u}{f_v}
}
du dv
\end{equation*}
}

Following the arguments used in \citep{schwartz1987pe} for Stokes' theorem in three dimensions, we first evaluate the loop integral along the differential element of the surface at the point \(\Bx(u_0, v_0)\) evaluated over the range \((du, dv)\), as shown in the infinitesimal loop of \cref{fig:loopIntegralInfinitesimal:loopIntegralInfinitesimalFig1}.

\imageFigure{../figures/gabook/loopIntegralInfinitesimalFig1}{Infinitesimal loop integral.}{fig:loopIntegralInfinitesimal:loopIntegralInfinitesimalFig1}{0.35}

Over the infinitesimal area, the loop integral decomposes into

\begin{dmath}\label{eqn:stokesTheoremGeometricAlgebra:1700}
\ointctrclockwise \Bf \cdot d\Bl
=
\int \Bf \cdot d\Bx_1
+\int \Bf \cdot d\Bx_2
+\int \Bf \cdot d\Bx_3
+\int \Bf \cdot d\Bx_4,
\end{dmath}

where the differentials along the curve are

\begin{dmath}\label{eqn:stokesTheoremGeometricAlgebra:1600}
\begin{aligned}
d\Bx_1 &= \evalbar{ \PD{u}{\Bx} }{v = v_0} du \\
d\Bx_2 &= \evalbar{ \PD{v}{\Bx} }{u = u_0 + du} dv \\
d\Bx_3 &= -\evalbar{ \PD{u}{\Bx} }{v = v_0 + dv} du \\
d\Bx_4 &= -\evalbar{ \PD{v}{\Bx} }{u = u_0} dv.
\end{aligned}
\end{dmath}

It is assumed that the parameterization change \((du, dv)\) is small enough that this loop integral can be considered planar (regardless of the dimension of the vector space).
Making use of the fact that \(\Bx^\perp \cdot \Bx_\alpha = 0\) for \(\alpha \in \setlr{u,v}\), the loop integral is

\begin{dmath}\label{eqn:stokesTheoremGeometricAlgebra:1620}
\ointctrclockwise \Bf \cdot d\Bl
=
\int
\lr{
f_u \Bx^u + f_v \Bx^v + f_\perp \Bx^\perp
}
\cdot
\Bigl(
\Bx_u(u, v_0) du - \Bx_u(u, v_0 + dv) du
+\Bx_v(u_0 + du, v) dv - \Bx_v(u_0, v) dv
\Bigr)
=
\int
f_u(u, v_0) du - f_u(u, v_0 + dv) du
+
f_v(u_0 + du, v) dv - f_v(u_0, v) dv
\end{dmath}

With the distances being infinitesimal, these differences can be rewritten as partial differentials

\begin{dmath}\label{eqn:stokesTheoremGeometricAlgebra:1860}
\ointctrclockwise \Bf \cdot d\Bl
=
\iint \lr{
-\PD{v}{f_u}
+\PD{u}{f_v}
}
du dv.
\end{dmath}

We can now sum over a larger area as in \cref{fig:loopIntegralInfinitesimalSum:loopIntegralInfinitesimalSumFig2}

\imageFigure{../figures/gabook/loopIntegralInfinitesimalSumFig2}{Sum of infinitesimal loops.}{fig:loopIntegralInfinitesimalSum:loopIntegralInfinitesimalSumFig2}{0.35}

All the opposing oriented loop elements cancel, so the integral around the complete boundary of the surface \(\Bx(u, v)\) is given by the \(u,v\) area integral of the partials difference.

We will see that Green's theorem is a special case of the Stokes' theorem.
This observation will also provide a geometric interpretation of the right hand side area integral of \cref{thm:stokesTheoremGeometricAlgebra:1660}, and allow for a coordinate free representation.

\paragraph{Special case:}

An important special case of Green's theorem is for a Euclidean two dimensional space where the vector function is

\begin{dmath}\label{eqn:stokesTheoremGeometricAlgebra:1720}
\Bf = P \Be_1 + Q \Be_2.
\end{dmath}

Here Green's theorem takes the form

\boxedEquation{eqn:stokesTheoremGeometricAlgebra:1710}{
\ointctrclockwise P dx + Q dy
=
\iint \lr{
\PD{x}{Q}
-\PD{y}{P}
}
dx dy.
}

         %\subsection{Problems}
      \section{Stokes' theorem}
         \subsection{Statement}
            %
% Copyright © 2016 Peeter Joot.  All Rights Reserved.
% Licenced as described in the file LICENSE under the root directory of this GIT repository.
%
\index{Stokes' theorem}
Stokes' theorem is fairly easy to state, but takes a fair amount of work to understand and unpack its implications.

%
% Copyright © 2013 Peeter Joot.  All Rights Reserved.
% Licenced as described in the file LICENSE under the root directory of this GIT repository.
%
\maketheorem{Stokes' Theorem}{thm:stokesTheoremGeometricAlgebra:1740}{

For blades \(F \in \bigwedge^{s}\), and \(m\) volume element \(d^k \Bx, s < k\),

\begin{equation*}%\label{eqn:stokesTheoremTheStatement:120}
\int_V d^k \Bx \cdot (\boldpartial \wedge F) = \int_{\partial V} d^{k-1} \Bx \cdot F.
\end{equation*}

Here the volume integral is over a \(m\) dimensional surface (manifold).  The derivative operator \(\boldpartial\) is called the vector derviative and is the projection of the gradient onto the tangent space of the manifold.  Integration over the boundary of \(V\) is indicated by \( \partial V \).
}

The vector derivative is defined by

\begin{equation}\label{eqn:stokesTheoremTheStatement:1400}
\boldpartial = \Bx^i \partial_i = \sum_i \Bx_i \PD{u^i}{}.
\end{equation}

where \( \Bx^i \) are reciprocal frame vectors dual to the tangent vector basis \( \Bx_i \) associated with the parameters \( u^1, u^2, \cdots \).
%These will be defined in more detail in the next section.
Once the volume element, vector product and the other concepts are defined, the proof of
Stokes theorem is really just a statement that

\boxedEquation{eqn:stokesTheoremGeometricAlgebra:2840}{
\int_V d^k \Bx \cdot (\Bx^i \partial_i \wedge F) =
\int_V \lr{ d^k \Bx \cdot \Bx^i } \cdot \partial_i F.
}

This dot product expansion applies to any degree blade and volume element provided the degree of the blade is less than that of the volume element (i.e. \(s < k\)).  That magic follows directly from \cref{thm:stokesTheoremGeometricAlgebra:1420}.


\index{oriented volume element}
This dot product defines the oriented surface ``area'' elements associated with the ``volume'' element \( d^k \Bx \).
That area element can be obtained from the mnemonic

\begin{dmath}\label{eqn:statement:1561}
\sum_i d^k \Bx \cdot \Bx^i,
\end{dmath}

with each of the i-th differentials evaluated.
This will be made clear by example.


         \subsection{One parameter specialization of Stokes' theorem.}
            %
% Copyright © 2016 Peeter Joot.  All Rights Reserved.
% Licenced as described in the file LICENSE under the root directory of this GIT repository.
%

An example parameterization with one parameter, and the corresponding differential with respect to that parameter, is plotted in
\cref{fig:oneParameterDifferential:oneParameterDifferentialFig1}, for a parameterization over \( [a, b] \in [0,1]\otimes[0,1] \).

\imageFigure{../figures/GAelectrodynamics/oneParameterDifferentialFig1}{One parameter manifold.}{fig:oneParameterDifferential:oneParameterDifferentialFig1}{0.3}

The differential with respect to the parameter \( a \) is

\begin{equation}\label{eqn:stokesTheoremCore:20}
d\Bx_a = \PD{a}{\Bx} da = \Bx_a da.
\end{equation}

On this curve the projection of the gradient has just one component

\begin{dmath}\label{eqn:stokesTheoremCore:40}
\boldpartial
=
\sum_\mu \Bx^\mu (\Bx_\mu \cdot \spacegrad)
=
\Bx^a \PD{a}{}
\equiv
\Bx^a \partial_a.
\end{dmath}

Please see \citep{aMacdonaldVAGC} for a full justification of the curvilinear coordinate representation of the vector derivative (or the gradient).
That text also discusses pertinent issues with the connectivity of the manifold ignored here.

Stokes' theorem for a one parameter manifold can only be expressed for scalar fields.
That is

\begin{dmath}\label{eqn:stokesTheoremCore:60}
\int d\Bx \cdot (\boldpartial \wedge \psi)
=
\int d\Bx \cdot \boldpartial \psi
=
\int da \PD{a}{ \psi }
= \evalbar{\psi}{\Delta a}.
\end{dmath}

Observe that the vector derivative can be replaced by the gradient since \( d\Bx \cdot \boldpartial = d\Bx \cdot \spacegrad \).
This is the case since dotting the
gradient with a differential element \( d\Bx \) on this curve, no component of the gradient that isn't colinear to the curve makes no contribution.

That means that Stokes' theorem for a one parameter curve is exactly the fundamental theorem of calculus for line integrals

%\begin{dmath}\label{eqn:stokesTheoremCore:80}
\boxedEquation{eqn:stokesTheoremCore:80}{
\int_{\Ba}^{\Bb} d\Bx \cdot \spacegrad \psi = \psi(\Bb) - \psi(\Ba).
}
%\end{dmath}

         \subsection{Two parameter specialization of Stokes' theorem.}
            %
% Copyright © 2016 Peeter Joot.  All Rights Reserved.
% Licenced as described in the file LICENSE under the root directory of this GIT repository.
%

An example parameterization with two parameters, and the corresponding differentials with respect to those parameters, is plotted in
\cref{fig:twoParameterDifferential:twoParameterDifferentialFig1}.

\imageFigure{../figures/GAelectrodynamics/twoParameterDifferentialFig1}{Two parameter manifold differentials.}{fig:twoParameterDifferential:twoParameterDifferentialFig1}{0.4}

Given parameters \( a, b \), the differentials along each of the parameterization directions are

\begin{dmath}\label{eqn:stokesTheoremCore:100}
\begin{aligned}
d\Bx_a &= \PD{a}{\Bx} da = \Bx_a da \\
d\Bx_b &= \PD{b}{\Bx} db = \Bx_b db.
\end{aligned}
\end{dmath}

The ``volume'' element for this parameterization (a surface area element) is

\begin{equation}\label{eqn:stokesTheoremCore:120}
d^2 \Bx
=
d\Bx_a \wedge
d\Bx_b
=
da db (\Bx_a \wedge \Bx_b).
\end{equation}

The vector derivative, the projection of the gradient onto the surface at the point of integration (also called the tangent space), now has two components

\begin{dmath}\label{eqn:stokesTheoremCore:200}
\boldpartial
=
\sum_\mu \Bx^\mu (\Bx_\mu \cdot \spacegrad)
=
\Bx^a \PD{a}{}
+
\Bx^b \PD{b}{}
\equiv
\Bx^a \partial_a
+
\Bx^b \partial_b.
\end{dmath}

The Stokes integral can be evaluated over this volume element for either scalar fields \( \psi \) or vector fields \( \Bf \), and takes the form

\begin{subequations}
\label{eqn:stokesTheoremCore:140}
\begin{equation}\label{eqn:stokesTheoremCore:160}
\int_A d^2 \Bx \cdot (\boldpartial \wedge \psi) =
\int_A (d^2 \Bx \cdot \boldpartial) \psi
=
\int_{\partial A} d^1 \Bx \psi
\end{equation}
\begin{equation}\label{eqn:stokesTheoremCore:180}
\int_A d^2 \Bx \cdot (\boldpartial \wedge \Bf) =
\int_A (d^2 \Bx \cdot \boldpartial) \cdot \Bf
=
\int_{\partial A} d^1 \Bx \cdot \Bf.
\end{equation}
\end{subequations}

To extract the full meaning of this the boundary differential \( d^1 \Bx \) must be computed.  This has the same structure for a vector or scalar field

\begin{dmath}\label{eqn:stokesTheoremCore:220}
\begin{aligned}
\int_A d^2 \Bx \cdot (\boldpartial \wedge \Bf)
&=
\int_A (d^2 \Bx \cdot \boldpartial) \cdot \Bf \\
&=
\sum_\mu \int_A (d^2 \Bx \cdot \Bx^\mu) \cdot \partial_\mu \Bf \\
&=
\sum_\mu \int_A da db  \lr{ \Bx_a \wedge \Bx_b ) \cdot \Bx^\mu } \cdot \partial_\mu \Bf \\
&=
\sum_\mu \int_A da db  \lr{ \Bx_a {\delta_b}^\mu - \Bx_b {\delta_a}^\mu } \cdot \partial_\mu \Bf \\
&=
\int_A da db  \lr{ \Bx_a \cdot \PD{b}{ \Bf} - \Bx_b \cdot \PD{a}{\Bf} }
\end{aligned}
\end{dmath}

While \( \Bx_a, \Bx_b \) both depend on both parameters \( a, b \), the differential form immediately above is still a perfect integral in the variables of the partials since \( \Bx_a \) is computed with \( b \) held fixed, and \( \Bx_b \) is computed with \( a \) held fixed.  Proceeding with the integrals that match the respective partials, this gives

\begin{dmath}\label{eqn:stokesTheoremCore:240}
\int_A d^2 \Bx \cdot (\boldpartial \wedge \Bf)
=
\int
da \Bx_a \cdot \evalbar{\Bf}{\Delta b}
-\int
db \Bx_b \cdot \evalbar{\Bf}{\Delta a}
=
\int
d\Bx_a \cdot \evalbar{\Bf}{\Delta b}
-\int
d\Bx_b \cdot \evalbar{\Bf}{\Delta a}.
\end{dmath}

This shows that the boundary differential \( d^1 \Bx \) in \cref{eqn:stokesTheoremCore:140} is given by

\begin{dmath}\label{eqn:stokesTheoremCore:260}
d^1 \Bx = d\Bx_a - d\Bx_b,
\end{dmath}

where it is implied that the field in question is evaluated at the boundaries of the parameter that has been eliminated by this first integration.  This boundary integral can be interpreted as the integral around a contour, as indicated in
\cref{fig:twoParameterDifferentialBoundary:twoParameterDifferentialBoundaryFig2}.

\imageFigure{../figures/GAelectrodynamics/twoParameterDifferentialBoundaryFig2}{Contour for two parameter surface boundary.}{fig:twoParameterDifferentialBoundary:twoParameterDifferentialBoundaryFig2}{0.4}

Additionally, as with the single parameter case, a substitution of the gradient does not change the result, since any component of the gradient that lies outside of the tangent space on the surface has a zero dot product with the surface volume element \( d^2 \Bx \).
This allows the two parameter Stokes integrals to be written as

%\begin{dmath}\label{eqn:stokesTheoremCore:280}
\boxedEquation{eqn:stokesTheoremCore:280}{
\begin{aligned}
\int_A d^2 \Bx \cdot \spacegrad \psi &= \ointclockwise d\Bx \psi \\
\int_A d^2 \Bx \cdot (\spacegrad \wedge \Bf) &= \ointclockwise d\Bx \cdot \Bf.
\end{aligned}
}
%\end{dmath}

It can be shown that this two parameter Stokes integral is equivalent to Green's theorem.

         \subsection{Three parameter specialization of Stokes' theorem.}
            %
% Copyright © 2016 Peeter Joot.  All Rights Reserved.
% Licenced as described in the file LICENSE under the root directory of this GIT repository.
%

An example parameterization with three parameters, and the corresponding differentials with respect to those parameters, and the outwards normals, are sketched in
\cref{fig:normalsOnVolumeAreaElement:normalsOnVolumeAreaElementFig11}.

\imageFigure{../figures/gabook/normalsOnVolumeAreaElementFig11}{Three parameter volume element.}{fig:normalsOnVolumeAreaElement:normalsOnVolumeAreaElementFig11}{0.4}

Given parameters \( a, b, c \), the differentials along each of the parameterization directions are

\begin{dmath}\label{eqn:stokesTheoremCore:1421}
\begin{aligned}
d\Bx_a &= \PD{a}{\Bx} da = \Bx_a da \\
d\Bx_b &= \PD{b}{\Bx} db = \Bx_b db \\
d\Bx_c &= \PD{c}{\Bx} dc = \Bx_c dc.
\end{aligned}
\end{dmath}

The ``volume'' element for this parameterization (a surface area element) is

\begin{equation}\label{eqn:stokesTheoremCore:1441}
d^3 \Bx
=
d\Bx_a
\wedge
d\Bx_b
\wedge
d\Bx_c
=
da db dc (\Bx_a \wedge \Bx_b \wedge \Bx_c).
\end{equation}

The vector derivative, the projection of the gradient onto the surface at the point of integration (also called the tangent space), now has three components

\begin{dmath}\label{eqn:stokesTheoremCore:1461}
\boldpartial
=
\sum_\mu \Bx^\mu (\Bx_\mu \cdot \spacegrad)
=
\Bx^a \PD{a}{}
+
\Bx^b \PD{b}{}
+
\Bx^c \PD{c}{}
\equiv
\Bx^a \partial_a
+
\Bx^b \partial_b
+
\Bx^c \partial_c.
\end{dmath}

The Stokes integral can be evaluated over this volume element for either scalar fields \( \psi \), vector fields \( \Bf \), or bivector fields \( B \) and takes the form

\begin{subequations}
\label{eqn:stokesTheoremCore:1481}
\begin{equation}\label{eqn:stokesTheoremCore:1501}
\int_V d^3 \Bx \cdot (\boldpartial \wedge \psi) =
\int_V (d^3 \Bx \cdot \boldpartial) \psi
=
\int_{\partial V} d^2 \Bx \psi
\end{equation}
\begin{equation}\label{eqn:stokesTheoremCore:1521}
\int_V d^3 \Bx \cdot (\boldpartial \wedge \Bf) =
\int_V (d^3 \Bx \cdot \boldpartial) \cdot \Bf
=
\int_{\partial V} d^2 \Bx \cdot \Bf
\end{equation}
\begin{equation}\label{eqn:stokesTheoremCore:1541}
\int_V d^3 \Bx \cdot (\boldpartial \wedge B) =
\int_V (d^3 \Bx \cdot \boldpartial) \cdot B
=
\int_{\partial V} d^2 \Bx \cdot B.
\end{equation}
\end{subequations}

When working with \R{3} vector spaces, \( \boldpartial = \spacegrad \), but in higher dimensional spaces, the gradient can also be substituted above due using the same arguments about projection onto the tangent space.

An explicit value for the differential form of the boundary integral is desired and can be obtained from the mnemonic \cref{eqn:stokesTheoremCore:1561}

\begin{dmath}\label{eqn:stokesTheoremCore:1581}
\sum_i d^3 \Bx \cdot \Bx^i
=
\sum_i da db dc \lr{ \Bx_a \wedge \Bx_b \wedge \Bx_c } \cdot \Bx^i
=
\sum_i da db dc \lr{
\Bx_a \wedge \Bx_b +
\Bx_b \wedge \Bx_c +
\Bx_c \wedge \Bx_a }.
\end{dmath}

The bounding form for the three parameter volume is therefore

\begin{dmath}\label{eqn:stokesTheoremCore:1601}
d^2 \Bx
=
d\Bx_a \wedge d\Bx_b +
d\Bx_b \wedge d\Bx_c +
d\Bx_c \wedge d\Bx_a.
\end{dmath}

         \subsection{Using scalar volume elements}
            %
% Copyright © 2016 Peeter Joot.  All Rights Reserved.
% Licenced as described in the file LICENSE under the root directory of this GIT repository.
%

FIXME: remove most of this and introduce inline with the oriented area and volume results.  This is already done for the \( d^2 \Bx \) integrals.

In \R{3} the area elements of
(FIXME: equation reference dead with rewrite)
%\cref{eqn:twoparameter:140}
, and volume elements of 
\cref{eqn:threeparameter:1481}
can be reexpressed as scalars, recovering a number of the integral calculus identities that are more familiar than the wedge product variants above.

The pseudoscalar volume element can be written

\begin{dmath}\label{eqn:scalarVolumeElement:1621}
d^3 \Bx = I dV,
\end{dmath}
and the (oriented) area elements can be written as

\begin{dmath}\label{eqn:scalarVolumeElement:1641}
d^2 \Bx \ncap = I dA,
\end{dmath}
or
\begin{dmath}\label{eqn:scalarVolumeElement:1661}
d^2 \Bx = I \ncap dA.
\end{dmath}

For \( \psi \in \bigwedge^0, \Bf \in \bigwedge^1, B \in \bigwedge^2 \), this gives

\begin{subequations}
\label{eqn:scalarVolumeElement:1681}
\begin{equation}\label{eqn:scalarVolumeElement:1701}
I \int_A dA \ncap \wedge \spacegrad \psi = \ointclockwise d\Bx \psi
\end{equation}
\begin{equation}\label{eqn:scalarVolumeElement:1721}
I \int_A dA \ncap \wedge \spacegrad \wedge \Bf = \ointclockwise d\Bx \cdot \Bf
\end{equation}
\begin{equation}\label{eqn:scalarVolumeElement:1741}
\int_V dV \spacegrad \psi = \int_{\partial V} dA \ncap \psi
\end{equation}
\begin{equation}\label{eqn:scalarVolumeElement:1761}
\int_V dV \spacegrad \wedge \Bf = \int_{\partial V} dA \ncap \wedge \Bf
\end{equation}
\begin{equation}\label{eqn:scalarVolumeElement:1781}
\int dV \spacegrad \wedge B = \int_{\partial V} dA \ncap \wedge B
\end{equation}
\end{subequations}

It is straightforward to re-express all the wedge products above in their dual forms.
With \( B = I \Bf \), that is

\begin{subequations}
\label{eqn:scalarVolumeElement:1801}
\begin{equation}\label{eqn:scalarVolumeElement:1821}
\int_A dA \ncap \cross \spacegrad \psi = \ointctrclockwise d\Bx \psi
\end{equation}
\begin{equation}\label{eqn:scalarVolumeElement:1841}
\int_A dA \ncap \cdot (\spacegrad \cross \Bf) = \ointctrclockwise d\Bx \cdot \Bf
\end{equation}
\begin{equation}\label{eqn:scalarVolumeElement:1861}
\int_V dV \spacegrad \psi = \int_{\partial V} dA \ncap \psi
\end{equation}
\begin{equation}\label{eqn:scalarVolumeElement:1881}
\int_V dV \spacegrad \cross \Bf = \int_{\partial V} dA \ncap \cross \Bf
\end{equation}
\begin{equation}\label{eqn:scalarVolumeElement:1901}
\int dV \spacegrad \cdot \Bf = \int_{\partial V} dA \ncap \cdot \Bf.
\end{equation}
\end{subequations}

Each of the cross product terms above can also be put into dual forms, giving

\begin{subequations}
\label{eqn:scalarVolumeElement:1801c}
\begin{equation}\label{eqn:scalarVolumeElement:1821c}
\int_A dA \ncap \cdot \lr{ I \spacegrad \psi } = \ointclockwise d\Bx \psi
\end{equation}
\begin{equation}\label{eqn:scalarVolumeElement:1841c}
\int_A dA \ncap \cdot (\spacegrad \cdot B) = \ointctrclockwise d\Bx \cdot (I B)
\end{equation}
\begin{equation}\label{eqn:scalarVolumeElement:1881c}
\int_V dV \spacegrad \cdot B = \int_{\partial V} dA \ncap \cdot B.
\end{equation}
\end{subequations}

Note that all of
\cref{eqn:scalarVolumeElement:1861}, \cref{eqn:scalarVolumeElement:1901}, and \cref{eqn:scalarVolumeElement:1881c} all have the same form

%\begin{equation}\label{eqn:scalarVolumeElement:1881d}
\boxedEquation{eqn:scalarVolumeElement:1881d}{
\int_V dV \spacegrad \cdot A = \int_{\partial V} dA \ncap \cdot A.
}
%\end{equation}
\index{divergence theorem}

This is also true for pseudoscalar grades, which can be demonstrated by multiplying both sides of \cref{eqn:scalarVolumeElement:1741} with \( I \).
This implies that \cref{eqn:scalarVolumeElement:1881d} is valid for any \R{3} multivector, generalizing the conventional divergence theorem over a 3D volume to all spatial grades.

         \subsection{Problems}
            %
% Copyright � CCYY Peeter Joot.  All Rights Reserved.
% Licenced as described in the file LICENSE under the root directory of this GIT repository.
%
\makeproblem{Stokes' theorem relation to Green's theorem}{problem:stokesAndGreens:1}{
Show that Stokes' theorem, in its two parameter form, applied to a vector field recovers Green's theorem.
\index{Green's theorem}
\index{Stokes' theorem}
} % problem

\makeanswer{problem:stokesAndGreens:1}{

To demonstrate this, expand the LHS of the Stokes identity

\begin{dmath}\label{eqn:stokesAndGreens:20}
\int_A d^2 \Bx \cdot (\boldpartial \wedge \Bf) = \ointclockwise d\Bx \cdot \Bf.
\end{dmath}

Assuming \( u, v\) parameterization

\begin{dmath}\label{eqn:stokesAndGreens:40}
\int_A d^2 \Bx \cdot (\boldpartial \wedge \Bf)
=
\int_A (d\Bx_u \wedge d\Bx_v) \cdot (\boldpartial \wedge \Bf)
=
\int_A ((d\Bx_u \wedge d\Bx_v) \cdot \Bx^u) \cdot \partial_u \Bf
+
\int_A ((d\Bx_u \wedge d\Bx_v) \cdot \Bx^v) \cdot \partial_v \Bf
=
-\int_A du dv \Bx_v \cdot \partial_u \Bf
+
\int_A du dv \Bx_u \cdot \partial_v \Bf
=
-\int_A du dv \Bx_v \cdot \partial_u \Bf
+
\int_A du dv \lr{
-\Bx_v \cdot \partial_u \Bf
+
\Bx_u \cdot \partial_v \Bf
}.
\end{dmath}

The coordinate expansion of \( \Bf \) with respect to the tangent space coordinates is

\begin{dmath}\label{eqn:stokesAndGreens:60}
\Bf = \Bx^u f_u + \Bx^v f_v + \Bf_\perp
\end{dmath}

where \( \Bf_\perp \) lies in normal to the tangent space at the point in question.
Because \( \Bx_v \) is computed with \( u \) held fixed and \( \Bx_u \) computed with \( v \) held fixed, the area integrand can be written

\begin{dmath}\label{eqn:stokesAndGreens:80}
-\Bx_v \cdot \partial_u \Bf
+
\Bx_u \cdot \partial_v \Bf
=
-\PD{u}{}\lr{ \Bx_v \cdot \Bf }
+\PD{v}{}\lr{ \Bx_u \cdot \Bf }
=
-\PD{u}{f_v}
+\PD{v}{f_u},
\end{dmath}

which gives
\begin{dmath}\label{eqn:stokesAndGreens:100}
\int_A du dv \lr{ -\PD{u}{f_v}
+\PD{v}{f_u}
}
=
\ointclockwise d\Bx \cdot \Bf,
\end{dmath}

which recovers \cref{thm:stokesTheoremGeometricAlgebra:1660} as desired.
} % answer

            %
% Copyright © 2016 Peeter Joot.  All Rights Reserved.
% Licenced as described in the file LICENSE under the root directory of this GIT repository.
%

\makeproblem{\R{3} dual forms of Stokes' theorem.}{problem:stokesTheoremCoreProblems:1}{
Prove
\makesubproblem{}{problem:stokesTheoremCoreProblems:1:a}
\cref{eqn:scalarVolumeElement:1681},
\makesubproblem{}{problem:stokesTheoremCoreProblems:1:b}
\cref{eqn:scalarVolumeElement:1801},
\makesubproblem{}{problem:stokesTheoremCoreProblems:1:c}
and \cref{eqn:scalarVolumeElement:1801c}.
} % problem

\makeanswer{problem:stokesTheoremCoreProblems:1}{

The volume elements are
\makeSubAnswer{}{problem:stokesTheoremCoreProblems:1:a}
\begin{subequations}
\label{eqn:stokesTheoremCoreProblems:20}
\begin{dmath}\label{eqn:stokesTheoremCoreProblems:40}
d^2 \Bx \cdot \spacegrad
=
dA \gpgradeone{ I \ncap \spacegrad }
=
dA I \ncap \wedge \spacegrad
\end{dmath}
\begin{dmath}\label{eqn:stokesTheoremCoreProblems:60}
d^2 \Bx \cdot (\spacegrad \wedge \BA)
=
dA \gpgradezero{ I \ncap \spacegrad \BA }
=
dA I \ncap \wedge \spacegrad \wedge \BA
\end{dmath}
\begin{dmath}\label{eqn:stokesTheoremCoreProblems:80}
d^3 \Bx \cdot \spacegrad \phi
=
dV \gpgradetwo{ I \spacegrad \phi }
=
dV I \spacegrad \phi
\end{dmath}
\begin{dmath}\label{eqn:stokesTheoremCoreProblems:100}
d^3 \Bx \cdot (\spacegrad \wedge \BA)
=
dV \gpgradeone{ I (\spacegrad \wedge \BA) }
=
dV I \spacegrad \wedge \BA
\end{dmath}
\begin{dmath}\label{eqn:stokesTheoremCoreProblems:120}
d^3 \Bx \cdot (\spacegrad \wedge B)
=
dV \gpgradezero{ I (\spacegrad \wedge B) }
=
dV I \spacegrad \wedge B.
\end{dmath}
\end{subequations}

The corresponding boundary forms are
\begin{subequations}
\label{eqn:stokesTheoremCoreProblems:140}
\begin{equation}\label{eqn:stokesTheoremCoreProblems:160}
d\Bx \psi
\end{equation}
\begin{dmath}\label{eqn:stokesTheoremCoreProblems:180}
d\Bx \cdot \BA
\end{dmath}
\begin{dmath}\label{eqn:stokesTheoremCoreProblems:200}
d^2 \Bx \psi
=
dA I \ncap \psi
\end{dmath}
\begin{dmath}\label{eqn:stokesTheoremCoreProblems:220}
d^2 \Bx \cdot \BA
=
dA \gpgradeone{ I \ncap \BA }
=
dA I \ncap \wedge \BA
\end{dmath}
\begin{dmath}\label{eqn:stokesTheoremCoreProblems:240}
d^2 \Bx \cdot B
=
dA \gpgradezero{ I \ncap B }
=
dA I \ncap \wedge B.
\end{dmath}
\end{subequations}

Assembling these pieces back into the integrals proves the relationships.

\makeSubAnswer{}{problem:stokesTheoremCoreProblems:1:b}

To show \cref{eqn:scalarVolumeElement:1841} note that

\begin{dmath}\label{eqn:stokesTheoremCoreProblems:260}
I (\Ba \wedge \Bb \wedge \Bc)
=
\gpgradezero{ I \Ba \wedge \Bb \wedge \Bc }
=
\gpgradezero{ I \Ba (\Bb \wedge \Bc) -
I \Ba \cdot (\Bb \wedge \Bc)
}
=
\gpgradezero{ I \Ba I(\Bb \cross \Bc) }
=
- \Ba \cdot (\Bb \cross \Bc).
\end{dmath}

To show \cref{eqn:scalarVolumeElement:1901} note that

\begin{dmath}\label{eqn:stokesTheoremCoreProblems:280}
\Ba \wedge (I \BA)
=
\Ba \wedge (I \BA)
=
\gpgradethree{ \Ba I \BA }
=
\gpgradethree{ I \Ba \cdot \BA }
=
I (\Ba \cdot \BA).
\end{dmath}

\makeSubAnswer{}{problem:stokesTheoremCoreProblems:1:c}

For vector \( \Ba \), these transformations all follow from

\begin{dmath}\label{eqn:stokesTheoremCoreProblems:300}
\Ba \cross \Bf
=
\gpgradeone{ -I \Ba \wedge \Bf}
=
\gpgradeone{ -I \Ba \Bf}
=
-\gpgradeone{ \Ba I \Bf}
=
-\Ba \cdot (I \Bf)
=
\Ba \cdot B.
\end{dmath}

} % answer


      \section{Fundamental theorem of geometric calculus}
         \subsection{Fundamental Theorem of Geometric Calculus}
            %
% Copyright � 2016 Peeter Joot.  All Rights Reserved.
% Licenced as described in the file LICENSE under the root directory of this GIT repository.
%
%{
%\input{../blogpost.tex}
%\renewcommand{\basename}{fundamentalTheoremOfCalculus}
%\renewcommand{\dirname}{notes/phy1520/}
%%\newcommand{\dateintitle}{}
%%\newcommand{\keywords}{}
%
%\input{../peeter_prologue_print2.tex}
%
%\usepackage{peeters_layout_exercise}
%\usepackage{peeters_braket}
%\usepackage{peeters_figures}
%\usepackage{siunitx}
%
%\beginArtNoToc
%
%\generatetitle{Fundamental theorem of geometric calculus}
%\label{chap:fundamentalTheoremOfCalculus}

\subsection{Hypervolume integral}
We wish to generalize the concepts of line, surface and volume integrals to hypervolumes and multivector functions, and define a hypervolume integral as

\makedefinition{Multivector integral.}{dfn:fundamentalTheoremOfCalculus:240}{
Given a hypervolume parameterized by \( k \) parameters, k-volume volume element \( d^k \Bx \), and
multivector functions \( F, G \), we define k-volume integral with the vector derivative acting to the right on \( F \) as
\begin{equation*}
\int d^k\Bx \lr{ \rboldpartial F },
\end{equation*}
a k-volume integral with the vector derivative acting to the left \( F \) as
\begin{equation*}
\int F d^k\Bx \lboldpartial,
\end{equation*}
and a k-volume integral with the vector derivative acting bidirectionally on \( F, G \) as
\begin{equation*}
\int F d^k\Bx \lrboldpartial G
\equiv
\int \lr{ F d^k\Bx \lboldpartial} G
+
\int F d^k\Bx \lr{ \rboldpartial G },
\end{equation*}
where the meaning given to these directionally acting derivative operations is
\begin{equation*}
F d^k \Bx \lrboldpartial G
=
F d^k \Bx \lr{ \sum_i \Bx^i {\stackrel{ \leftrightarrow }{\partial_i}} } G
=
(\partial_i F) d^k \Bx \sum_i \Bx^i G
+
F d^k \Bx \sum_i \Bx^i (\partial_i G)
\equiv
(F d^k \Bx \lboldpartial) G
+
F d^k \Bx (\rboldpartial G),
\end{equation*}
with \( \boldpartial \) acting on \( F \) and \( G \), but not the volume element \( d^k \Bx \), which may also be a function of the implied parameterization.
} % definition

The vector derivative (and gradient)
may not commute with \( F, G \) nor the volume element \( d^k \Bx \), so we are forced to use some notation to indicate what the vector derivative (or gradient) acts on.
In conventional right acting cases, where there is no ambiguity, arrows will usually be omitted, but braces may also be used to indicate the scope of derivative operators.
This bidirectional notation will also be used for the gradient, especially for volume integrals in \R{3} where the vector derivative is identitical to the gradient.

Some authors use overdots or ticks are used to indicate the exact scope of multivector derivative operators, as in
\begin{dmath}\label{eqn:fundamentalTheoremOfCalculus:260}
F d^k \Bx \boldpartial G =
\dot{F} d^k \Bx \dot{\boldpartial} G
+
F d^k \Bx \dot{\boldpartial} \dot{G}.
\end{dmath}
Here the (Hestenes) dot notation would have the advantage of emphasizing that the action of the vector derivative (or gradient) is on the functions \( F, G \), and not on the hypervolume element \( d^k \Bx \).
In this book, where we will use ticks to indicate whether gradients are with respect to primed \( \Bx' \) or unprimed \( \Bx \) variables, over arrows seemed like a better choice than dots to indicate operator scope, and have the advantage of being visually conspicuous.

\subsection{Fundamental theorem.}
\index{fundamental theorem of geometric calculus}

The fundamental theorem of geometric calculus is a generalization of many conventional scalar and vector integral theorems.
It is a powerful theorem, which we will use with Green's functions to solve Maxwell's equation, and to derive the geometric algebra form of Stokes' theorem.

\maketheorem{Fundamental theorem of geometric calculus}{thm:fundamentalTheoremOfCalculus:1}{
For multivectors \(F, G \), and a hypervolume element \(d^k \Bx\),
\begin{equation*}
\int_V F d^k \Bx \boldpartial G = \oint_{\partial V} F d^{k-1} \Bx G.
\end{equation*}
}

This theorem relates the hypervolume integral to the integral over the bounding surface of hypervolume.
Additional work is required to describe the precise meaning of the boundary differential \( d^{k-1} \Bx \).  We will do so for line, surface, and volume integrals, proving the theorem in a limited fashion for each of those cases as we go.

For a full proof of \cref{thm:fundamentalTheoremOfCalculus:1}, additional mathematical sublties must be considered.
For full proofs and additional details, the reader is referred to \citep{hestenes1985clifford}, \citep{doran2003gap}, \citep{aMacdonaldVAGC} and \citep{sobczyk2011fundamental}, which all
which all tackle different aspects of general geometric calculus.

Before considering multivector line, surface and volume integral specializations of
\cref{thm:fundamentalTheoremOfCalculus:1},
we will state Stokes' theorem in its geometric algebra form.

%}
%\EndArticle

         \subsection{Green's function for the gradient in Euclidean spaces.}
            \input{../gabookI/calculus/gradientGreensFunction.tex}
            % example:
            \input{../gabookI/calculus/biotSavartGreens.tex}
         \subsection{Problems}
            \input{../gabookI/calculus/helmholtzDerviationMultivector.tex}
%      \section{Helmholtz theorem}
%gabook: 45.1
      \section{Problem solutions}
         \shipoutAnswer

\part{Electromagnetism}
   \chapter{Maxwell's equations}
      \section{Problem solutions}
         \shipoutAnswer
   \chapter{Electrostatics}
      \section{Problem solutions}
         \shipoutAnswer
   \chapter{Magnetostatics}
      \section{Problem solutions}
         \shipoutAnswer
   \chapter{Constitutive relations}
      \section{Problem solutions}
         \shipoutAnswer
   \chapter{Boundary value conditions}
      \section{Problem solutions}
         \shipoutAnswer
   \chapter{Time harmonic fields}
      \section{Problem solutions}
         \shipoutAnswer
   \chapter{Polarization}
      \section{Problem solutions}
         \shipoutAnswer
   \chapter{Potentials}
      \section{Problem solutions}
         \shipoutAnswer
   \chapter{Green's functions}
      \section{Problem solutions}
         \shipoutAnswer
   \chapter{Wave equations}
      \section{Problem solutions}
         \shipoutAnswer
   \chapter{Radiation and scattering}
      \section{Problem solutions}
         \shipoutAnswer
%\end{itemize}



