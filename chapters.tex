%
% Copyright � 2016 Peeter Joot.  All Rights Reserved.
% Licenced as described in the file LICENSE under the root directory of this GIT repository.
%
%----------------------------------------------------------------------------------------
\part{Geometric Algebra.}
   \chapter{Geometric Algebra.}
      \section{Prerequisites.}
         
\section{Junk?}

\subsection{Problems}
%\makeproblem{Explicit squared norm}{problem:multivector:60}{
%   Given a coordinate representation of a vector with respect to a standard basis
%\begin{dmath}\label{eqn:multivector:240}
%   \Bx = \sum_{i = 1}^N x_i \Be_i,
%\end{dmath}
%
%show that the squared norm is
%\begin{dmath}\label{eqn:multivector:260}
%   \Norm{\Bx}^2 = \Bx \cdot \Bx = \sum_{i = 1}^N x_i^2 (\Be_i \cdot \Be_i).
%\end{dmath}
%
%Observe that for a Euclidean vector space this is the squared length in the Pythagorean sense.
%}
%
\makeproblem{Null vector}{problem:multivector:80}{
Given a two dimensional non-Euclidean vector space with basis elements satisfying
\( \gamma_0 \cdot \gamma_0 = 1 = -\gamma_1 \cdot \gamma_1 \), construct a vector that has a squared
norm of 0.  Such a vector is called a null vector.
%   \Bx = \gamma_0 + \gamma_1,
}


\subsection{basis, norm, ...}

%We will use a representation such as \( \Bv = x \Be_1 + y \Be_2 + z \Be_3 \) for such vectors, where the
%coordinates are always paired with their respective direction vectors, and will not use
%column vector of coordinates or tuples such as \( \Bv = (x, y, z)\).
%, \Bv = x \xcap + y \ycap + z \zcap, or \Bv = x \ahat_x + y \ahat_y + z \ahat_z.
\makedefinition{Coordinates.}{dfn:prerequisites:coordinates}{
%Given a basis \( B =
FIXME: define
} % definition

%\makedefinition{Basis and coordinates}{dfn:multivector:basis}{
%   If \( N \) is the dimension of a vector space \( V \), a set of \( N \) vectors \( B = \setlr{ \Ba_1, \Ba_2, \cdots , \Ba_N } \) is a basis for that vector space, if it is possible to form any vector \( \Bx \in V \) as a linear combination of those vectors \( \Ba_k \).  That is, there exists scalars \( c_k \) such that for any \( \Bx \in V \)
%
%\begin{equation*}
%   \Bx = \sum_{k = 1}^N c_k \Ba_k.
%\end{equation*}
%
%The numbers \( (c_1, c_2, \cdots, c_N ) \) are referred to as the coordinates of the vector \( \Bx \) with respect to the basis \( B \).
%}

\makedefinition{Standard basis, and dot product properties.}{dfn:multivector:standardbasis}{
   Any vector space \( V \) used in this book will be assumed to have been generated from a basis \( \setlr{ \Be_1, \Be_2, \cdots, \Be_N } \), associated with a dot product that has the properties

\begin{enumerate}
   \item \( \Be_i \cdot \Be_i = \pm 1 \).
   \item \( \Be_i \cdot \Be_j = 0 \) for any \( i \ne j \).
\end{enumerate}

Such a basis will called a standard basis.  When these dot products are always positive, the vector space is referred to as a Euclidean vector space.
}

\paragraph{FIXME: remove?}
There are many possible standard bases sets.  In \R{3}, it is conventional to refer to \( \Be_1, \Be_2, \Be_3 \) as the standard bases elements if these represent the directions of the x, y, and z directions respectively.  Unless otherwise noted \( \Be_k \) refers to the direction vector for the k-th direction in a standard basis for that space.
The only non-Euclidean vector space of interest in this book (for relativistic material), has a Minkowski dot product.  For such a space, the standard basis elements will be labelled \( \setlr{ \gamma_0, \gamma_1, \gamma_2, \gamma_3 } \), where for \( i \in [1,3] \), \( \gamma_0 \cdot \gamma_0 = \pm 1 = -\gamma_i \cdot \gamma_i \).  The positive sign convention will be used.

%GA requires the vector space to have an associated
%dot product \( \Bx \cdot \By \) that
%defines the notion of perpendicularity for the space.  We will want to extend the scalar multiplication operation of the vector
%space to complex numbers, but
%will not require a (complex) order dependent inner product \( \innerprod{\Bx}{\By} \) for our vector space.
%

\paragraph{The metric, length and normality.}

An abstract vector need not have an associated notion of length, nor a notion of perpendicularity (normality).
In abstract vector algebra, length and normality are provided by defining an associated dot product \(\Bx \cdot \By\), or inner product \(\innerprod{\Bx}{\By}\).
In GA, length and normality of two vectors are provided by a metric \(g(\Bx, \By)\).
Like the dot product where \( \Bx \cdot \By = \By \cdot \Bx\), this metric is independent of order, a property that is not generally required of the inner product.
However, unlike both the dot and inner products of abstract vector algebra, where \( \Bx \cdot \Bx \ge 0\), and \( \innerprod{\Bx}{\Bx} \ge 0\), the metric \(g(\Bx, \Bx)\) may be negative (i.e. for spacetime vectors).
If \(c \) is any real or complex number, the metric in GA is \( g(c \Bx, c \Bx) = c^2 g(\Bx, \Bx)\), unlike the inner product in complex spaces, where \( \innerprod{c \Bx}{c \Bx} = \Abs{c}^2 \innerprod{c \Bx}{c \Bx} \).
Effectively, this means that our underlying direction vectors are always real.

\subsection{Orientation}
We are familiar with the idea of an oriented line segment (a vector), a quantity that can be visualized as an arrow with direction and magnitude.
The idea of an oriented plane, volume, or hypervolume is probably less familiar.
An oriented plane segment, in addition to having a specific area and a direction in space, can be visualized as having a
circulation direction, or handedness.
In a three dimensional space, this circulation direction can be associated with one of the two possible normal directions for the plane.
An oriented volume, in addition to having a given magnitude, is considered to have an associated circulation direction along its surface.
In a three dimensional space, an oriented volume can be conceptualized as a volume with either an inwards or outwards normal.

\subsection{dot and metric original text}

Vectors are often represented with an implied basis, with tuples like \( \Bx = (x,y,z) \), or with column (or row) vectors like
\(
   \Bx =
\begin{bmatrix}
x \\
y \\
z
\end{bmatrix}
\).
The values \( x, y, z \) in these representations are called the coordinates of the vectors, but only have specific meaning once a direction and magnitude is associated with each coordinate (i.e. a basis is chosen).
In three dimensions, the simplest such basis choice (the standard basis), associates the respective coordinates with a set of mutually perpendicular (normal) directions.
This is conventionally a right handed triple of direction vectors of unit length, perhaps designated \( \xcap, \ycap, \zcap \) or \( \Be_1, \Be_2, \Be_3 \).

In GA, when working with coordinates, we generally prefer to make the basis explicit, so instead of writing a vector as a set of coordinates, these coordinates
will be explicitly paired with their associated basis vectors.
For example in \R{3} a vector with coordinates \( x, y, z \) will be written as

\begin{dmath}\label{eqn:prerequisites:280}
x \Be_1 + y \Be_2 + z \Be_3.
\end{dmath}

By convention, we understand that \( \Be_1, \Be_2, \Be_3 \) in \cref{eqn:prerequisites:280} are unit length vectors, and are all mutually perpendicular (orthonormal).
The vector space must be augmented with a dot product (or inner product) to provide a measure of length and normality.  

%\makedefinition{Inner product.}{dfn:prerequisites:innerproduct}{
%The inner product 
%} % definition

For \R{3}, the dot product satisfies the following conditions

\begin{equation}\label{eqn:prerequisites:320}
\Be_i \cdot \Be_j = \delta_{ij} \, \forall i, j \in [1,3],
\end{equation}

where \( \delta_{ij} \) is the Kronecker delta \( \delta_{ij} = 1 \) for \( i = j \) and \( \delta_{ij} = 0 \) for \( i \ne j \).
Specifying the action of the dot product on all the unit vectors, completely specifies the action of the dot product on any two vectors, provided one assumes that the dot product is a bilinear operator.
For example, given

\begin{dmath}\label{eqn:prerequisites:340}
\begin{aligned}
\Ba &= a_1 \Be_1 + a_2 \Be_2 + a_3 \Be_3 \\
\Bb &= b_1 \Be_1 + b_2 \Be_2 + b_3 \Be_3,
\end{aligned}
\end{dmath}

or \( \Ba = \sum_i a_i \Be_i, \Bb = \sum_j b_j \Be_j \), we recover the familiar coordinate description of the dot product

\begin{dmath}\label{eqn:prerequisites:360}
\Ba \cdot \Bb
=
\lr{ \sum_i a_i \Be_i } \cdot \lr{ \sum_j b_j \Be_j }
=
\sum_{i,j} a_i b_j \lr{ \Be_i \cdot \Be_j }
=
\sum_{i,j} a_i b_j \delta_{ij}
=
\sum_{i} a_i b_i.
\end{dmath}

Electromagnetism is intrinsically relativistic, and there will be circumstances where vectors with both space and time components are required.
In physics, these are called four-vectors, but we will call them spacetime vectors here to avoid confusion with \( k = 4 \) k-vectors.
Following \citep{doran2003gap}, the Dirac (matrix) notation will be used as the relativistic basis, so a spacetime vector might be written like

\begin{dmath}\label{eqn:prerequisites:300}
A = c t \gamma_0 + x \gamma_1 + y \gamma_2 + z \gamma_3.
\end{dmath}

It will be seen later that our spacetime vector representation has similar properties to Dirac matrices, but we need not refer to any specific matrix representation.

For spacetime vectors, we can also assume a dot product operation between the basis vectors.  For example, given two spacetime vectors

\begin{dmath}\label{eqn:prerequisites:380}
\begin{aligned}
A &= c t \gamma_0 + x \gamma_1 + y \gamma_2 + z \gamma_3 \\
B &= c t' \gamma_0 + x' \gamma_1 + y' \gamma_2 + z' \gamma_3,
\end{aligned}
\end{dmath}

if the action of a ``dot-product'' is known between all basis vectors \( \gamma_\mu, \mu \in [0,3] \), then it will be possible to compute the dot-product of any pair of four vectors as done above for the \R{3} example.  Special relativity constrains the properties of four-vector dot products, requiring the following of the four-vector basis

\begin{dmath}\label{eqn:prerequisites:400}
\left\{
\begin{array}{l l}
\gamma_\mu \cdot \gamma_\nu = 0 & \quad \mbox{ \( \mu \ne \nu ; \mu, \nu \in [0,3] \) } \\
\gamma_0 \cdot \gamma_0 = -\gamma_i \cdot \gamma_i = \pm 1 & \quad \mbox{ \( i \in [1,3] \) }
\end{array}
\right.
\end{dmath}

Strictly speaking, this is a specification of a metric, not a dot product, since this four vector dot product specification does not satisfy the positive definite property required by most dot product definitions (i.e. \( A \cdot A \ge 0 \)).
There is a sign ambiguity in the metric specification above.  The physics of relativity is independent of the sign convention used, but we will use the positive sign convention, consistent with field theory and most matrix representations of the Dirac matrices.
\footnote{In general relativitity, many authors will use the opposite sign convention.}

Stated explicitly, we use a metric where the basis vectors satisfy the following properties

\begin{dmath}\label{eqn:prerequisites:420}
\left\{
\begin{array}{l l}
\gamma_\mu \cdot \gamma_\nu = 0 & \quad \mbox{ \( \mu \ne \nu ; \mu, \nu \in [0,3] \) } \\
\gamma_i \cdot \gamma_i = -1& \quad \mbox{ \( i \in [1,3] \) } \\
\gamma_0 \cdot \gamma_0 = 1. &\\
\end{array}
\right.
\end{dmath}


%%%\makeproblem{}{problem:multivector:50}{
%%%The most general definition of an Euclidean norm satisfies all of the properties
%%%
%%%\begin{enumerate}
%%%   \item \( \Norm{\Bx} \ge 0 \), and \( \Norm{\Bx} = 0 \iff \Bx = 0 \).
%%%   \item \( \Norm{a \Bx} = \Abs{a} \Norm{\Bx} \).
%%%   \item \( \Norm{\Bx + \By} \le \Norm{\Bx} + \Norm{\By} \).
%%%\end{enumerate}
%%%
%%%If the coordinates of a vector with respect to the standard basis are \( x_i \) then show that the Euclidean norm defined in
%%%that the Pythagorean norm
%%%\begin{equation*}
%%%\Norm{\Bx}^2 = \sum_{i = 1}^N x_i^2,
%%%\end{equation*}
%%%
%%%satisfies these properties.
%%%} % problem
%%%

         \subsection{Vector space.}
            %
% Copyright © 2017 Peeter Joot.  All Rights Reserved.
% Licenced as described in the file LICENSE under the root directory of this GIT repository.
%
Vectors have many generalizations in mathematics, where
a number of disparate mathematical objects
%, such as
%directed ``arrows'', tuples of real or complex numbers, matrices, functions, polynomials, and quantum states
can all be considered vectors.
A vector space is an enumeration of the properties and operations that are common to a set of
vector-like objects, allowing them to be treated in a unified fashion, regardless of their representation and application.
%The definition of a vector space and some other basic ideas from linear algebra are all reviewed here.
%This review will set the stage for the definition of a \boldTextAndIndex{multivector space}, the GA analogue of a vector space.

\index{vector space}
\makedefinition{Vector space.}{def:prerequisites:vectorspace}{
A (real) vector space is a set \( V = \setlr{\Bx, \By, \Bz, \cdots} \), the elements of which are called vectors, which has an addition operation designated \( + \) and a scalar multiplication operation designated by juxtaposition, where the following axioms are satisfied
for all vectors \( \Bx, \By, \Bz \in V \) and scalars \( a, b \in \bbR \)

\begin{tablebox}[tabularx={X|Y}]{Vector space axioms.}
    V is closed under addition & \( \Bx + \By \in V \) \\ \hline
    V is closed under (scalar) multiplication & \( a \Bx \in V \) \\ \hline
    Addition is associative & \( (\Bx + \By) + \Bz = \Bx + (\By + \Bz) \) \\ \hline
    Addition is commutative & \( \By + \Bx = \Bx + \By \) \\ \hline
    There exists a zero element \( \Bzero \in V \)  & \( \Bx + \Bzero = \Bx \) \\ \hline
    For any \( \Bx \in V \) there exists a negative additive inverse \( -\Bx \in V \) & \( \Bx + (-\Bx) = \Bzero \) \\ \hline
    (Scalar) multiplication is distributive  & \( a( \Bx + \By ) = a \Bx + a \By \), \( (a + b)\Bx = a \Bx + b\Bx \) \\ \hline
    (Scalar) multiplication is associative & \( (a b) \Bx = a ( b \Bx ) \) \\ \hline
    There exists a multiplicative identity \( 1 \in V \) & \( 1 \Bx = \Bx \) \\ \hline
\end{tablebox}
}

% Fixme:
% Mo found this section confusing, because he can imagine places (like Fourier, spherical harmonics, ...)
% where we'd want to mix GA with infinite dimensional vector spaces.  Another example is my use of
% complex valued "phasors" in the plane wave and polarization section.
%
% Perhaps rework along the following lines:
%Some problems have been set below with some examples of vector spaces....
Despite the generality of this definition, the vector spaces used in GA are fairly restricted.
In particular, electrodynamic applications of GA require only two, three or four dimensional real vector spaces.
No vector spaces with matrix, polynomial, or complex tuple elements will be required, nor will any
infinite dimensional vector spaces.
%The only unconventional vector space of interest will be a ``space-time'' vector space containing a
%time like ``direction'', 1-3 spatial directions, and a generalized length operation that can be negative.

\index{\R{N}}
\makeproblem{\R{N}}{problem:prerequisites:RN}{
Define \R{N} as the set of tuples \( \setlr{ (x_1, x_2, \cdots) \mid x_i \in \bbR } \).
Show that \R{N} is a vector space when the
addition operation is defined as
\( \Bx + \By \equiv (x_1 + y_1, x_2 + y_2, \cdots) \)
, and
scalar multiplication
is defined as
\( a \Bx \equiv (a x_1 , a x_2 , \cdots ) \) for any
\( \Bx = (x_1, x_2, \cdots) \in \bbR^N \),
\( \By = (y_1, y_2, \cdots) \in \bbR^N \), and
\( a \in \bbR \).
} % problem


         \subsection{Basis, span and dimension.}
            %
% Copyright © 2017 Peeter Joot.  All Rights Reserved.
% Licenced as described in the file LICENSE under the root directory of this GIT repository.
%
\makedefinition{Linear combination}{dfn:prerequisites:linearcombination}{
Let \( S = \setlr{ \Bx_1, \Bx_2, \cdots, \Bx_k } \) be a subset of a vector space \( V \).
A linear combination of vectors in \( S \) is any sum
\begin{equation*}
a_1 \Bx_1
+
a_2 \Bx_2
+
\cdots
+
a_k \Bx_k.
\end{equation*}
} % definition

\makedefinition{Linear dependence.}{dfn:prerequisites:dependence}{
Let \( S = \setlr{ \Bx_1, \Bx_2, \cdots, \Bx_k } \) be a subset of a vector space \( V \).
This set \( S \) is linearly dependent if any equation
\begin{equation*}
0 =
a_1 \Bx_1
+
a_2 \Bx_2
+
\cdots
+
a_k \Bx_k,
\end{equation*}

can be constructed for which not all of the coefficients \( a_i \) are zero.
} % definition

\makedefinition{Linear independence.}{dfn:prerequisites:independence}{
Let \( S = \setlr{ \Bx_1, \Bx_2, \cdots, \Bx_k } \) be a subset of a vector space \( V \).
This set is linearly independent if the there are no equations with \( a_i \ne 0 \) such that
\begin{equation*}
0 =
a_1 \Bx_1
+
a_2 \Bx_2
+
\cdots
+
a_k \Bx_k.
\end{equation*}
} % definition

\makedefinition{Span.}{dfn:prerequisites:span}{
Let \( S = \setlr{ \Bx_1, \Bx_2, \cdots, \Bx_k } \) be a subset of a vector space \( V \).  The span
of this set is the set of all linear combinations of these vectors, denoted
\begin{equation*}
\Span(S) =
\setlr{
a_1 \Bx_1
+
a_2 \Bx_2
+
\cdots
+
a_k \Bx_k}.
\end{equation*}
} % definition

\makedefinition{Subspace.}{dfn:prerequisites:subspace}{
Let \( S = \setlr{ \Bx_1, \Bx_2, \cdots, \Bx_k } \) be a subset of a vector space \( V \).  This subset is
a subspace if \( S \) is a vector space under the multiplication and addition operations of the vector space \( V \).
} % definition

\makedefinition{Basis and dimension}{dfn:prerequisites:basisanddimension}{
Let \( S = \setlr{ \Bx_1, \Bx_2, \cdots, \Bx_n } \) be a linearly independent subset of \( V \).  This set is a basis if \( \Span(S) = V \).  The number of vectors \( n \) in this set is called the dimension of the space.
} % definition


         \subsection{Standard basis, length and normality.}
            %
% Copyright © 2017 Peeter Joot.  All Rights Reserved.
% Licenced as described in the file LICENSE under the root directory of this GIT repository.
%
\index{dot product}
\makedefinition{Dot product.}{dfn:prerequisites:dotproduct}{
Let \( \Bx, \By \) be vectors from a vector space \( V \).
A dot product \( \Bx \cdot \By \) is a mapping \( V \cross V \rightarrow \bbR \)
with the following properties

\begin{tcolorbox}[tab2,tabularx={X|Y},title=Dot product properties.,boxrule=0.5pt]
    Symmetric in both arguments & \( \Bx \cdot \By = \By \cdot \Bx \) \\ \hline
    Bilinear & \( (a \Bx + b \By) \cdot (a' \Bx' + b' \By' ) =
a a' (\Bx \cdot \Bx') + b b' (\By \cdot \By')
+
a b' (\Bx \cdot \By') + b a' (\By \cdot \Bx') \)
\\ \hline
    (Optional) Positive definite & \( \Bx \cdot \Bx \ge 0 \) \\ \hline
\end{tcolorbox}
} % definition

Because the dot product is bilinear, it is
specified completely by the dot products of a set of basis elements for the space.
For example,
given a basis \( \setlr{ \Be_1, \Be_2, \cdots, \Be_N} \), and two vectors

\begin{dmath}\label{eqn:prereq_standardbasis:240}
\begin{aligned}
   \Bx &= \sum_{i = 1}^N x_i \Be_i \\
   \By &= \sum_{i = 1}^N y_i \Be_i,
\end{aligned}
\end{dmath}

the dot product of the two is

\begin{dmath}\label{eqn:prereq_standardbasis:260}
\Bx \cdot \By
=
   \lr{ \sum_{i = 1}^N x_i \Be_i } \cdot
   \lr{ \sum_{j = 1}^N y_j \Be_j }
=
   \sum_{i,j = 1}^N x_i y_j \lr{ \Be_i \cdot \Be_j }.
\end{dmath}

Such an expansion in coordinates can be written in matrix form as

\begin{dmath}\label{eqn:prereq_standardbasis:280}
\Bx \cdot \By
=
\Bx^\T G \By,
\end{dmath}

where \( G \) is the symmetric matrix with elements \( g_{ij} = \Be_i \cdot \Be_j \).
This matrix \( G \), or its elements \( g_{ij} \) is also called the metric for the space.

In this book the metric is always diagonal, with all diagonal values having an absolute value of one.
The positive definite property \( \Bx^\T G \By \ge 0 \) is not required of the metric or its associated dot product.
This omission has specific
relevance in electrodynamics, since Maxwell's equations take their simplest form when expressed in terms of four-vector (relativistic) vector spaces, where some of the metric matrix elements are negative.

\index{length}
\makedefinition{Length}{dfn:prerequisites:norm}{
   The squared norm of a vector \( \Bx \) is defined as
\begin{equation*}
   \Norm{\Bx}^2 = \Bx \cdot \Bx,
\end{equation*}

a quantity that need not be positive.
The length of a vector \( \Bx \) is defined as
\begin{equation*}
\Norm{\Bx} =
\sqrt{\Abs{ \Bx \cdot \Bx }}.
\end{equation*}
}

%A vector space with an associated norm based length is called a normed vector space.
%Any dot product space is also a normed vector space.

\index{unit vector}
\makedefinition{Unit vector}{dfn:prerequisites:unitvector}{
   A vector \( \Bx \) is called a unit vector if its absolute squared norm is one (\( \Abs{\Bx \cdot \Bx} = 1 \)).
} % definition

%A unit vector \( \xcap \) may be generated from any vector \( \Bx \) that has a non-zero squared norm by computing
%
%\begin{dmath}\label{eqn:prereq_standardbasis:220}
%\xcap = \frac{\Bx}{\sqrt{\Abs{\Norm{\Bx}^2}}}.
%\end{dmath}
%
\index{normal}
\makedefinition{Normal}{dfn:prerequisites:normal}{
   Two vectors from a vector space \( V \) are normal, or orthogonal, if their dot product is zero, \( \Bx \cdot \By = 0 \,\forall \Bx, \By \in V \).
}

\index{orthonormal}
\makedefinition{Orthonormal}{dfn:prerequisites:orthonormal}{
   Two vectors \( \Bx, \By \) are orthonormal if they are both unit vectors and normal to each other (\( \Bx \cdot \By = 0 \), \( \Abs{\Bx \cdot \Bx} = \Abs{\By \cdot \By} = 1 \)).

   A set of vectors \( \setlr{ \Bx, \By, \cdots, \Bz } \) is an orthonormal set if all pairs of vectors in that set are orthonormal.
}

\index{standard basis}
\makedefinition{Standard basis.}{dfn:prerequisites:standardbasis}{
   A basis
\( \setlr{ \Be_1, \Be_2, \cdots, \Be_N} \) is called a standard basis if that set is orthonormal.
} % definition

\index{Euclidean space}
\makedefinition{Euclidean space.}{dfn:prerequisites:euclideanspace}{
   A vector space with basis
   \( \setlr{ \Be_1, \Be_2, \cdots, \Be_N} \) is called Euclidean if all the dot product pairs between the basis elements are not only orthonormal, but positive definite.
That is
\begin{equation*}
\Be_i \cdot \Be_j = \delta_{ij}.
\end{equation*}
} % definition

      \section{Definitions}
         \subsection{Multivector space.}
            %
% Copyright © 2017 Peeter Joot.  All Rights Reserved.
% Licenced as described in the file LICENSE under the root directory of this GIT repository.
%
%{
Geometric algebra takes a vector space and adds two additional operations, a vector multiplication operation, and a generalized addition operation that extends vector addition to include addition of scalars and products of vectors.
Multiplication of vectors is indicated by juxtaposition, for example, if \( \Bx, \By, \Be_1, \Be_2, \Be_3, \cdots \) are vectors, then some vector products are

\begin{dmath}\label{eqn:multivector:20}
\begin{aligned}
&\Bx \By, \Bx \By \Bx, \Bx \By \Bx \By, \\
&\Be_1 \Be_2, \Be_2 \Be_1, \Be_2 \Be_3, \Be_3 \Be_2, \Be_3 \Be_1, \Be_1 \Be_3, \\
&\Be_1 \Be_2 \Be_3, \Be_3 \Be_1 \Be_2, \Be_2 \Be_3 \Be_1, \Be_3 \Be_2 \Be_1, \Be_2 \Be_1 \Be_3, \Be_1 \Be_3 \Be_2, \\
&\Be_1 \Be_2 \Be_3 \Be_1, \Be_1 \Be_2 \Be_3 \Be_1 \Be_3 \Be_2, \cdots
\end{aligned}
\end{dmath}

Vector multiplication is constrained by a rule, the contraction axiom, which specifies that the square of vector is the squared length of that vector (i.e. a scalar).

In a sum of scalars, vectors, and vector products, such as
\begin{dmath}\label{eqn:multivector:40}
1 + 2 \Be_1 + 3 \Be_1 \Be_2 + 4 \Be_1 \Be_2 \Be_3,
\end{dmath}
\( \Be_1 \Be_2 \) is called a bivector, \( \Be_1 \Be_2 \Be_3 \) is called a trivector, and the sum itself is a multivector.

Put more formally

\index{bivector}
\index{2-vector}
\makedefinition{Bivector.}{dfn:multivector:60}{
A bivector, or 2-vector, is a sum of products of pairs of normal vectors.
Given an \( N \) dimensional vector space with an orthonormal basis \( \setlr{ \Be_1, \Be_2, \cdots } \),
a general bivector can be expressed as
\begin{equation*}
\sum_{1 \le i \ne j \le N} B_{ij} \Be_i \Be_j,
\end{equation*}
where \( B_{ij} \) is a scalar.
} % definition

Just as vectors can represent line segments with direction and magnitude, a bivector can represent an plane segment with an orientation (both the orientation of the plane in space, and a sidedness) and magnitude.
Fixme: pictures.
We will see that the products of normal vectors, like \( \Be_1 \Be_2 \) anticommute\footnote{Quantities that anticommute are unchanged if both the order and the sign are toggled.},
for example \( \Be_2 \Be_1 = -\Be_1 \Be_2 \).
This means that many of the products in \cref{eqn:multivector:20} are not independent, and that the definition of a bivector could be a more restrictive sum, such as \( \sum_{1 \le i < j \le N} b_{ij} \Be_i \Be_j \), where \( b_{ij} \) is an antisymmetric
\footnote{An indexed quantity such as \( b_{ij} \) is antisymmetric if toggling the order of indexes changes the sign, that is \( b_{ji} = -b_{ij} \).}
scalar.

\index{trivector}
\index{3-vector}
\makedefinition{Trivector.}{dfn:multivector:80}{
A trivector, or 3-vector, is a sum of products of triplets of mutually normal vectors.
Given an \( N \) dimensional vector space with an orthonormal basis \( \setlr{ \Be_1, \Be_2, \cdots } \),
a general trivector can be expressed as
\begin{equation*}
\sum_{1 \le i \ne j \ne k \le N} T_{ijk} \Be_i \Be_j \Be_k,
\end{equation*}
where \( \T_{ijk} \) is a scalar.
} % definition

A trivector can represent an oriented volume segment.
In a three dimensional space, this orientation describes a ``sidedness'' of the volume, perhaps represented with an outwards or inwards facing normal, or with an oriented cyclic direction on the surface.
In greater than three dimensions, a trivector can have a ``direction'' in the higher dimensional space, as well as a sidedness.
As was the case with the bivector, because not all the products \( \Be_i \Be_j \Be_k \) for any set of indexes \( i, j, k \) are independent, it is possible to form a trivector as a sum over a more restricted set, such as \( \sum_{1 \le i < j < k \le N} T_{ijk} \Be_i \Be_j \Be_k \).
In particular, in three dimensions, all trivectors can be expressed as scalar multiples of \( \Be_1 \Be_2 \Be_3 \).

\index{k-vector}
\index{grade}
\makedefinition{K-vector and grade.}{dfn:multivector:100}{
A k-vector is a sum of products of \( k \) mutually normal vectors.
Given an \( N \) dimensional vector space with an orthonormal basis \( \setlr{ \Be_1, \Be_2, \cdots } \),
a general k-vector can be expressed as
\begin{equation*}
\sum_{1 \le i_1 \ne i_2 \cdots \ne i_k \le N} K_{i_1 i_2 \cdots i_k} \Be_{i_1} \Be_{i_2} \cdots \Be_{i_k},
\end{equation*}
where \( K_{i_1 i_2 \cdots i_k} \) is a scalar.

The number \( k \) of normal vectors that generate a k-vector is called the grade.

A 1-vector is defined as a vector, and a 0-vector is defined as a scalar.
} % definition

We will see that the highest grade for a k-vector in an N dimensional vector space is \( N \).

\index{multivector}
\index{multivector space}
\makedefinition{Multivector space.}{def:multiplication:multivectorspace}{
   Given an N dimensional (generating) vector space \( V \) with an orthonormal basis \( \setlr{ \Be_1, \Be_2, \cdots, \Be_N } \),
%a basis \( \setlr{ \Bx_1, \Bx_2, \cdots } \), 
and a vector multiplication operation represented by juxtaposition,
a multivector is a sum of k-vectors, \( k \in [ 1, N ] \), such as
   \( a_0 + \sum_i a_i \Be_i + \sum_{i \ne j} a_{ij} \Be_i \Be_j + \sum_{i \ne j \ne k} a_{ijk} \Be_i \Be_j \Be_k + \cdots \), where \( a_0, a_i, a_{ij}, \cdots \) are scalars.

A multivector space is a set \( M = \setlr{ x, y, z, \cdots } \) of multivectors, where the following axioms are satisfied

\begin{tcolorbox}[tab2,tabularx={X|Y},title=Multivector space axioms.,boxrule=0.5pt]
    Contraction. & \( \Bx^2 = \Bx \cdot \Bx, \,\forall \Bx \in V \) \\ \hline
    Addition is closed. & \( x + y \in M \) \\ \hline
    Multiplication is closed. & \( x y \in M \) \\ \hline
    Addition is associative. & \( (x + y) + z = x + (y + z) \) \\ \hline
    Addition is commutative. & \( y + x = x + y \) \\ \hline
    There exists a zero element \( 0 \in M \).  & \( x + 0 = x \) \\ \hline
    There exists a negative additive inverse \( -x \in M \). & \( x + (-x) = 0 \) \\ \hline
    Multiplication is distributive.  & \( x( y + z ) = x y + x z \), \( (x + y)z = x z + y z \) \\ \hline
    Multiplication is associative. & \( (x y) z = x ( y z ) \) \\ \hline
    There exists a multiplicative identity \( 1 \). & \( 1 x = x \) \\ \hline
\end{tcolorbox}
}

Compared to the vector space, def'n. \ref{def:prerequisites:vectorspace}, the multivector space

\begin{itemize}
\item presumes a vector multiplication operation, which is not assumed to be commutative (order matters),
\item generalizes vector addition to multivector addition,
\item generalizes scalar multiplication to multivector multiplication (of which scalar multiplication and vector multiplication are special cases),
\item and most importantly, specifies a rule providing the meaning of a squared vector (the contraction axiom).
\end{itemize}

The contraction axiom is arguably the most important of the multivector space axioms, as it allows for multiplicative closure without an infinite dimensional multivector space.
The remaining set of non-contraction axioms of a multivector space are almost that of a field
\footnote{A mathematician would call a multivector space a non-commutative ring with identity \citep{van1943modern}, and could state the multivector space definition much more compactly without listing all the properties of a ring explicitly as done above.}
(as encountered in the study of complex inner products),
as they describe most of the properties one
would expect of a ``well behaved'' set of number-like quantities.
However, a field also requires a multiplicative inverse element for all elements of the space, which exists for some multivector subspaces, but not in general.

%These axioms may seem simple enough, especially since they are not that different from the familiar axioms of the vector space,
%but it will take considerable work to extract all their consequences.
%The subject of Geometric Algebra can be viewed as the study of the impliciations of the axioms
%of the multivector space.

%}

         \subsection{Nomenclature.}
            %
% Copyright © 2017 Peeter Joot.  All Rights Reserved.
% Licenced as described in the file LICENSE under the root directory of this GIT repository.
%
A fair amount of nomenclature and notation is unfortunately required before systematically examining the implications of the multivector space axioms that define geometric algebra.

\index{blade}
\index{grade}
\makedefinition{Blade and grade}{def:multiplication:blade}{
A product of \( k \) perpendicular vectors is called a k-blade, or a blade of grade \( k \).
A grade zero blade is a scalar.

The notation \( F \in \bigwedge^k \) is used in the literature to indicate that \( F \) is a blade of grade \( k \).
}

The maximum grade of a multivector is equal to the dimension of the generating vector space.
For example, for a multivector space generated by \R{3}, no k-vector can have grade greater than 3.

Examples of blades with grades 0, 1, 2, and 3 respectively are

\begin{dmath}\label{eqn:multivector:180}
\begin{aligned}
&1 \\
&\Be_1,\quad \Be_2,\quad \Be_3 \\
&\Be_1 \Be_2,\quad \Be_2 \Be_1,\quad \Be_1 \Be_2 + \Be_2 \Be_3 \\
&\Be_1 \Be_2 \Be_3,\quad \Be_1 \Be_3 \Be_2,\quad \Be_1 \Be_4 \Be_2
\end{aligned}
\end{dmath}

Multivectors which can be factored into perpendicular vector products, such as
\begin{dmath}\label{eqn:multiplication:220}
\Be_1 \Be_2 + 3 \Be_1 \Be_3
=
\Be_1 (\Be_2 + 3 \Be_3),
\end{dmath}

are blades.  In contrast, the following grade 2 multivectors

\begin{dmath}\label{eqn:multiplication:240}
\Be_1 \Be_2 + \Be_3 \Be_4,
\end{dmath}

and
\begin{dmath}\label{eqn:multiplication:260}
\Be_1 \Be_2 + \Be_2 \Be_3 + \Be_3 \Be_1,
\end{dmath}

which cannot be factored into two vector products, are not blades.

\index{k-vector}
\makedefinition{k-vector.}{dfn:multivector:kvector}{
A sum of k-blades is called a k-vector.
} % definition

Multivectors are therefore sums of k-vectors with different grades.

All the k-blade examples in 
\cref{eqn:multivector:180}
 are also k-vectors.
K-vectors with grades 2 and 3 are so pervasive that they are given special names.

\index{bivector}
\makedefinition{Bivector.}{dfn:multivector:bivector}{
A bivector, or 2-vector, is a k-vector with grade 2.
} % definition

The product \( \Be_1 \Be_2 \) is a bivector, as is \( \Be_2 \Be_3 + 3 \Be_4 \Be_1 \)
%Each of \( \Be_1 \Be_2, \Be_2 \Be_1, \Be_1 \Be_2 + \Be_2 \Be_3 \), and \( \Be_1 \Be_2 + \Be_3 \Be_4 \) are bivectors.
%All but the last of these represents an oriented plane segment.

\index{trivector}
\makedefinition{Trivector.}{dfn:multivector:trivector}{
A trivector, or 3-vector, is a k-vector with grade 3.
} % definition

%Quantities with higher grades than 3 are not generally given explicit names.
The multivector \( \Be_3 \Be_1 \Be_2 \) is a trivector, as is \( \Be_1 \Be_2 \Be_3 + 3 \Be_5 \Be_4 \Be_1 \).  The latter is not a blade.
%Each of \( \Be_1 \Be_2 \Be_3, \Be_1 \Be_3 \Be_2, \Be_1 \Be_4 \Be_2 \) are trivectors.
% , and represent oriented volumes.

\index{grade selection}
\makedefinition{Grade selection operator}{dfn:gradeselection:gradeselection}{
Given a set of k-vectors \( M_k, k \in [0,N] \), and any multivector of their sum

\begin{equation*}
M = \sum_{i = 0}^N M_i,
\end{equation*}

the grade selection operator is defined as

\begin{equation*}\label{eqn:gradeselection:40}
\gpgrade{M}{k} \equiv M_k.
\end{equation*}

Due to its importance, selection of the (scalar) zero grade is given the shorthand
\begin{equation*}
\gpgradezero{M} \equiv \gpgrade{M}{0} = M_0.
\end{equation*}
}

For example, if \( M = 3 - \Be_3 + 2 \Be_1 \Be_2 \), then
\begin{equation}\label{eqn:gradeselection:80}
\begin{aligned}
\gpgradezero{M} &= 3 \\
\gpgrade{M}{1} &= - \Be_3 \\
\gpgrade{M}{2} &= 2 \Be_1 \Be_2 \\
\gpgrade{M}{3} &= 0.
\end{aligned}
\end{equation}

\index{orthonormal blades}
\makedefinition{Orthonormal product shorthand.}{dfn:multivector:shorthand}{
Given an orthonormal basis \( \setlr{ \Be_1, \Be_2, \cdots } \), a multiple indexed quantity \( \Be_{ij\cdots k} \) should be interpreted as the product (in the same order) of the basis elements with those indexes

\begin{equation*}
\Be_{ij\cdots k} = \Be_i \Be_j \cdots \Be_k.
\end{equation*}
} % definition

For example,

\begin{equation}\label{eqn:multivector:360}
\begin{aligned}
\Be_{12} &= \Be_1 \Be_2 \\
\Be_{123} &= \Be_1 \Be_2 \Be_3 \\
\Be_{23121} &= \Be_2 \Be_3 \Be_1 \Be_2 \Be_1.
\end{aligned}
\end{equation}

\index{pseudoscalar}
\makedefinition{Pseudoscalar.}{def:multiplication:pseudoscalar}{
A blade with grade that matches the dimension of the space.
}

In a two dimensional space \( \Be_1 \Be_2 \) is a pseudoscalar, as is \( 3 \Be_2 \Be_1 \).  In a three dimensional space
\( \Be_3 \Be_1 \Be_2 \) is a pseudoscalar, as is \( - 7 \Be_3 \Be_1 \Be_2 \).
%A pseudoscalar has an implied orientation, which can be
%associated with the handedness of the underlying basis.
It is conventional to refer to

\begin{dmath}\label{eqn:definitions:320}
i = \Be_1 \Be_2,
\end{dmath}

as ``the pseudoscalar'' for a two dimensional space, and to

\begin{dmath}\label{eqn:definitions:340}
I = \Be_1 \Be_2 \Be_3,
\end{dmath}

as ``the pseudoscalar'' for a three dimensional space.



      \section{Analysis.}
         %
% Copyright © 2017 Peeter Joot.  All Rights Reserved.
% Licenced as described in the file LICENSE under the root directory of this GIT repository.
%
\paragraph{Unless otherwise stated, a Euclidean vector space with an orthonormal basis \( \setlr{\Be_1, \Be_2, \cdots } \) is assumed for the remainder of this chapter.}
Generalizations required for non-Euclidean spaces will be discussed when spacetime vectors are introduced.

         \subsection{Colinear vectors.}
            %
% Copyright © 2017 Peeter Joot.  All Rights Reserved.
% Licenced as described in the file LICENSE under the root directory of this GIT repository.
%
It was pointed out that the vector multiplication operation was not assumed to be commutative (order matters).
The only condition for which the product of two vectors is order independent, is when those vectors are colinear.

\index{commutation}
\maketheorem{Vector commutation.}{thm:multiplication:commutation}{
If \(\Bu\), and \(\Bv\) are non-zero colinear vectors, then they commute
\begin{equation*}
\Bu \Bv = \Bv \Bu.
\end{equation*}
} % theorem

The proof is simple.
Because these vectors are colinear there exists some scalar \( \alpha \) for which \( \Bv = \alpha \Bu \), so
\begin{dmath}\label{eqn:colinearVectors:380}
\Bv \Bu
=
(\alpha \Bu) \Bu
=
\alpha \Bu \Bu
=
\Bu \alpha \Bu
=
\Bu (\alpha \Bu)
=
\Bu \Bv.
\end{dmath}

The contraction axiom ensures that the product of two colinear vectors is a scalar.
In particular, the square of a unit vector, say \( \Bu \) is unity.
This should be highlighted explicitly, because this property will be used again and again
%\begin{equation}\label{eqn:colinearVectors:300}
\boxedEquation{eqn:multiplication:320}{
\Bu^2 = 1.
}
%\end{equation}

For example, the squares of any orthonormal basis vectors are unity \( (\Be_1)^2 = (\Be_2)^2 = (\Be_3)^3 = 1 \).

A corollary of
\cref{eqn:multiplication:320} is that we can factor \( 1 \) into
the square of any unit vector \( \Bu \)
\boxedEquation{eqn:multiplication:400}{
1 = \Bu \Bu.
}

This factorization trick will be used repeatedly in this book.

         \subsection{Normal vectors.}
            %
% Copyright © 2017 Peeter Joot.  All Rights Reserved.
% Licenced as described in the file LICENSE under the root directory of this GIT repository.
%
An interchange of the order of the factors of two orthogonal factors results in a change of sign,
for example \( \Be_2 \Be_1 = -\Be_1 \Be_2 \).
This is a consequence of the contraction axiom, and can be demonstrated by squaring the vector
\( \Be_1 + \Be_2 \) (\cref{fig:unitSum:unitSumFig1}).

\imageFigure{../figures/GAelectrodynamics/unitSumFig1}{\( \Be_1 + \Be_2 \).}{fig:unitSum:unitSumFig1}{0.3}
By the contraction axiom, the square of this vector is \( 2 \), so we have
\begin{dmath}\label{eqn:normalVectors:80}
2 =
(\Be_1 + \Be_2)^2
= (\Be_1 + \Be_2)(\Be_1 + \Be_2)
= \Be_1^2 + \Be_2^2 + \Be_2 \Be_1 + \Be_1 \Be_2.
= 2 + \Be_2 \Be_1 + \Be_1 \Be_2.
\end{dmath}

We conclude immediately that
\begin{dmath}\label{eqn:normalVectors:280}
\Be_2 \Be_1 + \Be_1 \Be_2 = 0,
\end{dmath}
or
%\begin{dmath}\label{eqn:normalVectors:140}
\boxedEquation{eqn:normalVectors:140}{
\Be_1 \Be_2 = -\Be_1 \Be_2.
}
%\end{dmath}

The same computation could have been performed for any two orthogonal vectors, so we conclude that any interchange of two orthogonal vectors changes the sign.

\index{anticommutation}
\maketheorem{Anticommutation}{thm:multiplication:anticommutationNormal}{
Let \(\Bu\), and \(\Bv\) be two orthogonal vectors, the product of which \( \Bu \Bv \) is a bivector.
Changing the order of these products toggles the sign of the bivector.
\begin{equation*}
\Bu \Bv = -\Bv \Bu.
\end{equation*}

This sign change on interchange is called anticommutation.  Conversely, if the product of two vectors is a bivector, those vectors are orthogonal.
} % theorem


         \subsection{2D multiplication table.}
            %
% Copyright © 2017 Peeter Joot.  All Rights Reserved.
% Licenced as described in the file LICENSE under the root directory of this GIT repository.
%
The multiplication table for the \R{2} geometric algebra can be computed with relative ease.
Many of the interesting products involve \( i = \Be_1 \Be_2 \), the unit pseudoscalar.
Using \cref{eqn:normalVectors:140} the imaginary nature of the pseudoscalar, mentioned early, can now be demonstrated explicitly
\begin{dmath}\label{eqn:2dMultiplication:220}
   \lr{ \Be_1 \Be_2 }^2
   =
   (\Be_1 \Be_2)(\Be_1 \Be_2)
   =
   -(\Be_1 \Be_2)(\Be_2 \Be_1)
   =
   -\Be_1 (\Be_2^2 ) \Be_1
   =
   -\Be_1^2
   = -1.
\end{dmath}

\index{complex imaginary}
Like the (scalar) complex imaginary, this bivector also squares to \( -1 \).
The only non-trivial products left to fill in the \R{2} multiplication table are those of the unit vectors with \( i \), products that are order dependent
\begin{dmath}\label{eqn:2dMultiplication:180}
\begin{aligned}
   \Be_1 i &= \Be_1 \lr{ \Be_1 \Be_2 } \\
           &= \lr{ \Be_1 \Be_1 } \Be_2 \\
           &= \Be_2 \\
   i \Be_1 &= \lr{ \Be_1 \Be_2 } \Be_1 \\
           &= \lr{ -\Be_2 \Be_1 } \Be_1 \\
           &= -\Be_2 \lr{ \Be_1 \Be_1 } \\
           &= -\Be_2 \\
   \Be_2 i &= \Be_2 \lr{ \Be_1 \Be_2 } \\
           &= \Be_2 \lr{ -\Be_2 \Be_1 } \\
           &= -\lr{ \Be_2 \Be_2 }\Be_1 \\
           &= -\Be_1 \\
   i \Be_2 &= \lr{ \Be_1 \Be_2 } \Be_2 \\
           &= \Be_1 \lr{ \Be_2 \Be_2 } \\
           &= \Be_1.
\end{aligned}
\end{dmath}

The multiplication table for the \R{2} multivector basis can now be tabulated

%FIXME: how to reference a tcolorbox table?
% examples in http://ctan.mirrors.hoobly.com/macros/latex/contrib/tcolorbox/tcolorbox.pdf section 5.1
% requires setting up a counter variable like some of the others (theorem environments)

% various options for prettier than default table:
% https://tex.stackexchange.com/a/135421/15
% https://tex.stackexchange.com/a/298109/15
% https://tex.stackexchange.com/a/112359/15
%\captionedTable{2D Multiplication table.}{tab:2dMultiplication:10}{
%\begin{tabular}{|l||l|l|l|l|}
%\hline
%        & \( 1 \) & \( \Be_1 \) & \( \Be_2 \) & \( \Be_1 \Be_2 \) \\ \hline
%\( 1 \) & \( 1 \) & \( \Be_1 \) & \( \Be_2 \) & \( \Be_1 \Be_2 \) \\ \hline
%\( \Be_1\) & \( \Be_1 \) & \( 1 \) & \( \Be_1 \Be_2 \) & \( \Be_2 \)\\ \hline
%\( \Be_2\) & \( \Be_2 \) & \( -\Be_1 \Be_2 \) & \( 1 \) & \( -\Be_1 \)\\ \hline
%\( \Be_1 \Be_2\) & \( \Be_1 \Be_2 \) & \( -\Be_2 \) & \( \Be_1 \) & \( -1 \) \\ \hline
%\end{tabular}
%}

%\label{tab:2dMultiplication:10}
\begin{tcolorbox}[tab2,tabularx={X||Y|Y|Y|Y},title=2D Multiplication table.,boxrule=0.5pt]
        & \( 1 \) & \( \Be_1 \) & \( \Be_2 \) & \( \Be_1 \Be_2 \) \\ \hline
\( 1 \) & \( 1 \) & \( \Be_1 \) & \( \Be_2 \) & \( \Be_1 \Be_2 \) \\ \hline
\( \Be_1\) & \( \Be_1 \) & \( 1 \) & \( \Be_1 \Be_2 \) & \( \Be_2 \)\\ \hline
\( \Be_2\) & \( \Be_2 \) & \( -\Be_1 \Be_2 \) & \( 1 \) & \( -\Be_1 \)\\ \hline
\( \Be_1 \Be_2\) & \( \Be_1 \Be_2 \) & \( -\Be_2 \) & \( \Be_1 \) & \( -1 \) \\ \hline
\end{tcolorbox}

\index{pseudoscalar}
It is important to point out that the
pseudoscalar \( i \) does not commute with either basis vector, but anticommutes with both, since \( i \Be_1 = - \Be_1 i \), and \( i \Be_2 = - \Be_2 i \).
By superposition \( i \) anticommutes with any vector in the x-y plane.

More generally, if \( \Bu \) and \( \Bv \) are orthonormal, and \( \Bx \in \Span\setlr{\Bu, \Bv} \) then the bivector \( \Bu \Bv \) anticommutes with \( \Bx \), or any other vector in this plane.

%\ref{tab:2dMultiplication:10}.


         \subsection{Plane rotations.}
            %
% Copyright © 2017 Peeter Joot.  All Rights Reserved.
% Licenced as described in the file LICENSE under the root directory of this GIT repository.
%
In \R{2} many of the interesting vector products involve the unit bivector \( i = \Be_1 \Be_2 \), the \R{2} unit pseudoscalar.
It is not a coincidence that the symbol for the complex imaginary \( i \) is used for this bivector.
The square of this bivector

\begin{dmath}\label{eqn:SimpleProducts2:220}
   \lr{ \Be_1 \Be_2 }^2
   =
   \Be_1 \Be_2
   \Be_1 \Be_2
   =
   \Be_1 \lr{ \Be_2 \Be_1 } \Be_2
   =
   \Be_1 \lr{ -\Be_1 \Be_2 } \Be_2
   =
   -\lr{ \Be_1 \Be_1 }
   \lr{ \Be_2 \Be_2 }
   = -1,
\end{dmath}

like the complex imaginary, also squares to \( -1 \).

In complex algebra, multiplication by \( \pm i \) will rotate a complex number \( z = x + i y \) by \( \pm \pi/2 \) radians.
Multiplying an \R{2} vector by \( i = \Be_1 \Be_2 \) also produces \( \pi/2 \) rotations, however the rotation direction depends on whether left of right multiplication is used.
Computing the left and right products of \( i \) with the \R{2} basis vectors provides a
simple illustration of these rotational effects

\begin{dmath}\label{eqn:SimpleProducts2:180}
\begin{aligned}
   \Be_1 i &= \Be_1 \lr{ \Be_1 \Be_2 } \\
           &= \lr{ \Be_1 \Be_1 } \Be_2 \\
           &= \Be_2 \\
   i \Be_1 &= \lr{ \Be_1 \Be_2 } \Be_1 \\
           &= \lr{ -\Be_2 \Be_1 } \Be_1 \\
           &= -\Be_2 \lr{ \Be_1 \Be_1 } \\
           &= -\Be_2 \\
   \Be_2 i &= \Be_2 \lr{ \Be_1 \Be_2 } \\
           &= \Be_2 \lr{ -\Be_2 \Be_1 } \\
           &= -\lr{ \Be_2 \Be_2 }\Be_1 \\
           &= -\Be_1 \\
   i \Be_2 &= \lr{ \Be_1 \Be_2 } \Be_2 \\
           &= \Be_1 \lr{ \Be_2 \Be_2 } \\
           &= \Be_1.
\end{aligned}
\end{dmath}

There are a number of noteworthy aspects of these calculations.

\begin{itemize}
\item The pseudoscalar \( i \) does not commute with either basis vector, but anticommutes with both, since \( i \Be_1 = - \Be_1 i \), and \( i \Be_2 = - \Be_2 i \).  By superposition \( i \) anticommutes with any vector in the plane.
\item The \( i \) products do rotate the basis vectors as claimed, which is
illustrated in \cref{fig:rotationOfe1:rotationOfe1Fig1}.
\item The products of
\cref{eqn:SimpleProducts2:220}
\cref{eqn:SimpleProducts2:180} can now be tabulated, constructing the geometric algebra multiplication table associated with the \R{2} standard basis.
\end{itemize}

\imageTwoFigures
{../figures/GAelectrodynamics/rotationOfe1Fig1}
{../figures/GAelectrodynamics/rotationOfe2Fig1}
{Multiplication by \( \Be_1 \Be_2 \).}{fig:rotationOfe1:rotationOfe1Fig1}{scale=0.5}

%\ref{tab:SimpleProducts2:10}.
%FIXME: how to reference a tcolorbox table?
% examples in http://ctan.mirrors.hoobly.com/macros/latex/contrib/tcolorbox/tcolorbox.pdf section 5.1
% requires setting up a counter variable like some of the others (theorem environments)

% various options for prettier than default table:
% https://tex.stackexchange.com/a/135421/15
% https://tex.stackexchange.com/a/298109/15
% https://tex.stackexchange.com/a/112359/15
%\captionedTable{2D Multiplication table.}{tab:SimpleProducts2:10}{
%\begin{tabular}{|l||l|l|l|l|}
%\hline
%        & \( 1 \) & \( \Be_1 \) & \( \Be_2 \) & \( \Be_1 \Be_2 \) \\ \hline
%\( 1 \) & \( 1 \) & \( \Be_1 \) & \( \Be_2 \) & \( \Be_1 \Be_2 \) \\ \hline
%\( \Be_1\) & \( \Be_1 \) & \( 1 \) & \( \Be_1 \Be_2 \) & \( \Be_2 \)\\ \hline
%\( \Be_2\) & \( \Be_2 \) & \( -\Be_1 \Be_2 \) & \( 1 \) & \( -\Be_1 \)\\ \hline
%\( \Be_1 \Be_2\) & \( \Be_1 \Be_2 \) & \( -\Be_2 \) & \( \Be_1 \) & \( -1 \) \\ \hline
%\end{tabular}
%}

%\label{tab:SimpleProducts2:10}
\begin{tcolorbox}[tab2,tabularx={X||Y|Y|Y|Y},title=2D Multiplication table.,boxrule=0.5pt]
        & \( 1 \) & \( \Be_1 \) & \( \Be_2 \) & \( \Be_1 \Be_2 \) \\ \hline
\( 1 \) & \( 1 \) & \( \Be_1 \) & \( \Be_2 \) & \( \Be_1 \Be_2 \) \\ \hline
\( \Be_1\) & \( \Be_1 \) & \( 1 \) & \( \Be_1 \Be_2 \) & \( \Be_2 \)\\ \hline
\( \Be_2\) & \( \Be_2 \) & \( -\Be_1 \Be_2 \) & \( 1 \) & \( -\Be_1 \)\\ \hline
\( \Be_1 \Be_2\) & \( \Be_1 \Be_2 \) & \( -\Be_2 \) & \( \Be_1 \) & \( -1 \) \\ \hline
\end{tcolorbox}

Given an arbitrary vector in a polar representation

\begin{dmath}\label{eqn:SimpleProducts2:280}
   \Bx = \rho \lr{ \Be_1 \cos\theta + \Be_2 \sin\theta },
\end{dmath}

left and right multiplication by the unit pseudoscalar gives

\begin{dmath}\label{eqn:SimpleProducts2:300}
\begin{aligned}
\Bx i
&= \Bx \Be_1 \Be_2 \\
&= \rho \lr{ \Be_1 \cos\theta + \Be_2 \sin\theta } \Be_1 \Be_2 \\
&= \rho \lr{ \Be_2 \cos\theta - \Be_1 \sin\theta } \\
i \Bx &= \Be_1 \Be_2 \Bx \\
&= \rho \Be_1 \Be_2 \lr{ \Be_1 \cos\theta + \Be_2 \sin\theta } \Be_1 \Be_2 \\
&= \rho \lr{ -\Be_2 \cos\theta + \Be_1 \sin\theta }.
\end{aligned}
\end{dmath}

It is left as a problem for the reader to show (using familiar methods, such as rotation matrices)
that \cref{eqn:SimpleProducts2:300} are the \( \pi/2 \) counterclockwise and clockwise rotations of \cref{eqn:SimpleProducts2:280} respectively.  These rotations are illustrated in \cref{fig:rotationOfV:rotationOfVFig1}.

\imageFigure{../figures/GAelectrodynamics/rotationOfVFig1}{\( \pi/2\) rotation using pseudoscalar multiplication.}{fig:rotationOfV:rotationOfVFig1}{0.3}

We can use Euler's formula with the \R{2} pseudoscalar representation of the complex imaginary

\begin{dmath}\label{eqn:SimpleProducts2:340}
e^{i \theta} = \cos\theta + i \sin\theta.
\end{dmath}

This can be justified by the fact that \( i = \Be_1 \Be_2 \) commutes with itself.

It is somewhat remarkable that \( \Be_1 \) can be directly factored from the
polar vector representation \cref{eqn:SimpleProducts2:280}, leaving a complex exponential.
This factorization relies on the trick mentioned earlier, utilizing a unit vector factorization of unity
\( 1 = \Be_1 \Be_1 \).  First factoring \( \Be_1 \) to the left,

\begin{dmath}\label{eqn:SimpleProducts2:940}
\Bx
=
\rho \lr{ \Be_1 \cos\theta + \Be_2 \sin\theta }
=
\rho \lr{ \Be_1 \cos\theta + (\Be_1 \Be_1) \Be_2 \sin\theta }
=
\rho \Be_1 \lr{ \cos\theta + \Be_1 \Be_2 \sin\theta }
=
\rho \Be_1 \lr{ \cos\theta + i \sin\theta }
=
\rho \Be_1 e^{i\theta},
\end{dmath}

a complex exponential (a multivector with grades 0,2) is left as a right factor.

Alternatively, by factoring \( \Be_1 \) to the right

\begin{dmath}\label{eqn:SimpleProducts2:960}
\Bx
=
\rho \lr{ \Be_1 \cos\theta + \Be_2 \sin\theta }
=
\rho \lr{ \Be_1 \cos\theta + \Be_2 (\Be_1 \Be_1) \sin\theta }
=
\rho \lr{ \cos\theta - \Be_1 \Be_2 \sin\theta } \Be_1
=
\rho \lr{ \cos\theta - i \sin\theta } \Be_1
=
\rho e^{-i\theta} \Be_1,
\end{dmath}

a complex exponential (with negative sign) is left factor.
The polar representation can therefore be expressed as either left or right complex exponential rotation of the vector \( \rho \Be_1 \).

\begin{equation}\label{eqn:SimpleProducts2:1120}
\rho \lr{ \Be_1 \cos\theta + \Be_2 \sin\theta }
= \rho e^{-i\theta} \Be_1 = \rho \Be_1 e^{i\theta}
\end{equation}

In general a positive right complex exponential multiplication (of any vector) rotates that vector counterclockwise (i.e. from \( \Be_1 \) to \( \Be_2 \)), whereas a positive left complex exponential multiplication would rotate that vector clockwise.  This is
illustrated in \cref{fig:rotationOfX:rotationOfXFig1}.
\imageFigure{../figures/GAelectrodynamics/rotationOfXFig1}{Rotation in a plane.}{fig:rotationOfX:rotationOfXFig1}{0.3}

\index{orientation}
\makedigression{Orientation}{
This is the first hint that a bivector can be thought of having a rotational sense, or orientation.  This is very similar to the orientation change that a vector undergoes by changing its sign.  As we think of vectors as oriented line segments, we will eventually come to think of bivectors as oriented plane segments, trivectors as oriented volume elements, and k-vectors as oriented hypervolumes.
}

         \subsection{Vector product, dot product and wedge product.}
            %
% Copyright © 2017 Peeter Joot.  All Rights Reserved.
% Licenced as described in the file LICENSE under the root directory of this GIT repository.
%
\index{vector product}
The product of two colinear vectors is a scalar, and the product of two normal vectors is a bivector.
The product of two general vectors is a multivector with structure to be determined.
A powerful way to examine this structure is to compute the product of two vectors in a polar representation with respect to the plane that they span.
Let \( \ucap \) and \( \vcap \) be an orthonormal pair of vectors in the plane of \( \Ba \) and \( \Bb \), oriented in a positive rotational sense as illustrated in
\cref{fig:Parallelogram:ParallelogramFig1}.
\imageFigure{../figures/GAelectrodynamics/ParallelogramFig1}{Two vectors in a plane.}{fig:Parallelogram:ParallelogramFig1}{0.3}

With respect to the plane basis \( \ucap \) and \( \vcap \), a
a polar representation of \( \Ba, \Bb \) is

\begin{dmath}\label{eqn:SimpleProducts2:1660}
\begin{aligned}
\Ba &= \Norm{\Ba} \ucap e^{ i_{ab} \theta_a } = \Norm{\Ba} e^{ -i_{ab} \theta_a } \ucap \\
\Bb &= \Norm{\Bb} \ucap e^{ i_{ab} \theta_b } = \Norm{\Bb} e^{ -i_{ab} \theta_b } \ucap,
\end{aligned}
\end{dmath}

where \( i_{ab} = \ucap \vcap \) is a unit pseudoscalar for the planar subspace spanned by \( \Ba \) and \( \Bb \).
The vector product of these two vectors is

\begin{dmath}\label{eqn:SimpleProducts2:1680}
\Ba \Bb
=
\lr{ \Norm{\Ba} e^{ -i_{ab} \theta_a } \ucap } \lr{ \Norm{\Bb} \ucap e^{ i_{ab} \theta_b } }
=
 \Norm{\Ba} \Norm{\Bb}
e^{ -i_{ab} \theta_a } ( \ucap \ucap ) e^{ i_{ab} \theta_b }
=
 \Norm{\Ba} \Norm{\Bb}
e^{ i_{ab} (\theta_b - \theta_a)}.
=
 \Norm{\Ba} \Norm{\Bb}
\lr{
\cos
(\theta_b - \theta_a)
+ i_{ab}
\sin
(\theta_b - \theta_a)
}.
\end{dmath}

We see that the product of two vectors is a multivector that has only grades 0 and 2.
This can be expressed symbolically as

\begin{dmath}\label{eqn:products:1800}
\Ba \Bb
=
\gpgradezero{ \Ba \Bb }
+
\gpgradetwo{ \Ba \Bb }.
\end{dmath}

We recognize the scalar grade of the vector product as the \R{N} dot product, but the grade 2 component of the vector product is something new that requires a name.
We respectively identify and define operators for these vector grade selection operations

\index{wedge product}
\index{dot product}
\makedefinition{Dot and wedge products of two vectors.}{dfn:products:dotandwedge}{
Given two vectors \( \Ba, \Bb \in \bbR^{N} \) the dot product is identified as the scalar grade of their product
\begin{equation*}
\gpgradezero{ \Ba \Bb }
=
\Ba \cdot \Bb
.
\end{equation*}

A wedge product of the vectors is defined as
\begin{equation*}
\Ba \wedge \Bb \equiv \gpgradetwo{ \Ba \Bb }.
\end{equation*}

Given this notation, the product of two vectors can be written
\begin{equation*}
\Ba \Bb = \Ba \cdot \Bb + \Ba \wedge \Bb.
\end{equation*}
} % definition

\index{grade selection}
Scalar grade selection of a product of two vectors is an important new tool.
There will be many circumstances where the easiest way to compute a dot product is using scalar grade selection.

The split of a vector product into dot and wedge product components is also important.
However, to utilize it, the properties of the wedge product have to be determined.
We also want to determine how exactly how the wedge product is related to the cross product, as they clearly have a similar structure.

To summarizing \cref{eqn:SimpleProducts2:1680} with our new operators, we write

\boxedEquation{eqn:SimpleProducts2:1700}{
\begin{aligned}
\Ba \Bb &= \Norm{\Ba} \Norm{\Bb} \exp\lr{ i_{ab} (\theta_b - \theta_a) } \\
\Ba \cdot \Bb &= \Norm{\Ba} \Norm{\Bb} \cos( \theta_b - \theta_a ) \\
\Ba \wedge \Bb &= i_{ab} \Norm{\Ba} \Norm{\Bb} \sin( \theta_b - \theta_a ),
\end{aligned}
}

\index{polar representation}
Two wedge product properties can be immediately deduced from this polar representation

\begin{itemize}
\item \( \Bb \wedge \Ba = - \Ba \wedge \Bb \).
\item \( \Ba \wedge (\alpha \Ba) = 0, \quad \forall \alpha \in \bbR \).
\end{itemize}

The cross product is also bilinear, so one can reasonably expect this of the wedge product.
This is much easier to demonstrate using a coordinate expansion.
To do so let

\begin{dmath}\label{eqn:SimpleProducts2:1160}
\begin{aligned}
\Ba &= \sum_i a_i \Be_i \\
\Bb &= \sum_i b_i \Be_i.
\end{aligned}
\end{dmath}

The product of these vectors is

\begin{dmath}\label{eqn:SimpleProducts2:1360}
\Ba \Bb
=
\lr{ \sum_i a_i \Be_i } \lr{ \sum_j b_j \Be_j }
=
\sum_{ij} a_i b_j \Be_i \Be_j
=
\sum_{i = j} a_i b_j \Be_i \Be_j
+
\sum_{i \ne j} a_i b_j \Be_i \Be_j
\end{dmath}

Since \( \Be_i \Be_i = 1 \), we see again that the scalar component of the product is the dot product \( \sum_i a_i b_i \).
The remaining grade 2 components are the wedge product, for which the coordinate expansion can be simplified further

\begin{dmath}\label{eqn:SimpleProducts2:1460}
\Ba \wedge \Bb
=
\sum_{i \ne j} a_i b_j \Be_i \Be_j
=
\sum_{i < j} a_i b_j \Be_i \Be_j
+
\sum_{j < i} a_i b_j \Be_i \Be_j
=
\sum_{i < j} a_i b_j \Be_i \Be_j
+
\sum_{i < j} a_j b_i \Be_j \Be_i
%=
%\sum_{i < j} a_i b_j \Be_i \Be_j
%+
%\sum_{i < j} a_j b_i (-\Be_i \Be_j)
=
\sum_{i < j} (a_i b_j - a_j b_i) \Be_i \Be_j.
\end{dmath}

\index{determinant!wedge product}
The scalar factors can be written as \( 2 x 2 \) determinants

\boxedEquation{eqn:SimpleProducts2:1320}{
\Ba \wedge \Bb
=
\sum_{i < j}
\begin{vmatrix}
a_i & a_j \\
b_i & b_j
\end{vmatrix}
\Be_i \Be_j.
}

It is now straightforward to show that the wedge product is distributive and bilinear (\cref{problem:products:bilinear}).
It is also simple to use \cref{eqn:SimpleProducts2:1320} to show that \( \Bb \wedge \Ba = -\Ba \wedge \Bb \) and \( \Ba \wedge \Ba = 0 \).

For \R{2} there is only one term in \cref{eqn:SimpleProducts2:1320}

\begin{dmath}\label{eqn:SimpleProducts2:1720}
\Ba \wedge \Bb
=
\begin{vmatrix}
a_1 & a_2 \\
b_1 & b_2
\end{vmatrix}
\Be_1 \Be_2.
\end{dmath}

\index{cross product}
We are used to writing the cross product as a \( 3 x 3 \) determinant, which can also be done with the coordinate expansion of the
\R{3} wedge product

\begin{dmath}\label{eqn:SimpleProducts2:1740}
\Ba \wedge \Bb
=
\sum_{ ij \in \setlr{ 12, 13, 23 } }
\begin{vmatrix}
a_i & a_j \\
b_i & b_j
\end{vmatrix}
\Be_i \Be_j
=
\begin{vmatrix}
\Be_2 \Be_3 & \Be_3 \Be_1 & \Be_1 \Be_2 \\
a_1 & a_2 & a_3 \\
b_1 & b_2 & b_3 \\
\end{vmatrix}.
\end{dmath}

Let's summarize the wedge product properties and relations we have found so far, comparing the \R{3} wedge product to the cross product

\begin{tcolorbox}[tab2,tabularx={X||Y|Y},title=Cross product and \R{3} wedge product comparison.,boxrule=0.5pt]
Property & Cross product & Wedge product
\\ \hline
Same vectors & \( \Ba \cross \Ba = 0 \) & \( \Ba \wedge \Ba = 0 \)
\\ \hline
Antisymmetry & \( \Bb \cross \Ba = -\Ba \cross \Bb \) & \( \Bb \wedge \Ba = -\Ba \wedge \Bb \)
\\ \hline
Linear & \( \Ba \cross (\alpha \Bb) = \alpha (\Ba \cross \Bb) \) &
\( \Ba \wedge (\alpha \Bb) = \alpha (\Ba \wedge \Bb) \)
\\ \hline
Distributive
& \( \Ba \cross (\Bb + \Bc) = \Ba \cross \Bb + \Ba \cross \Bc \)
& \( \Ba \wedge (\Bb + \Bc) = \Ba \wedge \Bb + \Ba \wedge \Bc \)
\\ \hline
Determinant expansion
&
\(
\Ba \cross \Bb
=
\begin{vmatrix}
\Be_1 & \Be_2 & \Be_3 \\
a_1 & a_2 & a_3 \\
b_1 & b_2 & b_3 \\
\end{vmatrix}
\)
&
\(
\Ba \wedge \Bb
=
\begin{vmatrix}
\Be_2 \Be_3 & \Be_3 \Be_1 & \Be_1 \Be_2 \\
a_1 & a_2 & a_3 \\
b_1 & b_2 & b_3 \\
\end{vmatrix}
\)
\\ \hline
Polar form &
\( \ncap_{ab} \Norm{\Ba} \Norm{\Bb} \sin( \theta_b - \theta_a )  \) &
\( i_{ab} \Norm{\Ba} \Norm{\Bb} \sin( \theta_b - \theta_a )  \)
\\ \hline
\end{tcolorbox}

All the wedge properties except the determinant expansion above are valid in any dimension.
It is reasonable to guess that the \R{3} wedge product is related to the cross product by some constant multivector factor \( i_{ab} = A \ncap_{ab} \).
In coordinate form, this requires a simultaneous solution to

\begin{dmath}\label{eqn:SimpleProducts2:1580}
\begin{aligned}
\Be_2 \Be_3 &= A \Be_1 \\
\Be_3 \Be_1 &= A \Be_2 \\
\Be_1 \Be_2 &= A \Be_3.
\end{aligned}
\end{dmath}

Multiplying on the right by \( \Be_1, \Be_2, \Be_3 \) respectively, this factor seems to be

\begin{equation}\label{eqn:SimpleProducts2:1600}
A = \Be_2 \Be_3 \Be_1 = \Be_3 \Be_1 \Be_2 = \Be_1 \Be_2 \Be_3,
\end{equation}

which are all permutations of the \R{3} unit pseudoscalar \( I = \Be_1 \Be_2 \Be_3 \).
This indicates that the cyclic permutations of the \R{3} pseudoscalar must all be identical (\cref{problem:SimpleProducts2:permutationspseudoscalar}).

We now have a coordinate free relationship for the \R{3} wedge product and the cross product

\boxedEquation{eqn:SimpleProducts2:1620}{
\Ba \wedge \Bb = I ( \Ba \cross \Bb ),
}

and can also express the
\R{3} vector product as a multivector combination of the dot and cross products

\boxedEquation{eqn:SimpleProducts2:1640}{
\Ba \Bb = \Ba \cdot \Bb + I(\Ba \cross \Bb).
}

This is a very important relationship, and will the key that allows
all the separate scalar and vector Maxwell's equations to be assembled into a single multivector equation.

\index{colinear vectors!wedge}
\makeproblem{Wedge product of colinear vectors.}{problem:SimpleProducts2:wedgecolinear}{
Given \( \Bb = \alpha \Ba \), use
\cref{eqn:SimpleProducts2:1320} to show that the wedge product of any pair of colinear vectors is zero.
} % problem

\makeproblem{Wedge product antisymmetry.}{problem:SimpleProducts2:1560}{
Prove that the wedge product is antisymmetric using using \cref{eqn:SimpleProducts2:1320}.
} % problem

\makeproblem{Permutations of the \R{3} pseudoscalar}{problem:SimpleProducts2:permutationspseudoscalar}{
Show that each of the permutations of
\cref{eqn:SimpleProducts2:1600} are all equal.
} % problem

\makeproblem{Wedge product distributivity and linearity.}{problem:products:bilinear}{
For vectors \( \Ba, \Bb, \Bc \) and \( \Bd \), and scalars \( \alpha, \beta \) use
\cref{eqn:SimpleProducts2:1320} to show that

\makesubproblem{}{problem:products:bilinear:a}

the wedge product is distributive
\begin{equation*}
(\Ba + \Bb) \wedge (\Bc + \Bd) =
\Ba \wedge \Bc
+
\Ba \wedge \Bd
+
\Bb \wedge \Bc
+
\Bb \wedge \Bd.
\end{equation*}

\makesubproblem{}{problem:products:bilinear:b}

and show that the wedge product is bilinear
\begin{equation*}
(\alpha \Ba) \wedge (\beta \Bb)
=
(\alpha \beta) (\Ba \wedge \Bb).
\end{equation*}

Note that these imply the wedge product also has the cross product filtering property \( \Ba \wedge (\Bb + \alpha \Ba) = \Ba \wedge \Bb \).
} % problem

%Answer (partial)
%Swapping \( \Ba \) and \( \Bb \),
%
%\begin{dmath}\label{eqn:products:1820}
%\Bb \wedge \Ba
%=
%\sum_{i < j}
%\begin{vmatrix}
%b_i & b_j \\
%a_i & a_j
%\end{vmatrix}
%\Be_i \Be_j
%=
%-\sum_{i < j}
%\begin{vmatrix}
%a_i & a_j \\
%b_i & b_j \\
%\end{vmatrix}
%\Be_i \Be_j
%=
%-\Ba \wedge \Bb,
%\end{dmath}
%
%proves that
%\cref{eqn:products:1780} holds for any vectors, and not just when they are normal.
%Because of this antisymmetry we also have
%
%\begin{dmath}\label{eqn:products:1840}
%\Ba \wedge \Ba = 0.
%\end{dmath}
%

         \subsection{Reverse.}
            %
% Copyright © 2017 Peeter Joot.  All Rights Reserved.
% Licenced as described in the file LICENSE under the root directory of this GIT repository.
%
\index{reverse}
%
% Copyright � 2016 Peeter Joot.  All Rights Reserved.
% Licenced as described in the file LICENSE under the root directory of this GIT repository.
%
\index{reverse}
\makedefinition{Reverse}{dfn:reverse:1}{

Let \( A \) be a multivector with j multivector factors,
\( A = B_1 B_2 \cdots B_j \),
not necessarily normal.
The reverse \( A^\dagger \), or reversion, of this multivector \( A \) is
\begin{equation*}
A^\dagger = B_j^\dagger B_{j-1}^\dagger \cdots B_1^\dagger.
\end{equation*}
Scalars and vectors are their own reverse, and
the reverse of a sum of multivectors is the sum of the reversions of its summands.
} % definition

Examples:
\begin{dmath}\label{eqn:reverseDefined:21}
\begin{aligned}
\lr{ 1 + 2 \Be_{12} + 3 \Be_{321} }^\dagger &= 1 + 2 \Be_{21} + 3 \Be_{123} \\
\lr{ (1 + \Be_1)(\Be_{23} - \Be_{12} }^\dagger &= (\Be_{32} + \Be_{12})(1 + \Be_1).
\end{aligned}
\end{dmath}


A useful relation for k-vectors that are composed entirely of products of normal vectors exists.  We call such k-vectors blades

\index{blade}
\index{grade}
\makedefinition{Blade.}{def:multiplication:blade}{
A product of \( k \) normal vectors is called a k-blade, or a blade of grade \( k \).
A grade zero blade is a scalar.

The notation \( F \in \bigwedge^k \) is used in the literature to indicate that \( F \) is a blade of grade \( k \).
}

Any k-blade is also a k-vector, but not all k-vectors are k-blades.  For example in \R{4} the bivector
\begin{dmath}\label{eqn:reverse:3}
\Be_{12} + \Be_{34},
\end{dmath}
is not a 2-blade, since it cannot be factored into normal products.
This will be relevant when formulating rotations since bivectors that are blades can be used to simply describe rotations or Lorentz boosts
\footnote{A rotation in spacetime.} whereas it is not easily possible to compute an exponential of a non-blade bivector argument.

The reverse of a blade is

\maketheorem{Reverse of k-blade.}{thm:reverse:kBlade}{
The reverse of a k-blade \( A_k = \Ba_1 \Ba_2 \cdots \Ba_k \) is given by
\begin{equation*}
A_k^\dagger = (-1)^{k(k-1)/2} A_k.
\end{equation*}
} % theorem

This can be proven by successive interchange of the factors
\begin{dmath}\label{eqn:reverse:81}
\begin{aligned}
A_k^\dagger
&= \Ba_k \Ba_{k-1} \cdots \Ba_1 \\
&= (-1)^{k-1} \Ba_1 \Ba_k \Ba_{k-1} \cdots \Ba_2 \\
&= (-1)^{k-1} (-1)^{k-2} \Ba_1 \Ba_2 \Ba_k \Ba_{k-1} \cdots \Ba_3 \\
&\qquad \vdots \\
&= (-1)^{k-1} (-1)^{k-2} \cdots (-1)^1 \Ba_1 \Ba_2 \cdots \Ba_k,
&= (-1)^{k(k-1)/2} \Ba_1 \Ba_2 \cdots \Ba_k. \qquad \qedmarker
\end{aligned}
\end{dmath}

A special, but important case, is the reverse of the \R{3} pseudoscalar, which is negated by reversion
\boxedEquation{eqn:reverse:103}{
I^\dagger = -I.
}


         \subsection{Imaginary nature of the \R{3} pseudoscalar.}
            % depends on reverse
            %
% Copyright © 2016 Peeter Joot.  All Rights Reserved.
% Licenced as described in the file LICENSE under the root directory of this GIT repository.
%

\index{complex imaginary}
\index{pseudoscalar}

Using the reversion operation it is simple to show that the \R{3} pseudoscalar
behaves like a complex imaginary with \( I^2 = -1 \)
\begin{dmath}\label{eqn:R3PseudoscalarSquare:3310}
I^2
=
I (-I^\dagger)
=
-
(\Be_1 \Be_2 \Be_3)(\Be_3 \Be_2 \Be_1)
=
-
\Be_1 \Be_2 \Be_2 \Be_1
=
-
\Be_1 \Be_1
=
-1.
\end{dmath}

         \subsection{Multivector dot product.}
            %
% Copyright © 2017 Peeter Joot.  All Rights Reserved.
% Licenced as described in the file LICENSE under the root directory of this GIT repository.
%

In general the product of two blades is a multivector, with a selection of different grades.
For example, the product of two bivectors may have grades 0, 2, or 4

\begin{dmath}\label{eqn:generalizedDot:601}
\Be_{12} \lr{ \Be_{21} + \Be_{23} + \Be_{34} }
=
1 + \Be_{13} + \Be_{1234}.
\end{dmath}

Similarily,
the product of a vector and bivector generally has grades 1 and 3

\begin{dmath}\label{eqn:generalizedDot:621}
\Be_1 \lr{ \Be_{12} + \Be_{23} }
=
\Be_2 + \Be_{123}.
\end{dmath}

We've identified the vector dot product with scalar grade selection of their vector product, the selection of the lowest grade of their product.
This motivates the definition of a general multivector dot product

\index{multivector dot product}
\makedefinition{Multivector dot product}{dfn:generalizedDot:100}{
The dot (or inner) product of two multivectors
\( A = \sum_{i = 0}^N \gpgrade{A}{i}, B = \sum_{i = 0}^N \gpgrade{B}{i} \)
is defined as
\begin{equation*}
A \cdot B \equiv
\sum_{i,j = 0}^N \gpgrade{ A_i B_j }{\Abs{i - j}}.
\end{equation*}
}

If \( A, B \) are k-vectors with equal grade, then the dot product is just the scalar selection of their product

\begin{dmath}\label{eqn:generalizedDot:580}
A \cdot B = \gpgradezero{ A B },
\end{dmath}

and if \( A, B \) are a k-vectors with grades \( i \ne j \) respectively, then their dot product is a single grade selection

\begin{dmath}\label{eqn:generalizedDot:581}
A \cdot B = \gpgrade{ A B }{\Abs{i - j}}.
\end{dmath}

\subsubsection{Dot product of a vector and bivector}

An important example of the generalized dot product is the dot product of a vector and bivector.
Unlike the dot product of two vectors, a vector-bivector dot product is order dependent.

The vector dot product is zero when the two vectors are normal.  This is also true if the vector and bivector are normal, that is, having no common factor, as in

\begin{equation}\label{eqn:generalizedDot:661}
\Be_1 \cdot \Be_{12} = \gpgradeone{ \Be_{123} } = 0.
\end{equation}

On the other hand, a non-zero vector-bivector dot product requires the vector to have some overlap with the bivector.
A bivector formed from the product of two normal vectors \( B = \Ba \Bb, \, \Ba \cdot \Bb = 0 \), will have a non-zero dot product with any vector that lies in \( \Span \setlr{ \Ba, \Bb} \)

\begin{dmath}\label{eqn:generalizedDot:681}
\lr{ \alpha \Ba + \beta \Bb } \cdot (\Ba\Bb)
=
\alpha \Norm{\Ba}^2 \Bb - \beta \Norm{\Bb}^2 \Ba.
\end{dmath}

There is a direct relationship between the
dot product of a vector and a 2-blade (i.e. wedge of two vectors) with a familiar \R{3} vector algebra result.  That expansion is

\maketheorem{Dot product of vector and 2-blade.}{thm:generalizedDot:wedgeDotDistribution}{
The dot product of a vector and a wedge product of two vectors distributes as
\begin{equation*}
\Ba \cdot \lr{ \Bb \wedge \Bc }
=
\lr{ \Bc \wedge \Bb } \cdot \Ba
=
( \Ba \cdot \Bb ) \Bc
-( \Ba \cdot \Bc ) \Bb.
\end{equation*}

For vectors in \R{3}, this dot product can be expressed as a triple cross product
\begin{equation*}
\Ba \cdot \lr{ \Bb \wedge \Bc }
=
\lr{ \Bb \cross \Bc } \cross \Ba.
\end{equation*}
} % theorem

There are (somewhat tricky) coordinate free ways to prove the distribution identity of this theorem, but a dumber simple expansion in coordinates also does the job.

\begin{dmath}\label{eqn:gradeselectionProblems:681}
\begin{aligned}
\Ba \cdot \lr{ \Bb \wedge \Bc } &= \sum_{i, j, k} a_i b_j c_k \Be_i \cdot (\Be_j \wedge \Be_k) \\
\lr{ \Bc \wedge \Bb } \cdot \Ba &= \sum_{i, j, k} a_i b_j c_k (\Be_k \wedge \Be_j) \cdot \Be_i
\end{aligned}
\end{dmath}

Since only \( j \ne k \) terms need be considered, either of these dot products can be written as a grade selection operation

\begin{equation}\label{eqn:generalizedDot:n}
\begin{aligned}
\Be_i \cdot (\Be_j \wedge \Be_k) 
&= \gpgradeone{ \Be_i \Be_j \Be_k } \\
&= \gpgradeone{ \Be_k \Be_j \Be_i } \\
&=
(\Be_k \wedge \Be_j) \cdot \Be_i.
\end{aligned}
\end{equation}

If all of \( i, j, k \) are unique then \( \gpgradeone{ \Be_i \Be_j \Be_k } = 0 \), so it is non-zero only when \( i \) equals one of \( j, k \).  For example

\begin{dmath}\label{eqn:generalizedDot:n}
\begin{aligned}
\gpgradeone{ \Be_1 \Be_1 \Be_2 } &= \Be_2 \\
\gpgradeone{ \Be_1 \Be_2 \Be_1 } &= -\Be_1.
\end{aligned}
\end{dmath}

In general
(\cref{problem:generalizedDot:distributionUnitVectors}), this grade selection can be expanded as 

\begin{equation}\label{eqn:generalizedDot:980}
\gpgradeone{ \Be_i \Be_j \Be_k } \\
=
\Be_k \lr{ \Be_j \cdot \Be_i }
-\Be_j \lr{ \Be_k \cdot \Be_i },
\end{equation}

so

\begin{dmath}\label{eqn:generalizedDot:n}
\Ba \cdot \lr{ \Bb \wedge \Bc } 
= \sum_{i, j, k} a_i b_j c_k \lr{ \Be_k \lr{ \Be_j \cdot \Be_i }
-\Be_j \lr{ \Be_k \cdot \Be_i } }
=
\lr{ \Ba \cdot \Bb } \Bc
- \lr{ \Ba \cdot \Bc } \Bb,
\end{dmath}

proving the distribution part of the theorem.

The \R{3} triple cross product part of this theorem, can be proven by repeated use of \( \Ba \wedge \Bb = I ( \Ba \cross \Bb ) \) in the expansion of the product of the vector and bivector

\begin{dmath}\label{eqn:generalizedDot:1000}
\Ba \lr{ \Bb \wedge \Bc }
=
\Ba I \lr{ \Bb \cross \Bc }
=
I \lr{ \Ba \cdot \lr{ \Bb \cross \Bc } }
+
I \Ba \wedge \lr{ \Bb \cross \Bc }
=
I \lr{ \Ba \cdot \lr{ \Bb \cross \Bc } }
+
I^2 \Ba \cross \lr{ \Bb \cross \Bc }
=
I \lr{ \Ba \cdot \lr{ \Bb \cross \Bc } }
+
\lr{ \Bb \cross \Bc } \cross \Ba.
\end{dmath}

This multivector has a pseudoscalar (grade 3) component, and a vector components.
The triple cross product identity follows since the
vector-bivector dot product selects just the vector component.

            \subsubsection{Problems.}
               %
% Copyright © 2017 Peeter Joot.  All Rights Reserved.
% Licenced as described in the file LICENSE under the root directory of this GIT repository.
%
\makeproblem{Index permuation in vector selection.}{problem:generalizedDot:distributionUnitVectorsa}{
Prove \cref{eqn:generalizedDot:1020}.
} % problem

\makeproblem{Dot product of unit vector with unit bivector.}{problem:generalizedDot:distributionUnitVectorsb}{
Prove \cref{eqn:generalizedDot:980}.
} % problem

         \subsection{Permutation within scalar selection.}
            As scalar selection is at the heart of the
generalized dot product, it is worth knowing
some of the ways that such a selection operation can be manipulated.

\maketheorem{Permutation of multivector products in scalar selection.}{theorem:scalarPermutation:1}{
The factors within a scalar grade selection of a pair of multivector products may be permuted or may be cyclically permuted
\begin{equation*}
\begin{aligned}
\gpgradezero{A B} &= \gpgradezero{B A} \\
\gpgradezero{A B \cdots Y Z} &= \gpgradezero{Z A B \cdots Y}.
\end{aligned}
\end{equation*}
} % problem

It is suffient to prove just the two multivector permutation case.
One simple, but inelegant method, is to first expand the pair of multivectors in coordinates.  Let

\begin{equation}\label{eqn:scalarSelectionPermutation:40}
\begin{aligned}
A &= a_0 + \sum_i a_i \Be_i + \sum_{i < j} a_{ij} \Be_{ij} + \cdots \\
B &= b_0 + \sum_i b_i \Be_i + \sum_{i < j} b_{ij} \Be_{ij} + \cdots
\end{aligned}
\end{equation}

Only the products of like unit blades can contribute scalar components to the sum, so the
the scalar selection of the products must have the form

\begin{dmath}\label{eqn:scalarSelectionPermutation:60}
\gpgradezero{ A B }
=
a_0 b_0 + \sum_i a_i b_i \Be_i^2 + \sum_{i < j} a_{ij} b_{ij} \Be_{ij}^2 + \cdots
\end{dmath}

This sum is also clearly equal to \( \gpgradezero{ B A } \), completing the proof.

         \subsection{Multivector wedge product.}
            %
% Copyright © 2017 Peeter Joot.  All Rights Reserved.
% Licenced as described in the file LICENSE under the root directory of this GIT repository.
%
We've identified the vector wedge product of two vectors with the selection of the highest grade of their product.
Looking back to the multivector products of \cref{eqn:generalizedDot:601}, and \cref{eqn:generalizedDot:621} as motivation,
a generalized wedge product can be defined that selects the highest grade terms of a given multivector product

\index{multivector wedge product}
\makedefinition{Multivector wedge product.}{dfn:generalizedWedge:480}{
For the multivectors \( A, B \) defined in \cref{dfn:generalizedDot:100}, the wedge (or outer) product is defined as
\begin{equation*}
A \wedge B
\equiv
\sum_{i,j = 0}^N \gpgrade{ A_i B_j }{i + j}.
\end{equation*}
} % definition

If \( A, B \) are a k-vectors with grades \( r, s \) respectively, then their wedge product is a single grade selection

\begin{dmath}\label{eqn:generalizedWedge:560}
A \wedge B = \gpgrade{ A B }{ r + s}.
\end{dmath}

The most important example of the generalized wedge is the wedge product of a vector with wedge of two vectors

\maketheorem{Wedge of three vectors.}{thm:generalizedWedge:vectorTwoBlade}{
The wedge product of three vectors is associative
\begin{equation*}
(\Ba \wedge \Bb) \wedge \Bc = \Ba \wedge (\Bb \wedge \Bc),
\end{equation*}
so can be written simply as \( \Ba \wedge \Bb \wedge \Bc \).
} % theorem

The proof follows directly from the definition

\begin{dmath}\label{eqn:generalizedWedge:580}
(\Ba \wedge \Bb) \wedge \Bc
=
\gpgradethree{ (\Ba \wedge \Bb) \Bc }
=
\gpgradethree{ (\Ba \Bb -\Ba \cdot \Bb) \Bc }
=
\gpgradethree{ \Ba \Bb \Bc }
-
(\Ba \cdot \Bb) \gpgradethree{ \Bc }
=
\gpgradethree{ \Ba \Bb \Bc },
\end{dmath}

where the grade-3 selection of a vector is zero by definition.
Similarly

\begin{dmath}\label{eqn:generalizedWedge:600}
\Ba \wedge (\Bb \wedge \Bc)
=
\gpgradethree{ \Ba (\Bb \wedge \Bc) }
=
\gpgradethree{ \Ba (\Bb \Bc - \Bb \cdot \Bc) }
=
\gpgradethree{ \Ba \Bb \Bc }
- (\Bb \cdot \Bc) \gpgradethree{ \Ba }
=
\gpgradethree{ \Ba \Bb \Bc },
\end{dmath}

which proves the theorem.
It is simple to show that the wedge of three vectors is completely antisymmetric (any interchange of vectors changes the sign), and that cyclic permutation \( \Ba \rightarrow \Bb \rightarrow \Bc \rightarrow \Ba \) of the vectors leaves it unchanged
(\cref{problem:generalizedWedge:tripleWedgeProperties}).
These properties are also common to the triple product of \R{3} vector algebra, a fact that is associated with the fact that there is also a determinant structure to the triple wedge product, which can be shown by direct expansion in coordinates

\begin{dmath}\label{eqn:generalizedWedge:620}
\Ba \wedge \Bb \wedge \Bc
=
\gpgradethree{ a_i b_j c_k \Be_i \Be_j \Be_k }
=
\sum_{i \ne j \ne k}
a_i b_j c_k \Be_i \Be_j \Be_k
=
\sum_{i < j < k}
\begin{vmatrix}
a_i & a_j & a_k \\
b_i & b_j & b_k \\
c_i & c_j & c_k \\
\end{vmatrix}
\Be_{i j k}.
\end{dmath}

This shows that the \R{3} wedge of three vectors is triple product times the pseudoscalar

\boxedEquation{eqn:generalizedWedge:640}{
\Ba \wedge \Bb \wedge \Bc
=
\lr{ \Ba \cdot (\Bb \cross \Bc) } I.
}

Note that the wedge of \( n \) vectors is also associative.
A full proof is possible by induction, which won't be done here.
Instead, as a hint of how to proceed if desired,
consider the coordinate expansion of a trivector wedged with a vector

\begin{dmath}\label{eqn:generalizedWedge:660}
(\Ba \wedge \Bb \wedge \Bc) \wedge \Bd
=
\sum_{i \ne j \ne k, l}
\gpgrade{
a_i b_j c_k
\Be_i \Be_j \Be_k
d_l \Be_l
}{4}
=
\sum_{i \ne j \ne k \ne l}
a_i b_j c_k d_l
\Be_i \Be_j \Be_k \Be_l.
\end{dmath}

This can be rewritten with any desired grouping \( ((\Ba \wedge \Bb) \wedge \Bc) \wedge \Bd = (\Ba \wedge \Bb) \wedge ( \Bc \wedge \Bd) = \Ba \wedge (\Bb \wedge \Bc \wedge \Bd) = \cdots \).
Observe that this can also be put into a determinant form like that of
\cref{eqn:generalizedWedge:620}.
Whenever the number of vectors matches the dimension of the underlying vector space, this will be a single determinant of all the coordinates of the vectors multiplied by the unit pseudoscalar for the vector space.


            \subsubsection{Problems.}
               %
% Copyright © 2017 Peeter Joot.  All Rights Reserved.
% Licenced as described in the file LICENSE under the root directory of this GIT repository.
%
\makeproblem{Properties of the wedge of three vectors.}{problem:generalizedWedge:tripleWedgeProperties}{
\makesubproblem{}{problem:generalizedWedge:tripleWedgeProperties:a}
Show that the wedge product of three vectors is completely antisymmetric.
\makesubproblem{}{problem:generalizedWedge:tripleWedgeProperties:b}
Show that the
wedge product of three vectors \( \Ba \wedge \Bb \wedge \Bc \) is invariant with respect to cyclic permutation.
} % problem

               %
% Copyright © 2016 Peeter Joot.  All Rights Reserved.
% Licenced as described in the file LICENSE under the root directory of this GIT repository.
%

\makeproblem{\R{4} wedge of a non-blade with itself.}{problem:gradeselection:r4nonzerobivectorwedgewithself}{
While the wedge product of a blade with itself is always zero, this is not generally true of the wedge products of arbitrary k-vectors in higher dimensional spaces.
To demonstrate this, show that the wedge of the bivector 
\( B = \Be_1 \Be_2 + \Be_3 \Be_4 \) with itself is non-zero.
Why is this bivector not a blade?
%, show that \( B \wedge B \ne 0 \).
} % problem

         \subsection{Duality.}
           %
% Copyright © 2017 Peeter Joot.  All Rights Reserved.
% Licenced as described in the file LICENSE under the root directory of this GIT repository.
%

\index{dual}
\makedefinition{Dual}{dfn:definitions:dual}{
The dual of a multivector is the product of that multivector with a pseudoscalar for a subspace that contains the multivector.  Such multiplication is referred to as a duality transformation, and can often be interpreted as an operation that produces a normal.
} % definition

The dual vectors to the \R{2} basis vectors are those same vectors rotated by \( \pi/2 \)

\begin{dmath}\label{eqn:definitions:360}
\begin{aligned}
\Be_1 \Be_{12} &= \Be_2 \\
\Be_2 \Be_{12} &= -\Be_1,
\end{aligned}
\end{dmath}

with an inverse duality transformation given by the multiplication with \( \Be_{12}^{-1} = \Be_{21} \)

\begin{dmath}\label{eqn:definitions:440}
\begin{aligned}
\Be_2 \Be_{21} &= \Be_1 \\
-\Be_1 \Be_{21} &= \Be_2.
\end{aligned}
\end{dmath}

The \R{3} duals to the basis vectors are bivectors

\begin{dmath}\label{eqn:definitions:380}
\begin{aligned}
\Be_1 \Be_{123} &= \Be_{23} \\
\Be_2 \Be_{123} &= \Be_{31} \\
\Be_3 \Be_{123} &= \Be_{12},
\end{aligned}
\end{dmath}

whereas the duals to those bivectors with respect to the pseudoscalar \( I^{-1} = \Be_{321} \) are the original basis vectors

\begin{dmath}\label{eqn:definitions:400}
\begin{aligned}
\Be_{23} \Be_{321} &= \Be_1 \\
\Be_{31} \Be_{321} &= \Be_2 \\
\Be_{12} \Be_{321} &= \Be_3.
\end{aligned}
\end{dmath}

In a sense that can be defined more precisely once the general dot product operator is defined, the dual to a given blade represents an object that is normal to the original blade.

The dual of any scalar is a pseudoscalar, whereas the dual of a pseudoscalar is a scalar.

A duality transformation can also be applied to multivectors.
For example in \R{2}, given \( M = 1 + i\), its dual is

\begin{dmath}\label{eqn:dual:460}
M i = i - 1.
\end{dmath}

Because this particular multivector had a complex structure the duality operation can be interpreted as a rotation of a vector.
How to geometrically interpret the duality transformation of a general multivector is not obvious.

%When working with multivector integrals it will be useful to consider the differential volume element a volume weighted pseudoscalar.

         \subsection{Projection and rejection.}
            
(cut)
The pythagorean property of these two vector components can also be checked.
Computing the squared length using \( \Norm{\By}^2 = \By \cdot \By = \By^2 \), the squared length of the projective component is

\begin{dmath}\label{eqn:SimpleProducts2:740}
\lr{ \lr{\Bx \cdot \ucap } \ucap }^2
=
\lr{\Bx \cdot \ucap }^2
=
(x_1 u_1 + x_2 u_2)^2
=
x_1^2 u_1^2 + x_2^2 u_2^2 + 2 x_1 x_2 u_1 u_2.
\end{dmath}

The squared length of the rejective component is
\begin{dmath}\label{eqn:SimpleProducts2:760}
\lr{ \lr{\Bx \wedge \ucap } \ucap }^2
=
-(\Bx \wedge \ucap) \ucap^2 (\Bx \wedge \ucap)
=
-
\lr{\begin{vmatrix}
   x_1 & x_2 \\
   u_1 & u_2
\end{vmatrix}}^2
(\Be_1 \Be_2)^2
=
x_1^2 u_2^2 + x_2^2 u_1^2 - 2 x_1 x_2 u_1 u_2.
\end{dmath}

Adding these together gives

\begin{dmath}\label{eqn:SimpleProducts2:780}
\lr{ \lr{\Bx \cdot \ucap } \ucap }^2 + \lr{ \lr{\Bx \wedge \ucap } \ucap }^2
=
x_1^2 u_1^2 + x_2^2 u_2^2
+x_1^2 u_2^2 + x_2^2 u_1^2
=
x_1^2 ( u_1^2 + u_2^2 )
+
x_2^2 ( u_1^2 + u_2^2 )
=
\Bx^2,
\end{dmath}

recovering the squared length of the vector as expected.
It is generally true in higher dimensions that the projection and rejection can be written as

\begin{dmath}\label{eqn:SimpleProducts2:800}
\begin{aligned}
\Proj_\ucap(\Bx) &= (\Bx \cdot \ucap) \ucap \\
\RejName_\ucap(\Bx) &= (\Bx \wedge \ucap) \ucap.
\end{aligned}
\end{dmath}

The Pythagorean aspect of this statement in higher degree spaces
will be demonstrated later in a coordinate free fashion after some additional identities have been derived.

The unit vector restriction defining the direction of projection and rejection can be relaxed in a compact fashion by introducing the vector \boldTextAndIndex{inverse}, which is always well defined and unique in a Euclidean space

\boxedEquation{eqn:SimpleProducts2:860}{
\inv{\Bu} \equiv \frac{\Bu}{\Bu^2}.
}

Now the projection and rejection onto the direction of \( \Bu \) are

\boxedEquation{eqn:SimpleProducts2:880}{
\begin{aligned}
\Proj_\Bu(\Bx) &= (\Bx \cdot \Bu) \inv{\Bu} \\
\RejName_\Bu(\Bx) &= (\Bx \wedge \Bu) \inv{\Bu}.
\end{aligned}
}

%\makelemma{\R{3} pseudoscalar commutation.}{dfn:projectionAndRejection:r3pcommutation}{
%The \R{3} pseudoscalar \( I = \Be_1 \Be_2 \Be_3 \) commutes with all \R{3} multivectors.
%} % lemma
%
%To prove this, it is sufficient to consider the commutation of \( I \) with each of the standard basis vectors \( \Be_1, \Be_2, \Be_3 \) (\cref{problem:projectionAndRejection:1160}).
%
Now the cross product form of the rejection equation can be determined

         \subsection{Normal factorization of the wedge product.}
            %
% Copyright © 2017 Peeter Joot.  All Rights Reserved.
% Licenced as described in the file LICENSE under the root directory of this GIT repository.
%
A general bivector has the form

\begin{dmath}\label{eqn:normalFactorization:1800}
B = \sum_{i \ne j} a_{ij} \Be_{ij},
\end{dmath}
which is not necessarily a blade.
%For example the bivector \( \Be_1 \Be_2 + \Be_3 \Be_4 \) cannot be factored into any product of normal vectors.
On the other hand, a wedge product is always a blade
\footnote{In \R{3} any bivector is also a blade \citep{ablamowicz2004lectures:chapter1}}

\index{wedge factorization}
\maketheorem{Wedge product normal factorization}{thm:SimpleProducts2:wnormalfactorize}{
The wedge product of any two non-colinear vectors \( \Ba, \Bb \) always has a normal (2-blade) factorization
\begin{equation*}
\Ba \wedge \Bb = \Bu \Bv, \quad \Bu \cdot \Bv = 0.
\end{equation*}
} % theorem

This can be proven by construction.
Pick \( \Bu = \Ba \) and \( \Bv = \Rej{\Ba}{\Bb} \), then

\begin{dmath}\label{eqn:normalFactorization:1840}
\Ba \Rej{\Ba}{\Bb}
=
\cancel{\Ba \cdot \Rej{\Ba}{\Bb}}
+
\Ba \wedge \Rej{\Ba}{\Bb}
=
\Ba \wedge \lr{ \Bb - \frac{\Bb \cdot \Ba}{\Norm{\Ba}^2} \Ba }
=
\Ba \wedge \Bb,
\end{dmath}
since \( \Ba \wedge (\alpha \Ba) = 0 \) for any scalar \( \alpha \).

The significance of \cref{thm:SimpleProducts2:wnormalfactorize} is that the square of any wedge product is negative

\begin{dmath}\label{eqn:normalFactorization:1820}
(\Bu \Bv)^2
=
(\Bu \Bv) (-\Bv \Bu)
=
-\Bu (\Bv^2) \Bu
=
- \Abs{\Bu}^2 \Abs{\Bv}^2,
\end{dmath}
which in turn means that exponentials with wedge product arguments can be used as rotation operators.

\makeproblem{\R{3} bivector factorization.}{problem:normalFactorization:1}{
Find some normal factorizations for the \R{3} bivector \( \Be_{12} + \Be_{23} + \Be_{31} \).
} % problem

\makeanswer{problem:normalFactorization:1}{
\begin{dmath*}
\Be_{12} +
\Be_{23} +
\Be_{31}
= \lr{ \Be_1 + \Be_2 - 2 \Be_3 } \frac{ \Be_2 - \Be_1 }{2}
= \frac{ \Be_3 - \Be_2 }{2} \lr{ 2 \Be_1 - \Be_2 - \Be_3 }.
\end{dmath*}

The respective parallelogram representations of these bivector factorizations are illustrated in
\cref{fig:bivectorFactorization:bivectorFactorizationFig1}, showing the bivectors face on, and from a side view that shows they are coplanar.

% \imageTwoFigures{path1}{path2}{fancy plots}{fig:blah}{scale=0.3}
\imageTwoFigures
{../figures/GAelectrodynamics/bivectorFactorizationFig2}
{../figures/GAelectrodynamics/bivectorFactorizationFig1}
{Two equivalent bivector factorizations.}
{fig:bivectorFactorization:bivectorFactorizationFig1}{scale=0.3}
} % answer

         \subsection{The wedge product as an oriented area.}
            %
% Copyright © 2017 Peeter Joot.  All Rights Reserved.
% Licenced as described in the file LICENSE under the root directory of this GIT repository.
%
The coordinate representation of the \R{2} wedge product (\cref{eqn:SimpleProducts2:1720}) had a single \( \Be_{12} \) bivector factor, whereas the expansion in coordinates for the general \R{N} wedge product was considerably messier (\cref{eqn:SimpleProducts2:1320}).
This difference is essentially just one of the choice of basis.

A simpler coordinate representation for the \R{N} wedge product follows by choosing an
orthonormal basis
for the planar subspace spanned by the wedge vectors.
Given vectors \( \Ba, \Bb \), let \( \setlr{\ucap, \vcap} \) be an orthonormal basis for the plane subspace
\( P = \Span\setlr{ \Ba, \Bb } \).
The coordinate representation in this basis is

\begin{dmath}\label{eqn:wedgeProductArea:1900}
\begin{aligned}
\Ba &= (\Ba \cdot \ucap) \ucap + (\Ba \cdot \vcap) \vcap \\
\Bb &= (\Bb \cdot \ucap) \ucap + (\Bb \cdot \vcap) \vcap.
\end{aligned}
\end{dmath}

Wedging these vectors using a grade 2 selection gives

\begin{dmath}\label{eqn:SimpleProducts2:1860}
\Ba \wedge \Bb
=
\gpgradetwo{
   \lr{
   (\Ba \cdot \ucap) \ucap + (\Ba \cdot \vcap) \vcap
   }
   \lr{
   (\Bb \cdot \ucap) \ucap + (\Bb \cdot \vcap) \vcap
   }
}
=
\gpgradetwo{
\cancel{
   (\Ba \cdot \ucap) (\Bb \cdot \ucap) \ucap^2
}
+
\cancel{
   (\Ba \cdot \vcap) (\Bb \cdot \vcap) \vcap^2
}
+
\lr{
      (\Ba \cdot \ucap)
   (\Bb \cdot \vcap)
   -
   (\Ba \cdot \vcap) (\Bb \cdot \ucap)
}
\ucap \vcap
}
=
\lr{
      (\Ba \cdot \ucap)
   (\Bb \cdot \vcap)
   -
   (\Ba \cdot \vcap) (\Bb \cdot \ucap)
}
\ucap \vcap.
\end{dmath}

Such a basis allows for the most compact (single term) coordinate representation of the wedge product

\begin{dmath}\label{eqn:SimpleProducts2:1880}
\Ba \wedge \Bb
=
\begin{vmatrix}
   \Ba \cdot \ucap & \Ba \cdot \vcap \\
   \Bb \cdot \ucap & \Bb \cdot \vcap
\end{vmatrix}
\ucap \vcap.
\end{dmath}

If a counterclockwise rotation by \( \pi/2 \) takes \( \ucap \) to \( \vcap \) the determinant will equal the area of the parallelogram spanned by \( \Ba \) and \( \Bb \).  Let that area be designated

\begin{dmath}\label{eqn:wedgeProductArea:1920}
A =
\begin{vmatrix}
   \Ba \cdot \ucap & \Ba \cdot \vcap \\
   \Bb \cdot \ucap & \Bb \cdot \vcap
\end{vmatrix}.
\end{dmath}

Any number of possible wedge (or normal) product representations of the same numeric 2-blade quantity are possible

\begin{dmath}\label{eqn:wedgeProductArea:1940}
\begin{aligned}
\Ba \wedge \Bb
&= (\Ba + \beta \Bb ) \wedge \Bb \\
&= \Ba \wedge ( \Bb + \alpha \Ba ) \\
&= (A \ucap) \wedge \vcap \\
&= \ucap \wedge (A \vcap) \\
&= (\alpha A \ucap) \wedge \frac{\vcap}{\alpha} \\
&= (\beta A \ucap') \wedge \frac{\vcap'}{\beta} \\
\end{aligned}
\end{dmath}

These equivalent values can be thought of as different possible parallelogram possible geometrical representations of the same object, provided the spanned area and relative ordering of the wedged vectors remains constant.
Some different parallelogram representations of a single 2-blade are illustrated in \cref{fig:parrallelograms:parrallelogramsFig1}.

\imageFigure{../figures/GAelectrodynamics/parrallelogramsFig1}{Parallelogram representations of wedge products.}{fig:parrallelograms:parrallelogramsFig1}{0.3}

An arbitrary 2-blade need not be ``factored'' into any pair of normal vectors, nor is there any a-priori reason to represent such an object as a wedge product of two specific vectors.  In this sense there is no definitive geometry to a 2-blade, and it can be represented as any fixed area with a given cyclic orientation, as
illustrated in \cref{fig:orientedAreasVariety:orientedAreasVarietyFig1}.

\imageFigure{../figures/GAelectrodynamics/orientedAreasVarietyFig1}{Different shape representations of a given bivector.}{fig:orientedAreasVariety:orientedAreasVarietyFig1}{0.2}

\index{parallelogram}
\makeproblem{Parallelogram area.}{problem:wedgeProductArea:R2parallelogramarea}{
Show that the area \( A \) of the parallelogram spanned by vectors

\begin{equation*}
\begin{aligned}
\Ba &= a_1 \Be_1 + a_2 \Be_2 \\
\Bb &= b_1 \Be_1 + b_2 \Be_2,
\end{aligned}
\end{equation*}

is
\begin{equation*}
A =
\pm
\begin{vmatrix}
   a_1 & a_2 \\
   b_1 & b_2 \\
\end{vmatrix}
,
\end{equation*}

and that the sign is positive if the rotation angle \( \theta \) that takes \( \acap \) to \( \bcap \) is positive, \( \theta \in (0,\pi) \).
} % problem


         \subsection{General rotation.}
            %
% Copyright © 2017 Peeter Joot.  All Rights Reserved.
% Licenced as described in the file LICENSE under the root directory of this GIT repository.
%
\Cref{eqn:2dMultiplication:180} showed that the \R{2} pseudoscalar anticommutes with any vector \( \Bx \in \bbR^{2} \),
\begin{dmath}\label{eqn:generalRotation:1760}
\Bx i = -i \Bx,
\end{dmath}
and that the sign of the bivector exponential argument must be negated to maintain the value of the vector \( \Bx \in \bbR^2 \) on interchange
\begin{dmath}\label{eqn:generalRotation:1820}
\Bx e^{i\theta}
=
e^{-i\theta} \Bx.
\end{dmath}

The higher dimensional generalization of these results are

\index{commutation}
\index{conjugation}
\maketheorem{Wedge and exponential commutation and conjugation rules.}{thm:SimpleProducts2:1780}{
Given two
non-colinear vectors \( \Ba, \Bb \), let
the planar subspace formed by their span be designated
\( S = \Span \setlr{ \Ba, \Bb } \).

\begin{enumerate}[(a)]
\item
Any vector \( \Bp \in S \) anticommutes with the wedge product \( \Ba \wedge \Bb \)
\begin{equation*}
\Bp (\Ba \wedge \Bb) = - (\Ba \wedge \Bb) \Bp.
\end{equation*}
\item
Any vector \( \Bn \) orthogonal to this plane (\( \Bn \cdot \Ba = \Bn \cdot \Bb = 0 \)) commutes with this wedge product
\begin{equation*}
\Bn (\Ba \wedge \Bb) = (\Ba \wedge \Bb) \Bn.
\end{equation*}
\item
Reversing the order of multiplication of a
vector \( \Bp \in S \) with an exponential
\( e^{ \Ba \wedge \Bb } \), requires the sign of the exponential argument to be negated
\begin{equation*}
\Bp e^{\Ba \wedge \Bb} = e^{-\Ba \wedge \Bb} \Bp.
\end{equation*}

This sign change on interchange will be called conjugation.
\item
Any orthogonal vectors \( \Bn \) commute with a such a complex exponential
\begin{equation*}
\Bn e^{\Ba \wedge \Bb} = e^{\Ba \wedge \Bb} \Bn.
\end{equation*}
\end{enumerate}
} % theorem

The proof relies on the fact that a orthogonal factorization of the wedge product is possible.
If \( \Bp \) is one of those factors, then the other is uniquely determined by the multivector equation \( \Ba \wedge \Bb = \Bp \Bq \), for which we must have \( \Bq = \inv{\Bx}(\Ba \wedge \Bb) \in S \) and \( \Bp \cdot \Bq = 0 \)
\footnote{The identities required to show that \( \Bq \) above has no trivector grades, and to evaluate it explicitly in terms of \( \Ba, \Bb, \Bx \), will be derived later.}
.
Then
\begin{dmath}\label{eqn:generalRotation:1780}
\Bp (\Ba \wedge \Bb)
= \Bp (\Bp \Bq)
= \Bp (-\Bq \Bp)
= -(\Bp \Bq) \Bp
=
-(\Ba \wedge \Bb) \Bp.
\end{dmath}

Any orthogonal vectors \( \Bn \) must also be perpendicular to the factors \( \Bp, \Bq \), with \( \Bn \cdot \Bp = \Bn \cdot \Bq = 0 \), so
\begin{dmath}\label{eqn:generalRotation:1800}
\Bn (\Ba \wedge \Bb)
= \Bn (\Bp \Bq)
= (-\Bp \Bn) \Bq
= -\Bp (-\Bq \Bn)
= (\Bp \Bq) \Bn
=
(\Ba \wedge \Bb) \Bn.
\end{dmath}

For the complex exponentials, introduce a unit pseudoscalar for the plane \( i = \pcap \qcap \) satisfying \( i^2 = -1 \) and a scalar rotation angle \( \theta = \ifrac{ (\Ba \wedge \Bb) }{i} \), then for vectors \( \Bp \in S \)
\begin{dmath}\label{eqn:generalRotation:1840}
\Bp e^{ \Ba \wedge \Bb }
=
\Bp e^{ i \theta }
=
\Bp \lr{ \cos\theta + i \sin\theta }
=
\lr{ \cos\theta - i \sin\theta } \Bp
=
e^{-i\theta} \Bp
=
e^{- \Ba \wedge \Bb} \Bp,
\end{dmath}
and for vectors \( \Bn \) orthogonal to \( S \)
\begin{dmath}\label{eqn:generalRotation:1860}
\Bn e^{ \Ba \wedge \Bb }
=
\Bn e^{ i \theta }
=
\Bn \lr{ \cos\theta + i \sin\theta }
=
\lr{ \cos\theta + i \sin\theta } \Bn
=
e^{i\theta} \Bn
=
e^{\Ba \wedge \Bb} \Bn,
\end{dmath}
which completes the proof.

The point of this somewhat abstract seeming theorem is to prepare for the statement of a general \R{N} rotation, which is

\index{rotation}
\makedefinition{General rotation}{dfn:generalRotation:generalrotation}{
Let \( B = \setlr{ \pcap, \qcap } \) be an orthonormal basis for a planar subspace with unit pseudoscalar \( i = \pcap \qcap \) where \( i^2 = -1\).
The rotation of a vector \( \Bx \) through an angle \( \theta \) with respect to this plane is
\begin{equation*}
R_\theta(\Bx) = e^{ - i \theta/2 } \Bx e^{ i\theta/2 }.
\end{equation*}

Here the rotation sense is that of the \( \pi/2 \) rotation from \( \pcap \) to \( \qcap \) in the subspace \( S = \Span B \).
} % definition

This statement did not make any mention of an orthogonal direction.
Such an orthogonal direction is not unique for dimensions higher than 3, nor defined for two dimensions.
Instead the rotational sense is defined by the ordering of the factors in the bivector \( i \).

To check that this operation has the desired semantics,
let \( \Bx = \Bx_\parallel + \Bx_\perp \), where \( \Bx_\parallel \in S \) and \( \Bx_\perp \cdot \Bp = 0 \,\forall \Bp \in S \).
Then
\begin{dmath}\label{eqn:generalRotation:1880}
R_\theta(\Bx)
=
e^{ - i \theta/2 } \Bx e^{ i\theta/2 }
=
e^{ - i \theta/2 } \lr{ \Bx_\parallel + \Bx_\perp } e^{ i\theta/2 }
=
\Bx_\parallel e^{ i\theta } +
\Bx_\perp e^{ - i \theta/2 } e^{ i\theta/2 }
=
\Bx_\parallel e^{ i\theta } + \Bx_\perp.
\end{dmath}

As desired, this rotation operation
rotates components of the vector that lie in the planar subspace \( S \) by \( \theta \), while leaving the components of the vector orthogonal to the plane unchanged, as illustrated in \cref{fig:Rotation:RotationFig1}.
This is what we can call rotation around a normal in \R{3}.

\imageFigure{../figures/GAelectrodynamics/RotationFig1}{Rotation with respect to the plane of a pseudoscalar.}{fig:Rotation:RotationFig1}{0.5}

         \subsection{Symmetric and antisymmetric vector sums.}
            
\maketheorem{Symmetric and antisymmetric vector products.}{thm:symmetricAndAntiSymmetricVectorSums:symmetricAndAnti}{
\begin{enumerate}
\item The dot product of vectors \( \Bx, \By \) can be written as
\begin{equation*}
\Bx \cdot \By = \inv{2}\lr{ \Bx \By + \By \Bx }.
\end{equation*}

This sum, including all permutations of the products of \( \Bx \) and \( \By \) is called a completely symmetric sum.
\item The wedge product of vectors \( \Bx, \By \) can be written as
\begin{equation*}
\Bx \wedge \By = \inv{2}\lr{ \Bx \By - \By \Bx }.
\end{equation*}

This sum, including all permutations of the products \( \Bx \) and \( \By \), with a sign change for any interchange, is called a completely antisymmetric sum.
\end{enumerate}
} % theorem

These identities highlight the symmetric and antisymmetric nature of the respective dot and wedge products in a coordinate free form, and will be useful in the manipulation of various identities.
The proof follows by direct compuation after first noting that the respect vector products are

\begin{subequations}
\label{eqn:symmetricAndAntiSymmetricVectorSums:660}
\begin{dmath}\label{eqn:symmetricAndAntiSymmetricVectorSums:640}
\Bx \By = \Bx \cdot \By + \Bx \wedge \By
\end{dmath}
\begin{dmath}\label{eqn:symmetricAndAntiSymmetricVectorSums:680}
\By \Bx
= \By \cdot \Bx + \By \wedge \Bx
= \Bx \cdot \By - \Bx \wedge \By.
\end{dmath}
\end{subequations}

In \cref{eqn:symmetricAndAntiSymmetricVectorSums:680} the interchange utilized the respective symmetric and antisymmetric nature of the dot and wedge products.

Adding and subtracting \cref{eqn:symmetricAndAntiSymmetricVectorSums:660} proves the result.

%Some authors will use \cref{eqn:SimpleProducts2:620} as the definitions of the dot and wedge products instead of defining them in terms of grade selection.
%Grade selection is preferred here since it allows for a generalization of the wedge product to multiple vectors in higher degree spaces in a particularily simple way, and also allows for the generalization of the dot and wedge products with higher order geometric structures to be discussed.

         \subsection{Reflection.}
            %
% Copyright © 2017 Peeter Joot.  All Rights Reserved.
% Licenced as described in the file LICENSE under the root directory of this GIT repository.
%
\index{reflection}

Geometrically the reflection of a vector \( \Bx \) across a line directed along \( \Bu \) is the difference of the projection and rejection

\begin{dmath}\label{eqn:SimpleProducts2:900}
\Bx'
= \lr{ \Bx \cdot \Bu }\Bu - \lr{ \Bx \wedge \Bu } \inv{\Bu }
= \lr{ \Bx \cdot \Bu - \Bx \wedge \Bu } \inv{\Bu }
\end{dmath}

Using the symmetric and antisymmetric sum representations of the dot and wedge products from
\cref{thm:symmetricAndAntiSymmetricVectorSums:symmetricAndAnti}
the reflection can be expressed as vector products

\begin{dmath}\label{eqn:reflection:n}
\Bx'
= \inv{2} \lr{ \cancel{\Bx \Bu} + \Bu \Bx - \cancel{\Bx \Bu} + \Bu \Bx } \inv{\Bu },
\end{dmath}

yielding a remarkably simple form in terms of vector products

\boxedEquation{eqn:SimpleProducts2:920}{
\Bx' = \Bu \Bx \inv{\Bu}.
}

An illustration of the geometry of reflection is provided in \cref{fig:reflection:reflectionFig1}.

\imageFigure{../figures/GAelectrodynamics/reflectionFig1}{Reflection}{fig:reflection:reflectionFig1}{0.3}

         \subsection{Linear systems.}
            %
% Copyright © 2017 Peeter Joot.  All Rights Reserved.
% Licenced as described in the file LICENSE under the root directory of this GIT repository.
%
\index{linear system}
\index{wedge product!linear solution}
Linear systems can be solved using the wedge product.  Illustrating by example, consider the following two variable equation in \R{N}

\begin{dmath}\label{eqn:solutionOfLinearSystem:1160}
\Ba x + \Bb y = \Bc.
\end{dmath}

To solve for \( x \) simply wedge with \( \Bb \), and to solve for \( y \) wedge with \( \Ba \)

\begin{dmath}\label{eqn:solutionOfLinearSystem:1180}
\begin{aligned}
\lr{ \Ba x + \cancel{\Bb} y } \wedge \Bb &= \Bc \wedge \Bb \\
\Ba \wedge \lr{ \cancel{\Ba} x + \Bb y } &= \Ba \wedge \Bc,
\end{aligned}
\end{dmath}

so, if the system has a solution, it is given by
\begin{dmath}\label{eqn:solutionOfLinearSystem:1200}
\begin{aligned}
x &= \inv{ \Ba \wedge \Bb } \Bc \wedge \Bb \\
y &= \inv{ \Ba \wedge \Bb } \Ba \wedge \Bc.
\end{aligned}
\end{dmath}

Higher order systems can be solved in the same fashion, but the equation to be solved must be wedged more times.

For example, if the \( n \) variable system
\begin{dmath}\label{eqn:solutionOfLinearSystem:1220}
\Ba_1 x_1 + \Ba_2 x_2 \cdots + \Ba_n x_n = \Bb,
\end{dmath}

has a solution, it is given by

\begin{dmath}\label{eqn:solutionOfLinearSystem:1240}
\begin{aligned}
x_1 &= \inv{ \Ba_1 \wedge \Ba_2 \wedge \cdots \wedge \Ba_n } \Bb \wedge \Ba_2 \wedge \cdots \wedge \Ba_n  \\
x_2 &= \inv{ \Ba_1 \wedge \Ba_2 \wedge \cdots \wedge \Ba_n } \Ba_1 \wedge \Bb \wedge \cdots \wedge \Ba_n  \\
    & \vdots \\
x_n &= \inv{ \Ba_1 \wedge \Ba_2 \wedge \cdots \wedge \Ba_n } \Ba_1 \wedge \Ba_2 \wedge \cdots \wedge \Bb.
\end{aligned}
\end{dmath}

If this system has no solution, then these n-vector ratios will not be scalars.

For the solution of \( x_k \), the numerator in each case is the denominator wedge product with the \( \Ba_k \) replaced by the solution vector \( \Bb \).
If this sounds like Cramer's rule, that is because the two are equivalent when the dimension of the vector equals the number of variables in the linear system.
\index{Cramer's rule}
For example, consider the solution for \( x_1 \) of \cref{eqn:solutionOfLinearSystem:1220} for an \R{3} system, with \( \Ba_1 = \Bu, \Ba_2 = \Bv, \Ba_3 = \Bw \)

\begin{equation}\label{eqn:solutionOfLinearSystem:1260}
x_1 =
\frac{ \Bb \wedge \Bv \wedge \Bw }
{ \Bu \wedge \Bv \wedge \Bw }
=
\frac{
\begin{vmatrix}
b_1 & v_1 & w_1 \\
b_2 & v_2 & w_2 \\
b_3 & v_3 & w_3 \\
\end{vmatrix}
\cancel{\Be_1 \Be_2 \Be_3}
}
{
\begin{vmatrix}
u_1 & v_1 & w_1 \\
u_2 & v_2 & w_2 \\
u_3 & v_3 & w_3 \\
\end{vmatrix}
\cancel{\Be_1 \Be_2 \Be_3}
},
\end{equation}

which is exactly the ratio of determinants found in the Cramer's rule solution of this problem.  We get Cramer's rule for free due to the antisymmetric structure of the wedge product.

Cramer's rule doesn't apply to cases where the dimension of the space exceeds the number of variables, but a wedge product solution does not have that restriction.  As an example, consider the two variable system \cref{eqn:solutionOfLinearSystem:1160} for vectors in \R{4} as follows

\begin{equation}\label{eqn:solutionOfLinearSystem:1280}
\Ba =
\begin{bmatrix}
1 \\
1 \\
0 \\
0
\end{bmatrix}, \qquad
\Bb =
\begin{bmatrix}
1 \\
0 \\
0 \\
1
\end{bmatrix}, \qquad
\Bb =
\begin{bmatrix}
1 \\
2 \\
0 \\
-1
\end{bmatrix}.
\end{equation}

Here's a (Mathematica) computation of the wedge products for the solution
\footnote{
Using the
%\href{https://github.com/jlaragonvera/Geometric-Algebra}{CliffordBasic.m}
CliffordBasic.m
geometric algebra module from
https://github.com/jlaragonvera/Geometric-Algebra.
}

\begin{mmaCell}[moredefined={a, b, c, iab, aWedgeB, cWedgeB, aWedgeC, x, y, e, OuterProduct, GeometricProduct}]{Input}
a = e[1] + e[2];
b = e[1] + e[4];
c = e[1] + 2 e[2] - e[4];
aWedgeB = OuterProduct[a, b]
cWedgeB = OuterProduct[c, b]
aWedgeC = OuterProduct[a, c]
iab = \mmaFrac{aWedgeB}{GeometricProduct[aWedgeB, aWedgeB]};
x = GeometricProduct[iab, cWedgeB]
y = GeometricProduct[iab, aWedgeC]
\end{mmaCell}

Which gives
\begin{dmath}\label{eqn:solutionOfLinearSystem:1300}
\begin{aligned}
\Ba \wedge \Bb &= \Be_{21} + \Be_{14} + \Be_{24} \\
\Bc \wedge \Bb &= 2 \Be_{21} + 2 \Be_{14} + 2 \Be_{24} \\
\Ba \wedge \Bc &= -\Be_{21} - \Be_{14} - \Be_{24} \\
x &= 2 \\
y &= -1.
\end{aligned}
\end{dmath}

\subsubsection{Example: intersection of two lines.}

As a concrete example, let's solve the intersection of two lines problem illustrated in \cref{fig:intersectionOfLines:intersectionOfLinesFig1}.

\imageFigure{../figures/GAelectrodynamics/intersectionOfLinesFig1}{Intersection of two lines.}{fig:intersectionOfLines:intersectionOfLinesFig1}{0.3}

In parametric form, the lines in this problem are

\begin{dmath}\label{eqn:solutionOfLinearSystem:1000}
\begin{aligned}
\Br_1(s) &= \Ba_0 + s( \Ba_1 - \Ba_2 ) \\
\Br_2(t) &= \Bb_0 + t( \Bb_1 - \Bb_2 ),
\end{aligned}
\end{dmath}

so the solution, if it exists, is found at the point satisfying the equality

\begin{dmath}\label{eqn:solutionOfLinearSystem:1020}
\Ba_0 + s( \Ba_1 - \Ba_2 ) = \Bb_0 + t( \Bb_1 - \Bb_2 ).
\end{dmath}

With
\begin{dmath}\label{eqn:solutionOfLinearSystem:1040}
\begin{aligned}
\Bu_1 &= \Ba_1 - \Ba_2 \\
\Bu_2 &= \Bb_1 - \Bb_2 \\
\Bd &= \Ba_0 - \Bb_0,
\end{aligned}
\end{dmath}

the desired equation to solve is

\begin{dmath}\label{eqn:solutionOfLinearSystem:1060}
\Bd + s \Bu_1 = t \Bu_2.
\end{dmath}

As with any linear system, we can
solve for \( s \) or \( t \) by
wedging both sides with one of \( \Bu_1 \) or \( \Bu_2 \)

\begin{dmath}\label{eqn:solutionOfLinearSystem:1080}
\begin{aligned}
\Bd \wedge \Bu_1 &= t \Bu_2 \wedge \Bu_1 \\
\Bd \wedge \Bu_2 + s \Bu_1 \wedge \Bu_2 &= 0,
\end{aligned}
\end{dmath}

In \R{2} these equations have a solution if \( \Bu_1 \wedge \Bu_2 \ne 0 \), and in \R{N} these have solutions if the bivectors on each sides of the equations describe the same plane (i.e. the bivectors on each side of \cref{eqn:solutionOfLinearSystem:1080} are related by a scalar factor).
Put another way, these have solutions when \( s \) and \( t \) are scalars with the values

\begin{dmath}\label{eqn:solutionOfLinearSystem:1100}
\begin{aligned}
s &= \frac{\Bu_2 \wedge \Bd}{\Bu_1 \wedge \Bu_2} \\
t &= \frac{\Bu_1 \wedge \Bd}{\Bu_1 \wedge \Bu_2}.
\end{aligned}
\end{dmath}

%%%In
%%%\R{2}
%%%with
%%%\begin{dmath}\label{eqn:solutionOfLinearSystem:1120}
%%%\begin{aligned}
%%%\Bu_1 &= u_{11} \Be_1 + u_{12} \Be_2 \\
%%%\Bu_2 &= u_{21} \Be_1 + u_{22} \Be_2 \\
%%%\Bd &= d_{1} \Be_1 + d_{2} \Be_2,
%%%\end{aligned}
%%%\end{dmath}
%%%
%%%the wedge products in \cref{eqn:solutionOfLinearSystem:1100}
%%%can be expressed explicitly as a (unit bivector scaled) determinants
%%%
%%%\begin{equation}\label{eqn:solutionOfLinearSystem:1140}
%%%%\begin{aligned}
%%%s =
%%%\frac{
%%%\begin{vmatrix}
%%%u_{21} & u_{22} \\
%%%d_1 & d_2
%%%\end{vmatrix}
%%%\Be_{12}
%%%}
%%%{
%%%\begin{vmatrix}
%%%u_{11} & u_{12} \\
%%%u_{21} & u_{22} \\
%%%\end{vmatrix}
%%%\Be_{12}
%%%}
%%%%=
%%%%\frac{
%%%%\begin{vmatrix}
%%%%u_{21} & u_{22} \\
%%%%d_1 & d_2
%%%%\end{vmatrix}
%%%%}
%%%%{
%%%%\begin{vmatrix}
%%%%u_{11} & u_{12} \\
%%%%u_{21} & u_{22} \\
%%%%\end{vmatrix}
%%%%}
%%%\qquad
%%%t =
%%%\frac{
%%%\begin{vmatrix}
%%%u_{11} & u_{12} \\
%%%d_1 & d_2
%%%\end{vmatrix}
%%%\Be_{12}
%%%}
%%%{
%%%\begin{vmatrix}
%%%u_{11} & u_{12} \\
%%%u_{21} & u_{22} \\
%%%\end{vmatrix}
%%%\Be_{12}
%%%}
%%%%=
%%%%\frac{
%%%%\begin{vmatrix}
%%%%u_{11} & u_{12} \\
%%%%d_1 & d_2
%%%%\end{vmatrix}
%%%%}
%%%%{
%%%%\begin{vmatrix}
%%%%u_{11} & u_{12} \\
%%%%u_{21} & u_{22} \\
%%%%\end{vmatrix}
%%%%}
%%%.
%%%%\end{aligned}
%%%\end{equation}
%%%
%%%Once the unit bivectors \( \Be_{12} \) are cancelled \cref{eqn:solutionOfLinearSystem:1140} is the Cramer's rule solution of the problem.  Cramer's rule is seen to follow directly from the use of the wedge product to eliminate factors that are not of interest.
%%%In a similar way, the use of the wedge product for a 3D intersection problem with three variables, will lead directly to the Cramer's rule solution.
%%%
\makeproblem{Intersection of a line and plane.}{problem:solutionOfLinearSystem:1}{
Let a line be parameterized by
\begin{equation*}
\Br(a) = \Bp + a \Ba,
\end{equation*}
and a plane be parameterized by
\begin{equation*}
\Br(b,c) = \Bq + b \Bb + c \Bc.
\end{equation*}
\makesubproblem{}{problem:solutionOfLinearSystem:1:a}
State the vector equation to be solved, and its solution for \( a \) in terms of a ratio of wedge products.
\makesubproblem{}{problem:solutionOfLinearSystem:1:b}
State the conditions for which the solution exist in \R{3} and \R{N}.
\makesubproblem{}{problem:solutionOfLinearSystem:1:c}
In terms of coordinates in \R{3} write out the ratio of wedge products as determinants and compare to the Cramer's rule solution.
} % problem

         %\subsection{Orientation.}
         %   %
% Copyright © 2017 Peeter Joot.  All Rights Reserved.
% Licenced as described in the file LICENSE under the root directory of this GIT repository.
%
FIXME: find a home for this.  Also have a number of figures in junk.pdf that should be brought in.

Perhaps after triple wedge products, and certainly after areas of wedge products.

Given this, it is reasonable to associate an orientation with 2-blades (like \( \ucap \vcap \) or \( \Be_{12} \)).
We will also see that an area can also be associated with a 2-blade.

Eventually this will justify a geometrical characterization of k-vector blades along the following lines

\begin{itemize}
\item A scalar (0-vector) represents an oriented (signed) point.
\item A vector (1-vector, also always a 1-blade) represents an oriented line segment.
\item A bivector, if it is also a 2-blade, represents an oriented area segment.  You could think of such a bivector as a representation of an area elements that has, say, an upper or lower side, just as a vector can be thought of as having a head and a tail.
\item A trivector, if also a 3-blade, represents an oriented volume segment.  We can associate an inwards or outward normal with the orientation of this volume element, or associate the orientation with a rotational sense as depicted in
\cref{fig:orientedVolume:orientedVolumeFig1}.
\imageFigure{../figures/GAelectrodynamics/orientedVolumeFig1}{Oriented Volume}{fig:orientedVolume:orientedVolumeFig1}{0.3}
\item A k-blade represents an oriented k-dimensional hypervolume element.
\end{itemize}

As with multivectors in general, it isn't clear whether there is a good geometric characterization for a k-vector that isn't also a blade.



     \section{A summary comparision.}
        %
% Copyright © 2017 Peeter Joot.  All Rights Reserved.
% Licenced as described in the file LICENSE under the root directory of this GIT repository.
%

Here is a summary that compares various GA expressions with close equivalents in traditional vector (or complex) algebra.

\begin{tablelabelbox}[tabularx={X||Y|Y}]{Comparison of traditional and geometric algebra identities.}{label=tab:gacompare:compare}
             & Geometric algebra & Traditional algebra
\\ \hline
Norm squared & \( \Bx^2 \)            & \( \Bx \cdot \Bx \)
\\ \hline
complex imaginary & \( \Be_{12}, \Be_{123}, \cdots \) & i
\\ \hline
complex number & \( x + \Be_{12} y \) & \( x + i y \)
\\ \hline
vector products & \begin{equation*}
\begin{aligned}
\Ba \Bb &= \Norm{\Ba} \Norm{\Bb} \exp\lr{ i_{ab} \Delta \theta } \\
\Ba \cdot \Bb &= \Norm{\Ba} \Norm{\Bb} \cos\lr{ \Delta \theta } \\
\Ba \wedge \Bb &= i_{ab} \Norm{\Ba} \Norm{\Bb} \sin\lr{ \Delta \theta } \\
\end{aligned}
\end{equation*}
& \begin{equation*}
\begin{aligned}
\Ba \cdot \Bb &= \Norm{\Ba} \Norm{\Bb} \cos\lr{ \Delta \theta } \\
\Ba \cross \Bb &= \ncap \Norm{\Ba} \Norm{\Bb} \sin\lr{ \Delta \theta } \\
\end{aligned}
\end{equation*},\quad (\( i_{ab} \ncap = \Be_{123} \)).
\\ \hline
%wedge/cross product antisymmetry
%& \( \Ba \wedge \Bb = -\Bb \wedge \Ba \)
%& \( \Ba \cross \Bb = -\Bb \cross \Ba \)
%\\ \hline
wedge/cross filtering
& \( \Ba \wedge (\alpha \Ba + \Bb) = \Ba \wedge \Bb \)
& \( \Ba \cross (\alpha \Ba + \Bb) = \Ba \cross \Bb \)
\\ \hline
wedge/cross coordinate expansion
& \(
\Ba \wedge \Bb
=
\sum_{i < j}
\begin{vmatrix}
a_i & a_j \\
b_i & b_j
\end{vmatrix}
\Be_i \Be_j \)
& \(
\Ba \cross \Bb
=
\sum_{i < j}
\begin{vmatrix}
a_i & a_j \\
b_i & b_j
\end{vmatrix}
\Be_i \cross \Be_j \)
\\ \hline
%wedge/cross determinant expansion
%& \( \Ba \wedge \Bb = \begin{vmatrix}
%\Be_2 \Be_3 & \Be_3 \Be_1 & \Be_1 \Be_2 \\
%a_1 & a_2 & a_3 \\
%b_1 & b_2 & b_3 \\
%\end{vmatrix} \)
%& \( \Ba \cross \Bb = \begin{vmatrix}
%\Be_1 & \Be_2 & \Be_3 \\
%a_1 & a_2 & a_3 \\
%b_1 & b_2 & b_3 \\
%\end{vmatrix} \)
%\\ \hline
Triple wedge (triple product) &
\( \Ba \wedge \Bb \wedge \Bc
=
\begin{vmatrix}
a_i & a_j & a_k \\
b_i & b_j & b_k \\
c_i & c_j & c_k \\
\end{vmatrix}
I \)
&
\( \Ba \cdot (\Bb \cross \Bc)
=
\begin{vmatrix}
a_i & a_j & a_k \\
b_i & b_j & b_k \\
c_i & c_j & c_k \\
\end{vmatrix}
\)
\\ \hline
Vector orthogonal to two vectors \( \Ba, \Bb \) & \( I (\Ba \wedge \Bb) \) & \( \Ba \cross \Bb \)
\\ \hline
Plane orthogonal to a vector \( \Bn \) & \( (I \Bn) \wedge (\Bx - \Bx_0) = 0 \) & \( \Bn \cdot (\Bx - \Bx_0) = 0 \)
\\ \hline
Plane containing vectors \( \Ba, \Bb \) & \( \Ba \wedge \Bb \wedge (\Bx - \Bx_0) = 0 \) & \( (\Ba \cross \Bb) \cdot (\Bx - \Bx_0) = 0 \)
\\ \hline
\end{tablelabelbox}



     \section{Problem solutions.}
         \shipoutAnswer

   \chapter{Multivector calculus.}
      \section{Reciprocal frames.}
         %
% Copyright � 2016 Peeter Joot.  All Rights Reserved.
% Licenced as described in the file LICENSE under the root directory of this GIT repository.
%
%{
%\input{../blogpost.tex}
%\renewcommand{\basename}{reciprocal}
%%\renewcommand{\dirname}{notes/phy1520/}
%\renewcommand{\dirname}{notes/ece1228-electromagnetic-theory/}
%%\newcommand{\dateintitle}{}
%%\newcommand{\keywords}{}
%
%\input{../peeter_prologue_print2.tex}
%
%\usepackage{peeters_layout_exercise}
%\usepackage{peeters_braket}
%\usepackage{peeters_figures}
%\usepackage{siunitx}
%%\usepackage{mhchem} % \ce{}
%%\usepackage{macros_bm} % \bcM
%%\usepackage{macros_qed} % \qedmarker
%%\usepackage{txfonts} % \ointclockwise
%
%\beginArtNoToc
%
%\generatetitle{Reciprocal frame vectors}
%%\chapter{reciprocal frame vectors}
%%\label{chap:reciprocal}
%
The end goal of this chapter is to be able to integrate multivector functions along curves and surfaces, known collectively as manifolds.
For our purposes, a manifold is defined by a parameterization, such as the vector valued function \( \Bx(a,b) \) where \( a, b\) are scalar parameters.  With one parameter the vector traces out a curve, with two a surface, three a volume, and so forth.
The respective partial derivatives of such a parameterized vector define a local basis for the surface at the point at which the partials are evaluated.
The span of such a basis is called the tangent space, and the partials that constitute it are not necessarily orthonormal, or even normal.

Unfortunately, in order to work with the curvilinear non-orthonormal bases that will be encountered in general integration theory, some
additional tools are required.

\begin{itemize}
\item
We introduce a reciprocal frame which partially generalizes the notion of normal to non-orthonormal bases.
\item
We will borrow the upper and lower index (tensor) notation from relativistic physics that is useful for the intrinsically non-orthonormal spaces encountered in that study, as this notation works well to define the reciprocal frame.
\end{itemize}

\index{reciprocal frame}
\makedefinition{Reciprocal frame}{dfn:reciprocal:frame}{
Given a basis for a subspace \( \setlr{ \Bx_1, \Bx_2, \cdots \Bx_n } \), where the vectors \( \Bx_i \) are not necessarily orthonormal, the reciprocal frame is defined as the set of vectors \( \setlr{ \Bx^1, \Bx^2, \cdots \Bx^n } \) satisfying

\begin{dmath*}
\Bx_i \cdot \Bx^j = {\delta_i}^j,
\end{dmath*}

where the vector \( \Bx^j \) is not the j-th power of \( \Bx \), but is a superscript index, the conventional way of denoting a reciprocal frame vector, and \( {\delta_i}^j \) is the Kronecker delta.
} % definition

This definition introduces mixed index variables for the first time in this text, which may be unfamiliar.  These are most often used in tensor algebra, where any expression that has pairs of upper and lower indexes implies a sum, and is called the summation (or Einstein) convention.  For example:

\begin{dmath}\label{eqn:reciprocal:400}
\begin{aligned}
a_i b^i &\equiv \sum_i a_i b^i \\
{A^{i}}_j B_i C^j &\equiv \sum_{i,j} {A^{i}}_j B_i C^j.
\end{aligned}
\end{dmath}

Summation convention will not be used explicitly in this text, as it deviates from normal practises in electrical engineering\footnote{Generally, when summation convention is used, explicit summation is only used explicitly when upper and lower indexes are not perfectly matched, but summation is still implied.  Readers of texts that use summation convention can check for proper matching of upper and lower indexes to ensure that the expressions make sense.  Such matching is the reason a mixed index Kronecker delta has been used in the definition of the reciprocal frame.}.

The most important application of a reciprocal frame is for the computation of the coordinates of a vector with respect to a non-orthonormal frame.
Let a vector \( \Ba \) have coordinates \( a^i \) with respect to a basis \( \setlr{ \Bx_i } \)

\begin{dmath}\label{eqn:reciprocal:20}
\Ba = \sum_j a^j \Bx_j,
\end{dmath}

where \( j \) is an index not a power\footnote{In tensor algebra, any index that is found in matched upper and lower index pairs, is known as a dummy summation index, whereas an index that is unmatched is known as a free index.  For example, in \( a^j b_{ij} \) (summation implied) \( j \) is a summation index, and \( i \) is a free index.  We are free to make a change of variables of any summation index, so for the same example we can write
\( a^k b_{ik} \).  These index tracking conventions are obvious when summation symbols are included explicitly, as we will do.}.

Dotting with the reciprocal frame vectors \( \Bx^i \) provides these coordinates \( a^i \)

\begin{dmath}\label{eqn:reciprocal:40}
\Ba \cdot \Bx^i
= \lr{\sum_j a^j \Bx_j} \cdot \Bx^i
= \sum_j a^j {\delta_j}^i
= a^i.
\end{dmath}

The vector can also be expressed with coordinates taken with respect to the reciprocal frame.  Let those coordinates be \( a_i \), so that

\begin{dmath}\label{eqn:reciprocal:60}
\Ba = \sum_i a_i \Bx^i.
\end{dmath}

Dotting with the basis vectors \( \Bx_i \) provides the reciprocal frame relative coordinates \( a_i \)

\begin{dmath}\label{eqn:reciprocal:80}
\Ba \cdot \Bx_i
= \lr{\sum_j a_j \Bx^j} \cdot \Bx_i
= \sum_j a_j {\delta^j}_i
= a_i.
\end{dmath}

We can summarize \cref{eqn:reciprocal:40} and \cref{eqn:reciprocal:80} by stating that a vector can be expressed in terms of coordinates relative to either the original or reciprocal basis as follows

\begin{equation}\label{eqn:reciprocal:420}
\Ba
= \sum_j \lr{ \Ba \cdot \Bx^j } \Bx_j
= \sum_j \lr{ \Ba \cdot \Bx_j } \Bx^j.
\end{equation}

In tensor algebra the basis is generally implied\footnote{
In tensor algebra, a vector, identified by the coordinates \( a^i \) is called a contravariant vector.
When that vector is identified by the coordinates \( a_i \) it is called a covariant vector.  These labels relate to how the coordinates transform with respect to norm preserving transformations.
We have no need of this nomenclature, since we never transform coordinates in isolation, but will always transform the coordinates along with their associated basis vectors.}.

%When doing tensor algebra manipulations, you'll generally have the freedom to swap any pairs of upper and lower indexes as done above.

An example of a 2D oblique Euclidean basis and a corresponding reciprocal basis is plotted in \cref{fig:obliqueReciprocal:obliqueReciprocalFig2}.
Also plotted are the superposition of the projections required to arrive at a given point \( (4,2) \)) along the \( \Be_1, \Be_2 \) directions or the \( \Be^1, \Be^2 \) directions.
In this plot, neither of the reciprocal frame vectors \( \Be^i \) are normal to the corresponding basis vectors \( \Be_i \).
When one of \( \Be_i \) is increased(decreased) in magnitude, there will be a corresponding decrease(increase) in the magnitude of \( \Be^i \), but if the orientation is remained fixed, the corresponding direction of the reciprocal frame vector stays the same.

\imageFigure{../figures/GAelectrodynamics/obliqueReciprocalFig2}{Oblique and reciprocal bases.}{fig:obliqueReciprocal:obliqueReciprocalFig2}{0.5}

How are the reciprocal frame vectors computed?  While these vectors have a natural GA representation, this is not intrinsically a GA problem, and can be solved with standard linear algebra, using a matrix inversion.
For example, given a 2D basis \( \setlr{ \Bx_1, \Bx_2 } \), the reciprocal basis can be assumed to have a coordinate representation in the original basis

\begin{dmath}\label{eqn:reciprocal:100}
\begin{aligned}
\Bx^1 &= a \Bx_1 + b \Bx_2 \\
\Bx^2 &= c \Bx_1 + d \Bx_2.
\end{aligned}
\end{dmath}

Imposing the constraints of \cref{dfn:reciprocal:frame} leads to a pair of 2x2 linear systems that are easily solved to find
\begin{dmath}\label{eqn:reciprocal:120}
\begin{aligned}
\Bx^1 &= \inv{ (\Bx_1)^2 (\Bx_2)^2 - \lr{ \Bx_1 \cdot \Bx_2}^2 } \lr{ (\Bx_2)^2 \Bx_1 - \lr{ \Bx_1 \cdot \Bx_2 } \Bx_2 } \\
\Bx^2 &= \inv{ (\Bx_1)^2 (\Bx_2)^2 - \lr{ \Bx_1 \cdot \Bx_2}^2 } \lr{ (\Bx_1)^2 \Bx_2 - \lr{ \Bx_1 \cdot \Bx_2 } \Bx_1 } \\
\end{aligned}
\end{dmath}

The reader may notice that for \R{3} the denominator is related to the norm of the cross product \( \Bx_1 \cross \Bx_2 \).
More generally, this can be expressed as the square of the bivector \( \Bx_1 \wedge \Bx_2 \)

\begin{dmath}\label{eqn:reciprocal:140}
-\lr{\Bx_1 \wedge \Bx_2 }^2
=
-\lr{\Bx_1 \wedge \Bx_2 } \cdot \lr{\Bx_1 \wedge \Bx_2 }
=
-\lr{ \lr{\Bx_1 \wedge \Bx_2 } \cdot \Bx_1 } \cdot \Bx_2
=
(\Bx_1)^2 (\Bx_2)^2 - \lr{\Bx_1 \cdot \Bx_2}^2.
\end{dmath}

Additionally, the numerators are each dot products of \( \Bx_1, \Bx_2 \) with that same bivector

\begin{dmath}\label{eqn:reciprocal:160}
\begin{aligned}
\Bx^1 &= \frac{\Bx_2 \cdot \lr{ \Bx_1 \wedge \Bx_2 } }{ \lr{ \Bx_1 \wedge \Bx_2}^2 } \\
\Bx^2 &= \frac{\Bx_1 \cdot \lr{ \Bx_2 \wedge \Bx_1 } }{ \lr{ \Bx_1 \wedge \Bx_2}^2 },
\end{aligned}
\end{dmath}

or

%\begin{dmath}\label{eqn:reciprocal:180}
\boxedEquation{eqn:reciprocal:180}{
\begin{aligned}
\Bx^1 &= \Bx_2 \cdot \inv{ \Bx_1 \wedge \Bx_2 } \\
\Bx^2 &= \Bx_1 \cdot \inv{ \Bx_2 \wedge \Bx_1 }.
\end{aligned}
}
%\end{dmath}

Geometrically, dotting with the bivector of the plane is a hybrid rotation and scaling operation.
For example, for \R{2} with \( \Bx_1 = a_1 \Be_1 + a_2 \Be_2, \Bx_2 = b_1 \Be_1 + b_2 \Be_2 \), that pseudoscalar for this basis is

\begin{dmath}\label{eqn:reciprocal:260}
\Bx_1 \wedge \Bx_2
=
\lr{ a_1 \Be_1 + a_2 \Be_2 } \wedge \lr{ b_1 \Be_1 + b_2 \Be_2 }
=
\lr{ a_1 b_2 - a_2 b_1 } \Be_{12}.
\end{dmath}

This has inverse
\begin{dmath}\label{eqn:reciprocal:280}
\inv{\Bx_1 \wedge \Bx_2 }
=
\inv{ a_1 b_2 - a_2 b_1 } \Be_{21}.
\end{dmath}

So for the \R{2} the reciprocal frame is just

\begin{dmath}\label{eqn:reciprocal:300}
\begin{aligned}
\Bx^1 &= \Bx_2 \frac{\Be_{21}}{ a_1 b_2 - a_2 b_1 } \\
\Bx^2 &= \Bx_1 \frac{\Be_{12}}{ a_1 b_2 - a_2 b_1 }
\end{aligned}
\end{dmath}

The vector \( \Bx^1 \) is obtained by rotating \( \Bx_2 \) by \( -\pi/2 \), and rescaling it.
The vector \( \Bx^2 \) is similarly obtained by a scaling and a rotation of \( \Bx_1 \) by \( \pi/2 \).

Generalizing \cref{eqn:reciprocal:180} is almost possible by inspection.
Given
a subspace spanned by a three vector basis \( \setlr{ \Bx_1, \Bx_2, \Bx_3 } \) the reciprocal frame vectors can be written as dot products

\begin{dmath}\label{eqn:reciprocal:320}
\begin{aligned}
\Bx^1 &= \lr{ \Bx_2 \wedge \Bx_3 } \cdot \lr{ \Bx^3 \wedge \Bx^2 \wedge \Bx^1 } \\
\Bx^2 &= \lr{ \Bx_3 \wedge \Bx_1 } \cdot \lr{ \Bx^1 \wedge \Bx^3 \wedge \Bx^2 } \\
\Bx^3 &= \lr{ \Bx_1 \wedge \Bx_2 } \cdot \lr{ \Bx^2 \wedge \Bx^1 \wedge \Bx^3 } \\
\end{aligned}
\end{dmath}

Each of those trivector terms equals \( - \Bx^1 \wedge \Bx^2 \wedge \Bx^3 \) and can be related to the (known) pseudoscalar \( \Bx_1 \wedge \Bx_2 \wedge \Bx_3 \) by observing that

\begin{dmath}\label{eqn:reciprocal:340}
\lr{ \Bx^1 \wedge \Bx^2 \wedge \Bx^3 } \cdot \lr{ \Bx_3 \wedge \Bx_2 \wedge \Bx_1 }
=
\Bx^1 \cdot \lr{ \Bx^2 \cdot \lr{ \Bx^3 \cdot \lr{ \Bx_3 \wedge \Bx_2 \wedge \Bx_1 } }}
=
\Bx^1 \cdot \lr{ \Bx^2 \cdot \lr{ \Bx_2 \wedge \Bx_1 } }
=
\Bx^1 \cdot \Bx_1
=
1,
\end{dmath}

which means that

\begin{dmath}\label{eqn:reciprocal:360}
-\Bx^1 \wedge \Bx^2 \wedge \Bx^3
= -\inv{ \Bx_3 \wedge \Bx_2 \wedge \Bx_1 }
= \inv{ \Bx_1 \wedge \Bx_2 \wedge \Bx_3 },
\end{dmath}

and

\boxedEquation{eqn:reciprocal:380}{
\begin{aligned}
\Bx^1 &= \lr{ \Bx_2 \wedge \Bx_3 } \cdot \inv{ \Bx_1 \wedge \Bx_2 \wedge \Bx_3 } \\
\Bx^2 &= \lr{ \Bx_3 \wedge \Bx_1 } \cdot \inv{ \Bx_1 \wedge \Bx_2 \wedge \Bx_3 } \\
\Bx^3 &= \lr{ \Bx_1 \wedge \Bx_2 } \cdot \inv{ \Bx_1 \wedge \Bx_2 \wedge \Bx_3 }
\end{aligned}
}

Geometrically, this trivector division is a duality transformation within the subspace spanned by the three vectors \( \Bx_1, \Bx_2, \Bx_3 \), also scaling the result so that the \( \Bx_i \cdot \Bx^j = {\delta_i}^j \) condition is satisfied.

It should be clear how to generalize the reciprocal basis calculation formulas of
\cref{eqn:reciprocal:180} and \cref{eqn:reciprocal:380} to higher dimensions if desired.
%}
%\EndNoBibArticle

         \subsection{Problems.}
            %
% Copyright © 2016 Peeter Joot.  All Rights Reserved.
% Licenced as described in the file LICENSE under the root directory of this GIT repository.
%
\makeproblem{Reciprocal frame for two dimensional subspace.}{problem:reciprocal:2dsubspaceRecip}{
Prove \cref{eqn:reciprocal:120}.
} % problem

\makeanswer{problem:reciprocal:2dsubspaceRecip}{

Assuming the representation of \cref{eqn:reciprocal:100}, the dot products are

\begin{dmath}\label{eqn:2dreciprocalMatrixCalculation:200}
\begin{aligned}
1 &= \Bx_1 \cdot \Bx^1 = a \Bx_1^2 + b \Bx_1 \cdot \Bx_2 \\
0 &= \Bx_2 \cdot \Bx^1 = a \Bx_2 \cdot \Bx_1 + b \Bx_2^2 \\
0 &= \Bx_1 \cdot \Bx^2 = c \Bx_1^2 + d \Bx_1 \cdot \Bx_2 \\
1 &= \Bx_2 \cdot \Bx^2 = c \Bx_2 \cdot \Bx_1 + d \Bx_2^2
\end{aligned}
\end{dmath}

This can be written out as a pair of matrix equations

\begin{dmath}\label{eqn:2dreciprocalMatrixCalculation:220}
\begin{aligned}
\begin{bmatrix}
1 \\
0
\end{bmatrix}
&=
\begin{bmatrix}
\Bx_1^2 & \Bx_1 \cdot \Bx_2 \\
\Bx_2 \cdot \Bx_1 & \Bx_2^2 \\
\end{bmatrix}
\begin{bmatrix}
a \\
b
\end{bmatrix} \\
\begin{bmatrix}
0 \\
1
\end{bmatrix}
&=
\begin{bmatrix}
\Bx_1^2 & \Bx_1 \cdot \Bx_2 \\
\Bx_2 \cdot \Bx_1 & \Bx_2^2 \\
\end{bmatrix}
\begin{bmatrix}
c \\
d
\end{bmatrix}.
\end{aligned}
\end{dmath}

The matrix inverse is
\begin{dmath}\label{eqn:2dreciprocalMatrixCalculation:240}
{
\begin{bmatrix}
\Bx_1^2 & \Bx_1 \cdot \Bx_2 \\
\Bx_2 \cdot \Bx_1 & \Bx_2^2 \\
\end{bmatrix}
}^{-1}
=
\inv{ \Bx_1^2 \Bx_2^2 - \lr{\Bx_1 \cdot \Bx_2}^2 }
\begin{bmatrix}
\Bx_2^2 & -\Bx_1 \cdot \Bx_2 \\
-\Bx_2 \cdot \Bx_1 & \Bx_1^2 \\
\end{bmatrix}
\end{dmath}

Multiplying by the \( (1,0) \), and \( (0,1) \) vectors picks out the respective columns, and gives \cref{eqn:reciprocal:120}.
} % answer

            %
% Copyright © 2016 Peeter Joot.  All Rights Reserved.
% Licenced as described in the file LICENSE under the root directory of this GIT repository.
%

\index{reciprocal frame}
\makeproblem{Two vector reciprocal frame}{problem:2subspaceR3reciprocalExample:2subspaceR3reciprocalExample}{
Calculate the reciprocal frame for the \R{3} subspace spanned by \( \setlr{ \Bx_1, \Bx_2 } \) where

\begin{dmath}\label{eqn:2subspaceR3reciprocalExample:20}
\begin{aligned}
\Bx_1 &= \Be_1 + 2 \Be_2 \\
\Bx_2 &= \Be_2 - \Be_3.
\end{aligned}
\end{dmath}
} % problem

\makeanswer{problem:2subspaceR3reciprocalExample:2subspaceR3reciprocalExample}{
The bivector for the plane spanned by this basis is

\begin{dmath}\label{eqn:2subspaceR3reciprocalExample:40}
\Bx_1 \wedge \Bx_2
=
\lr{ \Be_1 + 2 \Be_2 } \wedge
\lr{ \Be_2 - \Be_3 }
=
\Be_{12} - \Be_{13} - 2 \Be_{23}
=
\Be_{12} + \Be_{31} + 2 \Be_{32}.
\end{dmath}

This has the square
\begin{dmath}\label{eqn:2subspaceR3reciprocalExample:60}
\lr{ \Bx_1 \wedge \Bx_2 }^2
=
\lr{ \Be_{12} + \Be_{31} + 2 \Be_{32} }
\cdot
\lr{ \Be_{12} + \Be_{31} + 2 \Be_{32} }
=
-1 -1 -4
=
-6.
\end{dmath}

Dotting \( -\Bx_1 \) with the bivector is
\begin{dmath}\label{eqn:2subspaceR3reciprocalExample:80}
\Bx_1 \cdot \lr{ \Bx_2 \wedge \Bx_1 }
=
-\lr{ \Be_1 + 2 \Be_2 } \cdot \lr{\Be_{12} + \Be_{31} + 2 \Be_{32} }
=
-\lr{ \Be_2 - \Be_3 - 2 \Be_1 - 4 \Be_3 }
= 2 \Be_1 - \Be_2 + 5 \Be_3.
\end{dmath}

For \( \Bx_2 \) the dot product with the bivector is

\begin{dmath}\label{eqn:2subspaceR3reciprocalExample:100}
\Bx_2 \cdot \lr{ \Bx_1 \wedge \Bx_2 }
=
\lr{ \Be_2 - \Be_3 } \cdot \lr{\Be_{12} + \Be_{31} + 2 \Be_{32} }
=
- \Be_1 - 2 \Be_3 - \Be_1 - 2 \Be_2
=
- 2 \Be_1 - 2 \Be_2 - 2 \Be_3,
\end{dmath}
so
\begin{dmath}\label{eqn:2subspaceR3reciprocalExample:120}
\begin{aligned}
\Bx^1 &= \inv{3} \lr{ \Be_1 + \Be_2 + \Be_3 } \\
\Bx^2 &= \inv{6} \lr{ -2 \Be_1 + \Be_2 - 5 \Be_3 }.
\end{aligned}
\end{dmath}
It is easy to verify that this has the desired semantics.
} % answer

      \section{Curvilinear coordinates.}
         %
% Copyright © 2017 Peeter Joot.  All Rights Reserved.
% Licenced as described in the file LICENSE under the root directory of this GIT repository.
%
Curvilinear coordinates can be defined for any subspace spanned by a parameterized vector into that space.  Given, for example, a vector into a subspace parameterized by parameters \(u,v\)

\begin{dmath}\label{eqn:curvilinear:20}
\Bx = \Bx(u, v),
\end{dmath}

the partials with respect to these parameters

\begin{dmath}\label{eqn:curvilinear:40}
\begin{aligned}
d\Bx_u &= \PD{u}{\Bx} du \\
d\Bx_v &= \PD{v}{\Bx} dv
\end{aligned}
\end{dmath}

span the space at the point that these partials are evaluated.  In the language of differential forms, this localized subspace is called the tangent space.  It is generally desirable to consider parameterizations for which the tangent space volume element is non-zero.  In this case, that is

\begin{dmath}\label{eqn:curvilinear:60}
d\Bx_u \wedge d\Bx_v \ne 0.
\end{dmath}

The differentials form a basis for the tangent space, as do the partials themselves

\begin{dmath}\label{eqn:curvilinear:80}
\begin{aligned}
\Bx_u &= \PD{u}{\Bx} \\
\Bx_v &= \PD{v}{\Bx}.
\end{aligned}
\end{dmath}

There is no reason to presume that there is any orthonormality constraint on the basis \( \setlr{ \Bx_u, \Bx_v } \) for this two parameter subspace, so a reciprocal basis \( \setlr{ \Bx^u, \Bx^v } \)
must be used to compute coordinates.
%, defined by \( \Bx^i \cdot \Bx_j = {\delta^i}_j \),  must be used to compute coordinates.   [SEE: GAelectrodynamics: 3.1].

More generally, given a parameterization of \( \Bx(u_1, u_2, \cdots, u_k) \), a curvilinear basis defined on the tangent space is induced by the partials

\begin{dmath}\label{eqn:curvilinear:240}
\Bx_{u_i} = \PD{u_i}{\Bx}.
\end{dmath}

The volume element for the subspace is

\begin{dmath}\label{eqn:curvilinear:260}
d^k \Bx = du_1 du_2 \cdots du_k\,
\Bx_{u_1} \wedge
\Bx_{u_2} \wedge \cdots \wedge
\Bx_{u_k}.
\end{dmath}

Unlike a scalar volume, this volume element is oriented.  Any multivector can be expressed in terms of the curvilinear basis \( \setlr{ \Bx_{u_1}, \Bx_{u_2}, \cdots, \Bx_k} \), but computation of the curvilinear coordinates requires the reciprocal basis.  For example, a vector \( \Bf \) constrained to the tangent space admits a representation

\begin{dmath}\label{eqn:curvilinear:380}
\Bf = \sum_i a_i \Bx_{u_i}.
\end{dmath}

Dotting with \( \Bx^{u_j} \) gives

\begin{dmath}\label{eqn:curvilinear:280}
\Bf \cdot \Bx^{u_j}
= \sum_i a_i \Bx_{u_i} \cdot \Bx^{u_j}
= \sum_i a_i {\delta^i}_j
= a_j,
\end{dmath}

so
\begin{dmath}\label{eqn:curvilinear:300}
\Bf = \sum_i \lr{ \Bf \cdot \Bx^{u_i} } \Bx_{u_i}.
\end{dmath}

Higher grade multivector objects may also be represented in curvilinear coordinates.  For example, given a bivector constrained to the tangent space

\begin{dmath}\label{eqn:curvilinear:320}
B = \inv{2} \sum_{i, j} b_{ij} \Bx_{u_i} \wedge \Bx_{u_j},
\end{dmath}

the coordinates \( b_{ij} \) can be determined by dotting \( B \) with \( \Bx^{u_j} \wedge \Bx^{u_i} \), yielding

\begin{dmath}\label{eqn:curvilinear:340}
B \cdot \lr{ \Bx^{u_j} \wedge \Bx^{u_i} }
=
\inv{2} \sum_{i' , j'} b_{i'j'} \lr{ \Bx_{u_i'} \wedge \Bx_{u_j'} } \cdot \lr{ \Bx^{u_j} \wedge \Bx^{u_i} }
=
\inv{2} \sum_{i' , j'} b_{i'j'} \lr{ \lr{ \Bx_{u_i'} \wedge \Bx_{u_j'} } \cdot \Bx^{u_j} } \cdot \Bx^{u_i}
=
\inv{2} \sum_{i' , j'} b_{i'j'} \lr{  \Bx_{u_i'} {\delta_j'}^j - \Bx_{u_j'} {\delta_i'}^j } \cdot \Bx^{u_i}
=
\inv{2} \sum_{i' , j'} b_{i'j'} \lr{  {\delta_i'}^i {\delta_j'}^j - {\delta_j'}^i {\delta_i'}^j }
=
\inv{2} \lr{ b_{i j} - b_{ji} }.
\end{dmath}

When \( i \ne j \) this is \( b_{ij} \) and is zero otherwise.  The curvilinear representation of the bivector is therefore

\begin{dmath}\label{eqn:curvilinear:400}
B = \sum_{i < j} \lr{ B \cdot \lr{ \Bx^{u_j} \wedge \Bx^{u_i} }} \Bx_{u_i} \wedge \Bx_{u_j}.
\end{dmath}


         %
% Copyright � 2017 Peeter Joot.  All Rights Reserved.
% Licenced as described in the file LICENSE under the root directory of this GIT repository.
%
While the reciprocal frame can be computed explicitly, it can also be computed very simply by computing the gradient of the parameters themselves.  Two theorems relate the gradient and the reciprocal frame vectors.

\maketheorem{Gradient definition of reciprocal frame vectors}{thm:curvilinearGradient:1}{

Given a curvilinear basis \( \setlr{ \Bx_k } \), the reciprocal frame vectors are

\begin{dmath*}
\Bx^i = \spacegrad u_i.
\end{dmath*}
} % theorem

\maketheorem{Curvilinear representation of the gradient}{thm:curvilinearGradient:2}{

Given an n-parameter representation of a vector that spans an n-dimensional space

\begin{dmath*}
\Bx = \Bx(u_1, \cdots, u_n),
\end{dmath*}

the curvilinear representation of the gradient is

\begin{dmath*}
\spacegrad = \sum_i \Bx^i \PD{u_i}{}.
\end{dmath*}

It is often convenient to write this as

\begin{dmath*}
\spacegrad = \sum_{i=1}^n \Bx^i \partial_i,
\end{dmath*}

or the same with sums over mixed indexes implied.

} % theorem

The proof of both are both just applications of the chain rule.  Assuming \cref{thm:curvilinearGradient:1} to be true, then the dot products of the reciprocal frame vectors with the curvilinear basis vectors are

\begin{dmath}\label{eqn:curvilinearGradient:20}
\Bx^i \cdot \Bx_j
=
(\spacegrad u_i) \cdot \PD{u_j}{\Bx}
=
\sum_{r,s=1}^n
\lr{ \Be_r \PD{x_r}{u_i} } \cdot \lr{ \Be_s \PD{u_j}{x_s} }
=
\sum_{r,s=1}^n (\Be_r \cdot \Be_s)
\PD{x_r}{u_i} \PD{u_j}{x_s}
=
\sum_{r,s=1}^n \delta_{rs}
\PD{x_r}{u_i} \PD{u_j}{x_s}
=
\sum_{r=1}^n
\PD{x_r}{u_i} \PD{u_j}{x_r}
=
\PD{u_i}{u_j}
=
\delta_{ij}.
\end{dmath}

This shows that \( \Bx^i = \spacegrad u_i \) has the properties required of the reciprocal frame, proving the theorem.

The curvilinear representation of the gradient follows from the gradient representation of the reciprocal frame, and the chain rule.  The sum in \cref{thm:curvilinearGradient:2} expands as

\begin{dmath}\label{eqn:curvilinearGradient:40}
\sum_{i=1}^n
\Bx^i \PD{u_i}{F}
=
\sum_{i=1}^n
(\spacegrad u_i) \PD{u_i}{F}
=
\sum_{i,j=1}^n
\Be_j \PD{x_j}{u_i}
\PD{u_i}{F}
=
\sum_{j=1}^n
\Be_j
\PD{x_j}{F}
=
\spacegrad F,
\end{dmath}

which proves the result.

Note that the gradient representation of the reciprocal frame is mainly useful for theoretical reasons (i.e. the proof of the curvilinear representation of the gradient).  In many cases it will likely be more difficult to compute the reciprocal frame vectors using the gradient of the parameters than

An excellent (and more detailed) discussion of the relationships of the reciprocal frame and the gradient can be found in \citep{aMacdonaldVAGC}.


Many of the concepts are illuminated nicely by considering some examples.

         \subsection{Cylindrical coordinates.}
            %
% Copyright © 2017 Peeter Joot.  All Rights Reserved.
% Licenced as described in the file LICENSE under the root directory of this GIT repository.
%
%\index{cylindrical coordinates}
\index{polar coordinates}
\index{curvilinear coordinates}

We will now consider a simple concrete example of a vector parameterization, that of polar coordinates in \R{2}, as illustrated in
\cref{fig:curvilinearPolar:curvilinearPolarFig1}.

\imageFigure{../figures/GAelectrodynamics/curvilinearPolarFig1}{Polar coordinates.}{fig:curvilinearPolar:curvilinearPolarFig1}{0.3}

The parameterization is simplest in complex exponential form

\begin{dmath}\label{eqn:2Dcylindrical:100}
\Bx(\rho, \phi) = \rho \Be_1 \exp\lr{ \Be_{12} \phi }.
\end{dmath}

The curvilinear coordinate basis is therefore

\begin{subequations}
\label{eqn:2Dcylindrical:120}
\begin{dmath}\label{eqn:2Dcylindrical:140}
\Bx_\rho
= \PD{\rho}{} \lr{ \rho \Be_1 \exp\lr{ \Be_{12} \phi } }
= \Be_1 \exp\lr{ \Be_{12} \phi }
\end{dmath}
\begin{dmath}\label{eqn:2Dcylindrical:160}
\Bx_\phi
= \PD{\phi}{} \lr{ \rho \Be_1 \exp\lr{ \Be_{12} \phi } }
= \rho
\Be_1 \Be_{12} \exp\lr{ \Be_{12} \phi }
= \rho
\Be_2 \exp\lr{ \Be_{12} \phi }.
\end{dmath}
\end{subequations}

To show that these vectors are 
perpendicular, we can
select the scalar grade of their product, and
use \cref{thm:SimpleProducts2:1780}, property (c) to swap the vector and complex exponential, conjugating the exponential

\begin{dmath}\label{eqn:2Dcylindrical:640}
\Bx_\rho \cdot \Bx_\phi
=
\gpgradezero{
   \lr{ \Be_1 \exp\lr{ \Be_{12} \phi } }
   \lr{ \rho \Be_2 \exp\lr{ \Be_{12} \phi } }
}
=
\rho
\gpgradezero{
   \Be_1 \exp\lr{ \Be_{12} \phi }
   \exp\lr{ -\Be_{12} \phi } \Be_2
}
=
\rho
\gpgradezero{
   \Be_1
\Be_2
}
=
0.
\end{dmath}

We can use the same method to find the (squared) length of the vectors

\begin{dmath}\label{eqn:2Dcylindrical:680}
\Bx_\rho^2
=
\gpgradezero{
   \Be_1 \exp\lr{ \Be_{12} \phi }
   \Be_1 \exp\lr{ \Be_{12} \phi }
}
=
\gpgradezero{
   \Be_1 \exp\lr{ \Be_{12} \phi }
   \exp\lr{ -\Be_{12} \phi } \Be_1
}
=
\gpgradezero{
   \Be_1^2
}
= 1,
\end{dmath}

and

\begin{dmath}\label{eqn:2Dcylindrical:700}
\Bx_\phi^2
=
\gpgradezero{
   \lr{ \rho \Be_2 \exp\lr{ \Be_{12} \phi } }
   \lr{ \rho \Be_2 \exp\lr{ \Be_{12} \phi } }
}
=
\rho^2
\gpgradezero{
   \Be_2 \exp\lr{ \Be_{12} \phi }
   \exp\lr{ -\Be_{12} \phi } \Be_2
}
=
\rho^2
\gpgradezero{
\Be_2^2
}
= \rho^2.
\end{dmath}

The volume element for this subspace is
\begin{dmath}\label{eqn:2Dcylindrical:220}
d\Bx_\rho \wedge d\Bx_\phi
=
d\rho d\phi
\Bx_\rho \wedge \Bx_\phi
=
d\rho d\phi
\gpgradetwo{
\Bx_\rho \Bx_\phi
}
=
d\rho d\phi
\gpgradetwo{
\Be_1 \exp\lr{ \Be_{12} \phi } \rho
\Be_2 \exp\lr{ \Be_{12} \phi }
}
=
\rho d\rho d\phi
\gpgradetwo{
\Be_1 \Be_2 \exp\lr{ -\Be_{12} \phi }
\exp\lr{ \Be_{12} \phi }
}
=
\rho d\rho d\phi \Be_{12}.
\end{dmath}

Observe that the (oriented) volume of a circular region of radius \( r \) in this space has the expected result

\begin{dmath}\label{eqn:2Dcylindrical:360}
\int d\Bx_\rho \wedge d\Bx_\phi
=
\int_0^r \rho d\rho \int_0^{2\pi} d\phi \Be_{12}
= \pi r^2 \Be_{12}.
\end{dmath}

\index{reciprocal basis}
Noting that this is a normal set of vectors, the reciprocal basis can be found by inspection

\begin{dmath}\label{eqn:2Dcylindrical:180}
\begin{aligned}
\Bx^\rho &= \Be_1 \exp\lr{ \Be_{12} \phi } \\
\Bx^\phi &= \inv{\rho} \Be_2 \exp\lr{ \Be_{12} \phi }.
\end{aligned}
\end{dmath}

\index{gradient}
For completeness, it's worth verifying that the gradient representation of the reciprocal frame provides this same result.
The \( x, y \) variables are related to \( \rho, \phi \) through

\begin{dmath}\label{eqn:2Dcylindrical:620}
\begin{aligned}
x &= r \cos\phi \\
y &= r \sin\phi.
\end{aligned}
\end{dmath}

Rearranging slightly to facilitate evaluation of the \( x, y \) partials

\begin{dmath}\label{eqn:2Dcylindrical:500}
\begin{aligned}
\rho^2 &= x^2 + y^2 \\
\tan\phi &= y/x,
\end{aligned}
\end{dmath}

we can evaluate the components of the gradients by implicit differentiation

\begin{dmath}\label{eqn:2Dcylindrical:520}
\begin{aligned}
2 \rho \PD{x}{\rho} &= 2 x \\
2 \rho \PD{y}{\rho} &= 2 y \\
\inv{\cos^2\phi} \PD{x}{\phi} &= -\frac{y}{x^2} \\
\inv{\cos^2\phi} \PD{y}{\phi} &= \inv{x},
\end{aligned}
\end{dmath}

The gradients are
\begin{subequations}
\label{eqn:2Dcylindrical:540}
\begin{dmath}\label{eqn:2Dcylindrical:560}
\spacegrad \rho
= \inv{\rho} (\cos\phi, \sin\phi)
= \Be_1 e^{\Be_{12} \phi}
= \Bx^\rho
\end{dmath}
\begin{dmath}\label{eqn:2Dcylindrical:580}
\spacegrad \phi
=
\cos^2 \phi \lr{ -\frac{y}{x^2}, \inv{x} }
=
\inv{\rho} ( -\sin\phi, \cos\phi )
=
\frac{\Be_2}{\rho} ( \cos\phi + \Be_{12} \sin\phi )
=
\frac{\Be_2}{\rho} e^{ \Be_{12} \phi }
=
\Bx^\phi,
\end{dmath}
\end{subequations}

which is consistent with the result found by inspection as desired.

In this particular parameterization, it is convenient to define a locally orthonormal coordinate basis \( \setlr{ \rhocap, \phicap } \)

\begin{dmath}\label{eqn:2Dcylindrical:200}
\begin{aligned}
\rhocap &= \Bx_\rho = \Be_1 \exp\lr{ \Be_{12} \phi } \\
\phicap &= \inv{\rho} \Bx_\phi = \Be_2 \exp\lr{ \Be_{12} \phi },
\end{aligned}
\end{dmath}

so that \( \Bx^\rho = \Bx_\rho = \rhocap \), \( \Bx_\phi = \rho \rhocap \), and \( \Bx^\phi = \rhocap/\rho \), and the gradient is

\begin{dmath}\label{eqn:2Dcylindrical:600}
\spacegrad
=
\Bx^\rho \PD{\rho}{}
+ \Bx^\phi \PD{\phi}{}
=
\rhocap \PD{\rho}{}
+\inv{\rho} \phicap \PD{\phi}{}.
\end{dmath}

%%Given a vector \( \Bv = \Be_1 f(\rho, \phi) + \Be_2 g(\rho, \phi) \), the cylindrical representation \( \Bv = \Bv_\rho \rhocap + \Bv_\phi \phicap \) can be found by computing the dot products
%%
%%\begin{subequations}
%%\label{eqn:2Dcylindrical:420}
%%\begin{dmath}\label{eqn:2Dcylindrical:440}
%%\Bv \cdot \rhocap
%%=
%%\gpgradezero{ (\Be_1 f + \Be_2 g) \Be_1 e^{\Be_{12} \phi} }
%%=
%%f \cos\phi + g \sin\phi
%%\end{dmath}
%%\begin{dmath}\label{eqn:2Dcylindrical:460}
%%\Bv \cdot \phicap
%%=
%%\gpgradezero{ (\Be_1 f + \Be_2 g) \Be_2 e^{\Be_{12} \phi} }
%%=
%%g \cos\phi - f \sin\phi,
%%\end{dmath}
%%\end{subequations}
%%
%%so
%%\begin{dmath}\label{eqn:2Dcylindrical:480}
%%\Bv = \lr{ f \cos\phi + g \sin\phi } \rhocap + \lr{ g \cos\phi - f \sin\phi } \phicap.
%%\end{dmath}


         \subsection{Spherical coordinates.}
            %
% Copyright � 2017 Peeter Joot.  All Rights Reserved.
% Licenced as described in the file LICENSE under the root directory of this GIT repository.
%
\index{curvilinear coordinates}
\index{spherical coordinates}
The spherical vector parameterization admits a compact GA representation.
From the coordinate representation, some factoring gives

\begin{dmath}\label{eqn:curvilinearspherical:20}
\Bx
= r \lr{ \Be_1 \sin\theta \cos\phi + \Be_2 \sin\theta \sin\phi + \Be_3 \cos\theta }
= r \lr{ \sin\theta \Be_1 (\cos\phi + \Be_{12} \sin\phi ) + \Be_3 \cos\theta }
= r \lr{ \sin\theta \Be_1 e^{\Be_{12} \phi } + \Be_3 \cos\theta }
= r \Be_3 \lr{ \cos\theta + \sin\theta \Be_3 \Be_1 e^{\Be_{12} \phi } }.
\end{dmath}

With
\begin{dmath}\label{eqn:curvilinearspherical:40}
\begin{aligned}
i &= \Be_{12} \\
j &= \Be_{31} e^{i \phi},
\end{aligned}
\end{dmath}

this is

\begin{dmath}\label{eqn:curvilinearspherical:60}
\Bx = r \Be_3 e^{j \theta}.
\end{dmath}

The curvilinear basis vectors can now be computed

\begin{subequations}
\label{eqn:curvilinearspherical:80}
\begin{dmath}\label{eqn:curvilinearspherical:100}
\Bx_r = \Be_3 e^{j \theta}
\end{dmath}
\begin{dmath}\label{eqn:curvilinearspherical:120}
\Bx_\theta
= r \Be_3 j e^{j \theta}
= r \Be_3 \Be_{31} e^{i\phi} e^{j \theta}
= r \Be_1 e^{i\phi} e^{j \theta}
\end{dmath}
\begin{dmath}\label{eqn:curvilinearspherical:140}
\Bx_\phi
=
\PD{\phi}{} \lr{
r \Be_3 \sin\theta \Be_{31} e^{i \phi}
}
=
r \sin\theta \Be_1 \Be_{12} e^{i \phi}
=
r \sin\theta \Be_2 e^{i \phi}.
\end{dmath}
\end{subequations}

These are all mutually normal, which can be verified by computing dot products.
With that asserted, orthonormalization of the curvilinear basis is now possible by inspection

\begin{dmath}\label{eqn:curvilinearspherical:240}
\begin{aligned}
\rcap &= \Bx_r = \Be_3 e^{j \theta} \\
\thetacap &= \inv{r} \Bx_\theta = \Be_1 e^{i\phi} e^{j \theta} \\
\phicap &= \inv{r \sin\theta} \Bx_\phi = \Be_2 e^{i \phi},
\end{aligned}
\end{dmath}

so

\begin{dmath}\label{eqn:curvilinearspherical:260}
\begin{aligned}
\Bx^r &= \rcap = \Be_3 e^{j \theta} \\
\Bx^\theta &= \inv{r} \thetacap = \inv{r} \Be_1 e^{i\phi} e^{j \theta} \\
\Bx^\phi &= \inv{r \sin\theta} \phicap = \inv{r \sin\theta} \Be_2 e^{i \phi}.
\end{aligned}
\end{dmath}

\index{gradient!spherical}
In particular, this shows that the spherical representation of the gradient is
\begin{dmath}\label{eqn:curvilinearspherical:280}
\spacegrad
=
\Bx^r \PD{r}{}
+ \Bx^\theta \PD{\theta}{}
+ \Bx^\phi \PD{\phi}{}
=
\rcap \PD{r}{}
+\inv{r} \thetacap \PD{\theta}{}
+\inv{r \sin\theta} \phicap \PD{\phi}{}.
\end{dmath}

The spherical (oriented) volume element can also be computed in a compact fashion, without having to evaluate a very messy Jacobian determinant

\begin{dmath}\label{eqn:curvilinearspherical:300}
\Bx_r \wedge \Bx_\theta \wedge \Bx_\phi
=
\gpgradethree{
\Bx_r \Bx_\theta \Bx_\phi
}
=
\gpgradethree{
\Be_3 e^{j \theta}
r \Be_1 e^{i\phi} e^{j \theta}
r \sin\theta \Be_2 e^{i \phi}
}
=
r^2 \sin\theta
\gpgradethree{
\Be_3 e^{j \theta}
\Be_1 e^{i\phi} e^{j \theta}
\Be_2 e^{i \phi}
}
=
r^2 \sin\theta\, \Be_{123}
.
\end{dmath}

The final reduction is left as a problem for the student.
It is left to the student to evaluate whether this method is easier or more difficult than the conventional volume element Jacobean determinant expansion

\begin{dmath}\label{eqn:curvilinearspherical:320}
dV =
dr d\theta d\phi\,
\frac{\partial( x_1, x_2, x_3)}{\partial(r, \theta, \phi)}
=
dr d\theta d\phi\,
\begin{vmatrix}
\sin\theta \cos\phi & \sin\theta \sin\phi & \cos\theta \\
r \cos\theta \cos\phi & r \cos\theta \sin\phi & -r \sin\theta \\
-r \sin\theta \sin\phi & r \sin\theta \cos\phi & 0 \\
\end{vmatrix}.
\end{dmath}

Expanding this determinant (\cref{problem:curvilinearspherical:1}) gives
\begin{dmath}\label{eqn:curvilinearspherical:n}
dV =
dr d\theta d\phi\, r^2 \sin\theta,
\end{dmath}

consistent with \cref{eqn:curvilinearspherical:300}.

\makeproblem{Spherical volume Jacobean.}{problem:curvilinearspherical:1}{
Expand the determinant in \cref{eqn:curvilinearspherical:320}.
} % problem

%It is easily argued that both volume element calculation methods are best performed by a computer algebra system.
%FIXME: Wolfgang: ``please show this as an exercise!''
%  In EVA I had to use xu = r *(e1*sin(t)*cos(p)+e2*sin(t)*sin(p)+e3*cos(p)).
%  But the result deserves interpretation ..)


         \subsection{Toroidal coordinates.}
            %
% Copyright � 2012 Peeter Joot.  All Rights Reserved.
% Licenced as described in the file LICENSE under the root directory of this GIT repository.
%
\index{toroid}
\index{differential form}
%\imageFigure{../figures/gabook/toriodalSegment}{Toroidal parameterization.}{fig:toriodalSegment}{0.5}
\imageFigure{../figures/GAelectrodynamics/toroidFig1}{Toroidal parameterization.}{fig:toriodalSegment}{0.3}

Here is a 3D example of a parameterization with a non-orthogonal curvilinear basis, that of a
toroidal subspace specified by two angles and a radial distance to the center of the toroid, as illustrated in \cref{fig:toriodalSegment}.

The position vector on the surface of a toroid of radius \( \rho \) within the torus can be stated directly

\begin{subequations}
\begin{align}\label{eqn:torusCenterOfMassParameterization:1}
\Bx(\rho, \theta, \phi) &= e^{-j\theta/2} \left( \rho \Be_1 e^{ i \phi } + R \Be_3 \right) e^{j \theta/2} \\
i &= \Be_1 \Be_3 \\
j &= \Be_3 \Be_2
\end{align}
\end{subequations}

It happens that the unit bivectors \(i\) and \(j\) used in this construction happen
to have the
quaternion-ic properties \(i j = -j i\), and \(i^2 = j^2 = -1\) which can be verified easily.

After some regrouping the curvilinear basis is found to be

\begin{subequations}
\begin{align}\label{eqn:torusCenterOfMassParameterization:3}
\Bx_\rho &= \PD{\rho}{\Bx} = e^{-j\theta/2} \Be_1 e^{ i \phi } e^{j \theta/2} \\
\Bx_\theta &= \PD{\theta}{\Bx}
%&= e^{-j\theta/2} \left( \rho \inv{2} \left( -\Be_3 \Be_2 \Be_1 e^{ i \phi } + \Be_1 e^{ i \phi } \Be_3 \Be_2 \right) + R \Be_2 \right) e^{j \theta/2} \\
= e^{-j\theta/2} \left( R + \rho \sin\phi \right) \Be_2 e^{j \theta/2} \\
\Bx_\phi &= \PD{\phi}{\Bx} = e^{-j\theta/2} \rho \Be_3 e^{ i \phi } e^{j \theta/2}.
\end{align}
\end{subequations}

The oriented
volume element can be computed using a trivector selection operation, which conveniently wipes out a number of the interior exponentials
%\begin{align}\label{eqn:torusCenterOfMassParameterization:4}
\begin{dmath}\label{eqn:torusCenterOfMassParameterization:4}
\PD{\rho}{\Bx} \wedge \PD{\theta}{\Bx} \wedge \PD{\phi}{\Bx}
=
\rho \left( R + \rho \sin\phi \right) \gpgradethree{ e^{-j\theta/2} \Be_1 e^{ i \phi } \Be_2 \Be_3 e^{ i \phi } e^{j \theta/2} }.
%\end{align}
\end{dmath}

Note that \(\Be_1\) commutes with \(j = \Be_3 \Be_2\), so also with \(e^{-j\theta/2}\).
Also \(\Be_2 \Be_3 = -j\) anticommutes with \(i\), so
there is a conjugate commutation effect \(e^{i\phi} j = j e^{-i\phi}\).  This gives
\begin{dmath}\label{eqn:torusCenterOfMassParameterization:28}
\begin{aligned}
\gpgradethree{ e^{-j\theta/2} \Be_1 e^{ i \phi } \Be_2 \Be_3 e^{ i \phi } e^{j \theta/2} }
&=
-\gpgradethree{ \Be_1 e^{-j\theta/2} j e^{ -i \phi } e^{ i \phi } e^{j \theta/2} } \\
&=
-\gpgradethree{ \Be_1 e^{-j\theta/2} j e^{j \theta/2} } \\
&=
-\gpgradethree{ \Be_1 j } \\
&=
I.
\end{aligned}
\end{dmath}

Together the trivector grade selection reduces almost magically to just
\begin{equation}\label{eqn:torusCenterOfMassParameterization:5}
\PD{\rho}{\Bx} \wedge \PD{\theta}{\Bx} \wedge \PD{\phi}{\Bx}
=
\rho \left( R + \rho \sin\phi \right) I.
\end{equation}

\todo{Show this with Mathematica too.}

Thus the (scalar) volume element is
\begin{align}\label{eqn:torusCenterOfMassParameterization:6}
dV = \rho \left( R + \rho \sin\phi \right) d\rho d\theta d\phi.
\end{align}

As a check, it should be the case that the
volume of the complete torus using this volume element has the
expected \(V = (2 \pi R) (\pi r^2)\) value.

That volume is
\begin{align}\label{eqn:torusCenterOfMassParameterization:7}
V = \int_{\rho=0}^r \int_{\theta=0}^{2\pi} \int_{\phi=0}^{2\pi} \rho \left( R + \rho \sin\phi \right) d\rho d\theta d\phi.
\end{align}

The sine term conveniently vanishes over the \(2\pi\) interval, leaving just
\begin{align}\label{eqn:torusCenterOfMassParameterization:8}
V = \inv{2} r^2 R (2 \pi)(2 \pi),
\end{align}

as expected.


         \subsection{Problems.}
            %
% Copyright � CCYY Peeter Joot.  All Rights Reserved.
% Licenced as described in the file LICENSE under the root directory of this GIT repository.
%
\makeproblem{Spherical coordinate basis normality.}{problem:sphericaldot:1}{
\index{spherical coordinates}
Show that the spherical curvilinear basis of \cref{eqn:curvilinearspherical:80} are all mutually normal.
} % problem

\makeanswer{problem:sphericaldot:1}{
Computing the various dot products is made easier by noting that \( \Be_3 \) and \( e^{i \phi } \) commute, whereas \( e^{j\theta } \Be_3 = \Be_3 e^{-j\theta}, \Be_1 e^{i\phi} = e^{-i\phi} \Be_1, \Be_2 e^{i\phi} = e^{-i\phi} \Be_2 \) (since \( \Be_3 j \), \( \Be_1 i \) and \( \Be_2 i \) all anticommute)

\begin{subequations}
\label{eqn:sphericaldot:160}
\begin{dmath}\label{eqn:sphericaldot:180}
\Bx_r \cdot \Bx_\theta
=
\gpgradezero{
\Be_3 e^{j \theta} \Be_1 e^{i\phi} e^{j \theta}
}
=
\gpgradezero{
e^{j \theta} \Be_3 e^{j \theta} \Be_1 e^{i\phi}
}
=
\gpgradezero{
\Be_3 e^{-j \theta} e^{j \theta} \Be_1 e^{i\phi}
}
=
\gpgradezero{
\Be_3 \Be_1 e^{i\phi}
}
= 0
\end{dmath}
\begin{dmath}\label{eqn:sphericaldot:200}
\Bx_r \cdot \Bx_\phi
=
\gpgradezero{
\Be_3 e^{j \theta} r \sin\theta \Be_2 e^{i \phi}
}
=
r \sin\theta
\gpgradezero{
\Be_3 \lr{ \cos\theta + \Be_{31} \sin\theta e^{i\phi} } \Be_2 e^{i \phi}
}
=
r \sin^2\theta
\gpgradezero{
\Be_{1} e^{i\phi} \Be_2 e^{i \phi}
}
=
r \sin^2\theta
\gpgradezero{
\Be_{1} \Be_2
}
=
0
\end{dmath}
\begin{dmath}\label{eqn:sphericaldot:220}
\Bx_\theta \cdot \Bx_\phi
=
r \sin\theta
\gpgradezero{
\Be_1 e^{i\phi} e^{j \theta}
\Be_2 e^{i \phi}
}
=
r \sin\theta
\gpgradezero{
\Be_2 \Be_1 e^{j \theta}
}
=
r \sin\theta
\gpgradezero{
\Be_2 \Be_1 \lr{ \cos\theta + \Be_{31} \sin\theta e^{i \phi} }
}
=
r \sin^2\theta
\gpgradezero{
\Be_{32} e^{i \phi}
}
=
0.
\end{dmath}
\end{subequations}

} % answer

            %
% Copyright � CCYY Peeter Joot.  All Rights Reserved.
% Licenced as described in the file LICENSE under the root directory of this GIT repository.
%
\makeproblem{Spherical volume element pseudoscalar.}{problem:volumeselection:1}{
Confirm the grade three selection claim made in \cref{eqn:curvilinearspherical:300}.
} % problem

\makeanswer{problem:volumeselection:1}{

Using the commutation relations from last problem, first note that

\begin{dmath}\label{eqn:volumeselection:20}
\Be_1 e^{i\phi} e^{j \theta}
=
\Be_1 \lr{ \cos\theta e^{i\phi} + \sin\theta \Be_{31} e^{-i\phi} e^{i\phi} }
=
\Be_1 \lr{ \cos\theta + \sin\theta \Be_{31} e^{-i\phi} } e^{i\phi}
=
\lr{ \cos\theta - \sin\theta \Be_{31} e^{i\phi} } \Be_1 e^{i\phi}
=
e^{-j\theta}
\Be_1 e^{i\phi}.
\end{dmath}

This gives

\begin{dmath}\label{eqn:volumeselection:40}
\gpgradethree{
\Be_3 e^{j \theta}
\Be_1 e^{i\phi} e^{j \theta}
\Be_2 e^{i \phi}
}
=
\gpgradethree{
\Be_3
\Be_1 e^{i\phi}
\Be_2 e^{i \phi}
}
=
\gpgradethree{
\Be_3
\Be_1
\Be_2
}
=
\Be_{123}.
\end{dmath}

} % answer



%gabook: 31.1
%Also: Stokes chapter.
% Lots of examples there that should really be separated out from the stokes core content
%(now included here).
      \section{Green's theorem.}
         %
% Copyright © 2013 Peeter Joot.  All Rights Reserved.
% Licenced as described in the file LICENSE under the root directory of this GIT repository.
%
Given a two parameter (\(u,v\)) surface parameterization, the curvilinear coordinate representation of a vector \(\Bf\) has the form

\begin{dmath}\label{eqn:stokesTheoremGeometricAlgebra:1640}
\Bf = f_u \Bx^u + f_v \Bx^v + f_\perp \Bx^\perp.
\end{dmath}

We assume that the vector space is of dimension two or greater but otherwise unrestricted, and need not have an Euclidean basis.
Here \(f_\perp \Bx^\perp\) denotes the rejection of \(\Bf\) from the tangent space at the point of evaluation.
Green's theorem relates the integral around a closed curve to an ``area'' integral on that surface

\maketheorem{Green's Theorem}{thm:stokesTheoremGeometricAlgebra:1660}{
\index{Green's theorem}
\begin{equation*}
\ointctrclockwise \Bf \cdot d\Bl
=
\iint \lr{
-\PD{v}{f_u}
+\PD{u}{f_v}
}
du dv
\end{equation*}
}

Following the arguments used in \citep{schwartz1987pe} for Stokes' theorem in three dimensions, we first evaluate the loop integral along the differential element of the surface at the point \(\Bx(u_0, v_0)\) evaluated over the range \((du, dv)\), as shown in the infinitesimal loop of \cref{fig:loopIntegralInfinitesimal:loopIntegralInfinitesimalFig1}.

\imageFigure{../figures/gabook/loopIntegralInfinitesimalFig1}{Infinitesimal loop integral.}{fig:loopIntegralInfinitesimal:loopIntegralInfinitesimalFig1}{0.35}

Over the infinitesimal area, the loop integral decomposes into

\begin{dmath}\label{eqn:stokesTheoremGeometricAlgebra:1700}
\ointctrclockwise \Bf \cdot d\Bl
=
\int \Bf \cdot d\Bx_1
+\int \Bf \cdot d\Bx_2
+\int \Bf \cdot d\Bx_3
+\int \Bf \cdot d\Bx_4,
\end{dmath}

where the differentials along the curve are

\begin{dmath}\label{eqn:stokesTheoremGeometricAlgebra:1600}
\begin{aligned}
d\Bx_1 &= \evalbar{ \PD{u}{\Bx} }{v = v_0} du \\
d\Bx_2 &= \evalbar{ \PD{v}{\Bx} }{u = u_0 + du} dv \\
d\Bx_3 &= -\evalbar{ \PD{u}{\Bx} }{v = v_0 + dv} du \\
d\Bx_4 &= -\evalbar{ \PD{v}{\Bx} }{u = u_0} dv.
\end{aligned}
\end{dmath}

It is assumed that the parameterization change \((du, dv)\) is small enough that this loop integral can be considered planar (regardless of the dimension of the vector space).
Making use of the fact that \(\Bx^\perp \cdot \Bx_\alpha = 0\) for \(\alpha \in \setlr{u,v}\), the loop integral is

\begin{dmath}\label{eqn:stokesTheoremGeometricAlgebra:1620}
\ointctrclockwise \Bf \cdot d\Bl
=
\int
\lr{
f_u \Bx^u + f_v \Bx^v + f_\perp \Bx^\perp
}
\cdot
\Bigl(
\Bx_u(u, v_0) du - \Bx_u(u, v_0 + dv) du
+\Bx_v(u_0 + du, v) dv - \Bx_v(u_0, v) dv
\Bigr)
=
\int
f_u(u, v_0) du - f_u(u, v_0 + dv) du
+
f_v(u_0 + du, v) dv - f_v(u_0, v) dv
\end{dmath}

With the distances being infinitesimal, these differences can be rewritten as partial differentials

\begin{dmath}\label{eqn:stokesTheoremGeometricAlgebra:1860}
\ointctrclockwise \Bf \cdot d\Bl
=
\iint \lr{
-\PD{v}{f_u}
+\PD{u}{f_v}
}
du dv.
\end{dmath}

We can now sum over a larger area as in \cref{fig:loopIntegralInfinitesimalSum:loopIntegralInfinitesimalSumFig2}

\imageFigure{../figures/gabook/loopIntegralInfinitesimalSumFig2}{Sum of infinitesimal loops.}{fig:loopIntegralInfinitesimalSum:loopIntegralInfinitesimalSumFig2}{0.35}

All the opposing oriented loop elements cancel, so the integral around the complete boundary of the surface \(\Bx(u, v)\) is given by the \(u,v\) area integral of the partials difference.

We will see that Green's theorem is a special case of the Stokes' theorem.
This observation will also provide a geometric interpretation of the right hand side area integral of \cref{thm:stokesTheoremGeometricAlgebra:1660}, and allow for a coordinate free representation.

\paragraph{Special case:}

An important special case of Green's theorem is for a Euclidean two dimensional space where the vector function is

\begin{dmath}\label{eqn:stokesTheoremGeometricAlgebra:1720}
\Bf = P \Be_1 + Q \Be_2.
\end{dmath}

Here Green's theorem takes the form

\boxedEquation{eqn:stokesTheoremGeometricAlgebra:1710}{
\ointctrclockwise P dx + Q dy
=
\iint \lr{
\PD{x}{Q}
-\PD{y}{P}
}
dx dy.
}

         %\subsection{Problems}
      \section{Stokes' theorem.}
         \subsection{Statement.}
            %
% Copyright © 2016 Peeter Joot.  All Rights Reserved.
% Licenced as described in the file LICENSE under the root directory of this GIT repository.
%
\index{Stokes' theorem}
Stokes' theorem is fairly easy to state, but takes a fair amount of work to understand and unpack its implications.

%
% Copyright © 2013 Peeter Joot.  All Rights Reserved.
% Licenced as described in the file LICENSE under the root directory of this GIT repository.
%
\maketheorem{Stokes' Theorem}{thm:stokesTheoremGeometricAlgebra:1740}{

For blades \(F \in \bigwedge^{s}\), and \(m\) volume element \(d^k \Bx, s < k\),

\begin{equation*}%\label{eqn:stokesTheoremTheStatement:120}
\int_V d^k \Bx \cdot (\boldpartial \wedge F) = \int_{\partial V} d^{k-1} \Bx \cdot F.
\end{equation*}

Here the volume integral is over a \(m\) dimensional surface (manifold).  The derivative operator \(\boldpartial\) is called the vector derviative and is the projection of the gradient onto the tangent space of the manifold.  Integration over the boundary of \(V\) is indicated by \( \partial V \).
}

The vector derivative is defined by

\begin{equation}\label{eqn:stokesTheoremTheStatement:1400}
\boldpartial = \Bx^i \partial_i = \sum_i \Bx_i \PD{u^i}{}.
\end{equation}

where \( \Bx^i \) are reciprocal frame vectors dual to the tangent vector basis \( \Bx_i \) associated with the parameters \( u^1, u^2, \cdots \).
%These will be defined in more detail in the next section.
Once the volume element, vector product and the other concepts are defined, the proof of
Stokes theorem is really just a statement that

\boxedEquation{eqn:stokesTheoremGeometricAlgebra:2840}{
\int_V d^k \Bx \cdot (\Bx^i \partial_i \wedge F) =
\int_V \lr{ d^k \Bx \cdot \Bx^i } \cdot \partial_i F.
}

This dot product expansion applies to any degree blade and volume element provided the degree of the blade is less than that of the volume element (i.e. \(s < k\)).  That magic follows directly from \cref{thm:stokesTheoremGeometricAlgebra:1420}.


\index{oriented volume element}
This dot product defines the oriented surface ``area'' elements associated with the ``volume'' element \( d^k \Bx \).
That area element can be obtained from the mnemonic

\begin{dmath}\label{eqn:statement:1561}
\sum_i d^k \Bx \cdot \Bx^i,
\end{dmath}

with each of the i-th differentials evaluated.
This will be made clear by example.


         \subsection{One parameter specialization of Stokes' theorem.}
            %
% Copyright © 2016 Peeter Joot.  All Rights Reserved.
% Licenced as described in the file LICENSE under the root directory of this GIT repository.
%

An example parameterization with one parameter, and the corresponding differential with respect to that parameter, is plotted in
\cref{fig:oneParameterDifferential:oneParameterDifferentialFig1}, for a parameterization over \( [a, b] \in [0,1]\otimes[0,1] \).

\imageFigure{../figures/GAelectrodynamics/oneParameterDifferentialFig1}{One parameter manifold.}{fig:oneParameterDifferential:oneParameterDifferentialFig1}{0.3}

The differential with respect to the parameter \( a \) is

\begin{equation}\label{eqn:stokesTheoremCore:20}
d\Bx_a = \PD{a}{\Bx} da = \Bx_a da.
\end{equation}

On this curve the projection of the gradient has just one component

\begin{dmath}\label{eqn:stokesTheoremCore:40}
\boldpartial
=
\sum_\mu \Bx^\mu (\Bx_\mu \cdot \spacegrad)
=
\Bx^a \PD{a}{}
\equiv
\Bx^a \partial_a.
\end{dmath}

Please see \citep{aMacdonaldVAGC} for a full justification of the curvilinear coordinate representation of the vector derivative (or the gradient).
That text also discusses pertinent issues with the connectivity of the manifold ignored here.

Stokes' theorem for a one parameter manifold can only be expressed for scalar fields.
That is

\begin{dmath}\label{eqn:stokesTheoremCore:60}
\int d\Bx \cdot (\boldpartial \wedge \psi)
=
\int d\Bx \cdot \boldpartial \psi
=
\int da \PD{a}{ \psi }
= \evalbar{\psi}{\Delta a}.
\end{dmath}

Observe that the vector derivative can be replaced by the gradient since \( d\Bx \cdot \boldpartial = d\Bx \cdot \spacegrad \).
This is the case since dotting the
gradient with a differential element \( d\Bx \) on this curve, no component of the gradient that isn't colinear to the curve makes no contribution.

That means that Stokes' theorem for a one parameter curve is exactly the fundamental theorem of calculus for line integrals

%\begin{dmath}\label{eqn:stokesTheoremCore:80}
\boxedEquation{eqn:stokesTheoremCore:80}{
\int_{\Ba}^{\Bb} d\Bx \cdot \spacegrad \psi = \psi(\Bb) - \psi(\Ba).
}
%\end{dmath}

         \subsection{Two parameter specialization of Stokes' theorem.}
            %
% Copyright © 2016 Peeter Joot.  All Rights Reserved.
% Licenced as described in the file LICENSE under the root directory of this GIT repository.
%

An example parameterization with two parameters, and the corresponding differentials with respect to those parameters, is plotted in
\cref{fig:twoParameterDifferential:twoParameterDifferentialFig1}.

\imageFigure{../figures/GAelectrodynamics/twoParameterDifferentialFig1}{Two parameter manifold differentials.}{fig:twoParameterDifferential:twoParameterDifferentialFig1}{0.4}

Given parameters \( a, b \), the differentials along each of the parameterization directions are

\begin{dmath}\label{eqn:stokesTheoremCore:100}
\begin{aligned}
d\Bx_a &= \PD{a}{\Bx} da = \Bx_a da \\
d\Bx_b &= \PD{b}{\Bx} db = \Bx_b db.
\end{aligned}
\end{dmath}

The ``volume'' element for this parameterization (a surface area element) is

\begin{equation}\label{eqn:stokesTheoremCore:120}
d^2 \Bx
=
d\Bx_a \wedge
d\Bx_b
=
da db (\Bx_a \wedge \Bx_b).
\end{equation}

The vector derivative, the projection of the gradient onto the surface at the point of integration (also called the tangent space), now has two components

\begin{dmath}\label{eqn:stokesTheoremCore:200}
\boldpartial
=
\sum_\mu \Bx^\mu (\Bx_\mu \cdot \spacegrad)
=
\Bx^a \PD{a}{}
+
\Bx^b \PD{b}{}
\equiv
\Bx^a \partial_a
+
\Bx^b \partial_b.
\end{dmath}

The Stokes integral can be evaluated over this volume element for either scalar fields \( \psi \) or vector fields \( \Bf \), and takes the form

\begin{subequations}
\label{eqn:stokesTheoremCore:140}
\begin{equation}\label{eqn:stokesTheoremCore:160}
\int_A d^2 \Bx \cdot (\boldpartial \wedge \psi) =
\int_A (d^2 \Bx \cdot \boldpartial) \psi
=
\int_{\partial A} d^1 \Bx \psi
\end{equation}
\begin{equation}\label{eqn:stokesTheoremCore:180}
\int_A d^2 \Bx \cdot (\boldpartial \wedge \Bf) =
\int_A (d^2 \Bx \cdot \boldpartial) \cdot \Bf
=
\int_{\partial A} d^1 \Bx \cdot \Bf.
\end{equation}
\end{subequations}

To extract the full meaning of this the boundary differential \( d^1 \Bx \) must be computed.  This has the same structure for a vector or scalar field

\begin{dmath}\label{eqn:stokesTheoremCore:220}
\begin{aligned}
\int_A d^2 \Bx \cdot (\boldpartial \wedge \Bf)
&=
\int_A (d^2 \Bx \cdot \boldpartial) \cdot \Bf \\
&=
\sum_\mu \int_A (d^2 \Bx \cdot \Bx^\mu) \cdot \partial_\mu \Bf \\
&=
\sum_\mu \int_A da db  \lr{ \Bx_a \wedge \Bx_b ) \cdot \Bx^\mu } \cdot \partial_\mu \Bf \\
&=
\sum_\mu \int_A da db  \lr{ \Bx_a {\delta_b}^\mu - \Bx_b {\delta_a}^\mu } \cdot \partial_\mu \Bf \\
&=
\int_A da db  \lr{ \Bx_a \cdot \PD{b}{ \Bf} - \Bx_b \cdot \PD{a}{\Bf} }
\end{aligned}
\end{dmath}

While \( \Bx_a, \Bx_b \) both depend on both parameters \( a, b \), the differential form immediately above is still a perfect integral in the variables of the partials since \( \Bx_a \) is computed with \( b \) held fixed, and \( \Bx_b \) is computed with \( a \) held fixed.  Proceeding with the integrals that match the respective partials, this gives

\begin{dmath}\label{eqn:stokesTheoremCore:240}
\int_A d^2 \Bx \cdot (\boldpartial \wedge \Bf)
=
\int
da \Bx_a \cdot \evalbar{\Bf}{\Delta b}
-\int
db \Bx_b \cdot \evalbar{\Bf}{\Delta a}
=
\int
d\Bx_a \cdot \evalbar{\Bf}{\Delta b}
-\int
d\Bx_b \cdot \evalbar{\Bf}{\Delta a}.
\end{dmath}

This shows that the boundary differential \( d^1 \Bx \) in \cref{eqn:stokesTheoremCore:140} is given by

\begin{dmath}\label{eqn:stokesTheoremCore:260}
d^1 \Bx = d\Bx_a - d\Bx_b,
\end{dmath}

where it is implied that the field in question is evaluated at the boundaries of the parameter that has been eliminated by this first integration.  This boundary integral can be interpreted as the integral around a contour, as indicated in
\cref{fig:twoParameterDifferentialBoundary:twoParameterDifferentialBoundaryFig2}.

\imageFigure{../figures/GAelectrodynamics/twoParameterDifferentialBoundaryFig2}{Contour for two parameter surface boundary.}{fig:twoParameterDifferentialBoundary:twoParameterDifferentialBoundaryFig2}{0.4}

Additionally, as with the single parameter case, a substitution of the gradient does not change the result, since any component of the gradient that lies outside of the tangent space on the surface has a zero dot product with the surface volume element \( d^2 \Bx \).
This allows the two parameter Stokes integrals to be written as

%\begin{dmath}\label{eqn:stokesTheoremCore:280}
\boxedEquation{eqn:stokesTheoremCore:280}{
\begin{aligned}
\int_A d^2 \Bx \cdot \spacegrad \psi &= \ointclockwise d\Bx \psi \\
\int_A d^2 \Bx \cdot (\spacegrad \wedge \Bf) &= \ointclockwise d\Bx \cdot \Bf.
\end{aligned}
}
%\end{dmath}

It can be shown that this two parameter Stokes integral is equivalent to Green's theorem.

         \subsection{Three parameter specialization of Stokes' theorem.}
            %
% Copyright © 2016 Peeter Joot.  All Rights Reserved.
% Licenced as described in the file LICENSE under the root directory of this GIT repository.
%

An example parameterization with three parameters, and the corresponding differentials with respect to those parameters, and the outwards normals, are sketched in
\cref{fig:normalsOnVolumeAreaElement:normalsOnVolumeAreaElementFig11}.

\imageFigure{../figures/gabook/normalsOnVolumeAreaElementFig11}{Three parameter volume element.}{fig:normalsOnVolumeAreaElement:normalsOnVolumeAreaElementFig11}{0.4}

Given parameters \( a, b, c \), the differentials along each of the parameterization directions are

\begin{dmath}\label{eqn:stokesTheoremCore:1421}
\begin{aligned}
d\Bx_a &= \PD{a}{\Bx} da = \Bx_a da \\
d\Bx_b &= \PD{b}{\Bx} db = \Bx_b db \\
d\Bx_c &= \PD{c}{\Bx} dc = \Bx_c dc.
\end{aligned}
\end{dmath}

The ``volume'' element for this parameterization (a surface area element) is

\begin{equation}\label{eqn:stokesTheoremCore:1441}
d^3 \Bx
=
d\Bx_a
\wedge
d\Bx_b
\wedge
d\Bx_c
=
da db dc (\Bx_a \wedge \Bx_b \wedge \Bx_c).
\end{equation}

The vector derivative, the projection of the gradient onto the surface at the point of integration (also called the tangent space), now has three components

\begin{dmath}\label{eqn:stokesTheoremCore:1461}
\boldpartial
=
\sum_\mu \Bx^\mu (\Bx_\mu \cdot \spacegrad)
=
\Bx^a \PD{a}{}
+
\Bx^b \PD{b}{}
+
\Bx^c \PD{c}{}
\equiv
\Bx^a \partial_a
+
\Bx^b \partial_b
+
\Bx^c \partial_c.
\end{dmath}

The Stokes integral can be evaluated over this volume element for either scalar fields \( \psi \), vector fields \( \Bf \), or bivector fields \( B \) and takes the form

\begin{subequations}
\label{eqn:stokesTheoremCore:1481}
\begin{equation}\label{eqn:stokesTheoremCore:1501}
\int_V d^3 \Bx \cdot (\boldpartial \wedge \psi) =
\int_V (d^3 \Bx \cdot \boldpartial) \psi
=
\int_{\partial V} d^2 \Bx \psi
\end{equation}
\begin{equation}\label{eqn:stokesTheoremCore:1521}
\int_V d^3 \Bx \cdot (\boldpartial \wedge \Bf) =
\int_V (d^3 \Bx \cdot \boldpartial) \cdot \Bf
=
\int_{\partial V} d^2 \Bx \cdot \Bf
\end{equation}
\begin{equation}\label{eqn:stokesTheoremCore:1541}
\int_V d^3 \Bx \cdot (\boldpartial \wedge B) =
\int_V (d^3 \Bx \cdot \boldpartial) \cdot B
=
\int_{\partial V} d^2 \Bx \cdot B.
\end{equation}
\end{subequations}

When working with \R{3} vector spaces, \( \boldpartial = \spacegrad \), but in higher dimensional spaces, the gradient can also be substituted above due using the same arguments about projection onto the tangent space.

An explicit value for the differential form of the boundary integral is desired and can be obtained from the mnemonic \cref{eqn:stokesTheoremCore:1561}

\begin{dmath}\label{eqn:stokesTheoremCore:1581}
\sum_i d^3 \Bx \cdot \Bx^i
=
\sum_i da db dc \lr{ \Bx_a \wedge \Bx_b \wedge \Bx_c } \cdot \Bx^i
=
\sum_i da db dc \lr{
\Bx_a \wedge \Bx_b +
\Bx_b \wedge \Bx_c +
\Bx_c \wedge \Bx_a }.
\end{dmath}

The bounding form for the three parameter volume is therefore

\begin{dmath}\label{eqn:stokesTheoremCore:1601}
d^2 \Bx
=
d\Bx_a \wedge d\Bx_b +
d\Bx_b \wedge d\Bx_c +
d\Bx_c \wedge d\Bx_a.
\end{dmath}

         \subsection{Using scalar volume elements}
            %
% Copyright © 2016 Peeter Joot.  All Rights Reserved.
% Licenced as described in the file LICENSE under the root directory of this GIT repository.
%

FIXME: remove most of this and introduce inline with the oriented area and volume results.  This is already done for the \( d^2 \Bx \) integrals.

In \R{3} the area elements of
(FIXME: equation reference dead with rewrite)
%\cref{eqn:twoparameter:140}
, and volume elements of 
\cref{eqn:threeparameter:1481}
can be reexpressed as scalars, recovering a number of the integral calculus identities that are more familiar than the wedge product variants above.

The pseudoscalar volume element can be written

\begin{dmath}\label{eqn:scalarVolumeElement:1621}
d^3 \Bx = I dV,
\end{dmath}
and the (oriented) area elements can be written as

\begin{dmath}\label{eqn:scalarVolumeElement:1641}
d^2 \Bx \ncap = I dA,
\end{dmath}
or
\begin{dmath}\label{eqn:scalarVolumeElement:1661}
d^2 \Bx = I \ncap dA.
\end{dmath}

For \( \psi \in \bigwedge^0, \Bf \in \bigwedge^1, B \in \bigwedge^2 \), this gives

\begin{subequations}
\label{eqn:scalarVolumeElement:1681}
\begin{equation}\label{eqn:scalarVolumeElement:1701}
I \int_A dA \ncap \wedge \spacegrad \psi = \ointclockwise d\Bx \psi
\end{equation}
\begin{equation}\label{eqn:scalarVolumeElement:1721}
I \int_A dA \ncap \wedge \spacegrad \wedge \Bf = \ointclockwise d\Bx \cdot \Bf
\end{equation}
\begin{equation}\label{eqn:scalarVolumeElement:1741}
\int_V dV \spacegrad \psi = \int_{\partial V} dA \ncap \psi
\end{equation}
\begin{equation}\label{eqn:scalarVolumeElement:1761}
\int_V dV \spacegrad \wedge \Bf = \int_{\partial V} dA \ncap \wedge \Bf
\end{equation}
\begin{equation}\label{eqn:scalarVolumeElement:1781}
\int dV \spacegrad \wedge B = \int_{\partial V} dA \ncap \wedge B
\end{equation}
\end{subequations}

It is straightforward to re-express all the wedge products above in their dual forms.
With \( B = I \Bf \), that is

\begin{subequations}
\label{eqn:scalarVolumeElement:1801}
\begin{equation}\label{eqn:scalarVolumeElement:1821}
\int_A dA \ncap \cross \spacegrad \psi = \ointctrclockwise d\Bx \psi
\end{equation}
\begin{equation}\label{eqn:scalarVolumeElement:1841}
\int_A dA \ncap \cdot (\spacegrad \cross \Bf) = \ointctrclockwise d\Bx \cdot \Bf
\end{equation}
\begin{equation}\label{eqn:scalarVolumeElement:1861}
\int_V dV \spacegrad \psi = \int_{\partial V} dA \ncap \psi
\end{equation}
\begin{equation}\label{eqn:scalarVolumeElement:1881}
\int_V dV \spacegrad \cross \Bf = \int_{\partial V} dA \ncap \cross \Bf
\end{equation}
\begin{equation}\label{eqn:scalarVolumeElement:1901}
\int dV \spacegrad \cdot \Bf = \int_{\partial V} dA \ncap \cdot \Bf.
\end{equation}
\end{subequations}

Each of the cross product terms above can also be put into dual forms, giving

\begin{subequations}
\label{eqn:scalarVolumeElement:1801c}
\begin{equation}\label{eqn:scalarVolumeElement:1821c}
\int_A dA \ncap \cdot \lr{ I \spacegrad \psi } = \ointclockwise d\Bx \psi
\end{equation}
\begin{equation}\label{eqn:scalarVolumeElement:1841c}
\int_A dA \ncap \cdot (\spacegrad \cdot B) = \ointctrclockwise d\Bx \cdot (I B)
\end{equation}
\begin{equation}\label{eqn:scalarVolumeElement:1881c}
\int_V dV \spacegrad \cdot B = \int_{\partial V} dA \ncap \cdot B.
\end{equation}
\end{subequations}

Note that all of
\cref{eqn:scalarVolumeElement:1861}, \cref{eqn:scalarVolumeElement:1901}, and \cref{eqn:scalarVolumeElement:1881c} all have the same form

%\begin{equation}\label{eqn:scalarVolumeElement:1881d}
\boxedEquation{eqn:scalarVolumeElement:1881d}{
\int_V dV \spacegrad \cdot A = \int_{\partial V} dA \ncap \cdot A.
}
%\end{equation}
\index{divergence theorem}

This is also true for pseudoscalar grades, which can be demonstrated by multiplying both sides of \cref{eqn:scalarVolumeElement:1741} with \( I \).
This implies that \cref{eqn:scalarVolumeElement:1881d} is valid for any \R{3} multivector, generalizing the conventional divergence theorem over a 3D volume to all spatial grades.

         \subsection{Problems}
            %
% Copyright � CCYY Peeter Joot.  All Rights Reserved.
% Licenced as described in the file LICENSE under the root directory of this GIT repository.
%
\makeproblem{Stokes' theorem relation to Green's theorem}{problem:stokesAndGreens:1}{
Show that Stokes' theorem, in its two parameter form, applied to a vector field recovers Green's theorem.
\index{Green's theorem}
\index{Stokes' theorem}
} % problem

\makeanswer{problem:stokesAndGreens:1}{

To demonstrate this, expand the LHS of the Stokes identity

\begin{dmath}\label{eqn:stokesAndGreens:20}
\int_A d^2 \Bx \cdot (\boldpartial \wedge \Bf) = \ointclockwise d\Bx \cdot \Bf.
\end{dmath}

Assuming \( u, v\) parameterization

\begin{dmath}\label{eqn:stokesAndGreens:40}
\int_A d^2 \Bx \cdot (\boldpartial \wedge \Bf)
=
\int_A (d\Bx_u \wedge d\Bx_v) \cdot (\boldpartial \wedge \Bf)
=
\int_A ((d\Bx_u \wedge d\Bx_v) \cdot \Bx^u) \cdot \partial_u \Bf
+
\int_A ((d\Bx_u \wedge d\Bx_v) \cdot \Bx^v) \cdot \partial_v \Bf
=
-\int_A du dv \Bx_v \cdot \partial_u \Bf
+
\int_A du dv \Bx_u \cdot \partial_v \Bf
=
-\int_A du dv \Bx_v \cdot \partial_u \Bf
+
\int_A du dv \lr{
-\Bx_v \cdot \partial_u \Bf
+
\Bx_u \cdot \partial_v \Bf
}.
\end{dmath}

The coordinate expansion of \( \Bf \) with respect to the tangent space coordinates is

\begin{dmath}\label{eqn:stokesAndGreens:60}
\Bf = \Bx^u f_u + \Bx^v f_v + \Bf_\perp
\end{dmath}

where \( \Bf_\perp \) lies in normal to the tangent space at the point in question.
Because \( \Bx_v \) is computed with \( u \) held fixed and \( \Bx_u \) computed with \( v \) held fixed, the area integrand can be written

\begin{dmath}\label{eqn:stokesAndGreens:80}
-\Bx_v \cdot \partial_u \Bf
+
\Bx_u \cdot \partial_v \Bf
=
-\PD{u}{}\lr{ \Bx_v \cdot \Bf }
+\PD{v}{}\lr{ \Bx_u \cdot \Bf }
=
-\PD{u}{f_v}
+\PD{v}{f_u},
\end{dmath}

which gives
\begin{dmath}\label{eqn:stokesAndGreens:100}
\int_A du dv \lr{ -\PD{u}{f_v}
+\PD{v}{f_u}
}
=
\ointclockwise d\Bx \cdot \Bf,
\end{dmath}

which recovers \cref{thm:stokesTheoremGeometricAlgebra:1660} as desired.
} % answer

            %
% Copyright © 2016 Peeter Joot.  All Rights Reserved.
% Licenced as described in the file LICENSE under the root directory of this GIT repository.
%

\makeproblem{\R{3} dual forms of Stokes' theorem.}{problem:stokesTheoremCoreProblems:1}{
Prove
\makesubproblem{}{problem:stokesTheoremCoreProblems:1:a}
\cref{eqn:scalarVolumeElement:1681},
\makesubproblem{}{problem:stokesTheoremCoreProblems:1:b}
\cref{eqn:scalarVolumeElement:1801},
\makesubproblem{}{problem:stokesTheoremCoreProblems:1:c}
and \cref{eqn:scalarVolumeElement:1801c}.
} % problem

\makeanswer{problem:stokesTheoremCoreProblems:1}{

The volume elements are
\makeSubAnswer{}{problem:stokesTheoremCoreProblems:1:a}
\begin{subequations}
\label{eqn:stokesTheoremCoreProblems:20}
\begin{dmath}\label{eqn:stokesTheoremCoreProblems:40}
d^2 \Bx \cdot \spacegrad
=
dA \gpgradeone{ I \ncap \spacegrad }
=
dA I \ncap \wedge \spacegrad
\end{dmath}
\begin{dmath}\label{eqn:stokesTheoremCoreProblems:60}
d^2 \Bx \cdot (\spacegrad \wedge \BA)
=
dA \gpgradezero{ I \ncap \spacegrad \BA }
=
dA I \ncap \wedge \spacegrad \wedge \BA
\end{dmath}
\begin{dmath}\label{eqn:stokesTheoremCoreProblems:80}
d^3 \Bx \cdot \spacegrad \phi
=
dV \gpgradetwo{ I \spacegrad \phi }
=
dV I \spacegrad \phi
\end{dmath}
\begin{dmath}\label{eqn:stokesTheoremCoreProblems:100}
d^3 \Bx \cdot (\spacegrad \wedge \BA)
=
dV \gpgradeone{ I (\spacegrad \wedge \BA) }
=
dV I \spacegrad \wedge \BA
\end{dmath}
\begin{dmath}\label{eqn:stokesTheoremCoreProblems:120}
d^3 \Bx \cdot (\spacegrad \wedge B)
=
dV \gpgradezero{ I (\spacegrad \wedge B) }
=
dV I \spacegrad \wedge B.
\end{dmath}
\end{subequations}

The corresponding boundary forms are
\begin{subequations}
\label{eqn:stokesTheoremCoreProblems:140}
\begin{equation}\label{eqn:stokesTheoremCoreProblems:160}
d\Bx \psi
\end{equation}
\begin{dmath}\label{eqn:stokesTheoremCoreProblems:180}
d\Bx \cdot \BA
\end{dmath}
\begin{dmath}\label{eqn:stokesTheoremCoreProblems:200}
d^2 \Bx \psi
=
dA I \ncap \psi
\end{dmath}
\begin{dmath}\label{eqn:stokesTheoremCoreProblems:220}
d^2 \Bx \cdot \BA
=
dA \gpgradeone{ I \ncap \BA }
=
dA I \ncap \wedge \BA
\end{dmath}
\begin{dmath}\label{eqn:stokesTheoremCoreProblems:240}
d^2 \Bx \cdot B
=
dA \gpgradezero{ I \ncap B }
=
dA I \ncap \wedge B.
\end{dmath}
\end{subequations}

Assembling these pieces back into the integrals proves the relationships.

\makeSubAnswer{}{problem:stokesTheoremCoreProblems:1:b}

To show \cref{eqn:scalarVolumeElement:1841} note that

\begin{dmath}\label{eqn:stokesTheoremCoreProblems:260}
I (\Ba \wedge \Bb \wedge \Bc)
=
\gpgradezero{ I \Ba \wedge \Bb \wedge \Bc }
=
\gpgradezero{ I \Ba (\Bb \wedge \Bc) -
I \Ba \cdot (\Bb \wedge \Bc)
}
=
\gpgradezero{ I \Ba I(\Bb \cross \Bc) }
=
- \Ba \cdot (\Bb \cross \Bc).
\end{dmath}

To show \cref{eqn:scalarVolumeElement:1901} note that

\begin{dmath}\label{eqn:stokesTheoremCoreProblems:280}
\Ba \wedge (I \BA)
=
\Ba \wedge (I \BA)
=
\gpgradethree{ \Ba I \BA }
=
\gpgradethree{ I \Ba \cdot \BA }
=
I (\Ba \cdot \BA).
\end{dmath}

\makeSubAnswer{}{problem:stokesTheoremCoreProblems:1:c}

For vector \( \Ba \), these transformations all follow from

\begin{dmath}\label{eqn:stokesTheoremCoreProblems:300}
\Ba \cross \Bf
=
\gpgradeone{ -I \Ba \wedge \Bf}
=
\gpgradeone{ -I \Ba \Bf}
=
-\gpgradeone{ \Ba I \Bf}
=
-\Ba \cdot (I \Bf)
=
\Ba \cdot B.
\end{dmath}

} % answer


      \section{Fundamental theorem of geometric calculus.}
         \subsection{Fundamental Theorem of Geometric Calculus.}
            %
% Copyright � 2016 Peeter Joot.  All Rights Reserved.
% Licenced as described in the file LICENSE under the root directory of this GIT repository.
%
%{
%\input{../blogpost.tex}
%\renewcommand{\basename}{fundamentalTheoremOfCalculus}
%\renewcommand{\dirname}{notes/phy1520/}
%%\newcommand{\dateintitle}{}
%%\newcommand{\keywords}{}
%
%\input{../peeter_prologue_print2.tex}
%
%\usepackage{peeters_layout_exercise}
%\usepackage{peeters_braket}
%\usepackage{peeters_figures}
%\usepackage{siunitx}
%
%\beginArtNoToc
%
%\generatetitle{Fundamental theorem of geometric calculus}
%\label{chap:fundamentalTheoremOfCalculus}

\subsection{Hypervolume integral}
We wish to generalize the concepts of line, surface and volume integrals to hypervolumes and multivector functions, and define a hypervolume integral as

\makedefinition{Multivector integral.}{dfn:fundamentalTheoremOfCalculus:240}{
Given a hypervolume parameterized by \( k \) parameters, k-volume volume element \( d^k \Bx \), and
multivector functions \( F, G \), we define k-volume integral with the vector derivative acting to the right on \( F \) as
\begin{equation*}
\int d^k\Bx \lr{ \rboldpartial F },
\end{equation*}
a k-volume integral with the vector derivative acting to the left \( F \) as
\begin{equation*}
\int F d^k\Bx \lboldpartial,
\end{equation*}
and a k-volume integral with the vector derivative acting bidirectionally on \( F, G \) as
\begin{equation*}
\int F d^k\Bx \lrboldpartial G
\equiv
\int \lr{ F d^k\Bx \lboldpartial} G
+
\int F d^k\Bx \lr{ \rboldpartial G },
\end{equation*}
where the meaning given to these directionally acting derivative operations is
\begin{equation*}
F d^k \Bx \lrboldpartial G
=
F d^k \Bx \lr{ \sum_i \Bx^i {\stackrel{ \leftrightarrow }{\partial_i}} } G
=
(\partial_i F) d^k \Bx \sum_i \Bx^i G
+
F d^k \Bx \sum_i \Bx^i (\partial_i G)
\equiv
(F d^k \Bx \lboldpartial) G
+
F d^k \Bx (\rboldpartial G),
\end{equation*}
with \( \boldpartial \) acting on \( F \) and \( G \), but not the volume element \( d^k \Bx \), which may also be a function of the implied parameterization.
} % definition

The vector derivative (and gradient)
may not commute with \( F, G \) nor the volume element \( d^k \Bx \), so we are forced to use some notation to indicate what the vector derivative (or gradient) acts on.
In conventional right acting cases, where there is no ambiguity, arrows will usually be omitted, but braces may also be used to indicate the scope of derivative operators.
This bidirectional notation will also be used for the gradient, especially for volume integrals in \R{3} where the vector derivative is identitical to the gradient.

Some authors use overdots or ticks are used to indicate the exact scope of multivector derivative operators, as in
\begin{dmath}\label{eqn:fundamentalTheoremOfCalculus:260}
F d^k \Bx \boldpartial G =
\dot{F} d^k \Bx \dot{\boldpartial} G
+
F d^k \Bx \dot{\boldpartial} \dot{G}.
\end{dmath}
Here the (Hestenes) dot notation would have the advantage of emphasizing that the action of the vector derivative (or gradient) is on the functions \( F, G \), and not on the hypervolume element \( d^k \Bx \).
In this book, where we will use ticks to indicate whether gradients are with respect to primed \( \Bx' \) or unprimed \( \Bx \) variables, over arrows seemed like a better choice than dots to indicate operator scope, and have the advantage of being visually conspicuous.

\subsection{Fundamental theorem.}
\index{fundamental theorem of geometric calculus}

The fundamental theorem of geometric calculus is a generalization of many conventional scalar and vector integral theorems.
It is a powerful theorem, which we will use with Green's functions to solve Maxwell's equation, and to derive the geometric algebra form of Stokes' theorem.

\maketheorem{Fundamental theorem of geometric calculus}{thm:fundamentalTheoremOfCalculus:1}{
For multivectors \(F, G \), and a hypervolume element \(d^k \Bx\),
\begin{equation*}
\int_V F d^k \Bx \boldpartial G = \oint_{\partial V} F d^{k-1} \Bx G.
\end{equation*}
}

This theorem relates the hypervolume integral to the integral over the bounding surface of hypervolume.
Additional work is required to describe the precise meaning of the boundary differential \( d^{k-1} \Bx \).  We will do so for line, surface, and volume integrals, proving the theorem in a limited fashion for each of those cases as we go.

For a full proof of \cref{thm:fundamentalTheoremOfCalculus:1}, additional mathematical sublties must be considered.
For full proofs and additional details, the reader is referred to \citep{hestenes1985clifford}, \citep{doran2003gap}, \citep{aMacdonaldVAGC} and \citep{sobczyk2011fundamental}, which all
which all tackle different aspects of general geometric calculus.

Before considering multivector line, surface and volume integral specializations of
\cref{thm:fundamentalTheoremOfCalculus:1},
we will state Stokes' theorem in its geometric algebra form.

%}
%\EndArticle

         \subsection{Green's function for the gradient in Euclidean spaces.}
            %\input{../gabookI/calculus/gradientGreensFunction.tex}
            %
% Copyright � 2016 Peeter Joot.  All Rights Reserved.
% Licenced as described in the file LICENSE under the root directory of this GIT repository.
%
%{
\index{Green's function}

\subsection{Motivation.}

We will now introduce Green's functions, which provide a general method of solving many of the linear differential equations that will be encountered in electromagnetism.

One such linear differential equation is the inhomogeneous wave equation

\begin{dmath}\label{eqn:gradientGreensFunctionEuclidean:162}
\lr{ \spacegrad^2 - \inv{c^2} \PDSq{t}{} } F(\Bx, t) = B(\Bx, t)
\end{dmath}

The time harmonic (frequency domain) representation of the wave equation can be found by assuming a that fixed frequency solution exists.
In complex notation, that means that we can assume that all sources and fields have a complex exponential time dependence of the form
\footnote{This is the engineering convention for the time dependence.
The reader must take care when reading the literature, since some authors (notably in physics) use the opposite sign convention
\( F(\Bx, t) = \Real\lr{ F(\Bx) e^{-i \omega t }} \).}

\index{time harmonic}
\index{frequency domain}
\begin{dmath}\label{eqn:gradientGreensFunctionEuclidean:60}
F(\Bx, t) = \Real\lr{ F(\Bx) e^{j \omega t} },
\end{dmath}

where \( j \) is a scalar imaginary that need not be represented by any geometrical imaginary such as \( \Be_{123}, \Be_{12}, \cdots \).
After substitution of the time harmonic representation into \cref{eqn:gradientGreensFunctionEuclidean:162}, the problem is reduced to finding a solution that is a function of space and time to one that is purely spatial

\begin{dmath}\label{eqn:gradientGreensFunctionEuclidean:159}
\lr{ \spacegrad^2 + \frac{\omega^2}{c^2} } F(\Bx) = B(\Bx).
\end{dmath}

Superposition of discrete or continuous combinations of fixed frequency solutions, once found, can be used to determine more general solutions to the original wave equation \cref{eqn:gradientGreensFunctionEuclidean:162}.

We will writing \( \omega^2/c^2 = k^2 \), to obtain the standard form of the Helmholtz equation we wish to solve

\index{Helmholtz equation}
\index{second order Helmholtz equation}
\begin{dmath}\label{eqn:gradientGreensFunctionEuclidean:160}
\lr{ \spacegrad^2 + k^2 } F(\Bx) = B(\Bx).
\end{dmath}

This is a linear differential equation that is second order with respect to the gradient.
Despite employing a complex representations of the fields and sources, our vector basis is still a real valued Euclidean basis, and we will have no reason to introduce complex inner products spaces into the mix.
We will also encounter statics problems that have no time dependence in electromagnetism.
Some of these problems have the structure of \cref{eqn:gradientGreensFunctionEuclidean:160} with \( k = 0 \), and for those problems the fields and sources are real.

\index{Helmholtz operator}
Observe that the Helmholtz operator can be factored into operators that are first order in the gradient

\begin{dmath}\label{eqn:gradientGreensFunctionEuclidean:161}
\lr{ \spacegrad - j k }\lr{ \spacegrad + j k } F(\Bx) = B(\Bx).
\end{dmath}

We will see that the time harmonic Maxwell's equation, in its GA form, is a first order equation in the gradient of the form

\index{first order Helmholtz equation}
\begin{dmath}\label{eqn:gradientGreensFunctionEuclidean:180}
\lr{ \spacegrad + j k } F(\Bx) = J(\Bx),
\end{dmath}

where \( F \) is a (complex) 1,2 multivector, and \( J \) is a (complex) multivector containing all the charge and current density contributions.
Our initial goal is to develop the Green's function toolbox that can be used to solve first and second order Helmholtz equations of the form
\cref{eqn:gradientGreensFunctionEuclidean:180} and
\cref{eqn:gradientGreensFunctionEuclidean:160} respectively.

\subsection{Unbounded superposition solutions for the Helmholtz equation.}

We can utilize \cref{eqn:gradientGreensFunctionEuclidean:160} to illustrate the Green's function technique.
As this equation is a linear differential operator relating the wave and the driving sources,
it is reasonable to assume that the solution also has a general linear structure, such as

\begin{dmath}\label{eqn:gradientGreensFunctionEuclidean:100}
F(\Bx) = \int dV' B(\Bx') G(\Bx, \Bx') + F_0(\Bx),
\end{dmath}

where the function \( G(\Bx, \Bx') \) is called the Green's function for the Helmholtz operator, and \( F_0 \) is any particular solution to the inhomogeneous Helmholtz equation \( \lr{ \spacegrad^2 + k^2 } F_0 = 0 \).
Operating on \cref{eqn:gradientGreensFunctionEuclidean:100} with the Helmholtz operator \( \spacegrad + k^2 \) we find that the Green's function must
satisfy

\begin{dmath}\label{eqn:gradientGreensFunctionEuclidean:140}
\lr{ \spacegrad^2 + k^2 } G(\Bx, \Bx') = \delta(\Bx - \Bx').
\end{dmath}

While it is possible \citep{schwinger1998classical} to derive the Green's function using Fourier transform techniques, we will state the result instead, which is well known

\index{Helmholtz!Green's function}
\index{Green's function!Helmholtz}
\maketheorem{Green's function for the Helmholtz operator.}{thm:gradientGreensFunctionEuclidean:3}{
The advancing (causal), and the receding (acausal) Green's functions satisfying
\cref{eqn:gradientGreensFunctionEuclidean:140} are respectively

\begin{equation*}
\begin{aligned}
G_{\textrm{adv}}(\Bx, \Bx') &= -\frac{e^{-j k \Norm{ \Bx - \Bx' } }}{ 4 \pi \Norm{\Bx - \Bx'}} \\
G_{\textrm{rec}}(\Bx, \Bx') &= -\frac{e^{j k \Norm{ \Bx - \Bx' } }}{ 4 \pi \Norm{\Bx - \Bx'}}.
\end{aligned}
\end{equation*}
} % theorem

We will use the advancing Green's function, and refer to this function as \( G(\Bx, \Bx') \) without any subscript.
A demonstration that these Green's function representations are valid can be found in \cref{chap:helmholtzGreens}.

\index{Laplacian!Green's function}
\index{Green's function!Laplacian}
Observe that as a special case, the Helmholtz Green's function reduces to the Green's function for the Laplacian when \( k = 0 \)

\begin{dmath}\label{eqn:gradientGreensFunctionEuclidean:80}
G(\Bx, \Bx') = -\inv{ 4 \pi \Norm{\Bx - \Bx'}}.
\end{dmath}

\subsection{Bounded superposition solutions for the Helmholtz equation.}

When the presumed solution is a superposition of only states in a bounded region, such as

\begin{dmath}\label{eqn:gradientGreensFunctionEuclidean:200}
F(\Bx) = \int_V dV' B(\Bx') G(\Bx, \Bx') + F_0(\Bx),
\end{dmath}

then life gets a bit more interesting.
For such problems, we require Green's theorem, which must be generalized slightly for use with multivector fields.
The basic idea is that we can relate the Laplacian's of the Green's function and the field
\( F(\Bx') (\spacegrad')^2 G(\Bx, \Bx') = G(\Bx, \Bx') (\spacegrad')^2 F(\Bx') + \cdots \).
That relation is usually expressed in terms of the difference of the two in the integral domain

\maketheorem{Green's theorem}{thm:gradientGreensFunctionEuclidean:220}{
Given a multivector function \( F \) and a scalar function \( G \)
\begin{equation*}
\int_V \lr{ F \spacegrad^2 G - G \spacegrad^2 F } dV = \int_{\partial V} \lr{ F \ncap \cdot \spacegrad G - G \ncap \cdot \spacegrad F },
\end{equation*}
where \( \partial V \) is the boundary of the volume \( V \).
} % theorem

A straightforward, but perhaps inelligant way of proving this theorem is to expand the sandwich difference in coordinates

\begin{dmath}\label{eqn:gradientGreensFunctionEuclidean:260}
F \spacegrad^2 G - G \spacegrad^2 F
=
\sum_k F \partial_k \partial_k G - G \partial_k \partial_k F
=
\sum_k \partial_k \lr{
F \partial_k G - G \partial_k F
}
-
(\partial_k F)(\partial_k G) + (\partial_k G)(\partial_k F).
\end{dmath}

Since \( G \) is a scalar, the last two terms cancel, and we can integrate

\begin{dmath}\label{eqn:gradientGreensFunctionEuclidean:280}
\int_V \lr{ F \spacegrad^2 G - G \spacegrad^2 F } dV
=
\sum_k \int_V \partial_k \lr{ F \partial_k G - G \partial_k F }.
\end{dmath}

Each integral above involves one component of the gradient.
From
%the fundamental theorem of geometric calculus
\cref{thm:fundamentalTheoremOfCalculus:1}
we know that
\begin{dmath}\label{eqn:gradientGreensFunctionEuclidean:300}
\int_V \spacegrad Q dV = \int_{\partial V} \ncap Q dA,
\end{dmath}

for any multivector \( Q \).
Equating components gives

\begin{dmath}\label{eqn:gradientGreensFunctionEuclidean:340}
\int_V \partial_k Q dV = \int_{\partial V} \ncap \cdot \Be_k Q dA,
\end{dmath}

which can be substituted into \cref{eqn:gradientGreensFunctionEuclidean:280} to find

\begin{dmath}\label{eqn:gradientGreensFunctionEuclidean:360}
\int_V \lr{ F \spacegrad^2 G - G \spacegrad^2 F } dV
=
\sum_k \int_{\partial V} \ncap \cdot \Be_k \lr{ F \partial_k G - G \partial_k F } dA
=
\int_{\partial V} \lr{ F (\ncap \cdot \spacegrad) G - G (\ncap \cdot \spacegrad) F } dA,
\end{dmath}

which proves the theorem.
For our application of
{thm:gradientGreensFunctionEuclidean:3} to the Helmholtz problem, we
are actually interested in a antisymmetric sandwich of the Helmholtz operator by the function \( F \) and the scalar (Green's) function \( G \), but
that reduces to a sandwich of Laplacian's

\begin{dmath}\label{eqn:gradientGreensFunctionEuclidean:240}
F \lr{ \spacegrad^2 + k^2 } G - G \lr{ \spacegrad^2 + k^2 } F
=
F \spacegrad^2 G + \cancel{F k^2 G} - G \spacegrad^2 F - \cancel{G k^2 F}
=
F \spacegrad^2 G - G \spacegrad^2 F,
\end{dmath}

so

\begin{dmath}\label{eqn:gradientGreensFunctionEuclidean:380}
\int_V F(\Bx') \lr{ (\spacegrad')^2 + k^2 } G(\Bx, \Bx')
=
\int_V G(\Bx, \Bx') \lr{ (\spacegrad')^2 + k^2} F(\Bx') dV'
+
\int_{\partial V} \lr{ F(\Bx') (\ncap' \cdot \spacegrad') G(\Bx, \Bx') - G(\Bx, \Bx') (\ncap' \cdot \spacegrad') F(\Bx') } dA'
\end{dmath}

This shows that if we assume the Green's function satisfies
the delta function condition
\cref{eqn:gradientGreensFunctionEuclidean:140}
%that was also true for the unbounded case
, then the general solution to \cref{eqn:gradientGreensFunctionEuclidean:160} is

\boxedEquation{eqn:gradientGreensFunctionEuclidean:400}{
\begin{aligned}
F(\Bx) &=
\int_V G(\Bx, \Bx') B( \Bx' ) dV' \\
&+
\int_{\partial V} \lr{
 G(\Bx, \Bx') (\ncap' \cdot \spacegrad') F(\Bx')
-F(\Bx') (\ncap' \cdot \spacegrad') G(\Bx, \Bx')
} dA'.
\end{aligned}
}

We are also free to add in any specific solution \( F_0(\Bx) \) that satisfies the
homogeneous Helmholtz equation.
There is also freedom to add any solution of the homogeneous Helmholtz equation to the Green's function itself, so it is not unique.
For a bounded superposition we generally desire that the solution \( F \) and its normal derivative, or the Green's function \( G \) (and it's normal derivative) or an appropriate combination of the two are zero on the boundary, so that the surface integral is killed.

\subsection{First order gradient Green's functions.}

\index{gradient!Green's function}
\index{Green's function!gradient representation}

We will see that the GA formulation of the statics equations (no time dependence), all have the form

\begin{dmath}\label{eqn:gradientGreensFunctionEuclidean:420}
\spacegrad F(\Bx) = J(\Bx),
\end{dmath}

where \( F, J \) are multivector fields and sources respectively.  We can assume an unbounded superposition solution

\begin{dmath}\label{eqn:gradientGreensFunctionEuclidean:440}
F(\Bx) = \int G(\Bx, \Bx') J(\Bx') dV' + F_0(\Bx),
\end{dmath}

where \( F_0 \) is any solution to the homogeneous gradient equation \( \spacegrad F_0 = 0 \), and operate on this presumed solution with the gradient to find

\begin{dmath}\label{eqn:gradientGreensFunctionEuclidean:460}
J(\Bx)
= \int \spacegrad G(\Bx, \Bx') J(\Bx') dV',
\end{dmath}

so the Green's function \( G \) for this system must satisfy

\begin{dmath}\label{eqn:gradientGreensFunctionEuclidean:480}
\spacegrad G(\Bx, \Bx') = \delta( \Bx - \Bx' ).
\end{dmath}

We will now show that this Green's function is vector valued as follows

\maketheorem{Green's function for the gradient}{thm:gradientGreensFunctionEuclidean:1}{
A Green's function that satisfies \cref{eqn:gradientGreensFunctionEuclidean:480} is
\begin{equation*}
   G(\Bx, \Bx') = \inv{4 \pi} \frac{\Bx - \Bx'}{\Norm{\Bx-\Bx'}^3}.
\end{equation*}
At points \( \Bx \ne \Bx' \), \( \spacegrad \wedge G = 0 \), or
\( \rspacegrad G = G \lspacegrad \).
} % theorem

To prove this, observe that we know the Laplacian representation of the delta function, so
the Green's function for the gradient can be written as

\begin{equation}\label{eqn:gradientGreensFunctionEuclidean:481}
\spacegrad G(\Bx, \Bx') = \delta( \Bx - \Bx' ) = \spacegrad^2 \lr{ -\inv{4\pi} \inv{ \Norm{\Bx - \Bx'} } }.
\end{equation}

GA provides us the rather beautiful and remarkable ability to factor the Laplacian in to a product of gradients \( \spacegrad^2 = \spacegrad \spacegrad \), so the gradient's Green's function is

\begin{dmath}\label{eqn:gradientGreensFunctionEuclidean:580}
G(\Bx, \Bx')
= \spacegrad \lr{ -\inv{4\pi} \inv{ \Norm{\Bx - \Bx'} } }
= -\inv{4\pi} \rcap \PD{r}{} \inv{r},
\end{dmath}

where \( \Br = \Bx - \Bx', r = \Norm{\Br} \), and \( \rcap = \Br/r \).  Proceeding with the derivatives, we find
\begin{dmath}\label{eqn:gradientGreensFunctionEuclidean:600}
G(\Bx, \Bx')
= -\inv{4\pi} \rcap \lr{ -\inv{r^2} }
= \frac{\rcap}{4 \pi r^2},
\end{dmath}

as claimed.  To show that the Green's function commutes with the gradient at points \( \Bx \ne \Bx' \) we can compute the (bivector) curl

\begin{dmath}\label{eqn:gradientGreensFunctionEuclidean:620}
\spacegrad \wedge \frac{\Br}{r^3}
=
\lr{ \spacegrad \inv{r^3}} \wedge \Br
+
\inv{r^3} \spacegrad \wedge \Br.
\end{dmath}

Since \( \spacegrad \inv{r^m} \propto \rcap \) the first wedge is zero.  The second wedge is also zero, which is easily demonstrated by coordinate expansion
\begin{dmath}\label{eqn:gradientGreensFunctionEuclidean:640}
\spacegrad \wedge \Br
=
\sum_{m,n} (\Be_m \partial_m) \wedge (\Be_n r_n)
=
\sum_{m,n} (\Be_m \wedge \Be_n) \partial_m r_n
=
\sum_{m,n} (\Be_m \wedge \Be_n) \delta_{m n}.
\end{dmath}

This last sum is zero since it is the symmetric sum of an antisymmetric quantity, which completes the proof.

We can determine the structure of an unbounded superposition solution through application of
%the fundamental theorem of geometric calculus
\cref{thm:fundamentalTheoremOfCalculus:1}

\begin{dmath}\label{eqn:gradientGreensFunctionEuclidean:500}
\int_V G(\Bx, \Bx') \lrspacegrad' F(\Bx') dV'
=
\int_V (G(\Bx, \Bx') \lspacegrad') F(\Bx') dV'
+
\int_V G(\Bx, \Bx') (\rspacegrad' F(\Bx')) dV'
=
\int_{\partial V} G(\Bx, \Bx') \ncap' F(\Bx') dA'.
\end{dmath}

As with the Helmholtz equation, we can presuming that the Green's function of
\cref{thm:gradientGreensFunctionEuclidean:1}
for the unbounded superposition solution also applies here.  Since

\begin{dmath}\label{eqn:gradientGreensFunctionEuclidean:520}
G(\Bx, \Bx') \lspacegrad'
=
-G(\Bx', \Bx) \lspacegrad'
=
-
\spacegrad' G(\Bx', \Bx)
= -\delta(\Bx - \Bx'),
\end{dmath}

we have
\begin{dmath}\label{eqn:gradientGreensFunctionEuclidean:540}
-F(\Bx)
+
\int_V G(\Bx, \Bx') J(\Bx') dV'
=
\int_{\partial V} G(\Bx, \Bx') \ncap' F(\Bx') dA',
\end{dmath}

or
\boxedEquation{eqn:gradientGreensFunctionEuclidean:560}{
F(\Bx)
=
\int_V G(\Bx, \Bx') J(\Bx') dV'
-
\int_{\partial V} G(\Bx, \Bx') \ncap' F(\Bx') dA'.
}

We are also free to add any specific solution \( F_0 \) to the gradient equation \( \spacegrad F_0 = 0 \).
Because the Green's function is not unique (we can add any solution \( G_0 \) of the gradient equation \( \spacegrad G_0 = 0 \)),
it may be desirable for bounded problems to construct Green's functions that are zero on the boundary of the integration volume.

\subsection{Green's function for the first order Helmholtz equation.}

The non-zero frequency generalization of the gradient equations above are multivector equations of the form

\begin{dmath}\label{eqn:gradientGreensFunctionEuclidean:660}
\lr{ \spacegrad + j k } F(\Bx) = J(\Bx).
\end{dmath}

As above, this can be solved by assuming a superposition solution

\begin{dmath}\label{eqn:gradientGreensFunctionEuclidean:680}
F(\Bx) = \int G(\Bx, \Bx') J(\Bx') dV',
\end{dmath}

where
\begin{dmath}\label{eqn:gradientGreensFunctionEuclidean:700}
\lr{ \spacegrad + j k } G(\Bx, \Bx')  = \delta(\Bx - \Bx').
\end{dmath}

FIXME: this basic idea has been repeated over and over again.  Probably worth stating it once up front.

This Green's function is multivector valued

\maketheorem{Green's function for the first order Helmholtz operator.}{thm:gradientGreensFunctionEuclidean:720}{
The Green's function satisfying
\begin{equation*}
\lr{ \rspacegrad + j k } G(\Bx, \Bx') = G(\Bx, \Bx') \lr{ -\lspacegrad' + j k } = \delta(\Bx - \Bx'),
\end{equation*}
is
\begin{equation*}
G(\Bx, \Bx') = \frac{e^{-j k r}}{4 \pi r} \lr{ j k \lr{ 1 + \rcap } + \frac{\rcap}{r} },
\end{equation*}
where \( \Br = \Bx - \Bx', r = \Norm{\Br} \) and \( \rcap = \Br/r \).
} % theorem

If we denote the (advanced) Green's function for the 2nd order Helmholtz operator
\cref{thm:gradientGreensFunctionEuclidean:3}
as \( \phi(\Bx, \Bx') \), we must have

\begin{equation}\label{eqn:gradientGreensFunctionEuclidean:740}
\lr{ \rspacegrad + j k } G(\Bx, \Bx') = \delta(\Bx - \Bx') =
\lr{ \rspacegrad + j k } \lr{ \rspacegrad - j k } \phi(\Bx, \Bx'),
\end{equation}

we see that the Green's function is given by
\begin{dmath}\label{eqn:gradientGreensFunctionEuclidean:760}
G(\Bx, \Bx')
=
\lr{ \rspacegrad - j k } \phi(\Bx, \Bx').
\end{dmath}

This can be computed directly

\begin{dmath}\label{eqn:gradientGreensFunctionEuclidean:780}
G(\Bx, \Bx')
=
\lr{ \rspacegrad - j k } \rcap \PD{r}{} \lr{ -\frac{e^{-j k r}}{4 \pi r} }
=
\frac{-e^{-j k r}}{4 \pi}
\lr{
\rcap \lr{ -\frac{j k}{r} - \inv{ r^2 } } - \frac{j k}{r}
}
=
\frac{e^{-j k r}}{4 \pi}
\lr{
j k \lr{ 1 + \rcap } + \frac{\rcap}{r}
},
\end{dmath}

as claimed.
Observe that since \( \phi \) is scalar valued, we can also rewrite
\cref{eqn:gradientGreensFunctionEuclidean:760} in terms of a right acting operator

\begin{dmath}\label{eqn:gradientGreensFunctionEuclidean:800}
G(\Bx, \Bx')
=
\phi(\Bx, \Bx')
\lr{ \lspacegrad - j k }
=
\phi(\Bx, \Bx')
\lr{ -\lspacegrad' - j k },
\end{dmath}

so
\begin{equation}\label{eqn:gradientGreensFunctionEuclidean:820}
G(\Bx, \Bx') \lr{ -\lspacegrad' + j k } =
\phi(\Bx, \Bx') \lr{ (\lspacegrad')^2 + k^2 }
=
\delta(\Bx - \Bx').
\end{equation}

This is relavant for bounded superposition states, which we will discuss next now that the proof of
\cref{thm:gradientGreensFunctionEuclidean:720} is complete.  In particular, we may rearrange
\cref{eqn:gradientGreensFunctionEuclidean:500}

\begin{dmath}\label{eqn:gradientGreensFunctionEuclidean:840}
\begin{aligned}
-\int_V \lr{ G(\Bx, \Bx') \lspacegrad'} F(\Bx') dV'
&=
\int_V G(\Bx, \Bx') \lr{ \rspacegrad' F(\Bx')} dV' \\
&-
\int_{\partial V} G(\Bx, \Bx') \ncap' F(\Bx') dA',
\end{aligned}
\end{dmath}

and add \( \int_V G(\Bx, \Bx') j k F(\Bx') dV' \) to both sides to find

\begin{dmath}\label{eqn:gradientGreensFunctionEuclidean:860}
\begin{aligned}
\int_V \lr{ G(\Bx, \Bx') \lr{ -\lspacegrad' + j k } } F(\Bx') dV'
&=
\int_V G(\Bx, \Bx') \lr{ \lr{ \rspacegrad' + j k } F(\Bx') } dV' \\
&-
\int_{\partial V} G(\Bx, \Bx') \ncap' F(\Bx') dA'.
\end{aligned}
\end{dmath}

Utilizing \cref{thm:gradientGreensFunctionEuclidean:720}, and substituting \( J(\Bx') \) from \cref{eqn:gradientGreensFunctionEuclidean:660}, we find that one solution is

\begin{dmath}\label{eqn:gradientGreensFunctionEuclidean:880}
F(\Bx)
=
\int_V G(\Bx, \Bx') J(\Bx') dV'
-
\int_{\partial V} G(\Bx, \Bx') \ncap' F(\Bx') dA'.
\end{dmath}

We are free to
add any specific solution \( F_0 \) that satisfies the homogeneous equation \( \lr{ \spacegrad + j k } F_0 = 0 \).

%}

         %\subsection{Problems}
         \subsection{Helmholtz theorem.}
            %
% Copyright © 2016 Peeter Joot.  All Rights Reserved.
% Licenced as described in the file LICENSE under the root directory of this GIT repository.
%
In conventional electromagnetism Maxwell's equations are posed in terms of separate divergence and curl equations, so it is desirable to show that knowning the divergence and curl of a function and it's normal characteristics on the boundary of an integraion volume determine that function uniquely, known as the Helmholtz theorem
\maketheorem{Helmholtz first theorem.}{thm:helmholtzDerviationMultivectorStatement:1}{
If vector \(\BM\) is defined by its divergence

\begin{dmath}\label{eqn:helmholtzDerviationMultivectorStatement:20}
\spacegrad \cdot \BM = s
\end{dmath}

and its curl
\begin{dmath}\label{eqn:helmholtzDerviationMultivectorStatement:40}
\spacegrad \cross \BM = \BC
\end{dmath}

within a region and its normal component \( \BM_{\txtn} \) over the boundary, then \( \BM \) is
uniquely specified.
} % theorem

It could be argued that Helmholtz's theorem is irrelavent when using the GA formalism, since we consolidate the separate divergence and curl equations into one gradient operator.
We include a proof here regardless, since it can be performed in a compact and interesting fashion using
the fundamental theorem of geometric calculus \cref{thm:fundamentalTheoremOfCalculus:1}.


            %
% Copyright © 2016 Peeter Joot.  All Rights Reserved.
% Licenced as described in the file LICENSE under the root directory of this GIT repository.
%
The gradient of the vector \( \BM \) can be written as a single even grade multivector

\begin{equation}\label{eqn:helmholtzDerviationMultivector:60}
\spacegrad \BM
= \spacegrad \cdot \BM + I \spacegrad \cross \BM
= s + I \BC.
\end{equation}

%Observe that the Laplacian of \( \BM \) is vector valued
%
%\begin{dmath}\label{eqn:helmholtzDerviationMultivector:760}
%\spacegrad^2 \BM = \spacegrad s + I \spacegrad \BC.
%\end{dmath}
%
%This means that \( \spacegrad \BC \) must be a bivector \( \spacegrad \BC = \spacegrad \wedge \BC \), or that \( \BC \) has zero divergence
%
%\begin{dmath}\label{eqn:helmholtzDerviationMultivector:780}
%\spacegrad \cdot \BC = 0.
%\end{dmath}

This can be used to attempt to discover the relation between the vector \( \BM \) and its divergence and curl.
The vector \( \BM \) can be expressed at the point of interest as a convolution with the delta function at all other points in space

\begin{dmath}\label{eqn:helmholtzDerviationMultivector:80}
\BM(\Bx) = \int_V dV' \delta(\Bx - \Bx') \BM(\Bx').
\end{dmath}

The Laplacian representation of the delta function in \R{3} is

\begin{dmath}\label{eqn:helmholtzDerviationMultivector:100}
\delta(\Bx - \Bx') = -\inv{4\pi} \spacegrad^2 \inv{\Norm{\Bx - \Bx'}},
\end{dmath}

so \( \BM \) can be represented as the following convolution

\begin{dmath}\label{eqn:helmholtzDerviationMultivector:120}
\BM(\Bx) = -\inv{4\pi} \int_V dV' \spacegrad^2 \inv{\Norm{\Bx - \Bx'}} \BM(\Bx').
\end{dmath}

%As noted in \cref{eqn:helmholtzDerviationMultivector:460} the Laplacian of a vector can be factored as
%
%\begin{dmath}\label{eqn:helmholtzDerviationMultivector:140}
%\spacegrad^2 \Ba
%=
%\spacegrad (\spacegrad \cdot \Ba)
%-
%\spacegrad \cross (\spacegrad \cross \Ba).
%\end{dmath}
%
%Note that the last term can be written in cross product notation using \( \Bc \cdot (\Ba \wedge \Bb) = -\Bc \cross (\Ba \cross \Bb) \) if desired.

Using this relation and proceeding with a few applications of the chain rule, plus the fact that \( \spacegrad 1/\Norm{\Bx - \Bx'} = -\spacegrad' 1/\Norm{\Bx - \Bx'} \), we find
%
%I previously posted a Geometric Algebra attack on the Helmholtz theorem.  Here is
%
%Here's a third way of deriving the Helmholtz theorem inversion relation.  This is a refinement of the traditional vector algebra solution that led to \cref{eqn:helmholtzDerviationMultivector:200}, that uses a factorization of the Laplacian directly, deferring any expansion in terms of dot and cross (or wedge) products until the very end.
%
%Starting from the first line of \cref{eqn:helmholtzDerviationMultivector:160}, we have

\begin{dmath}\label{eqn:helmholtzDerviationMultivector:720}
-4 \pi \BM(\Bx)
= \int_V dV' \spacegrad^2 \inv{\Norm{\Bx - \Bx'}} \BM(\Bx')
= \gpgradeone{\int_V dV' \spacegrad^2 \inv{\Norm{\Bx - \Bx'}} \BM(\Bx')}
= -\gpgradeone{\int_V dV' \spacegrad \lr{ \spacegrad' \inv{\Norm{\Bx - \Bx'}}} \BM(\Bx')}
= -\gpgradeone{\spacegrad \int_V dV' \lr{
\spacegrad' \frac{\BM(\Bx')}{\Norm{\Bx - \Bx'}}
-\frac{\spacegrad' \BM(\Bx')}{\Norm{\Bx - \Bx'}}
} }
=
-\gpgradeone{\spacegrad \int_{\partial V} dA'
\ncap \frac{\BM(\Bx')}{\Norm{\Bx - \Bx'}}
 }
+\gpgradeone{\spacegrad \int_V dV'
\frac{s(\Bx') + I\BC(\Bx')}{\Norm{\Bx - \Bx'}}
 }
=
-\gpgradeone{\spacegrad \int_{\partial V} dA'
\ncap \frac{\BM(\Bx')}{\Norm{\Bx - \Bx'}}
 }
+\spacegrad \int_V dV'
\frac{s(\Bx')}{\Norm{\Bx - \Bx'}}
+\spacegrad \cdot \int_V dV'
\frac{I\BC(\Bx')}{\Norm{\Bx - \Bx'}}.
\end{dmath}

By inserting a no-op grade selection operation in the second step, the trivector terms that would show up in subsequent steps are automatically filtered out.
%the troublesome trivector term that shows up in my first purely Geometric Algebra
%attempt is eliminated.
This leaves us with a boundary term dependent on the surface and the normal and tangential components of \( \BM \).
Added to that is a pair of volume integrals that provide the unique dependence of \( \BM \) on its divergence and curl.
When the surface is taken to infinity, which requires \( \Norm{\BM}/\Norm{\Bx - \Bx'} \rightarrow 0 \), then the dependence of \( \BM \) on its divergence and curl is unique.

In order to express final result in traditional vector algebra form, a couple transformations are required.
The first is that

\begin{equation}\label{eqn:helmholtzDerviationMultivector:800}
\gpgradeone{ \Ba I \Bb } = I^2 \Ba \cross \Bb = -\Ba \cross \Bb.
\end{equation}

For the grade selection in the boundary integral, note that

\begin{dmath}\label{eqn:helmholtzDerviationMultivector:740}
\gpgradeone{ \spacegrad \ncap \BX }
=
\gpgradeone{ \spacegrad (\ncap \cdot \BX) }
+
\gpgradeone{ \spacegrad (\ncap \wedge \BX) }
=
\spacegrad (\ncap \cdot \BX)
+
\gpgradeone{ \spacegrad I (\ncap \cross \BX) }
=
\spacegrad (\ncap \cdot \BX)
-
\spacegrad \cross (\ncap \cross \BX).
\end{dmath}

These give

%\begin{dmath}\label{eqn:helmholtzDerviationMultivector:721}
\boxedEquation{eqn:helmholtzDerviationMultivector:721}{
\begin{aligned}
\BM(\Bx)
&=
\spacegrad \inv{4\pi} \int_{\partial V} dA' \ncap \cdot \frac{\BM(\Bx')}{\Norm{\Bx - \Bx'}}
-
\spacegrad \cross \inv{4\pi} \int_{\partial V} dA' \ncap \cross \frac{\BM(\Bx')}{\Norm{\Bx - \Bx'}} \\
&-\spacegrad \inv{4\pi} \int_V dV'
\frac{s(\Bx')}{\Norm{\Bx - \Bx'}}
+\spacegrad \cross \inv{4\pi} \int_V dV'
\frac{\BC(\Bx')}{\Norm{\Bx - \Bx'}}.
\end{aligned}
}
%\end{dmath}

      \section{Problem solutions.}
         \shipoutAnswer

\part{Electromagnetism.}
   \chapter{Electromagnetism.}
      \section{Maxwell and Lorentz equations.}
         %
% Copyright © 2017 Peeter Joot.  All Rights Reserved.
% Licenced as described in the file LICENSE under the root directory of this GIT repository.
%
Maxwell's equations provide an abstraction, the field, that aggregates the effects of an arbitrary electric charge and current
distribution on a ``test'' charge distribution.
The test charge is assumed to be small and isolated enough that it does not also appreciably change the fields themselves.
Once the fields are determined, the Lorentz force equation can be used to determine the dynamics of the
test particle.
These dynamics can be determined without having to
compute all the interactions of that charge with all the charges and currents in space, nor having to continually account for
the interactions of those charge with each other.

We will use vector differential form of Maxwell's equations with antenna theory extensions (fictious magnetic sources) as our starting point

\index{magnetic charge density}
\index{electric charge density}
\index{magnetic current density}
\index{electric current density}

\begin{subequations}
\label{eqn:freespace:3399}
\begin{dmath}\label{eqn:freespace:3100}
%\spacegrad \cross \BE = - \PD{t}{\BB}
\spacegrad \cross \BE = - \BM - \PD{t}{\BB}
\end{dmath}
\begin{dmath}\label{eqn:freespace:3120}
%\spacegrad \cross \BB = \mu_0 \lr{ \BJ + \epsilon_0 \PD{t}{\BE} }
\spacegrad \cross \BH = \BJ + \PD{t}{\BD}
\end{dmath}
\begin{dmath}\label{eqn:freespace:3140}
%\spacegrad \cdot \BE = \frac{\rho}{\epsilon_0}
\spacegrad \cdot \BD = \rho
\end{dmath}
\begin{dmath}\label{eqn:freespace:3160}
%\spacegrad \cdot \BB = 0.
\spacegrad \cdot \BB = \rho_\txtm.
\end{dmath}
\end{subequations}

These equations relate the primary electric and magnetic fields

\begin{itemize}
	\item \( \BE(\Bx, t) \) : Electric field intensity [\si{V/m}] (Volts/meter)
	\item \( \BH(\Bx, t) \) : Magnetic field intensity [\si{A/m}] (Amperes/meter),
\end{itemize}

and the induced electric and magnetic fields

\begin{itemize}
	\item \( \BD(\Bx, t) \) : Electric flux density (or displacement vector) [\si{C/m}] (Coulombs/meter)
	\item \( \BB(\Bx, t) \) : Magnetic flux density [\si{W/m^2}] (Webers/square meter),
\end{itemize}

to the charge densities

\begin{itemize}
	\item \( \rho(\Bx, t) \) : Electric charge density [\si{C/m^3}] (Coulombs/cubic meter)
   \item \( \rho_\txtm(\Bx, t) \) : Magnetic charge density [\si{W/m^3}] (Webers/cubic meter),
\end{itemize}

and the current densities

\begin{itemize}
	\item \( \BJ(\Bx, t) \) : Electric current density [\si{A/m^2}] (Amperes/square meter),
   \item \( \BM(\Bx, t) \) : Magnetic current density [\si{V/m^2}] (Volts/square meter).
\end{itemize}

All of the fields and sources can vary in space and time, and are specified here in SI units.
The sources \( \BM, \rho_\txtm \) can be considered fictional, representing physical phenomena such as infinitesimal current loops.

In general, the relationship between the electric and magnetic fields (constitutivity relationships) may be complicated
non-isotropic tensor operators, functions of all of \( \BE, \BD, \BB \) and \( \BH \).
It will often be useful to assume that these constitutive relationships between the electric and magnetic fields are independent

\begin{subequations}
\label{eqn:freespace:300}
\begin{dmath}\label{eqn:freespace:320}
\BB = \mu \BH
\end{dmath}
\begin{dmath}\label{eqn:freespace:340}
\BD = \epsilon \BE,
\end{dmath}
\end{subequations}

where \( \epsilon = \epsilon_r \epsilon_0 \) is the permittivity of the medium [\si{F/m}] (Farads/meter), and \( \mu = \mu_r \mu_0 \) is the permeability of the medium [\si{H/m}] (Henries/meter).
The permittivity and permeability may be functions of both time and position, and model the materials that the fields are propagating through.
In free space \( \mu_r = 1 \) and \( \epsilon_r = 1\) so these relationships are simply

\begin{subequations}
\label{eqn:freespace:301}
\begin{dmath}\label{eqn:freespace:321}
\BB = \mu_0 \BH
\end{dmath}
\begin{dmath}\label{eqn:freespace:341}
\BD = \epsilon_0 \BE,
\end{dmath}
\end{subequations}

where

\begin{itemize}
\item \( \epsilon_0 = 8.85 \times 10^{-12} \si{C^2/N/m^2}\) : Permittivity of free space (Coulombs squared/Newton/square meter)
\item \( \mu_0 = 4 \pi \times 10^{-7} \si{N/A^2}\) : Permeability of free space (Newtons/Ampere-squared).
\end{itemize}

These constants are related to the speed of light, \( c = 3.00 \times 10^8 \si{m/s} \) by \( \mu_0 \epsilon_0 = 1/c^2 \).

Antenna theory extends Maxwell's equations with fictional magnetic charge and current densities that are useful to model
real phenomena such as infinitesimal current loops.

The fields and sources are all real valued functions of both space and time.
In many situations it will be desirable to work with a time harmonic (frequency-domain phasor) form of Maxwell's equations.
Two conventions exist for the frequency dependence
\index{time harmonic}

\begin{subequations}
\label{eqn:freespace:900}
\begin{dmath}\label{eqn:freespace:20}
\BY(\Bx, t) = \Real( \BY(\Bx, \omega) e^{j\omega t} )
\end{dmath}
\begin{dmath}\label{eqn:freespace:40}
\BY(\Bx, t) = \Real( \BY(\Bx, \omega) e^{-i\omega t} ).
\end{dmath}
\end{subequations}

\Cref{eqn:freespace:20} is the engineering convention (used herein), and \cref{eqn:freespace:40} is the physics convention.
Care is required by the reader to understand which form of frequency dependence has been assumed.
In either representation the field (or source) \( \BY(\Bx, \Bomega) \) is allowed to be complex valued, but we still require the basis to be Euclidean, and do not require complex inner product spaces.  In such a representation, the square of a vector is still a scalar, but may be complex valued.
%%%Given this frequency dependence Maxwell's equations take the form
%%%
%%%%\input{../ece1229-antenna/MaxwellsTimeHarmonic.tex}
%%%\begin{subequations}
%%%\label{eqn:freespace:99}
%%%\begin{dmath}\label{eqn:freespace:100}
%%%\spacegrad \cross \BE = - \BM - j \omega \BB
%%%\end{dmath}
%%%\begin{dmath}\label{eqn:freespace:120}
%%%\spacegrad \cross \BH = \BJ + j \omega \BD
%%%\end{dmath}
%%%\begin{dmath}\label{eqn:freespace:140}
%%%\spacegrad \cdot \BD = \rho
%%%\end{dmath}
%%%\begin{dmath}\label{eqn:freespace:160}
%%%\spacegrad \cdot \BB = \rho_\txtm.
%%%\end{dmath}
%%%\end{subequations}

Continuous models for charge and current distributions are used in Maxwell's equations, despite the
fact that charges (i.e. electrons) are particles, and are not distributed in space.
The discrete nature of electronic charge can be modelled using a delta function representation of the charge and current densities

\begin{dmath}\label{eqn:freespace:240}
\begin{aligned}
\rho(\Bx, t) &= \sum_a q_a \delta( \Bx - \Bx_a( t) ) \\
\BJ(\Bx, t) &= \sum_a q_a \Bv_a( \Bx, \Bt ).
\end{aligned}
\end{dmath}

This model is inherently non-quantum mechanical, as it assumes that it is possible to
simultaneous measure the position and velocity of an electron.
%Additionally, irrespective of an electron's wave function distribution, this model requires that all electron interactions occur at fixed points in space and time.

The dynamics of particle interaction with the fields are provided by the
Lorentz force and power equations

\begin{subequations}
\label{eqn:freespace:180}
\begin{dmath}\label{eqn:freespace:200}
\ddt{\Bp} = q \lr{ \BE + \Bv \cross \BB }
\end{dmath}
\begin{dmath}\label{eqn:freespace:220}
\ddt{\calE} = q \BE \cdot \Bv.
\end{dmath}
\end{subequations}

Both the energy and the momentum relations of \cref{eqn:freespace:180} are stated, since
the simplest (relativistic) form of the Lorentz force equation directly encodes both.
For readers unfamiliar with \cref{eqn:freespace:220}, \cref{problem:freespace:LorentzPower} provides a derivation method.
%energy and momentum are intimate partners in the context of special relativity, and Maxwell's equations themselves are inherently relativistic.

The quantities involved in the Lorentz equations are

\begin{itemize}
	\item \( \Bp(\Bx, t) \) : Test particle momentum [\si{kg\, m/s}] (Kilogram meters/second)
	\item \( \calE(\Bx, t) \) : Test particle kinetic energy [\si{J}] (Joules, kilogram meter^2/second^2)
	\item \( q \) : Test particle charge [\si{C}] (Coulombs)
	\item \( \Bv \) : Test particle velocity [\si{m/s}] (Meters/second)
\end{itemize}

The task of extracting all the physical meaning from the Maxwell and Lorentz equations is a difficult one.
Our attempt to do so will use the formalism of geometric algebra.


         \subsection{Problems.}
            %
% Copyright © 2017 Peeter Joot.  All Rights Reserved.
% Licenced as described in the file LICENSE under the root directory of this GIT repository.
%
\makeoproblem{Lorentz power and force relationship.}{problem:freespace:LorentzPower}{\S 17 \citep{landau1980classical}}{
Using the relativistic definitions of momentum and energy
\begin{equation*}
\begin{aligned}
	\Bp(\Bx, t) &= \frac{m \Bv}{\sqrt{1-\Bv^2/c^2}} \\
	\calE(\Bx, t) &= \frac{m c^2}{\sqrt{1-\Bv^2/c^2}},
\end{aligned}
\end{equation*}
show that \( d\calE/dt = \Bv \cdot d\Bp/dt \), and use this to derive
\cref{eqn:freespace:220} from \cref{eqn:freespace:200}.
} % problem

      \section{Electrostatics.}
         %
% Copyright © 2017 Peeter Joot.  All Rights Reserved.
% Licenced as described in the file LICENSE under the root directory of this GIT repository.
%
The study of
field and charge distributions that are independent of time is called electrostatics.
With such conditions, the freespace representation of Maxwell's equations \cref{eqn:freespace:99} with no magnetic sources is simply

\begin{subequations}
\label{eqn:electrostatics:99}
\begin{dmath}\label{eqn:electrostatics:100}
\spacegrad \cross \BE = 0
\end{dmath}
\begin{dmath}\label{eqn:electrostatics:120}
\spacegrad \cross \BB = 0
\end{dmath}
\begin{dmath}\label{eqn:electrostatics:140}
\spacegrad \cdot \BE = \frac{\rho}{\epsilon_0}
\end{dmath}
\begin{dmath}\label{eqn:electrostatics:160}
\spacegrad \cdot \BB = 0.
\end{dmath}
\end{subequations}

All the complicated coupling of the electric and magnetic fields is eliminated, and the only source term remaining is a time independent charge density \( \rho = \rho(\Bx) \).

Utilizing \cref{eqn:SimpleProducts2:1640}, the geometric product of the gradient \( \spacegrad \) with a vector \( \Bb \) is

\begin{dmath}\label{eqn:electrostatics:240}
\spacegrad \Bb = \spacegrad \cdot \Bb + I(\spacegrad \cross \Bb).
\end{dmath}

\Cref{eqn:electrostatics:240} can be used to rewrite the electrostatic Maxwell equations (\cref{eqn:electrostatics:99}), as a pair of multivector gradient equations

\begin{subequations}
\label{eqn:electrostatics:360}
\begin{equation}\label{eqn:electrostatics:380}
\spacegrad \BE = \frac{\rho}{\epsilon_0}
\end{equation}
\begin{equation}\label{eqn:electrostatics:400}
\spacegrad \BB = 0.
\end{equation}
\end{subequations}

\subsection{Enclosed charge.}

The charge in a volume can be related to the electric field by integrating \cref{eqn:electrostatics:380}

\begin{dmath}\label{eqn:electrostatics:420}
\int_V d^3 \Bx \spacegrad \BE = \inv{\epsilon_0} \int_V d^3 \Bx \rho(\Bx).
\end{dmath}

This is an oriented integral, where \( d^3 \Bx \) is a pseudoscalar volume element, such as
\( d^3 \Bx = (\Be_1 dx) \wedge (\Be_2 dy) \wedge (\Be_3 dz) = I dx dy dz \).

The LHS integral can be evaluated using the fundamental theorem \cref{thm:fundamentalTheoremOfCalculus:1}

\begin{dmath}\label{eqn:electrostatics:461}
\int_{\partial V} d^2 \Bx \BE = \frac{I}{\epsilon_0} \int_V dV \rho(\Bx).
\end{dmath}

An outward normal \( \ncap \) can be used to
parameterize the bivector surface area element \( d^2 \Bx = I \ncap dA \), which allows the pseudoscalar factors on both
sides to be cancelled

%\begin{dmath}\label{eqn:electrostatics:460}
\boxedEquation{eqn:electrostatics:460}{
\int_{\partial V} dA \ncap \BE = \frac{1}{\epsilon_0} \int_V dV \rho(\Bx).
}
%\end{dmath}

This is a multivector equation which must be simultaneously satisfied by its scalar and bivector components

\begin{subequations}
\label{eqn:electrostatics:481}
\begin{dmath}\label{eqn:electrostatics:501}
\int_{\partial V} dA \ncap \cdot \BE = \frac{1}{\epsilon_0} \int_V dV \rho(\Bx)
\end{dmath}
\begin{dmath}\label{eqn:electrostatics:521}
\int_{\partial V} dA \ncap \wedge \BE = 0.
\end{dmath}
\end{subequations}

The first equation is the familiar relationship between the divergence and the enclosed charge, which could have been derived from \cref{eqn:electrostatics:140} directly.
The second provides a constraint on the tangential components of the field with respect to the enclosed volume, and could have been derived from
\cref{eqn:electrostatics:100} directly.
The multivector equation \cref{eqn:electrostatics:460} encodes both of these relationships, simultaneously incorporating the contributions of the Maxwell divergence and curl equations for the electric field, relating both to the enclosed charge.

\subsection{Electric potential.}

If a gradient representation of the electric field is assumed

\begin{dmath}\label{eqn:electrostatics:561}
\BE(\Bx) = -\spacegrad \phi(\Bx),
\end{dmath}

then
inserting
this assumed representation into \cref{eqn:electrostatics:380} provides the
Poisson equation directly

%\begin{dmath}\label{eqn:electrostatics:581}
\boxedEquation{eqn:electrostatics:581}{
\spacegrad^2 \phi = -\frac{\rho}{\epsilon_0}.
}
%\end{dmath}

%This has solution \cref{eqn:electrostatics:601} but can also be solved using
\subsection{Inverting the gradient equations.}

From \cref{thm:gradientGreensFunctionEuclidean:1}, the
\R{3} Green's function for the gradient (on an infinite spherical bounding surface) is

\begin{dmath}\label{eqn:electrostatics:260}
G(\Bx, \Bx') = \inv{4 \pi} \frac{\Bx - \Bx'}{\Norm{\Bx - \Bx'}^3}.
\end{dmath}

The Green's function convolution that inverts the electric field gradient equation is
\begin{dmath}\label{eqn:electrostatics:621}
\BE(\Bx)
= \int_V dV' G(\Bx, \Bx') \spacegrad' \BE(\Bx')
= \int_V dV' G(\Bx, \Bx') \lr{ \inv{\epsilon_0}\rho(\Bx') }
= \inv{4\pi} \int_V dV' \frac{\Bx - \Bx'}{ \Abs{\Bx - \Bx'}^3 } \lr{ \inv{\epsilon_0}\rho(\Bx') },
\end{dmath}

or
%\begin{dmath}\label{eqn:electrostatics:340}
\boxedEquation{eqn:electrostatics:340}{
\BE(\Bx) =
\inv{4 \pi \epsilon_0} \int dV' \rho(\Bx') \frac{\Bx - \Bx'}{\Norm{\Bx - \Bx'}^3},
}
%\end{dmath}

which is Coulomb's law.

The convolution for the magnetic field is trivial
\begin{dmath}\label{eqn:electrostatics:641}
\BB(\Bx)
= \int_V dV' G(\Bx, \Bx') \spacegrad' \BB(\Bx')
= \int_V dV' G(\Bx, \Bx') (0),
\end{dmath}

so the magnetic field is zero everywhere
\begin{dmath}\label{eqn:electrostatics:320}
\BB(\Bx) = 0.
\end{dmath}

%Question: would a non-zero magnetic field solution be possible if a Green's function for a finite bounded surface were to be used instead?

\subsection{Poisson equation solution.}

Provided \( \Bx \ne \Bx' \), it is simple to show that (\cref{problem:electrostatics:gradrelation})

\begin{dmath}\label{eqn:electrostatics:541}
\spacegrad \Norm{\Bx - \Bx'}^k = k (\Bx - \Bx') \Norm{ \Bx - \Bx' }^{k-2}.
\end{dmath}

In particular

\begin{dmath}\label{eqn:electrostatics:542}
\spacegrad \inv{\Norm{\Bx - \Bx'}} = - \frac{(\Bx - \Bx')}{\Norm{ \Bx - \Bx' }^{3}}.
\end{dmath}

Inserting \cref{eqn:electrostatics:542} into Coulomb's law \cref{eqn:electrostatics:340} gives
\begin{dmath}\label{eqn:electrostatics:661}
\BE(\Bx)
=
-\inv{4 \pi \epsilon_0} \int dV' \rho(\Bx') \spacegrad \inv{\Norm{\Bx - \Bx'}}
=
- \spacegrad \inv{4 \pi \epsilon_0} \int dV' \frac{\rho(\Bx')}{\Norm{\Bx - \Bx'}},
\end{dmath}

which implicitly provides the solution to the Poisson equation \cref{eqn:electrostatics:581}

%\begin{dmath}\label{eqn:electrostatics:601}
\boxedEquation{eqn:electrostatics:601}{
\phi(\Bx) = \inv{4 \pi \epsilon_0} \int dV' \frac{ \rho(\Bx') }{\Norm{\Bx - \Bx'}}.
}
%\end{dmath}

The solution \cref{eqn:electrostatics:601} could also have been obtained from the Green's function for the Laplacian
\cref{eqn:helmholtzDerviationMultivector:100}.

         \subsection{Example: Straight line charge.}
            %
% Copyright © 2017 Peeter Joot.  All Rights Reserved.
% Licenced as described in the file LICENSE under the root directory of this GIT repository.
%
%\makeexample{Line charge.}{example:linecharge:linecharge}{
\index{line charge}
In this example the electric field is calculated at a point on the z-axis, due to a line charge density of \( \lambda \) along a segment \( [a,b] \) of the x-axis.
This is illustrated in \cref{fig:linecharge:linechargeFig1}.
\imageFigure{../figures/GAelectrodynamics/linechargeFig1}{Line charge density.}{fig:linecharge:linechargeFig1}{0.3}

Introducing a unit imaginary \( i = \Be_{13} \) for the rotation from the x-axis to the z-axis, the field point observation point is

\begin{dmath}\label{eqn:linecharge:120}
\Bx = r \Be_1 e^{i \theta}.
\end{dmath}

The charge element point is \( \Bx' = x \Be_1 \), so the difference can now be written with \( \Be_1 \) factored to the left or to the right

\begin{equation}\label{eqn:linecharge:20}
\Bx - \Bx'
= \Be_1\lr{ r e^{i\theta} - x }
= \lr{ r e^{-i\theta} - x } \Be_1.
\end{equation}

This allows the squared vector length to be calculated as a product of complex conjugates

\begin{dmath}\label{eqn:linecharge:40}
\lr{ \Bx - \Bx' }^2
= \lr{ r e^{-i\theta} - x } \Be_1 \Be_1\lr{ r e^{i\theta} - x }
= \lr{ r e^{-i\theta} - x } \lr{ r e^{i\theta} - x }
= r^2 + x^2 - r x \lr{ e^{i\theta} + e^{-i\theta} }
= r^2 + x^2 - 2 r x \cos\theta.
\end{dmath}

The total electric field is therefore
\begin{dmath}\label{eqn:linecharge:60}
\BE
= \frac{\lambda}{4 \pi \epsilon_0} \int_a^b dx \frac{ r \Be_1 e^{i\theta} - x \Be_1 }{ \lr{ r^2 + x^2 - 2 x r \cos\theta }^{3/2} }
= \frac{\lambda \Be_1}{4 \pi \epsilon_0 r} \int_{a/r}^{b/r} du \frac{ e^{i\theta} - u }{ \lr{ 1 + u^2 - 2 u \cos\theta }^{3/2} }.
\end{dmath}

This integral can be evaluated by table lookup or using tools like Mathematica.
For \( \theta = \pi/2 \)

\begin{dmath}\label{eqn:linecharge:80}
\int
du \frac{ e^{i\theta} - u }{ \lr{ 1 + u^2 - 2 u \cos\theta }^{3/2} }
= \frac{1 + i u}{\sqrt{1 + u^2}},
\end{dmath}

and for other angles \( \theta \neq n \pi/2 \)

\begin{dmath}\label{eqn:linecharge:100}
\int
du \frac{ e^{i\theta} - u }{ \lr{ 1 + u^2 - 2 u \cos\theta }^{3/2} }
= \frac{(1 -u e^{-i\theta}) \sqrt{1 + u^2 - 2 u \cos\theta}}{(1 + u^2) \sin(2\theta)}.
\end{dmath}

\index{complex plane}
The interesting takeaway is not the form of the solution, but the fact that GA allows an introduction of a ``complex plane'' for many problems that have polar representations in a plane.
When such a complex plane is introduced,
existing Computer Algebra Systems (CAS), like Mathematica, can be utilized for the grunt work of the evaluation.

Observe that the numerator factors like \( \Be_1 (1 + i u) \) and \( \Be_1(1 - u e^{-i\theta}) \)
compactly describe the direction of the vector field at the observation point.
Either of these can be expanded explicitly in sines and cosines if desired

\begin{dmath}\label{eqn:linecharge:140}
\begin{aligned}
\Be_1 (1 + i u) &= \Be_1 + u \Be_3 \\
\Be_1(1 - u e^{-i\theta}) &= \Be_1(1 - u \cos\theta) + u \Be_3 \sin\theta.
\end{aligned}
\end{dmath}

%} % example

         \subsection{Example: Circular line charge.}
            %
% Copyright © 2017 Peeter Joot.  All Rights Reserved.
% Licenced as described in the file LICENSE under the root directory of this GIT repository.
%
%\makeexample{Circular line charge.}{example:circularlinecharge:circularlinecharge}{
In this example we will examine the electric field due to a line charge density of \( \lambda \) along a circular arc segment \( \phi' \in [a,b] \), of radius \( r \) in the x-y plane.
The field will be evaluated at the
spherical coordinate point \( (R, \theta, \phi) \), as illustrated in \cref{fig:circularlinecharge:circularlinechargeFig1}.  (FIXME: better figure).

\imageFigure{../figures/GAelectrodynamics/circularlinechargeFig1}{Circular line charge.}{fig:circularlinecharge:circularlinechargeFig1}{0.3}

To setup this problem, first consider a GA factorization of the spherical radial point.  Let
\( i = \Be_{12} \), so that

\begin{dmath}\label{eqn:circularlinecharge:20}
\Bx
= R \lr{ \Be_1 \sin\theta \cos\phi + \Be_2 \sin\theta \sin\phi + \Be_3 \cos\theta }
= R \Be_1 \sin\theta ( \cos\phi + \Be_{12} \sin\phi ) + \Be_3 \cos\theta
= R \Be_1 \sin\theta e^{i \phi} + \Be_3 \cos\theta
= R \Be_3 \lr{ \cos\theta + \Be_{31} e^{i \phi} \sin\theta }.
\end{dmath}

Let the bivector for the rotational plane between \( R \Be_3 \rightarrow \Bx \) be represented by

\begin{dmath}\label{eqn:circularlinecharge:40}
j = \Be_{31} e^{i \phi}.
\end{dmath}

The bivector \( j \) is a function of the azimuthal angle \( \phi \), but encodes all the geometry of the rotation.
The observation point now has the simple representation

\begin{dmath}\label{eqn:circularlinecharge:60}
\Bx = R \Be_3 e^{j \theta },
\end{dmath}

and is the product of a polar directed vector with a complex exponential whos argument is the polar rotation angle.
To sum the contributions of the charge elements we need the distance between the charge element and the observation point.  That vector difference is

\begin{dmath}\label{eqn:circularlinecharge:80}
\Bx - \Bx'
=
R \Be_3 e^{j \theta } - r \Be_1 e^{i \phi'}.
\end{dmath}

Compare this to the tuple representation

\begin{dmath}\label{eqn:circularlinecharge:100}
\Bx - \Bx'
= ( R \sin\theta \cos\phi - r \cos\phi', R \sin\theta \sin\phi - r \cos\phi', \cos\theta ),
\end{dmath}

for which the prospect of working with is considerably less attractive.  The squared length of \cref{eqn:circularlinecharge:80} is
\begin{dmath}\label{eqn:circularlinecharge:120}
(\Bx - \Bx')^2
=
R^2 + r^2 - 2 R r \lr{\Be_3 e^{j \theta} } \cdot \lr{ \Be_1 e^{i \phi'} }.
\end{dmath}

The dot product of unit vectors in \cref{eqn:circularlinecharge:120} can be reduced using scalar grade selection

\begin{dmath}\label{eqn:circularlinecharge:140}
\lr{\Be_3 e^{j \theta }} \cdot \lr{ \Be_1 e^{i \phi'} }
=
\gpgradezero{
\lr{ \Be_1 \sin\theta e^{i \phi} } \lr{ \Be_1 e^{i \phi'}}
}
=
\sin\theta
\gpgradezero{
e^{-i \phi} e^{i \phi'}
}
=
\sin\theta \cos( \phi' - \phi ),
\end{dmath}

so
\begin{dmath}\label{eqn:circularlinecharge:160}
\Norm{ \Bx - \Bx' }
=
\sqrt{
R^2 + r^2 - 2 R r \sin\theta \cos( \phi' - \phi )
}.
\end{dmath}

The electric field is
\begin{dmath}\label{eqn:circularlinecharge:180}
\BE = \frac{1}{4 \pi \epsilon_0} \int_a^b \lambda r d\phi' \frac{ R \Be_3 e^{j \theta } - r \Be_1 e^{i \phi'} }{\lr{ R^2 + r^2 - 2 R r \sin\theta \cos( \phi' - \phi ) }^{3/2} }.
\end{dmath}

Non-dimensionalizing \cref{eqn:circularlinecharge:180} with \( u = r/R \), a change of variables \( \alpha = \phi' - \phi \), and noting that \( i \phicap = \Be_1 e^{i \phi} \), the field is

\begin{dmath}\label{eqn:circularlinecharge:200}
\BE
= \frac{\lambda r}{4 \pi \epsilon_0 R^2} \int_{a-\phi}^{b-\phi} d\alpha \frac{ \Be_3 e^{j \theta } - u \Be_1 e^{i\phi} e^{i \alpha} }{\lr{ 1 + u^2 - 2 u \sin\theta \cos \alpha }^{3/2} }
= \frac{\lambda r}{4 \pi \epsilon_0 R^2} \int_{a-\phi}^{b-\phi} d\alpha \frac{ \rcap + u \phicap i e^{i \alpha } }{\lr{ 1 + u^2 - 2 u \sin\theta \cos \alpha }^{3/2} }.
\end{dmath}

The radial integral \( \int dx \lr{ a^2 - 2 b \cos x }^{-3/2} \) scales the radial electric field component without rotating it.
The azimuthal integral \( \int dx \lr{ a^2 - 2 b \cos x }^{-3/2} \) scales and rotates the azimuthal electric field component within the azimuthal plane.
Evaluating \cref{eqn:circularlinecharge:200} using CAS software with no direct GA support requires care, since the bivectors \( i, j \) do not commute and cannot be assembled into a single complex integrand as was possible for the linear line charge.  These integrals can be computed separately without trouble.

%} % example

         \subsection{Problems.}
            %
% Copyright © 2017 Peeter Joot.  All Rights Reserved.
% Licenced as described in the file LICENSE under the root directory of this GIT repository.
%
\makeproblem{Radial gradient.}{problem:electrostatics:gradrelation}{
Prove \cref{eqn:electrostatics:541}.
} % problem


      \section{Magnetostatics.}
         Magnetostatics is the study of Maxwell's equations where
a time independent restriction of the fields is imposed, and
it assumed that there are no static charge distributions.
These constraints simpify Maxwell's equations considerably

\begin{subequations}
\label{eqn:magnetostatics:99}
\begin{dmath}\label{eqn:magnetostatics:100}
\spacegrad \cross \BE = 0
\end{dmath}
\begin{dmath}\label{eqn:magnetostatics:120}
\spacegrad \cross \BB = \mu_0 \BJ
\end{dmath}
\begin{dmath}\label{eqn:magnetostatics:140}
\spacegrad \cdot \BE = 0
\end{dmath}
\begin{dmath}\label{eqn:magnetostatics:160}
\spacegrad \cdot \BB = 0.
\end{dmath}
\end{subequations}

\Cref{eqn:magnetostatics:240} can be used to rewrite the magnetostatic Maxwell equations (\cref{eqn:magnetostatics:99}), as a pair of multivector gradient equations.  The electric field equation is just

\begin{equation}\label{eqn:magnetostatics:400}
\spacegrad \BE = 0,
\end{equation}

and for the magnetic field, we have

\begin{dmath}\label{eqn:magnetostatics:420}
\spacegrad \BB
=
\spacegrad \cdot \BB
I (\spacegrad \cross \BB),
\end{dmath}

so
\begin{dmath}\label{eqn:magnetostatics:380}
\spacegrad \BB
=
I \mu_0 \BJ.
\end{dmath}

Constraints must be imposed on the current density for \cref{eqn:magnetostatics:380} to be satisfied.  This can be seen by left multiplying with the gradient

\begin{dmath}\label{eqn:magnetostatics:440}
\spacegrad^2 \BB
= \mu_0 I \spacegrad \BJ
= \mu_0 I \lr{ \spacegrad \cdot \BJ + \spacegrad \wedge \BJ }
= \mu_0 \lr{ I (\spacegrad \cdot \BJ) - \spacegrad \cross \BJ }
\end{dmath}

The left hand side is a vector, whereas the right hand side has vector and pseudoscalar grades.
This means that the divergence of the current density must be zero

\begin{dmath}\label{eqn:magnetostatics:460}
\spacegrad \cdot \BJ = 0.
\end{dmath}

\subsection{Vector potential.}

Similar to electrostatics where it was assumed that the electric field could be expressed as the gradient of a scalar potential,
a vector potential \( \BA \) solution for the dual of the magnetic field can be assumed

\begin{dmath}\label{eqn:magnetostatics:480}
\spacegrad \BA = I \BB.
\end{dmath}

As the right hand side is a bivector, we must have \( \spacegrad \cdot \BA = 0 \) for this presumed solution to be valid.
Assuming (for now) a zero divergence constraint for the vector potential, then \cref{eqn:magnetostatics:380} is reduced to

\begin{dmath}\label{eqn:magnetostatics:540}
\spacegrad^2 \BA = -\mu_0 \BJ,
\end{dmath}

which can be solved immediately

\begin{dmath}\label{eqn:magnetostatics:560}
\BA(\Bx) = \frac{\mu_0}{4\pi} \int dV' \frac{ \BJ(\Bx') }{\Norm{\Bx - \Bx'}}.
\end{dmath}

The zero divergence constraint for the vector potential is easily dealt with by adding a gradient to the vector potential with the
transformation \( \BA \rightarrow \overbar{\BA} + \spacegrad \chi \).  This gives

\begin{dmath}\label{eqn:magnetostatics:500}
\spacegrad \BA
=
\spacegrad \overbar{\BA} + \spacegrad^2 \chi
=
\spacegrad \cdot \overbar{\BA} + \spacegrad \wedge \overbar{\BA} + \spacegrad^2 \chi,
\end{dmath}

which has the required bivector grade when \( \spacegrad \cdot \overbar{\BA} = -\spacegrad^2 \chi \), or

\begin{dmath}\label{eqn:magnetostatics:520}
\chi(\Bx) = \inv{4\pi} \int dV' \frac{ \spacegrad' \cdot \overbar{\BA}(\Bx') }{\Norm{\Bx - \Bx'}}.
\end{dmath}

\subsection{Enclosed charge density.}

A volume integral of \cref{eqn:magnetostatics:380} provides a relationship between the total enclosed current density and the magnetic field.  The fundamental theorem gives

\begin{dmath}\label{eqn:magnetostatics:580}
-\mu_0 I
\int_V d^3 \Bx \BJ =
\int_V d^3 \Bx \spacegrad \BB =
\int_{\partial V} d^2 \Bx \BB.
\end{dmath}

With a normal parameterization of the oriented surface area element \( d^2 \Bx = I \ncap dA \), and \( d^3 \Bx = I dV \),
\cref{eqn:magnetostatics:580} is reduced to

\begin{dmath}\label{eqn:magnetostatics:600}
\int_{\partial V} dA I \ncap \BB = \mu_0  \int_V dV \BJ.
\end{dmath}

This can be split into two grades

\begin{subequations}
\label{eqn:magnetostatics:620}
\begin{dmath}\label{eqn:magnetostatics:640}
I \int_{\partial V} dA \ncap \cdot \BB = 0
\end{dmath}
\begin{dmath}\label{eqn:magnetostatics:660}
\int_{\partial V} dA \ncap \cross \BB = -\mu_0  \int_V dV \BJ.
\end{dmath}
\end{subequations}

\Cref{eqn:magnetostatics:640} should not be a suprise since it is a direct consequence of Gauss's law \( \spacegrad \cdot \BB = 0 \).  \Cref{eqn:magnetostatics:660} provides a relationship between the tangential components of the magnetic field and the total enclosed current density.

\subsection{Biot-Savart law.}

The magnetostatic Maxwell equation \cref{eqn:magnetostatics:380} can be inverted directly using the Green's function for the gradient

\begin{dmath}\label{eqn:biotSavartGreens:40}
I \BB(\Bx)
= \int_V dV' G(\Bx, \Bx') \spacegrad' I \BB(\Bx')
\end{dmath}

This expansion can be simplified by inserting a no-op grade selection operation

\begin{dmath}\label{eqn:magnetostatics:680}
I \BB(\Bx)
= \gpgradetwo{ \int_V dV' G(\Bx, \Bx') \spacegrad' I \BB(\Bx') }
= \int_V dV' \gpgradetwo{ G(\Bx, \Bx') (-\mu_0 \BJ(\Bx')) }
= \inv{4\pi} \int_V dV' \frac{\Bx - \Bx'}{ \Abs{\Bx - \Bx'}^3 } \wedge (-\mu_0 \BJ(\Bx'))
= \frac{\mu_0}{4\pi} \int_V dV' \BJ(\Bx') \wedge \frac{\Bx - \Bx'}{ \Abs{\Bx - \Bx'}^3 }.
\end{dmath}

This is the Biot-Savart law in its GA form.  The traditional expression requires only a duality transformation \( \BJ \wedge \Bf = I ( \BJ \cross \Bf) \), or

\begin{dmath}\label{eqn:magnetostatics:700}
\BB(\Bx)
= \frac{\mu_0}{4\pi} \int_V dV' \BJ(\Bx') \cross \frac{\Bx - \Bx'}{ \Abs{\Bx - \Bx'}^3 }.
\end{dmath}

The freedom to insert a no-op vector grade selection may seem like a sneaky move.  It is possible to avoid that
by showing that the scalar integrands discarded in \cref{eqn:magnetostatics:680} are explicitly zero.

\begin{dmath}\label{eqn:biotSavartGreens:60}
-\frac{\mu_0}{4\pi} \int_V dV' \frac{\Bx - \Bx'}{ \Abs{\Bx - \Bx'}^3 } \cdot \BJ(\Bx')
= \frac{\mu_0}{4\pi} \int_V dV' \lr{ \spacegrad \inv{ \Abs{\Bx - \Bx'} }} \cdot \BJ(\Bx')
= -\frac{\mu_0}{4\pi} \int_V dV' \lr{ \spacegrad' \inv{ \Abs{\Bx - \Bx'} }} \cdot \BJ(\Bx')
= -\frac{\mu_0}{4\pi} \int_V dV' \lr{
\spacegrad' \cdot \frac{\BJ(\Bx')}{ \Abs{\Bx - \Bx'} }
-
\frac{\spacegrad' \cdot \BJ(\Bx')}{ \Abs{\Bx - \Bx'} }
}.
\end{dmath}

By \cref{eqn:magnetostatics:460}, the divergence of the current density is zero, which kills the second term.  The divergence theorem can be used to express the remaining integral as a surface integral, so

\begin{dmath}\label{eqn:biotSavartGreens:100}
-\frac{\mu_0}{4\pi} \int_V dV' \frac{\Bx - \Bx'}{ \Abs{\Bx - \Bx'}^3 } \cdot \BJ(\Bx')
= -\frac{\mu_0}{4\pi} \int_V dV' \spacegrad' \cdot \frac{\BJ(\Bx')}{ \Abs{\Bx - \Bx'} }
= -\frac{\mu_0}{4\pi} \int_{\partial V} dA' \ncap \cdot \frac{\BJ(\Bx')}{ \Abs{\Bx - \Bx'} }.
\end{dmath}

Provided the normal component of \( \BJ(\Bx')/\Abs{\Bx - \Bx'} \) vanishes on the boundary of an infinite sphere, we see that the
the scalar selection of the convolution integral is zero, justifying the (sneaky) bivector selection operation.


      \section{Maxwell's equation (GA).}
         %
% Copyright © 2016 Peeter Joot.  All Rights Reserved.
% Licenced as described in the file LICENSE under the root directory of this GIT repository.
%
\index{Maxwell's equation}
We will work with a multivector representation of the fields in isotropic media satisfying the
constituency relationships from \cref{eqn:freespace:300}, and define a multivector field that includes both electric and magnetic components

\makedefinition{Electromagnetic field strength.}{dfn:isotropicMaxwells:640}{
The \textit{electromagnetic field strength} ([\si{V/m}] (Volts/meter)) is defined as
\begin{equation*}
F = \BE + I \eta \BH \quad(= \BE + I c \BB),
\end{equation*}
where
\begin{itemize}
\item \( \eta = \sqrt{\mu/\epsilon} \) (\( [\Omega] \) Ohms), is the impedance of the media.
\item \( c = 1/\sqrt{\epsilon\mu} \) ([\si{m/s}] meters/second), is the group velocity of a wave in the media.  When \( \epsilon = \epsilon_0, \mu = \mu_0 \), \( c \) is the speed of light.
\end{itemize}
\( F \) is called the \textit{F}araday by some authors.
} % definition

The factors of \( \eta \) (or \( c \)) that multiply the magnetic fields are for dimensional consistency, since \( [\sqrt{\epsilon} \BE] = [\sqrt{\mu} \BH] = [\BB/\sqrt{\mu}]\).
The justification for imposing a dual (or complex) structure on the electromagnetic field strength can be found in the historical development of
Maxwell's equations, but we will also see such a structure arise naturally in short order.

No information is lost by imposing the complex structure of
\cref{dfn:isotropicMaxwells:640}, since we can always obtain the
electric field vector \( \BE \) and the magnetic field bivector \( I \BH \) by grade selection
from the electromagnetic field strength when desired
\begin{dmath}\label{eqn:isotropicMaxwells:620}
\begin{aligned}
\BE &= \gpgradeone{ F } \\
I \BH &= \inv{\eta} \gpgradetwo{ F }.
\end{aligned}
\end{dmath}

We will also
define a multivector current containing all charge densities and current densities
\makedefinition{Multivector current.}{dfn:isotropicMaxwells:660}{
The \textit{current} ([\si{A/m^2}] (Amperes/square meter)) is defined as
\begin{equation*}
J = \eta \lr{ c \rho - \BJ } + I \lr{ c \rho_\txtm - \BM }.
\end{equation*}
} % definition
When the fictitious magnetic source terms \((\rho_\txtm, \BM)\) are included, the current has one grade for each possible source (scalar, vector, bivector, trivector).  With only conventional electric sources, the current is still a multivector, but contains only scalar and vector grades.

Given the multivector field and current, it is now possible to state Maxwell's equation (singular) in its geometric algebra form
\maketheorem{Maxwell's equation.}{dfn:isotropicMaxwells:680}{
Maxwell's equation is a multivector equation relating the change in the electromagnetic field strength to charge and current densities and is written as
\begin{equation*}
\stgrad F = J.
\end{equation*}
} % theorem
Maxwell's equation in this form will be the starting place for all the subsequent analysis in this book.
As mentioned in \cref{chap:GreensFunctions}, the operator \( \spacegrad + (1/c) \partial_t \) will be called the \textit{spacetime gradient}\footnote{This form of spacetime gradient is given a special symbol by a number of authors, but there is no general agreement on what to use.
Instead of entering the fight, it will be written it out in full in this book.}.

To prove \cref{dfn:isotropicMaxwells:680} we
first insert the
isotropic
constituency relationships from \cref{eqn:freespace:300} into
\cref{eqn:freespace:3399}, so that we are working with two field variables instead of four
\begin{dmath}\label{eqn:isotropicMaxwells:500}
\begin{aligned}
\spacegrad \cdot \BE &= \inv{\epsilon} \rho \\
\spacegrad \cross \BE &= - \BM - \mu \PD{t}{\BH} \\
\spacegrad \cdot \BH &= \inv{\mu} \rho_\txtm \\
\spacegrad \cross \BH &= \BJ + \epsilon \PD{t}{\BE}
\end{aligned}
\end{dmath}
Inserting \( \Ba = \spacegrad \) into \cref{eqn:SimpleProducts2:1640} the vector product of the gradient with another vector
\begin{dmath}\label{eqn:isotropicMaxwells:520}
\spacegrad \Bb = \spacegrad \cdot \Bb + I \spacegrad \cross \Bb.
\end{dmath}
The respective dot and cross products for \( \BE \) and \( \BH \) in
\cref{eqn:isotropicMaxwells:500}
can be grouped using \cref{eqn:isotropicMaxwells:520} into multivector gradient equations
\begin{dmath}\label{eqn:isotropicMaxwells:540}
\begin{aligned}
\spacegrad \BE &= \inv{\epsilon} \rho + I \lr{ - \BM - \mu \PD{t}{\BH} } \\
\spacegrad \BH &= \inv{\mu} \rho_\txtm + I \lr{ \BJ + \epsilon \PD{t}{\BE} }.
\end{aligned}
\end{dmath}
Multiplying the gradient equation for the magnetic field by \( \eta I \) so that both equations have the same dimensions, and so that the electric field appears in both equations as \( \BE \) and not \( I \BE \), we find
\begin{dmath}\label{eqn:isotropicMaxwells:560}
\begin{aligned}
\spacegrad \BE        + \inv{c} \PD{t}{} (I \eta \BH) &= \inv{\epsilon}\rho - I \BM  \\
\spacegrad I \eta \BH + \inv{c} \PD{t}{\BE}           &= I c \rho_\txtm - \eta \BJ,
\end{aligned}
\end{dmath}
where \( \mu/\eta = \eta \epsilon = 1/c \) was used to simplify things slightly, and all the field contributions have been moved to the left hand side.
The first multivector equation has only scalar and bivector grades, whereas the second has only vector and trivector grades.  This means that if we add these equations, we can recover each by grade selection, and no information is lost.  That sum is
\begin{dmath}\label{eqn:isotropicMaxwells:580}
\stgrad \lr{ \BE + I \eta \BH } = \eta\lr{ c \rho - \BJ } + I \lr{ c \rho_\txtm - \BM }.
\end{dmath}
Application of \cref{dfn:isotropicMaxwells:640} and \cref{dfn:isotropicMaxwells:660} to
\cref{eqn:isotropicMaxwells:580} proves the theorem, verifying the
assertion that Maxwell's equations can be consolidated into a single multivector equation.
There is a lot of information packed into this single equation.
Where possible, we want to work with the multivector form of Maxwell's equation, either in the
compact form of \cref{dfn:isotropicMaxwells:680} or the explicit form of \cref{eqn:isotropicMaxwells:580},
and not decompose Maxwell's equation into the conventional representation by grade selection operations.

\subsubsection{Problems.}
\makeproblem{Dot and cross product relation to vector product.}{problem:isotropicMaxwells:700}{
Using coordinate expansion, convince yourself of the validity of \cref{eqn:isotropicMaxwells:520}.
} % problem
\makeproblem{Extracting the conventional Maxwell's equations.}{problem:isotropicMaxwells:720}{
Apply grade 0,1,2, and 3 selection operations to \cref{eqn:isotropicMaxwells:580}.  Determine the multiplicative (scalar or trivector) constants required to obtain \cref{eqn:isotropicMaxwells:500} from the equations that result from such grade selection operations.
} % problem

      \section{Wave equation.}
            %
% Copyright © 2016 Peeter Joot.  All Rights Reserved.
% Licenced as described in the file LICENSE under the root directory of this GIT repository.
%
%\section{Wave equation.}
\index{wave equation}
Having assembled all of Maxwell's equations into \cref{eqn:maxwellsEquations:460}, some results now follow almost trivially.
One such result is the wave equation in space free of sources.
In such a region, Maxwell's equation is just

\begin{dmath}\label{eqn:maxwellsEquations:480}
\lr{ \spacegrad + \inv{c} \PD{t}{} } F = 0.
\end{dmath}

This can be multiplied from the left with the multivector operator \( \spacegrad - \inv{c} \PD{t}{} \), to give

\begin{dmath}\label{eqn:maxwellsEquations:500}
0 =
\lr{ \spacegrad - \inv{c} \PD{t}{} }
\lr{ \spacegrad + \inv{c} \PD{t}{} } F
=
\lr{ \spacegrad^2 - \inv{c^2} \PDSq{t}{} } F,
\end{dmath}

or

\begin{dmath}\label{eqn:maxwellsEquations:520}
\spacegrad^2 F = \inv{c^2} \PDSq{t}{F}.
\end{dmath}

Since \( \spacegrad^2 \) is a scalar operator, selection of the vector and bivector components of \cref{eqn:maxwellsEquations:520} gives

\begin{dmath}\label{eqn:maxwellsEquations:540}
\begin{aligned}
\spacegrad^2 \BE &= \inv{c^2} \PDSq{t}{\BE} \\
\spacegrad^2 (I \BH) &= \inv{c^2} \PDSq{t}{(I \BH)} \\
\end{aligned}
\end{dmath}

These equations can be solved independently, provided the solutions are also constrained by Maxwell's equation \cref{eqn:maxwellsEquations:480}.


      \section{Poynting vector.}
         %
% Copyright © 2017 Peeter Joot.  All Rights Reserved.
% Licenced as described in the file LICENSE under the root directory of this GIT repository.
%

In conventional electromagnetism the energy and momentum density of the fields are

\index{energy density}
\index{momentum density}
\index{Poynting vector}
\index{energy flux}
\begin{dmath}\label{eqn:poyntingF:20}
\begin{aligned}
\calE &= \inv{2} \lr{ \BD \cdot \BE + \BB \cdot \BH } \\
\bcP c &= \inv{c} \BS = \inv{c} \BE \cross \BH.
\end{aligned}
\end{dmath}

where \( \calE \) is the energy density, \( \BS \) is the Poynting vector representing energy flux through a surface per unit time, and \( \bcP \) is the momentum density of the fields.
In geometric algebra, it is arguably more natural to write the Poynting vector as a bivector-vector dot product

\begin{dmath}\label{eqn:poyntingF:1100}
\BS = \lr{ I \BH } \cdot \BE,
\end{dmath}

however, we can do better, relating both the
energy and momentum densities to a single multivector formed from the products of the electrodynamic field \( F \) with its reverse \( F^\dagger \)

\boxedEquation{eqn:poyntingF:60}{
T(1) \equiv \inv{2} \epsilon F F^\dagger = \calE + \bcP c = \calE + \frac{\BS}{c},
}

so
\begin{dmath}\label{eqn:poyntingF:40}
\begin{aligned}
\calE &= \inv{2} \epsilon \gpgradezero{ F F^\dagger } \\
\bcP c &= \inv{2} \epsilon \gpgradeone{ F F^\dagger } \\
\BS &= \inv{2 \eta} \gpgradeone{ F F^\dagger }.
\end{aligned}
\end{dmath}

This dispenses with any requirement to refer to electric or magnetic field components in isolation.

Expanding \( T(1) \) in terms of \( \BE, \BH \) gives

\begin{dmath}\label{eqn:poyntingF:80}
T(1)
=
\inv{2} \epsilon F F^\dagger
=
\inv{2} \epsilon \lr{ \BE + I \eta \BH } \lr{ \BE - I \eta \BH }
=
\inv{2} \epsilon \lr{ \BE^2 + \eta^2 \BH^2 }
+
\inv{2} I \epsilon \eta \lr{ \BH \BE - \BE \BH }
=
\inv{2} \lr{ \BD \cdot \BE + \BH \cdot \BB }
+
\frac{I}{c} \BH \wedge \BE
=
\inv{2} \lr{ \BD \cdot \BE + \BH \cdot \BB }
+
\frac{1}{c} \BE \cross \BH
=
\calE + \frac{\BS}{c},
\end{dmath}

as claimed in \cref{eqn:poyntingF:20} and \cref{eqn:poyntingF:60}.  \( T(1) \) has one scalar component, and three vector components, and represents four
four of the sixteen components of a larger energy-momentum tensor.  The geometric algebra form of the complete energy-momentum tensor is
\index{energy-momentum tensor}
\index{Maxwell stress tensor}

%\begin{dmath}\label{eqn:poyntingF:760}
\boxedEquation{eqn:poyntingF:760}{
T(a) = \inv{2} \epsilon F a F^\dagger,
}
%\end{dmath}

where \( a \) is one of \( 1, \Be_1, \Be_2 \) or \( \Be_3 \), or any linear combination of these 0,1 grade multivector elements.  Observe that \( T(a) \) is a linear operator with respect to any parameter \( a = \alpha + \Ba, \alpha \in \bbR, \Ba \in \bbR^3 \).  \( T(a) \) only 0 and 1 grade elements, which we have seen for scalar parameters \( a \).  We will see by direct expansion that this is also the case for vector parameters.
Such an expansion of \( T(\Be_k) \) is harder to do algebaicly than \cref{eqn:poyntingF:80}
\footnote{Such an expansion is a worthwhile problem to develop GA manipulation skills.  The reader is encouraged to try this independently first, and to refer to
\cref{chap:stressTensorAlgebraically}
for hints if required.}.
On the other hand, it is easy to expand the energy-momentum tensor \( T(a) \) symbolically by brute force using a GA computer algebra package.
A Mathematica expansion of the elements of \( T(a) \) gives:

\begin{subequations}
\label{eqn:poyntingF:800}
\begin{equation}\label{eqn:poyntingF:1020}
\begin{aligned}
T(1)
&= \frac{\epsilon}{2} \lr{E_1^2 + E_2^2 + E_3^2} + \frac{\epsilon \eta^2}{2} \lr{H_1^2 + H_2^2 + H_3^2} \\
&+ \Be_1 \eta \epsilon \lr{E_2 H_3 - E_3 H_2} \\
&+ \Be_2 \eta \epsilon \lr{E_3 H_1 - E_1 H_3} \\
&+ \Be_3 \eta\epsilon \lr{E_1 H_2 - E_2 H_1}
\end{aligned}
\end{equation}
\begin{equation}\label{eqn:poyntingF:1040}
\begin{aligned}
T(\Be_1)
&= \eta \epsilon \lr{E_3 H_2 - E_2 H_3} \\
& + \frac{1}{2} \Be_1 \epsilon \lr{E_1^2 - E_2^2 - E_3^2} + \frac{\epsilon \eta^2}{2} \lr{ H_1^2 -  H_2^2 -  H_3^2} \\
& + \Be_2 \epsilon \lr{E_1 E_2 + \eta^2 H_1 H_2} \\
& + \Be_3 \epsilon \lr{E_1 E_3 + \eta^2 H_1 H_3}
\end{aligned}
\end{equation}
\begin{equation}\label{eqn:poyntingF:1060}
\begin{aligned}
T(\Be_2)
&= \eta \epsilon \lr{E_1 H_3 - E_3 H_1} \\
& + \Be_1 \epsilon \lr{E_1 E_2 + \eta^2 H_1 H_2} \\
& + \frac{1}{2} \Be_2 \epsilon \lr{-E_1^2 + E_2^2 - E_3^2 } + \frac{\epsilon \eta^2}{2} \lr{-H_1^2 +  H_2^2 -  H_3^2} \\
& + \Be_3 \epsilon \lr{E_2 E_3 + \eta^2 H_2 H_3}
\end{aligned}
\end{equation}
\begin{equation}\label{eqn:poyntingF:1080}
\begin{aligned}
T(\Be_3)
&= \eta \epsilon \lr{E_2 H_1 - E_1 H_2} \\
& + \Be_1 \epsilon \lr{E_1 E_3 + \eta^2 H_1 H_3} \\
& + \Be_2 \epsilon \lr{E_2 E_3 + \eta^2 H_2 H_3} \\
& + \frac{1}{2} \Be_3 \epsilon \lr{-E_1^2 - E_2^2 + E_3^2 } + \frac{\epsilon \eta^2}{2} \lr{ -H_1^2 -  H_2^2 + H_3^2}
\end{aligned}
\end{equation}
\end{subequations}

The components of the multivectors \( T(a) \) that we are calling the energy-momentum tensor, are more conventionally written out
as a symmetric tensor \( \Theta^{ij} \) as follows

\begin{dmath}\label{eqn:poyntingF:840}
\begin{aligned}
\Theta^{00} &= \frac{\epsilon}{2} \lr{ \BE^2 + \eta^2 \BH^2 } \\
\Theta^{0i} &= \inv{c} \lr{ \BE \cross \BH } \cdot \Be_i \\
\Theta^{ij} &= -\epsilon \lr{ E_i E_j + \eta^2 H_i H_j - \inv{2} \delta_{ij} \lr{ \BE^2 + \eta^2 \BH^2 } }.
\end{aligned}
\end{dmath}

The names and notation for this tensor components varies.
\( \Theta^{\alpha\beta}, \alpha, \beta \in \setlr{0,1,2,3} \) as defined in \cref{eqn:poyntingF:840} is called the symmetric
stress tensor \citep{jackson1975cew},
whereas other authors call this the energy-momentum tensor and express it as \( T^{\alpha\beta} \) \citep{landau1980classical}, \citep{doran2003gap}.
The sign conventions and notation for the spatial components \( \Theta^{ij}, i, j \in \setlr{1,2,3} \) vary as well, but all authors appear to call this subset the Maxwell stress tensor.
The Maxwell stress tensor is written as \( \sigma_{ij} (=-\Theta^{ij}) \) \citep{landau1980classical}, or as
\( T_{ij} (=-\Theta^{ij}) \)
\citep{griffiths1999introduction},
\citep{jackson1975cew}.

The symmetric stress tensor components of \cref{eqn:poyntingF:840}
are related to the multivector representation expanded in \cref{eqn:poyntingF:800} by

\begin{dmath}\label{eqn:poyntingF:820}
\begin{aligned}
&\gpgradezero{ T(1) }
%= \calE
=
{\Theta_0}^0 = \Theta^{00} \\
&\gpgradeone{ T(1) } \cdot \Be_i
%= \frac{\BS}{c} \cdot \Be_i
= {\Theta_0}^i = \Theta^{0i} \\
&\gpgradezero{ T(\Be_i) }
%= -\frac{\BS}{c} \cdot \Be_i
= {\Theta_i}^0 = -\Theta^{i0} \\
&\gpgradeone{ T(\Be_i) } \cdot \Be_j = {\Theta_i}^j = -\Theta^{ij}.
\end{aligned}
\end{dmath}

The Maxwell stress tensor components \( \gpgradeone{T(\Be_k)} \) can be found expressed in a dyadic notation (\citep{griffiths1999introduction}, \citep{jackson1975cew}) as follows

\begin{dmath}\label{eqn:poyntingF:1140}
\gpgradeone{ T(\Ba) }
=
\sum_i a_i \gpgradeone{ T(\Be_i) }
=
\sum_{i,j} a_i \lr{ \gpgradeone{ T(\Be_i) } \cdot \Be_j } \Be_j
=
\sum_{i,j} a_i T_{ij} \Be_j
\equiv
\Ba \cdot \lrT,
\end{dmath}

so that
\begin{dmath}\label{eqn:poyntingF:1160}
\lr{ \Ba \cdot \lrT } \cdot \Bb
=
\sum_{i,j} a_i T_{ij} b_j.
\end{dmath}

With the Maxwell stress tensor parameterized by a vector, we don't really have any need for this dyadic notation, but it is worth mentioning to
understand how the two formalisms are related.

The complete specification of the energy-momentum tensor for a parameter \( a = \alpha + \Ba = \alpha + \sum_k a_k \Be_k \) is

%\begin{dmath}\label{eqn:poyntingF:1120}
\boxedEquation{eqn:poyntingF:1120}{
T(\alpha + \Ba)
=
\alpha \lr{
   \calE + \frac{\BS}{c}
}
-
\Ba \cdot \frac{\BS}{c}
+ \BT(\Ba),
%+ \gpgradeone{ T(\Ba) }.
}
%\end{dmath}

where the shorthand \( \BT(\Ba) = \gpgradeone{T(\Ba)} \) has been introduced for the Maxwell stress tensor.


         %
% Copyright © 2017 Peeter Joot.  All Rights Reserved.
% Licenced as described in the file LICENSE under the root directory of this GIT repository.
%
\subsection{Complex power.}
TODO.
%\index{complex power}

      \section{Plane waves.}
         %
% Copyright © 2016 Peeter Joot.  All Rights Reserved.
% Licenced as described in the file LICENSE under the root directory of this GIT repository.
%
%\section{Plane waves.}
\index{plane wave}
\index{time harmonic}
\index{source free}
In the time harmonic representation for source free conditions Maxwell's equation \cref{dfn:isotropicMaxwells:680} is just
\begin{dmath}\label{eqn:planewaves:560}
\begin{aligned}
F &= \BE + \eta I \BH \\
\lr{ \spacegrad + j k } F &= 0,
\end{aligned}
\end{dmath}
where \( k = \omega/v \) is the wave number.
It is now possible to examine what constraints Maxwell's equation imposes on plane waves of the form
%\begin{dmath}\label{eqn:planewaves:580}
%\begin{aligned}
%\BE(\Bk) &= \BE_0 e^{-j \Bk \cdot \Bx} \\
%\BH(\Bk) &= \BH_0 e^{-j \Bk \cdot \Bx},
%\end{aligned}
%\end{dmath}
\begin{dmath}\label{eqn:planewaves:600}
F(\Bk) = F_0 e^{-j \Bk \cdot \Bx}.
\end{dmath}

where \( F = \Real( F(\Bk) e^{j \omega t} ) \).
Let's seek a phasor solution \( F(\Bk) \) for the total electromagnetic field, but drop the explicit frequency dependence for simplicity

%
% Copyright © 2016 Peeter Joot.  All Rights Reserved.
% Licenced as described in the file LICENSE under the root directory of this GIT repository.
%
%\section{Plane waves}
\index{plane wave}
The gradient action on the electromagnetic field is

\begin{dmath}\label{eqn:frequencydomainPlaneWaves:160}
\spacegrad F_0 e^{-j \Bk \cdot \Bx}
=
\sum_{m = 1}^3 \Be_m \partial_m
F_0 e^{-j \Bk \cdot \Bx}
=
\sum_{m = 1}^3 \Be_m
F_0
\lr{ -j k_m }
e^{-j \Bk \cdot \Bx}
=
-j \Bk F_0,
\end{dmath}
so

\begin{dmath}\label{eqn:frequencydomainPlaneWaves:180}
j k (1 - \kcap) F_0 = 0.
\end{dmath}

This means that the field must be of the form

%\begin{dmath}\label{eqn:frequencydomainPlaneWaves:200}
\boxedEquation
{eqn:frequencydomainPlaneWaves:200}
{
F = (1 + \kcap) \BE_0 e^{-j \Bk \cdot \Bx},
}
%\end{dmath}
where \( \BE_0 \) is a vector valued complex constant, and \( \kcap \cdot \BE_0 = 0 \).
The dot product constraint follows from the requirement that the \( I \BH \propto \kcap \BE_0 \) portion of the electromagnetic field is a bivector.
The time domain representation of the field is
\begin{dmath}\label{eqn:frequencydomainPlaneWaves:460}
F = (1 + \kcap) \Real{ \BE_0 e^{-j \Bk \cdot \Bx} },
\end{dmath}
but we will see later
instead of using a scalar imaginary \( j \), it is possible to use either the unit bivector for the transverse plane or the \R{3} unit pseudoscalar as the imaginary, and that a plane wave of any polarization can be encoded without any requirement to take real parts.

From \cref{eqn:frequencydomainPlaneWaves:200} the interdependence of the electric and magnetic field portions of the field can be read off immediately.
Those are

\begin{subequations}
\label{eqn:frequencydomainPlaneWaves:220}
\begin{dmath}\label{eqn:frequencydomainPlaneWaves:221}
\BE = \BE_0 e^{-j \Bk \cdot \Bx}
\end{dmath}
\begin{dmath}\label{eqn:frequencydomainPlaneWaves:222}
I \BH = \inv{\eta} \kcap \BE_0 e^{-j \Bk \cdot \Bx},
\end{dmath}
\end{subequations}

or
\begin{dmath}\label{eqn:frequencydomainPlaneWaves:380}
I \BH = \inv{\eta} \kcap \BE.
\end{dmath}

\index{pseudoscalar!spherical}
Since the \R{3} pseudoscalar can be written as

\begin{dmath}\label{eqn:frequencydomainPlaneWaves:400}
I = \kcap \Ecap \Hcap,
\end{dmath}
the directions \( \kcap, \Ecap, \Hcap \) must form a right handed triple.
It is thus expected that the magnetic field is perpendicular to the propagation direction, and that the electric and magnetic fields are explicitly perpendicular, facts that are easily verified

\begin{subequations}
\label{eqn:frequencydomainPlaneWaves:440}
\begin{dmath}\label{eqn:frequencydomainPlaneWaves:260}
\kcap \cdot \BH
= \gpgradezero{ \kcap (-I \kcap \BE_0) } e^{-j \Bk \cdot \Bx}
= -\gpgradezero{ I \BE_0 } e^{-j \Bk \cdot \Bx}
= 0
\end{dmath}
\begin{dmath}\label{eqn:frequencydomainPlaneWaves:280}
\BE \cdot \BH
=
\gpgradezero{ \BE \lr{ -\frac{I}{\eta}} \kcap \BE }
=
-\inv{\eta} \BE^2
\gpgradezero{ \kcap I }
=
0.
\end{dmath}
\end{subequations}

In conventional vector treatments of electromagnetic field theory the field relationships of \cref{eqn:frequencydomainPlaneWaves:220} and the propagation directions are written out explicitly as cross products, instead of multivector equations.
Those cross product relations are obtained easily

\begin{subequations}
\label{eqn:frequencydomainPlaneWaves:420}
\begin{dmath}\label{eqn:frequencydomainPlaneWaves:240}
\BH
= -I \inv{\eta} \kcap \BE
= -I \inv{\eta} (\kcap \wedge \BE)
= -I \inv{\eta} I (\kcap \cross \BE)
= \inv{\eta} \kcap \cross \BE
\end{dmath}
\begin{dmath}\label{eqn:frequencydomainPlaneWaves:300}
\BE
= \eta \kcap I \BH
= \eta I \kcap \wedge \BH
= \eta I^2 \kcap \cross \BH
= \eta \BH \cross \kcap
\end{dmath}
\begin{dmath}\label{eqn:frequencydomainPlaneWaves:340}
\kcap
= I \Hcap \Ecap
= I (\Hcap \wedge \Ecap)
= I^2 (\Hcap \cross \Ecap)
= \Ecap \cross \Hcap.
\end{dmath}
\end{subequations}



      \section{Polarization.}
         %
% Copyright © 2017 Peeter Joot.  All Rights Reserved.
% Licenced as described in the file LICENSE under the root directory of this GIT repository.
%
%{
\index{plane wave}
\index{polarization}
In a discussion of polarization, it is convenient to align the propagation direction along a fixed direction, usually the z-axis.
Setting \( \kcap = \Be_3, \beta z = \Bk \cdot \Bx \) in \cref{eqn:frequencydomainPlaneWaves:200} the plane wave representation of the field is

\begin{dmath}\label{eqn:polarization:20}
\begin{aligned}
F(\Bx, \omega) &= (1 + \Be_3) \BE e^{-j \beta z} \\
F(\Bx, t) &= \Real\lr{ F(\Bx, \omega) e^{j \omega t} }.
\end{aligned}
\end{dmath}

Here the imaginary \( j \) has no intrinsic geometrical interpretation, \( \BE = \BE_\txtr + j \BE_\txti \) is allowed to have complex values, and all components of \( \BE \) is perpendicular to the propagation direction (\( \Be_\txtr \cdot \Be_3 = \BE_\txti \cdot \Be_3 = 0 \)).
\index{Jones vector}
A common representation of the electric field components is the Jones vector \( (c_1, c_2) \), which specifies complex coefficients for the electric field phasor in each of the possible directions

\begin{dmath}\label{eqn:polarization:120}
\BE = c_1 \Be_1 + c_2 \Be_2,
\end{dmath}

where \( c_1, c_2 \) are complex valued, say

\begin{dmath}\label{eqn:polarization:140}
\begin{aligned}
c_1 &= \alpha_1 + j \beta_1 \\
c_2 &= \alpha_2 + j \beta_2.
\end{aligned}
\end{dmath}

The tuple \( (c_1, c_2) \) is called the Jones vector, and compactly encodes the geometry of the pattern that the electric field traces out in the transverse plane.


      \section{Boundary value conditions.}
         %
% Copyright © 2017 Peeter Joot.  All Rights Reserved.
% Licenced as described in the file LICENSE under the root directory of this GIT repository.
%
\index{boundary values}
The difference in the normal and tangential components of the electromagnetic field spanning a surface on which there are
a surface current or surface charge density can be related to those surface sources.

These relationships can be determined by integrating Maxwell's equation over the
pillbox configuration illustrated in \cref{fig:ps3Problem1Pillbox:ps3Problem1PillboxFig1}.

%\imageFigure{../figures/ece1228-electromagnetic-theory/ps3Problem1PillboxFig1}{Pillbox integration volume.}{fig:ps3Problem1Pillbox:ps3Problem1PillboxFig1}{0.2}
\imageFigure{../figures/GAelectrodynamics/pillboxIntegrationVolumeFig1}{pillboxIntegrationVolumeFigllbox integration volume.1}{fig:ps3Problem1Pillbox:ps3Problem1PillboxFig1}{0.3}

An assumption that the sources are primarily constrained to the surface can be written as

\begin{dmath}\label{eqn:boundary:20}
J = J_s \delta(y),
\end{dmath}

where the \( y \) coordinate is locally normal to the surface at any given point.
In terms of the scalar and vector potentials, such a surface source model is

\begin{dmath}\label{eqn:boundary:40}
J = \lr{ \eta\lr{ v \rho_s - \BJ_s } + I \lr{ v \rho_{ms} - \BM_s } }
\delta(y).
\end{dmath}

It will be
simplest to demonstrate the boundary relationships in the frequency domain, where Maxwell's equation can be written as either

\begin{subequations}
\label{eqn:boundary:60}
\begin{dmath}\label{eqn:boundary:80}
\spacegrad F = -j k F + J,
\end{dmath}

or

\begin{dmath}\label{eqn:boundary:100}
\spacegrad I F = -j k I F + I J.
\end{dmath}
\end{subequations}

Application of contraction operations gives

\begin{subequations}
\label{eqn:boundary:120}
\begin{dmath}\label{eqn:boundary:140}
\spacegrad \cdot F
= \gpgrade{-j k F + J}{0,1}
= -j k \BE + \eta( v \rho_s - \BJ_s ) \delta(y)
\end{dmath}
\begin{dmath}\label{eqn:boundary:160}
\spacegrad \cdot (I F)
= \gpgrade{-j k I F + I J}{0,1}
= j k \eta \BH - ( v \rho_{ms} - \BM_s ) \delta(y).
\end{dmath}
\end{subequations}

Each of these contraction operations can be evaluated over the pillbox volume above using the divergence theorem, however, the delta function integrals are problematic.
Those integrals dependent on \( \eta \) and \( v \) which vary across the surface, but are also dependent on the delta function surface contribution, which is valid at only the surface.
Consider the vector potential term for electric sources as an example, where the volume integral of that term is

\begin{dmath}\label{eqn:boundary:180}
-\int dV \eta \BJ_s \delta(y)
=
-\int_{y=0}^{h/2} \int dA \eta_2 \BJ_s \delta(y)
-\int_{y=-h/2}^0 \int dA \eta_1 \BJ_s \delta(y).
\end{dmath}

The delta function is only well defined when integrated across the \( y = 0 \) point.
This problem can be overcome by applying grade selection operations to each of the components of \cref{eqn:boundary:120}, and then rearranging so that all the medium specific contributions to the integrals are factored away from the delta functions

\begin{subequations}
\label{eqn:boundary:200}
\begin{dmath}\label{eqn:boundary:220}
\gpgradezero{\spacegrad \cdot \lr{ \epsilon F}} = \rho_s \delta(y)
\end{dmath}
\begin{dmath}\label{eqn:boundary:240}
\gpgradeone{\spacegrad \cdot \lr{ \inv{\eta} F}} = -j \frac{k}{\eta} \BE - \BJ_s \delta(y)
\end{dmath}
\begin{dmath}\label{eqn:boundary:260}
\gpgradezero{\spacegrad \cdot \lr{ I \inv{v}F}} = - \rho_{ms} \delta(y)
\end{dmath}
\begin{dmath}\label{eqn:boundary:280}
\gpgradeone{\spacegrad \cdot \lr{ I F}} = j k \eta \BH + \BM_s \delta(y).
\end{dmath}
\end{subequations}

Each of the grade selections picks off one of \( \BD, \BB, \BH \) or \( \BE \), so this could have been obtained directly from the conventional set of individual Maxwell equations, however, it is instructional to see how to work with the complete electromagnetic field \( F \).
This also provides a method of evaluating the boundary conditions that is both coordinate free, and uses the same integral form for all the boundary conditions.

Application of the multivector \R{3} divergence theorem, as stated informally in \cref{eqn:stokesTheoremCore:1881d} gives

\begin{subequations}
\label{eqn:boundary:300}
\begin{dmath}\label{eqn:boundary:320}
\gpgradezero{\int dV \ncap \cdot \lr{ \epsilon F}} = \Delta A \rho_s
\end{dmath}
\begin{dmath}\label{eqn:boundary:340}
\gpgradeone{\int dV \ncap \cdot \lr{ \inv{\eta} F}} = -j \omega \int_y dy \int dA \BD - \Delta A \BJ_s
\end{dmath}
\begin{dmath}\label{eqn:boundary:360}
\gpgradezero{\int dV \ncap \cdot \lr{ I \inv{v}F}} = - \Delta A \rho_{ms}
\end{dmath}
\begin{dmath}\label{eqn:boundary:380}
\gpgradeone{\int dV \ncap \cdot \lr{ I F}} = j \omega \int_y dy \int dA \BB + \Delta A \BM_s
\end{dmath}
\end{subequations}

The \( y \) (normal) integral components of the volume integrals are all assumed to vanish as \( \Delta y \rightarrow 0 \), leaving

%\begin{dmath}\label{eqn:boundary:420}
\boxedEquation{eqn:boundary:420}{
\begin{aligned}
\gpgradezero{\ncap (\epsilon_2 F_2 - \epsilon_1 F_1) } &= \rho_s \\
\gpgradeone{\ncap \lr{\inv{\eta_2} F_2 - \inv{\eta_1} F_1 } } &= - \BJ_s \\
\gpgradezero{\ncap I \lr{ \inv{v_2}F_2 - \inv{v_1} F_1 } } &= - \rho_{ms} \\
\gpgradeone{\ncap I (F_2 - F_1)} &= \BM_s
\end{aligned}
}
%\end{dmath}

These can, of course, each be written in terms of the constituent fields if desired

%\begin{dmath}\label{eqn:boundary:440}
%\begin{aligned}
%\ncap \cdot (\BD_2 - \BD_1) &= \rho_s \\
%I \ncap \wedge \lr{ \BH_2 - \BH_1 } &= - \BJ_s \\
%-\ncap \cdot (\BB_2 - \BB_1) &= - \rho_{ms} \\
%I \ncap \wedge (\BE_2 - \BE_1)} &= \BM_s,
%\end{aligned}
%\end{dmath}
%
%or
%
\begin{dmath}\label{eqn:boundary:460}
\begin{aligned}
\ncap \cdot \lr{ \BD_2 - \BD_1 } &= \rho_s \\
\ncap \cross \lr{ \BH_2 - \BH_1 } &= \BJ_s \\
\ncap \cdot \lr{ \BB_2 - \BB_1 } &= \rho_{ms} \\
\ncap \cross \lr{ \BE_2 - \BE_1 } &= -\BM_s.
\end{aligned}
\end{dmath}

The crazy jumble of dot products, cross products and field components in this conventional statement of the boundary conditions is seen to follow systematically from Maxwell's equation \cref{eqn:boundary:80}, and reflects the fact that the components of Maxwell's equation have to be treated individually by grade when evaluating the boundary integrals.

      \section{Multivector potential.}
         %
% Copyright � 2017 Peeter Joot.  All Rights Reserved.
% Licenced as described in the file LICENSE under the root directory of this GIT repository.
%
%{

\section{Multivector potential.}

\subsection{General potential representation.}

For both electrostatics and magnetostatics, where Maxwell's equations are both a pair of gradients, we were able to require that the respective scalar and vector potentials were both gradients.  For electrodynamics where Maxwell's equation is

\begin{dmath}\label{eqn:potentialSection:1800}
\lr{ \spacegrad + \inv{v} \PD{t}{} } F = J,
\end{dmath}

it seems more reasonable to demand a different structure of the potential, say

\begin{dmath}\label{eqn:potentialSection:1820}
F = \lr{ \spacegrad - \inv{v} \PD{t}{} } A,
\end{dmath}

where \( A \) is a multivector potential that may contain all grades, with structure to be determined.
If such a multivector potential can be found, then Maxwell's equation is reduced to a single wave equation

\begin{dmath}\label{eqn:potentialSection:1840}
\lr{ \spacegrad^2 - \inv{v^2} \PDSq{t}{} } A = J,
\end{dmath}

which can be thought of as one wave equation for each multivector grade of the multivector source \( J \).

Some thought shows that the guess \cref{eqn:potentialSection:1820} is not quite right, as it allows for the invalid possibility that \( F \) has scalar or pseudoscalar grades.
While it is possible to impose constraints (a gauge choice) on potential \( A \) that ensure
\( F \) has only the vector and bivector grades,
in general,
a grade selection filter must be imposed

\boxedEquation{eqn:potentialSection:1860}{
F
=
\gpgrade{ \lr{ \spacegrad - \inv{v} \PD{t}{} } A }{1,2}.
}

We will find that the desired representation of the multivector potential is

\begin{dmath}\label{eqn:potentialSection:40}
A =
      - \phi
      + v \BA
      + \eta I \lr{ -\phi_m + v \BF }.
\end{dmath}

Here

\begin{enumerate}
\item \( \phi \) is the scalar potential \si{V} (Volts).
\item \( \BA \) is the vector potential \si{W/m} (Webers/meter).
\item \( \phi_m \) is the scalar potential for (fictitious) magnetic current sources \si{A} (Amperes).
\item \( \BF \) is the vector potential for (fictitious) magnetic current sources \si{C} (Coulombs).
\end{enumerate}

This specific breakdown of \( A \) into scalar and vector potentials, and dual (pseudoscalar and bivector) potentials has been chosen to match existing SI conventions, specifically those of \citep{balanis2005antenna}.

\subsection{Electric sources.}

For a multivector current with only electric sources

\begin{dmath}\label{eqn:potentialSection:1880}
J = \eta \lr{ v \rho - \BJ },
\end{dmath}

we can construct a multivector potential with only scalar and vector grades
\begin{dmath}\label{eqn:potentialSection:1900}
A = - \phi + v \BA.
\end{dmath}

The resulting field is

\begin{dmath}\label{eqn:potentialSection:80}
F
=
\BE + I \eta \BH
=
\gpgrade{ \lr{ \spacegrad - \inv{v} \PD{t}{} }
\lr{
      - \phi
      + v \BA
}
}{1,2}
=
-\spacegrad \phi
-\PD{t}{\BA}
+ v \spacegrad \wedge \BA
=
-\spacegrad \phi
-\PD{t}{\BA}
+ v \spacegrad \wedge \BA.
\end{dmath}

The respective electric and magnetic fields can be extracted using a duality transformation for the bivector curl

\begin{dmath}\label{eqn:potentialSection:1920}
F
=
-\spacegrad \phi
-\PD{t}{\BA}
+ I v \spacegrad \cross \BA,
\end{dmath}

from which we can read off the field components

\begin{dmath}\label{eqn:potentialSection:100}
\begin{aligned}
\BE &= -\spacegrad \phi -\PD{t}{\BA} \\
\mu \BH &= \spacegrad \cross \BA.
\end{aligned}
\end{dmath}

Observe that the grade selection encodes the precise reciepe required to produce the desired combination of gradients, curls and time partials.

The potential representation of the field \cref{eqn:potentialSection:80} is only a solution if Maxwell's equation is also satisfied, or

\begin{subequations}
\label{eqn:potentialSection:1940}
\begin{dmath}\label{eqn:potentialSection:1960}
\lr{ \spacegrad^2 - \inv{v^2} \PDSq{t}{} } \lr{ -\phi + v\BA } = \eta \lr{ v \rho - \BJ }
\end{dmath}
\begin{dmath}\label{eqn:potentialSection:1980}
\lr{ \spacegrad + \inv{v} \PD{t}{} } \gpgrade{ \lr{ \spacegrad - \inv{v} \PD{t}{} } \lr{ -\phi + v\BA } }{0,3} = 0.
\end{dmath}
\end{subequations}

The simplest way to satisfy the constraint equation \cref{eqn:potentialSection:1980} is to impose the Lorenz gauge condition

\begin{dmath}\label{eqn:potentialSection:2000}
0
= \gpgrade{ \lr{ \spacegrad - \inv{v} \PD{t}{} } \lr{ -\phi + v\BA } }{0,3}
= v \spacegrad \cdot \BA + \inv{v} \PD{t}{\phi},
\end{dmath}

or
\begin{dmath}\label{eqn:potentialSection:2020}
\spacegrad \cdot \BA + \inv{v^2} \PD{t}{\phi} = 0.
\end{dmath}

In the frequency domain \( \PDi{t}{} \leftrightarrow j \omega = j k v \), this gauge choice allows the scalar and vector potentials to be completely decoupled

\begin{dmath}\label{eqn:potentialSection:2040}
\phi = \frac{j v^2}{\omega} \spacegrad \cdot \BA,
\end{dmath}

so the multivector potential is completely determined by a single vector potential

\begin{dmath}\label{eqn:potentialSection:2060}
A =
-\frac{j v^2}{\omega} \spacegrad \cdot \BA + v \BA,
\end{dmath}

Maxwell's equation is reduced to a Helmholtz equation

\begin{dmath}\label{eqn:potentialSection:2080}
\lr{ \spacegrad^2 + k^2} A = J,
\end{dmath}

and the field is simply
\begin{dmath}\label{eqn:potentialSection:2100}
F = \lr{ \spacegrad - j\omega } A.
\end{dmath}

\subsection{Magnetic sources.}

Fixme: to in time domain.

As a second example, consider a multivector potential for magnetic sources in the frequency domain, containing bivector and pseudoscalar components

\begin{dmath}\label{eqn:potentialSection:120}
F
=
\BE + I \eta \BH
=
\gpgrade{ \lr{ \spacegrad - j k }
\lr{
      - I \eta \phi_m
      + I \eta v \BF
}
}{1,2}
=
I \eta v \spacegrad \wedge \BF
-\eta I j k v \BF
- I \eta \spacegrad \phi_m
=
- \eta v \spacegrad \cross \BF
+ \eta I \lr
{
-\spacegrad \phi_m
-j \omega \BF
},
\end{dmath}

which recovers the expected potential representations of the fields

\begin{dmath}\label{eqn:potentialSection:140}
\begin{aligned}
\BE &= -\inv{\epsilon} \spacegrad \cross \BF \\
\BH &= -\spacegrad \phi_m
-j \omega \BF.
\end{aligned}
\end{dmath}




\subsection{rewrite from here.}

\subsection{Gauge transformations}

Because the potential representation of the field is expressed as a grade 1,2 selection, the addition of scalar or pseudoscalar components to the grade selection will not alter the field.  In particular, it is possible to alter the multivector potential

\begin{dmath}\label{eqn:potentialSection:160}
A \rightarrow A + \lr{ \spacegrad + \inv{v} \PD{t}{}} \psi,
\end{dmath}

where \( \psi \) is any multivector field with scalar and pseudoscalar grades, without changing the field

\begin{dmath}\label{eqn:potentialSection:180}
F
\rightarrow
\gpgrade{
   \lr{ \spacegrad - \inv{v} \PD{t}{} }
   \lr{ A + \lr{ \spacegrad + \inv{v} \PD{t}{}} \psi }
}{1,2}
=
F +
\gpgrade{
   \lr{ \spacegrad^2 - \inv{v^2} \PDSq{t}{}} \psi
}{1,2}
.
\end{dmath}

That last grade selection is zero, since \( \psi \) has no vector or bivector grades, demonstrating that the electromagnetic field is invariant with respect to this multivector potential transformation.

It is worth looking how such a transformation impacts each grade of the potential.  Let \( \psi = v \psi^\e + \eta v I \psi^\m \), where \( \psi^\e \) and \( \psi^\m \) are both scalar fields.  The gauge transformation provides the mapping

\begin{subequations}
\label{eqn:potentialSection:220}
\begin{dmath}\label{eqn:potentialSection:200}
- \phi \rightarrow - \phi + \PD{t}{} \psi^\e
\end{dmath}
\begin{dmath}\label{eqn:potentialSection:240}
v \BA \rightarrow v \BA + v \spacegrad \psi^\e
\end{dmath}
\begin{dmath}\label{eqn:potentialSection:260}
I v \BF \rightarrow I v \BF + I v \spacegrad \psi^\m
\end{dmath}
\begin{dmath}\label{eqn:potentialSection:280}
- I \eta \phi_m \rightarrow -I \eta \phi_m + I \eta \PD{t}{} \psi^\m,
\end{dmath}
\end{subequations}

or

\begin{subequations}
\label{eqn:potentialSection:400}
\begin{dmath}\label{eqn:potentialSection:420}
\phi \rightarrow \phi - \PD{t}{} \psi^\e
\end{dmath}
\begin{dmath}\label{eqn:potentialSection:440}
\BA \rightarrow \BA + \spacegrad \psi^\e
\end{dmath}
\begin{dmath}\label{eqn:potentialSection:460}
\BF \rightarrow \BF + \spacegrad \psi^\m
\end{dmath}
\begin{dmath}\label{eqn:potentialSection:480}
\phi_m \rightarrow \phi_m - \PD{t}{} \psi^\m.
\end{dmath}
\end{subequations}

These have the alternation of sign that is found in the usual recipe for gauge transformation of the scalar and vector potentials.  In conventional electromagnetism, the first two relations are usually found by observing it is possible to add any gradient to the vector potential, and then finding the transformation consequences that that choice imposes on the electric field.  With the grade selection formulation of the electromagnetic field, this special coupling of the field potentials comes for free without having to consider the curl of a specific field component.

Note that the latter two dual transformation relationships are for magnetic sources, and are usually expressed in the frequency domain, where the gauge transformations take the form

\begin{subequations}
\label{eqn:potentialSection:300}
\begin{dmath}\label{eqn:potentialSection:320}
\phi \rightarrow \phi - j \omega \psi^\e
\end{dmath}
\begin{dmath}\label{eqn:potentialSection:340}
\BA \rightarrow \BA + \spacegrad \psi^\e
\end{dmath}
\begin{dmath}\label{eqn:potentialSection:360}
\BF \rightarrow \BF + \spacegrad \psi^\m
\end{dmath}
\begin{dmath}\label{eqn:potentialSection:380}
\phi_m \rightarrow \phi_m -j \omega \psi^\m.
\end{dmath}
\end{subequations}

\subsection{Lorenz gauge}

With the flexibility to alter make a gauge transformation of the potential, it is useful to examine the conditions for which it is possible to express the electromagnetic field without any grade selection operation.  That is

\begin{dmath}\label{eqn:potentialSection:1720}
F
=
\lr{ \spacegrad - \inv{v} \PD{t}{} }
\lr{
      - \phi
      + v \BA
      + \eta I \lr{ -\phi_m + v \BF }
}.
\end{dmath}

There should be no a-priori assumption that such a field representation has no scalar, nor no pseudoscalar components, which can be seen by the explicit expansion in grades

\begin{dmath}\label{eqn:potentialSection:1640}
\begin{aligned}
F
&=
\lr{ \spacegrad - \inv{v} \PD{t}{} } A \\
&=
\lr{ \spacegrad - \inv{v} \PD{t}{} } \lr{ -\phi + v \BA + \eta I \lr{ -\phi_m + v \BF } } \\
&=
\inv{v} \partial_t \phi
+ v \spacegrad \cdot \BA  \\
&-\spacegrad \phi
+ I \eta v \spacegrad \wedge \BF
- \partial_t \BA  \\
&+ v \spacegrad \wedge \BA
- \eta I \spacegrad \phi_m
- I \eta \partial_t \BF \\
&+ \eta I \inv{v} \partial_t \phi_m
+ I \eta v \spacegrad \cdot \BF,
\end{aligned}
\end{dmath}

so if this potential representation has only vector and bivector grades, it must be true that

\begin{dmath}\label{eqn:potentialSection:1660}
\begin{aligned}
\inv{v} \partial_t \phi + v \spacegrad \cdot \BA &= 0 \\
\inv{v} \partial_t \phi_m + v \spacegrad \cdot \BF &= 0.
\end{aligned}
\end{dmath}

The first is the well known Lorenz gauge condition, whereas the second is the dual of that condition for magnetic sources.

Should one of these conditions, say the Lorenz condition for the electric source potentials, be non-zero, then it is possible to make a potential transformation for which this condition is zero

\begin{dmath}\label{eqn:potentialSection:1680}
0 \ne
\inv{v} \partial_t \phi + v \spacegrad \cdot \BA
=
\inv{v} \partial_t (\phi' - \partial_t \psi) + v \spacegrad \cdot (\BA' + \spacegrad \psi)
=
\inv{v} \partial_t \phi' + v \spacegrad \BA'
+ v \lr{ \spacegrad^2 - \inv{v^2} \partial_{tt} } \psi,
\end{dmath}

so if \( \inv{v} \partial_t \phi' + v \spacegrad \BA' \) is zero, \( \psi \) must be found such that
\begin{dmath}\label{eqn:potentialSection:1700}
\inv{v} \partial_t \phi + v \spacegrad \cdot \BA
= v \lr{ \spacegrad^2 - \inv{v^2} \partial_{tt} } \psi.
\end{dmath}

Such a gauge transformation requires a non-homogeneous wave equation solution, or equivalently in the frequency domain requires the solution of a Helmholtz equation

\begin{dmath}\label{eqn:potentialSection:1740}
\inv{v} j \omega \phi + v \spacegrad \cdot \BA
= v \lr{ \spacegrad^2 + k^2 } \psi.
\end{dmath}

A similar transformation is also clearly possible to eliminate any pseudoscalar grades in \cref{eqn:potentialSection:1720}.  Such a potential representation is desirable since
Maxwell's equations for such a potential are completely decoupled

\begin{dmath}\label{eqn:potentialSection:1760}
\lr{ \spacegrad^2 - \inv{v^2} \PDSq{t}{} } A = J,
\end{dmath}

which is equivalent to precisely one non-homogenious wave equation for each grade source and potential

\begin{dmath}\label{eqn:potentialSection:1600}
\begin{aligned}
\lr{ \spacegrad^2 - \inv{v^2} \PDSq{t}{} } \phi &= - \inv{\epsilon} \rho \\
\lr{ \spacegrad^2 - \inv{v^2} \PDSq{t}{} } \BA &= - \mu \BJ \\
\lr{ \spacegrad^2 - \inv{v^2} \PDSq{t}{} } \phi_m &= - \frac{I}{\mu} \rho_m \\
\lr{ \spacegrad^2 - \inv{v^2} \PDSq{t}{} } \BF &= - I \epsilon \BM,
\end{aligned}
\end{dmath}

or equivalently, in the frequency domain, a forced Helmholtz equation for each grade

\begin{dmath}\label{eqn:potentialSection:1780}
\begin{aligned}
\lr{ \spacegrad^2 + k^2 } \phi &= - \inv{\epsilon} \rho \\
\lr{ \spacegrad^2 + k^2 } \BA &= - \mu \BJ \\
\lr{ \spacegrad^2 + k^2 } \phi_m &= - \frac{1}{\mu} \rho_m \\
\lr{ \spacegrad^2 + k^2 } \BF &= - \epsilon \BM.
\end{aligned}
\end{dmath}

%}

      \section{Lorentz force.}
         %
% Copyright © 2017 Peeter Joot.  All Rights Reserved.
% Licenced as described in the file LICENSE under the root directory of this GIT repository.
%
We now wish to express the Lorentz force equation \cref{eqn:freespace:200} in its geometric algebra form.
Like the energy momentum tensor, there is value to introduce a multivector with energy and momentum components

\makedefinition{Energy momentum multivector.}{dfn:lorentzForce:300}{
For a particle with energy \( \calE \) and momentum \( \Bp \), we define the \textit{energy momentum multivector} as
\begin{equation*}
T = \calE + c \Bp.
\end{equation*}
} % definition

\makedefinition{Multivector charge.}{dfn:lorentzForce:280}{
We may define a \textit{multivector charge} that includes both the magnitude and velocity (relative to the speed of light) of the charged particle.
\begin{equation*}
Q = \int_V J dV,
\end{equation*}
where \( \BJ = \rho_\txte \Bv_\txte, \BM = \rho_\txtm \Bv_\txtm \).
For electric charges this is
\begin{equation*}
Q = q_\txte \lr{ 1 + \Bv_\txte/c },
\end{equation*}
and for magnetic charges
\begin{equation*}
Q = I q_\txtm \lr{ 1 + \Bv_\txtm/c },
\end{equation*}
where \( q_\txte = \int_V \rho_\txte dV, q_\txtm = \int_V \rho_\txtm dV \).
} % definition

With a multivector charge defined, the Lorentz force equation can be stated in terms of the total electromagnetic field strength
\maketheorem{Lorentz force and power.}{thm:lorentzForce:300}{
The respective power and force experienced by particles with electric (and/or magnetic) charges is described by
\cref{dfn:lorentzForce:280} is
\begin{equation*}
\inv{c} \frac{dT}{dt} = \gpgrade{ F Q^\dagger }{0,1} = \inv{2} \lr{ F^\dagger Q + F Q^\dagger }.
\end{equation*}
where \( \gpgradezero{dT/dt} = \ifrac{d\calE}{dt} \) is the power and \( \gpgradeone{dT/dt} = c \ifrac{d\Bp}{dt} \) is the force on the particle, and
\( Q^\dagger \) is the electric or magnetic charge/velocity multivector of \cref{dfn:lorentzForce:280}.
The conventional representation of the Lorentz force/power equations
\begin{equation*}
\begin{aligned}
\gpgradeone{ F Q^\dagger } &= \ddt{\Bp} = q \lr{ \BE + \Bv \cross \BB } \\
c \gpgradezero{ F Q^\dagger } &= \ddt{\calE} = q \BE \cdot \Bv.
\end{aligned}
\end{equation*}
%given by \cref{eqn:freespace:180}
may be recovered by grade selection operations.
For magnetic particles, such a grade selection gives
\begin{equation*}
\begin{aligned}
\gpgradeone{ F Q^\dagger } &= \frac{d\Bp}{dt} = q_\txtm \lr{ c \BB - \inv{c} \Bv_\txtm \cross \BE } \\
c \gpgradezero{ F Q^\dagger } &= \frac{d\calE}{dt} = \inv{\eta} q_\txtm \BB \cdot \frac{\Bv_\txtm}{c}.
\end{aligned}
\end{equation*}
} % theorem

To prove
\cref{thm:lorentzForce:300},
we can expand the multivector product
\( F q \lr{ 1 + \ifrac{\Bv}{c} } \) into its constituent grades
\begin{dmath}\label{eqn:lorentzForce:40}
q F \lr{ 1 + \frac{\Bv}{c} }
=
q
\lr{ \BE + I c \BB }
\lr{ 1 + \frac{\Bv}{c} }
=
q \BE
+ q I \BB \Bv
+ \frac{q}{c} \BE \Bv
+ q c I \BB
=
  \frac{q}{c} \BE \cdot \Bv
+ q \lr{ \BE + \Bv \cross \BB }
+ q \lr{ c I \BB + \inv{c} \BE \wedge \Bv }
+ q (I \BB) \wedge \Bv.
\end{dmath}

We see the (c-scaled) particle power relationship
\cref{eqn:freespace:220}
in the grade zero component and the Lorentz force \cref{eqn:freespace:220} in the grade 1 component.
A substitution \( q \rightarrow -I q_\txtm, \Bv \rightarrow \Bv_\txtm \), and subsequent grade 0,1 selection gives
\begin{dmath}\label{eqn:lorentzForce:320}
\gpgrade{
-I q_\txtm F \lr{ 1 + \frac{\Bv_\txtm}{c} }
}{0,1}
=
- I q_\txtm \lr{ c I \BB + \inv{c} \BE \wedge \Bv_\txtm }
- I q_\txtm I \BB \cdot \Bv_\txtm
=
q_\txtm \lr{ c \BB - \inv{c} \Bv_\txtm \cross \BE }
+
q_\txtm \BB \cdot \Bv_\txtm.
\end{dmath}
The grade one component of this multivector has the
required form for the dual Lorentz force equation
from \cref{thm:poyntingTheoremRewrite:1420}.
%, as determined from the conservation relationships for the energy momentum tensor in
%\cref{eqn:poyntingLorentzForce:140}.
Scaling the grade zero component by \( c \) completes the proof.

%FIXME: did I have an energy momentum tensor derivation of the time rate of change of energy for a magnetic charge density?

%As the electric and magnetic field contributions to the force are subsumed by the total electromagnetic field strength \( F \), \cref{thm:lorentzForce:300} puts the electric and magnetic fields on equal footing.

      \section{Dielectric and magnetic media.}
         %
% Copyright © 2017 Peeter Joot.  All Rights Reserved.
% Licenced as described in the file LICENSE under the root directory of this GIT repository.
%
So far, we've considered only media where the linear constitutive relationships \cref{eqn:freespace:301} hold.
Without such assumptions the GA formalism for Maxwell's equations cannot be written as a single equation with one multivector field, but requires two equations and two multivector fields.

The two multivector fields are
\begin{dmath}\label{eqn:inMatter:40}
\begin{aligned}
G &= \BE + I c \BB \\
F &= \BD + \frac{I}{c} \BH,
\end{aligned}
\end{dmath}
for which Maxwell's equations are
\begin{dmath}\label{eqn:inMatter:60}
\begin{aligned}
\gpgrade{ \stgrad F }{0,1} &= \rho - \frac{\BJ}{c} \\
\gpgrade{ \stgrad G }{2,3} &= I \lr{ c \rho_m - \BM }.
\end{aligned}
\end{dmath}

Here \( c \) is a non-dimensionalizing constant with dimensions [L/T], but is otherwise unspecified.
Direct expansion can be used to show that \cref{eqn:inMatter:60} is equivalent to Maxwell's equations.
Doing so for each of the grades in turn, we have

\begin{subequations}
\label{eqn:inMatter:80}
\begin{dmath}\label{eqn:inMatter:100}
\rho
=
\gpgradezero{ \stgrad F }
=
\gpgradezero{ \stgrad \lr{ \BD + \frac{I}{c} \BH } }
=
\spacegrad \cdot \BD
\end{dmath}
\begin{dmath}\label{eqn:inMatter:120}
- \frac{\BJ}{c}
=
\gpgradeone{ \stgrad F }
=
\gpgradeone{ \stgrad \lr{ \BD + \frac{I}{c} \BH } }
=
\inv{c} \PD{t}{\BD} + \frac{I}{c} \spacegrad \wedge \BH
=
\inv{c} \PD{t}{\BD} - \frac{1}{c} \spacegrad \cross \BH
\end{dmath}
\begin{dmath}\label{eqn:inMatter:140}
- I \BM
=
\gpgrade{ \stgrad G }{2}
=
\gpgrade{ \stgrad \lr{ \BE + I c \BB} }{2}
=
\spacegrad \wedge \BE + I \PD{t}{\BB}
\end{dmath}
\begin{dmath}\label{eqn:inMatter:160}
I c \rho_m
=
\gpgrade{ \stgrad G }{3}
=
\gpgrade{ \stgrad \lr{ \BE + I c \BB} }{3}
=
c I \spacegrad \cdot \BB.
\end{dmath}
\end{subequations}

After rearranging and cancelling common factors of \( c, I \) Maxwell's equations are recovered
\begin{dmath}\label{eqn:inMatter:180}
\begin{aligned}
\spacegrad \cdot \BD &= \rho \\
\spacegrad \cross \BH &= \BJ + \PD{t}{\BD}  \\
\spacegrad \cross \BE &= -\BM - \PD{t}{\BB} \\
\spacegrad \cdot \BB &= \rho_m.
\end{aligned}
\end{dmath}

One possible strategy for solving these equations is to impose an additional set of constraints on the grades in question
\begin{dmath}\label{eqn:inMatter:200}
\begin{aligned}
\gpgrade{ \stgrad F }{2,3} &= 0 \\
\gpgrade{ \stgrad G }{0,1} &= 0,
\end{aligned}
\end{dmath}
so that all the grade selection filters can be cleared
\begin{dmath}\label{eqn:inMatter:220}
\begin{aligned}
\stgrad F &= \rho - \frac{\BJ}{c} \\
\stgrad G &= I \lr{ c \rho_m - \BM }.
\end{aligned}
\end{dmath}

Each of these now separately has the form of Maxwell's equation, and could be solved separately, subject to the constraint equations.
Only if \( G, F \) can be related by a constant factor, say \( \epsilon G = F \), can these be summed directly (after non-dimensional scaling) to form Maxwell's equation.
Other non-constraint strategies for solving \cref{eqn:inMatter:60} would require additional thought and study.

%   \section{Radiation and scattering}
%TODO.
   %   \subsection{Problem solutions}
   %      \shipoutAnswer

%\part{First draft fragments to rewrite.}
%   \chapter{Maxwell's equations}
%      %
% Copyright © 2016 Peeter Joot.  All Rights Reserved.
% Licenced as described in the file LICENSE under the root directory of this GIT repository.
%
\section{Conventional differential form}

The differential form of Maxwell's equations, with extensions for magnetic sources, is the starting point for all the analysis in these notes.  Those equations are

\input{../ece1229-antenna/MaxwellsStatement.tex}

The magnetic sources can be considered fictional, but are useful for modelling real phenomina such as infinitesimal current loops, especially in antenna theory.

\input{../ece1229-antenna/MaxwellsFieldAndSourceDescription.tex}

The fields and sources are all real valued functions of both space and time.  In many situations it will be desirable to work with a time harmonic (frequency-domain phasor) form of Maxwell's equations.  In engineering, a time harmonic representation presumes that all sources and fields have a frequency dependence of the form
\index{time harmonic}

\begin{dmath}\label{eqn:maxwellsEquations:20}
\bcY(\Bx, t) = \Real( \BY(\Bx, \omega) e^{j\omega t} ),
\end{dmath}

where the field (or source) \( \BY(\Bx, \Bomega) \) is allowed to be complex valued.  Given this frequency dependence Maxwell's equations take the form

\input{../ece1229-antenna/MaxwellsTimeHarmonic.tex}

Note that the time harmonic convention typically used in physics literature presumes a frequency dependence of the form

\begin{dmath}\label{eqn:maxwellsEquations:40}
\bcY(\Bx, t) = \Real( \BY(\Bx, \omega) e^{-i\omega t} ),
\end{dmath}

which alters the sign of any imaginary originating from a time derivative.  Care is required by the reader to understand which form of frequency dependence has been assumed.

\section{GA differential form}

Geometric Algebra admits a number of alternative representations of Maxwell's equations.  The first follows from expressing the cross products all as wedge products, leaving a pair of bivector and a pair of scalar equations

\begin{subequations}
\begin{dmath}\label{eqn:maxwellsEquations:60}
\spacegrad \wedge \bcE = - I \bcM - \PD{t}{I\bcB}
\end{dmath}
\begin{dmath}\label{eqn:maxwellsEquations:80}
\spacegrad \wedge \bcH = I \bcJ + I \PD{t}{\bcD}
\end{dmath}
\begin{dmath}\label{eqn:maxwellsEquations:100}
\spacegrad \cdot \bcD = q_\txte
\end{dmath}
\begin{dmath}\label{eqn:maxwellsEquations:120}
\spacegrad \cdot \bcB = q_\txtm.
\end{dmath}
\end{subequations}

Alternatively, the duality transformation \( \Ba \wedge \Bb = -I \Ba \cdot (I \Bb) \) allows Maxwell's equations to be all written as dot products

\begin{subequations}
\begin{dmath}\label{eqn:maxwellsEquations:140}
\spacegrad \cdot (I \bcE) = \bcM + \PD{t}{\bcB}
\end{dmath}
\begin{dmath}\label{eqn:maxwellsEquations:160}
\spacegrad \cdot (I \bcH) = -\bcJ - \PD{t}{\bcD}
\end{dmath}
\begin{dmath}\label{eqn:maxwellsEquations:180}
\spacegrad \cdot \bcD = q_\txte
\end{dmath}
\begin{dmath}\label{eqn:maxwellsEquations:200}
\spacegrad \cdot \bcB = q_\txtm,
\end{dmath}
\end{subequations}

or, using the duality transformation \( \Ba \cdot \Bb = -I (\Ba \wedge (I \Bb) \), Maxwell's equations can all be written as wedge products

\begin{subequations}
\begin{dmath}\label{eqn:maxwellsEquations:220}
\spacegrad \wedge \bcE = - I \bcM - \PD{t}{I\bcB}
\end{dmath}
\begin{dmath}\label{eqn:maxwellsEquations:240}
\spacegrad \wedge \bcH = I \bcJ + I \PD{t}{\bcD}
\end{dmath}
\begin{dmath}\label{eqn:maxwellsEquations:260}
\spacegrad \wedge (I\bcD) = I q_\txte
\end{dmath}
\begin{dmath}\label{eqn:maxwellsEquations:280}
\spacegrad \wedge (I\bcB) = I q_\txtm.
\end{dmath}
\end{subequations}

Each of these forms can be useful in different circumstances, however the real power of GA in electromagnetism follows from presuming constituative relationships between the pairs of electric and magnetic fields

\begin{subequations}
\label{eqn:maxwellsEquations:300}
\begin{dmath}\label{eqn:maxwellsEquations:320}
\bcB = \mu \bcH
\end{dmath}
\begin{dmath}\label{eqn:maxwellsEquations:340}
\bcD = \epsilon \bcE,
\end{dmath}
\end{subequations}

where \( \epsilon \) is the permitivitity of the medium [\si{F/m}] (Farads/meter), and \( \mu \) is the permeability of the medium [\si{H/m}] (Henries/meter).
The permitivitity and permeability may be functions of both time and position, and model the materials that the fields are propagating through.  In general, the these may be non-isotropic tensor operators, however, unless otherwise specified, isotropic media will be assumed in these notes.

With this constitutative relationship assumed (and a bit of rescaling), the dot and wedge products of \cref{eqn:maxwellsEquations:60}, \cref{eqn:maxwellsEquations:100} can be added, as can those of \cref{eqn:maxwellsEquations:80}, \cref{eqn:maxwellsEquations:120}.  This reduces Maxwell's equations to a pair of first order coupled gradient equations

\begin{subequations}
\label{eqn:maxwellsEquations:361}
\begin{dmath}\label{eqn:maxwellsEquations:360}
\spacegrad \bcE = \inv{\epsilon} q_\txte - I \bcM - \mu \PD{t}{(I\bcH)}
\end{dmath}
\begin{dmath}\label{eqn:maxwellsEquations:380}
\spacegrad (I \bcH) = \frac{I q_\txtm}{\mu} - \bcJ - \epsilon \PD{t}{\bcE}.
\end{dmath}
\end{subequations}

Note that it is more natural to work with a bivector magnetic field \( I \bcH \) in GA, than it is to work with a vector field \( \bcH \).  Observe that, when magnetic sources are included, this pair of coupled equations have sources of each grade (scalar, vector, bivector, and pseudoscalar).

The multivector equation \cref{eqn:maxwellsEquations:360} has grades 0,2 (scalar and bivector), whereas the multivector equation \cref{eqn:maxwellsEquations:380} has grades 1,3 (vector, pseudoscalar).  This means that arbitrary linear combinations of these equations, such as \( \spacegrad (a \bcE + b I \bcH ) \), are possible without any loss of information, since the original equations can then be recovered by grade selection.  To determine a desirable scaling of such a sum, these equations can be non-dimensionalized by expressing the fields as \( \sqrt{\epsilon} \bcE, \sqrt{\mu} \bcH \)

\begin{subequations}
\begin{dmath}\label{eqn:maxwellsEquations:400}
\spacegrad \sqrt{\epsilon} \bcE = \inv{\sqrt{\epsilon}} q_\txte - I \sqrt{\epsilon} \bcM - \sqrt{\epsilon \mu} \PD{t}{(I\sqrt{\mu} \bcH)}
\end{dmath}
\begin{dmath}\label{eqn:maxwellsEquations:420}
\spacegrad (I \sqrt{\mu} \bcH) = \frac{I q_\txtm}{\sqrt{\mu}} - \sqrt{\mu} \bcJ - \sqrt{\epsilon\mu} \PD{t}{\sqrt{\epsilon} \bcE}.
\end{dmath}
\end{subequations}

The dimensions of both differential operators are now equal \( [\spacegrad] = [\sqrt{\epsilon\mu} \PDi{t}{}] = 1/L \), allowing the remaining two multivector Maxwell equations to be decoupled into a single first order equation to solve for the multivector field \( \sqrt{\epsilon} \bcE + I \sqrt{\mu} \bcH \)

\begin{dmath}\label{eqn:maxwellsEquations:440}
\lr{ \spacegrad + \sqrt{\epsilon\mu} \PDi{t}{} }
\lr{ \sqrt{\epsilon} \bcE
\pm
I \sqrt{\mu} \bcH
}
=
\inv{\sqrt{\epsilon}} q_\txte
- I \sqrt{\epsilon} \bcM
\pm \lr{
+ \frac{I q_\txtm}{\sqrt{\mu}}
- \sqrt{\mu} \bcJ
}
.
\end{dmath}

Whether or not to add or subtract is essentially a phase choice for the electric field relative to the magnetic field.  It is conventional to pick the sum rather than the difference.  In engineering, with \( \bcE \) and \( \bcH \) as the primary fields, Maxwell's equation can now be expressed in its multivector form

\boxedEquation{eqn:maxwellsEquations:460}{
\begin{aligned}
\bcF &= \bcE + \eta I \bcH \\
\lr{ \spacegrad + \inv{v} \PD{t}{} } \bcF
&=
\inv{\epsilon v} \lr{ v q_\txte - \bcJ }
+ I \lr{ v q_\txtm - \bcM }
,
\end{aligned}
}

where \( \eta = \sqrt{\mu/\epsilon} \) (\( [\Omega] \) Ohms)
is the impedance of the media
, and \( v = 1/\sqrt{\epsilon\mu} \)
([\si{m/s}] meters/second)
is the group velocity of a wave in the media.

In the time harmonic representation the electromagnetic field will be of the form

\begin{dmath}\label{eqn:maxwellsEquations:620}
F = \BE + \eta I \BH,
\end{dmath}

where \( \BE \) and \( \BH \) are complex.

\makedigression{
\input{../frequencydomain/frequencydomainMaxwellsExtraction.tex}
}

\section{Wave equation.}

Having assembled all of Maxwell's equations into \cref{eqn:maxwellsEquations:460}, some results now follow almost trivially.  One such result is the wave equation in space free of sources.  In such a region, Maxwell's equation is just

\begin{dmath}\label{eqn:maxwellsEquations:480}
\lr{ \spacegrad + \inv{v} \PD{t}{} } \bcF = 0.
\end{dmath}

This can be multiplied with \( \spacegrad - \inv{v} \PD{t}{} \), to give

\begin{dmath}\label{eqn:maxwellsEquations:500}
0 =
\lr{ \spacegrad - \inv{v} \PD{t}{} }
\lr{ \spacegrad + \inv{v} \PD{t}{} } \bcF
=
\lr{ \spacegrad^2 - \inv{v^2} \PDSq{t}{} } \bcF,
\end{dmath}

or

\begin{dmath}\label{eqn:maxwellsEquations:520}
\spacegrad^2 \bcF = \inv{v^2} \PDSq{t}{\bcF}.
\end{dmath}

Since \( \spacegrad^2 \) is a scalar operator, selection of the vector and bivector components of \cref{eqn:maxwellsEquations:520} gives

\begin{dmath}\label{eqn:maxwellsEquations:540}
\begin{aligned}
\spacegrad^2 \bcE &= \inv{v^2} \PDSq{t}{\bcE} \\
\spacegrad^2 (I \bcH) &= \inv{v^2} \PDSq{t}{(I \bcH)} \\
\end{aligned}
\end{dmath}

These equations can be solved independently, provided the solutions are also constrained by Maxwell's equation \cref{eqn:maxwellsEquations:480}.

\section{Plane waves.}

In the time harmonic representation for source free conditions Maxwell's equation \cref{eqn:maxwellsEquations:460} is just
\begin{dmath}\label{eqn:maxwellsEquations:560}
\begin{aligned}
F &= \BE + \eta I \BH \\
\lr{ \spacegrad + j k } F &= 0,
\end{aligned}
\end{dmath}

where \( k = \omega/v \) is the wave number.  It is now possible to examine what constraints Maxwell's equation imposes on plane waves of the form

\begin{dmath}\label{eqn:maxwellsEquations:580}
\begin{aligned}
\BE &= \BE_0 e^{-j \Bk \cdot \Bx} \\
\BH &= \BH_0 e^{-j \Bk \cdot \Bx},
\end{aligned}
\end{dmath}

or
\begin{dmath}\label{eqn:maxwellsEquations:600}
F = F_0 e^{-j \Bk \cdot \Bx}.
\end{dmath}

%
% Copyright © 2016 Peeter Joot.  All Rights Reserved.
% Licenced as described in the file LICENSE under the root directory of this GIT repository.
%
%\section{Plane waves}
\index{plane wave}
The gradient action on the electromagnetic field is

\begin{dmath}\label{eqn:frequencydomainPlaneWaves:160}
\spacegrad F_0 e^{-j \Bk \cdot \Bx}
=
\sum_{m = 1}^3 \Be_m \partial_m
F_0 e^{-j \Bk \cdot \Bx}
=
\sum_{m = 1}^3 \Be_m
F_0
\lr{ -j k_m }
e^{-j \Bk \cdot \Bx}
=
-j \Bk F_0,
\end{dmath}
so

\begin{dmath}\label{eqn:frequencydomainPlaneWaves:180}
j k (1 - \kcap) F_0 = 0.
\end{dmath}

This means that the field must be of the form

%\begin{dmath}\label{eqn:frequencydomainPlaneWaves:200}
\boxedEquation
{eqn:frequencydomainPlaneWaves:200}
{
F = (1 + \kcap) \BE_0 e^{-j \Bk \cdot \Bx},
}
%\end{dmath}
where \( \BE_0 \) is a vector valued complex constant, and \( \kcap \cdot \BE_0 = 0 \).
The dot product constraint follows from the requirement that the \( I \BH \propto \kcap \BE_0 \) portion of the electromagnetic field is a bivector.
The time domain representation of the field is
\begin{dmath}\label{eqn:frequencydomainPlaneWaves:460}
F = (1 + \kcap) \Real{ \BE_0 e^{-j \Bk \cdot \Bx} },
\end{dmath}
but we will see later
instead of using a scalar imaginary \( j \), it is possible to use either the unit bivector for the transverse plane or the \R{3} unit pseudoscalar as the imaginary, and that a plane wave of any polarization can be encoded without any requirement to take real parts.

From \cref{eqn:frequencydomainPlaneWaves:200} the interdependence of the electric and magnetic field portions of the field can be read off immediately.
Those are

\begin{subequations}
\label{eqn:frequencydomainPlaneWaves:220}
\begin{dmath}\label{eqn:frequencydomainPlaneWaves:221}
\BE = \BE_0 e^{-j \Bk \cdot \Bx}
\end{dmath}
\begin{dmath}\label{eqn:frequencydomainPlaneWaves:222}
I \BH = \inv{\eta} \kcap \BE_0 e^{-j \Bk \cdot \Bx},
\end{dmath}
\end{subequations}

or
\begin{dmath}\label{eqn:frequencydomainPlaneWaves:380}
I \BH = \inv{\eta} \kcap \BE.
\end{dmath}

\index{pseudoscalar!spherical}
Since the \R{3} pseudoscalar can be written as

\begin{dmath}\label{eqn:frequencydomainPlaneWaves:400}
I = \kcap \Ecap \Hcap,
\end{dmath}
the directions \( \kcap, \Ecap, \Hcap \) must form a right handed triple.
It is thus expected that the magnetic field is perpendicular to the propagation direction, and that the electric and magnetic fields are explicitly perpendicular, facts that are easily verified

\begin{subequations}
\label{eqn:frequencydomainPlaneWaves:440}
\begin{dmath}\label{eqn:frequencydomainPlaneWaves:260}
\kcap \cdot \BH
= \gpgradezero{ \kcap (-I \kcap \BE_0) } e^{-j \Bk \cdot \Bx}
= -\gpgradezero{ I \BE_0 } e^{-j \Bk \cdot \Bx}
= 0
\end{dmath}
\begin{dmath}\label{eqn:frequencydomainPlaneWaves:280}
\BE \cdot \BH
=
\gpgradezero{ \BE \lr{ -\frac{I}{\eta}} \kcap \BE }
=
-\inv{\eta} \BE^2
\gpgradezero{ \kcap I }
=
0.
\end{dmath}
\end{subequations}

In conventional vector treatments of electromagnetic field theory the field relationships of \cref{eqn:frequencydomainPlaneWaves:220} and the propagation directions are written out explicitly as cross products, instead of multivector equations.
Those cross product relations are obtained easily

\begin{subequations}
\label{eqn:frequencydomainPlaneWaves:420}
\begin{dmath}\label{eqn:frequencydomainPlaneWaves:240}
\BH
= -I \inv{\eta} \kcap \BE
= -I \inv{\eta} (\kcap \wedge \BE)
= -I \inv{\eta} I (\kcap \cross \BE)
= \inv{\eta} \kcap \cross \BE
\end{dmath}
\begin{dmath}\label{eqn:frequencydomainPlaneWaves:300}
\BE
= \eta \kcap I \BH
= \eta I \kcap \wedge \BH
= \eta I^2 \kcap \cross \BH
= \eta \BH \cross \kcap
\end{dmath}
\begin{dmath}\label{eqn:frequencydomainPlaneWaves:340}
\kcap
= I \Hcap \Ecap
= I (\Hcap \wedge \Ecap)
= I^2 (\Hcap \cross \Ecap)
= \Ecap \cross \Hcap.
\end{dmath}
\end{subequations}


\section{Poynting theorem}

Poynting's theorem describes the relationship between the flux of energy through a surface bounding a volume.
The theorem follows from computing the divergence of the Poynting vector \( \bcS = \bcE \cross \bcH \).  With the GA toolbox at hand, this divergence can be written as a scalar selection

\begin{equation}\label{eqn:maxwellsEquations:640}
\spacegrad \cdot \lr{ \bcE \cross \bcH }
=
\gpgradezero{ \spacegrad (-I) \lr{ \bcE \wedge \bcH } }
=
-\gpgradezero{ \spacegrad \bcE I \bcH }.
\end{equation}

Here the gradient is acting on everything to the right, however, allowing the gradient to act bidirectionally, and employing the
the flexibility to use cyclic permutation within a scalar selection
(\(\gpgradezero{AB} = \gpgradezero{BA}\))
, allows for the easy application of the chain rule

\begin{dmath}\label{eqn:maxwellsEquations:760}
-\gpgradezero{ \spacegrad \bcE I \bcH }
=
-\gpgradezero{ I \bcH \lrspacegrad \bcE }
=
-\gpgradezero{ I \bcH (\rspacegrad \bcE) }
-\gpgradezero{ (I \bcH \lspacegrad) \bcE }.
\end{dmath}

Explicit left and right acting gradients are required because the gradient operator does not commute with the vector fields.

The gradient action on \( \bcE \) (from the left) is given by \cref{eqn:maxwellsEquations:360}.  The right acting gradient action on \( I \bcH \) is given by reversing
all the products in
%\spacegrad \bcE = \inv{\epsilon} q_\txte - I \bcM - \mu \PD{t}{(I\bcH)}
\cref{eqn:maxwellsEquations:380}

\begin{dmath}\label{eqn:maxwellsEquations:660}
I \bcH \lspacegrad = \frac{I q_\txtm}{\mu} + \bcJ + \epsilon \PD{t}{\bcE}.
\end{dmath}

This gives
\begin{dmath}\label{eqn:maxwellsEquations:680}
0
=
\spacegrad \cdot \lr{ \bcE \cross \bcH }
+
\gpgradezero
{
I \bcH
\lr{ \inv{\epsilon} q_\txte - I \bcM - \mu \PD{t}{(I\bcH)} }
+
\lr{ \frac{I q_\txtm}{\mu} + \bcJ + \epsilon \PD{t}{\bcE} } \bcE
},
%=
%\spacegrad \cdot \lr{ \bcE \cross \bcH }
%+
%\bcM \cdot \bcH + \PD{t}{\bcB} \cdot \bcH
%+ \bcJ \cdot \bcE + \PD{t}{\bcD} \cdot \bcE,
\end{dmath}

or
%\begin{dmath}\label{eqn:maxwellsEquations:700}
\boxedEquation{eqn:maxwellsEquations:720}{
0 =
\spacegrad \cdot \lr{ \bcE \cross \bcH }
+
\bcH \cdot \bcM + \bcJ \cdot \bcE
+ \PD{t}{\bcB} \cdot \bcH
+ \PD{t}{\bcD} \cdot \bcE.
}
%\end{dmath}

The last two terms is the time rate of change of the energy density.  Consider the change of energy density through a volume with neither electric nor magnetic current sources in that region of space

\begin{dmath}\label{eqn:maxwellsEquations:740}
\PD{t}{} \int_V
\inv{2} dV \lr{
\bcB \cdot \bcH
+ \bcD \cdot \bcE
}
=
-\int_{\partial V} dA \ncap \cdot \bcS.
\end{dmath}

Here \( \ncap \) is the outward normal, so if the energy contained in the volume is decreasing, then \( \bcS \) must represent the energy per unit area that leaves the volume.  The direction of the Poynting vector is the direction that the energy is leaving the volume.  Only the components of the Poynting vector that are colinear with the surface normal will result in energy leaving or entering the volume.


%      \section{Potentials}
%         %
% Copyright � 2017 Peeter Joot.  All Rights Reserved.
% Licenced as described in the file LICENSE under the root directory of this GIT repository.
%
%{

\section{Multivector potential.}

\subsection{General potential representation.}

For both electrostatics and magnetostatics, where Maxwell's equations are both a pair of gradients, we were able to require that the respective scalar and vector potentials were both gradients.  For electrodynamics where Maxwell's equation is

\begin{dmath}\label{eqn:potentialSection:1800}
\lr{ \spacegrad + \inv{v} \PD{t}{} } F = J,
\end{dmath}

it seems more reasonable to demand a different structure of the potential, say

\begin{dmath}\label{eqn:potentialSection:1820}
F = \lr{ \spacegrad - \inv{v} \PD{t}{} } A,
\end{dmath}

where \( A \) is a multivector potential that may contain all grades, with structure to be determined.
If such a multivector potential can be found, then Maxwell's equation is reduced to a single wave equation

\begin{dmath}\label{eqn:potentialSection:1840}
\lr{ \spacegrad^2 - \inv{v^2} \PDSq{t}{} } A = J,
\end{dmath}

which can be thought of as one wave equation for each multivector grade of the multivector source \( J \).

Some thought shows that the guess \cref{eqn:potentialSection:1820} is not quite right, as it allows for the invalid possibility that \( F \) has scalar or pseudoscalar grades.
While it is possible to impose constraints (a gauge choice) on potential \( A \) that ensure
\( F \) has only the vector and bivector grades,
in general,
a grade selection filter must be imposed

\boxedEquation{eqn:potentialSection:1860}{
F
=
\gpgrade{ \lr{ \spacegrad - \inv{v} \PD{t}{} } A }{1,2}.
}

We will find that the desired representation of the multivector potential is

\begin{dmath}\label{eqn:potentialSection:40}
A =
      - \phi
      + v \BA
      + \eta I \lr{ -\phi_m + v \BF }.
\end{dmath}

Here

\begin{enumerate}
\item \( \phi \) is the scalar potential \si{V} (Volts).
\item \( \BA \) is the vector potential \si{W/m} (Webers/meter).
\item \( \phi_m \) is the scalar potential for (fictitious) magnetic current sources \si{A} (Amperes).
\item \( \BF \) is the vector potential for (fictitious) magnetic current sources \si{C} (Coulombs).
\end{enumerate}

This specific breakdown of \( A \) into scalar and vector potentials, and dual (pseudoscalar and bivector) potentials has been chosen to match existing SI conventions, specifically those of \citep{balanis2005antenna}.

\subsection{Electric sources.}

For a multivector current with only electric sources

\begin{dmath}\label{eqn:potentialSection:1880}
J = \eta \lr{ v \rho - \BJ },
\end{dmath}

we can construct a multivector potential with only scalar and vector grades
\begin{dmath}\label{eqn:potentialSection:1900}
A = - \phi + v \BA.
\end{dmath}

The resulting field is

\begin{dmath}\label{eqn:potentialSection:80}
F
=
\BE + I \eta \BH
=
\gpgrade{ \lr{ \spacegrad - \inv{v} \PD{t}{} }
\lr{
      - \phi
      + v \BA
}
}{1,2}
=
-\spacegrad \phi
-\PD{t}{\BA}
+ v \spacegrad \wedge \BA
=
-\spacegrad \phi
-\PD{t}{\BA}
+ v \spacegrad \wedge \BA.
\end{dmath}

The respective electric and magnetic fields can be extracted using a duality transformation for the bivector curl

\begin{dmath}\label{eqn:potentialSection:1920}
F
=
-\spacegrad \phi
-\PD{t}{\BA}
+ I v \spacegrad \cross \BA,
\end{dmath}

from which we can read off the field components

\begin{dmath}\label{eqn:potentialSection:100}
\begin{aligned}
\BE &= -\spacegrad \phi -\PD{t}{\BA} \\
\mu \BH &= \spacegrad \cross \BA.
\end{aligned}
\end{dmath}

Observe that the grade selection encodes the precise reciepe required to produce the desired combination of gradients, curls and time partials.

The potential representation of the field \cref{eqn:potentialSection:80} is only a solution if Maxwell's equation is also satisfied, or

\begin{subequations}
\label{eqn:potentialSection:1940}
\begin{dmath}\label{eqn:potentialSection:1960}
\lr{ \spacegrad^2 - \inv{v^2} \PDSq{t}{} } \lr{ -\phi + v\BA } = \eta \lr{ v \rho - \BJ }
\end{dmath}
\begin{dmath}\label{eqn:potentialSection:1980}
\lr{ \spacegrad + \inv{v} \PD{t}{} } \gpgrade{ \lr{ \spacegrad - \inv{v} \PD{t}{} } \lr{ -\phi + v\BA } }{0,3} = 0.
\end{dmath}
\end{subequations}

The simplest way to satisfy the constraint equation \cref{eqn:potentialSection:1980} is to impose the Lorenz gauge condition

\begin{dmath}\label{eqn:potentialSection:2000}
0
= \gpgrade{ \lr{ \spacegrad - \inv{v} \PD{t}{} } \lr{ -\phi + v\BA } }{0,3}
= v \spacegrad \cdot \BA + \inv{v} \PD{t}{\phi},
\end{dmath}

or
\begin{dmath}\label{eqn:potentialSection:2020}
\spacegrad \cdot \BA + \inv{v^2} \PD{t}{\phi} = 0.
\end{dmath}

In the frequency domain \( \PDi{t}{} \leftrightarrow j \omega = j k v \), this gauge choice allows the scalar and vector potentials to be completely decoupled

\begin{dmath}\label{eqn:potentialSection:2040}
\phi = \frac{j v^2}{\omega} \spacegrad \cdot \BA,
\end{dmath}

so the multivector potential is completely determined by a single vector potential

\begin{dmath}\label{eqn:potentialSection:2060}
A =
-\frac{j v^2}{\omega} \spacegrad \cdot \BA + v \BA,
\end{dmath}

Maxwell's equation is reduced to a Helmholtz equation

\begin{dmath}\label{eqn:potentialSection:2080}
\lr{ \spacegrad^2 + k^2} A = J,
\end{dmath}

and the field is simply
\begin{dmath}\label{eqn:potentialSection:2100}
F = \lr{ \spacegrad - j\omega } A.
\end{dmath}

\subsection{Magnetic sources.}

Fixme: to in time domain.

As a second example, consider a multivector potential for magnetic sources in the frequency domain, containing bivector and pseudoscalar components

\begin{dmath}\label{eqn:potentialSection:120}
F
=
\BE + I \eta \BH
=
\gpgrade{ \lr{ \spacegrad - j k }
\lr{
      - I \eta \phi_m
      + I \eta v \BF
}
}{1,2}
=
I \eta v \spacegrad \wedge \BF
-\eta I j k v \BF
- I \eta \spacegrad \phi_m
=
- \eta v \spacegrad \cross \BF
+ \eta I \lr
{
-\spacegrad \phi_m
-j \omega \BF
},
\end{dmath}

which recovers the expected potential representations of the fields

\begin{dmath}\label{eqn:potentialSection:140}
\begin{aligned}
\BE &= -\inv{\epsilon} \spacegrad \cross \BF \\
\BH &= -\spacegrad \phi_m
-j \omega \BF.
\end{aligned}
\end{dmath}




\subsection{rewrite from here.}

\subsection{Gauge transformations}

Because the potential representation of the field is expressed as a grade 1,2 selection, the addition of scalar or pseudoscalar components to the grade selection will not alter the field.  In particular, it is possible to alter the multivector potential

\begin{dmath}\label{eqn:potentialSection:160}
A \rightarrow A + \lr{ \spacegrad + \inv{v} \PD{t}{}} \psi,
\end{dmath}

where \( \psi \) is any multivector field with scalar and pseudoscalar grades, without changing the field

\begin{dmath}\label{eqn:potentialSection:180}
F
\rightarrow
\gpgrade{
   \lr{ \spacegrad - \inv{v} \PD{t}{} }
   \lr{ A + \lr{ \spacegrad + \inv{v} \PD{t}{}} \psi }
}{1,2}
=
F +
\gpgrade{
   \lr{ \spacegrad^2 - \inv{v^2} \PDSq{t}{}} \psi
}{1,2}
.
\end{dmath}

That last grade selection is zero, since \( \psi \) has no vector or bivector grades, demonstrating that the electromagnetic field is invariant with respect to this multivector potential transformation.

It is worth looking how such a transformation impacts each grade of the potential.  Let \( \psi = v \psi^\e + \eta v I \psi^\m \), where \( \psi^\e \) and \( \psi^\m \) are both scalar fields.  The gauge transformation provides the mapping

\begin{subequations}
\label{eqn:potentialSection:220}
\begin{dmath}\label{eqn:potentialSection:200}
- \phi \rightarrow - \phi + \PD{t}{} \psi^\e
\end{dmath}
\begin{dmath}\label{eqn:potentialSection:240}
v \BA \rightarrow v \BA + v \spacegrad \psi^\e
\end{dmath}
\begin{dmath}\label{eqn:potentialSection:260}
I v \BF \rightarrow I v \BF + I v \spacegrad \psi^\m
\end{dmath}
\begin{dmath}\label{eqn:potentialSection:280}
- I \eta \phi_m \rightarrow -I \eta \phi_m + I \eta \PD{t}{} \psi^\m,
\end{dmath}
\end{subequations}

or

\begin{subequations}
\label{eqn:potentialSection:400}
\begin{dmath}\label{eqn:potentialSection:420}
\phi \rightarrow \phi - \PD{t}{} \psi^\e
\end{dmath}
\begin{dmath}\label{eqn:potentialSection:440}
\BA \rightarrow \BA + \spacegrad \psi^\e
\end{dmath}
\begin{dmath}\label{eqn:potentialSection:460}
\BF \rightarrow \BF + \spacegrad \psi^\m
\end{dmath}
\begin{dmath}\label{eqn:potentialSection:480}
\phi_m \rightarrow \phi_m - \PD{t}{} \psi^\m.
\end{dmath}
\end{subequations}

These have the alternation of sign that is found in the usual recipe for gauge transformation of the scalar and vector potentials.  In conventional electromagnetism, the first two relations are usually found by observing it is possible to add any gradient to the vector potential, and then finding the transformation consequences that that choice imposes on the electric field.  With the grade selection formulation of the electromagnetic field, this special coupling of the field potentials comes for free without having to consider the curl of a specific field component.

Note that the latter two dual transformation relationships are for magnetic sources, and are usually expressed in the frequency domain, where the gauge transformations take the form

\begin{subequations}
\label{eqn:potentialSection:300}
\begin{dmath}\label{eqn:potentialSection:320}
\phi \rightarrow \phi - j \omega \psi^\e
\end{dmath}
\begin{dmath}\label{eqn:potentialSection:340}
\BA \rightarrow \BA + \spacegrad \psi^\e
\end{dmath}
\begin{dmath}\label{eqn:potentialSection:360}
\BF \rightarrow \BF + \spacegrad \psi^\m
\end{dmath}
\begin{dmath}\label{eqn:potentialSection:380}
\phi_m \rightarrow \phi_m -j \omega \psi^\m.
\end{dmath}
\end{subequations}

\subsection{Lorenz gauge}

With the flexibility to alter make a gauge transformation of the potential, it is useful to examine the conditions for which it is possible to express the electromagnetic field without any grade selection operation.  That is

\begin{dmath}\label{eqn:potentialSection:1720}
F
=
\lr{ \spacegrad - \inv{v} \PD{t}{} }
\lr{
      - \phi
      + v \BA
      + \eta I \lr{ -\phi_m + v \BF }
}.
\end{dmath}

There should be no a-priori assumption that such a field representation has no scalar, nor no pseudoscalar components, which can be seen by the explicit expansion in grades

\begin{dmath}\label{eqn:potentialSection:1640}
\begin{aligned}
F
&=
\lr{ \spacegrad - \inv{v} \PD{t}{} } A \\
&=
\lr{ \spacegrad - \inv{v} \PD{t}{} } \lr{ -\phi + v \BA + \eta I \lr{ -\phi_m + v \BF } } \\
&=
\inv{v} \partial_t \phi
+ v \spacegrad \cdot \BA  \\
&-\spacegrad \phi
+ I \eta v \spacegrad \wedge \BF
- \partial_t \BA  \\
&+ v \spacegrad \wedge \BA
- \eta I \spacegrad \phi_m
- I \eta \partial_t \BF \\
&+ \eta I \inv{v} \partial_t \phi_m
+ I \eta v \spacegrad \cdot \BF,
\end{aligned}
\end{dmath}

so if this potential representation has only vector and bivector grades, it must be true that

\begin{dmath}\label{eqn:potentialSection:1660}
\begin{aligned}
\inv{v} \partial_t \phi + v \spacegrad \cdot \BA &= 0 \\
\inv{v} \partial_t \phi_m + v \spacegrad \cdot \BF &= 0.
\end{aligned}
\end{dmath}

The first is the well known Lorenz gauge condition, whereas the second is the dual of that condition for magnetic sources.

Should one of these conditions, say the Lorenz condition for the electric source potentials, be non-zero, then it is possible to make a potential transformation for which this condition is zero

\begin{dmath}\label{eqn:potentialSection:1680}
0 \ne
\inv{v} \partial_t \phi + v \spacegrad \cdot \BA
=
\inv{v} \partial_t (\phi' - \partial_t \psi) + v \spacegrad \cdot (\BA' + \spacegrad \psi)
=
\inv{v} \partial_t \phi' + v \spacegrad \BA'
+ v \lr{ \spacegrad^2 - \inv{v^2} \partial_{tt} } \psi,
\end{dmath}

so if \( \inv{v} \partial_t \phi' + v \spacegrad \BA' \) is zero, \( \psi \) must be found such that
\begin{dmath}\label{eqn:potentialSection:1700}
\inv{v} \partial_t \phi + v \spacegrad \cdot \BA
= v \lr{ \spacegrad^2 - \inv{v^2} \partial_{tt} } \psi.
\end{dmath}

Such a gauge transformation requires a non-homogeneous wave equation solution, or equivalently in the frequency domain requires the solution of a Helmholtz equation

\begin{dmath}\label{eqn:potentialSection:1740}
\inv{v} j \omega \phi + v \spacegrad \cdot \BA
= v \lr{ \spacegrad^2 + k^2 } \psi.
\end{dmath}

A similar transformation is also clearly possible to eliminate any pseudoscalar grades in \cref{eqn:potentialSection:1720}.  Such a potential representation is desirable since
Maxwell's equations for such a potential are completely decoupled

\begin{dmath}\label{eqn:potentialSection:1760}
\lr{ \spacegrad^2 - \inv{v^2} \PDSq{t}{} } A = J,
\end{dmath}

which is equivalent to precisely one non-homogenious wave equation for each grade source and potential

\begin{dmath}\label{eqn:potentialSection:1600}
\begin{aligned}
\lr{ \spacegrad^2 - \inv{v^2} \PDSq{t}{} } \phi &= - \inv{\epsilon} \rho \\
\lr{ \spacegrad^2 - \inv{v^2} \PDSq{t}{} } \BA &= - \mu \BJ \\
\lr{ \spacegrad^2 - \inv{v^2} \PDSq{t}{} } \phi_m &= - \frac{I}{\mu} \rho_m \\
\lr{ \spacegrad^2 - \inv{v^2} \PDSq{t}{} } \BF &= - I \epsilon \BM,
\end{aligned}
\end{dmath}

or equivalently, in the frequency domain, a forced Helmholtz equation for each grade

\begin{dmath}\label{eqn:potentialSection:1780}
\begin{aligned}
\lr{ \spacegrad^2 + k^2 } \phi &= - \inv{\epsilon} \rho \\
\lr{ \spacegrad^2 + k^2 } \BA &= - \mu \BJ \\
\lr{ \spacegrad^2 + k^2 } \phi_m &= - \frac{1}{\mu} \rho_m \\
\lr{ \spacegrad^2 + k^2 } \BF &= - \epsilon \BM.
\end{aligned}
\end{dmath}

%}

%      \section{Problem solutions}
%         \shipoutAnswer
%%   \chapter{Time harmonic fields}
%%      \section{Frequency domain}
%%         \input{../frequencydomain/frequencydomainMaxwells.tex}
%%      \section{Plane waves}
%%         %
% Copyright © 2016 Peeter Joot.  All Rights Reserved.
% Licenced as described in the file LICENSE under the root directory of this GIT repository.
%
%\section{Plane waves}
\index{plane wave}
The gradient action on the electromagnetic field is

\begin{dmath}\label{eqn:frequencydomainPlaneWaves:160}
\spacegrad F_0 e^{-j \Bk \cdot \Bx}
=
\sum_{m = 1}^3 \Be_m \partial_m
F_0 e^{-j \Bk \cdot \Bx}
=
\sum_{m = 1}^3 \Be_m
F_0
\lr{ -j k_m }
e^{-j \Bk \cdot \Bx}
=
-j \Bk F_0,
\end{dmath}
so

\begin{dmath}\label{eqn:frequencydomainPlaneWaves:180}
j k (1 - \kcap) F_0 = 0.
\end{dmath}

This means that the field must be of the form

%\begin{dmath}\label{eqn:frequencydomainPlaneWaves:200}
\boxedEquation
{eqn:frequencydomainPlaneWaves:200}
{
F = (1 + \kcap) \BE_0 e^{-j \Bk \cdot \Bx},
}
%\end{dmath}
where \( \BE_0 \) is a vector valued complex constant, and \( \kcap \cdot \BE_0 = 0 \).
The dot product constraint follows from the requirement that the \( I \BH \propto \kcap \BE_0 \) portion of the electromagnetic field is a bivector.
The time domain representation of the field is
\begin{dmath}\label{eqn:frequencydomainPlaneWaves:460}
F = (1 + \kcap) \Real{ \BE_0 e^{-j \Bk \cdot \Bx} },
\end{dmath}
but we will see later
instead of using a scalar imaginary \( j \), it is possible to use either the unit bivector for the transverse plane or the \R{3} unit pseudoscalar as the imaginary, and that a plane wave of any polarization can be encoded without any requirement to take real parts.

From \cref{eqn:frequencydomainPlaneWaves:200} the interdependence of the electric and magnetic field portions of the field can be read off immediately.
Those are

\begin{subequations}
\label{eqn:frequencydomainPlaneWaves:220}
\begin{dmath}\label{eqn:frequencydomainPlaneWaves:221}
\BE = \BE_0 e^{-j \Bk \cdot \Bx}
\end{dmath}
\begin{dmath}\label{eqn:frequencydomainPlaneWaves:222}
I \BH = \inv{\eta} \kcap \BE_0 e^{-j \Bk \cdot \Bx},
\end{dmath}
\end{subequations}

or
\begin{dmath}\label{eqn:frequencydomainPlaneWaves:380}
I \BH = \inv{\eta} \kcap \BE.
\end{dmath}

\index{pseudoscalar!spherical}
Since the \R{3} pseudoscalar can be written as

\begin{dmath}\label{eqn:frequencydomainPlaneWaves:400}
I = \kcap \Ecap \Hcap,
\end{dmath}
the directions \( \kcap, \Ecap, \Hcap \) must form a right handed triple.
It is thus expected that the magnetic field is perpendicular to the propagation direction, and that the electric and magnetic fields are explicitly perpendicular, facts that are easily verified

\begin{subequations}
\label{eqn:frequencydomainPlaneWaves:440}
\begin{dmath}\label{eqn:frequencydomainPlaneWaves:260}
\kcap \cdot \BH
= \gpgradezero{ \kcap (-I \kcap \BE_0) } e^{-j \Bk \cdot \Bx}
= -\gpgradezero{ I \BE_0 } e^{-j \Bk \cdot \Bx}
= 0
\end{dmath}
\begin{dmath}\label{eqn:frequencydomainPlaneWaves:280}
\BE \cdot \BH
=
\gpgradezero{ \BE \lr{ -\frac{I}{\eta}} \kcap \BE }
=
-\inv{\eta} \BE^2
\gpgradezero{ \kcap I }
=
0.
\end{dmath}
\end{subequations}

In conventional vector treatments of electromagnetic field theory the field relationships of \cref{eqn:frequencydomainPlaneWaves:220} and the propagation directions are written out explicitly as cross products, instead of multivector equations.
Those cross product relations are obtained easily

\begin{subequations}
\label{eqn:frequencydomainPlaneWaves:420}
\begin{dmath}\label{eqn:frequencydomainPlaneWaves:240}
\BH
= -I \inv{\eta} \kcap \BE
= -I \inv{\eta} (\kcap \wedge \BE)
= -I \inv{\eta} I (\kcap \cross \BE)
= \inv{\eta} \kcap \cross \BE
\end{dmath}
\begin{dmath}\label{eqn:frequencydomainPlaneWaves:300}
\BE
= \eta \kcap I \BH
= \eta I \kcap \wedge \BH
= \eta I^2 \kcap \cross \BH
= \eta \BH \cross \kcap
\end{dmath}
\begin{dmath}\label{eqn:frequencydomainPlaneWaves:340}
\kcap
= I \Hcap \Ecap
= I (\Hcap \wedge \Ecap)
= I^2 (\Hcap \cross \Ecap)
= \Ecap \cross \Hcap.
\end{dmath}
\end{subequations}

%%      \section{Problem solutions}
%%         \shipoutAnswer
%   \chapter{Polarization}
%      %
% Copyright © 2017 Peeter Joot.  All Rights Reserved.
% Licenced as described in the file LICENSE under the root directory of this GIT repository.
%
\paragraph{Real Phasor representation.}

A real time dependent field, represented in terms of a complex vector valued phasor \( \tilde{\BA} \), is formed by taking the real part of the product of that phasor with the phase exponential

\begin{dmath}\label{eqn:ellipticalWaves:20}
\bcA
= \Real\lr{ \tilde{\BA} e^{j \Bk \cdot \Bx -j \omega t} }
=
\BA_r \cos\lr{ \Bk \cdot \Bx - \omega t }
- \BA_i \sin\lr{ \Bk \cdot \Bx - \omega t }.
\end{dmath}

In the complex representation above, the imaginary \( j \) is not interpreted geometrically, but like the unit pseudoscalar \( I = \Be_1 \Be_2 \Be_3 \), squares to \( -1 \) and commutes with all grades.  It is therefore possible to express the field using the pseudoscalar as the imaginary

\begin{dmath}\label{eqn:ellipticalWaves:40}
\bcA
=
\inv{2} \BA_r
\lr{ e^{I \phi} + e^{-I\phi} }
   - \inv{2 I} \BA_i
   \lr{ e^{I \phi} - e^{-I\phi} }
=
\inv{2}\lr{ \BA_r + I \BA_i } e^{I \phi}
+
\inv{2}\lr{ \BA_r - I \BA_i } e^{-I \phi}
=
\inv{2} \lr{ \BA e^{I \phi} + \lr{ \BA e^{I \phi} }^\dagger }
,
\end{dmath}

where the phase angle was written as

\begin{dmath}\label{eqn:ellipticalWaves:400}
\phi = \Bk \cdot \Bx - \omega t,
\end{dmath}

and where the field magnitude and orientation has been specified by a ``complex'' (grade-1,3) multivector

\begin{dmath}\label{eqn:ellipticalWaves:120}
\BA = \BA_r + I \BA_i,
\end{dmath}

and its reverse \( \BA^\dagger \).
This has the structure of a real-part operation, where the real part is represented by half the multivector plus its reverse.  This is in fact one way of expressing the vector grade selection operation for the grade-1,3 multivector \( \BA e^{I\phi} \), which can also be considered a phasor representation

\begin{dmath}\label{eqn:ellipticalWaves:80}
\BA e^{I \phi}
=
\lr{ \BA_r + I \BA_i }
\lr{ \cos\phi + I \sin\phi }
=
\lr{ \BA_r + I \BA_i }
\lr{ \cos\phi + I \sin\phi }
=
\BA_r \cos\phi - \BA_i \sin\phi
+ I \BA_i \cos\phi + I \BA_r \sin\phi.
\end{dmath}

Adding this to its reverse (which negates the sign of the pseudoscalar, but not the vector), eliminates all the bivector components of this multivector phasor representation.  It is now possible to represent the field completely in terms of real vectors and a vector grade selection operation

\begin{dmath}\label{eqn:ellipticalWaves:100}
\bcA = \gpgradeone{ \BA e^{I \phi } }.
\end{dmath}

\paragraph{Electromagnetic plane wave.}

The electromagnetic field, with \( \BE = \BE_r + j \BE_i \), where \( \BE_r \cdot \kcap = \BE_i \cdot \kcap = 0 \), for a plane wave is

\begin{dmath}\label{eqn:ellipticalWaves:140}
   F = \Real \lr{ \lr{ 1 + \kcap } \BE e^{j \phi} }.
\end{dmath}

(I have a derivation of this elsewhere, but there is also one in \citep{doran2003gap})

The real representation, with multivector phasor \( \BE = \BE_r + I \BE_i \), is

\begin{dmath}\label{eqn:ellipticalWaves:160}
F
=
\lr{ 1 + \kcap } \gpgradeone{ \BE e^{I \phi } }
=
\inv{2} \lr{ 1 + \kcap } \lr{ \BE e^{I \phi } + \BE^\dagger e^{-I \phi } }.
\end{dmath}

Note that this is not equal to \(
\inv{2} \lr{
   \lr{ 1 + \kcap } \BE e^{I \phi }
   +
\lr{ \lr{ 1 + \kcap } \BE e^{I \phi } }^\dagger } \), since \( \lr{ \kcap \BE }^\dagger = -\kcap \BE^\dagger \).
% because \( \kcap \) is normal to both the \( \BE_r \) and \( \BE_i \) vectors.

Should the electric and magnetic fields be desired explicitly, they can be obtained by the grade selection, with

\begin{equation}\label{eqn:ellipticalWaves:220}
F = \bcE + I \eta \bcH
=
\gpgradeone{ \BE e^{I \phi } } +
\gpgradetwo{ \kcap \BE e^{I \phi } },
\end{equation}

where this split into electric (vector) and magnetic (bivector) field components was facilitated by
the fact that
\( \kcap \gpgradeone{ \BE e^{I \phi } } = \gpgradetwo{ \kcap \BE e^{I \phi } } \) [exercise].

\paragraph{Circular waves}

The use of the 3D pseudoscalar above to express the sine and cosines was arbitrary, and isn't the only option.  Another obvious choice is the pseudoscalar for the plane normal to the propagation direction.  One such unit pseudoscalar is \( i = I \kcap \), for which \( i^\dagger = -i \), and \( i^2 = -1 \) (as was also the case with the 3D pseudoscalar).
With \( \BE = \BE_r + i \BE_i \), the electromagnetic field can be represented as

\begin{dmath}\label{eqn:ellipticalWaves:240}
F
=
\lr{ 1 + \kcap } \gpgradeone{ \BE e^{i \phi } }.
\end{dmath}

Observe that for this choice of pseudoscalar, the grade selection is a no-op, so the electromagnetic field is real, and is just

\begin{dmath}\label{eqn:ellipticalWaves:260}
F
=
\lr{ 1 + \kcap } \BE e^{i \phi }.
\end{dmath}

For example, with \( \BE = E_0 \Be_1 \), and \( \kcap = \Be_3 \), \( i = \kcap I = \Be_1 \Be_2 \), this is

\begin{dmath}\label{eqn:ellipticalWaves:280}
F
=
\lr{ 1 + \kcap } \BE e^{i \phi }
=
E \lr{ 1 + \Be_3 } \Be_1 \lr{ \cos\phi + \Be_1 \Be_2 \sin\phi }
=
E \lr{ 1 + \Be_3 } \lr{ \Be_1 \cos\phi - \Be_2 \sin\phi }.
\end{dmath}

\paragraph{Linear polarized waves.}

The example above was of a circularly polarized state.  The linear polarized plane wave states can be obtained by superposition.  For example, again with \( \kcap = \Be_3, i = \Be_1 \Be_2 \), linear plane electric field configurations with cosine and sine phase follow from

\begin{dmath}\label{eqn:ellipticalWaves:300}
\begin{aligned}
   \inv{2} E_0 \Be_1 \lr{ e^{i \phi} + e^{-i\phi} } &= E_0 \Be_1 \cos\phi \\
   \inv{2} E_0 \Be_1 \lr{ e^{i \phi} - e^{-i\phi} } &= E_0 \Be_2 \sin\phi.
\end{aligned}
\end{dmath}

\paragraph{Elliptically polarized waves.}

While a circle can be parameterized as

\begin{dmath}\label{eqn:ellipticalWaves:320}
\Br(\phi)
=
r \Be_1 e^{i \phi}
=
\Be_1 \lr{ \cos\phi + i \sin\phi }
=
\Be_1 \cos\phi + \Be_2 \sin\phi,
\end{dmath}

an ellipse can be parameterized as
\begin{dmath}\label{eqn:ellipticalWaves:340}
\Br(\phi)
=
a \Be_1 \cos\phi + b \Be_2 \sin\phi.
\end{dmath}

If \( a, b \) are the semi-major/minor axes of the ellipse (i.e. \( a > b \)),
and \( \Ba = a \Be_1 e^{i\alpha} \) is the vectoral representation of the semimajor axis (not necessarily placed along \( \Be_1 \)),
and \( e = \sqrt{1 - (b/a)^2} \) is the eccentricity of the ellipse,
then an elliptic parameterization can be written
\citep{hestenes1999nfc}
in the compact form

\begin{dmath}\label{eqn:ellipticalWaves:360}
\Br(\phi)
=
e \Ba \cosh( \Atanh(b/a) + i \phi).
\end{dmath}

This is also real and has only vector grades, so the electromagnetic field for a general elliptic wave has the form

\begin{dmath}\label{eqn:ellipticalWaves:380}
F
=
e \lr{ 1 + \kcap } \BE_a
\cosh\lr{ \Atanh\lr{ \frac{\Norm{\BE_b}}{\Norm{\BE_a}}} + I \kcap \phi},
\end{dmath}

where \( \BE_a(\BE_b) \) are the electric field components lying along the semi-major(minor) axes, and the propagation direction \( \kcap \) is normal to both \( \BE_a \) and \(\BE_b\).

\paragraph{Problems.}

\makeproblem{}{problem:ellipticalWaves:1}{
Given \( \BE = \BE_r + I \BE_i \), and \( \kcap \cdot \BE_r = \kcap \cdot \BE_i = 0 \), show that
\( \kcap \gpgradeone{ \BE e^{I \phi } } = \gpgradetwo{ \kcap \BE e^{I \phi } } \).
Also show that \( \gpgradetwo{ \kcap \BE e^{I \phi } } \) can be expanded as an antisymmetric sum of the multivector \( \kcap \BE e^{I\phi} \) and its reverse.
} % problem

\makeanswer{problem:ellipticalWaves:1}{
\begin{dmath}\label{eqn:ellipticalWaves:180}
\gpgradetwo{ \kcap \BE e^{I \phi } }
=
\gpgradetwo{ \kcap \lr{ \BE_r + I \BE_i} e^{I \phi } }
=
\gpgradetwo{ \kcap \lr{ \BE_r \cos\phi - \BE_i \sin\phi + I \BE_i \cos\phi + I \BE_r \sin\phi } }
=
\kcap \wedge \BE_r \cos\phi - \kcap \wedge \BE_i \sin\phi
=
\kcap \BE_r \cos\phi - \kcap \BE_i \sin\phi
=
\kcap \lr{ \BE_r \cos\phi - \kcap \BE_i \sin\phi }
=
\kcap \gpgradeone{ \BE e^{I\phi} }.
\end{dmath}

For the second part, we have

\begin{dmath}\label{eqn:ellipticalWaves:200}
\inv{2} \lr{ \kcap \BE e^{I \phi } - \lr{ \kcap \BE e^{I \phi } }^\dagger }
=
\inv{2} \lr{ \kcap \BE e^{I \phi } - e^{-I \phi } \BE^\dagger \kcap }
=
\inv{2} \lr{ \kcap \BE e^{I \phi } + e^{-I \phi } \kcap \BE^\dagger }
=
\frac{\kcap}{2} \lr{ \BE e^{I \phi } + \BE^\dagger e^{-I \phi } }
=
\frac{\kcap}{2} \lr{ \lr{\BE_r + I \BE_i} \lr{ \cos\phi + I \sin\phi } + \lr{\BE_r - I \BE_i} \lr{ \cos\phi - I \sin\phi } }
=
\frac{\kcap}{2} \lr{
   \BE_r \cos\phi - \BE_i \sin\phi + I \lr{ \BE_i \cos\phi + \BE_r \sin\phi }
+  \BE_r \cos\phi - \BE_i \sin\phi - I \lr{ \BE_i \cos\phi + \BE_r \sin\phi }
}
=
\kcap \lr{ \BE_r \cos\phi - \BE_i \sin\phi }
=
\kcap \gpgradeone{ \BE e^{I \phi} }
=
\gpgradetwo{ \kcap \BE e^{I \phi} }
.
\end{dmath}

} % answer

%      \section{Problem solutions}
%         \shipoutAnswer
%%</TEMPORARY COMMENT OUT FOR QUICKER BUILD.>
