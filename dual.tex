%
% Copyright © 2017 Peeter Joot.  All Rights Reserved.
% Licenced as described in the file LICENSE under the root directory of this GIT repository.
%

\index{dual}
\makedefinition{Dual}{dfn:definitions:dual}{
The dual of a multivector is the product of that multivector with a pseudoscalar for a subspace that contains the multivector.
Such multiplication is referred to as a duality transformation, and can often be interpreted as an operation that produces a normal.
} % definition

The dual vectors to the \R{2} basis vectors are those same vectors rotated by \( \pi/2 \)

\begin{dmath}\label{eqn:definitions:360}
\begin{aligned}
\Be_1 \Be_{12} &= \Be_2 \\
\Be_2 \Be_{12} &= -\Be_1,
\end{aligned}
\end{dmath}

with an inverse duality transformation given by the multiplication with \( \Be_{12}^{-1} = \Be_{21} \)

\begin{dmath}\label{eqn:definitions:440}
\begin{aligned}
\Be_2 \Be_{21} &= \Be_1 \\
-\Be_1 \Be_{21} &= \Be_2.
\end{aligned}
\end{dmath}

The \R{3} duals to the basis vectors are bivectors

\begin{dmath}\label{eqn:definitions:380}
\begin{aligned}
\Be_1 \Be_{123} &= \Be_{23} \\
\Be_2 \Be_{123} &= \Be_{31} \\
\Be_3 \Be_{123} &= \Be_{12},
\end{aligned}
\end{dmath}

whereas the duals to those bivectors with respect to the pseudoscalar \( I^{-1} = \Be_{321} \) are the original basis vectors

\begin{dmath}\label{eqn:definitions:400}
\begin{aligned}
\Be_{23} \Be_{321} &= \Be_1 \\
\Be_{31} \Be_{321} &= \Be_2 \\
\Be_{12} \Be_{321} &= \Be_3.
\end{aligned}
\end{dmath}

In a sense that can be defined more precisely once the general dot product operator is defined, the dual to a given blade represents an object that is normal to the original blade.

The dual of any scalar is a pseudoscalar, whereas the dual of a pseudoscalar is a scalar.

A duality transformation can also be applied to multivectors.
For example in \R{2}, given \( M = 1 + i\), its dual is

\begin{dmath}\label{eqn:dual:460}
M i = i - 1.
\end{dmath}

Because this particular multivector had a complex structure the duality operation can be interpreted as a rotation of a vector.
How to geometrically interpret the duality transformation of a general multivector is not obvious.

%When working with multivector integrals it will be useful to consider the differential volume element a volume weighted pseudoscalar.
