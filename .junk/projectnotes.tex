\paragraph{Context for the project}

An ECE professor from Universit� di Perugia, Prof. Mauro Mongiardo, has reached out to me to collaborate on a book and papers related to applications of geometric algebra (GA) in electromagnetism, particularly focused on engineering applications in the frequency domain.  As discussed, I am interested in perusing this work for two reasons:

\begin{enumerate}
\item
It is intrinsically interesting to me, and I have a strong impression that there is a great deal of potentially interesting engineering applications.
\item
Doing this work in the context of an M.Eng project will help satisfy the ECE graduation requirements for the M.Eng degree program I am enrolled in.  This is especially true given that the electromagnetics group course offerings in recent years has been particularly limited.
\end{enumerate}

I've read and written considerably about applications of geometric algebra outside of a university context.  That writing is scattered throughout the following notes compilations (and probably other locations)

\begin{itemize}
\item Exploring Physics with Geometric Algebra, Part I \citep{gabookI}
\item Exploring Physics with Geometric Algebra, Part II \citep{gabookII}
\item Classical Mechanics \citep{classicalmechanics}
\item Continuum Mechanics \citep{phy454}
\item Advanced Antenna Theory \citep{ece1229}
\end{itemize}

Much of the research into geometric algebra applications to electromagnetism has been in the context of relativistic electromagnetism, where Maxwell's equations take a particularly simple form

\begin{dmath}\label{eqn:projectnotes:20}
\grad F = \inv{\epsilon} J.
\end{dmath}

It can be shown that this consolidates the standard pair of relativistic tensor relations for Maxwell's equations

\begin{dmath}\label{eqn:projectnotes:40}
\begin{aligned}
\partial_\mu F^{\mu\nu} &= \inv{\epsilon} J^\nu \\
\epsilon^{\alpha\beta\gamma\kappa} \partial_\beta F_{\gamma\kappa} &= 0.
\end{aligned}
\end{dmath}

into a single multivector equation.  There is a lot that is left unspecified in the relation above.  A more complete statement has to also define the operators, fields and sources, which are

\begin{dmath}\label{eqn:projectnotes:60}
\begin{aligned}
F &= \BE + I c \BB \\
I &= \gamma_0 \gamma_1 \gamma_2 \gamma_3 \\
\BE &= \sum_{k = 1}^3 \gamma_k \gamma_0 E^k \\
\BB &= \sum_{k = 1}^3 \gamma_k \gamma_0 B^k \\
\BJ &= \sum_{k = 1}^3 \gamma_k \gamma_0 J^k \\
\grad &= \gamma^\mu \partial_\mu = \gamma^\mu \PD{x^\mu}{} \\
x^0 &= c t \\
\partial_0 &= \inv{c} \PD{t}{} \\
c &= 1/\sqrt{\mu\epsilon} \\
J &= \gamma_0 \lr{ c \rho - \BJ },
\end{aligned}
\end{dmath}

where \( \setlr{ \gamma_\mu } \) is a relativistic four-vector basis satisfying \( (\gamma_0)^2 = 1 \), \( (\gamma^k)^2 = -1 \), and \( \gamma^\mu \gamma_\mu = 1 \).  The geometric algebra over this Minkowski (or Dirac) basis is referred to as the Space Time Algebra, or STA \citep{doran2003gap}.  In the STA representation, all spatial vectors \((\BE, \BB, \BJ)\) are represented as bivectors, as is the electromagnetic field \( F \).  The STA form of Maxwell's equation allows problems to be formulated without any explicit reference to either electric or magnetic fields, quantities that are observer dependent.  Lorentz boosts that translate from an observe frame to can be formulated as easily as rotations, which is especially powerful given that rotations in geometric algebra have such a compact representation.
This power comes with a level of abstraction that makes the subject impalpable for applications in engineering.

There is also a considerable learning curve for geometric algebra, and that learning curve is made still steeper by requiring the electromagnetic practitioner to also deal with the relativistic abstractions.

\paragraph{Maxwell's equation in a Euclidean basis}

Maxwell's equation can also be expressed in a compact geometric algebra multivector equation, without the use of
four vectors and non-Euclidean geometries, namely

\begin{dmath}\label{eqn:projectnotes:80}
\lr{ \inv{c} \PD{t}{} + \spacegrad } F = \inv{\epsilon} J
\end{dmath}

where
\begin{dmath}\label{eqn:projectnotes:120}
\begin{aligned}
F &= \BE + I c \BB \\
I &= \Be_1 \Be_2 \Be_3 \\
c &= 1/\sqrt{\mu\epsilon} \\
J &= c \rho - \BJ.
\end{aligned}
\end{dmath}

A choice of a fixed observer frame fixes the representation of the four-gradient, expressed here as a multivector operator, but result is otherwise identical to the representation of Maxwell's equation \cref{eqn:projectnotes:20}.
It is straightforward to show that this representation is equivalent to the normal vectoral form of Maxwell's equations

\begin{dmath}\label{eqn:projectnotes:100}
\begin{aligned}
\spacegrad \cdot \BE &= \inv{\epsilon} \rho \\
\spacegrad \cdot \BB &= 0 \\
\spacegrad \cross \BE &= -\PD{t}{\BB} \\
\spacegrad \cross \BB &= \mu \lr{ \BJ + \epsilon \PD{t}{\BE} }.
\end{aligned}
\end{dmath}

While the fixed observer frame GA representation of Maxwell's equation \cref{eqn:projectnotes:80} is only subtly different from the STA representation, I believe that this representation is preferable to the study of electromagnetism with respect to engineering applications.  Not having to deal with non-Euclidean geometries and four vectors should considerably reduce the learning curve required to exploit the compact GA representation in real world applications.  A compact representation alone is clearly not the only desirable attribute, for if that were the case, engineers would work exclusively with the tensor form of Maxwell's equations \cref{eqn:projectnotes:40}.


As engineers, having time as an independent variable, and an assumption that the geometry we have to deal with is Euclidean, are definite prerequisites!

\paragraph{Suggested Syllabus topics}

M.Eng project on Engineering applications of Geometric Algebra to engineering electromagnetism.

\begin{itemize}
\item Literature search and summary of non-relativistic treatments of electromagnetism in the formalism of geometric algebra.
\item Express the fundamentals of electromagnetism in a fashion that is natural using GA, while also attempting to present that material in a way that does not overwhelm the student with excessive GA theorems.
\item Attempt to determine what the minimal amount of GA theory that must be presented to a new student that will allow the student to focus to be on applications of electromagnetism.
\item Explore topics that have natural expression in relativistic GA, and determine where possible the most natural expression of those relationships in a Euclidean and explicit time formalism.  For example the energy momentum relationship in STA form is
that includes the continuum Lorentz force and Poynting relationships as special cases has an STA form

\begin{dmath}\label{eqn:projectnotes:140}
\begin{aligned}
\grad \cdot T(a) &= \inv{\epsilon c} \gpgradezero{ F a J } \\
T(a) &= -\inv{2} F a F,
\end{aligned}
\end{dmath}

where \( a = a^\mu \gamma_\mu \) is a four vector.  The stress-energy tensor, is usually expressed as the considerably more complex form

\begin{dmath}\label{eqn:projectnotes:160}
T^{\mu\nu} = \frac{1}{\mu_0} \lr{
F^{\mu \alpha}F^\nu{}_{\alpha} - \frac{1}{4} \eta^{\mu\nu}F_{\alpha\beta} F^{\alpha\beta}
}.
\end{dmath}

Note that the Poynting theorem can be recovered from \cref{eqn:projectnotes:140} using a timelike vector such as \( a = \gamma_0 \), whereas the continuum Lorentz force relationships follow by selecting values of \( a \) that are spacelike.

Is there a natural representation Euclidean representation of the energy momentum relationships that is useful for engineering applications, while still highlighting the underlying connections between the Poynting and Lorentz relationships as aspects of a single relativistic concept?
\item Investigate electromagnetic applications of geometric algebra to topics that have a natural geometric bias that is hard to formulate in traditional vector algebra.  One possible example of such an application could be to exploit the ability to simply express rotations in GA for constructions such as Bessel beams.
\item
Because many computational tools exist for standard vector techniques, engineering adoption and exploitation of GA would be facilitated by also enhancing existing CAS systems with support for the underlying product and selection operations.
Pauli matrices allow a compact representation of GA objects and operators (as do Dirac matrices for STA).  This could provide a mechanism for implementing symbolic GA computation engines with existing tools, with potentially less effort then required to implement symbolic GA CAS systems for the general N dimensional non-Euclidean spaces of litte interest to engineering applications.  It would be worthwhile to explore or implement symbolic GA CAS packages for some subset of the languages that may find use in engineering applications:
\begin{itemize}
\item Mathematica.
\item SymPy using a Python or Julia front end.
\item Maxima.
\end{itemize}
\end{itemize}
