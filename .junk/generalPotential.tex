%
% Copyright � 2018 Peeter Joot.  All Rights Reserved.
% Licenced as described in the file LICENSE under the root directory of this GIT repository.
%
%{
%\input{../latex/blogpost.tex}
%\renewcommand{\basename}{generalPotential}
%%\renewcommand{\dirname}{notes/phy1520/}
%\renewcommand{\dirname}{notes/ece1228-electromagnetic-theory/}
%%\newcommand{\dateintitle}{}
%%\newcommand{\keywords}{}
%
%\input{../latex/peeter_prologue_print2.tex}
%
%\usepackage{peeters_layout_exercise}
%\usepackage{peeters_braket}
%\usepackage{peeters_figures}
%\usepackage{siunitx}
%%\usepackage{mhchem} % \ce{}
%%\usepackage{macros_bm} % \bcM
%%\usepackage{macros_qed} % \qedmarker
%%\usepackage{txfonts} % \ointclockwise
%
%\beginArtNoToc
%
%\generatetitle{Time domain multivector potentials}
%%\chapter{Time domain multivector potentials}
\label{chap:generalPotential}

For time dependent fields and sources we need a time derivative in the derivative operator that produces the field from the potential.
\makedefinition{Multivector potential.}{thm:generalPotential:80}{
We call \( A \) the \textit{multivector potential} for the field if
\begin{equation*}
F = \gpgrade{\conjstgrad A}{1,2},
\end{equation*}
and \( (\spacegrad + (1/c)\partial_t) F = J \).
%  That is
%\begin{equation*}
%\stgrad \gpgrade{\conjstgrad A}{1,2} = J.
%\end{equation*}
} % definition

We may write Maxwell's equation in terms of the multivector potential as a modified wave equation
\begin{dmath}\label{eqn:generalPotential:20}
%\lr{ \spacegrad^2 - \inv{c^2} \PDSq{t}{} }
\dLambertian
A = J +
\stgrad \gpgrade{\conjstgrad A}{0,3}.
\end{dmath}
As we saw in statics the pesky grade selection term on the right can be gauge transformed away.  Before formulating that
gauge transformation,
it's worthwhile to unpack
\cref{thm:staticPotentials:380} to show how the electric and magnetic fields are related to the multivector potential, and
get some intuition about the quantity to be transformed away.

\maketheorem{Fields and the potential wave equations.}{thm:generalPotential:40}{
In terms of the potential components, the electric field vector and the magnetic field bivector are
\begin{equation*}
\begin{aligned}
\BE &=
\gpgrade{\conjstgrad A}{1}
=
   - \spacegrad \phi
   - \PD{t}{\BA}
   - \inv{\epsilon} \spacegrad \cross \BF \\
I \eta \BH &=
\gpgrade{\conjstgrad A}{2}
=
   I \eta
   \lr{
      - \spacegrad \phi_\txtm
      - \PD{t}{\BF}
      + \inv{\mu} \spacegrad \cross \BA
   }
.
\end{aligned}
\end{equation*}
The potentials are related to the sources by
%\begin{equation*}
%\begin{aligned}
%- \lr{ \spacegrad^2 - \inv{c^2} \PDSq{t}{} } \phi &= \frac{\rho}{\epsilon} + \inv{c} \PD{t}{} \lr{ c \spacegrad \cdot \BA + \inv{c} \PD{t}{\phi} } \\
%c \lr{ \spacegrad^2 - \inv{c^2} \PDSq{t}{} } \BA &= -\eta \BJ + \spacegrad \lr{ c \spacegrad \cdot \BA + \inv{c} \PD{t}{\phi} } \\
%\eta c I \lr{ \spacegrad^2 - \inv{c^2} \PDSq{t}{} } \BF &= - I \BM + \spacegrad \cdot \lr{ I \eta\lr{ c \spacegrad \cdot \BF + \inv{c} \PD{t}{\phi_\txtm} } } \\
%-I \eta \lr{ \spacegrad^2 - \inv{c^2} \PDSq{t}{} } \phi_\txtm &= I c \rho_\txtm + \inv{c} \PD{t}{} I \eta\lr{ c \spacegrad \cdot \BF + \inv{c} \PD{t}{\phi_\txtm} }
%\end{aligned}
%\end{equation*}
\begin{equation*}
\begin{aligned}
%\lr{ \spacegrad^2 - \inv{c^2} \PDSq{t}{} }
\dLambertian
\phi &= -\frac{\rho}{\epsilon} - \PD{t}{} \lr{ \spacegrad \cdot \BA + \inv{c^2} \PD{t}{\phi} } \\
%\lr{ \spacegrad^2 - \inv{c^2} \PDSq{t}{} }
\dLambertian
\BA &= -\mu \BJ + \spacegrad \lr{ \spacegrad \cdot \BA + \inv{c^2} \PD{t}{\phi} } \\
%\lr{ \spacegrad^2 - \inv{c^2} \PDSq{t}{} }
\dLambertian
\BF &= - \epsilon \BM + \spacegrad \lr{ \spacegrad \cdot \BF + \inv{c^2} \PD{t}{\phi_\txtm} } \\
%\lr{ \spacegrad^2 - \inv{c^2} \PDSq{t}{} }
\dLambertian
\phi_\txtm &= -\frac{\rho_\txtm}{\mu} - \PD{t}{} \lr{ \spacegrad \cdot \BF + \inv{c^2} \PD{t}{\phi_\txtm} }
\end{aligned}
\end{equation*}
} % theorem

To prove \cref{thm:generalPotential:40} we start by expanding \( (\spacegrad - (1/c)\partial_t) A \) using
\cref{dfn:unpackStaticPotential:80} and then group by grade to find
\begin{dmath}\label{eqn:generalPotential:60}
\begin{aligned}
\conjstgrad A
&=
\conjstgrad \lr{  - \phi
      + c \BA
      + \eta I \lr{ -\phi_m + c \BF } } \\
&=
- \spacegrad \phi + c \spacegrad \cdot \BA + c \spacegrad \wedge \BA + \inv{c} \PD{t}{\phi} - \PD{t}{\BA} \\
&\quad + I \eta
\lr{
- \spacegrad \phi_\txtm + c \spacegrad \cdot \BF + c \spacegrad \wedge \BF + \inv{c} \PD{t}{\phi_\txtm} - \PD{t}{\BF}
} \\
&=
c \spacegrad \cdot \BA
+ \inv{c} \PD{t}{\phi}
\\
&
+
\mathLabelBox[ labelstyle={below of=m\themathLableNode, below of=m\themathLableNode} ]
{
   - \spacegrad \phi
   - \PD{t}{\BA}
   - \inv{\epsilon} \spacegrad \cross \BF
}
{
\(\BE\)
}
+
\mathLabelBox[ labelstyle={below of=m\themathLableNode, below of=m\themathLableNode} ]
{
   I \eta
   \lr{
      - \spacegrad \phi_\txtm
      - \PD{t}{\BF}
      + \inv{\mu} \spacegrad \cross \BA
   }
}
{\(I \eta \BH\)
} \\
&
+ I \eta\lr{
  c \spacegrad \cdot \BF
+ \inv{c} \PD{t}{\phi_\txtm}
},
\end{aligned}
\end{dmath}
which shows the claimed field split.
Unpacking Maxwell's equation by grade selection gives
\begin{equation*}
\begin{aligned}
-
%\lr{ \spacegrad^2 - \inv{c^2} \PDSq{t}{} }
\dLambertian
\phi &= \frac{\rho}{\epsilon} + \inv{c} \PD{t}{} \lr{ c \spacegrad \cdot \BA + \inv{c} \PD{t}{\phi} } \\
c
%\lr{ \spacegrad^2 - \inv{c^2} \PDSq{t}{} }
\dLambertian
\BA &= -\eta \BJ + \spacegrad \lr{ c \spacegrad \cdot \BA + \inv{c} \PD{t}{\phi} } \\
\eta c I
%\lr{ \spacegrad^2 - \inv{c^2} \PDSq{t}{} }
\dLambertian
\BF &= - I \BM + \spacegrad \cdot \lr{ I \eta\lr{ c \spacegrad \cdot \BF + \inv{c} \PD{t}{\phi_\txtm} } } \\
-I \eta
%\lr{ \spacegrad^2 - \inv{c^2} \PDSq{t}{} }
\dLambertian
\phi_\txtm &= I c \rho_\txtm + \inv{c} \PD{t}{} I \eta\lr{ c \spacegrad \cdot \BF + \inv{c} \PD{t}{\phi_\txtm} }
\end{aligned}
\end{equation*}
Using \( \eta = \mu c, \eta c \epsilon = 1 \), and
\( \spacegrad \cdot (I \psi) = I \spacegrad \psi \) for scalar \(\psi\), a bit
of rearrangement completes the proof.

%}
%\EndNoBibArticle
