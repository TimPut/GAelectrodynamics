%
% Copyright © 2018 Peeter Joot.  All Rights Reserved.
% Licenced as described in the file LICENSE under the root directory of this GIT repository.
%
%{
\index{vector potential}

Similar to electrostatics where it was assumed that the electric field could be expressed as the gradient of a scalar potential,
a vector potential \( \BA \) solution for the dual of the magnetic field can be assumed
\begin{dmath}\label{eqn:magnetostatics_vectorPotential:480}
\spacegrad \BA = I \BB.
\end{dmath}

As the right hand side is a bivector, we must have \( \spacegrad \cdot \BA = 0 \) for this presumed solution to be valid.
Assuming (for now) a zero divergence constraint for the vector potential, then \cref{eqn:magnetostatics:380} is reduced to
%\begin{dmath}\label{eqn:magnetostatics_vectorPotential:540}
\boxedEquation{eqn:magnetostatics_vectorPotential:540}{
\spacegrad^2 \BA = -\mu \BJ,
}
%\end{dmath}
which can be solved immediately
%\begin{dmath}\label{eqn:magnetostatics_vectorPotential:560}
\boxedEquation{eqn:magnetostatics_vectorPotential:560}{
\BA(\Bx) = \frac{\mu}{4\pi} \int dV' \frac{ \BJ(\Bx') }{\Norm{\Bx - \Bx'}}.
}
%\end{dmath}

The zero divergence constraint for the vector potential is easily dealt with by adding a gradient to the vector potential with the
transformation \( \BA \rightarrow \overbar{\BA} + \spacegrad \chi \).
This gives
\begin{dmath}\label{eqn:magnetostatics_vectorPotential:500}
\spacegrad \BA
=
\spacegrad \overbar{\BA} + \spacegrad^2 \chi
=
\spacegrad \cdot \overbar{\BA} + \spacegrad \wedge \overbar{\BA} + \spacegrad^2 \chi,
\end{dmath}
which has the required bivector grade when \( \spacegrad \cdot \overbar{\BA} = -\spacegrad^2 \chi \), or
\begin{dmath}\label{eqn:magnetostatics_vectorPotential:520}
\chi(\Bx) = \inv{4\pi} \int dV' \frac{ \spacegrad' \cdot \overbar{\BA}(\Bx') }{\Norm{\Bx - \Bx'}}.
\end{dmath}
%}
