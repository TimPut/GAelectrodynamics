%
% Copyright � 2016 Peeter Joot.  All Rights Reserved.
% Licenced as described in the file LICENSE under the root directory of this GIT repository.
%
%{
%\input{../blogpost.tex}
%\renewcommand{\basename}{gradeselection}
%%\renewcommand{\dirname}{notes/phy1520/}
%\renewcommand{\dirname}{notes/ece1228-electromagnetic-theory/}
%%\newcommand{\dateintitle}{}
%%\newcommand{\keywords}{}
%
%\input{../peeter_prologue_print2.tex}
%
%\usepackage{peeters_layout_exercise}
%\usepackage{peeters_braket}
%\usepackage{peeters_figures}
%\usepackage{siunitx}
%%\usepackage{mhchem} % \ce{}
%%\usepackage{macros_bm} % \bcM
%\usepackage{macros_qed} % \qedmarker
%%\usepackage{txfonts} % \ointclockwise
%
%\beginArtNoToc
%
%\generatetitle{XXX}
%%\chapter{XXX}
%%\label{chap:gradeselection}
%
Having defined the axioms and definitions of Geometric Algebra, it desirable to define the grade selection operator, the dot product operator and the wedge product operator, and consider some simple examples of each.

(cut)
As an example, consider two vectors in a 2D space

\begin{dmath}\label{eqn:gradeselection:140}
\begin{aligned}
\Ba  &= \lr{ x \Be_1 + y \Be_2 } \\
\Ba' &= \lr{ x' \Be_1 + y' \Be_2 },
\end{aligned}
\end{dmath}

for which this definition of the dot product gives

\begin{dmath}\label{eqn:gradeselection:160}
\Ba \cdot \Ba'
=
\gpgrade{ \Ba \Ba' }{\Abs{1 - 1}}
=
\gpgradezero{ \Ba \Ba' }
=
\gpgradezero{ \lr{ x \Be_1 + y \Be_2 } \lr{ x' \Be_1 + y' \Be_2 } }
=
\gpgradezero{ x x' \Be_1^2 + y y' \Be_2^2 + (x y' - y x') \Be_1 \Be_2 }
=
x x' + y y'.
\end{dmath}

It is left to the reader (\cref{problem:gradeselection:RnDotProduct}) to show that this definition also reduces to the traditional \R{n} dot product.

As a second example, consider the dot product of a vector with a bivector.  With \( \Ba \) as defined in \cref{eqn:gradeselection:140} and \( i = \Be_1 \Be_2 \)

\begin{dmath}\label{eqn:gradeselection:240}
\Ba \cdot i
=
\gpgrade{ \Ba i }{1}
=
\gpgrade{ \lr{ x \Be_1 + y \Be_2 } \Be_1 \Be_2 }{1}
=
\gpgrade{ x \Be_1^2 \Be_2 + y \Be_2 (-\Be_2 \Be_1) }{1}
=
\gpgrade{ x \Be_2 - y \Be_1 }{1}
=
x \Be_2 - y \Be_1.
\end{dmath}

This particular dot product is trivial, since the product \( \Ba i \) has only a vector component.
In this example \( i \) is the pseudoscalar for the two dimensional space, and it can be observed that multiplication of a vector from the right serves to rotate the vector by 90 degrees.  It is not a coincidence that this is strikingly similar to the action of the imaginary from complex algebra.  It can be shown (\cref{problem:gradeselection:PlaneRotations})
that \( e^{i\theta} \) acts as a rotation operator as it does in complex algebra, and that a GA representation of complex numbers is possible (\cref{problem:gradeselection:ComplexNumbers}).


%}
%\EndNoBibArticle
