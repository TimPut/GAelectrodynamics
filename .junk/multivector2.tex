%
% Copyright © 2017 Peeter Joot.  All Rights Reserved.
% Licenced as described in the file LICENSE under the root directory of this GIT repository.
%
\subsection{Irreducible products}

Armed with the contraction axiom and \cref{eqn:multiplication:140} it is now possible to show how to put a multivector into an irreducible form.  As an example, consider

\begin{equation}\label{eqn:SimpleProducts:20}
M = \Be_3 \Be_3 + 2 \Be_1 \Be_2 \Be_1 + \Be_2 \Be_3 - 5 \Be_3 \Be_1 \Be_3 \Be_2 + \Be_4 \Be_1 \Be_4 \Be_2 \Be_3 + \Be_1 \Be_2 \Be_1 \Be_3 \Be_4 \Be_5.
\end{equation}

Application of the contraction axiom shows that the first term is a scalar

\begin{equation}\label{eqn:SimpleProducts:40}
\Be_3 \Be_3 = 1.
\end{equation}

The second term is a vector, as it is possible to reorder normal products (changing sign each time) and regroup terms to apply the contraction axiom, as follows

\begin{dmath}\label{eqn:SimpleProducts:60}
2 \Be_1 \Be_2 \Be_1
=
2 \Be_1 \lr{ \Be_2 \Be_1 }
=
2 \Be_1 \lr{ - \Be_1 \Be_2 }
=
-2 \Be_1 \Be_1 \Be_2
=
-2 \lr{ \Be_1 \Be_1 } \Be_2
=
-2 \Be_2.
\end{dmath}

The third term is a bivector and cannot be reduced further.  The fourth term is also a bivector

\begin{dmath}\label{eqn:SimpleProducts:80}
- 5 \Be_3 \Be_1 \Be_3 \Be_2
=
- 5 \lr{ \Be_3 \Be_1 } \Be_3 \Be_2
=
+ 5 \lr{ \Be_1 \Be_3 } \Be_3 \Be_2
=
+ 5 \Be_1 \lr{ \Be_3 \Be_3 } \Be_2
=
+ 5 \Be_1 \Be_2.
\end{dmath}

As the fifth term has repeated indexes, is is also reducible too

\begin{dmath}\label{eqn:SimpleProducts:100}
\Be_4 \Be_1 \Be_4 \Be_2 \Be_3
=
\lr{ \Be_4 \Be_1} \Be_4 \Be_2 \Be_3
=
-\lr{ \Be_1 \Be_4} \Be_4 \Be_2 \Be_3
=
- \Be_1 \lr{ \Be_4 \Be_4 } \Be_2 \Be_3
=
- \Be_1 \Be_2 \Be_3.
\end{dmath}

The reader should demonstrate that the final term has grade four, and can be reduced to \( -\Be_2 \Be_3 \Be_4 \Be_5 \).

\begin{dmath}\label{eqn:SimpleProducts:120}
M = 1 - 2 \Be_2  + \Be_2 \Be_3 + 5 \Be_1 \Be_2 - \Be_1 \Be_2 \Be_3 -\Be_2 \Be_3 \Be_4 \Be_5.
  = 1 - 2 \Be_2  + \Be_{23} + 5 \Be_{12} - \Be_{123} -\Be_{2345}.
\end{dmath}

\subsection{Mixed grade sums}
In traditional vector algebra, the
``weird'' sum of a scalar and vector is forbidden and undefined, but is explicitly allowed in GA.  For example,

\begin{dmath}\label{eqn:multivector:240}
1 + \Be_1,
\end{dmath}

is a simple mixed grade multivector.
Such mixed grade mathematical objects are not only well defined in GA, but are required to represent some vector products.  One of the simplest examples is the following vector product

\begin{dmath}\label{eqn:multivector:260}
\Be_1 ( \Be_1 + \Be_2 )
=
\Be_1 \Be_1 + \Be_1 \Be_2
=
\Be_1 \cdot \Be_1 + \Be_1 \Be_2
=
1 + \Be_1 \Be_2,
\end{dmath}

where the last step assumes the vector space is Euclidean.

