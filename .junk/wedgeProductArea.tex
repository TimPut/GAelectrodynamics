
A simple, but inelegant, way to prove this is to specify a coordinate system for which \( \Ba, \Bb \) both lie in the \( x,y \) plane.  Then \( \Ba \wedge \Bb = \alpha i \) for some \( \alpha \), and the anticommutation part of the theorem follows from
%the \R{2} result
\cref{eqn:SimpleProducts2:1760}.  For the normal commutation part of the theorem, pick any vector normal to the \(x, y\) plane, say \( \Be_3\), for which we have

\begin{dmath}\label{eqn:SimpleProducts2:1780}
\Be_3  (\Ba \wedge \Bb)
=
\Be_3 \alpha i
=
\alpha \Be_3 \Be_1 \Be_2
=
\alpha (-\Be_1 \Be_3) \Be_2
=
-\alpha \Be_1 (\Be_3 \Be_2)
=
-\alpha \Be_1 (-\Be_2 \Be_3)
= (\Ba \wedge \Bb) \Be_3.
\end{dmath}

In dimensions with more normals, say \( \Be_4, \cdots \), the steps of \cref{eqn:SimpleProducts2:1780} can be repeated.  The general normal commuation result follows by superposition.
