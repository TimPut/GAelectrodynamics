%
% Copyright © 2017 Peeter Joot.  All Rights Reserved.
% Licenced as described in the file LICENSE under the root directory of this GIT repository.
%

Poynting's theorem, the conservation relationship between energy and momentum density (or more generally, the energy-momentum tensor) and the sources, can be stated in terms of the multivector field \( F \) and the multivector current \( J \).
To derive this relationship we can act on (all terms of) \( F a F^\dagger \) with the space+time derivative operator \( \spacegrad + (1/c) \partial_t \).
We do so within a scalar selection operation, which simplifies things, and allows for cyclic permutation of the multivector factors (i.e. \(\gpgradezero{ABC} = \gpgradezero{CAB}\)).
\begin{dmath}\label{eqn:poyntingTheorem:100}
\frac{\epsilon}{2} \gpgradezero{ \lr{ c \spacegrad + \PD{t}{} } F a F^\dagger }
=
\frac{\epsilon}{2} \gpgradezero{ \lr{ c \spacegrad + \PD{t}{} } \dot{F} a F^\dagger }
+
\frac{\epsilon}{2} \gpgradezero{ \lr{ c \spacegrad + \PD{t}{} } F a \dot{F}^\dagger }
=
\frac{\epsilon}{2} \gpgradezero{ c J a F^\dagger }
+
\frac{\epsilon}{2} \gpgradezero{ \dot{F}^\dagger \lr{ c \spacegrad + \PD{t}{} } F a },
\end{dmath}
where
the over-dot notation of
\citep{hestenes1999nfc} was used to indicate the desired action of the derivative operators in the
chain rule expansion of
\cref{eqn:poyntingTheorem:100}
, as the gradient may not commute with \( F \).  Another application of Maxwell's equation reduces this further
\begin{dmath}\label{eqn:poyntingTheorem:960}
\frac{\epsilon}{2} \gpgradezero{ \lr{ c \spacegrad + \PD{t}{} } F a F^\dagger }
=
\frac{\epsilon}{2} \gpgradezero{ c J F^\dagger a }
+
\frac{\epsilon}{2} \gpgradezero{ \lr{ \lr{ c \spacegrad + \PD{t}{} } F }^\dagger F a }
=
c \frac{\epsilon}{2} \gpgradezero{ F^\dagger J a + J^\dagger F a },
\end{dmath}
or
\boxedEquation{eqn:poyntingF:980}{
c \spacegrad \cdot \gpgradeone{ \frac{\epsilon}{2} F a F^\dagger }
+ \PD{t}{} \gpgradezero{ \frac{\epsilon}{2} F a F^\dagger }
=
\frac{1}{2 \eta} \gpgradezero{ a \lr{ F^\dagger J + J^\dagger F} }.
}

For \( a = 1 \), since scalars are reversion invariant (\(\alpha^\dagger = \alpha\) for any scalars \( \alpha \))
\begin{equation}\label{eqn:poyntingTheorem:1000}
\gpgradezero{ F^\dagger J }
=
\gpgradezero{ F^\dagger J }^\dagger
=
\gpgradezero{ J^\dagger F },
\end{equation}
so the
multivector form of Poynting's theorem with respect to the time variation of the energy of the field is
\index{Poynting theorem}
\boxedEquation{eqn:poyntingF:220}{
c \spacegrad \cdot \gpgradeone{ \frac{\epsilon}{2} F F^\dagger }
+ \PD{t}{} \gpgradezero{ \frac{\epsilon}{2} F F^\dagger }
=
\inv{\eta} \gpgradezero{ J^\dagger F }.
}

The conventional statement of this theorem in terms of \( \BD, \BE, \BB, \BH, \BJ, \BM \) follows by direct substitution.
The multivector current \( J \) and its reverse are
\begin{dmath}\label{eqn:poyntingTheorem:160}
\begin{aligned}
J &= \eta \lr{ c \rho - \BJ } + I \lr{ c \rho_m - \BM } \\
J^\dagger &= \eta \lr{ c \rho - \BJ } - I \lr{ c \rho_m - \BM },
\end{aligned}
\end{dmath}
so
\begin{dmath}\label{eqn:poyntingTheorem:180}
0 =
\spacegrad \cdot \BS
-
\inv{\eta}
\lr{
- \eta \BJ \cdot \BE
- \eta \BM \cdot \BH
}
+ \PD{t}{\calE},
\end{dmath}
or
\boxedEquation{eqn:poyntingF:200}{
\spacegrad \cdot \BS + \BJ \cdot \BE + \BM \cdot \BH
%+ \PD{t}{\calE} = 0.
+ \PD{t}{\BB} \cdot \BH
+ \PD{t}{\BD} \cdot \BE = 0.
}

The sum of the last two terms is the time rate of change of the energy density.
In particular,
with neither electric nor magnetic current sources in a region of space,
the change of energy density through a volume is matched by a corresponding flux through the bounding surface
\begin{dmath}\label{eqn:poyntingTheorem:740}
\PD{t}{} \int_V
\inv{2} dV \lr{
\BB \cdot \BH
+ \BD \cdot \BE
}
=
-\int_{\partial V} dA \ncap \cdot \BS.
\end{dmath}

Here \( \ncap \) is the outward normal, so if the energy contained in the volume is decreasing, then \( \BS \) must represent the energy per unit area that leaves the volume.
The direction of the Poynting vector is the direction that the energy is leaving the volume.
Only the components of the Poynting vector that are colinear with the surface normal will result in energy leaving or entering the volume.

