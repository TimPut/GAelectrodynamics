
\index{wave equation}
%Having assembled all of Maxwell's equations into \cref{dfn:isotropicMaxwells:680}, some results now follow almost trivially.
%One such result is the wave equation in space free of sources.
%In such a region, Maxwell's equation is just
%\begin{dmath}\label{eqn:waveequation:480}
%\lr{ \spacegrad + \inv{c} \PD{t}{} } F = 0.
%\end{dmath}
%
%This can be multiplied from the left with the multivector operator \( \spacegrad - \inv{c} \PD{t}{} \), to give
%\begin{dmath}\label{eqn:waveequation:500}
%0 =
%\lr{ \spacegrad - \inv{c} \PD{t}{} }
%\lr{ \spacegrad + \inv{c} \PD{t}{} } F
%=
%\lr{ \spacegrad^2 - \inv{c^2} \PDSq{t}{} } F,
%\end{dmath}
%or
In source free conditions \cref{eqn:continuity:110} reduces to an homogeneous wave equation
%\begin{dmath}\label{eqn:waveequation:520}
\boxedEquation{eqn:continuity:520}{
\spacegrad^2 F = \inv{c^2} \PDSq{t}{F}.
}
%\end{dmath}

The solutions of the wave equation are well known, allowing us to immediately state that the solution is
\begin{dmath}\label{eqn:continuity:560}
F(\Bx, t) = f(\Norm{\Bx} \pm c t),
\end{dmath}
where \( f \) is any grade 1,2 multivector, provided that \( F \) also satisfies the constraints imposed by
Maxwell's equation \cref{dfn:isotropicMaxwells:680}.

In conventional electromagnetism, we have independent wave equations for
each of the electric and magnetic fields.
We can obtain those by applying vector and bivector grade selection operations to
\cref{eqn:continuity:520} to find
\begin{dmath}\label{eqn:waveequation:540}
\begin{aligned}
\spacegrad^2 \BE &= \inv{c^2} \PDSq{t}{\BE} \\
\spacegrad^2 (I \BH) &= \inv{c^2} \PDSq{t}{(I \BH)}.
\end{aligned}
\end{dmath}
The pseudoscalar factors in the magnetic field wave equation can be eliminated by multiplying with
\( -I \), which yields the conventional pair of vector wave equations for the electric and magnetic fields.  Any solutions
to
\cref{eqn:waveequation:540} are also constrained by Maxwell's equation.
