%
% Copyright © 2018 Peeter Joot.  All Rights Reserved.
% Licenced as described in the file LICENSE under the root directory of this GIT repository.
%
%{

\index{gauge transformation}

Because the potential representation of the field is expressed as a grade 1,2 selection, the addition of scalar or pseudoscalar components to the grade selection will not alter the field.
In particular, it is possible to alter the multivector potential
\begin{dmath}\label{eqn:potentialSection_gauge:160}
A \rightarrow A + \stgrad \psi,
\end{dmath}
where \( \psi \) is any multivector field with scalar and pseudoscalar grades, without changing the field
\begin{dmath}\label{eqn:potentialSection_gauge:180}
F
\rightarrow
\gpgrade{
   \conjstgrad
   \lr{ A + \stgrad \psi }
}{1,2}
=
F +
\gpgrade{
   \lr{ \spacegrad^2 - \inv{c^2} \PDSq{t}{}} 
%\dLambertian
\psi
}{1,2}
.
\end{dmath}

That last grade selection is zero, since \( \psi \) has no vector or bivector grades, demonstrating that the electromagnetic field is invariant with respect to this multivector potential transformation.

It is worth looking how such a transformation impacts each grade of the potential.
Let \( \psi = c \psi^\e + \eta c I \psi^\m \), where \( \psi^\e \) and \( \psi^\m \) are both scalar fields.
The gauge transformation provides the mapping

\begin{subequations}
\label{eqn:potentialSection_gauge:220}
\begin{dmath}\label{eqn:potentialSection_gauge:200}
- \phi \rightarrow - \phi + \PD{t}{} \psi^\e
\end{dmath}
\begin{dmath}\label{eqn:potentialSection_gauge:240}
c \BA \rightarrow c \BA + c \spacegrad \psi^\e
\end{dmath}
\begin{dmath}\label{eqn:potentialSection_gauge:260}
I c \BF \rightarrow I c \BF + I c \spacegrad \psi^\m
\end{dmath}
\begin{dmath}\label{eqn:potentialSection_gauge:280}
- I \eta \phi_m \rightarrow -I \eta \phi_m + I \eta \PD{t}{} \psi^\m,
\end{dmath}
\end{subequations}

or

\begin{subequations}
\label{eqn:potentialSection_gauge:400}
\begin{dmath}\label{eqn:potentialSection_gauge:420}
\phi \rightarrow \phi - \PD{t}{} \psi^\e
\end{dmath}
\begin{dmath}\label{eqn:potentialSection_gauge:440}
\BA \rightarrow \BA + \spacegrad \psi^\e
\end{dmath}
\begin{dmath}\label{eqn:potentialSection_gauge:460}
\BF \rightarrow \BF + \spacegrad \psi^\m
\end{dmath}
\begin{dmath}\label{eqn:potentialSection_gauge:480}
\phi_m \rightarrow \phi_m - \PD{t}{} \psi^\m.
\end{dmath}
\end{subequations}

These have the alternation of sign that is found in the usual recipe for gauge transformation of the scalar and vector potentials.
In conventional electromagnetism, the first two relations are usually found by observing it is possible to add any gradient to the vector potential, and then finding the transformation consequences that that choice imposes on the electric field.
With the grade selection formulation of the electromagnetic field, this special coupling of the field potentials comes for free without having to consider the curl of a specific field component.

Note that the latter two dual transformation relationships are for magnetic sources, and are usually expressed in the frequency domain, where the gauge transformations take the form

\begin{subequations}
\label{eqn:potentialSection_gauge:300}
\begin{dmath}\label{eqn:potentialSection_gauge:320}
\phi \rightarrow \phi - j \omega \psi^\e
\end{dmath}
\begin{dmath}\label{eqn:potentialSection_gauge:340}
\BA \rightarrow \BA + \spacegrad \psi^\e
\end{dmath}
\begin{dmath}\label{eqn:potentialSection_gauge:360}
\BF \rightarrow \BF + \spacegrad \psi^\m
\end{dmath}
\begin{dmath}\label{eqn:potentialSection_gauge:380}
\phi_m \rightarrow \phi_m -j \omega \psi^\m.
\end{dmath}
\end{subequations}

%}
