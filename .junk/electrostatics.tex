%
% Copyright © 2017 Peeter Joot.  All Rights Reserved.
% Licenced as described in the file LICENSE under the root directory of this GIT repository.
%
(cut)

\begin{subequations}
\label{eqn:electrostatics:99}
\begin{dmath}\label{eqn:electrostatics:100}
\spacegrad \cross \BE = 0
\end{dmath}
\begin{dmath}\label{eqn:electrostatics:120}
\spacegrad \cross \BB = 0
\end{dmath}
\begin{dmath}\label{eqn:electrostatics:140}
\spacegrad \cdot \BE = \frac{\rho}{\epsilon}
\end{dmath}
\begin{dmath}\label{eqn:electrostatics:160}
\spacegrad \cdot \BB = 0.
\end{dmath}
\end{subequations}

All the complicated coupling of the electric and magnetic fields is eliminated, and the only source term remaining is a time independent charge density \( \rho = \rho(\Bx) \).

Utilizing \cref{eqn:SimpleProducts2:1640}, the geometric product of the gradient \( \spacegrad \) with a vector \( \Ba = \Bx(\Bx) \) is
\begin{dmath}\label{eqn:electrostatics:240}
\spacegrad \Ba = \spacegrad \cdot \Ba + I(\spacegrad \cross \Ba).
\end{dmath}

\Cref{eqn:electrostatics:240} can be used to rewrite the electrostatic Maxwell equations (\cref{eqn:electrostatics:99}), as a pair of multivector gradient equations
%\begin{subequations}
%\label{eqn:electrostatics:360}
%\begin{equation}\label{eqn:electrostatics:380}
%\begin{equation}\label{eqn:electrostatics:400}
\boxedEquation{eqn:electrostatics:380}{
\begin{aligned}
\spacegrad \BE &= \frac{\rho}{\epsilon} \\
\spacegrad \BB &= 0.
\end{aligned}
}
%\end{subequations}

