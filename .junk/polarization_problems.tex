%
% Copyright © 2018 Peeter Joot.  All Rights Reserved.
% Licenced as described in the file LICENSE under the root directory of this GIT repository.
%
%{

\makeproblem{Circular polarization coefficients relationship to the Jones vector.}{problem:polarization:1}{
By substituting \cref{eqn:polarization_circular:220} into \cref{eqn:polarization_circular:200}, and comparing to \cref{eqn:polarization_circular:160},
show that the circular state coefficients have the following relationship to the Jones vector coordinates
\begin{equation*}
\begin{aligned}
\alpha_\txtL &= \lr{ \alpha_1 + \beta_2 }/2 + i \lr{ -\alpha_2 + \beta_1 }/2 \\
\alpha_\txtR &= \lr{ \alpha_1 - \beta_2 }/2 + i \lr{ -\alpha_2 - \beta_1 }/2,
\end{aligned}
\end{equation*}
and use this to prove \cref{eqn:polarization_circular:260}.
} % problem

\makeproblem{Pseudoscalar Jones vector.}{problem:polarization:2}{
With the Jones vector defined in terms of the \R{3} pseudoscalar
\begin{equation*}
\begin{aligned}
c_1 &= \alpha_1 + I \beta_1 \\
c_2 &= \alpha_2 + I \beta_2,
\end{aligned}
\end{equation*}
calculate the values \( \alpha_\txtL', \alpha_\txtR' \) of \cref{eqn:polarization_pseudoscalarImaginary:480} in terms of this Jones vector.
} % problem
%}
