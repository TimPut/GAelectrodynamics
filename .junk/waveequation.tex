%
% Copyright © 2016 Peeter Joot.  All Rights Reserved.
% Licenced as described in the file LICENSE under the root directory of this GIT repository.
%
%\section{Wave equation.}
\index{wave equation}
%Having assembled all of Maxwell's equations into \cref{dfn:isotropicMaxwells:680}, some results now follow almost trivially.
%One such result is the wave equation in space free of sources.
%In such a region, Maxwell's equation is just
%\begin{dmath}\label{eqn:waveequation:480}
%\lr{ \spacegrad + \inv{c} \PD{t}{} } F = 0.
%\end{dmath}
%
%This can be multiplied from the left with the multivector operator \( \spacegrad - \inv{c} \PD{t}{} \), to give
%\begin{dmath}\label{eqn:waveequation:500}
%0 =
%\lr{ \spacegrad - \inv{c} \PD{t}{} }
%\lr{ \spacegrad + \inv{c} \PD{t}{} } F
%=
%\lr{ \spacegrad^2 - \inv{c^2} \PDSq{t}{} } F,
%\end{dmath}
%or
In source free conditions
\begin{dmath}\label{eqn:waveequation:520}
\spacegrad^2 F = \inv{c^2} \PDSq{t}{F}.
\end{dmath}

Since \( \spacegrad^2 \) is a scalar operator, selection of the vector and bivector components of \cref{eqn:waveequation:520} gives
\begin{dmath}\label{eqn:waveequation:540}
\begin{aligned}
\spacegrad^2 \BE &= \inv{c^2} \PDSq{t}{\BE} \\
\spacegrad^2 (I \BH) &= \inv{c^2} \PDSq{t}{(I \BH)}.
\end{aligned}
\end{dmath}

These equations can be solved independently, provided the solutions are also constrained by Maxwell's equation \cref{eqn:waveequation:480}.
