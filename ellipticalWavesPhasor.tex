%
% Copyright © 2017 Peeter Joot.  All Rights Reserved.
% Licenced as described in the file LICENSE under the root directory of this GIT repository.
%
\paragraph{Real Phasor representation.}

A real time dependent field, represented in terms of a complex vector valued phasor \( \tilde{\BA} \), is formed by taking the real part of the product of that phasor with the phase exponential

\begin{dmath}\label{eqn:ellipticalWaves:20}
\bcA
= \Real\lr{ \tilde{\BA} e^{j \Bk \cdot \Bx -j \omega t} }
=
\BA_r \cos\lr{ \Bk \cdot \Bx - \omega t }
- \BA_i \sin\lr{ \Bk \cdot \Bx - \omega t }.
\end{dmath}

In the complex representation above, the imaginary \( j \) is not interpreted geometrically, but like the unit pseudoscalar \( I = \Be_1 \Be_2 \Be_3 \), squares to \( -1 \) and commutes with all grades.  It is therefore possible to express the field using the pseudoscalar as the imaginary

\begin{dmath}\label{eqn:ellipticalWaves:40}
\bcA
=
\inv{2} \BA_r
\lr{ e^{I \phi} + e^{-I\phi} }
   - \inv{2 I} \BA_i
   \lr{ e^{I \phi} - e^{-I\phi} }
=
\inv{2}\lr{ \BA_r + I \BA_i } e^{I \phi}
+
\inv{2}\lr{ \BA_r - I \BA_i } e^{-I \phi}
=
\inv{2} \lr{ \BA e^{I \phi} + \lr{ \BA e^{I \phi} }^\dagger }
,
\end{dmath}

where the phase angle was written as

\begin{dmath}\label{eqn:ellipticalWaves:400}
\phi = \Bk \cdot \Bx - \omega t,
\end{dmath}

and where the field magnitude and orientation has been specified by a ``complex'' (grade-1,3) multivector

\begin{dmath}\label{eqn:ellipticalWaves:120}
\BA = \BA_r + I \BA_i,
\end{dmath}

and its reverse \( \BA^\dagger \).
This has the structure of a real-part operation, where the real part is represented by half the multivector plus its reverse.  This is in fact one way of expressing the vector grade selection operation for the grade-1,3 multivector \( \BA e^{I\phi} \), which can also be considered a phasor representation

\begin{dmath}\label{eqn:ellipticalWaves:80}
\BA e^{I \phi}
=
\lr{ \BA_r + I \BA_i }
\lr{ \cos\phi + I \sin\phi }
=
\lr{ \BA_r + I \BA_i }
\lr{ \cos\phi + I \sin\phi }
=
\BA_r \cos\phi - \BA_i \sin\phi
+ I \BA_i \cos\phi + I \BA_r \sin\phi.
\end{dmath}

Adding this to its reverse (which negates the sign of the pseudoscalar, but not the vector), eliminates all the bivector components of this multivector phasor representation.  It is now possible to represent the field completely in terms of real vectors and a vector grade selection operation

\begin{dmath}\label{eqn:ellipticalWaves:100}
\bcA = \gpgradeone{ \BA e^{I \phi } }.
\end{dmath}

\paragraph{Electromagnetic plane wave.}

The electromagnetic field, with \( \BE = \BE_r + j \BE_i \), where \( \BE_r \cdot \kcap = \BE_i \cdot \kcap = 0 \), for a plane wave is

\begin{dmath}\label{eqn:ellipticalWaves:140}
   F = \Real \lr{ \lr{ 1 + \kcap } \BE e^{j \phi} }.
\end{dmath}

(I have a derivation of this elsewhere, but there is also one in \citep{doran2003gap})

The real representation, with multivector phasor \( \BE = \BE_r + I \BE_i \), is

\begin{dmath}\label{eqn:ellipticalWaves:160}
F
=
\lr{ 1 + \kcap } \gpgradeone{ \BE e^{I \phi } }
=
\inv{2} \lr{ 1 + \kcap } \lr{ \BE e^{I \phi } + \BE^\dagger e^{-I \phi } }.
\end{dmath}

Note that this is not equal to \(
\inv{2} \lr{
   \lr{ 1 + \kcap } \BE e^{I \phi }
   +
\lr{ \lr{ 1 + \kcap } \BE e^{I \phi } }^\dagger } \), since \( \lr{ \kcap \BE }^\dagger = -\kcap \BE^\dagger \).
% because \( \kcap \) is normal to both the \( \BE_r \) and \( \BE_i \) vectors.

Should the electric and magnetic fields be desired explicitly, they can be obtained by the grade selection, with

\begin{equation}\label{eqn:ellipticalWaves:220}
F = \bcE + I \eta \bcH
=
\gpgradeone{ \BE e^{I \phi } } +
\gpgradetwo{ \kcap \BE e^{I \phi } },
\end{equation}

where this split into electric (vector) and magnetic (bivector) field components was facilitated by
the fact that
\( \kcap \gpgradeone{ \BE e^{I \phi } } = \gpgradetwo{ \kcap \BE e^{I \phi } } \) [exercise].

\paragraph{Circular waves}

The use of the 3D pseudoscalar above to express the sine and cosines was arbitrary, and isn't the only option.  Another obvious choice is the pseudoscalar for the plane normal to the propagation direction.  One such unit pseudoscalar is \( i = I \kcap \), for which \( i^\dagger = -i \), and \( i^2 = -1 \) (as was also the case with the 3D pseudoscalar).
With \( \BE = \BE_r + i \BE_i \), the electromagnetic field can be represented as

\begin{dmath}\label{eqn:ellipticalWaves:240}
F
=
\lr{ 1 + \kcap } \gpgradeone{ \BE e^{i \phi } }.
\end{dmath}

Observe that for this choice of pseudoscalar, the grade selection is a no-op, so the electromagnetic field is real, and is just

\begin{dmath}\label{eqn:ellipticalWaves:260}
F
=
\lr{ 1 + \kcap } \BE e^{i \phi }.
\end{dmath}

For example, with \( \BE = E_0 \Be_1 \), and \( \kcap = \Be_3 \), \( i = \kcap I = \Be_1 \Be_2 \), this is

\begin{dmath}\label{eqn:ellipticalWaves:280}
F
=
\lr{ 1 + \kcap } \BE e^{i \phi }
=
E \lr{ 1 + \Be_3 } \Be_1 \lr{ \cos\phi + \Be_1 \Be_2 \sin\phi }
=
E \lr{ 1 + \Be_3 } \lr{ \Be_1 \cos\phi - \Be_2 \sin\phi }.
\end{dmath}

\paragraph{Linear polarized waves.}

The example above was of a circularly polarized state.  The linear polarized plane wave states can be obtained by superposition.  For example, again with \( \kcap = \Be_3, i = \Be_1 \Be_2 \), linear plane electric field configurations with cosine and sine phase follow from

\begin{dmath}\label{eqn:ellipticalWaves:300}
\begin{aligned}
   \inv{2} E_0 \Be_1 \lr{ e^{i \phi} + e^{-i\phi} } &= E_0 \Be_1 \cos\phi \\
   \inv{2} E_0 \Be_1 \lr{ e^{i \phi} - e^{-i\phi} } &= E_0 \Be_2 \sin\phi.
\end{aligned}
\end{dmath}

\paragraph{Elliptically polarized waves.}

While a circle can be parameterized as

\begin{dmath}\label{eqn:ellipticalWaves:320}
\Br(\phi)
=
r \Be_1 e^{i \phi}
=
\Be_1 \lr{ \cos\phi + i \sin\phi }
=
\Be_1 \cos\phi + \Be_2 \sin\phi,
\end{dmath}

an ellipse can be parameterized as
\begin{dmath}\label{eqn:ellipticalWaves:340}
\Br(\phi)
=
a \Be_1 \cos\phi + b \Be_2 \sin\phi.
\end{dmath}

If \( a, b \) are the semi-major/minor axes of the ellipse (i.e. \( a > b \)),
and \( \Ba = a \Be_1 e^{i\alpha} \) is the vectoral representation of the semimajor axis (not necessarily placed along \( \Be_1 \)),
and \( e = \sqrt{1 - (b/a)^2} \) is the eccentricity of the ellipse,
then an elliptic parameterization can be written
\citep{hestenes1999nfc}
in the compact form

\begin{dmath}\label{eqn:ellipticalWaves:360}
\Br(\phi)
=
e \Ba \cosh( \Atanh(b/a) + i \phi).
\end{dmath}

This is also real and has only vector grades, so the electromagnetic field for a general elliptic wave has the form

\begin{dmath}\label{eqn:ellipticalWaves:380}
F
=
e \lr{ 1 + \kcap } \BE_a
\cosh\lr{ \Atanh\lr{ \frac{\Norm{\BE_b}}{\Norm{\BE_a}}} + I \kcap \phi},
\end{dmath}

where \( \BE_a(\BE_b) \) are the electric field components lying along the semi-major(minor) axes, and the propagation direction \( \kcap \) is normal to both \( \BE_a \) and \(\BE_b\).

\paragraph{Problems.}

\makeproblem{}{problem:ellipticalWaves:1}{
Given \( \BE = \BE_r + I \BE_i \), and \( \kcap \cdot \BE_r = \kcap \cdot \BE_i = 0 \), show that
\( \kcap \gpgradeone{ \BE e^{I \phi } } = \gpgradetwo{ \kcap \BE e^{I \phi } } \).
Also show that \( \gpgradetwo{ \kcap \BE e^{I \phi } } \) can be expanded as an antisymmetric sum of the multivector \( \kcap \BE e^{I\phi} \) and its reverse.
} % problem

\makeanswer{problem:ellipticalWaves:1}{
\begin{dmath}\label{eqn:ellipticalWaves:180}
\gpgradetwo{ \kcap \BE e^{I \phi } }
=
\gpgradetwo{ \kcap \lr{ \BE_r + I \BE_i} e^{I \phi } }
=
\gpgradetwo{ \kcap \lr{ \BE_r \cos\phi - \BE_i \sin\phi + I \BE_i \cos\phi + I \BE_r \sin\phi } }
=
\kcap \wedge \BE_r \cos\phi - \kcap \wedge \BE_i \sin\phi
=
\kcap \BE_r \cos\phi - \kcap \BE_i \sin\phi
=
\kcap \lr{ \BE_r \cos\phi - \kcap \BE_i \sin\phi }
=
\kcap \gpgradeone{ \BE e^{I\phi} }.
\end{dmath}

For the second part, we have

\begin{dmath}\label{eqn:ellipticalWaves:200}
\inv{2} \lr{ \kcap \BE e^{I \phi } - \lr{ \kcap \BE e^{I \phi } }^\dagger }
=
\inv{2} \lr{ \kcap \BE e^{I \phi } - e^{-I \phi } \BE^\dagger \kcap }
=
\inv{2} \lr{ \kcap \BE e^{I \phi } + e^{-I \phi } \kcap \BE^\dagger }
=
\frac{\kcap}{2} \lr{ \BE e^{I \phi } + \BE^\dagger e^{-I \phi } }
=
\frac{\kcap}{2} \lr{ \lr{\BE_r + I \BE_i} \lr{ \cos\phi + I \sin\phi } + \lr{\BE_r - I \BE_i} \lr{ \cos\phi - I \sin\phi } }
=
\frac{\kcap}{2} \lr{
   \BE_r \cos\phi - \BE_i \sin\phi + I \lr{ \BE_i \cos\phi + \BE_r \sin\phi }
+  \BE_r \cos\phi - \BE_i \sin\phi - I \lr{ \BE_i \cos\phi + \BE_r \sin\phi }
}
=
\kcap \lr{ \BE_r \cos\phi - \BE_i \sin\phi }
=
\kcap \gpgradeone{ \BE e^{I \phi} }
=
\gpgradetwo{ \kcap \BE e^{I \phi} }
.
\end{dmath}

} % answer
