%
% Copyright © 2017 Peeter Joot.  All Rights Reserved.
% Licenced as described in the file LICENSE under the root directory of this GIT repository.
%
\index{vector product}
The product of two colinear vectors is a scalar, and the product of two normal vectors is a bivector.
The product of two general vectors is a multivector with structure to be determined.
A powerful way to examine this structure is to compute the product of two vectors in a polar representation with respect to the plane that they span.
Let \( \ucap \) and \( \vcap \) be an orthonormal pair of vectors in the plane of \( \Ba \) and \( \Bb \), oriented in a positive rotational sense as illustrated in
\cref{fig:Parallelogram:ParallelogramFig1}.
\imageFigure{../figures/GAelectrodynamics/ParallelogramFig1}{Two vectors in a plane.}{fig:Parallelogram:ParallelogramFig1}{0.3}

With respect to the orthonormal vectors \( \ucap \) and \( \vcap \), a
a polar representation of \( \Ba, \Bb \) is

\begin{dmath}\label{eqn:SimpleProducts2:1660}
\begin{aligned}
\Ba &= \Norm{\Ba} \ucap e^{ i_{ab} \theta_a } = \Norm{\Ba} e^{ -i_{ab} \theta_a } \ucap \\
\Bb &= \Norm{\Bb} \ucap e^{ i_{ab} \theta_b } = \Norm{\Bb} e^{ -i_{ab} \theta_b } \ucap,
\end{aligned}
\end{dmath}

where \( i_{ab} = \ucap \vcap \) is a unit pseudoscalar for the planar subspace spanned by \( \Ba \) and \( \Bb \).
The vector product of these two vectors is

\begin{dmath}\label{eqn:SimpleProducts2:1680}
\Ba \Bb
=
\lr{ \Norm{\Ba} e^{ -i_{ab} \theta_a } \ucap } \lr{ \Norm{\Bb} \ucap e^{ i_{ab} \theta_b } }
=
 \Norm{\Ba} \Norm{\Bb}
e^{ -i_{ab} \theta_a } ( \ucap \ucap ) e^{ i_{ab} \theta_b }
=
 \Norm{\Ba} \Norm{\Bb}
e^{ i_{ab} (\theta_b - \theta_a)}.
=
 \Norm{\Ba} \Norm{\Bb}
\lr{
\cos
(\theta_b - \theta_a)
+ i_{ab}
\sin
(\theta_b - \theta_a)
}.
\end{dmath}

We see that the product of two vectors is a multivector that has only grades 0 and 2.
This can be expressed symbolically as

\begin{dmath}\label{eqn:products:1800}
\Ba \Bb
=
\gpgradezero{ \Ba \Bb }
+
\gpgradetwo{ \Ba \Bb }.
\end{dmath}

We recognize the scalar grade of the vector product as the \R{N} dot product, but the grade 2 component of the vector product is something new that requires a name.
We respectively identify and define operators for these vector grade selection operations

\index{wedge product}
\index{dot product}
\makedefinition{Dot and wedge products of two vectors.}{dfn:products:dotandwedge}{
Given two vectors \( \Ba, \Bb \in \bbR^{N} \) the dot product is identified as the scalar grade of their product
\begin{equation*}
\gpgradezero{ \Ba \Bb }
=
\Ba \cdot \Bb
.
\end{equation*}

A wedge product of the vectors is defined as a grade-2 selection operation
\begin{equation*}
\Ba \wedge \Bb \equiv \gpgradetwo{ \Ba \Bb }.
\end{equation*}

Given this notation, the product of two vectors can be written
\begin{equation*}
\Ba \Bb = \Ba \cdot \Bb + \Ba \wedge \Bb.
\end{equation*}
} % definition

\index{grade selection}
Scalar grade selection of a product of two vectors is an important new tool.
There will be many circumstances where the easiest way to compute a dot product is using scalar grade selection.

The split of a vector product into dot and wedge product components is also important.
However, to utilize it, the properties of the wedge product have to be determined.
We also want to determine exactly how the wedge product is related to the cross product, as they clearly have a similar structure.

Summarizing \cref{eqn:SimpleProducts2:1680} with our new operators, we write

\boxedEquation{eqn:SimpleProducts2:1700}{
\begin{aligned}
\Ba \Bb &= \Norm{\Ba} \Norm{\Bb} \exp\lr{ i_{ab} (\theta_b - \theta_a) } \\
\Ba \cdot \Bb &= \Norm{\Ba} \Norm{\Bb} \cos( \theta_b - \theta_a ) \\
\Ba \wedge \Bb &= i_{ab} \Norm{\Ba} \Norm{\Bb} \sin( \theta_b - \theta_a ),
\end{aligned}
}

\index{polar representation}
Two wedge product properties can be immediately deduced from this polar representation

\begin{itemize}
\item \( \Bb \wedge \Ba = - \Ba \wedge \Bb \).
\item \( \Ba \wedge (\alpha \Ba) = 0, \quad \forall \alpha \in \bbR \).
\end{itemize}

The cross product is also bilinear, so one can reasonably expect this of the wedge product.
This is much easier to demonstrate using a coordinate expansion.
To do so let

\begin{dmath}\label{eqn:SimpleProducts2:1160}
\begin{aligned}
\Ba &= \sum_i a_i \Be_i \\
\Bb &= \sum_i b_i \Be_i.
\end{aligned}
\end{dmath}

The product of these vectors is

\begin{dmath}\label{eqn:SimpleProducts2:1360}
\Ba \Bb
=
\lr{ \sum_i a_i \Be_i } \lr{ \sum_j b_j \Be_j }
=
\sum_{ij} a_i b_j \Be_i \Be_j
=
\sum_{i = j} a_i b_j \Be_i \Be_j
+
\sum_{i \ne j} a_i b_j \Be_i \Be_j
\end{dmath}

Since \( \Be_i \Be_i = 1 \), we see again that the scalar component of the product is the dot product \( \sum_i a_i b_i \).
The remaining grade 2 components are the wedge product, for which the coordinate expansion can be simplified further

\begin{dmath}\label{eqn:SimpleProducts2:1460}
\Ba \wedge \Bb
=
\sum_{i \ne j} a_i b_j \Be_i \Be_j
=
\sum_{i < j} a_i b_j \Be_i \Be_j
+
\sum_{j < i} a_i b_j \Be_i \Be_j
=
\sum_{i < j} a_i b_j \Be_i \Be_j
+
\sum_{i < j} a_j b_i \Be_j \Be_i
%=
%\sum_{i < j} a_i b_j \Be_i \Be_j
%+
%\sum_{i < j} a_j b_i (-\Be_i \Be_j)
=
\sum_{i < j} (a_i b_j - a_j b_i) \Be_i \Be_j.
\end{dmath}

\index{determinant!wedge product}
The scalar factors can be written as \( 2 x 2 \) determinants

\boxedEquation{eqn:SimpleProducts2:1320}{
\Ba \wedge \Bb
=
\sum_{i < j}
\begin{vmatrix}
a_i & a_j \\
b_i & b_j
\end{vmatrix}
\Be_i \Be_j.
}

It is now straightforward to show that the wedge product is distributive and bilinear (\cref{problem:products:bilinear}).
It is also simple to use \cref{eqn:SimpleProducts2:1320} to show that \( \Bb \wedge \Ba = -\Ba \wedge \Bb \) and \( \Ba \wedge \Ba = 0 \).

For \R{2} there is only one term in \cref{eqn:SimpleProducts2:1320}

\begin{dmath}\label{eqn:SimpleProducts2:1720}
\Ba \wedge \Bb
=
\begin{vmatrix}
a_1 & a_2 \\
b_1 & b_2
\end{vmatrix}
\Be_1 \Be_2.
\end{dmath}

\index{cross product}
We are used to writing the cross product as a \( 3 x 3 \) determinant, which can also be done with the coordinate expansion of the
\R{3} wedge product

\begin{equation}\label{eqn:SimpleProducts2:1740}
\Ba \wedge \Bb
=
\sum_{ ij \in \setlr{ 12, 13, 23 } }
\begin{vmatrix}
a_i & a_j \\
b_i & b_j
\end{vmatrix}
\Be_i \Be_j
=
\begin{vmatrix}
\Be_2 \Be_3 & \Be_3 \Be_1 & \Be_1 \Be_2 \\
a_1 & a_2 & a_3 \\
b_1 & b_2 & b_3 \\
\end{vmatrix}.
\end{equation}

Let's summarize the wedge product properties and relations we have found so far, comparing the \R{3} wedge product to the cross product

\begin{tcolorbox}[tab2,tabularx={X||Y|Y},title=Cross product and \R{3} wedge product comparison.,boxrule=0.5pt]
Property & Cross product & Wedge product
\\ \hline
Same vectors & \( \Ba \cross \Ba = 0 \) & \( \Ba \wedge \Ba = 0 \)
\\ \hline
Antisymmetry & \( \Bb \cross \Ba = -\Ba \cross \Bb \) & \( \Bb \wedge \Ba = -\Ba \wedge \Bb \)
\\ \hline
Linear & \( \Ba \cross (\alpha \Bb) = \alpha (\Ba \cross \Bb) \) &
\( \Ba \wedge (\alpha \Bb) = \alpha (\Ba \wedge \Bb) \)
\\ \hline
Distributive
& \( \Ba \cross (\Bb + \Bc) = \Ba \cross \Bb + \Ba \cross \Bc \)
& \( \Ba \wedge (\Bb + \Bc) = \Ba \wedge \Bb + \Ba \wedge \Bc \)
\\ \hline
Determinant expansion
&
\(
\Ba \cross \Bb
=
\begin{vmatrix}
\Be_1 & \Be_2 & \Be_3 \\
a_1 & a_2 & a_3 \\
b_1 & b_2 & b_3 \\
\end{vmatrix}
\)
&
\(
\Ba \wedge \Bb
=
\begin{vmatrix}
\Be_2 \Be_3 & \Be_3 \Be_1 & \Be_1 \Be_2 \\
a_1 & a_2 & a_3 \\
b_1 & b_2 & b_3 \\
\end{vmatrix}
\)
\\ \hline
Polar form &
\( \ncap_{ab} \Norm{\Ba} \Norm{\Bb} \sin( \theta_b - \theta_a )  \) &
\( i_{ab} \Norm{\Ba} \Norm{\Bb} \sin( \theta_b - \theta_a )  \)
\\ \hline
\end{tcolorbox}

All the wedge properties except the determinant expansion above are valid in any dimension.
It is reasonable to guess that the \R{3} wedge product is related to the cross product by some constant multivector factor \( i_{ab} = A \ncap_{ab} \).
In coordinate form, this requires a simultaneous solution to

\begin{dmath}\label{eqn:SimpleProducts2:1580}
\begin{aligned}
\Be_2 \Be_3 &= A \Be_1 \\
\Be_3 \Be_1 &= A \Be_2 \\
\Be_1 \Be_2 &= A \Be_3.
\end{aligned}
\end{dmath}

Multiplying on the right by \( \Be_1, \Be_2, \Be_3 \) respectively, this factor seems to be

\begin{equation}\label{eqn:SimpleProducts2:1600}
A = \Be_2 \Be_3 \Be_1 = \Be_3 \Be_1 \Be_2 = \Be_1 \Be_2 \Be_3,
\end{equation}

which are all permutations of the \R{3} unit pseudoscalar \( I = \Be_1 \Be_2 \Be_3 \).
This indicates that the cyclic permutations of the \R{3} pseudoscalar must all be identical (\cref{problem:SimpleProducts2:permutationspseudoscalar}).

We now have a coordinate free relationship for the \R{3} wedge product and the cross product

\boxedEquation{eqn:SimpleProducts2:1620}{
\Ba \wedge \Bb = I ( \Ba \cross \Bb ),
}

and can also express the
\R{3} vector product as a multivector combination of the dot and cross products

\boxedEquation{eqn:SimpleProducts2:1640}{
\Ba \Bb = \Ba \cdot \Bb + I(\Ba \cross \Bb).
}

This is a very important relationship.

In particular, for electromagnetism, \cref{eqn:SimpleProducts2:1640} can
be used to combine (the scalar) Guass's law with (the vector) Maxwell-Faraday equation, and to
combine (the scalar) Gauss's law for magnetism with (the vector) Ampere-Maxwell equation.
Such dot plus cross product sums will yield
a pair of multivector equations that can be further merged.  The resulting multivector equation will be
called
Maxwell equation (singular), and will be the starting point of all our electromagnetic analysis.

\index{colinear vectors!wedge}
\makeproblem{Wedge product of colinear vectors.}{problem:SimpleProducts2:wedgecolinear}{
Given \( \Bb = \alpha \Ba \), use
\cref{eqn:SimpleProducts2:1320} to show that the wedge product of any pair of colinear vectors is zero.
} % problem

\makeproblem{Wedge product antisymmetry.}{problem:SimpleProducts2:1560}{
Prove that the wedge product is antisymmetric using using \cref{eqn:SimpleProducts2:1320}.
} % problem

\makeproblem{Permutations of the \R{3} pseudoscalar}{problem:SimpleProducts2:permutationspseudoscalar}{
Show that each of the permutations of
\cref{eqn:SimpleProducts2:1600} are all equal.
} % problem

\makeproblem{Wedge product distributivity and linearity.}{problem:products:bilinear}{
For vectors \( \Ba, \Bb, \Bc \) and \( \Bd \), and scalars \( \alpha, \beta \) use
\cref{eqn:SimpleProducts2:1320} to show that

\makesubproblem{}{problem:products:bilinear:a}

the wedge product is distributive
\begin{equation*}
(\Ba + \Bb) \wedge (\Bc + \Bd) =
\Ba \wedge \Bc
+
\Ba \wedge \Bd
+
\Bb \wedge \Bc
+
\Bb \wedge \Bd.
\end{equation*}

\makesubproblem{}{problem:products:bilinear:b}

and show that the wedge product is bilinear
\begin{equation*}
(\alpha \Ba) \wedge (\beta \Bb)
=
(\alpha \beta) (\Ba \wedge \Bb).
\end{equation*}

Note that these imply the wedge product also has the cross product filtering property \( \Ba \wedge (\Bb + \alpha \Ba) = \Ba \wedge \Bb \).
} % problem

%Answer (partial)
%Swapping \( \Ba \) and \( \Bb \),
%
%\begin{dmath}\label{eqn:products:1820}
%\Bb \wedge \Ba
%=
%\sum_{i < j}
%\begin{vmatrix}
%b_i & b_j \\
%a_i & a_j
%\end{vmatrix}
%\Be_i \Be_j
%=
%-\sum_{i < j}
%\begin{vmatrix}
%a_i & a_j \\
%b_i & b_j \\
%\end{vmatrix}
%\Be_i \Be_j
%=
%-\Ba \wedge \Bb,
%\end{dmath}
%
%proves that
%\cref{eqn:products:1780} holds for any vectors, and not just when they are normal.
%Because of this antisymmetry we also have
%
%\begin{dmath}\label{eqn:products:1840}
%\Ba \wedge \Ba = 0.
%\end{dmath}
%
