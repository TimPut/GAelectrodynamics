%
% Copyright © 2018 Peeter Joot.  All Rights Reserved.
% Licenced as described in the file LICENSE under the root directory of this GIT repository.
%
%{

Given curvilinear coordinates defined on a subspace \cref{dfn:curvilinearThree:280}, we don't have enough parameters to define the gradient.  For calculus on the k-dimensional subspace, we define the vector derivative

\index{vector derivative}
\makedefinition{Vector derivative}{thm:gradient:100}{
Given an k-parameter vector parameterization
\( \Bx = \Bx(u_1, u_2, \cdots, u_k) \) of \R{N} with \( k \le N \),
and curvilinear basis elements \( \Bx_i = \PDi{u_i}{\Bx} \), the vector derivative \( \boldpartial \) is defined as
\begin{dmath*}
\boldpartial = \sum_{i=1}^k \Bx^i \partial_i.
\end{dmath*}
} % theorem

When the dimension of the subspace (number of parameters) equals the dimension of the underlying vector space, the vector derivative equals the gradient.  Otherwise we can write

\begin{dmath}\label{eqn:vectorDerivative:101}
\spacegrad = \boldpartial + \spacegrad_\perp,
\end{dmath}

and can think of the vector derivative as the projection of the gradient onto the tangent space at the point of evaluation.

Please see \citep{aMacdonaldVAGC} for an excellent introduction of the reciprocal frame, the gradient, and the vector derivative, and for
details about the connectivity of the manifold ignored here.

%}
