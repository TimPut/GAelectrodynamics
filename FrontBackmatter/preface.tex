%
% Copyright � 2016 Peeter Joot.  All Rights Reserved.
% Licenced as described in the file LICENSE under the root directory of this GIT repository.
%

%
%\chapter{Preface}
% this suppresses an explicit chapter number for the preface.
\chapter*{Preface}%\normalsize
  \addcontentsline{toc}{chapter}{Preface}

A book on Geometric Algebra applications to electromagnetism.

The target audience are undergraduate student with sufficient background in electromagnetism that knowledge of Maxwell's equations can be presumed.
Alternatives to the usual 3D vector or coordinate based derivations of various problems will be presented that highlight some of the ways that Geometric Algebra can be used to tackle electromagnetic problems.
%the students that come from physics but are still undergraduate. I can possibly introduce some material also to antenna students (just to derive the classical Hertz dipole and magnetic dipole radiation in a (possibly) simpler way).

%\withproblemsetsMessage{
%\textcolor{Maroon}{
%\textit{THIS DOCUMENT IS REDACTED.  THE PROBLEM SET SOLUTIONS AND ASSOCIATED MATHEMATICA CODE IS NOT VISIBLE.  PLEASE EMAIL ME FOR THE FULL VERSION IF YOU ARE NOT TAKING THIS COURSE.}
%}
%}

\paragraph{These notes contain:}

\begin{itemize}
\item An introduction to Geometric Algebra (GA).
\item Application of Geometric Algebra to electromagnetism, with a focus on engineering applications.
\end{itemize}

There are two potential audiences for these notes.  The first is for a student new to GA faced with the learning curve of both GA itself and the notational changes needed to apply it to electromagnetism.  The second audience is the individual who already has some knowledge of GA, and may want to skim yet another boilerplate ``Introduction to Geometric Algebra'' to get an idea of the notation in use, and move on to the electromagnetism applications.  To serve both potential audiences, much of the substance of the introductory GA material has been deferred to the problems.  Students new to GA should attempt all these problems before falling back to just reading the solutions.

Peeter Joot \quad peeterjoot@protonmail.com

Prof. Mauro Mongiardo \quad mauro.mongiardo@gmail.com
