%
% Copyright © 2017 Peeter Joot.  All Rights Reserved.
% Licenced as described in the file LICENSE under the root directory of this GIT repository.
%

We've found solutions for a number of static charge and current distributions.

\begin{enumerate}[(a)]
\item For constant electric sources along the z-axis
(\cref{eqn:statics_infiniteLineCharge:180})
, with current \( \BJ \) moving with velocity \( \Bv = v \Be_3 \), the field had the form \( F = E \rhocap \lr{ 1 - \Bv/c } \).
\item For constant magnetic sources along the z-axis
(\cref{problem:statics:240})
, with current \( \BM \) moving with velocity \( \Bv = v \Be_3 \), the field had the form \( F = \eta H I \rhocap \lr{ 1 - \Bv/c } \).
\item For constant electric sources in the x-y plane
(\cref{eqn:statics_infinitePlane:320})
, with current \( \BJ \) moving with velocity \( \Bv = v \Be_1 e^{i\theta}, i = \Be_{12} \), the field had the form \( F = E \Be_3 \lr{ 1 - \Bv/c } \).
\item For constant magnetic sources in the x-y plane
(\cref{problem:statics:241})
, with current \( \BM \) moving with velocity \( \Bv = v \Be_1 e^{i\theta}, i = \Be_{12} \), the field had the form \( F = \eta H i \lr{ 1 - \Bv/c } \).
\end{enumerate}

In all cases the field has the form \( F = A ( 1 - \Bv/c ) \), where \( A \) is either a vector or a bivector that anticommutes with the current velocity \( \Bv \), so the energy momentum tensor \( T(1) \) has the form
\begin{dmath}\label{eqn:poyntingStaticExamples:860}
T(1)
= \frac{\epsilon}{2} A ( 1 - \Bv/c )^2 A^\dagger
= \frac{\epsilon}{2} A A^\dagger ( 1 + \Bv/c )^2
= \frac{\epsilon}{2} A A^\dagger \lr{ 1 + \lr{ \frac{\Bv}{c} }^2 + 2 \frac{\Bv}{c} }.
\end{dmath}

For the electric sources this is
\begin{dmath}\label{eqn:poyntingStaticExamples:880}
\calE + \frac{\BS}{c} = \frac{\epsilon}{2} E^2 \lr{ 1 + \lr{ \frac{\Bv}{c} }^2 + 2 \frac{\Bv}{c} },
\end{dmath}
or
\begin{dmath}\label{eqn:poyntingStaticExamples:900}
\begin{aligned}
\calE &= \frac{\epsilon}{2} E^2 \lr{ 1 + \lr{ \frac{\Bv}{c} }^2 } \\
\BS &= \epsilon E^2 \Bv.
\end{aligned}
\end{dmath}

For the magnetic sources this is
\begin{dmath}\label{eqn:poyntingStaticExamples:920}
\calE + \frac{\BS}{c} = \frac{\mu}{2} H^2 \lr{ 1 + \lr{ \frac{\Bv}{c} }^2 + 2 \frac{\Bv}{c} },
\end{dmath}
or
\begin{dmath}\label{eqn:poyntingStaticExamples:940}
\begin{aligned}
\calE &= \frac{\mu}{2} H^2 \lr{ 1 + \lr{ \frac{\Bv}{c} }^2 } \\
\BS &= \mu H^2 \Bv.
\end{aligned}
\end{dmath}

There are three terms in the multivector \( (1 -\Bv/c)^2 = 1 + \lr{ \ifrac{\Bv}{c} }^2 + 2 \ifrac{\Bv}{c} \).  For electric sources,
the first scalar term is due to the charge distribution, and provides the electric field contribution to the energy density.
The second scalar term is due to the current distribution, and provides the magnetic field contribution to the energy density.
The final vector term, proportional to the current velocity contributes to the Poynting vector, showing that the field momentum travels along the direction of the current in these static configurations.

Calculation of the \( T(\Be_k) \) tensor components is generally more involved.
Let's do this calculation for each of the fields above in turn to illustrate.

\paragraph{(a):}

To calculate \( T(\Be_3) \) we can reduce the following products
\begin{dmath}\label{eqn:poyntingStaticExamples:960}
F \Be_3 F^\dagger
=
E^2 \rhocap \lr{ 1 - \Bv/c} \Be_3 \lr{1 - \Bv/c} \rhocap
=
-E^2 \Be_3 \rhocap \lr{1 - \Bv/c}^2 \rhocap
=
-E^2 \Be_3 \rhocap \lr{1 + \Bv^2/c^2 - 2 \Bv/c} \rhocap
=
-E^2 \Be_3 \rhocap^2 \lr{1 + \Bv^2/c^2 + 2 \Bv/c}
=
-E^2 \Be_3 \lr{1 + \Bv^2/c^2 + 2 \Bv/c}.
\end{dmath}

Since
\begin{dmath}\label{eqn:poyntingStaticExamples:980}
T(\Be_k)
= - \frac{\BS}{c} \cdot \Be_k + \BT(\Be_k).
\end{dmath}

This means that \( \BS \cdot \Be_3 = \epsilon E^2 v \), as already found.  The vector component of this tensor element is
\begin{dmath}\label{eqn:poyntingStaticExamples:1000}
\BT(\Be_3) =
- \frac{\epsilon}{2} E^2 \Be_3 \lr{1 + \Bv^2/c^2 }.
\end{dmath}

This component of the stress tensor is aligned along the same axis as the velocity.
Calculation of the other stress tensor components is easiest in cylindrical coordinates.  Along the radial direction
\begin{dmath}\label{eqn:poyntingStaticExamples:1020}
\begin{aligned}
\rhocap \lr{ 1 - \Bv/c } \rhocap \lr{1 - \Bv/c} \rhocap
&=
\rhocap^2 \lr{ 1 + \Bv/c} \lr{1 - \Bv/c} \rhocap \\
&= \lr{1 - \Bv^2/c^2} \rhocap,
\end{aligned}
\end{dmath}
and along the azimuthal direction
\begin{dmath}\label{eqn:poyntingStaticExamples:1100}
\begin{aligned}
\rhocap \lr{ 1 - \Bv/c } \thetacap \lr{1 - \Bv/c} \rhocap
&=
\rhocap \thetacap \lr{1 + \Bv/c} \lr{1 - \Bv/c} \rhocap \\
&=
-\thetacap \rhocap^2 \lr{1 - \Bv^2/c^2} \\
&=
-\thetacap \lr{1 - \Bv^2/c^2}.
\end{aligned}
\end{dmath}

Since \( T(\Ba) \) is a linear operator for any vector parameters \( \Ba \), it cannot have any grade zero component along any directions \( \Be \cdot \Be_3 = 0 \).
No grade zero component of \( T(\Be_1), T(\Be_2) \) implies that the Poynting vector is zero along the \( \Be_1 \) and \( \Be_2 \) directions respectively, as we saw above in
\cref{eqn:poyntingStaticExamples:900}.
%\begin{dmath}\label{eqn:poyntingStaticExamples:1120}
%\rhocap \rhocap \rhocap = \rhocap
%\rhocap \thetacap \rhocap = -\thetacap
%\end{dmath}
%
%The reflections above can be computed explicitly
%
%\begin{dmath}\label{eqn:poyntingStaticExamples:1040}
%\rhocap \Be_1 \rhocap
%= \Be_1 e^{i \theta} \Be_1 \Be_1 e^{i \theta}
%= \Be_1 e^{2 i \theta},
%\end{dmath}
%
%and
%\begin{dmath}\label{eqn:poyntingStaticExamples:1060}
%\rhocap \Be_2 \rhocap
%= \Be_1 e^{i \theta} \Be_2 \Be_1 e^{i \theta}
%= \Be_{121} e^{2 i \theta}
%= -\Be_{2} e^{2 i \theta}.
%\end{dmath}

In summary
\begin{dmath}\label{eqn:poyntingStaticExamples:1080}
\begin{aligned}
\BT(\rhocap) &= \frac{\epsilon}{2} E^2 \lr{1 -\Bv^2/c^2} \rhocap \\
\BT(\thetacap) &= -\frac{\epsilon}{2} E^2 \lr{1 -\Bv^2/c^2} \thetacap \\
%\BT(\Be_1) &= \frac{\epsilon}{2} E^2 (1 -\Bv^2/c^2) \Be_1 e^{2 i \theta} \\
%\BT(\Be_2) &= -\frac{\epsilon}{2} E^2 (1 -\Bv^2/c^2) \Be_2 e^{2 i \theta} \\
\BT(\Be_3) &= -\frac{\epsilon}{2} E^2 \lr{1 + \Bv^2/c^2 } \Be_3.
\end{aligned}
\end{dmath}

For this field that \( \BT(\rhocap) \) is entirely radial, whereas \( \BT(\thetacap) \) is entirely azimuthal.

In terms of an arbitrary vector in cylindrical coordinates
\begin{dmath}\label{eqn:poyntingStaticExamples:1200}
\Ba = a_\rho \rhocap + a_\theta \thetacap + a_z \Be_3,
\end{dmath}
the grade one component of the tensor is
\begin{dmath}\label{eqn:poyntingStaticExamples:1140}
\BT(\Ba) =
\frac{\epsilon}{2} E^2 \lr{1 -\Bv^2/c^2} \lr{ a_\rho \rhocap - a_\theta \thetacap }
-\frac{\epsilon}{2} E^2 \lr{1 +\Bv^2/c^2} a_z \Be_3.
\end{dmath}

\paragraph{(b):}

For \( F = \eta H I \rhocap \lr{ 1 - \Bv/c } \), and \( \Bv = v \Be_3 \) we have
\begin{dmath}\label{eqn:poyntingStaticExamples:1160}
F \Ba F^\dagger
=
\eta^2 H^2 I \rhocap \lr{ 1 - \Bv/c } \Ba \lr{ 1 - \Bv/c } \rhocap (-I)
=
\eta^2 H^2 \rhocap \lr{ 1 - \Bv/c } \Ba \lr{ 1 - \Bv/c } \rhocap.
\end{dmath}

We can write the tensor components immediately, since
\cref{eqn:poyntingStaticExamples:1160}
has exactly the same structure as the tensor components computed in part (a) above.  That is
\begin{dmath}\label{eqn:poyntingStaticExamples:1180}
\BT(\Ba) =
\frac{\mu}{2} H^2 \lr{1 -\Bv^2/c^2} \lr{ a_\rho \rhocap - a_\theta \thetacap }
-\frac{\mu}{2} H^2 \lr{1 +\Bv^2/c^2} a_z \Be_3.
\end{dmath}

\paragraph{(c):}

For \( F = E \Be_3 \lr{ 1 - \Bv/c } \), and \( \Bv = v \rhocap \), we have
\begin{dmath}\label{eqn:poyntingStaticExamples:1220}
F \Ba F^\dagger
=
E^2 \Be_3 \lr{ 1 - (v/c) \rhocap } \Ba \lr{ 1 - (v/c) \rhocap } \Be_3,
\end{dmath}
so we need the following grade selections
\begin{equation}\label{eqn:poyntingStaticExamples:1260}
\begin{aligned}
\gpgradeone{ \Be_3 \lr{ 1 - (v/c) \rhocap } \rhocap \lr{ 1 - (v/c) \rhocap } \Be_3 }
&=
\gpgradeone{ \Be_3 \rhocap \lr{ 1 - (v/c) \rhocap }^2 \Be_3 } \\
&=
\gpgradeone{ \Be_3 \rhocap \lr{ 1 + (v^2/c^2) - 2 (v/c) \rhocap } \Be_3 } \\
&=
\lr{ 1 + (v^2/c^2) } \Be_3 \rhocap \Be_3 \\
&=
-\lr{ 1 + (v^2/c^2) } \rhocap \\
\gpgradeone{ \Be_3 \lr{ 1 - (v/c) \rhocap } \thetacap \lr{ 1 - (v/c) \rhocap } \Be_3 }
&=
\gpgradeone{ \Be_3 \thetacap \lr{ 1 + (v/c) \rhocap } \lr{ 1 - (v/c) \rhocap } \Be_3 } \\
&=
\gpgradeone{ \Be_3 \thetacap \lr{ 1 - (v^2/c^2) } \Be_3 } \\
&=
-\lr{ 1 - (v^2/c^2) } \thetacap \\
\gpgradeone{ \Be_3 \lr{ 1 - (v/c) \rhocap } \Be_3 \lr{ 1 - (v/c) \rhocap } \Be_3 }
&=
\gpgradeone{ \lr{ 1 + (v/c) \rhocap } \lr{ 1 - (v/c) \rhocap } \Be_3 } \\
&=
\lr{ 1 - (v^2/c^2) } \Be_3.
\end{aligned}
\end{equation}

So the Maxwell stress tensor components of interest are
\begin{dmath}\label{eqn:poyntingStaticExamples:1320}
\BT(\Ba)
=
-\frac{\epsilon}{2} E^2 \lr{ 1 + (\Bv^2/c^2) } a_\rho \rhocap
+
\frac{\epsilon}{2} E^2 \lr{ 1 - (\Bv^2/c^2) } \lr{ a_z \Be_3 - a_\theta \thetacap }.
\end{dmath}

\paragraph{(d):}

For \( F = \eta H i \lr{ 1 - \Bv/c }, i = \Be_{12} \), and \( \Bv = v \rhocap \), we can use a duality transformation for the unit bivector \( i \)
\begin{dmath}\label{eqn:poyntingStaticExamples:1340}
F = \eta H I \Be_3 \lr{ 1 - \Bv/c },
\end{dmath}
so
\begin{dmath}\label{eqn:poyntingStaticExamples:1360}
F \Ba F^\dagger = \eta^2 H^2 \Be_3 \lr{ 1 - \Bv/c } \Ba \lr{ 1 - \Bv/c } \Be_3.
\end{dmath}

\Cref{eqn:poyntingStaticExamples:1360} has the structure found in part (c) above, so
\begin{dmath}\label{eqn:poyntingStaticExamples:1380}
\BT(\Ba)
=
-\frac{\mu}{2} H^2 \lr{ 1 + (\Bv^2/c^2) } a_\rho \rhocap
+
\frac{\mu}{2} H^2 \lr{ 1 - (\Bv^2/c^2) } \lr{ a_z \Be_3 - a_\theta \thetacap }.
\end{dmath}
