%
% Copyright © 2017 Peeter Joot.  All Rights Reserved.
% Licenced as described in the file LICENSE under the root directory of this GIT repository.
%
%{
\index{complex imaginary}
\index{pseudoscalar}
We've seen that bivectors like \( \Be_{12} \) square to minus one.  Geometric algebra has infinitely many such imaginary numbers, which can be utilized to introduce problem specific ``complex planes'' as desired.  In three dimensonal and higher spaces, imaginary representations
(such as the \R{3} pseudoscalar) with grades higher than two are also possible.

Using the reversion relationship of \cref{eqn:reverse:103}, we can see that the \( I \) behaves as an imaginary

\begin{dmath}\label{eqn:R3PseudoscalarSquare:10}
I^2
=
I (-I^\dagger)
=
-
(\Be_1 \Be_2 \Be_3)(\Be_3 \Be_2 \Be_1)
=
-
\Be_1 \Be_2 \Be_2 \Be_1
=
-
\Be_1 \Be_1
=
-1.
\end{dmath}

Given many possible imaginary representations, complex and complex-like numbers can be represented in GA for any k-vector \( i \) that satisfies \( i^2 = -1 \) since the multivector

\begin{dmath}\label{eqn:complex:260}
z = x + i y,
\end{dmath}

will have all the required properties of a complex number.

For example, in Euclidean spaces we could use either of

\begin{dmath}\label{eqn:complex:280}
\begin{aligned}
i &= \frac{\Bu \wedge \Bv}{\sqrt{-\lr{\Bu\wedge\Bv}^2}} \\
i &= \frac{\Bu \wedge \Bv \wedge \Bw}{\sqrt{-\lr{\Bu\wedge\Bv\wedge \Bw}^2}},
\end{aligned}
\end{dmath}

provided \( \Bu, \Bv (,\Bw) \) are linearly independent.  Given a set of orthonormal vectors \( \Bu, \Bv (,\Bw) \), then

\begin{dmath}\label{eqn:complex:300}
\begin{aligned}
i &= \Bu \Bv \\
i &= \Bu \Bv \Bw,
\end{aligned}
\end{dmath}

are also suitable as imaginaries.

Other complex number like representations are also possible with GA.  Quaternions, which are often used in computer graphics to represent rotations,
are the set \( q \in \setlr{ a + x \Bi + y \Bj + z \Bk \mid a, x, y, z \in \bbR} \) where
\begin{equation}\label{eqn:2dMultiplication:240}
\begin{aligned}
\Bi^2 &= \Bj^2 = \Bk^2 = -1 \\
\Bi \Bj &= \Bk = -\Bj \Bi \\
\Bj \Bk &= \Bi = -\Bk \Bj \\
\Bk \Bi &= \Bj = -\Bi \Bk.
\end{aligned}
\end{equation}

Like complex numbers, quaternions can be represented in GA as 0,2 multivectors, but require three imaginaries instead of one.

Other complex like representations are also possible in GA, provided suitable conjugation operations are defined.  For example,
an operation called Clifford conjugation (or spatial reversal) designated \( \overbar{A} \) is introduced in \citep{baylis2004electrodynamics}
that toggles the sign of any multivector components with grade \( g \mod 4 = 1,2 \).  Illustrating by example, given a multivector
\( A = 1 + \Be_1 + \Be_{12} + \Be_{123} \), the Clifford conjugate is

\begin{dmath}\label{eqn:galiterature:60}
\overbar{A} = 1 - \Be_1 - \Be_{12} + \Be_{123},
\end{dmath}

leaving the sign of the scalar and pseudoscalar components untouched.  Such a complex conjugation like operation allows (0,1), (0,2), (1,3) or (2,3) multivectors to encode relativistic ``proper length'' using geometric algebras built from real Euclidean vector spaces.

\makeproblem{Quaternions.}{problem:2dMultiplication:quaternions}{
Show that the relations \cref{eqn:2dMultiplication:240} are satisfied by the unit bivectors \( \Bi = \Be_{32}, \Bj = \Be_{13}, \Bk = \Be_{21} \), demonstrating that quaternions, like complex numbers, may be represented as multivector subspaces.
} % problem
%}
