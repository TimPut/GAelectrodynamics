%
% Copyright � CCYY Peeter Joot.  All Rights Reserved.
% Licenced as described in the file LICENSE under the root directory of this GIT repository.
%
\makeproblem{Complex Poynting vector}{problem:complexPower:1}{
Show that the
} % problem

\makeanswer{problem:complexPower:1}{

\begin{dmath}\label{eqn:complexPower:n}
\bcS
= -\bcE \cdot (I \bcH)
= -\Real \lr{ \BE e^{j \omega t} } \cdot (I \Real \lr{ \BH e^{j \omega t}} )
= -\inv{4} \gpgradeone{ \lr{ \BE e^{j \omega t} + \BE^\conj e^{-j \omega t} } I \lr{
\BH e^{j \omega t}
+ \BH^\conj e^{-j \omega t}
}
}
=
-\inv{4}
\lr{
\BE \cdot (I \BH^\conj)
+
\BE^\conj \cdot (I \BH)
+
\BE \cdot (I \BH) e^{ 2 j \omega t }
+
\BE^\conj \cdot (I \BH^\conj) e^{ -2 j \omega t }
}.
\end{dmath}

This shows that the time harmonic representation of the Poynting vector is

\begin{dmath}\label{eqn:complexPower:n}
\bcS = -\inv{2} \Real \lr{ \BE \cdot (I \BH^\conj) + \BE \cdot (I \BH) e^{ 2 j \omega t } }.
\end{dmath}

Defining the complex Poynting vector 

\begin{dmath}\label{eqn:complexPower:n}
\BS = -\inv{2} \BE \cdot (I \BH^\conj),
\end{dmath}

In particular, the average of the Poynting vector over a single period, is just the real part of this complex Poynting vector

\begin{dmath}\label{eqn:complexPower:n}
\inv{T} \int_0^T dt \bcS = \Real \BS.
\end{dmath}

} % answer
