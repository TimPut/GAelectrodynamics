%
% Copyright © 2017 Peeter Joot.  All Rights Reserved.
% Licenced as described in the file LICENSE under the root directory of this GIT repository.
%
\subsection{Field energy and momentum density and the stress energy tensor.}

In conventional electromagnetism the energy and momentum density of the fields are

\index{energy density}
\index{momentum density}
\index{Poynting vector}
\index{energy flux}
\index{stress tensor}
\begin{dmath}\label{eqn:poyntingF:20}
\begin{aligned}
\calE &= \inv{2} \lr{ \BD \cdot \BE + \BB \cdot \BH } \\
\bcP c &= \inv{c} \BS = \inv{c} \BE \cross \BH = \inv{c} \lr{ I \BH } \cdot \BE.
\end{aligned}
\end{dmath}

where \( \calE \) is the energy density, \( \BS \) is the Poynting vector representing energy flux through a surface per unit time, and \( \bcP \) is the momentum density of the fields.

In GA formalism, the energy and momentum densities are related by grade selection from a single multivector formed from the electrodynamic field \( F \)

\boxedEquation{eqn:poyntingF:60}{
T(1) \equiv \inv{2} \epsilon F F^\dagger = \calE + \bcP c = \calE + \frac{\BS}{c},
}

or
\begin{dmath}\label{eqn:poyntingF:40}
\begin{aligned}
\calE &= \inv{2} \epsilon \gpgradezero{ F F^\dagger } \\
\bcP c &= \inv{2} \epsilon \gpgradeone{ F F^\dagger } \\
\BS &= \inv{2 \eta} \gpgradeone{ F F^\dagger }.
\end{aligned}
\end{dmath}

Expanding \( T(1) \) in terms of \( \BE, \BH \) gives

\begin{dmath}\label{eqn:poyntingF:80}
T(1)
=
\inv{2} \epsilon F F^\dagger
=
\inv{2} \epsilon \lr{ \BE + I \eta \BH } \lr{ \BE - I \eta \BH }
=
\inv{2} \epsilon \lr{ \BE^2 + \eta^2 \BH^2 }
+
\inv{2} I \epsilon \eta \lr{ \BH \BE - \BE \BH }
=
\inv{2} \lr{ \BD \cdot \BE + \BH \cdot \BB }
+
\frac{I}{c} \BH \wedge \BE
=
\inv{2} \lr{ \BD \cdot \BE + \BH \cdot \BB }
+
\frac{1}{c} \BE \cross \BH
=
\calE + \frac{\BS}{c},
\end{dmath}

as claimed in \cref{eqn:poyntingF:20} and \cref{eqn:poyntingF:60}.  \( T(1) \) has one scalar component, and three vector components, and represents four
four of the sixteen components of the complete stress energy tensor.  The geometric algebra form of the complete stress energy tensor is

\begin{dmath}\label{eqn:poyntingF:760}
T(a) = \inv{2} \epsilon F a F^\dagger,
\end{dmath}

where \( a \) is one of \( 1, \Be_1, \Be_2 \) or \( \Be_3 \).
Each \( T(a) \) is a 0,1 multivector containing one scalar component and three vector components.
An expansion of \( T(\Be_k) \) is harder to do in closed form than \cref{eqn:poyntingF:80}, but a
brute force expansion using Mathematica of all the \( T(a) \) multivectors gives:

\begin{subequations}
\label{eqn:poyntingF:800}
\begin{equation}\label{eqn:poyntingF:1020}
\begin{aligned}
T(1)
&= \frac{\epsilon}{2} \lr{E_1^2 + E_2^2 + E_3^2} + \frac{\epsilon \eta^2}{2} \lr{H_1^2 + H_2^2 + H_3^2} \\
&+ \Be_1 \eta \epsilon \lr{E_2 H_3 - E_3 H_2} \\
&+ \Be_2 \eta \epsilon \lr{E_3 H_1 - E_1 H_3} \\
&+ \Be_3 \eta\epsilon \lr{E_1 H_2 - E_2 H_1}
\end{aligned}
\end{equation}
\begin{equation}\label{eqn:poyntingF:1040}
\begin{aligned}
T(\Be_1)
&= \eta \epsilon \lr{E_3 H_2 - E_2 H_3} \\
& + \frac{1}{2} \Be_1 \epsilon \lr{E_1^2 - E_2^2 - E_3^2} + \frac{\epsilon \eta^2}{2} \lr{ H_1^2 -  H_2^2 -  H_3^2} \\
& + \Be_2 \epsilon \lr{E_1 E_2 + \eta^2 H_1 H_2} \\
& + \Be_3 \epsilon \lr{E_1 E_3 + \eta^2 H_1 H_3}
\end{aligned}
\end{equation}
\begin{equation}\label{eqn:poyntingF:1060}
\begin{aligned}
T(\Be_2)
&= \eta \epsilon \lr{E_1 H_3 - E_3 H_1} \\
& + \Be_1 \epsilon \lr{E_1 E_2 + \eta^2 H_1 H_2} \\
& + \frac{1}{2} \Be_2 \epsilon \lr{-E_1^2 + E_2^2 - E_3^2 } + \frac{\epsilon \eta^2}{2} \lr{-H_1^2 +  H_2^2 -  H_3^2} \\
& + \Be_3 \epsilon \lr{E_2 E_3 + \eta^2 H_2 H_3}
\end{aligned}
\end{equation}
\begin{equation}\label{eqn:poyntingF:1080}
\begin{aligned}
T(\Be_3)
&= \eta \epsilon \lr{E_2 H_1 - E_1 H_2} \\
& + \Be_1 \epsilon \lr{E_1 E_3 + \eta^2 H_1 H_3} \\
& + \Be_2 \epsilon \lr{E_2 E_3 + \eta^2 H_2 H_3} \\
& + \frac{1}{2} \Be_3 \epsilon \lr{-E_1^2 - E_2^2 + E_3^2 } + \frac{\epsilon \eta^2}{2} \lr{ -H_1^2 -  H_2^2 + H_3^2}
\end{aligned}
\end{equation}
\end{subequations}

The components of the multivectors \( T(a) \) that we are calling the stress energy tensor, are more conventionally written out
as a symmetric tensor \( \Theta^{ij} \) as follows
\footnote{Following conventions from \citep{jackson1975cew}, but translating to SI units.}
\begin{dmath}\label{eqn:poyntingF:840}
\begin{aligned}
\Theta^{00} &= \frac{\epsilon}{2} \lr{ \BE^2 + \eta^2 \BH^2 } \\
\Theta^{0i} &= \inv{c} \lr{ \BE \cross \BH } \cdot \Be_i \\
\Theta^{ij} &= -\epsilon \lr{ E_i E_j + \eta^2 H_i H_j - \inv{2} \delta_{ij} \lr{ \BE^2 + \eta^2 \BH^2 } }.
\end{aligned}
\end{dmath}

The symmetric stress tensor components of \cref{eqn:poyntingF:840}
are related to the multivector representation expanded in \cref{eqn:poyntingF:800} by

\begin{dmath}\label{eqn:poyntingF:820}
\begin{aligned}
&\gpgradezero{ T(1) }
%= \calE
=
{\Theta_0}^0 = \Theta^{00} \\
&\gpgradeone{ T(1) } \cdot \Be_i
%= \frac{\BS}{c} \cdot \Be_i
= {\Theta_0}^i = \Theta^{0i} \\
&\gpgradezero{ T(\Be_i) }
%= -\frac{\BS}{c} \cdot \Be_i
= {\Theta_i}^0 = -\Theta^{i0} \\
&\gpgradeone{ T(\Be_i) } \cdot \Be_j = {\Theta_i}^j = -\Theta^{ij}.
\end{aligned}
\end{dmath}

\subsection{Poynting's theorem.}

Poynting's theorem, the conservation relationship between energy and momentum density (or more generally, the stress tensor) and the sources, can be stated in terms of the multivector field \( F \) and the multivector current \( J \).
To derive this relationship we can act on (all terms of) \( F a F^\dagger \) with the space+time derivative operator \( \spacegrad + (1/c) \partial_t \).
We do so within a scalar selection operation, which simplifies things, and allows for cyclic permutation of the multivector factors (i.e. \(\gpgradezero{ABC} = \gpgradezero{CAB}\)).

\begin{dmath}\label{eqn:poyntingF:100}
\frac{\epsilon}{2} \gpgradezero{ \lr{ c \spacegrad + \PD{t}{} } F a F^\dagger }
=
\frac{\epsilon}{2} \gpgradezero{ \lr{ c \spacegrad + \PD{t}{} } \dot{F} a F^\dagger }
+
\frac{\epsilon}{2} \gpgradezero{ \lr{ c \spacegrad + \PD{t}{} } F a \dot{F}^\dagger }
=
\frac{\epsilon}{2} \gpgradezero{ c J a F^\dagger }
+
\frac{\epsilon}{2} \gpgradezero{ \dot{F}^\dagger \lr{ c \spacegrad + \PD{t}{} } F a },
\end{dmath}

where
the over-dot notation of
\citep{hestenes1999nfc} was used to indicate the desired action of the derivative operators in the
chain rule expansion of
\cref{eqn:poyntingF:100}
, as the gradient may not commute with \( F \).  Another application of Maxwell's equation reduces this further

\begin{dmath}\label{eqn:poyntingF:960}
\frac{\epsilon}{2} \gpgradezero{ \lr{ c \spacegrad + \PD{t}{} } F a F^\dagger }
=
\frac{\epsilon}{2} \gpgradezero{ c J F^\dagger a }
+
\frac{\epsilon}{2} \gpgradezero{ \lr{ \lr{ c \spacegrad + \PD{t}{} } F }^\dagger F a }
=
c \frac{\epsilon}{2} \gpgradezero{ F^\dagger J a + J^\dagger F a },
\end{dmath}

or
\boxedEquation{eqn:poyntingF:980}{
c \spacegrad \cdot \gpgradeone{ \frac{\epsilon}{2} F a F^\dagger }
+ \PD{t}{} \gpgradezero{ \frac{\epsilon}{2} F a F^\dagger }
=
\frac{1}{2 \eta} \gpgradezero{ a \lr{ F^\dagger J + J^\dagger F} }.
}

For \( a = 1 \), since scalars are reversion invariant (\(\alpha^\dagger = \alpha\) for any scalars \( \alpha \))

\begin{equation}\label{eqn:poyntingF:1000}
\gpgradezero{ F^\dagger J }
=
\gpgradezero{ F^\dagger J }^\dagger
=
\gpgradezero{ J^\dagger F },
\end{equation}

so the
multivector form of Poynting's theorem with respect to the time variation of the energy of the field is

\index{Poynting theorem}
\boxedEquation{eqn:poyntingF:220}{
c \spacegrad \cdot \gpgradeone{ \frac{\epsilon}{2} F F^\dagger }
+ \PD{t}{} \gpgradezero{ \frac{\epsilon}{2} F F^\dagger }
=
\inv{\eta} \gpgradezero{ J^\dagger F }.
}

The conventional statement of this theorem in terms of \( \BD, \BE, \BB, \BH, \BJ, \BM \) follows by direct substitution.
The multivector current \( J \) and its reverse are

\begin{dmath}\label{eqn:poyntingF:160}
\begin{aligned}
J &= \eta \lr{ c \rho - \BJ } + I \lr{ c \rho_m - \BM } \\
J^\dagger &= \eta \lr{ c \rho - \BJ } - I \lr{ c \rho_m - \BM },
\end{aligned}
\end{dmath}

so

\begin{dmath}\label{eqn:poyntingF:180}
0 =
\spacegrad \cdot \BS
-
\inv{\eta}
\lr{
- \eta \BJ \cdot \BE
- \eta \BM \cdot \BH
}
+ \PD{t}{\calE},
\end{dmath}

or
\boxedEquation{eqn:poyntingF:200}{
\spacegrad \cdot \BS + \BJ \cdot \BE + \BM \cdot \BH
%+ \PD{t}{\calE} = 0.
+ \PD{t}{\BB} \cdot \BH
+ \PD{t}{\BD} \cdot \BE = 0.
}

The sum of the last two terms is the time rate of change of the energy density.
In particular,
with neither electric nor magnetic current sources in a region of space,
the change of energy density through a volume is matched by a corresponding flux through the bounding surface

\begin{dmath}\label{eqn:maxwellsEquations:740}
\PD{t}{} \int_V
\inv{2} dV \lr{
\BB \cdot \BH
+ \BD \cdot \BE
}
=
-\int_{\partial V} dA \ncap \cdot \BS.
\end{dmath}

Here \( \ncap \) is the outward normal, so if the energy contained in the volume is decreasing, then \( \BS \) must represent the energy per unit area that leaves the volume.
The direction of the Poynting vector is the direction that the energy is leaving the volume.
Only the components of the Poynting vector that are colinear with the surface normal will result in energy leaving or entering the volume.

