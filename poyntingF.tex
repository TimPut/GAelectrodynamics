%
% Copyright © 2017 Peeter Joot.  All Rights Reserved.
% Licenced as described in the file LICENSE under the root directory of this GIT repository.
%
\subsection{Poynting theorem}
In conventional electromagnetism the energy and momentum density of the fields are

\index{energy density}
\index{momentum density}
\index{Poynting vector}
\index{energy flux}
\begin{dmath}\label{eqn:poyntingF:20}
\begin{aligned}
\calE &= \inv{2} \lr{ \BD \cdot \BE + \BB \cdot \BH } \\
\bcP v &= \inv{v} \BS = \inv{v} \BE \cross \BH = \inv{v} \lr{ I \BH } \cdot \BE.
\end{aligned}
\end{dmath}

where \( \calE \) is the energy density, \( \BS \) is the Poynting vector representing energy flux through a surface per unit time, and \( \bcP \) is the momentum density of the fields.

In GA formalism, the energy and momentum densities are related by grade selection from a single multivector formed from the electrodynamic field \( F \)

\boxedEquation{eqn:poyntingF:60}{
\inv{2} \epsilon F F^\dagger = \calE + \bcP v,
}

or
\begin{dmath}\label{eqn:poyntingF:40}
\begin{aligned}
\calE &= \inv{2} \epsilon \gpgradezero{ F F^\dagger } \\
\bcP v &= \inv{2} \epsilon \gpgradeone{ F F^\dagger } \\
\BS &= \inv{2 \eta} \gpgradeone{ F F^\dagger }.
\end{aligned}
\end{dmath}

It might be reasonable to call \( \epsilon F F^\dagger/2 \) the energy-momentum multivector for the field.
\Cref{eqn:poyntingF:40} can be demonstrated by direct expansion

\begin{dmath}\label{eqn:poyntingF:80}
\inv{2} \epsilon F F^\dagger
=
\inv{2} \epsilon \lr{ \BE + I \eta \BH } \lr{ \BE - I \eta \BH }
=
\inv{2} \epsilon \lr{ \BE^2 + \eta^2 \BH^2 }
+
\inv{2} I \epsilon \eta \lr{ \BH \BE - \BE \BH }
=
\inv{2} \lr{ \BD \cdot \BE + \BH \cdot \BB }
+
\frac{I}{v} \BH \wedge \BE
=
\inv{2} \lr{ \BD \cdot \BE + \BH \cdot \BB }
+
\frac{1}{v} \BE \cross \BH
=
\calE + \frac{\BS}{v}.\qedmarker
\end{dmath}

Poynting's theorem can be derived directly from \cref{eqn:poyntingF:60}.  To do so, first consider the scalar grade selection of the gradient acting on \( F F^\dagger \)

\begin{dmath}\label{eqn:poyntingF:100}
\gpgradezero{ \spacegrad F F^\dagger }
=
\gpgradezero{ F^\dagger \lrspacegrad F }
=
\gpgradezero{ (F^\dagger \lspacegrad) F + F^\dagger (\rspacegrad F) }
=
\gpgradezero{ (\rspacegrad F)^\dagger F + F^\dagger (\rspacegrad F) }
\end{dmath}

The gradient operator does not commute with the field \( F \), however, we may utilize
cyclic permutation within a scalar selection (i.e. \(\gpgradezero{ABC} = \gpgradezero{CAB}\)), provided
the gradient acts bidirectionally.  Rearranging Maxwell's equation for \( \spacegrad F \) gives

\begin{dmath}\label{eqn:poyntingF:120}
\spacegrad F = J - \inv{v} \PD{t}{F},
\end{dmath}

so
\begin{dmath}\label{eqn:poyntingF:140}
\frac{\epsilon}{2} \gpgradezero{ \spacegrad F F^\dagger }
=
\frac{\epsilon}{2} \gpgradezero{ \lr{ J - \inv{v} \PD{t}{F}}^\dagger F + F^\dagger \lr{ J - \inv{v} \PD{t}{F}} }
=
\frac{\epsilon}{2} \gpgradezero{ J^\dagger F + F^\dagger J - \inv{v} \PD{t}{} F^\dagger F }.
\end{dmath}

Scalars are reversion invariant \( \gpgradezero{ J^\dagger F } = \gpgradezero{ F^\dagger J } \), so

\boxedEquation{eqn:poyntingF:n}{
v \gpgradezero{ \spacegrad \frac{\epsilon}{2} F F^\dagger }
+ \PD{t}{} \gpgradezero{ \frac{\epsilon}{2} F F^\dagger }
=
\inv{\eta} \gpgradezero{ J^\dagger F }.
}

\index{Poynting theorem}
This is the multivector form of Poynting's theorem.  
The conventional statement of this theorem in terms of \( \BD, \BE, \BB, \BH, \BJ, \BM \) follows by direct substitution.  
The particular multivector current \( J \) and its reverse are

\begin{dmath}\label{eqn:poyntingF:160}
\begin{aligned}
J &= \eta \lr{ v \rho - \BJ } + I \lr{ v \rho_m - \BM } \\
J^\dagger &= \eta \lr{ v \rho - \BJ } - I \lr{ v \rho_m - \BM },
\end{aligned}
\end{dmath}

so

\begin{dmath}\label{eqn:poyntingF:180}
0 = 
\spacegrad \cdot \BS
-
\inv{\eta}
\lr{
- \eta \BJ \cdot \BE
- \eta \BM \cdot \BH
}
+ \PD{t}{\calE},
\end{dmath}

or
\boxedEquation{eqn:poyntingF:200}{
\spacegrad \cdot \BS + \BJ \cdot \BE + \BM \cdot \BH 
%+ \PD{t}{\calE} = 0.
+ \PD{t}{\BB} \cdot \BH
+ \PD{t}{\BD} \cdot \BE = 0.
}

\subsection{Complex power.}
TODO.
%\index{complex power}
