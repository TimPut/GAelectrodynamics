%
% Copyright © 2017 Peeter Joot.  All Rights Reserved.
% Licenced as described in the file LICENSE under the root directory of this GIT repository.
%

In conventional electromagnetism the energy and momentum density of the fields are

\index{energy density}
\index{momentum density}
\index{Poynting vector}
\index{energy flux}
\begin{dmath}\label{eqn:poyntingF:20}
\begin{aligned}
\calE &= \inv{2} \lr{ \BD \cdot \BE + \BB \cdot \BH } \\
\bcP c &= \inv{c} \BS = \inv{c} \BE \cross \BH.
\end{aligned}
\end{dmath}

where \( \calE \) is the energy density, \( \BS \) is the Poynting vector representing energy flux through a surface per unit time, and \( \bcP \) is the momentum density of the fields.
In geometric algebra, it is arguably more natural to write the Poynting vector as a bivector-vector dot product

\begin{dmath}\label{eqn:poyntingF:1100}
\BS = \lr{ I \BH } \cdot \BE,
\end{dmath}

however, we can do better, relating both the
energy and momentum densities to a single multivector formed from the products of the electrodynamic field \( F \) with its reverse \( F^\dagger \)

\boxedEquation{eqn:poyntingF:60}{
T(1) \equiv \inv{2} \epsilon F F^\dagger = \calE + \bcP c = \calE + \frac{\BS}{c},
}

so
\begin{dmath}\label{eqn:poyntingF:40}
\begin{aligned}
\calE &= \inv{2} \epsilon \gpgradezero{ F F^\dagger } \\
\bcP c &= \inv{2} \epsilon \gpgradeone{ F F^\dagger } \\
\BS &= \inv{2 \eta} \gpgradeone{ F F^\dagger }.
\end{aligned}
\end{dmath}

This dispenses with any requirement to refer to electric or magnetic field components in isolation.

Expanding \( T(1) \) in terms of \( \BE, \BH \) gives

\begin{dmath}\label{eqn:poyntingF:80}
T(1)
=
\inv{2} \epsilon F F^\dagger
=
\inv{2} \epsilon \lr{ \BE + I \eta \BH } \lr{ \BE - I \eta \BH }
=
\inv{2} \epsilon \lr{ \BE^2 + \eta^2 \BH^2 }
+
\inv{2} I \epsilon \eta \lr{ \BH \BE - \BE \BH }
=
\inv{2} \lr{ \BD \cdot \BE + \BH \cdot \BB }
+
\frac{I}{c} \BH \wedge \BE
=
\inv{2} \lr{ \BD \cdot \BE + \BH \cdot \BB }
+
\frac{1}{c} \BE \cross \BH
=
\calE + \frac{\BS}{c},
\end{dmath}

as claimed in \cref{eqn:poyntingF:20} and \cref{eqn:poyntingF:60}.  \( T(1) \) has one scalar component, and three vector components, and represents four
four of the sixteen components of a larger energy-momentum tensor.  The geometric algebra form of the complete energy-momentum tensor is
\index{energy-momentum tensor}
\index{Maxwell stress tensor}

%\begin{dmath}\label{eqn:poyntingF:760}
\boxedEquation{eqn:poyntingF:760}{
T(a) = \inv{2} \epsilon F a F^\dagger,
}
%\end{dmath}

where \( a \) is one of \( 1, \Be_1, \Be_2 \) or \( \Be_3 \), or any linear combination of these 0,1 grade multivector elements.  Observe that \( T(a) \) is a linear operator with respect to any parameter \( a = \alpha + \Ba, \alpha \in \bbR, \Ba \in \bbR^3 \).  \( T(a) \) only 0 and 1 grade elements, which we have seen for scalar parameters \( a \).  We will see by direct expansion that this is also the case for vector parameters.
Such an expansion of \( T(\Be_k) \) is harder to do algebaicly than \cref{eqn:poyntingF:80}
\footnote{Such an expansion is a worthwhile problem to develop GA manipulation skills.  The reader is encouraged to try this independently first, and to refer to
\cref{chap:stressTensorAlgebraically}
for hints if required.}.
On the other hand, it is easy to expand the energy-momentum tensor \( T(a) \) symbolically by brute force using a GA computer algebra package.
A Mathematica expansion of the elements of \( T(a) \) gives:

\begin{subequations}
\label{eqn:poyntingF:800}
\begin{equation}\label{eqn:poyntingF:1020}
\begin{aligned}
T(1)
&= \frac{\epsilon}{2} \lr{E_1^2 + E_2^2 + E_3^2} + \frac{\epsilon \eta^2}{2} \lr{H_1^2 + H_2^2 + H_3^2} \\
&+ \Be_1 \eta \epsilon \lr{E_2 H_3 - E_3 H_2} \\
&+ \Be_2 \eta \epsilon \lr{E_3 H_1 - E_1 H_3} \\
&+ \Be_3 \eta\epsilon \lr{E_1 H_2 - E_2 H_1}
\end{aligned}
\end{equation}
\begin{equation}\label{eqn:poyntingF:1040}
\begin{aligned}
T(\Be_1)
&= \eta \epsilon \lr{E_3 H_2 - E_2 H_3} \\
& + \frac{1}{2} \Be_1 \epsilon \lr{E_1^2 - E_2^2 - E_3^2} + \frac{\epsilon \eta^2}{2} \lr{ H_1^2 -  H_2^2 -  H_3^2} \\
& + \Be_2 \epsilon \lr{E_1 E_2 + \eta^2 H_1 H_2} \\
& + \Be_3 \epsilon \lr{E_1 E_3 + \eta^2 H_1 H_3}
\end{aligned}
\end{equation}
\begin{equation}\label{eqn:poyntingF:1060}
\begin{aligned}
T(\Be_2)
&= \eta \epsilon \lr{E_1 H_3 - E_3 H_1} \\
& + \Be_1 \epsilon \lr{E_1 E_2 + \eta^2 H_1 H_2} \\
& + \frac{1}{2} \Be_2 \epsilon \lr{-E_1^2 + E_2^2 - E_3^2 } + \frac{\epsilon \eta^2}{2} \lr{-H_1^2 +  H_2^2 -  H_3^2} \\
& + \Be_3 \epsilon \lr{E_2 E_3 + \eta^2 H_2 H_3}
\end{aligned}
\end{equation}
\begin{equation}\label{eqn:poyntingF:1080}
\begin{aligned}
T(\Be_3)
&= \eta \epsilon \lr{E_2 H_1 - E_1 H_2} \\
& + \Be_1 \epsilon \lr{E_1 E_3 + \eta^2 H_1 H_3} \\
& + \Be_2 \epsilon \lr{E_2 E_3 + \eta^2 H_2 H_3} \\
& + \frac{1}{2} \Be_3 \epsilon \lr{-E_1^2 - E_2^2 + E_3^2 } + \frac{\epsilon \eta^2}{2} \lr{ -H_1^2 -  H_2^2 + H_3^2}
\end{aligned}
\end{equation}
\end{subequations}

The components of the multivectors \( T(a) \) that we are calling the energy-momentum tensor, are more conventionally written out
as a symmetric tensor \( \Theta^{ij} \) as follows

\begin{dmath}\label{eqn:poyntingF:840}
\begin{aligned}
\Theta^{00} &= \frac{\epsilon}{2} \lr{ \BE^2 + \eta^2 \BH^2 } \\
\Theta^{0i} &= \inv{c} \lr{ \BE \cross \BH } \cdot \Be_i \\
\Theta^{ij} &= -\epsilon \lr{ E_i E_j + \eta^2 H_i H_j - \inv{2} \delta_{ij} \lr{ \BE^2 + \eta^2 \BH^2 } }.
\end{aligned}
\end{dmath}

The names and notation for this tensor components varies.
\( \Theta^{\alpha\beta}, \alpha, \beta \in \setlr{0,1,2,3} \) as defined in \cref{eqn:poyntingF:840} is called the symmetric
stress tensor \citep{jackson1975cew},
whereas other authors call this the energy-momentum tensor and express it as \( T^{\alpha\beta} \) \citep{landau1980classical}, \citep{doran2003gap}.
The sign conventions and notation for the spatial components \( \Theta^{ij}, i, j \in \setlr{1,2,3} \) vary as well, but all authors appear to call this subset the Maxwell stress tensor.
The Maxwell stress tensor is written as \( \sigma_{ij} (=-\Theta^{ij}) \) \citep{landau1980classical}, or as
\( T_{ij} (=-\Theta^{ij}) \)
\citep{griffiths1999introduction},
\citep{jackson1975cew}.

The symmetric stress tensor components of \cref{eqn:poyntingF:840}
are related to the multivector representation expanded in \cref{eqn:poyntingF:800} by

\begin{dmath}\label{eqn:poyntingF:820}
\begin{aligned}
&\gpgradezero{ T(1) }
%= \calE
=
{\Theta_0}^0 = \Theta^{00} \\
&\gpgradeone{ T(1) } \cdot \Be_i
%= \frac{\BS}{c} \cdot \Be_i
= {\Theta_0}^i = \Theta^{0i} \\
&\gpgradezero{ T(\Be_i) }
%= -\frac{\BS}{c} \cdot \Be_i
= {\Theta_i}^0 = -\Theta^{i0} \\
&\gpgradeone{ T(\Be_i) } \cdot \Be_j = {\Theta_i}^j = -\Theta^{ij}.
\end{aligned}
\end{dmath}

The Maxwell stress tensor components \( \gpgradeone{T(\Be_k)} \) can be found expressed in a dyadic notation (\citep{griffiths1999introduction}, \citep{jackson1975cew}) as follows

\begin{dmath}\label{eqn:poyntingF:1140}
\gpgradeone{ T(\Ba) }
=
\sum_i a_i \gpgradeone{ T(\Be_i) }
=
\sum_{i,j} a_i \lr{ \gpgradeone{ T(\Be_i) } \cdot \Be_j } \Be_j
=
\sum_{i,j} a_i T_{ij} \Be_j
\equiv
\Ba \cdot \lrT,
\end{dmath}

so that
\begin{dmath}\label{eqn:poyntingF:1160}
\lr{ \Ba \cdot \lrT } \cdot \Bb
=
\sum_{i,j} a_i T_{ij} b_j.
\end{dmath}

With the Maxwell stress tensor parameterized by a vector, we don't really have any need for this dyadic notation, but it is worth mentioning to
understand how the two formalisms are related.

The complete specification of the energy-momentum tensor for a parameter \( a = \alpha + \Ba = \alpha + \sum_k a_k \Be_k \) is

%\begin{dmath}\label{eqn:poyntingF:1120}
\boxedEquation{eqn:poyntingF:1120}{
T(\alpha + \Ba)
=
\alpha \lr{
   \calE + \frac{\BS}{c}
}
-
\Ba \cdot \frac{\BS}{c}
+ \BT(\Ba),
%+ \gpgradeone{ T(\Ba) }.
}
%\end{dmath}

where the shorthand \( \BT(\Ba) = \gpgradeone{T(\Ba)} \) has been introduced for the Maxwell stress tensor.

