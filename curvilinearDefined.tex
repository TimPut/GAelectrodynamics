%
% Copyright © 2017 Peeter Joot.  All Rights Reserved.
% Licenced as described in the file LICENSE under the root directory of this GIT repository.
%
\index{curvilinear coordinates}
Curvilinear coordinates can be defined for any subspace spanned by a parameterized vector into that space.
%Consider a continuous subspace parameterized by a two parameter vector function \( \Bx = \Bx(u_1, u_2) \) that is differentiable with respect to either parameter
As an example, consider a two parameter planar subspace of parameterized by the following continuous vector function

\begin{dmath}\label{eqn:curvilinearDefined:480}
\Bx(u_1, u_2) = u_1 \Be_1 \frac{\sqrt{3}}{2} \cosh\lr{ \Atanh(1/2) + \Be_{12} u_2 },
\end{dmath}

where \( u_1 \in [0,1] \) and \( u_2 \in [0, \pi/2] \).
This parameterization spans the first quadrant of the ellipse with semi-major axis length 1, and semi-minor axis length \( 1/2 \)
\footnote{
A parameterization of an elliptic area may or may not not be of much use in electrodynamics.  It does, however, provide a fairly simple but non-trivial example of a non-orthonormal parameterization.}
Contours for this parameterization are plotted in \cref{fig:ellipticalContours:ellipticalContoursFig1}.
The radial contours are for fixed values of \( u_2 \) and the elliptical contours fix the value of \( u_1 \), and depict a set of ellipic curves
with a semi-major/major axis ratio of \( 1/2 \).

\imageFigure{../figures/GAelectrodynamics/ellipticalContoursFig1}{Contours for an elliptical region.}{fig:ellipticalContours:ellipticalContoursFig1}{0.3}

We define a curvilinear basis associated with each point in the region by the partials

\begin{dmath}\label{eqn:curvilinearDefined:80}
\begin{aligned}
\Bx_{1} &= \PD{u_1}{\Bx} \\
\Bx_{2} &= \PD{u_2}{\Bx}.
\end{aligned}
\end{dmath}

For our the function \cref{eqn:curvilinearDefined:480} our curvilinear basis elements are

\begin{dmath}\label{eqn:curvilinearDefined:520}
\begin{aligned}
\Bx_{1} &= \Be_1 \frac{\sqrt{3}}{2} \cosh\lr{ \Atanh(1/2) + \Be_{12} u_2 } \\
\Bx_{2} &= u_1 \Be_2 \frac{\sqrt{3}}{2} \sinh\lr{ \Atanh(1/2) + \Be_{12} u_2 }.
\end{aligned}
\end{dmath}

We form vector valued differentials for each parameter

\begin{dmath}\label{eqn:curvilinearDefined:500}
\begin{aligned}
d\Bx_{1} &= \Bx_1 du_1 \\
d\Bx_{2} &= \Bx_2 du_2.
\end{aligned}
\end{dmath}

For \cref{eqn:curvilinearDefined:480},
the values of these differentials \( d\Bx_1, d\Bx_2 \) with \( du_1 = du_2 = 0.1 \) are plotted
in
\cref{fig:ellipticalContours:ellipticalContoursFig2}
for the points
\( (u_1, u_2) = (0.7, 5 \pi/20), (0.9, 3 \pi/20), (1.0, 5 \pi/20) \)
in
(dark-thick) red, blue and purple respectively.

\imageFigure{../figures/GAelectrodynamics/ellipticalContoursFig2}{Differentials for an elliptical parameterization.}{fig:ellipticalContours:ellipticalContoursFig2}{0.3}

In this case and in general there is no reason to presume that there is any orthonormality constraint on the basis \( \setlr{ \Bx_{1}, \Bx_{2} } \) for a given two parameter subspace.

Should we wish to calculate the reciprocal frame
for \cref{eqn:curvilinearDefined:480}
, we would find
(\cref{problem:curvilinearDefined:560}) that

\begin{dmath}\label{eqn:curvilinearDefined:540}
\begin{aligned}
\Bx^{1} &= \Be_1 \sqrt{3} \sinh\lr{ \Atanh(1/2) + \Be_{12} u_2 } \\
\Bx^{2} &= \frac{\Be_2}{u_1} \sqrt{3} \cosh\lr{ \Atanh(1/2) + \Be_{12} u_2 }.
\end{aligned}
\end{dmath}

These are plotted (scaled by \( da = 0.1 \) so they fit in the image nicely) in \cref{fig:ellipticalContours:ellipticalContoursFig2} using thin light arrows.

When evaluating surface integrals, we will form
oriented (bivector) area elements from the wedge product of the differentials

\begin{dmath}\label{eqn:curvilinearDefined:60}
d^2 \Bx \equiv d\Bx_{1} \wedge d\Bx_{2}.
\end{dmath}

This absolute value of this area element \( \sqrt{-(d^2 \Bx)^2} \) is the area of the parallelogram spanned by \( d\Bx_1, d\Bx_2 \).
In this example, all such area elements lie in the \( x-y \) plane, but that need not be the case.

Also note that we will only perform integrals for those parametrizations for which the area element \( d^2 \Bx \) is non-zero.

%If the spacing between the contours is made small enough, the boundaries of each partition will define a planar region at the point of evaluation.
%All points in the interior will be accessible by a combination of the vectors formed from the partials of \( \Bx \) at that point.

\makeproblem{Elliptic curvilinear and reciprocal basis.}{problem:curvilinearDefined:560}{
From \cref{eqn:curvilinearDefined:480}, compute the
curvilinear coordinates \cref{eqn:curvilinearDefined:520}, and the reciprocal frame vectors \cref{eqn:curvilinearDefined:540}.
Check using scalar grade selection that \( \Bx^i \cdot \Bx_j = {\delta^i}_j \).
Hints: Given \( \mu = \Atanh(1/2) \),
\begin{itemize}
\item \( \cosh( \mu + i \theta ) \Be_2 = \Be_2 \cosh( \mu - i \theta ) \).
\item \( \Real\lr{ \cosh( \mu - i \theta ) \sinh( \mu + i \theta ) } = 2/3 \).
\end{itemize}
} % problem

\paragraph{fixme:}
don't introduce the idea of tangent space until a 3D example.
Remove the \R{3} reference above, and keep this first example planar.

At the point of evaluation, the span of these differentials is called the tangent space.
In this particular case the tangent space at all points in the region is the entire x-y plane.
These partials locally span the tangent space at a given point on the surface.

%%%%\makeexample{A two parameter surface.}{example:curvilinearDefined:420}{
%%%%As an illustration, consider a two parameter subspace of \R{3}
%%%%
%%%%\begin{dmath}\label{eqn:curvilinearDefined:440}
%%%%\Bx(u_1, u_2)
%%%%=
%%%%u_1 \Be_1 + u_2 \Be_2 + \lr{(u_1 - 2 u_2)^2 + \cos(u_2) \sin(5 u_1 + u_2)} \Be_3.
%%%%\end{dmath}
%%%%
%%%%The tangent space associated with this parameterization is spanned by the vectors
%%%%
%%%%\begin{dmath}\label{eqn:curvilinearDefined:460}
%%%%\begin{aligned}
%%%%\Bx_{u_1}
%%%%&=
%%%%\Be_1 + \lr{
%%%%   2 (u_1 - 2 u_2) + 5 \cos(u_2) \cos(5 u_1 + u_2)
%%%%} \Be_3 \\
%%%%\Bx_{u_2}
%%%%&=
%%%%\Be_2 +
%%%%\lr{
%%%%   -4 (u_1 - 2 u_2)
%%%%- \sin(u_2) \sin(5 u_1 + u_2)
%%%%+ \cos(u_2) \cos(5 u_1 + u_2)
%%%%} \Be_3.
%%%%\end{aligned}
%%%%\end{dmath}
%%%%
%%%%This tangent space is plotted in \cref{fig:twoParameterDifferentialField:twoParameterDifferentialFieldFig1}.
%%%%
%%%%FIXME: Wolfgang: ``Plot the TpM at special point \( P(0.3,1.5,0.3) \) in Fig.  Calculate using hand and CAS MM.''
%%%%} % example
%%%%
%%%%\imageFigure{../figures/GAelectrodynamics/twoParameterDifferentialFieldFig1}{Curvilinear coordinates along a two parameter surface.}{fig:twoParameterDifferentialField:twoParameterDifferentialFieldFig1}{0.3}

