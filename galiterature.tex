%
% Copyright © 2017 Peeter Joot.  All Rights Reserved.
% Licenced as described in the file LICENSE under the root directory of this GIT repository.
%
%{
The notation and nomenclature used to express Maxwell's equation in the GA literature has not been standardized.
Here are some variations that could be helpful when examining the literature.

\paragraph{Space Time Algebra (STA).  \citep{doran2003gap}}

Maxwell's equation is written
\begin{dmath}\label{eqn:galiterature:80}
\begin{aligned}
\grad F &= J \\
F &= \BE + I \BB \\
I &= \gamma_0 \gamma_1 \gamma_2 \gamma_3 \\
J &= \gamma_\mu J^\mu = \gamma_0 \lr{ \rho - \BJ } \\
\grad &= \gamma^\mu \partial_\mu = \gamma_0 \lr{ \partial_t + \spacegrad }.
\end{aligned}
\end{dmath}

STA uses a relativistic basis \( \setlr{ \gamma_\mu } \) and its dual \( \setlr{ \gamma^\mu} \) for which \( \gamma_0^2 = -\gamma_k^2 = 1, k \in 1,2,3 \), and \( \gamma^\mu \cdot \gamma_\nu = {\delta^\mu}_\nu \).
Spatial vectors are expressed in terms of the Pauli basis \( \sigma_i = \gamma_i \gamma_0 \), which are bivectors that behave as Euclidean basis vectors (squaring to unity, and all mutually anticommutative).
\( F \) is called the electromagnetic field strength (and is not a 1,2 multivector, but a bivector), \( \grad \) is called the vector derivative operator, \( \spacegrad \) called the three-dimensional vector derivative operator, and \( J \) is called the spacetime current (and is a vector, not a multivector).
The physicist's ``natural units'' \( c = \epsilon_0 = \mu_0 \) are typically used in STA.
The d'Alambertian in STA is \( \Box = \grad^2 = \partial_t^2 - \spacegrad^2 \), although the earliest
formulation of STA \citep{hestenes1966space} used \( \Box \) for the vector derivative.
Only spatial vectors are in bold, and all other multivectors are non-bold.
STA is inherently relativistic, and can be used to obtain many of the results in this book more directly.

Maxwell's equations as expressed in \cref{eqn:maxwellsEquations:460} can be converted to their STA form (in SI units) by setting \( \Be_i = \gamma_i \gamma_0 \) and by left multipliplying both sides by \( \gamma_0 \).

\paragraph{Algebra of Physical Space (APS).  \citep{baylis2004electrodynamics}}
Maxwell's equation is written as

\begin{dmath}\label{eqn:galiterature:40}
\begin{aligned}
\overbar{\partial} \BF &= \inv{\epsilon_0 c} \overline{\jmath} \\
\BF &= \BE + i c \BB \\
i &= \Be_{123} \\
\partial &= \inv{c} \partial_t - \spacegrad \\
j &= \inv{\epsilon_0 c} \lr{ \rho c + \Bj }.
\end{aligned}
\end{dmath}

\( \BF \) is called the Faraday, \( \partial \) the gradient, \( j \) the current density, and
0,1 multivectors are called paravectors.
An operation called Clifford conjugation (or spatial reversal) designated \( \overbar{A} \) is introduced that toggles the sign of any multivector components with grade \( g \mod 4 = 1,2 \).
For example given a multivector \( A = 1 + \Be_1 + \Be_{12} + \Be_{123} \), the Clifford conjugate is

\begin{dmath}\label{eqn:galiterature:60}
\overbar{A} = 1 - \Be_1 - \Be_{12} + \Be_{123},
\end{dmath}

leaving the sign of the scalar and pseudoscalar components untouched.
This Clifford conjugation operation is also used to express relativistic proper length and Lorentz transformations.
The d'Alambertian is written as \( \Box = \partial \overbar{\partial} = (1/c^2) \partial_t^2 - \spacegrad^2 \).
There are a large (arguably confusing) variety of conjugation and complex-like selection operations in APS.
While APS is only cosmetically different than \cref{eqn:maxwellsEquations:460} the treatment in \citep{baylis2004electrodynamics} is inherently relativisitic, and carries the additional learning curve of both special relativity and relativistic geometric algebra.

\paragraph{Jancewicz.  \citep{jancewicz1988multivectors}}

% pg. 78
Maxwell's equation in linear isotropic media is written as

\begin{dmath}\label{eqn:galiterature:20}
\begin{aligned}
\calD f &= \tilde{\jmath} \\
\calD &= \spacegrad + \sqrt{\epsilon\mu} \PD{t}{} \\
f &= \Be + \Bcap \\
\Be &= \sqrt{\epsilon} \BE \\
\Bcap &= \inv{\sqrt{\mu}} I \BB \\
I &= \Be_{123} \\
\tilde{\jmath} &= \inv{\sqrt{\epsilon}}\rho - \sqrt{\mu} \Bj \\
\end{aligned}
\end{dmath}

Jancewicz works with fields that have been re-dimensionalized to the same units, uses an overhat bold notation for bivectors (which are sometimes called volutors).
\( \calD \) is called the cliffor differential operator, \( f \) the electromagnetic cliffor, and \( \tilde{\jmath} \) the density of electric sources.
The d'Alambertian is written
as \( \Box = \calD^\conj \calD = \spacegrad^2 - \epsilon\mu \partial_t^2 \), where
\( \calD^\conj = \spacegrad - \sqrt{\epsilon\mu} \partial_t \).

%}
