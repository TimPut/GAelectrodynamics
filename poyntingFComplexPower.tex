%
% Copyright © 2017 Peeter Joot.  All Rights Reserved.
% Licenced as described in the file LICENSE under the root directory of this GIT repository.
%
\index{complex power}

The time domain representation of a phasor based field \( F = F(\omega) \) is

\begin{dmath}\label{eqn:poyntingFComplexPower:20}
F(t)
= \Real\lr{ F e^{j \omega t} }
= \inv{2} \lr{ F e^{j \omega t} + F^\conj e^{-j\omega t} }.
\end{dmath}

The energy-momentum multivector for the field is therefore
\begin{dmath}\label{eqn:poyntingFComplexPower:40}
\inv{2} \epsilon F(t) {F(t)}^\dagger
=
\frac{\epsilon}{8}
\lr{
F e^{j \omega t} + F^\conj e^{-j\omega t}
}
\lr{
F^\dagger e^{j \omega t} + \lr{F^\conj}^\dagger e^{-j\omega t}
}
=
\frac{\epsilon}{8}
\lr{
F F^\dagger e^{2 j \omega t}
+
\lr{ F F^\dagger e^{2 j \omega t} }^\conj
+
F^\conj F^\dagger + \lr{ F^\conj F^\dagger }^\conj
},
\end{dmath}
which can be written as

%\begin{dmath}\label{eqn:poyntingFComplexPower:60}
\boxedEquation{eqn:poyntingFComplexPower:60}{
\calE + \frac{\BS}{v}
=
\inv{2} \epsilon F(t) {F(t)}^\dagger
=
\frac{\epsilon}{4} \Real \lr{ F^\conj F^\dagger + F F^\dagger e^{2 j \omega t} }.
}
%\end{dmath}

Assuming real \( \epsilon, \mu \), lets look at how these terms expand in terms of \( \BD = \epsilon \BE, \BB = \mu \BH\), starting with the constant contribution to the energy and momentum

% e eta = sqrt( e^2 mu/e ) = sqrt( e mu ) = 1/v
% e eta^2 = e mu/e = mu
\begin{dmath}\label{eqn:poyntingFComplexPower:80}
\inv{4} \epsilon F^\conj F^\dagger
=
\inv{4} \epsilon \lr{ \BE^\conj + I \eta \BH^\conj } \lr{ \BE - I \eta \BH }
=
\inv{4} \lr{ \BE^\conj \epsilon \BE + \epsilon \eta^2 \BH^\conj \BH
+ I \epsilon \eta \lr{ \BH^\conj \BE - \BE^\conj \BH }
}
=
\inv{4} \lr{
\epsilon \Abs{\BE}^2 + \mu \Abs{\BH}^2
+ \frac{I}{v} \lr{ \BH^\conj \BE - \BE^\conj \BH }
}.
\end{dmath}

The scalar terms are already real, but the real part of the vector term is
\begin{dmath}\label{eqn:poyntingFComplexPower:200}
\frac{I}{4 v} \Real \lr{ \BH^\conj \BE - \BE^\conj \BH }
=
\frac{I}{8 v} \lr{
\BH^\conj \BE - \BE^\conj \BH
+ \BH \BE^\conj - \BE \BH^\conj
}
=
\frac{I}{8 v} \lr{
2 \BH^\conj \wedge \BE
+ 2 \BH \wedge \BE^\conj
}
=
\frac{1}{4 v} \lr{
\BE \cross \BH^\conj
+ \BE^\conj \cross \BH
}
=
\frac{1}{2 v} \Real \lr{
\BE \cross \BH^\conj
}.
\end{dmath}

The \( \epsilon F F^\dagger \) factor of \( e^{2 j \omega t} \) above was expanded in \cref{eqn:poyntingF:80}, so the energy momentum multivector is

\begin{dmath}\label{eqn:poyntingFComplexPower:220}
\begin{aligned}
\calE + \frac{\BS}{v}
&=
\inv{4} \lr{
\epsilon \Abs{\BE}^2 + \mu \Abs{\BH}^2 }
+
\frac{1}{2 v} \Real \lr{
\BE \cross \BH^\conj
} \\
&+
\Real
\lr{
   \lr{
     \inv{4} \lr{ \BD \cdot \BE + \BH \cdot \BB}
   + \frac{1}{2 v} \BE \cross \BH
   }
   e^{2 j \omega t }
}.
\end{aligned}
\end{dmath}

It is conventional to introduce a complex Poynting vector

\begin{equation}\label{eqn:poyntingFComplexPower:240}
\calS = \inv{2} \BE \cross \BH^\conj = \inv{2} \lr{ I \BH^\conj } \cdot \BE,
\end{equation}
so the energy and momentum split of \( \epsilon F F^\dagger/2 \) is
\begin{dmath}\label{eqn:poyntingFComplexPower:260}
\begin{aligned}
\calE &=
\inv{4} \lr{
\epsilon \Abs{\BE}^2 + \mu \Abs{\BH}^2 }
+
\inv{4} \Real
\lr{
   \lr{ \BD \cdot \BE + \BH \cdot \BB}
   e^{2 j \omega t }
} \\
\BS &= \Real \calS
+
\inv{2} \Real
\lr{
\lr{ \BE \cross \BH }
   e^{2 j \omega t }
}.
\end{aligned}
\end{dmath}

Averaging over one period \( T \) kills the sinusoidal contributions, so the steady state energy and Poynting vectors are just

\begin{dmath}\label{eqn:poyntingFComplexPower:280}
\begin{aligned}
\inv{T} \int_\tau^{\tau+T} \calE(t) dt &=
\inv{4} \lr{
\epsilon \Abs{\BE}^2 + \mu \Abs{\BH}^2 } \\
\inv{T} \int_\tau^{\tau+T} \BS(t) dt &= \Real \calS.
\end{aligned}
\end{dmath}

Much of the preceding discussion shows how to determine the conventional energy and Poynting vector expressions in terms of separate electric and magnetic fields.
While this demonstrate that the GA approach yields equivalent results, what we really want to do is to
avoid separate electric and magnetic fields, using
\cref{eqn:poyntingFComplexPower:60} to describe the energy and momentum of the field in terms of the composite field \( F \) directly.
When using \( F \) instead of the component fields,
scalar and vector grade selection can be used to decouple the energy and momentum terms when desired.
