%
% Copyright © 2017 Peeter Joot.  All Rights Reserved.
% Licenced as described in the file LICENSE under the root directory of this GIT repository.
%
\index{linear combination}
\makedefinition{Linear combination}{dfn:prerequisites:linearcombination}{
Let \( S = \setlr{ \Bx_1, \Bx_2, \cdots, \Bx_k } \) be a subset of a vector space \( V \).
A linear combination of vectors in \( S \) is any sum
\begin{equation*}
a_1 \Bx_1
+
a_2 \Bx_2
+
\cdots
+
a_k \Bx_k.
\end{equation*}
} % definition

For example, if \( \Bx_1 = \Be_1 + \Be_2, \Bx_2 = \Be_1 - \Be_2 \), then \( 2 \Bx_1 + 3 \Bx_2 = 5 \Be_1 - \Be_2 \) is a linear combination.

\index{linear dependence}
\makedefinition{Linear dependence.}{dfn:prerequisites:dependence}{
Let \( S = \setlr{ \Bx_1, \Bx_2, \cdots, \Bx_k } \) be a subset of a vector space \( V \).
This set \( S \) is linearly dependent if one can construct any equation
\begin{equation*}
\Bzero =
a_1 \Bx_1
+
a_2 \Bx_2
+
\cdots
+
a_k \Bx_k,
\end{equation*}
for which not all of the coefficients \( a_i, \, 1 \le i \le k \) are zero.
} % definition

For example, the vectors \( \Bx_1 = \Be_1 + \Be_2, \Bx_2 = \Be_1 - \Be_2, \Bx_3 = \Be_1 + 3 \Be_2 \) are linearly dependent since \( 2 \Bx_1 - \Bx_2 - \Bx_3 = 0 \), and the vectors \( \By_1 = \Be_1 + \Be_2 + \Be_3, \By_2 = \Be_1 + \Be_3, \By_3 = 3 \Be_1 + \Be_2 + 3 \Be_3 \) are linearly dependent since \( \By_1 + 2 \By_2 - \By_3 = 0 \).

\index{linear independence}
\makedefinition{Linear independence.}{dfn:prerequisites:independence}{
Let \( S = \setlr{ \Bx_1, \Bx_2, \cdots, \Bx_k } \) be a subset of a vector space \( V \).
This set is linearly independent if there are no equations with \( a_i \ne 0,\, 1 \le i \le k\) such that
\begin{equation*}
\Bzero =
a_1 \Bx_1
+
a_2 \Bx_2
+
\cdots
+
a_k \Bx_k.
\end{equation*}
} % definition

For example, the vectors
\( \Bx_1 = \Be_1 + \Be_2, \Bx_2 = \Be_1 - \Be_2, \Bx_3 = 2 \Be_1 + \Be_3 \), are
linearly independent, as are the vectors
\( \By_1 = \Be_1 + \Be_2 + \Be_3, \By_2 = \Be_1 + \Be_3, \By_3 = \Be_2 + \Be_3 \).

\index{span}
\makedefinition{Span.}{dfn:prerequisites:span}{
Let \( S = \setlr{ \Bx_1, \Bx_2, \cdots, \Bx_k } \) be a subset of a vector space \( V \).
The span
of this set is the set of all linear combinations of these vectors, denoted
\begin{equation*}
\Span(S) =
\setlr{
a_1 \Bx_1
+
a_2 \Bx_2
+
\cdots
+
a_k \Bx_k}.
\end{equation*}
} % definition

\index{subspace}
\makedefinition{Subspace.}{dfn:prerequisites:subspace}{
Let \( S = \setlr{ \Bx_1, \Bx_2, \cdots, \Bx_k } \) be a subset of a vector space \( V \).
This subset is
a subspace if \( S \) is a vector space under the multiplication and addition operations of the vector space \( V \).
} % definition

\index{basis}
\index{dimension}
\makedefinition{Basis and dimension}{dfn:prerequisites:basisanddimension}{
Let \( S = \setlr{ \Bx_1, \Bx_2, \cdots, \Bx_n } \) be a linearly independent subset of \( V \).
This set is a basis if \( \Span(S) = V \).
The number of vectors \( n \) in this set is called the dimension of the space.
} % definition

