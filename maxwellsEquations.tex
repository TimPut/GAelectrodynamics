%
% Copyright © 2016 Peeter Joot.  All Rights Reserved.
% Licenced as described in the file LICENSE under the root directory of this GIT repository.
%
\section{Maxwell's equations for antenna theory.}

\makedigression{
\input{../frequencydomain/frequencydomainMaxwellsExtraction.tex}
}

\section{Plane waves.}

In the time harmonic representation for source free conditions Maxwell's equation \cref{eqn:maxwellsEquations:460} is just
\begin{dmath}\label{eqn:maxwellsEquations:560}
\begin{aligned}
F &= \BE + \eta I \BH \\
\lr{ \spacegrad + j k } F &= 0,
\end{aligned}
\end{dmath}

where \( k = \omega/v \) is the wave number.  It is now possible to examine what constraints Maxwell's equation imposes on plane waves of the form

\begin{dmath}\label{eqn:maxwellsEquations:580}
\begin{aligned}
\BE &= \BE_0 e^{-j \Bk \cdot \Bx} \\
\BH &= \BH_0 e^{-j \Bk \cdot \Bx},
\end{aligned}
\end{dmath}

or
\begin{dmath}\label{eqn:maxwellsEquations:600}
F = F_0 e^{-j \Bk \cdot \Bx}.
\end{dmath}

%
% Copyright © 2016 Peeter Joot.  All Rights Reserved.
% Licenced as described in the file LICENSE under the root directory of this GIT repository.
%
%\section{Plane waves}
\index{plane wave}
The gradient action on the electromagnetic field is

\begin{dmath}\label{eqn:frequencydomainPlaneWaves:160}
\spacegrad F_0 e^{-j \Bk \cdot \Bx}
=
\sum_{m = 1}^3 \Be_m \partial_m
F_0 e^{-j \Bk \cdot \Bx}
=
\sum_{m = 1}^3 \Be_m
F_0
\lr{ -j k_m }
e^{-j \Bk \cdot \Bx}
=
-j \Bk F_0,
\end{dmath}
so

\begin{dmath}\label{eqn:frequencydomainPlaneWaves:180}
j k (1 - \kcap) F_0 = 0.
\end{dmath}

This means that the field must be of the form

%\begin{dmath}\label{eqn:frequencydomainPlaneWaves:200}
\boxedEquation
{eqn:frequencydomainPlaneWaves:200}
{
F = (1 + \kcap) \BE_0 e^{-j \Bk \cdot \Bx},
}
%\end{dmath}
where \( \BE_0 \) is a vector valued complex constant, and \( \kcap \cdot \BE_0 = 0 \).
The dot product constraint follows from the requirement that the \( I \BH \propto \kcap \BE_0 \) portion of the electromagnetic field is a bivector.
The time domain representation of the field is
\begin{dmath}\label{eqn:frequencydomainPlaneWaves:460}
F = (1 + \kcap) \Real{ \BE_0 e^{-j \Bk \cdot \Bx} },
\end{dmath}
but we will see later
instead of using a scalar imaginary \( j \), it is possible to use either the unit bivector for the transverse plane or the \R{3} unit pseudoscalar as the imaginary, and that a plane wave of any polarization can be encoded without any requirement to take real parts.

From \cref{eqn:frequencydomainPlaneWaves:200} the interdependence of the electric and magnetic field portions of the field can be read off immediately.
Those are

\begin{subequations}
\label{eqn:frequencydomainPlaneWaves:220}
\begin{dmath}\label{eqn:frequencydomainPlaneWaves:221}
\BE = \BE_0 e^{-j \Bk \cdot \Bx}
\end{dmath}
\begin{dmath}\label{eqn:frequencydomainPlaneWaves:222}
I \BH = \inv{\eta} \kcap \BE_0 e^{-j \Bk \cdot \Bx},
\end{dmath}
\end{subequations}

or
\begin{dmath}\label{eqn:frequencydomainPlaneWaves:380}
I \BH = \inv{\eta} \kcap \BE.
\end{dmath}

\index{pseudoscalar!spherical}
Since the \R{3} pseudoscalar can be written as

\begin{dmath}\label{eqn:frequencydomainPlaneWaves:400}
I = \kcap \Ecap \Hcap,
\end{dmath}
the directions \( \kcap, \Ecap, \Hcap \) must form a right handed triple.
It is thus expected that the magnetic field is perpendicular to the propagation direction, and that the electric and magnetic fields are explicitly perpendicular, facts that are easily verified

\begin{subequations}
\label{eqn:frequencydomainPlaneWaves:440}
\begin{dmath}\label{eqn:frequencydomainPlaneWaves:260}
\kcap \cdot \BH
= \gpgradezero{ \kcap (-I \kcap \BE_0) } e^{-j \Bk \cdot \Bx}
= -\gpgradezero{ I \BE_0 } e^{-j \Bk \cdot \Bx}
= 0
\end{dmath}
\begin{dmath}\label{eqn:frequencydomainPlaneWaves:280}
\BE \cdot \BH
=
\gpgradezero{ \BE \lr{ -\frac{I}{\eta}} \kcap \BE }
=
-\inv{\eta} \BE^2
\gpgradezero{ \kcap I }
=
0.
\end{dmath}
\end{subequations}

In conventional vector treatments of electromagnetic field theory the field relationships of \cref{eqn:frequencydomainPlaneWaves:220} and the propagation directions are written out explicitly as cross products, instead of multivector equations.
Those cross product relations are obtained easily

\begin{subequations}
\label{eqn:frequencydomainPlaneWaves:420}
\begin{dmath}\label{eqn:frequencydomainPlaneWaves:240}
\BH
= -I \inv{\eta} \kcap \BE
= -I \inv{\eta} (\kcap \wedge \BE)
= -I \inv{\eta} I (\kcap \cross \BE)
= \inv{\eta} \kcap \cross \BE
\end{dmath}
\begin{dmath}\label{eqn:frequencydomainPlaneWaves:300}
\BE
= \eta \kcap I \BH
= \eta I \kcap \wedge \BH
= \eta I^2 \kcap \cross \BH
= \eta \BH \cross \kcap
\end{dmath}
\begin{dmath}\label{eqn:frequencydomainPlaneWaves:340}
\kcap
= I \Hcap \Ecap
= I (\Hcap \wedge \Ecap)
= I^2 (\Hcap \cross \Ecap)
= \Ecap \cross \Hcap.
\end{dmath}
\end{subequations}


\section{Poynting theorem}

Poynting's theorem describes the relationship between the flux of energy through a surface bounding a volume.
The theorem follows from computing the divergence of the Poynting vector \( \BS = \BE \cross \BH \).  In terms of \( \BE \) and \( \BH \) the Poynting vector can be written in dual form as a dot product

\begin{equation}\label{eqn:maxwellsEquations:780}
\BE \cross \BH
=
\gpgradeone{ I (\BH \wedge \BE) }
=
\gpgradeone{ I \BH \BE }
=
(I \BH) \cdot \BE.
\end{equation}

Similarly, the Poynting divergence is most compactly expressed as a scalar selection operation

\begin{equation}\label{eqn:maxwellsEquations:640}
\spacegrad \cdot \lr{ \BE \cross \BH }
=
\gpgradezero{ \spacegrad I \lr{ \BH \wedge \BE } }
=
\gpgradezero{ \spacegrad I \BH \BE }.
\end{equation}

Here the gradient is acting on everything to the right, however, allowing the gradient to act bidirectionally, and employing the
the flexibility to use cyclic permutation within a scalar selection
(i.e. \(\gpgradezero{ABC} = \gpgradezero{CAB}\))
, allows for the easy application of the chain rule

\begin{dmath}\label{eqn:maxwellsEquations:760}
\gpgradezero{ \spacegrad I \BH \BE }
=
\gpgradezero{ \BE \lrspacegrad I \BH }
=
\gpgradezero{ (\BE \lspacegrad) I \BH }
+\gpgradezero{ \BE (\rspacegrad I \BH) }
\end{dmath}

Explicit left and right acting gradients are required because the gradient operator does not commute with the vector fields.

The gradient action on \( I \BH \) (from the left) is given by
\cref{eqn:maxwellsEquations:380}.  The right acting gradient action on \( \BE \) is given by reversing all the products in
%\spacegrad \BE = \inv{\epsilon} \rho - I \BM - \mu \PD{t}{(I\BH)}
\cref{eqn:maxwellsEquations:360} (in particular noting that \( I^\dagger = -I \) )

\begin{dmath}\label{eqn:maxwellsEquations:660}
%I \BH \lspacegrad = \frac{I \rho_\txtm}{\mu} + \BJ + \epsilon \PD{t}{\BE}.
\BE \lspacegrad = \inv{\epsilon} \rho + I \BM + \mu \PD{t}{(I\BH)}.
\end{dmath}

This gives
\begin{dmath}\label{eqn:maxwellsEquations:680}
\spacegrad \cdot \BS
=
\gpgradezero
{
\lr{ \inv{\epsilon} \rho + I \BM + \mu \PD{t}{(I\BH)} } I \BH
+
\BE
\lr
{
\frac{I \rho_\txtm}{\mu} - \BJ - \epsilon \PD{t}{\BE}
}
}
,
\end{dmath}

or
%\begin{dmath}\label{eqn:maxwellsEquations:700}
\boxedEquation{eqn:maxwellsEquations:720}{
0 =
\spacegrad \cdot \BS
+
\BH \cdot \BM + \BJ \cdot \BE
+ \PD{t}{\BB} \cdot \BH
+ \PD{t}{\BD} \cdot \BE.
}
%\end{dmath}

The last two terms is the time rate of change of the energy density.  To illustrate this consider the change of energy density through a volume with neither electric nor magnetic current sources in that region of space

\begin{dmath}\label{eqn:maxwellsEquations:740}
\PD{t}{} \int_V
\inv{2} dV \lr{
\BB \cdot \BH
+ \BD \cdot \BE
}
=
-\int_{\partial V} dA \ncap \cdot \BS.
\end{dmath}

Here \( \ncap \) is the outward normal, so if the energy contained in the volume is decreasing, then \( \BS \) must represent the energy per unit area that leaves the volume.  The direction of the Poynting vector is the direction that the energy is leaving the volume.  Only the components of the Poynting vector that are colinear with the surface normal will result in energy leaving or entering the volume.

\section{Complex power}

The time harmonic expansion of the Poynting vector is

\begin{dmath}\label{eqn:maxwellsEquations:820}
\BS
= (I \BH) \cdot \BE
=
(I \Real \lr{ \BH e^{j \omega t}} ) \cdot
\Real \lr{ \BE e^{j \omega t} }
= \inv{4} \gpgradeone{
I \lr{
\BH e^{j \omega t}
+ \BH^\conj e^{-j \omega t}
}
\lr{ \BE e^{j \omega t} + \BE^\conj e^{-j \omega t} }
}
=
\inv{4}
\lr{
(I \BH^\conj) \cdot \BE
+(I \BH) \cdot \BE^\conj
+
(I \BH) \cdot \BE e^{ 2 j \omega t }
+
(I \BH^\conj) \cdot \BE^\conj e^{ 2 j \omega t }
}.
\end{dmath}

This shows that the time harmonic representation of the Poynting vector is

\begin{dmath}\label{eqn:maxwellsEquations:840}
\BS = \inv{2} \Real \lr{ (I \BH^\conj) \cdot \BE + (I \BH) \cdot \BE e^{ 2 j \omega t } }.
\end{dmath}

This motivates the definition of a complex Poynting vector

\begin{dmath}\label{eqn:maxwellsEquations:860}
\BS = \inv{2} (I \BH^\conj) \cdot \BE.
\end{dmath}

Note that this is conventionally written as \( \BS = (\ifrac{1}{2}) \BE \cross \BH^\conj \).  Observe that on average (over a given period), the \( e^{2 j \omega t} \) component of the real Poynting vector has no contribution to the power flux

\begin{dmath}\label{eqn:maxwellsEquations:880}
\inv{T} \int_0^T dt \BS = \Real \BS.
\end{dmath}
