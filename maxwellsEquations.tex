%
% Copyright © 2016 Peeter Joot.  All Rights Reserved.
% Licenced as described in the file LICENSE under the root directory of this GIT repository.
%
\section{Conventional differential form}

The differential form of Maxwell's equations, with extensions for magnetic sources, is the starting point for all the analysis in these notes.  Those equations are

\input{../ece1229-antenna/MaxwellsStatement.tex}

The magnetic sources can be considered fictional, but are useful for modelling real phenomina such as infinitesimal current loops, especially in antenna theory.

\input{../ece1229-antenna/MaxwellsFieldAndSourceDescription.tex}

The fields and sources are all real valued functions of both space and time.  In many situations it will be desirable to work with a time harmonic (frequency-domain phasor) form of Maxwell's equations.  In engineering, a time harmonic representation presumes that all sources and fields have a frequency dependence of the form
\index{time harmonic}

\begin{dmath}\label{eqn:maxwellsEquations:20}
\bcY(\Bx, t) = \Real( \BY(\Bx, \omega) e^{j\omega t} ),
\end{dmath}

where the field (or source) \( \BY(\Bx, \Bomega) \) is allowed to be complex valued.  Given this frequency dependence Maxwell's equations take the form

\input{../ece1229-antenna/MaxwellsTimeHarmonic.tex}

Note that the time harmonic convention typically used in physics literature presumes a frequency dependence of the form

\begin{dmath}\label{eqn:maxwellsEquations:40}
\bcY(\Bx, t) = \Real( \BY(\Bx, \omega) e^{-i\omega t} ),
\end{dmath}

which alters the sign of any imaginary originating from a time derivative.  Care is required by the reader to understand which form of frequency dependence has been assumed.

\section{GA differential form}

Geometric Algebra admits a number of alternative representations of Maxwell's equations.  The first follows from expressing the cross products all as wedge products, leaving a pair of bivector and a pair of scalar equations

\begin{subequations}
\begin{dmath}\label{eqn:maxwellsEquations:60}
\spacegrad \wedge \bcE = - I \bcM - \PD{t}{I\bcB}
\end{dmath}
\begin{dmath}\label{eqn:maxwellsEquations:80}
\spacegrad \wedge \bcH = I \bcJ + I \PD{t}{\bcD}
\end{dmath}
\begin{dmath}\label{eqn:maxwellsEquations:100}
\spacegrad \cdot \bcD = q_\txte
\end{dmath}
\begin{dmath}\label{eqn:maxwellsEquations:120}
\spacegrad \cdot \bcB = q_\txtm.
\end{dmath}
\end{subequations}

Alternatively, the duality transformation \( \Ba \wedge \Bb = -I \Ba \cdot (I \Bb) \) allows Maxwell's equations to be all written as dot products

\begin{subequations}
\begin{dmath}\label{eqn:maxwellsEquations:140}
\spacegrad \cdot (I \bcE) = \bcM + \PD{t}{\bcB}
\end{dmath}
\begin{dmath}\label{eqn:maxwellsEquations:160}
\spacegrad \cdot (I \bcH) = -\bcJ - \PD{t}{\bcD}
\end{dmath}
\begin{dmath}\label{eqn:maxwellsEquations:180}
\spacegrad \cdot \bcD = q_\txte
\end{dmath}
\begin{dmath}\label{eqn:maxwellsEquations:200}
\spacegrad \cdot \bcB = q_\txtm,
\end{dmath}
\end{subequations}

or, using the duality transformation \( \Ba \cdot \Bb = -I (\Ba \wedge (I \Bb) \), Maxwell's equations can all be written as wedge products

\begin{subequations}
\begin{dmath}\label{eqn:maxwellsEquations:220}
\spacegrad \wedge \bcE = - I \bcM - \PD{t}{I\bcB}
\end{dmath}
\begin{dmath}\label{eqn:maxwellsEquations:240}
\spacegrad \wedge \bcH = I \bcJ + I \PD{t}{\bcD}
\end{dmath}
\begin{dmath}\label{eqn:maxwellsEquations:260}
\spacegrad \wedge (I\bcD) = I q_\txte
\end{dmath}
\begin{dmath}\label{eqn:maxwellsEquations:280}
\spacegrad \wedge (I\bcB) = I q_\txtm.
\end{dmath}
\end{subequations}

Each of these forms can be useful in different circumstances, however the real power of GA in electromagnetism follows from presuming constituative relationships between the pairs of electric and magnetic fields

\begin{subequations}
\label{eqn:maxwellsEquations:300}
\begin{dmath}\label{eqn:maxwellsEquations:320}
\bcB = \mu \bcH
\end{dmath}
\begin{dmath}\label{eqn:maxwellsEquations:340}
\bcD = \epsilon \bcE,
\end{dmath}
\end{subequations}

where \( \epsilon \) is the permitivitity of the medium [\si{F/m}] (Farads/meter), and \( \mu \) is the permeability of the medium [\si{H/m}] (Henries/meter).
The permitivitity and permeability may be functions of both time and position, and model the materials that the fields are propagating through.  In general, the these may be non-isotropic tensor operators, however, unless otherwise specified, isotropic media will be assumed in these notes.

With this constitutative relationship assumed (and a bit of rescaling), the dot and wedge products of \cref{eqn:maxwellsEquations:60}, \cref{eqn:maxwellsEquations:100} can be added, as can those of \cref{eqn:maxwellsEquations:80}, \cref{eqn:maxwellsEquations:120}.  This reduces Maxwell's equations to a pair of first order coupled gradient equations

\begin{subequations}
\begin{dmath}\label{eqn:maxwellsEquations:360}
\spacegrad \bcE = \inv{\epsilon} q_\txte - I \bcM - \mu \PD{t}{(I\bcH)}
\end{dmath}
\begin{dmath}\label{eqn:maxwellsEquations:380}
\spacegrad (I \bcH) = \frac{I q_\txtm}{\mu} - \bcJ - \epsilon \PD{t}{\bcE}.
\end{dmath}
\end{subequations}

Note that it is more natural to work with a bivector magnetic field \( I \bcH \) in GA, than it is to work with a vector field \( \bcH \).  Observe that, when magnetic sources are included, this pair of coupled equations have sources of each grade (scalar, vector, bivector, and pseudoscalar).

The multivector equation \cref{eqn:maxwellsEquations:360} has grades 0,2 (scalar and bivector), whereas the multivector equation \cref{eqn:maxwellsEquations:380} has grades 1,3 (vector, pseudoscalar).  This means that arbitrary linear combinations of these equations, such as \( \spacegrad (a \bcE + b I \bcH ) \), are possible without any loss of information, since the original equations can then be recovered by grade selection.  To determine a desirable scaling of such a sum, these equations can be non-dimensionalized by expressing the fields as \( \sqrt{\epsilon} \bcE, \sqrt{\mu} \bcH \)

\begin{subequations}
\begin{dmath}\label{eqn:maxwellsEquations:400}
\spacegrad \sqrt{\epsilon} \bcE = \inv{\sqrt{\epsilon}} q_\txte - I \sqrt{\epsilon} \bcM - \sqrt{\epsilon \mu} \PD{t}{(I\sqrt{\mu} \bcH)}
\end{dmath}
\begin{dmath}\label{eqn:maxwellsEquations:420}
\spacegrad (I \sqrt{\mu} \bcH) = \frac{I q_\txtm}{\sqrt{\mu}} - \sqrt{\mu} \bcJ - \sqrt{\epsilon\mu} \PD{t}{\sqrt{\epsilon} \bcE}.
\end{dmath}
\end{subequations}

The dimensions of both differential operators are now equal \( [\spacegrad] = [\sqrt{\epsilon\mu} \PDi{t}{}] = 1/L \), allowing the remaining two multivector Maxwell equations to be decoupled into a single first order equation to solve for the multivector field \( \sqrt{\epsilon} \bcE + I \sqrt{\mu} \bcH \)

\begin{dmath}\label{eqn:maxwellsEquations:440}
\lr{ \spacegrad + \sqrt{\epsilon\mu} \PDi{t}{} }
\lr{ \sqrt{\epsilon} \bcE
\pm
I \sqrt{\mu} \bcH
}
=
\inv{\sqrt{\epsilon}} q_\txte
- I \sqrt{\epsilon} \bcM
\pm \lr{
+ \frac{I q_\txtm}{\sqrt{\mu}}
- \sqrt{\mu} \bcJ
}
.
\end{dmath}

Whether or not to add or subtract is essentially a phase choice for the electric field relative to the magnetic field.  It is conventional to pick the sum rather than the difference.  In engineering, with \( \bcE \) and \( \bcH \) as the primary fields, Maxwell's equation can now be expressed in its multivector form

\boxedEquation{eqn:maxwellsEquations:460}{
\begin{aligned}
\bcF &= \bcE + \eta I \bcH \\
\lr{ \spacegrad + \inv{v} \PD{t}{} } \bcF
&=
\inv{\epsilon v} \lr{ v q_\txte - \bcJ }
+ I \lr{ v q_\txtm - \bcM }
,
\end{aligned}
}

where \( \eta = \sqrt{\mu/\epsilon} \) (\( [\Omega] \) Ohms)
is the impedance of the media
, and \( v = 1/\sqrt{\epsilon\mu} \)
([\si{m/s}] meters/second)
is the group velocity of a wave in the media

\section{Wave equation.}



\section{Plane waves.}


