%
% Copyright © 2016 Peeter Joot.  All Rights Reserved.
% Licenced as described in the file LICENSE under the root directory of this GIT repository.
%
\makeproblem{Dot product of vectors as symmetric sum}{problem:gradeselection:dotprod}{
Show that the dot product of two vectors can be written as a symmetric sum

\begin{dmath}\label{eqn:gradeselection:600}
\Bx \cdot \By = \inv{2} \lr{ \Bx \By + \By \Bx }.
\end{dmath}
} % problem

\makeanswer{problem:gradeselection:dotprod}{
There are a few ways that this can be demonstrated.  The first relies on the classical definition of the dot product.  Expanding the square of a vector sum gives

\begin{dmath}\label{eqn:gradeselectionProblems:700}
(\Bx + \By)^2 = \Bx^2 + \By^2 + \Bx \By + \By \Bx.
\end{dmath}

By comparison this must also be equal to

\begin{dmath}\label{eqn:gradeselectionProblems:720}
\Norm{\Bx + \By}^2 = \Bx^2 + \By^2 + 2 \Bx \cdot \By,
\end{dmath}

so
\begin{dmath}\label{eqn:gradeselectionProblems:740}
\Bx \By + \By \Bx = 2 \Bx \cdot \By.
\end{dmath}

This might be viewed as a cheat, since it is not using the dot product as defined by grade zero selection according to \cref{dfn:gradeselection:100}.  Using that definition will produce the same result

\begin{dmath}\label{eqn:gradeselectionProblems:760}
\gpgradezero{ (\Bx + \By)^2 }
=
\gpgradezero{ \Bx^2 + \By^2 + \Bx \By + \By \Bx }
=
\Bx^2 + \By^2
+
\gpgradezero{
\Bx \By }
+ \gpgradezero{ \By \Bx }.
\end{dmath}

It was shown in \cref{problem:gradeselection:cyclicpermutationtwo} that \( \gpgradezero{ \Bx \By } = \gpgradezero{ \By \Bx } \) so
\begin{dmath}\label{eqn:gradeselectionProblems:780}
2 \gpgradezero{ \Bx \By } = \Bx \By + \By \Bx.
\end{dmath}

Using \cref{dfn:gradeselection:100}, this completes the problem.
} % answer
