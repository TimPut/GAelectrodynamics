%
% Copyright © 2017 Peeter Joot.  All Rights Reserved.
% Licenced as described in the file LICENSE under the root directory of this GIT repository.
%
Maxwell's equations provide an abstraction, the field, that aggregates the effects of an arbitrary electric charge and current
distribution on a ``test'' charge distribution.
The test charge is assumed to be small and isolated enough that it does not also appreciably change the fields themselves.
Once the fields are determined, the Lorentz force equation can be used to determine the dynamics of the
test particle.  These dynamics can be determined without having to
compute all the interactions of that charge with all the charges and currents in space, nor having to continually account for
the interactions of those charge with each other.

We will start with the traditional vector differential form of Maxwell's equations in free space

\begin{subequations}
\label{eqn:freespace:99}
\begin{dmath}\label{eqn:freespace:100}
\spacegrad \cross \BE = - \PD{t}{\BB}
\end{dmath}
\begin{dmath}\label{eqn:freespace:120}
\spacegrad \cross \BB = \mu_0 \lr{ \BJ + \epsilon_0 \PD{t}{\BE} }
\end{dmath}
\begin{dmath}\label{eqn:freespace:140}
\spacegrad \cdot \BE = \frac{\rho}{\epsilon_0}
\end{dmath}
\begin{dmath}\label{eqn:freespace:160}
\spacegrad \cdot \BB = 0.
\end{dmath}
\end{subequations}

These equations relate the electric and magnetic fields

\begin{itemize}
	\item \( \BE(\Bx, t) \) : Electric field intensity [\si{V/m}] (Volts/meter)
	\item \( \BB(\Bx, t) \) : Magnetic flux density [\si{W/m^2}] (Webers/square meter)
\end{itemize}

to charge and current densities

\begin{itemize}
	\item \( \rho(\Bx, t) \) : Electric charge density [\si{C/m^3}] (Coulombs/cubic meter)
	\item \( \BJ(\Bx, t) \) : Electric current density [\si{A/m^2}] (Amperes/square meter),
\end{itemize}

both of which can vary in space and time, and are specified here in SI units.  The constants are

\begin{itemize}
\item \( \epsilon_0 = 8.85 \times 10^{-12} \si{C^2/N/m^2}\) : Permutivity of free space (Coulombs squared/Newton/square meter)
\item \( \mu_0 = 4 \pi \times 10^{-7} \si{N/A^2}\) : Permeability of free space (Newtons/Ampere-squared).
\end{itemize}

These constants are related to the speed of light, \( c = 3.00 \times 10^8 \si{m/s} \) by \( \mu_0 \epsilon_0 = 1/c^2 \).

Continuous models for charge and current distributions are used in Maxwell's equations, despite the
fact that charges (i.e. electrons) are particles, and are not distributed in space.
The discrete nature of electronic charge can be modelled using a delta function representation of the charge and current densities

\begin{dmath}\label{eqn:freespace:240}
\begin{aligned}
\rho(\Bx, t) &= \sum_a q_a \delta( \Bx - \Bx_a( t) ) \\
\BJ(\Bx, t) &= \sum_a q_a \Bv_a \delta( \Bx - \Bx_a( t) ).
\end{aligned}
\end{dmath}

This model is inherently non-quantum mechanical, as it assumes that it is possible to
simultaneous measure the position and velocity of an electron.
%Additionally, irrespective of an electron's wave function distribution, this model requires that all electron interactions occur at fixed points in space and time.

The dynamics of particle interaction with the fields are provided by the
Lorentz force and power equations

\begin{subequations}
\label{eqn:freespace:180}
\begin{dmath}\label{eqn:freespace:200}
\ddt{\Bp} = q \lr{ \BE + \Bv \cross \BB }
\end{dmath}
\begin{dmath}\label{eqn:freespace:220}
\ddt{\calE} = q \BE \cdot \Bv.
\end{dmath}
\end{subequations}

The quantities involved are

\begin{itemize}
	\item \( \Bp(\Bx, t) \) : Test particle momentum [\si{kg\, m/s}] (Kilogram meters/second)
	\item \( \calE(\Bx, t) \) : Test particle kinetic energy [\si{J}] (Joules, kilogram meter^2/second^2)
	\item \( q \) : Test particle charge [\si{C}] (Coulombs)
	\item \( \Bv \) : Test particle velocity [\si{m/s}] (Meters/second)
\end{itemize}

As Maxwell's equations are inherently relativistic, and
energy and momentum are intimately coupled in special relativity,
the Lorentz power equation has been included here for completeness, although it can be argued that this is redundant
(\cref{problem:freespace:LorentzPower}).

\makeoproblem{Lorentz power and force relationship.}{problem:freespace:LorentzPower}{\S 17 \citep{landau1980classical}}{
Using the relativistic definitions of momentum and energy
\begin{equation*}
\begin{aligned}
	\Bp(\Bx, t) &= \frac{m \Bv}{\sqrt{1-\Bv^2/c^2}} \\
	\calE(\Bx, t) &= \frac{m c^2}{\sqrt{1-\Bv^2/c^2}},
\end{aligned}
\end{equation*}
show that \( d\calE/dt = \Bv \cdot d\Bp/dt \), and use this to derive
\cref{eqn:freespace:220} from \cref{eqn:freespace:200}.
} % problem
