%
% Copyright � 2018 Peeter Joot.  All Rights Reserved.
% Licenced as described in the file LICENSE under the root directory of this GIT repository.
%
%{
%
% Copyright © 2018 Peeter Joot.  All Rights Reserved.
% Licenced as described in the file LICENSE under the root directory of this GIT repository.
%
%\input{../latex/blogpost.tex}
\newcommand{\authorname}{Peeter Joot}
\newcommand{\email}{peeter.joot@mail.utoronto.ca, 920798560}
\newcommand{\basename}{FIXMEbasenameUndefined}
\newcommand{\dirname}{notes/FIXMEdirnameUndefined/}
\renewcommand{\basename}{ece2500report}
%\renewcommand{\dirname}{notes/phy1520/}
\renewcommand{\dirname}{notes/ece1228-electromagnetic-theory/}
%\newcommand{\dateintitle}{}
%\newcommand{\keywords}{}

% too many alphabets fix:
% https://tex.stackexchange.com/a/243541/15
\newcommand\hmmax{0}
\newcommand\bmmax{0}

%\input{../latex/peeter_prologue_print2.tex}
\newcommand{\onlineurl}{http://peeterjoot.com/archives/math2018/\basename.pdf}
\newcommand{\sourcepath}{\dirname\basename.tex}
\newcommand{\generatetitle}[1]{\chapter{#1}}

% this has a different implementation in various thisbook.sty's:
\newcommand{\underlineAndIndex}[1]{\underline{#1}}
\newcommand{\paragraphAndIndex}[1]{\paragraph{#1}}

\newcommand{\vcsinfo}{%
\section*{}
\noindent{\color{DarkOliveGreen}{\rule{\linewidth}{0.1mm}}}
\paragraph{Document version}
%\paragraph{\color{Maroon}{Document version}}
{
\small
\begin{itemize}
\item Available online at:\\
\href{\onlineurl}{\onlineurl}
\item Git Repository: \input{./.revinfo/gitRepo.tex}
\item Source: \sourcepath
\item last commit: \input{./.revinfo/gitCommitString.tex}
\item commit date: \input{./.revinfo/gitCommitDate.tex}
\end{itemize}
}
}

%\PassOptionsToPackage{dvipsnames,svgnames}{xcolor}
\PassOptionsToPackage{square,numbers}{natbib}

\documentclass{scrreprt}
%\documentclass[12pt]{scrreprt}
%\usepackage[left=1.2cm,right=1.2cm,top=1.5cm,bottom=1.5cm]{geometry}
%\usepackage[left=1.5cm,right=1.5cm,top=1.5cm,bottom=1.5cm]{geometry}
%\usepackage[top=1.5cm,bottom=1.5cm]{geometry}
%\usepackage[left=3cm,right=3cm,top=4cm,bottom=4cm]{geometry}
\usepackage[left=2.4cm,right=2.4cm,top=2.9cm,bottom=2.9cm]{geometry}
%\usepackage[left=1.9cm,right=1.9cm,top=2.2cm,bottom=2.2cm]{geometry}
%\usepackage{geometry}

%\documentclass{scrreprt}
%\usepackage[left=2cm,right=2cm]{geometry}

%\documentclass[10pt]{scrreprt}
%\usepackage[left=2cm,right=2cm]{geometry}

\usepackage[svgnames]{xcolor}
%\usepackage[english]{cleveref}
\usepackage{peeters_layout}
%\usepackage{peeters_layout_exercise}

\usepackage{natbib}

\usepackage[
colorlinks=true,
bookmarks=false,
pdfauthor={\authorname, \email},
backref
]{hyperref}

% http://tex.stackexchange.com/questions/75773/how-to-reference-problems-by-the-text-label-in-an-exercise-envioronment
\usepackage[english]{cleveref}
\crefname{Exercise}{exercise}{exercises}
\Crefname{Exercise}{Exercise}{Exercises}

\crefname{theorem}{theorem}{theorems}
\Crefname{theorem}{Theorem}{Theorems}

\crefname{lemma}{lemma}{lemmas}
\Crefname{lemma}{Lemma}{Lemmas}

\usepackage{tikz}

\RequirePackage{titlesec}
\RequirePackage{ifthen}

% http://stackoverflow.com/questions/4932910/date-in-the-tabular-environment
\makeatletter
\let\insertdate\@date
\makeatother

\titleformat{\chapter}[display]
{\bfseries\Large}
{\color{DarkSlateGrey}\filleft \authorname
\ifthenelse{\isundefined{\studentnumber}}{}{\\ \studentnumber}
\ifthenelse{\isundefined{\email}}{}{\\ \email}
\ifthenelse{\isundefined{\dateintitle}}{}{\\ \insertdate}
%\ifthenelse{\isundefined{\coursename}}{}{\\ \coursename} % put in title instead.
}
{4ex}
{\color{DarkOliveGreen}{\titlerule}\color{Maroon}
\vspace{2ex}%
\filright}
[\vspace{2ex}%
\color{DarkOliveGreen}\titlerule
]

\newcommand{\beginArtWithToc}[0]{\begin{document}\tableofcontents}
\newcommand{\beginArtNoToc}[0]{\begin{document}}
\newcommand{\EndNoBibArticle}[0]{\end{document}}
\newcommand{\EndArticle}[0]{\bibliography{Bibliography}\bibliographystyle{plainnat}\end{document}}

%
%\newcommand{\citep}[1]{\cite{#1}}

\colorSectionsForArticle

% FIXME: this doesn't work anymore, now that peeters_layout_exercise.sty is pulled out.
%% (a), (b), (c), ... numbering in ex (exercise pulled in by peeters_layout)
%\renewcommand{\QuestionNB}{\alph{Question}.\ }
%\renewcommand{\theQuestion}{\alph{Question}}

% have \cref fixup in book_layout.sty.  Use this macro instead:
\newcommand{\eqnref}[1]{eq. (\ref{#1})}
\newcommand{\Eqnref}[1]{Eq. (\ref{#1})}

%
% example:
%\statmechchapcite{nonIntegralBinomialSeries}
\newcommand{\quantumsolidschapcite}[1]{\citep{phy487:#1}}
%\newcommand{\statmechchapcite}[1]{\citep{phy452:#1}}
%\newcommand{\statmechchapcite}[1]{\cref{chap:#1}}

\usepackage{peeters_layout_exercise}
\usepackage{peeters_braket}
\usepackage{peeters_figures}

%\newcommand{\dAlembertian}[0]{\square}
\newcommand{\dAlembertian}[0]{\Box}

\newcommand{\dispdot}[2][.2ex]{\dot{\raisebox{0pt}[\dimexpr\height+#1][\depth]{$#2$}}}% \dispdot[<displace>]{<stuff>}
\newcommand{\dotBJ}[0]{\dispdot{\mathbf{J}}}

\newcommand{\stgrad}[0]{\lr{ \spacegrad + \inv{c} \PD{t}{}}}
\newcommand{\conjstgrad}[0]{\lr{ \spacegrad - \inv{c} \PD{t}{}}}
\newcommand{\stgradi}[0]{\lr{ \spacegrad + (1/c) \PDi{t}{}}}
\newcommand{\conjstgradi}[0]{\lr{ \spacegrad - (1/c) \PDi{t}{}}}

\usepackage{siunitx}
%\usepackage{mhchem} % \ce{}
\usepackage{macros_bm} % \bcM
\usepackage{macros_cal}
%\usepackage{macros_qed} % \qedmarker
\usepackage{txfonts} % \ointclockwise
\usepackage{enumerate}
\usepackage{mmacells}



\beginArtNoToc

\generatetitle{Project report ECE2500.  Geometric Algebra for Electrical Engineers}

\section{Motivation.}
This is the report for an ECE2500 M.Eng project course.

\subsection{Goals.}
This project had a few goals
\begin{enumerate}
\item Perform a literature review of applications of geometric algebra
to the study of electromagnetism.
\item Identify the subset of the literature that had direct relevance to electrical engineering.
\item Create a complete, and as compact as possible, introduction of the prerequisites required
for a graduate or advanced undergraduate electrical engineering student to be able to apply
geometric algebra to problems in electromagnetism.
%\item Develop some computer algebra methods for geometric algebra manipulation.
\end{enumerate}

\subsection{What is geometric algebra?}

In a geometric sense, vector algebra provides a representation of directed line segments.
Geometric algebra generalizes vector algebra and also provides an algebraic representation of points, plane segments, volumes, and higher degree geometric objects (hypervolumes.)
The geometric algebra representation of planes, volumes and hypervolumes requires a vector dot product, a vector multiplication operation, and a generalized addition operation.
The dot product provides the length of a vector and a test for vector perpendicularity.
The vector multiplication operation is used to construct
directed plane segments (bivectors),
and directed volumes (trivectors), which are built from the respective products of two or three mutually perpendicular vectors.
The geometric algebra
sum of scalars, vectors, or products of vectors is called a multivector.

The power to multiply and add scalars, vectors, and products of vectors can be exploited to simplify much of electromagnetism.
In particular, Maxwell's equations for isotropic media can be merged into a single multivector equation
\begin{dmath}\label{eqn:quaternion2maxwellWithGA:20}
\stgrad \lr{ \BE + I \eta \BH } = \eta\lr{ c \rho - \BJ },
\end{dmath}
where \( \spacegrad \) is the gradient, \( I = \Be_1 \Be_2 \Be_3 \) is the ordered product of the three \R{3} basis vectors, \( c = 1/\sqrt{\mu\epsilon}\) is the group velocity of the medium, \( \eta = \sqrt{\mu/\epsilon} \) is the impedance of the medium, \( \BE, \BH \) are the electric and magnetic fields, and
\( \rho \) and \( \BJ \) are the charge and current densities.
We will write this as
\begin{dmath}\label{eqn:ece2500report:40}
\stgrad F = J,
\end{dmath}
where \( F = \BE + I \eta \BH \) is the combined (multivector) electromagnetic field, and \( J = \eta\lr{ c \rho - \BJ } \) is the multivector current.
The complex interdependencies of Maxwell's equations require the student to learn a variety of special techniques and tricks to find solutions.
With Maxwell's equations reduced a single equation, we may throw the
complete Green's function toolbox into the fight, and directly
invert Maxwell's equation for \( F \) for any charge and current density distribution by convolving with the Green's function for the operator \( \stgradi \) as follows
\begin{dmath}\label{eqn:ece2500report:60}
F(\Bx, t)
= \int dt' dV' G(\Bx, \Bx' ; t, t') J(\Bx', t').
\end{dmath}
Green's functions may also be applied to static and frequency domain field configurations.

Working with the combined field \( F \) shows the
hidden structure behind a number of seemingly
disparate ideas in electromagnetism.
This project explored a number of ideas along these lines.
For example, a
Green's function solutions for the static field configurations simultaneously yields Coulomb's and the Biot-Savart law.
Plane, circular and elliptical waves may be expressed compactly in a multivector form, naturally expressing the mutual perpendicularity of the electric field, magnetic field and the propagation directions, as well
as the relationships between the electric and magnetic field components.
The field energy density and Poynting vectors have a simple multivector form expressed in terms of \( F \) alone.
Calculations of radiation pressure can be performed using only the normal component of what is known as the energy momentum tensor in the conventional representation, which has a particularly compact multivector
representation.

It is my belief that systematically working through a variety of introductory and advanced topics in electromagnetism using geometric algebra would provide significant insight, and would provide the electrical engineer or physics practitioner a powerful new set of tools and procedures.

\subsection{Results.}

Much of the geometric algebra literature for electrodynamics is presented with a relativistic bias, or assumes high levels of
physics and
mathematical
sophistication.
The aim of this work was an attempt to make the study of electromagnetism using geometric algebra more accessible, especially to an electrical engineering audience.
In particular, this project explored non-relativistic applications of geometric algebra to electromagnetism.
The end product of this project was a fairly small self contained book, titled ``Geometric Algebra for Electrical Engineers''.
This book includes an introduction to Euclidean geometric algebra focused on \R{2} and \R{3} (64 pages), an introduction to geometric calculus and multivector Green's functions (64 pages), applications to electromagnetism (82 pages), plus some appendices.

Many of the fundamental results of electromagnetism are derived directly from
\cref{eqn:ece2500report:40}, the multivector Maxwell's equation, in a streamlined and compact fashion.
This includes some new results, and many of the existing non-relativistic results from the geometric algebra literature.
As a
conceptual bridge, the book includes many examples of how to extract
familiar conventional results from simpler multivector representations.
Also included in the book are some sample calculations exploiting unique capabilities that geometric algebra provides.
In particular, vectors in a plane may be manipulated much like complex numbers, which has a number of advantages over working with coordinates explicitly.

The book produced in this project provides the prerequisite material for such exploration, and some first steps along the path of such an expedition.

\section{This report.}

This report summarizes some results from the book, omitting all figures, citations, Mathematica listings\footnote{Using Mathematica modules developed in this project, and other existing Mathematica modules.}, details from worked examples, and (most) explainations and derivations.
This report attempts to provide an overview that may be used as a road map for the book for further exploration.
While considerable work was done to make the book pedagodically complete, that is not possible in this report due to space constraints.

\section{Chapter I: Geometric Algebra.}
Chapter I of the book provides a self contained introduction to geometric algebra, with all the prerequisites required for geometric calculus and the subsequent applications to electromagnetism.
Due to space constraints in this report, no attempt to summarize that introduction will be made here.

\section{Chapter II: Geometric Calculus.}

Chapter II had two goals.
The first goal was to present the theory of multivector integration along curves and surfaces (i.e. manifold calculus), and the second goal was to derive all the multivector Green's functions required for application to problems of electromagnetism.
The multivector integration theory incorporates all of vector calculus as well as the more abstract theory of differential forms or integration on manifolds, but also generalizes both.
The key ideas of this chapter will be summarized below.
The book contains derivations, many examples, figures, Mathematica listings, and detailed explaination of many of the ideas summarized here.

\subsection{Reciprocal frames.}
Most elementary texts on electromagnetism will state or derive Stokes theorem and the divergence theorem in their rectangular coordinate form.
The fundamental theorem of geometric calculus is simplest to prove in a the more general context of curvilinear coordinates.
This requires some additional mathematical baggage, one part of which is the reciprocal frame, a key concept to working with
curvilinear and non-orthonormal bases.

%
% Copyright � 2018 Peeter Joot.  All Rights Reserved.
% Licenced as described in the file LICENSE under the root directory of this GIT repository.
%
\makedefinition{Reciprocal frame}{dfn:reciprocal:frame}{
Given a basis for a subspace \( \setlr{ \Bx_1, \Bx_2, \cdots \Bx_n } \), not 
necessarily orthonormal, the reciprocal frame is defined as the set of vectors \( \setlr{ \Bx^1, \Bx^2, \cdots \Bx^n } \) satisfying
\begin{dmath*}
\Bx_i \cdot \Bx^j = {\delta_i}^j,
\end{dmath*}
where the vector \( \Bx^j \) is not the j-th power of \( \Bx \), but is a superscript index, the conventional way of denoting a reciprocal frame vector, and \( {\delta_i}^j \) is the Kronecker delta.
} % definition


Any orthonormal basis is also its own reciprocal frame (or basis).
Mixed index (upper and lower) variables are used when working with curvilinear coordinates.
Summation (Einstein) convention was not used in the book, so sums over matched pairs of upper and lower indexes, are
marked with an explicit summation symbol.  Since geometric algebra identities are often coordinate free, this is not a terrible imposition.

A vector may be expressed in terms of either the curvilinear or reciprocal basis
\begin{equation}\label{eqn:ece2500report:2600}
\Ba
= \sum_i \lr{ \Ba \cdot \Bx^i } \Bx_i
= \sum_i \lr{ \Ba \cdot \Bx_i } \Bx^i.
\end{equation}
Either basis may be used to compute coordinates \( a_i = \Ba \cdot \Bx_i, a^i = \Ba \cdot \Bx^i \).  Coordinates are never used with an implied basis,
as is typical in both
engineering (column representation of vectors), and
relativisitic physics (four-vectors), so we have
no need of the covariant nor contravariant tensor nomenclature despite using mixed index representations.
The book contains a great deal of material to help the reader become familiar with curvilinear coordinates and reciprocal frames, including methods of computation, figures, hand calculations, and Mathematica listings.

\subsection{Curvilinear coordinates and hypervolume elements.}
The book provides a laymans introduction to manifold calculus in its geometric algebra form, where a manifold is, loosely speaking,
a connected and orientable subspace defined by a vector parameterization \( \Bx = \Bx(u_1, u_2, \cdots ) \).
Here connectivity is mentioned in passing, but is generally ignored in the book.
Infinitesimal partitioning (i.e. triangularization) is also generally ignored in the book, although some references are provided for further study by the interested student.

Given a manifold defined by a parameterization,
curvilinear basis elements are defined by partials with respect to the parameters \( \Bx_i = \PDi{u_i}{\Bx} \), as are the associated vector valued differentials \( d\Bx_i = \Bx_i du_i \).
Bivector valued area elements \( d^2 \Bx \equiv d\Bx_1 \wedge d\Bx_2 \) are formed by wedging the two vector differentials for a two parameter manifold, whereas trivector valued volume elements  \( d^3 \Bx \equiv d\Bx_1 \wedge d\Bx_2 \wedge d\Bx_3 \) are formed by wedging all three of the differentials for the parameterized space.
In general, it is assumed in the book that no (hyper)volume element \( d^k \Bx = d\Bx_1 \wedge \cdots \wedge d\Bx_k \) is ever zero (and thus never changes sign).
Such hypervolume elements have an orientation, as changing the order of any two differentials toggles the sign.
Orientability is also ignored in the book, and it should be assumed that the theorems stated do not hold for mobius-strip like manifolds.
For Euclidean spaces, the focus of the book, the absolute values of such area and volume differentials are \( dA = \sqrt{} -( d^2 \Bx)^2, dV = \sqrt{} -( d^3 \Bx)^2 \), which are the respective areas and volumes of the paralellogram and paralellopiped spanned by the vector differentials in each of the parameterization directions.

The required concepts are explained in detail in the book.
Many examples, figures, and Mathematica listings are used to facilitate that explaination.

\subsection{Gradient and vector derivative.}

The final tool required to formulate multivector integration theory is the vector derivative, the projection of the gradient onto the integration manifold.

%
% Copyright � 2018 Peeter Joot.  All Rights Reserved.
% Licenced as described in the file LICENSE under the root directory of this GIT repository.
%
\maketheorem{Reciprocal frame vectors}{thm:curvilinearGradient:1}{
Given a curvilinear basis with elements \( \Bx_i = \PDi{u_i}{\Bx} \), the \textit{reciprocal frame} vectors are given by
\begin{dmath*}
\Bx^i = \spacegrad u_i.
\end{dmath*}
} % theorem


%
% Copyright � 2018 Peeter Joot.  All Rights Reserved.
% Licenced as described in the file LICENSE under the root directory of this GIT repository.
%
\maketheorem{Curvilinear representation of the gradient}{thm:curvilinearGradient:2}{
Given an N-parameter vector parameterization
\( \Bx = \Bx(u_1, u_2, \cdots, u_N) \)
of \R{N},
with curvilinear basis elements \( \Bx_i = \PDi{u_i}{\Bx} \), the gradient can be expressed as
\begin{dmath*}
\spacegrad = \sum_i \Bx^i \PD{u_i}{}.
\end{dmath*}
It is often convenient to define \( \partial_i \equiv \PDi{u_i}{} \), so that the gradient can be expressed in mixed index representation
\begin{dmath*}
\spacegrad = \sum_i \Bx^i \partial_i.
\end{dmath*}
%or the same with sums over mixed indexes implied.
} % theorem


%
% Copyright � 2018 Peeter Joot.  All Rights Reserved.
% Licenced as described in the file LICENSE under the root directory of this GIT repository.
%
\makedefinition{Vector derivative}{dfn:gradient:100}{
Given an k-parameter vector parameterization
\( \Bx = \Bx(u_1, u_2, \cdots, u_k) \) of \R{N} with \( k \le N \),
and curvilinear basis elements \( \Bx_i = \PDi{u_i}{\Bx} \), the vector derivative \( \boldpartial \) is defined as
\begin{dmath*}
\boldpartial = \sum_{i=1}^k \Bx^i \partial_i.
\end{dmath*}
} % theorem


When the dimension of the subspace (number of parameters) equals the dimension of the underlying vector space, the vector derivative and gradient are identical.
In other situtations what is the vector derivative?  This can be answered by introducing the concept of the tangent space.
The tangent space \( T_\Bp \) is a curvilinear concept that represents the span of the differentials \( \Bx_i \) at the point of evaluation \( \Bp \).  For a curve this is the tangent line, and for a surface, this is the tangent plane.  Both are illustrated in the book.
The projection of any vector onto a k-dimensional tangent space with basis \( \setlr{\Bx_i} \) all evaluated at the point \( \Bp \) is just \(
\Proj_{T_\Bp} \Bf = \sum_{i = 1}^k \Bx^i (\Bx_i \cdot \Bf) \).
For the vector derivative, this is just
\begin{equation}\label{eqn:ece2500report:2640}
\Proj_{T_\Bp} \spacegrad
= \sum_{i = 1}^k \sum_{j = 1}^N \Bx^i (\Bx_i \cdot \Bx^j) \partial_j
= \sum_{i = 1}^k \sum_{j = 1}^N \Bx^i {\delta_i}^j \partial_j
= \sum_{i = 1}^k \Bx^i \partial_i,
\end{equation}
which is the vector derivative.
The vector derivative is the projection of the gradient onto the tangent space at the point of evaluation.

In the book many many examples, figures, and Mathematica calculations are used to illustrate and explain the vector derivative and associated curvilinear parameterizations.
This includes calculations of curvilinear coordinates, area and volume elements, and the vector derivative
for polar, spherical and a toroidal coordinate systems.

\subsection{Integration theory.}

The vector derivative
may not commute the functions it acts on nor a k-volume element \( d^k \Bx \), so we are forced to use some notation to indicate what the vector derivative (or gradient) acts on.
%
% Copyright � 2018 Peeter Joot.  All Rights Reserved.
% Licenced as described in the file LICENSE under the root directory of this GIT repository.
%
\makedefinition{Multivector k-volume integral.}{dfn:fundamentalTheoremOfCalculus:240}{
Given a hypervolume parameterized by \( k \) parameters, k-volume volume element \( d^k \Bx \), and
multivector functions \( F, G \), a k-volume integral with the vector derivative acting to the right on \( G \) is written as
\begin{equation*}
\int_V d^k\Bx \rboldpartial G,
\end{equation*}
a k-volume integral with the vector derivative acting to the left on \( F \) is written as
\begin{equation*}
\int_V F d^k\Bx \lboldpartial,
\end{equation*}
and a \textit{multivector k-volume integral} with the vector derivative acting bidirectionally on \( F, G \) is written as
\begin{equation*}
\int_V F d^k\Bx \lrboldpartial G
\equiv
\int_V \lr{ F d^k\Bx \lboldpartial} G
+
\int_V F d^k\Bx \lr{ \rboldpartial G }.
\end{equation*}
The explicit meaning of these directionally acting derivative operations is given by the following chain rule coordinate expansion
\begin{dmath*}
F d^k \Bx \lrboldpartial G
=
F d^k \Bx \lr{ \sum_i \Bx^i {\stackrel{ \leftrightarrow }{\partial_i}} } G
=
(\partial_i F) d^k \Bx \sum_i \Bx^i G
+
F d^k \Bx \sum_i \Bx^i (\partial_i G)
\equiv
(F d^k \Bx \lboldpartial) G
+
F d^k \Bx (\rboldpartial G),
\end{dmath*}
with \( \boldpartial \) acting on \( F \) and \( G \), but not the volume element \( d^k \Bx \), which may also be a function of the implied parameterization.
} % definition

In conventional right acting cases, where there is no ambiguity, arrows will usually be omitted, but braces may also be used to indicate the scope of derivative operators.
This bidirectional notation will also be used for the gradient, especially for volume integrals in \R{3} where the vector derivative is identitical to the gradient.
Some authors use the Hestenes dot notation, with overdots or primes to indicatating the exact scope of multivector derivative operators, as in
\begin{dmath}\label{eqn:fundamentalTheoremOfCalculus:260}
\dot{F} d^k \Bx \dot{\boldpartial} \dot{G} =
\dot{F} d^k \Bx \dot{\boldpartial} G
+
F d^k \Bx \dot{\boldpartial} \dot{G}.
\end{dmath}
The dot notation has the advantage of emphasizing that the action of the vector derivative (or gradient) is on the functions \( F, G \), and not on the hypervolume element \( d^k \Bx \).
However, in the book, where primed operators such as \( \spacegrad' \) are used to indicate that derivatives are taken with respect to primed \( \Bx' \) variables, a mix of dots and ticks would have been confusing.

The generalization of line, surface and volume integrals to hypervolumes and multivector functions can now be stated.
%
% Copyright � 2018 Peeter Joot.  All Rights Reserved.
% Licenced as described in the file LICENSE under the root directory of this GIT repository.
%
\maketheorem{Fundamental theorem of geometric calculus}{thm:fundamentalTheoremOfCalculus:1}{
Given
multivectors \(F, G \),
a parameterization \( \Bx = \Bx(u_1, u_2, \cdots) \), with hypervolume element \( d^k \Bx = d^k u I_k \), where
\( I_k = \Bx_1 \wedge \Bx_2 \wedge \cdots \wedge \Bx_k \), the hypervolume integral is related to the boundary integral by
\begin{equation*}
\int_V F d^k \Bx \lrboldpartial G = \int_{\partial V} F d^{k-1} \Bx G,
\end{equation*}
where \( \partial V \) represents the boundary of the volume, and \( d^{k-1} \Bx \) is the hypersurface element.
This is called the \textit{Fundamental theorem of geometric calculus}.

The hypersurface element and boundary integral is defined for \( k > 1 \) as
\begin{equation*}
\int_{\partial V} F d^{k-1} \Bx G
\equiv
\sum_i \int d^{k-1} u_i \evalbar{ \lr{ F \lr{ I_k \cdot \Bx^i} G }}{\Delta u_i},
\end{equation*}
where \( d^{k-1} u_i \) is the product of all \( du_j \) except for \( du_i \).
For
\( k = 1 \) the hypersurface element and associated
boundary ``integral''
is really just convenient general shorthand, and
should be taken to mean the evaluation of the \( F G \) multivector product over the range of the parameter
\begin{equation*}
\int_{\partial V} F d^{0} \Bx G
\equiv
\evalbar{ F G }{\Delta u_1}.
\end{equation*}
} % theorem


The fundamental theorem of geometric calculus is a generalization of many conventional scalar and vector integral theorems, and relates a hypervolume integral to its boundary.
This is a a powerful theorem, used in the book with Green's functions to solve Maxwell's equation.

The book includes a general k-volume proof of
\cref{thm:fundamentalTheoremOfCalculus:1} in the appendix.
Since most of the complexity of the proof is some hellish index manipulation required for the general k-volume case,
to aid accessibility, this theorem is stated and proven separately in the book for each of the line, surface, and volume integral cases.

\subsubsection{Stokes' theorem.}
An important consequence of the fundamental theorem of geometric calculus is the geometric algebra form of Stokes' theorem.
The Stokes' theorem that we know from conventional vector calculus relates \R{3} surface integrals to the line integral around a bounding surface.
The geometric algebra form of Stokes' theorem is equivalent to Stokes' theorem from the theory of differential forms, a more general theorem.
%, which relates
%hypervolume integrals of blades\footnote{Blades are isomorphic to the k-forms found in the theory of differential forms.} to the integrals over their hypersurface boundaries, a much more general result.

%
% Copyright � 2018 Peeter Joot.  All Rights Reserved.
% Licenced as described in the file LICENSE under the root directory of this GIT repository.
%
\maketheorem{Stokes' theorem}{thm:stokesTheoremGeometricAlgebra:1740}{
Given a \(k\) volume element \(d^k \Bx \) and an s-blade \( F, s < k \)
\begin{equation*}
\int_V d^k \Bx \cdot (\boldpartial \wedge F) = \int_{\partial V} d^{k-1} \Bx \cdot F.
\end{equation*}
}


In the book we will see that most of the well known scalar and vector integral theorems can easily be derived as direct consequences of \cref{thm:stokesTheoremGeometricAlgebra:1740}, itself a special case of \cref{thm:fundamentalTheoremOfCalculus:1}.
This report states those results for the special cases of line, area and volume integrals.

%
% Copyright � 2018 Peeter Joot.  All Rights Reserved.
% Licenced as described in the file LICENSE under the root directory of this GIT repository.
%
\maketheorem{Multivector line integral.}{thm:lineintegral:100}{
Given an connected curve \( C \) parameterized by a single parameter, and multivector functions \( F, G \), the line integral is related to the boundary by
\begin{equation*}
\int_C F d^1\Bx \boldpartial G
= \evalbar{F G}{\Delta a}.
\end{equation*}
} % theorem


%
% Copyright � 2018 Peeter Joot.  All Rights Reserved.
% Licenced as described in the file LICENSE under the root directory of this GIT repository.
%
\maketheorem{Line integral of a scalar function (Stokes').}{thm:lineintegral:180}{
Given a scalar function \( f \), its \textit{line integral} is given by
\begin{equation*}
\int_C d^1\Bx \cdot \boldpartial f =
\int_C d^1\Bx \cdot \spacegrad f = \evalbar{f}{\Delta a}.
\end{equation*}
} % theorem


%
% Copyright � 2018 Peeter Joot.  All Rights Reserved.
% Licenced as described in the file LICENSE under the root directory of this GIT repository.
%
\maketheorem{Multivector surface integral.}{thm:surfaceintegral:100}{
Given a connected surface \( S \) with an area element \( d^x \Bx = d\Bx_1 \wedge d\Bx_2 \), and multivector functions \( F, G \), the \textit{surface integral} is related to the boundary by
\begin{equation*}
\int_S F d^2\Bx \lrboldpartial G
= \ointclockwise_{\partial S} F d\Bx G,
\end{equation*}
where \( \partial S \) is the boundary of the surface \( S \).
} % theorem


%
% Copyright � 2018 Peeter Joot.  All Rights Reserved.
% Licenced as described in the file LICENSE under the root directory of this GIT repository.
%
\maketheorem{Surface integral of scalar function (Stokes').}{thm:surfaceintegral:420}{
Given a scalar function \( f(\Bx) \) its \textit{surface integral} is given by
\begin{equation*}
\int_S d^2 \Bx \cdot \boldpartial f =
\int_S d^2 \Bx \cdot \spacegrad f = \ointclockwise_{\partial S} d\Bx f.
\end{equation*}
In \R{3} this can be written as
\begin{equation*}
\int_S dA \ncap \cross \spacegrad f = \ointctrclockwise_{\partial S} d\Bx f,
\end{equation*}
where \( \ncap \) is the normal specified by \( d^2 \Bx = I \ncap dA \).
} % theorem


%
% Copyright � 2018 Peeter Joot.  All Rights Reserved.
% Licenced as described in the file LICENSE under the root directory of this GIT repository.
%
\maketheorem{Surface integral of a vector function (Stokes').}{thm:surfaceintegral:500}{
Given a vector function \( \Bf(\Bx) \) the \textit{surface integral} is given by
\begin{equation*}
\int_S d^2 \Bx \cdot (\spacegrad \wedge \Bf) = \ointclockwise_{\partial S} d\Bx \cdot \Bf.
\end{equation*}
In \R{3} this can be written as
\begin{equation*}
\int_S dA\, \ncap \cdot \lr{ \spacegrad \cross \Bf} = \ointctrclockwise_{\partial S} d\Bx \cdot \Bf,
\end{equation*}
where \( \ncap \) is the normal specified by \( d^2 \Bx = I \ncap dA \).
} % theorem


%
% Copyright � 2018 Peeter Joot.  All Rights Reserved.
% Licenced as described in the file LICENSE under the root directory of this GIT repository.
%
\maketheorem{Green's theorem.}{thm:surfaceintegral:620}{
Given a vector \( \Bf = \sum_i f_i \Bx^i \) in \R{N}, and a surface parameterized by \( \Bx = \Bx(u_1, u_2) \)
\begin{equation*}
\int_S du_1 du_2 \lr{ \PD{u_2}{f_1} - \PD{u_1}{f_2} }
=
\ointclockwise_{\partial S} du_1 f_1 + du_2 f_2.
\end{equation*}
This is
often stated for \R{2} with a Cartesian \(x,y\) parameterization, such as \( \Bf = P \Be_1 + Q \Be_2 \).
In that case
\begin{equation*}
\int_S dx dy \lr{ \PD{y}{P} - \PD{x}{Q} }
=
\ointclockwise_{\partial S} P dx + Q dy.
\end{equation*}
} % theorem


%
% Copyright � 2018 Peeter Joot.  All Rights Reserved.
% Licenced as described in the file LICENSE under the root directory of this GIT repository.
%
\maketheorem{Multivector volume integral.}{thm:volumeintegral:100}{
Given an connected volume \( V \) with a volume element \( d^3 \Bx = d\Bx_1 \wedge d\Bx_2 \wedge d\Bx_3 \), and multivector functions \( F, G \), a \textit{volume integral} can be reduced to a surface integral as follows
\begin{equation*}
\int_V F d^3\Bx \lrboldpartial G
= \ointctrclockwise_{\partial V} F d^2\Bx G,
\end{equation*}
where \( \partial V \) is the boundary of the volume \( V \), and \( d^2 \Bx \) is the counterclockwise oriented area element on the boundary of the volume.
In \R{3} with \( d^3 \Bx = I dV \), \( d^2 \Bx = I \ncap dA \), this integral can be written using a scalar volume element
\begin{equation*}
\int_V dV F \lrboldpartial G
= \int_{\partial V} dA F \ncap G,
\end{equation*}
} % theorem


%
% Copyright � 2018 Peeter Joot.  All Rights Reserved.
% Licenced as described in the file LICENSE under the root directory of this GIT repository.
%
\maketheorem{Volume integral of vector function (Stokes').}{thm:volumeintegral:1661}{
The specialization of \textit{Stokes' theorem} for a
volume integral of the (bivector) curl of a vector function \( \Bf(\Bx) \), relates the volume integral to a surface area
over the boundary as follows
\begin{equation*}
\int_V d^3 \Bx \cdot \lr{ \boldpartial \wedge \Bf } =
\int_V d^3 \Bx \cdot \lr{ \spacegrad \wedge \Bf } = \ointctrclockwise_{\partial V} d^2\Bx \cdot \Bf.
\end{equation*}
In \R{3}, this can be written as
\begin{equation*}
\int_V dV\, \spacegrad \cross \Bf = \int_{\partial V} dA\, \ncap \cross \Bf,
\end{equation*}
or with a duality transformation \( \Bf = I B \), where \( B \) is a bivector
\begin{equation*}
\int_V dV\, \spacegrad \cdot B = \int_{\partial V} dA\, \ncap \cdot B,
\end{equation*}
where \( \ncap \) is the normal specified by \( d^2 \Bx = I \ncap dA, \) and \( d^3 \Bx = I dV \).
} % theorem


%
% Copyright � 2018 Peeter Joot.  All Rights Reserved.
% Licenced as described in the file LICENSE under the root directory of this GIT repository.
%
\maketheorem{Volume integral of bivector function (Stokes', divergence).}{thm:volumeintegral:1681}{
Given a bivector function \( B(\Bx) \), the volume
integral of the (trivector) curl is related to a surface integral by
\begin{equation*}
\int_V d^3 \Bx \cdot \lr{ \boldpartial \wedge B } =
\int_V d^3 \Bx \cdot \lr{ \spacegrad \wedge B } = \ointctrclockwise_{\partial V} d^2\Bx \cdot B.
\end{equation*}
In \R{3}, this can be written as
\begin{equation*}
\int_V dV\, \spacegrad \wedge B = \int_{\partial V} dA\, \ncap \wedge B,
\end{equation*}
which yields the \textit{divergence theorem} after making a duality transformation \( B(\Bx) = I \Bf(\Bx) \), where \( \Bf \) is a vector, by
\begin{equation*}
\int_V dV\, \spacegrad \cdot \Bf = \int_{\partial V} dA\, \ncap \cdot \Bf,
\end{equation*}
where \( \ncap \) is the normal specified by \( d^2 \Bx = I \ncap dA, \) and \( d^3 \Bx = I dV \).
} % theorem


%
% Copyright � 2018 Peeter Joot.  All Rights Reserved.
% Licenced as described in the file LICENSE under the root directory of this GIT repository.
%
\maketheorem{Divergence theorem.}{thm:volumeintegral:2661}{
Given an \R{3} multivector \( M \) containing only grades 0,1, or 2
\begin{equation*}
\int_V dV \spacegrad \cdot M = \int_{\partial V} dA \ncap \cdot M,
\end{equation*}
where \( \ncap \) is the outwards normal to the surface bounding \( V \).
} % theorem


All of \cref{thm:stokesTheoremGeometricAlgebra:1740}-\cref{thm:volumeintegral:2661} are direct consequences (or specializations) of
\cref{thm:fundamentalTheoremOfCalculus:1}.

\subsection{Multivector Fourier transform and phasors.}
The book makes use of time harmonic (frequency domain) representations when convienient.
Provided we utilize a scalar (non-geometric) imaginary, a standard Fourier transform generalizes to multivectors.

%
% Copyright � 2018 Peeter Joot.  All Rights Reserved.
% Licenced as described in the file LICENSE under the root directory of this GIT repository.
%
\makedefinition{Multivector Fourier transform pairs}{dfn:greensFunctionOverview:280}{
The Fourier transform pair for a multivector valued function \( F(\Bx, t) \) will be written as
\begin{equation*}
\begin{aligned}
F(\Bx, t) &= \int F_\omega(\Bx) e^{j \omega t} d\omega \\
F_\omega(\Bx) &= \inv{2 \pi} \int F(\Bx, t) e^{-j \omega t} dt,
\end{aligned}
\end{equation*}
where \( j \) is an arbitrary scalar imaginary that commutes with all multivectors.
} % definition

By non-geometric, we mean that the imaginary need not have a geometric interpretation such as \( j = \Be_{12}, \Be_{123}, \cdots \).  Using a geometric representation of the imaginary requires special care as it may not commute with the multivector functions.  This is an active research topic in the literature, and was avoided in the book.

The book uses the engineering convention for
phasors, with a positive sign on the angular frequency complex exponentials (i.e. \( e^{j\omega t} \), not \( e^{-i \omega t} \)).

%
% Copyright � 2018 Peeter Joot.  All Rights Reserved.
% Licenced as described in the file LICENSE under the root directory of this GIT repository.
%
\makedefinition{Multivector phasor representation.}{dfn:greensFunctionOverview:300}{
The \textit{phasor representation} \( F(\Bx) \) of a multivector valued (real) function \( F(\Bx, t) \) is defined implicitly as
\begin{equation*}
F(\Bx, t) = \Real\lr{ F(\Bx) e^{j \omega t} },
\end{equation*}
where \( j \) is an arbitrary scalar imaginary that commutes with all multivectors.
} % definition


The complex valued multivector \( F(\Bx) \) is still generated from the real Euclidean basis for \R{3}, so there will be
no reason to introduce complex inner products spaces into the mix.

\subsection{Multivector Green's functions.}
An attempt to motivate the concept of a Green's function is provided in the book, omitted here.
The book specifies the chosen sign convention for the Green's functions used, as those vary in the literature.
Bounded vs. unbounded integration volumes and some theorems related to bounded applications of Green's functions (Green's theorem, and a related identity for the (first order) gradient) are also discussed.

%
% Copyright � 2018 Peeter Joot.  All Rights Reserved.
% Licenced as described in the file LICENSE under the root directory of this GIT repository.
%
\maketheorem{Green's function for the Helmholtz operator.}{thm:gradientGreensFunctionEuclidean:3}{
The advancing (causal), and the receding (acausal) \textit{Green's functions for the Helmholtz operator} satisfying
%\cref{eqn:greensFunctionHelmholtz:420} are respectively
\begin{equation*}
\lr{ \spacegrad^2 + k^2 } G(\Bx, \Bx') = \delta(\Bx - \Bx').
\end{equation*}
is
\begin{equation*}
\begin{aligned}
G_{\textrm{adv}}(\Bx, \Bx') &= -\frac{e^{-j k \Norm{ \Bx - \Bx' } }}{ 4 \pi \Norm{\Bx - \Bx'}} \\
G_{\textrm{rec}}(\Bx, \Bx') &= -\frac{e^{j k \Norm{ \Bx - \Bx' } }}{ 4 \pi \Norm{\Bx - \Bx'}}.
\end{aligned}
\end{equation*}
} % theorem


We will use the advancing (causal) Green's function, and refer to this function as \( G(\Bx, \Bx') \) without any subscript.
Observe that as a special case, the Helmholtz Green's function reduces to the Green's function for the Laplacian when \( k = 0 \)
\begin{dmath}\label{eqn:greensFunctionHelmholtz:80}
G(\Bx, \Bx') = -\inv{ 4 \pi \Norm{\Bx - \Bx'}}.
\end{dmath}

The Helmholtz operator can be factored in geometric algebra as \( \spacegrad^2 + k^2 = \lr{ \spacegrad \pm j k }\lr{ \spacegrad \mp j k } \).
We will call these factors first order Helmholtz operators, and derive their (multivector valued) Green's functions, which are as follows.
%
% Copyright � 2018 Peeter Joot.  All Rights Reserved.
% Licenced as described in the file LICENSE under the root directory of this GIT repository.
%
\maketheorem{Green's function for the first order Helmholtz operator.}{thm:gradientGreensFunctionEuclidean:720}{
The \textit{Green's function for the first order Helmholtz operator} \( \spacegrad + j k \) satisfies
\begin{equation*}
\lr{ \rspacegrad + j k } G(\Bx, \Bx') = G(\Bx, \Bx') \lr{ -\lspacegrad' + j k } = \delta(\Bx - \Bx'),
\end{equation*}
and has the value
\begin{equation*}
G(\Bx, \Bx') = \frac{e^{-j k r}}{4 \pi r} \lr{ j k \lr{ 1 + \rcap } + \frac{\rcap}{r} },
\end{equation*}
where \( \Br = \Bx - \Bx', r = \Norm{\Br} \) and \( \rcap = \Br/r \), and \( \spacegrad' \) denotes differentiation with respect to \( \Bx' \).
} % theorem


This theorem is proven in the book from the Green's function for the (second order) Helmholtz operator.
The solution of the
the first order Helmholtz system \( \lr{ \spacegrad + j k } F = J \) follows immediately
\begin{dmath}\label{eqn:greensFunctionFirstOrderHelmholtz:880}
F(\Bx)
=
\int_V G(\Bx, \Bx') J(\Bx') dV'
-
\int_{\partial V} G(\Bx, \Bx') \ncap' F(\Bx') dA'
+ F_0,
\end{dmath}
where \( F_0 \) is a solution to the homogeneous first order Helmholtz equation \( \lr{ \spacegrad + j k } F_0 = 0 \).
Given a ``well-behaved'' \( F(\Bx') \) the boundary term is required to vanish when the integral is taken to infinity, which leaves a solution that depends only on the sources though the convolution with the Green's function.

A special but important case of \cref{thm:gradientGreensFunctionEuclidean:720}
is the \( k = 0 \) condition, which provides the
Green's function for the gradient, which is vector valued
\begin{equation}\label{eqn:greensFunctionFirstOrderHelmholtz:900}
G(\Bx, \Bx' ; k = 0) = \inv{4 \pi} \frac{\rcap}{r^2}.
\end{equation}

For the wave equation operator, it is helpful to introduce a d'Lambertian operator.  Because the sign of this operator varies in the literature, we must state our choice explicitly.
%
% Copyright � 2018 Peeter Joot.  All Rights Reserved.
% Licenced as described in the file LICENSE under the root directory of this GIT repository.
%
\makedefinition{d'Lambertian (wave equation) operator.}{dfn:continuity:120}{
Let
\begin{equation*}
\dAlembertian =
\conjstgrad
\stgrad
=
\spacegrad^2 - \inv{c^2} \PDSq{t}{}.
\end{equation*}
} % definition


The d'Lambertian operator may be factored in geometric algebra as \( \dAlembertian = \conjstgradi \stgradi \).
The operator \( \stgradi \) will be called the \textit{spacetime gradient}\footnote{This form of spacetime gradient is given a special symbol (\(\gamma_0 \grad, \overbar{D}, \calD, \cdots\)) by a number of authors, but there is no general agreement on what to use.
Instead of entering the fight, it will be written it out in full in the book.}.

In the book, the Green's function for the spacetime gradient is derived from the from the Green's function for the d'Lambertian, also well known.
%
% Copyright � 2018 Peeter Joot.  All Rights Reserved.
% Licenced as described in the file LICENSE under the root directory of this GIT repository.
%
\maketheorem{Green's function for the spacetime gradient.}{thm:greensFunctionSpacetimeGradient:120}{
The \textit{Green's function for the spacetime gradient} \( \spacegrad + (1/c) \partial_t \) satisfies
\begin{equation*}
\stgrad G(\Bx - \Bx', t - t') = \delta(\Bx - \Bx') \delta(t - t'),
\end{equation*}
and has the value
\begin{equation*}
G(\Bx - \Bx', t - t')
=
\inv{4\pi} \lr{
- \frac{\rcap}{r^2} \PD{r}{}
+ \frac{\rcap}{r}
+ \inv{c r} \PD{t}{}
}
\delta( -r/c + t - t' ),
\end{equation*}
where \( \Br = \Bx - \Bx', r = \Norm{\Br} \) and \( \rcap = \Br/r \).
} % theorem

Like the Green's function for the first order Helmholtz operator, the Green's function for the spacetime gradient is also multivector.
These are all the Green's functions required for the electromagnetic solutions considered in chapter 3 of the book.

\subsection{Helmholtz theorem.}
In the book first and second order proofs of Helmholtz theorem are provided, both using geometric algebra techniques.
Disussion of those derivations are omitted here for brevity.
\section{Chapter III: Electromagnetism.}
\subsection{Conventional Maxwell's equations.}
In the book it is presumed that the reader is familiar with Maxwell's equations.
No attempt to motivate them was made.
The starting point was Maxwell's equations with antenna theory extensions (fictious magnetic sources)
\begin{dmath}\label{eqn:ece2500report:2540}
\begin{aligned}
\spacegrad \cross \BE &= - \BM - \PD{t}{\BB} \\
\spacegrad \cross \BH &= \BJ + \PD{t}{\BD} \\
\spacegrad \cdot \BD &= \rho \\
\spacegrad \cdot \BB &= \rho_\txtm.
\end{aligned}
\end{dmath}
where \( \BE, \BH, \BD, \BB \) are the conventional electric and magnetic field intensities and flux densities,
\( \rho, \rho_\txtm \) are the electric and (fictitious-)magnetic charge densities,
and \( \BJ, \BM \) are the electric and (fictitious-)magnetic current densities.

There is some limited discussion of the geometric algebra form of Maxwell's equations for more general media at the end of the book, however
much of the book presumes isotropic constitutive relationships between the electric and magnetic fields
\begin{dmath}
\label{eqn:freespace:300}
\begin{aligned}
\BB &= \mu \BH \\
\BD &= \epsilon \BE,
\end{aligned}
\end{dmath}
where \( \epsilon = \epsilon_r \epsilon_0 \) is the permittivity of the medium, and \( \mu = \mu_r \mu_0 \) is the permeability of the medium.
\subsection{Maxwell's equation.}
For isotropic media and constitutive relationships \cref{eqn:freespace:300} a multivector that includes both electric and magnetic fields is defined.
%
% Copyright � 2018 Peeter Joot.  All Rights Reserved.
% Licenced as described in the file LICENSE under the root directory of this GIT repository.
%
\makedefinition{Electromagnetic field strength.}{dfn:isotropicMaxwells:640}{
The \textit{electromagnetic field strength} ([\si{V/m}] (Volts/meter)) is defined as
\begin{equation*}
F = \BE + I \eta \BH \quad(= \BE + I c \BB),
\end{equation*}
where
\begin{itemize}
\item \( \eta = \sqrt{\mu/\epsilon} \) (\( [\Omega] \) Ohms), is the impedance of the media.
\item \( c = 1/\sqrt{\epsilon\mu} \) ([\si{m/s}] meters/second), is the group velocity of a wave in the media.
When \( \epsilon = \epsilon_0, \mu = \mu_0 \), \( c \) is the speed of light.
\end{itemize}
\( F \) is called the \textit{F}araday by some authors.
} % definition


The factors of \( \eta \) (or \( c \)) that multiply the magnetic field are for dimensional consistency, since \( [\sqrt{\epsilon} \BE] = [\sqrt{\mu} \BH] = [\BB/\sqrt{\mu}]\).
The justification for imposing a dual (or complex) structure on the electromagnetic field strength can be found in the historical development of
Maxwell's equations.
This structure also arises naturally when assembling the multivector Maxwell's equation.

No information is lost by imposing the complex structure of
\cref{dfn:isotropicMaxwells:640}, since we can always obtain the
electric field vector \( \BE \) and the magnetic field bivector \( I \BH \) by grade selection
from the electromagnetic field strength when desired using \(
\BE = \gpgradeone{ F },
I \BH = (1/\eta) \gpgradetwo{ F } \).

A multivector current containing all charge densites and current densities is defined as follows.
%
% Copyright � 2018 Peeter Joot.  All Rights Reserved.
% Licenced as described in the file LICENSE under the root directory of this GIT repository.
%
\makedefinition{Multivector current.}{dfn:isotropicMaxwells:660}{
The \textit{current} ([\si{A/m^2}] (Amperes/square meter)) is defined as
\begin{equation*}
J = \eta \lr{ c \rho - \BJ } + I \lr{ c \rho_\txtm - \BM }.
\end{equation*}
} % definition

When fictitious magnetic source terms \((\rho_\txtm, \BM)\) are included, the current has one grade for each possible source (scalar, vector, bivector, trivector).
With only conventional electric sources, the current is still a multivector, but contains only scalar and vector grades.

Given the multivector field and current, it is now possible to state the multivector form of Maxwell's equation (singular).
%
% Copyright � 2018 Peeter Joot.  All Rights Reserved.
% Licenced as described in the file LICENSE under the root directory of this GIT repository.
%
\maketheorem{Maxwell's equation.}{dfn:isotropicMaxwells:680}{
\textit{Maxwell's equation} is a multivector equation relating the change in the electromagnetic field strength to charge and current densities and is written as
\begin{equation*}
\stgrad F = J.
\end{equation*}
} % theorem


%See the book for a proof of \cref{dfn:isotropicMaxwells:680}.
The workhorse of the proof is the identity \( \spacegrad \Bb = \spacegrad \cdot \Bb + I \spacegrad \cross \Bb \) which allows Maxwell's equations into two gradient equations, one for each of \( \spacegrad \BE \), and \( \spacegrad \BH \)
\begin{dmath}\label{eqn:ece2500report:2660}
\begin{aligned}
\spacegrad \BE &= \inv{\epsilon} \rho + I \lr{ - \BM - \mu \PD{t}{\BH} } \\
\spacegrad \BH &= \inv{\mu} \rho_\txtm + I \lr{ \BJ + \epsilon \PD{t}{\BE} },
\end{aligned}
\end{dmath}
which can then be further grouped after dimensional rescaling to find
\begin{dmath}\label{eqn:isotropicMaxwells:580}
\stgrad \lr{ \BE + I \eta \BH } = \eta\lr{ c \rho - \BJ } + I \lr{ c \rho_\txtm - \BM },
\end{dmath}
which is \cref{dfn:isotropicMaxwells:680} expanded explictly.  There is a lot of information packed into this single equation.
All the subsequent analysis in the book utilizes the multivector form of Maxwell's equation.

\subsection{Wave equation and continuity.}
It can be
argued that the conventional form
\cref{eqn:ece2500report:2540}
of Maxwell's equations has a built in redundancy since continuity equations on the charge and current densities couple some of the equations.
An opposing argument is also possible, where the continuity equations are viewed as neccessary consequences of Maxwell's equation.
This amounts to a statement that the multivector current \( J \) is not completely unconstrained.

%
% Copyright � 2018 Peeter Joot.  All Rights Reserved.
% Licenced as described in the file LICENSE under the root directory of this GIT repository.
%
\maketheorem{Electromagnetic wave equation and continuity conditions.}{thm:continuity:600}{
The electromagnetic field is a solution to the non-homogeneous wave equation
\begin{equation*}
%\lr{ \spacegrad^2 - \inv{c^2} \PDSq{t}{} }
\dAlembertian
F =
\conjstgrad J.
\end{equation*}
In source free conditions, this reduces to a homogeneous wave equation, with group velocity \( c \), the speed of the wave in the media.
When expanded explicitly in terms of electric and magnetic fields, and charge and current densities, this single equation resolves to a
non-homogeneous wave equation for each of the electric and magnetic fields
\begin{equation*}
\begin{aligned}
%\lr{ \spacegrad^2 - \inv{c^2} \PDSq{t}{} }
\dAlembertian
\BE
%&= \gpgrade{\conjstgrad J}{1}
&= \inv{\epsilon} \spacegrad \rho + \mu \PD{t}{\BJ} + \spacegrad \cross \BM \\
%\lr{ \spacegrad^2 - \inv{c^2} \PDSq{t}{} }
\dAlembertian
\BH
%&= \inv{I \eta} \gpgrade{\conjstgrad J}{2}
&= \inv{\mu} \spacegrad \rho_\txtm + \epsilon \PD{t}{\BM} - \spacegrad \cross \BJ,
\end{aligned}
\end{equation*}
as well as a pair of continuity equations coupling the respective charge and current densities
\begin{equation*}
\begin{aligned}
\spacegrad \cdot \BJ + \PD{t}{\rho} &= 0 \\
\spacegrad \cdot \BM + \PD{t}{\rho_\txtm} &= 0.
\end{aligned}
\end{equation*}
} % theorem


The proof is in the book, but basically just requires operating on Maxwell's equation with \( \conjstgradi \), which yields two equations
equations, one for grades 1,2 and one for grades 0,3
\begin{dmath}\label{eqn:continuity:130}
\begin{aligned}
\dAlembertian
F &= \gpgrade{ \conjstgrad J }{1,2} \\
                                           0 &= \gpgrade{ \conjstgrad J }{0,3}.
\end{aligned}
\end{dmath}
Expansion of the grade 0,3 selection of \cref{eqn:continuity:130} provides the continuity equations.
\subsection{Plane waves.}
With all sources zero,
the free space Maxwell's equation as given by \cref{dfn:isotropicMaxwells:680} for the
electromagnetic field strength reduces to just
\begin{dmath}\label{eqn:planewavesMultivector:300}
\stgrad F(\Bx, t) = 0.
\end{dmath}

Utilizing a phasor representation of the form \cref{dfn:greensFunctionOverview:300},
we will define the
phasor representation of the field as follows.
%
% Copyright � 2018 Peeter Joot.  All Rights Reserved.
% Licenced as described in the file LICENSE under the root directory of this GIT repository.
%
\makedefinition{Plane wave.}{dfn:planewavesMultivector:680}{
We represent the
electromagnetic field strength
\textit{plane wave} solution of Maxwell's equation in phasor form as
\begin{equation*}
F(\Bx, t) = \Real \lr{ F(\Bk) e^{ j \omega t }  },
\end{equation*}
where the complex valued multivector \( F(\Bk) \) also has a presumed exponential dependence
\begin{equation*}
F(\Bk)
=
\tilde{F}
e^{ -j \Bk \cdot \Bx }.
\end{equation*}
} % definition


In the book, we show that solutions of the electromagnetic field wave equation have the following form.
%
% Copyright � 2018 Peeter Joot.  All Rights Reserved.
% Licenced as described in the file LICENSE under the root directory of this GIT repository.
%
\maketheorem{Plane wave solutions to Maxwell's equation.}{thm:planewavesMultivector:620}{
Single frequency \textit{plane wave solutions of Maxwell's equation} have the form
\begin{equation*}
F(\Bx, t)
=
\Real \lr{
\lr{ 1 + \kcap }
\kcap \wedge \BE\,
e^{-j \Bk \cdot \Bx + j \omega t}
}
,
\end{equation*}
where \( \Norm{\Bk} = \omega/c \), \( \kcap = \Bk/\Norm{\Bk} \) is the unit vector pointing along the propagation direction, and \( \BE \) is any complex-valued vector variable.
When a \( \BE \cdot \Bk = 0 \) constraint is imposed on the vector variable \( \BE \), that variable can be interpreted as the electric field, and the solution reduces to
\begin{equation*}
F(\Bx, t)
=
\Real \lr{
\lr{ 1 + \kcap }
\BE\,
e^{-j \Bk \cdot \Bx + j \omega t}
}
,
\end{equation*}
showing that the field phasor \( F(\Bk) = \BE(\Bk) + I \eta \BH(\Bk) \) splits naturally into electric and magnetic components
\begin{equation*}
\begin{aligned}
\BE(\Bk) &= \BE e^{-j \Bk \cdot \Bx} \\
\eta \BH(\Bk) &= \kcap \cross \BE \, e^{-j \Bk \cdot \Bx},
\end{aligned}
\end{equation*}
where the directions \( \kcap, \BE, \BH \) form a right handed triple.
} % theorem


A full proof and discussion is in the book.
The key step is that after insertion of the presumed phasor relationship, we find that
\begin{dmath}\label{eqn:planewavesMultivector:60}
0
=
-j \lr{ \Bk - \frac{\omega}{c} } F(\Bk),
\end{dmath}
which can be satisfied by insisting that \( F \) has a \( \Bk + \omega/c \) factor and that \( \Norm{\Bk} = \omega/c \).
The observation that
\( \kcap, \BE, \BH \) form a right handed triple, is expressed geometrically by \( I = \kcap \Ecap \Hcap \), from which we can also find \( \kcap = \Ecap \cross \Hcap \).

\subsection{Statics solution.}
If we restrict attention to time invariant fields (\( \partial_t F = 0\)) and time invariant sources (\(\partial_t J = 0\)),
Maxwell's equation is reduced to an invertible first order gradient equation
\begin{dmath}\label{eqn:statics:20}
\spacegrad F(\Bx) = J(\Bx),
\end{dmath}

%
% Copyright � 2018 Peeter Joot.  All Rights Reserved.
% Licenced as described in the file LICENSE under the root directory of this GIT repository.
%
\maketheorem{Maxwell's statics solution.}{thm:statics:100}{
The solution to the Maxwell statics equation is given by
\begin{equation*}
F(\Bx)
= \inv{4\pi} \int_V dV' \frac{\gpgrade{(\Bx - \Bx') J(\Bx')}{1,2}}{\Norm{\Bx - \Bx'}^3} + F_0,
\end{equation*}
where \( F_0 \) is any function for which \( \spacegrad F_0 = 0 \).
The explicit expansion in electric and magnetic fields and charge and current densities is given by
\begin{equation*}
\begin{aligned}
\BE(\Bx)
&=
\inv{4\pi} \int_V dV' \inv{\Norm{\Bx - \Bx'}^3}
\lr{
{\color{DarkOliveGreen}
   \inv{\epsilon}(\Bx - \Bx') \rho(\Bx')
}
   +
   (\Bx - \Bx') \cross \BM(\Bx')
} \\
\BH(\Bx)
&=
\inv{4\pi} \int_V dV' \inv{\Norm{\Bx - \Bx'}^3}
\lr{
{\color{Maroon}
  \BJ(\Bx') \cross (\Bx - \Bx')
}
+ \inv{\mu} (\Bx - \Bx') \rho_m(\Bx')
}.
\end{aligned}
\end{equation*}
} % theorem


The solution incorporates a {\color{DarkOliveGreen}Coulomb's law} contribution and a {\color{Maroon}Biot-Savart law} contribution, as well as magnetic analogues if applicable.

The proof is essentially a convolution with the (vector valued) Green's function for the (first order) gradient \cref{eqn:greensFunctionFirstOrderHelmholtz:900}.
\subsection{Statics: Enclosed charge.}
In conventional electrostatics we obtain a relation between the normal electric field component and the enclosed charge by integrating the electric field divergence.
The geometric algebra generalization relates the product of the normal and the electromagnetic field strength related to the enclosed multivector current as follows.
%
% Copyright � 2018 Peeter Joot.  All Rights Reserved.
% Licenced as described in the file LICENSE under the root directory of this GIT repository.
%
\maketheorem{Enclosed multivector current.}{thm:enclosedCurrent:60}{
The \textit{enclosed multivector current} in the volume is related to the surface integral of \( \ncap F \) over the boundary of the volume by
\begin{equation*}
\int_{\partial V} dA \ncap F = \int_V dV J.
\end{equation*}
This is a multivector equation, carrying information for each grade in the multivector current, and after explicit expansion is equivalent to
\begin{equation*}
\begin{aligned}
\int_{\partial V} dA\, \ncap \cdot \BE        &=  \inv{\epsilon} \int_V dV\, \rho \\
\int_{\partial V} dA\, \ncap \cross \BH       &=                 \int_V dV\, \BJ \\
\int_{\partial V} dA\, \ncap \cross \BE       &=               - \int_V dV\, \BM \\
\int_{\partial V} dA\, \ncap \cdot \BH        &=  \inv{\mu} \int_V dV\, \rho_\txtm.
\end{aligned}
\end{equation*}
} % theorem


The proof requires evaluation of the volume integral of the gradient of the field using \cref{thm:volumeintegral:100}, then a grade selection.  Full details are in the book.
The results of the grade selection
could have obtained directly from Maxwell's equations in their conventional form.
However, integration of the conventional Maxwell's equations would not have shown that the crazy mix of
fields, sources, dot and cross products in \cref{eqn:enclosedCurrent:60} had a hidden structure as simple as
\( \int_{\partial V} dA \ncap F = \int_V dV J \).

\subsection{Statics: Enclosed current.}
Ampere's law may be generalized to line integrals of the total electromagnetic field strength.
%
% Copyright � 2018 Peeter Joot.  All Rights Reserved.
% Licenced as described in the file LICENSE under the root directory of this GIT repository.
%
\maketheorem{Line integral of the field.}{thm:amperes:280}{
The \textit{line integral of the electromagnetic field strength} is
\begin{equation*}
\ointclockwise_{\partial A} d\Bx\, F
=
I \int_A dA \lr{ \ncap J - \PD{n}{F} },
\end{equation*}
where \( \PDi{n}{F} = \lr{ \ncap \cdot \spacegrad } F \).
Expressed in terms of the conventional consistent fields and sources, this multivector relationship expands to four equations, one for each grade
\begin{equation*}
\begin{aligned}
\ointclockwise_{\partial A} d\Bx \cdot \BE &=  \int_A dA\, \ncap \cdot \BM \\
\ointclockwise_{\partial A} d\Bx \cross \BH
&=
\int_A dA
\lr{
   - \ncap \cross \BJ
   + \frac{ \ncap \rho_\txtm }{\mu}
   - \PD{n}{\BH}
} \\
\ointclockwise_{\partial A} d\Bx \cross \BE &=
\int_A dA
\lr{
     \ncap \cross \BM
   + \frac{\ncap \rho}{\epsilon}
   - \PD{n}{\BE}
} \\
\ointclockwise_{\partial A} d\Bx \cdot \BH &= -\int_A dA\, \ncap \cdot \BJ.
\end{aligned}
\end{equation*}
} % theorem

Flipping the direction of integration in the the last of the scalar equations in
\cref{thm:amperes:280}
provides the conventional form of Ampere's law
\begin{equation}\label{eqn:amperes:20}
\ointctrclockwise_{\partial A} d\Bx \cdot \BH = \int_A \ncap \cdot \BJ = I_{\textrm{enc}}.
\end{equation}
The proof and additional details can be found in the book.

It is worth pointing out that for pure magnetostatics problems where \( J = \eta \BJ, F = I \eta \BH \), that Ampere's law can be written in a trivector form
\begin{equation}\label{eqn:amperes:260}
\ointclockwise_{\partial A} d\Bx \wedge F = I \int_A dA\, \ncap \cdot J = I \eta \int_A dA\, \ncap \cdot \BJ.
\end{equation}
This encodes the fact that the magnetic field component of the total electromagnetic field strength is most naturally expressed in
geometric algebra as a bivector.

\subsection{Statics: Example field calculations.}
A number of worked examples were calculated to illustrate geometric algebra techniques.
\begin{itemize}
\item A finite line charge with line charge density \( \lambda \).  This problem is worked with conventional and geometric algebra.  With the conventional approach a compact factorization of the end result is possible by introducing a 3x3 rotation matrix.
In the geometric algebra approach the field observation point is directly encoded in ``complex exponential'' form using
\( i = \Be_1 \Be_3 \) to represent the pseudoscalar for the x-z plane.
The (electric) field is found to have the form \( F = (\lambda/r) \int du \lr{ e^{i\theta} - u } f(u) \), where
\( f(u) \) is a scalar function (specified in the book.)
The scalar portion of the integral is strictly a scale factor for the component of the field that lies along the x-axis, whereas the ``complex exponential'' factor of the integrand represents a rotational term along the direction \( \Be_1 e^{i\theta} = \Be_1 \cos\theta + \Be_3 \sin\theta \).
In problems like this, with only two degrees of freedom, there will often be a complex like representation possible using geometric algebra.
\item The field for infinite static charge and current densities lying along the z-axis \( \rho(\Bx) = \lambda \delta(x) \delta(y), \BJ(\Bx) = \Bv \rho(\Bx) \) is found to be \( F = \lambda \rhocap \lr{ 1 - \Bv/c}/(2 \pi \epsilon R)\).
The field splits naturally into electric (vector) and magnetic (bivector) grades as \( F = \BE \lr{ 1 - \Bv/c } = \BE + I \lr{ \ifrac{\Bv}{c} \cross \BE } \).
\item A similar problem is left for the reader, who is asked to compute the field for
the magnetic charge density \( \rho_m = \lambda_m \delta(x) \delta(y) \), and current density \( \BM = v \Be_3 \rho_m = \Bv \rho_m \).
That field is
\( F = \lambda_m c I \rhocap \lr{ 1 - \ifrac{\Bv}{c} }/(4 \pi R) \), which may be split into electric and magnetic components as
\( F = \BB \cross \Bv + c I \BB \), where \( \BB = \lambda_m \rhocap/(4 \pi R) \).
\item The field for a uniform infinite planar charge density \( \rho(\Bx) = \sigma \delta(z) \) and associated current density \( \BJ(\Bx) = \Bv \rho(\Bx) \), where \( \Bv = v \Be_1 e^{i\theta}, \quad i = \Be_{12} \) is found to be
\( F = \sigma \sgn(z) \Be_3 \lr{ 1 - \ifrac{\Bv}{c}}/(4 \pi \epsilon) \).
As should be expected by superposition, the field splits neatly into electric field (vector) and magnetic field (bivector) components associated with the respective pure electrostatic and magnetostatics problems.
\item As a problem the reader is asked to show that the field for an infinite planar
magnetic charge density \( \rho_m = \sigma_m \delta(z) \), and current density \( \BM = \Bv \rho_m, \Bv = v \Be_1 e^{i\theta}, i = \Be_{12}\) is \( F = \ifrac{\sigma_m c \sgn(z)}{(4 \pi)} i \lr{ 1 - \ifrac{\Bv}{c} } \).
\item The field for a line charge density \( \lambda \) along a circular arc segment \( \phi' \in [a,b] \), of radius \( r \) in the x-y plane is found to be \( F = \ifrac{\lambda r}{(4 \pi \epsilon_0 R^2)} \int_{a-\phi}^{b-\phi} du \lr{ \rcap + \phicap u i e^{i \alpha } } \lr{ 1 + u^2 - 2 u \sin\theta \cos \alpha }^{-3/2} \), where \( i = \Be_{12} \).
This problem is often given as an example or problem in electrostatics, but usually for circular charge distribution, and an observation point on the z-axis where symmetries kill off all but the z-axis component of the field.
The freedom to represent rotational terms as complex exponentials in geometric algebra allows the more general problem to be calculated without much additional difficulty.
The resulting integrals can be evaluated easily with any existing numerical integration software as the vector factors \( \rcap, \phicap \) may be pulled out of the integrals, leaving strict scalar or complex valued integrands.
\item To illustrate the algebraic flexibility available, the circular ring charge problem is tackled in cylindrical coordinates instead of spherical (as previous).
For a static charge line density \( \lambda \) on a ring at \( z = 0 \), and an azimuthal current density \( \BJ = \Bv \rho \), we find a closed form solution for the field is found.
The symmetry of the ring configuration allows for a closed form solution (numerical integration not required) of the field, but comes with the cost of requiring
elliptic integrals, which are detailed in the book along with the derivation and plots of the resulting fields.
\item The final worked statics problem in the book is a use of Ampere's law, to compute the magnetic field for a
pair of z-axis oriented electric currents of magnitude \( I_1, I_2 \) flowing through the \( \Bp_1, \Bp_2 \) on the x-y plane.
The geometry and derivation is detailed in the book, but we use the multivector line integral form of Ampere's law \( \ointctrclockwise_{\partial A} d\Bx F = -I \int_A dA \Be_3 (-\eta \BJ) = I \eta I_\txte \), and superposition to compute the field
\( F = \sum_{k = 1,2} \ifrac{\eta I_k}{(2 \pi)} \ifrac{1}{(\Be_3 \wedge \lr{ \Br - \Bp_k})} \).
The bivector (magnetic) nature of a field with only electric current density sources is naturally represented by the wedge product \( \Be_3 \wedge \lr{ \Br - \Bp_k} \) which is a vector product of \( \Be_3 \) and the projection of \( \Br - \Bp_k \) onto the x-y plane.
\end{itemize}
\subsection{Dynamics.}
Maxwell's equation (\cref{dfn:isotropicMaxwells:680}) is invertable, with solution.
%
% Copyright � 2018 Peeter Joot.  All Rights Reserved.
% Licenced as described in the file LICENSE under the root directory of this GIT repository.
%
\maketheorem{Jefimenkos solution.}{thm:jefimenkosEquations:120}{
The general solution of Maxwell's equation is given by
\begin{equation*}
F(\Bx, t)
=
F_0(\Bx, t)
+
\inv{4 \pi}
\int dV'
\lr{
   \frac{\rcap}{r^2} J(\Bx', t_r)
   +
   \inv{c r} \lr{ 1 + \rcap } \dispdot{J}(\Bx', t_r)
},
\end{equation*}
where \( F_0(\Bx, t) \) is any specific solution of the homogeneous equation \( \lr{ \spacegrad + (1/c) \partial_t } F_0 = 0 \),
time derivatives are denoted by overdots, and all times are evaluated at the retarded time \( t_r = t - r/c \).
When expanded in terms of the electric and magnetic fields (ignoring magnetic sources), the non-homogeneous portion of this solution is known as
Jefimenkos' equations
\begin{dmath}\label{eqn:jefimenkosEquations:100}
\begin{aligned}
\BE &=
\inv{4 \pi}
\int dV'
\lr{
\frac{\rcap}{\epsilon r} \lr{
\frac{\rho(\Bx', t_r)}{r} + \frac{\dispdot{\rho}(\Bx', t_r) }{c} }
   - \frac{\eta }{ c r } \dotBJ(\Bx', t_r)
} \\
\BH &=
\inv{4 \pi}
\int dV'
\lr{
   \frac{1}{c r} \dotBJ(\Bx', t_r)
+
   \frac{1}{r^2} \BJ(\Bx', t_r)
} \cross \rcap,
\end{aligned}
\end{dmath}
%which checks against Griffiths.
} % theorem

This is found fairly easily using the Green's function for the spacetime gradient \cref{thm:greensFunctionSpacetimeGradient:120}, and the details can be found in the book.
Unlike the conventional approach, we are able to find the field directly without first having to determine the retarded time potentials, nor having to take their derivatives.

\subsection{Energy and momentum.}
The energy and momentum section of the book discusses field energy density, the Poynting vector, Maxwell stress tensor, and more generally, the energy momentum tensor.
The results in the conventional and geometric algebra formalism are detailed, showing how the two relate.
%
% Copyright � 2018 Peeter Joot.  All Rights Reserved.
% Licenced as described in the file LICENSE under the root directory of this GIT repository.
%
\makedefinition{(Conventional) Energy and momentum density and Poynting vector.}{dfn:poyntingF:1220}{
The quantities \( \calE \) and \( \bcP \) defined as
%\label{eqn:poyntingF:20}
\begin{equation*}
\begin{aligned}
\calE &
%=
%\inv{2} \lr{ \BD \cdot \BE + \BB \cdot \BH }
= \inv{2} \lr{ \epsilon \BE^2 + \mu \BH^2 } \\
\bcP c &= \inv{c} \BE \cross \BH,
\end{aligned}
\end{equation*}
are known respectively as the field energy and momentum densities.
\( \BS = c^2 \bcP = \BE \cross \BH \) is called the Poynting vector.
} % definition

In geometric algebra the energy momentum tensor, and the Maxwell stress tensor may be represented as linear grade 0,1 multivector valued functions of a grade 0,1 multivector, as follows.
%
% Copyright � 2018 Peeter Joot.  All Rights Reserved.
% Licenced as described in the file LICENSE under the root directory of this GIT repository.
%
\makedefinition{Energy momentum and Maxwell stress tensors.}{dfn:poyntingF:1200}{
We define the \textit{energy momentum tensor} as
\begin{equation*}
T(a) = \inv{2} \epsilon F a F^\dagger,
\end{equation*}
where \( a \) is a grade \((0,1)\)-multivector parameter.
We introduce a shorthand notation for grade one selection with vector valued parameters
\begin{equation*}
\BT(\Ba) = \gpgradeone{T(\Ba)},
\end{equation*}
and call this the \textit{Maxwell stress tensor}.
} % definition


%
% Copyright � 2018 Peeter Joot.  All Rights Reserved.
% Licenced as described in the file LICENSE under the root directory of this GIT repository.
%
\maketheorem{Expansion of the energy momentum tensor.}{thm:poyntingF:1240}{
Given a scalar parameter \( \alpha \), and a vector parameter \( \Ba = \sum_k a_k \Be_k \), the energy momentum tensor of
\cref{dfn:poyntingF:1200} is a grade 0,1 multivector, and may be expanded in terms of \( \calE, \BS \) and \( \BT(\Ba) \) as
%\label{eqn:poyntingF:1120}
\begin{equation*}
T(\alpha + \Ba)
=
\alpha \lr{
   \calE + \frac{\BS}{c}
}
-
\Ba \cdot \frac{\BS}{c}
+ \BT(\Ba),
\end{equation*}
where \( \BT(\Be_i) \cdot \Be_j = -\Theta^{ij} \), or \( \BT(\Ba) = \Ba \cdot \lrT \).
} % theorem


\Cref{thm:poyntingF:1240} relates the geometric algebra definition of the energy momentum tensor to the quantities found in the conventional
electromagnetism literature.
In the book, the conventional indexed representation is detailed more completely for comparision purposes.

Associated with the energy momentum tensor are a number of conservation relationships, which are most compactly stated utilizing the adjoint of the energy momenutum tensor.
%
% Copyright � 2018 Peeter Joot.  All Rights Reserved.
% Licenced as described in the file LICENSE under the root directory of this GIT repository.
%
\index{\(\overbar{A}(x)\)}
\makedefinition{Adjoint.}{dfn:poyntingTheorem:1120}{
The \textit{adjoint} \( \overbar{A}(x) \) of a linear operator \( A(x) \) is defined implicitly by the scalar selection
\begin{equation*}
\gpgradezero{ y \overbar{A}(x) } =
\gpgradezero{ x A(y) }.
\end{equation*}
} % definition

%
% Copyright � 2018 Peeter Joot.  All Rights Reserved.
% Licenced as described in the file LICENSE under the root directory of this GIT repository.
%
\maketheorem{Poynting's theorem (differential form.)}{thm:poyntingTheorem:1180}{
The adjoint energy momentum tensor of the spacetime gradient satisfies the following multivector equation
\begin{equation*}
\overbar{T}(\spacegrad + (1/c)\partial_t) = \frac{\epsilon}{2} \lr{ F^\dagger J + J^\dagger F }.
\end{equation*}
The multivector \( F^\dagger J + J^\dagger F \) can only have scalar and vector grades, since it equals its reverse.
This equation can be put into a form that is more obviously a conservation law by stating it as a set of
scalar grade identities
\begin{equation*}
\spacegrad \cdot \gpgradeone{ T(a) } + \inv{c} \PD{t}{} \gpgradezero{ T(a) }
=
\frac{\epsilon}{2} \gpgradezero{ a( F^\dagger J + J \dagger F) },
\end{equation*}
or as a pair of scalar and vector grade conservation relationships
%%which expands to the multivector equation
%\begin{equation*}
%\inv{c} \PD{t}{} \lr{ \calE - \frac{\BS}{c} }
%+ \spacegrad \cdot \frac{\BS}{c}
%+ \BT(\spacegrad)
%=
%-\inv{c} \lr{ \BE \cdot \BJ + \BH \cdot \BM }
%+
%\rho \BE + \epsilon \BE \cross \BM
%+
%\rho_\txtm \BH + \mu \BJ \cross \BH,
%\end{equation*}
%or as separate scalar and vector equations
\begin{equation*}
\begin{aligned}
\inv{c} \PD{t}{\calE} + \spacegrad \cdot \frac{\BS}{c} &= -\inv{c} \lr{ \BE \cdot \BJ + \BH \cdot \BM } \\
-\inv{c^2} \PD{t}{\BS} + \BT(\spacegrad) &= \rho \BE + \epsilon \BE \cross \BM + \rho_\txtm \BH + \mu \BJ \cross \BH.
\end{aligned}
\end{equation*}
Conventionally, only the scalar grade relating the time rate of change of the energy density to the flux of the Poynting vector, is called Poynting's theorem.
Here the more general multivector (adjoint) relationship is called \textit{Poynting's theorem}, which includes conservation laws relating for the field energy and momentum densities and conservation laws relating the Poynting vector components and the Maxwell stress tensor.
} % theorem

or in an integral form
%
% Copyright � 2018 Peeter Joot.  All Rights Reserved.
% Licenced as described in the file LICENSE under the root directory of this GIT repository.
%
\maketheorem{Poynting's theorem (integral form.)}{thm:poyntingTheoremRewrite:1420}{
\begin{dmath}\label{eqn:poyntingTheoremRewrite:1400}
\begin{aligned}
&\PD{t}{}
\int_V dV\, \calE 
=
-\int_{\partial V} dA\, \ncap \cdot \BS
-
\int_V dV \lr{
   \BJ \cdot \BE
   +
   \BM \cdot \BH
} \\
&
\int_V dV \lr{ \rho \BE + \BJ \cross \BB }
+ \int_V dV \lr{ \rho_\txtm \BH - \epsilon \BM \cross \BE }
=
-
\PD{t}{ }
\int_V dV \bcP
+
\int_{\partial V} dA\, \BT(\ncap).
\end{aligned}
\end{dmath}
} % theorem


As the field in the volume is carrying the (electromagnetic) momentum \( \Bp_{\textrm{em}} = \int_V dV \bcP \), we can identify the sum of the Maxwell stress tensor's normal component over the bounding integral as time rate of change of the mechanical and electromagnetic momentum
\begin{equation}\label{eqn:ece2500report:2560}
\frac{d}{dt} \Bp_{\textrm{mech}} + \frac{d}{dt} \Bp_{\textrm{em}} = \int_{\partial V} dA \BT(\ncap).
\end{equation}
The rate of change of mechanical momentum density \( \ifrac{d\Bp_{\textrm{mech}}}{dt} \) is, in fact, the continuous equivalent of the Lorentz force, which is found to be a direct consequence of conservation relationships associated with Maxwell's equation.

There were many details left out above.
Please refer to the book for more information.

\subsubsection{Example energy momentum calculations.}
To illustrate the ideas above, the energy momentum tensor components for all of the static fields computed previously are determined.
For brievity, these are omitted from this report, but it should be noted that we see that geometric algebra allows for a particularly compact coordinate free representation of the energy momentum tensor components.
\subsubsection{Complex power.}
The geometric algebra forms of the \( T(1) \) (field energy density and Poynting vector) are found to be
%
% Copyright � 2018 Peeter Joot.  All Rights Reserved.
% Licenced as described in the file LICENSE under the root directory of this GIT repository.
%
\maketheorem{Complex power representation.}{thm:poyntingFComplexPower:300}{
Given a time domain representation of a phasor based field \( F = F(\omega) \)
\begin{equation*}
F(t)
= \Real\lr{ F e^{j \omega t} },
\end{equation*}
the energy momentum tensor multivector \( T(1) \) has the representation
\begin{equation*}
T(1) = \calE + \frac{\BS}{c}
=
\frac{\epsilon}{4} \Real \lr{ F^\conj F^\dagger + F F^\dagger e^{2 j \omega t} }.
\end{equation*}
With the usual definition of the complex Poynting vector
%\label{eqn:poyntingFComplexPower:240}
\begin{equation*}
\calS = \inv{2} \BE \cross \BH^\conj = \inv{2} \lr{ I \BH^\conj } \cdot \BE,
\end{equation*}
the energy and momentum components of \( T(1) \), for real \( \mu, \epsilon \) are
%\label{eqn:poyntingFComplexPower:260}
\begin{equation*}
\begin{aligned}
\calE &=
\inv{4} \lr{
\epsilon \Abs{\BE}^2 + \mu \Abs{\BH}^2 }
+
\inv{4} \Real
\lr{
   \lr{ \epsilon \BE^2 + \mu \BH^2}
   e^{2 j \omega t }
} \\
\BS &= \Real \calS
+
\inv{2} \Real
\lr{
\lr{ \BE \cross \BH }
   e^{2 j \omega t }
}.
\end{aligned}
\end{equation*}
} % theorem


\subsection{Lorentz force.}
The Lorentz force equation can be stated in terms of the total electromagnetic field strength and current density
%
% Copyright � 2018 Peeter Joot.  All Rights Reserved.
% Licenced as described in the file LICENSE under the root directory of this GIT repository.
%
\maketheorem{Lorentz force and power.}{thm:lorentzForce:300}{
Given an energy momentum multivector \( T = \calE + c \Bp \), and a charge associated with a small bounded multivector current density \( Q = \int_V J dV \),
the respective power and force experienced by a particle with electric (and/or magnetic) charge is
\begin{equation*}
\inv{c} \frac{dT}{dt} = \gpgrade{ F Q^\dagger }{0,1} = \inv{2} \lr{ F^\dagger Q + F Q^\dagger }.
\end{equation*}
where \( \gpgradezero{dT/dt} = \ifrac{d\calE}{dt} \) is the power and \( \gpgradeone{dT/dt} = c \ifrac{d\Bp}{dt} \) is the force on the particle.
The conventional representation of the Lorentz force/power equations
\begin{equation*}
\begin{aligned}
\gpgradeone{ F Q^\dagger } &= \ddt{\Bp} = q \lr{ \BE + \Bv \cross \BB } \\
c \gpgradezero{ F Q^\dagger } &= \ddt{\calE} = q \BE \cdot \Bv.
\end{aligned}
\end{equation*}
%given by \cref{eqn:freespace:180}
may be recovered by grade selection operations.
For magnetic particles, such a grade selection gives
\begin{equation*}
\begin{aligned}
\gpgradeone{ F Q^\dagger } &= \frac{d\Bp}{dt} = q_\txtm \lr{ c \BB - \inv{c} \Bv \cross \BE } \\
c \gpgradezero{ F Q^\dagger } &= \frac{d\calE}{dt} = \inv{\eta} q_\txtm \BB \cdot \frac{\Bv}{c}.
\end{aligned}
\end{equation*}
} % theorem


\subsubsection{Constant magnetic field.}
As another example of geometric algebra in action, the Lorentz force equation for a constant external magnetic field bivector \( F = I c \BB \)
\begin{dmath}\label{eqn:lorentzForce_constantMagnetic:60}
m \frac{d\Bv}{dt} = q F \cdot \frac{\Bv}{c},
\end{dmath}
is solved in a fashion unique to this algebra.
With
\( \Omega = -\ifrac{q F}{m c} \), the Lorentz force equation is reduced to \( \ifrac{d\Bv}{dt} = \Bv \cdot \Omega \), which may be solved
using a multivector integration factor.
The solution is shown to be
\begin{dmath}\label{eqn:lorentzForce_constantMagnetic:200}
\Bv(t) = e^{-\Omega t/2} \Bv(0) e^{\Omega t/2}.
\end{dmath}
Any component of the initial velocity \( \Bv(0)_\perp \) perpendicular to the \( \Omega \) plane is untouched by this rotation operation, whereas components of the initial velocity \( \Bv(0)_\parallel \) that lie in the \( \Omega \) plane will trace out a circular path, so the velocity of the charged particle traces out a helical path.

More general examples are considered in the literature cited in the book.

\subsection{Polarization}
Following the usual convention, the geometric algebra treatment of polarization in the book
aligns the propagation direction along the z-axis.
The field is
\begin{dmath}\label{eqn:polarization:20}
\begin{aligned}
F(\Bx, \omega) &= (1 + \Be_3) \BE e^{-j \beta z} \\
F(\Bx, t) &= \Real\lr{ F(\Bx, \omega) e^{j \omega t} },
\end{aligned}
\end{dmath}
where \( \BE \cdot \Be_3 = 0 \).
Here the imaginary \( j \) has no intrinsic geometrical interpretation, but we are able to dispense with it and use geometric imaginaries instead.
This is done by first assuming the electric field is given by \( \BE = \lr{ \alpha_1 + j \beta_1 } \Be_1 + \lr{ \alpha_2 + j \beta_2 } \Be_2 \), so that the
time domain representation of the field is given by
\begin{dmath}\label{eqn:polarization_circular:160}
F(\Bx, t) = (1 + \Be_3) \lr{
\lr{ \alpha_1 \Be_1 + \alpha_2 \Be_2 } \cos\lr{ \omega t - \beta z }
-\lr{ \beta_1 \Be_1 + \beta_2 \Be_2 } \sin\lr{ \omega t - \beta z }
}.
\end{dmath}

Two geometric representations are possible.
The first uses the pseudoscalar for the transverse plane \( \Be_{12} \), denoted \( i \) here, and the other uses the \R{3} pseudoscalar as the imaginary.
\subsubsection{Transverse plane imaginary.}
%
% Copyright � 2018 Peeter Joot.  All Rights Reserved.
% Licenced as described in the file LICENSE under the root directory of this GIT repository.
%
\maketheorem{Circular polarization coefficients.}{thm:polarizationRewrite:700}{
The time domain representation of the field in \cref{eqn:polarization_circular:160} can be stated in terms of the total phase as
\begin{equation*}
F = \lr{ 1 + \Be_3 } \Be_1 \lr{ \alpha_\txtR e^{i\phi} + \alpha_\txtL e^{-i\phi} },
\end{equation*}
where the \textit{circular polarization coeffecients} are given by
\begin{equation*}
\begin{aligned}
\alpha_\txtR &= \inv{2}\lr{ c_1 + i c_2 } \\
\alpha_\txtL &= \inv{2}\lr{ c_1 - i c_2 }^\dagger,
\end{aligned}
\end{equation*}
where \( c_1, c_2 \) are 0,2 grade multivector representations of the Jones vector coordinates
\begin{equation*}
\begin{aligned}
c_1 &= \alpha_1 + i \beta_1 \\
c_2 &= \alpha_2 + i \beta_2,
\end{aligned}
\end{equation*}
and \( \phi(z,t) = \omega t - \beta z \) is the phase angle.
} % theorem


\begin{itemize}
\item
Linear polarization at an angle \( \psi\) from the x-axis in the transverse plane is given by
\( \alpha_\txtR = \inv{2}\Norm{\BE} e^{i(\psi + \theta)},
\alpha_\txtL = \inv{2}\Norm{\BE} e^{i(\psi - \theta)} \), for which the field is \( F = \lr{ 1 + \Be_3 } \Norm{\BE} \Be_1 e^{i \psi} \cos(\phi + \theta) \),
where \( \theta \) is an initial phase angle.
\item
Following the IEEE antenna convention, we define right(left) circular polarization as the
a change in phase that
results in the electric field tracing out a (clockwise,counterclockwise) circle
\begin{dmath}\label{eqn:polarization_circular:180}
\begin{aligned}
\BE_\txtR &= \Norm{\BE} \lr{ \Be_1 \cos\phi + \Be_2 \sin\phi } = \Norm{\BE} \Be_1 \exp\lr{  \Be_{12} \phi } \\
\BE_\txtL &= \Norm{\BE} \lr{ \Be_1 \cos\phi - \Be_2 \sin\phi } = \Norm{\BE} \Be_1 \exp\lr{ -\Be_{12} \phi }.
\end{aligned}
\end{dmath}
Right and left circular polarization in this representation are given by
\(\alpha_\txtR = \Norm{\BE}, \alpha_\txtL = 0 \) and \(\alpha_\txtL = \Norm{\BE}, \alpha_\txtR = 0 \) respectively.
The right(left) polarized fields are just
\( F = (1 + \Be_3) \Norm{\BE} \Be_1 e^{\pm i(\omega t - k z)} \).
\item An ellipically polzarized field is given by
\( \alpha_\txtR = \inv{2}\lr{ E_a - E_b },
\alpha_\txtL = \inv{2}\lr{ E_a + E_b } \), or
\begin{dmath}\label{eqn:ece2500report:2580}
F = \inv{2} (1 + \Be_3) \Be_1 \lr{ (E_a + E_b) e^{i\phi} + (E_a - E_b) e^{-i\phi} }
\end{dmath}
A hyperbolic parameterization of the elliptically polarized wave is also discussed in the book
\begin{dmath}\label{eqn:polarization_elliptical:380}
\begin{aligned}
F &= e E_a \lr{ 1 + \Be_3 } \Be_1 e^{ i \psi } \cosh\lr{ m + i \phi} \\
m &= \tanh^{-1}\lr{ E_b/E_a } \\
e &= \sqrt{1 - {(E_b/E_a)}^2 },
\end{aligned}
\end{dmath}
where \( E_a(E_b) \) are the magnitudes of the electric field components lying along the semi-major axis directed along \(
\begin{bmatrix}
\Be_1 \\
\Be_2
\end{bmatrix}
e^{i\psi} \) respectively.
Additional discussion and diagrams can be found in the book.
\end{itemize}

Each polarization considered above (linear, circular, elliptical) have the same general form
\begin{dmath}\label{eqn:polarizationRewrite:760}
F = \lr{ 1 + \Be_3 } \Be_1 e^{i\psi} f(\phi),
\end{dmath}
where \( f(\phi) \) is a complex valued function (i.e. grade 0,2).
The structure of \cref{eqn:polarizationRewrite:760} could be more general than considered so far.
For example, a Gaussian modulation could be added into the mix with \( f(\phi) = e^{i \phi - (\phi/\sigma)^2/2 } \).
The simple complex structure that encodes all the phase dependence allows the
energy, momentum and Maxwell stress tensor to be computed easily.
%
% Copyright � 2018 Peeter Joot.  All Rights Reserved.
% Licenced as described in the file LICENSE under the root directory of this GIT repository.
%
\maketheorem{Plane wave energy momentum tensor components.}{thm:polarizationRewrite:780}{
The \textit{energy momentum tensor components} for the plane wave given by \cref{eqn:polarizationRewrite:760} are
\begin{equation*}
\begin{aligned}
T(1) &= -T(\Be_3) = \epsilon \lr{ 1 + \Be_3 } f f^\dagger \quad \lr{ = \calE + \frac{\BS}{c} } \\
T(\Be_1) &= T(\Be_2) = 0.
\end{aligned}
\end{equation*}
} % theorem


Only the propagation direction of a plane wave, regardless of its polarization (or even whether or not there are Gaussian or other damping factors), carries any energy or momentum, and only the propagation direction component of the Maxwell stress tensor \( \BT(\Ba) \) is non-zero.

Using \cref{thm:polarizationRewrite:780} the energy momentum vector may be computed for each of the polarizations considered above.
\begin{itemize}
\item
For the linearly polarization \( T(1) = \ifrac{\epsilon}{2} \lr{ 1 + \Be_3 } \Norm{\BE}^2 \cos^2( \phi + \theta ) \).
\item For the circularly polarization \( T(1) = \ifrac{\epsilon}{2} (1 + \Be_3) \Norm{\BE}^2 \).
A circularly polarized wave carries maximum energy and momentum, whereas the energy and momentum of a linearly polarized wave
oscillates with the phase angle.
\item For the elliptical polarization \(
T(1)
= \ifrac{\epsilon}{2} \lr{ 1 + \Be_3 } e^2 \lr{ E_b^2 + 2 \lr{ E_a^2 - E_b^2} \cos^2 \phi } \).
As expected, the phase dependent portion of the energy momentum tensor vanishes as the wave function approaches circular polarization.
\end{itemize}

\subsubsection{Pseudoscalar imaginary.}
Alternatively, it is possible to encode the sines and cosines in the time domain representation of the field in terms of the \R{3} pseudoscalar.

%
% Copyright � 2018 Peeter Joot.  All Rights Reserved.
% Licenced as described in the file LICENSE under the root directory of this GIT repository.
%
\maketheorem{Circular polarization coefficients.}{thm:polarizationRewrite:940}{
The time domain representation of the field in \cref{eqn:polarization_circular:160} can be stated in terms of the total phase as
\begin{equation*}
F = \lr{ 1 + \Be_3 } \Be_1 \lr{ \alpha_\txtR e^{-I\phi} + \alpha_\txtL e^{I\phi} },
\end{equation*}
where the \textit{circular polarization coefficients} are
\begin{equation*}
\begin{aligned}
\alpha_\txtR &= \inv{2}\lr{ c_1 + I c_2 }^\dagger \\
\alpha_\txtL &= \inv{2}\lr{ c_1 - I c_2 },
\end{aligned}
\end{equation*}
where \( c_1, c_2 \) are the 0,2 grade multivector representation of the Jones vector coordinates
\begin{equation*}
\begin{aligned}
c_1 &= \alpha_1 + I \beta_1 \\
c_2 &= \alpha_2 + I \beta_2,
\end{aligned}
\end{equation*}
defined here as 0,3 complex numbers, using \( I \) as the imaginary.
} % theorem


There appear to be some advantages to pseudoscalar description of polarization, especially for computing energy momentum tensor components since \( I \) commutes with all grades.
For example, we can see practically by inspection that
\begin{equation}\label{eqn:polarization_pseudoscalarImaginary:620}
T(1) = \calE + \frac{\BS}{v} =
\epsilon \lr{ 1 + \Be_3 } \lr{ \Abs{\alpha_\txtR}^2 + \Abs{\alpha_\txtL}^2 },
\end{equation}
where the absolute value is computed using the reverse as the conjugation operation \( \Abs{z}^2 = z z^\dagger \).

\subsection{Transverse fields in a waveguide.}
One topic from waveguide theory is considered in the book.
%
% Copyright � 2018 Peeter Joot.  All Rights Reserved.
% Licenced as described in the file LICENSE under the root directory of this GIT repository.
%
\maketheorem{Transverse and propagation field solutions.}{thm:transverseField:348}{
Given a field propagating along the z-axis (either forward or backwards), with angular frequency \( \omega \), represented by the real part of
\begin{equation*}
F(x, y, z, t) = F(x, y) e^{j \omega t \mp j k z},
\end{equation*}
where \( \BE = \BE_z + \BE_t, \BH = \BH_z + \BH_t \), and \( \BA_z = (\BA \cdot \Be_3) \Be_3, \BA \in \BE, \BH \), the multivector field components in the axial and transverse ``directions''
\( F = F_z + F_t \)
are given by
\begin{equation*}
\begin{aligned}
F_z &= \inv{2} \lr{ F + \Be_3 F \Be_3 } \\
F_t &= \inv{2} \lr{ F - \Be_3 F \Be_3 },
\end{aligned}
\end{equation*}
and related to each other by
\begin{equation*}
\begin{aligned}
F_t &= j \inv{ \frac{\omega}{c} \mp k \Be_3 } \spacegrad_t F_z \\
F_z &= j \inv{ \frac{\omega}{c} \mp k \Be_3 } \spacegrad_t F_t,
\end{aligned}
\end{equation*}
Written out explicitly, the transverse field component expands as
\begin{equation*}
\begin{aligned}
\BE_t &=
\frac{j}{{\frac{\omega}{c}}^2 - k^2}
\lr{
   \pm k \spacegrad_t E_z
   + \frac{\omega \eta}{c} \Be_3 \cross \spacegrad_t H_z
}
\\
\eta \BH_t &=
\frac{j}{{\frac{\omega}{c}}^2 - k^2}
\lr{
   \pm k \eta \spacegrad_t H_z
   -
   \frac{\omega}{c}
   \Be_3 \cross \spacegrad_t E_z
}.
\end{aligned}
\end{equation*}
} % theorem


Details and proof are in the book.
This and many other examples in the book show that we pay a significant additional cost to work with separate electric and magnetic field components compared to working with the complete electromagnetic field multivector \( F \) in its entirety.

\subsection{Multivector potential.}
Conventional electromagnetism utilizes scalar and vector potentials, which may be generalized to a multivector potential containing their sums.
\input{multivector_potential}

The grades of the multivector potentials may be chosen to match SI conventions (as specified in Balanis' ``Antenna Theory'',
which includes fictitious magnetic sources),
%
% Copyright � 2018 Peeter Joot.  All Rights Reserved.
% Licenced as described in the file LICENSE under the root directory of this GIT repository.
%
\maketheorem{Fields and the potential wave equations.}{thm:generalPotential:40}{
Given
%\label{eqn:gaugeTransformation:1111}
\begin{equation*}
A =
      - \phi
      + c \BA
      + \eta I \lr{ -\phi_m + c \BF },
\end{equation*}
where
\begin{enumerate}
\item \( \phi \) is the scalar potential \si{V} (Volts).
\item \( \BA \) is the vector potential \si{W/m} (Webers/meter).
\item \( \phi_m \) is the scalar potential for (fictitious) magnetic sources \si{A} (Amperes).
\item \( \BF \) is the vector potential for (fictitious) magnetic sources \si{C} (Coulombs),
\end{enumerate}
the electric field vector and the magnetic field bivector associated with a potential \( A \) are
\begin{equation*}
\begin{aligned}
\BE &=
\gpgrade{\conjstgrad A}{1}
=
   - \spacegrad \phi
   - \PD{t}{\BA}
   - \inv{\epsilon} \spacegrad \cross \BF \\
I \eta \BH &=
\gpgrade{\conjstgrad A}{2}
=
   I \eta
   \lr{
      - \spacegrad \phi_\txtm
      - \PD{t}{\BF}
      + \inv{\mu} \spacegrad \cross \BA
   }
.
\end{aligned}
\end{equation*}
The potentials are related to the sources by
\begin{equation*}
\begin{aligned}
\dAlembertian
\phi &= -\frac{\rho}{\epsilon} - \PD{t}{} \lr{ \spacegrad \cdot \BA + \inv{c^2} \PD{t}{\phi} } \\
\dAlembertian
\BA &= -\mu \BJ + \spacegrad \lr{ \spacegrad \cdot \BA + \inv{c^2} \PD{t}{\phi} } \\
\dAlembertian
\BF &= - \epsilon \BM + \spacegrad \lr{ \spacegrad \cdot \BF + \inv{c^2} \PD{t}{\phi_\txtm} } \\
\dAlembertian
\phi_\txtm &= -\frac{\rho_\txtm}{\mu} - \PD{t}{} \lr{ \spacegrad \cdot \BF + \inv{c^2} \PD{t}{\phi_\txtm} }
\end{aligned}
\end{equation*}
} % theorem


Also detailed in the book is the multivector formulation of gauge transformation that allows the grade selection operation in
\cref{thm:generalPotential:80} to be removed.

%
% Copyright � 2018 Peeter Joot.  All Rights Reserved.
% Licenced as described in the file LICENSE under the root directory of this GIT repository.
%
\maketheorem{Gauge invariance.}{thm:gaugeTransformation:60}{
The spacetime gradient of a grade \((0,3)\)-multivector \( \Psi \) may be added to a multivector potential
\begin{equation*}
A' = A + \stgrad \Psi,
\end{equation*}
without changing the field.
That is
\begin{equation*}
F
= \gpgrade{\conjstgrad A}{1,2}
= \gpgrade{\conjstgrad A'}{1,2}.
\end{equation*}
} % theorem


We say that we are working in the Lorenz gauge, if the 0,3 grades of \( \conjstgradi A \) are zero, or a transformation that kills those grades is made.
%
% Copyright � 2018 Peeter Joot.  All Rights Reserved.
% Licenced as described in the file LICENSE under the root directory of this GIT repository.
%
\maketheorem{Lorentz gauge transformation.}{thm:gaugeTransformation:140}{
Given any multivector potential \( A \) solution of Maxwell's equation, the transformation
\begin{equation*}
A' = A - \stgrad \Psi,
\end{equation*}
where
\begin{equation*}
\dAlembertian \Psi = \gpgrade{ \conjstgrad A }{0,3},
\end{equation*}
allows Maxwell's equation to be written in wave equation form
\begin{equation*}
\dAlembertian A' = J.
\end{equation*}
} % theorem


Please see the book for proofs and additional details, including the explcit integral solution of \( \Psi \) required for the transformation to the Lorentz gauge.

\subsection{Far field.}
The geometric algebra form for the far field associated with a vector (or dual-vector) potential has a fairly simple coordinate free form.
%
% Copyright � 2018 Peeter Joot.  All Rights Reserved.
% Licenced as described in the file LICENSE under the root directory of this GIT repository.
%
\maketheorem{Far field representation.}{thm:potentialSection_farfield:1}{
Given a spherical wave vector(dual-vector) potentials with representations
%\label{eqn:potentialSection_farfield:2400}
\begin{equation*}
\begin{aligned}
\BA &= \frac{e^{-j k r}}{r} \bcA( \theta, \phi ) \\
\BF &= \frac{e^{-j k r}}{r} \bcF( \theta, \phi ),
\end{aligned}
\end{equation*}
the \textit{far field} (\(r \gg 1 \)) electromagnetic fields are given respectively
%\label{eqn:potentialSection_farfield:2520}{
\begin{equation*}
\begin{aligned}
F &= -j \omega \lr{ 1 + \rcap } \lr{ \rcap \wedge \BA} \\
F &= -j \omega \eta I \lr{ 1 + \rcap } \lr{ \rcap \wedge \BF }.
\end{aligned}
\end{equation*}
} % theorem

Noting that \( \rcap (\rcap \wedge \BA) \) is the rejection of the radial component of \( \BA \) (i.e. is a vector not a multivector) allows the far-field solution to easily be split into electric and magnetic field components if desired.
These are detailed in the book, along with the proof of \cref{thm:potentialSection_farfield:1}.
Also included in the book is an example calculation of \( F = -j \omega \lr{ 1 + \rcap } \lr{ \rcap \wedge \BA} \) for the dipole vector potential.

\subsection{Dielectric and magnetic media.}
The majority of the electromagnetic theory covered in the book focused on fields with the
isotropic constitutive relationships \cref{eqn:freespace:300}.
For more general consitutive relationships the geometric algebra form of Maxwell's equations requires a pair of
multivector equations, fields and sources as follows.
%
% Copyright � 2018 Peeter Joot.  All Rights Reserved.
% Licenced as described in the file LICENSE under the root directory of this GIT repository.
%
\maketheorem{Maxwell's equations in media.}{thm:dielectric:20}{
Maxwell's equations in media are
\begin{equation*}
\begin{aligned}
\gpgrade{ \stgrad F }{0,1} &= J_\txte \\
\gpgrade{ \stgrad G }{2,3} &= I J_\txtm,
\end{aligned}
\end{equation*}
where \( c \) is the group velocity of \( F, G \) in the medium,
the fields are grade 1,2 multivectors
\begin{equation*}
\begin{aligned}
F &= \BD + \frac{I}{c}\BH \\
G &= \BE + I c \BB,
\end{aligned}
\end{equation*}
and the sources are grade 0,1 multivectors
\begin{equation*}
\begin{aligned}
J_\txte &= \rho - \inv{c}\BJ \\
J_\txtm &= c \rho_\txtm - \BM.
\end{aligned}
\end{equation*}
} % theorem


Along with some discussion of solution of these more complicated equations, gauge-like transformations of the fields are discussed.
More work is required to fully flesh out this topic, especially given that \( F, G \) may be directly coupled, allowing neither field to be solved for independently.

\subsection{Boundary value conditions.}
Independent of the techniques used to find the multivector fields \( F, G \) for electromagnetism in matter, we may use \cref{thm:dielectric:20} to easily derive the boundary
value conditions for the fields spanning a surface with surface currents or charges, or even just a discontinuity in the media.
%
% Copyright � 2018 Peeter Joot.  All Rights Reserved.
% Licenced as described in the file LICENSE under the root directory of this GIT repository.
%
\maketheorem{Boundary value relations.}{thm:boundarySurfaceSources:480}{
The difference in the normal and tangential components of the electromagnetic field spanning a surface on which there are
a surface current or surface charge or current densities \( J_\txte = J_{\textrm{es}} \delta(n), J_\txtm = J_{\textrm{ms}} \delta(n) \)
can be related to those surface sources as follows
%\label{eqn:boundarySurfaceSources:420}
\begin{equation*}
\begin{aligned}
\gpgrade{\ncap (F_2 - F_1) }{0,1} &= J_{\textrm{es}} \\
\gpgrade{\ncap (G_2 - G_1) }{2,3} &= I J_{\textrm{ms}},
\end{aligned}
\end{equation*}
where \( F_k = \BD_k + I \BH_k/c, G_k = \BE_k + I c \BB_k, k = 1,2 \) are the fields in the
where \( \ncap = \ncap_2 = -\ncap_1 \) is the outwards facing normal in the second medium.
In terms of the conventional constituent fields, these may be written
%\label{eqn:boundarySurfaceSources:460}
\begin{equation*}
\begin{aligned}
\ncap \cdot \lr{ \BD_2 - \BD_1 } &= \rho_\txts \\
\ncap \cross \lr{ \BH_2 - \BH_1 } &= \BJ_\txts \\
\ncap \cdot \lr{ \BB_2 - \BB_1 } &= \rho_{\textrm{ms}} \\
\ncap \cross \lr{ \BE_2 - \BE_1 } &= -\BM_\txts.
\end{aligned}
\end{equation*}
} % theorem


A simple proof is possible by integrating the pair of Maxwell's equations over the pillbox configuration, allowing the height \( n \) of that pillbox above or below the surface to tend to zero,
and the area of the pillbox top to also tend to zero, using \cref{thm:volumeintegral:100} to transform the multivector integrals to boundary integrals.
Figures and the full proof are available in the book.

Note that in the special case where there are surface charge and current densities along the interface surface, but the media is uniform (\(\epsilon_1 = \epsilon_2, \mu_1 = \mu_2\)), then the field and current relationship has a particularily simple form% \citep{chappell2014geometric}
\begin{dmath}\label{eqn:boundarySurfaceSources:421}
\ncap (F_2 - F_1) = J_s.
\end{dmath}

\section{Conclusions.}

The book produced on this project is really only a starting point, and is in no sense publication ready.
Many more worked examples, problems, figures and computer algebra listings should be added.
In depth applications of derived geometric algebra relationships to problems customarily tackled with separate electric and magnetic field equations should also be incorporated.
There are also theoretical holes, topics covered in any conventional introductory electromagnetism text, that are missing.
Examples include the Fresnel relationships for transmission and reflection at an interface, in depth treatment of waveguides, dipole radiation and motion of charged particles, bound charges, and meta materials to name a few.

That said the book provides a great deal of information in one place, and addresses a fundamental problem for the application of geometric algebra to applied electromagnetism.
While Maxwell's equation in its
geometric algebra form \cref{eqn:quaternion2maxwellWithGA:20} has striking compactness and simplicity, encountering this representation
leaves the student with more questions than answers.
A student needs to know how to work with the representation if it is to be useful.
Electromagnetism has a large set of well developed tools available, and it is not obvious how many of these apply to the GA form of Maxwell's equations.

The book attempted to fill in a number of those conceptual holes, detailing how to work with the multivector Maxwell equation directly, and
how many of the conventional tools we use in electromagnetism (Stokes' and divergence laws, Green's functions, ...) generalize from vector calculus to multivector calculus.
Many of the tools required for the application of geometric algebra to electromagnetism are detailed, avoiding the requirement to try to find answers scattered in the literature,
much of which is not
easily accessible.
A conceptual map is provided that shows the connections between familiar conventional results and their (often simpler) geometric algebra equivalents.
There is a repeated implicit argument that it is simpler and more natural to work with a electromagnetic field \( F = \BE + I \eta \BH \) incorporating both electric and magnetic fields.
Geometric algebra provides compact and interesting ways to obtain the familiar identities and relationships, and also connects ideas that are not neccesarily obviously related in the conventional formalism.
The student is provided with a powerful theoretical framework, without sacrificing the ability to use any familiar conventional tools when desired.

\EndNoBibArticle
