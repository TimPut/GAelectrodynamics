%
% Copyright � 2018 Peeter Joot.  All Rights Reserved.
% Licenced as described in the file LICENSE under the root directory of this GIT repository.
%
%{
\input{../latex/blogpost.tex}
\renewcommand{\basename}{ece2500report}
%\renewcommand{\dirname}{notes/phy1520/}
\renewcommand{\dirname}{notes/ece1228-electromagnetic-theory/}
%\newcommand{\dateintitle}{}
%\newcommand{\keywords}{}

\input{../latex/peeter_prologue_print2.tex}

\usepackage{peeters_layout_exercise}
\usepackage{peeters_braket}
\usepackage{peeters_figures}
%\usepackage{siunitx}
%\usepackage{mhchem} % \ce{}
%\usepackage{macros_bm} % \bcM
%\usepackage{macros_qed} % \qedmarker
\usepackage{txfonts} % \ointclockwise

\beginArtNoToc

\generatetitle{Project report ECE2500.  Geometric Algebra for Electrical Engineers}

\section{Motivation.}
This is the report for an ECE2500 M.Eng project course.

\subsubsection{Goals.}
This project had a few goals
\begin{enumerate}
\item Perform a literature review of applications of geometric algebra
%\footnote{Geometric algebra will be defined precisely later, along with bivector, trivector, multivector and other geometric algebra generalizations of the vector.}
to the study of electromagnetism.
\item Identify the subset of the literature that had direct relevance to electrical engineering.
\item Create a complete, and as compact as possible, introduction of the prerequisites required
for a graduate or advanced undergraduate electrical engineering student to be able to apply
geometric algebra to problems in electromagnetism.
\end{enumerate}

Geometric algebra
generalizes vectors, providing algebraic representations of not just directed line segments, but also points, plane segments, volumes, and higher degree geometric objects (hypervolumes.)
The geometric algebra representation of planes, volumes and hypervolumes requires a vector dot product, a vector multiplication operation, and a generalized addition operation.
The dot product
provides the length of a vector and a test for whether or not any two vectors are perpendicular.
The vector multiplication operation is used to construct
directed plane segments (bivectors),
and directed volumes (trivectors), which are built from the respective products of two or three mutually perpendicular vectors.
The addition operation allows for sums of scalars, vectors, or any products of vectors.  Such a sum is called a multivector.

The power to add scalars, vectors, and products of vectors can be exploited to simplify much of electromagnetism.
In particular, Maxwell's equations for isotropic media can be merged into a single multivector equation
\begin{dmath}\label{eqn:quaternion2maxwellWithGA:20}
\stgrad \lr{ \BE + I c \BB } = \eta\lr{ c \rho - \BJ },
\end{dmath}
where \( \spacegrad \) is the gradient, \( I = \Be_1 \Be_2 \Be_3 \) is the ordered product of the three \R{3} basis vectors, \( c = 1/\sqrt{\mu\epsilon}\) is the group velocity of the medium, \( \eta = \sqrt{\mu/\epsilon} \), \( \BE, \BB \) are the electric and magnetic fields, and
\( \rho \) and \( \BJ \) are the charge and current densities.
We will write this as
\begin{dmath}\label{eqn:ece2500report:40}
\stgrad F = J,
\end{dmath}
where \( F = \BE + I c \BB \) is the combined (multivector) electromagnetic field, and \( J = \eta\lr{ c \rho - \BJ } \) is the multivector current.
As a single PDE, the complete Green's function toolbox may be thrown at
\cref{eqn:ece2500report:40}, inverting Maxwell's equation for the electromagnetic field, given any charge and current density distribution
\begin{dmath}\label{eqn:ece2500report:60}
F(\Bx, t)
= \int dt' dV' G(\Bx, \Bx' ; t, t') J(\Bx', t').
\end{dmath}
Green's functions may also be applied to static and frequency domain field configurations.
Solving for, or working with, the combined field \( F \) shows the
hidden structure behind a number of seemingly
disparate ideas in electromagnetism.
This project explored a number of ideas along these lines.
For example, a
Green's function solutions for the static field configurations simultaneously yields Coulomb's and the Biot-Savart law.
Plane, circular and elliptical waves may be expressed compactly in a multivector form, naturally expressing the mutual perpendicularity of the electric field, magnetic field and the propagation directions, as well
as the relationships between the electric and magnetic field components.
The field energy density and Poynting vectors have a simple multivector form expressed in terms of \( F \) alone.
Calculations of radiation pressure can be performed using only the normal component of what is known as the energy momentum tensor in the conventional representation, which has a particularly compact multivector
representation.

Much of the geometric algebra literature for electrodynamics is presented with a relativistic bias, or assumes high levels of mathematical or physics sophistication.
The aim of this work was an attempt to make the study of electromagnetism using geometric algebra more accessible, especially to an electrical engineering audience.
In particular, this project explored non-relativistic applications of geometric algebra to electromagnetism.
The end product of this project was a fairly small self contained book, titled ``Geometric Algebra for Electrical Engineers''.
This book includes an introduction to Euclidean geometric algebra focused on \R{2} and \R{3} (64 pages), an introduction to geometric calculus and multivector Green's functions (64 pages), applications to electromagnetism (75 pages), and some appendices.
This report summarizes results from this book, omitting most derivations, and attempts to provide an overview that may be used as a road map for the book for further exploration.
Many of the fundamental results of electromagnetism are derived directly from
\cref{eqn:ece2500report:40}, the multivector Maxwell's equation, in a streamlined and compact fashion.
This includes some new results, and many of the existing non-relativistic results from the geometric algebra literature.
As a
conceptual bridge, the book includes many examples of how to extract
familiar conventional results from simpler multivector representations.
Also included are some sample calculations exploiting unique capabilities that geometric algebra provides.  In particular, vectors in a plane may be manipulated much like complex numbers, which has a number of advantages over working with coordinates explicitly.

It is my belief that systematically working through all of the introductory and advanced topics in electromagnetism using geometric algebra
would provide significant insight, as well as a new set of powerful tools and procedures of practical value to the electrical engineer or physics practitioner.
The book produced in this project provides the prerequisite material for such exploration, and some first steps along the path of such an expedition.

\section{Geometric Algebra.}
\section{Geometric Calculus.}
\section{Electromagnetism.}

%}
\EndArticle
