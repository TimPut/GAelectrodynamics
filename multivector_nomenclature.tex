%
% Copyright © 2017 Peeter Joot.  All Rights Reserved.
% Licenced as described in the file LICENSE under the root directory of this GIT repository.
%
A fair amount of nomenclature and notation is unfortunately required before systematically examining the implications of the multivector space axioms that define geometric algebra.

\index{blade}
\index{grade}
\makedefinition{Blade and grade}{def:multiplication:blade}{
A product of \( k \) perpendicular vectors is called a k-blade, or a blade of grade \( k \).
A grade zero blade is a scalar.

The notation \( F \in \bigwedge^k \) is used in the literature to indicate that \( F \) is a blade of grade \( k \).
}

The maximum grade of a multivector is equal to the dimension of the generating vector space.
For example, for a multivector space generated by \R{3}, no k-vector can have grade greater than 3.

Examples of blades with grades 0, 1, 2, and 3 respectively are

\begin{dmath}\label{eqn:multivector:180}
\begin{aligned}
&1 \\
&\Be_1,\quad \Be_2,\quad \Be_3 \\
&\Be_1 \Be_2,\quad \Be_2 \Be_1,\quad \Be_1 \Be_2 + \Be_2 \Be_3 \\
&\Be_1 \Be_2 \Be_3,\quad \Be_1 \Be_3 \Be_2,\quad \Be_1 \Be_4 \Be_2
\end{aligned}
\end{dmath}

Multivectors which can be factored into perpendicular vector products, such as
\begin{dmath}\label{eqn:multiplication:220}
\Be_1 \Be_2 + 3 \Be_1 \Be_3
=
\Be_1 (\Be_2 + 3 \Be_3),
\end{dmath}

are blades.  In contrast, the following grade 2 multivectors

\begin{dmath}\label{eqn:multiplication:240}
\Be_1 \Be_2 + \Be_3 \Be_4,
\end{dmath}

and
\begin{dmath}\label{eqn:multiplication:260}
\Be_1 \Be_2 + \Be_2 \Be_3 + \Be_3 \Be_1,
\end{dmath}

which cannot be factored into two vector products, are not blades.

\index{k-vector}
\makedefinition{k-vector.}{dfn:multivector:kvector}{
A sum of k-blades is called a k-vector.
} % definition

Multivectors are therefore sums of k-vectors with different grades.

All the k-blade examples above are also k-vectors.
K-vectors with grades 2 and 3 are so pervasive that they are given special names.

\index{bivector}
\makedefinition{Bivector.}{dfn:multivector:bivector}{
A bivector, or 2-vector, is a k-vector with grade 2.
} % definition

The product \( \Be_1 \Be_2 \) is a bivector, as is \( \Be_2 \Be_3 + 3 \Be_4 \Be_1 \)
%Each of \( \Be_1 \Be_2, \Be_2 \Be_1, \Be_1 \Be_2 + \Be_2 \Be_3 \), and \( \Be_1 \Be_2 + \Be_3 \Be_4 \) are bivectors.
%All but the last of these represents an oriented plane segment.

\index{trivector}
\makedefinition{Trivector.}{dfn:multivector:trivector}{
A trivector, or 3-vector, is a k-vector with grade 3.
} % definition

%Quantities with higher grades than 3 are not generally given explicit names.
The multivector \( \Be_3 \Be_1 \Be_2 \) is a trivector, as is \( \Be_1 \Be_2 \Be_3 + 3 \Be_5 \Be_4 \Be_1 \).  The latter is not a blade.
%Each of \( \Be_1 \Be_2 \Be_3, \Be_1 \Be_3 \Be_2, \Be_1 \Be_4 \Be_2 \) are trivectors.
% , and represent oriented volumes.

\index{grade selection}
\makedefinition{Grade selection operator}{dfn:gradeselection:gradeselection}{
Given a set of k-vectors \( M_k, k \in [0,N] \), and any multivector of their sum

\begin{equation*}
M = \sum_{i = 0}^N M_i,
\end{equation*}

the grade selection operator is defined as

\begin{equation*}\label{eqn:gradeselection:40}
\gpgrade{M}{k} \equiv M_k.
\end{equation*}

Due to its importance, selection of the (scalar) zero grade is given the shorthand
\begin{equation*}
\gpgradezero{M} \equiv \gpgrade{M}{0} = M_0.
\end{equation*}
}

For example, if \( M = 3 - \Be_3 + 2 \Be_1 \Be_2 \), then
\begin{equation}\label{eqn:gradeselection:80}
\begin{aligned}
\gpgradezero{M} &= 3 \\
\gpgrade{M}{1} &= - \Be_3 \\
\gpgrade{M}{2} &= 2 \Be_1 \Be_2 \\
\gpgrade{M}{3} &= 0.
\end{aligned}
\end{equation}

\index{orthonormal blades}
\makedefinition{Orthonormal product shorthand.}{dfn:multivector:shorthand}{
Given an orthonormal basis \( \setlr{ \Be_1, \Be_2, \cdots } \), a multiple indexed quantity \( \Be_{ij\cdots k} \) should be interpretted as the product (in the same order) of the basis elements with those indexes

\begin{equation*}
\Be_{ij\cdots k} = \Be_i \Be_j \cdots \Be_k.
\end{equation*}
} % definition

For example,

\begin{equation}\label{eqn:multivector:360}
\begin{aligned}
\Be_{12} &= \Be_1 \Be_2 \\
\Be_{123} &= \Be_1 \Be_2 \Be_3 \\
\Be_{23121} &= \Be_2 \Be_3 \Be_1 \Be_2 \Be_1.
\end{aligned}
\end{equation}

\index{pseudoscalar}
\makedefinition{Pseudoscalar.}{def:multiplication:pseudoscalar}{
A blade with grade that matches the dimension of the space.
}

In a two dimensional space \( \Be_1 \Be_2 \) is a pseudoscalar, as is \( 3 \Be_2 \Be_1 \).  In a three dimensional space
\( \Be_3 \Be_1 \Be_2 \) is a pseudoscalar, as is \( - 7 \Be_3 \Be_1 \Be_2 \).
%A pseudoscalar has an implied orientation, which can be
%associated with the handedness of the underlying basis.
It is conventional to refer to

\begin{dmath}\label{eqn:definitions:320}
i = \Be_1 \Be_2,
\end{dmath}

as ``the pseudoscalar'' for a two dimensional space, and to

\begin{dmath}\label{eqn:definitions:340}
I = \Be_1 \Be_2 \Be_3,
\end{dmath}

as ``the pseudoscalar'' for a three dimensional space.


