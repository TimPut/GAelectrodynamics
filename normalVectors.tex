%
% Copyright © 2017 Peeter Joot.  All Rights Reserved.
% Licenced as described in the file LICENSE under the root directory of this GIT repository.
%
The simplest bivectors are products of two different orthonormal vectors.  For \R{3} all such bivectors that can be formed from the basis elements are \( \Be_1 \Be_2, \Be_2 \Be_1, \Be_2 \Be_3, \Be_3 \Be_2, \Be_3 \Be_1, \Be_1 \Be_3 \).

These are not all independent, which an be demonstrated simply by consider the square of the vector \( \Be_1 + \Be_2 \),
as sketched in \cref{fig:unitSum:unitSumFig1}.
\imageFigure{../figures/GAelectrodynamics/unitSumFig1}{\( \Be_1 + \Be_2 \).}{fig:unitSum:unitSumFig1}{0.3}
By the contraction axiom, the square of this vector is \( 2 \)

\begin{dmath}\label{eqn:multivector:420}
\lr{ \Be_1 + \Be_2 }^2 =
\lr{ \Be_1 + \Be_2 } \cdot
\lr{ \Be_1 + \Be_2 }
=
\Be_1 \cdot \Be_1
+
\cancel{\Be_1 \cdot \Be_2}
+
\cancel{\Be_2 \cdot \Be_1}
+
\Be_2 \cdot \Be_2
=
2.
\end{dmath}

Computing this same square by expansion in vector products must give the same result, but for that we get

\begin{dmath}\label{eqn:gaTutorial:80}
(\Be_1 + \Be_2)^2
= (\Be_1 + \Be_2)(\Be_1 + \Be_2)
= \Be_1^2 + \Be_2 \Be_1 + \Be_1 \Be_2 + \Be_2^2
= 1 + \Be_2 \Be_1 + \Be_1 \Be_2 + 1
= 2 + \Be_2 \Be_1 + \Be_1 \Be_2.
\end{dmath}

The right hand side is a mixed grade multivector with grades zero and two, however, it must also equal \( 2 \).
The only possible solution requires that the sum of all the grade two components of this equation are zero

\begin{dmath}\label{eqn:multivector:280}
\Be_2 \Be_1 + \Be_1 \Be_2 = 0,
\end{dmath}

or
%\begin{dmath}\label{eqn:multiplication:140}
\boxedEquation{eqn:multiplication:140}{
\Be_1 \Be_2 = -\Be_1 \Be_2.
}
%\end{dmath}

The same computation could have been performed for any two orthonormal vectors, so we conclude that any interchange of two orthonormal vectors changes the sign.  In general this is true of any normal vectors.

\maketheorem{Anticommutation}{thm:multiplication:anticommutationNormal}{
Let \(\Bu\), and \(\Bv\) be two normal vectors, the product of which \( \Bu \Bv \) is a bivector.
Changing the order of these products toggles the sign of the bivector.

\begin{equation*}
\Bu \Bv = -\Bv \Bu.
\end{equation*}

This sign change on interchange is called anticommutation.
} % theorem

