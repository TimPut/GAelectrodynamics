%
% Copyright � 2018 Peeter Joot.  All Rights Reserved.
% Licenced as described in the file LICENSE under the root directory of this GIT repository.
%
%{
\input{../latex/blogpost.tex}
\renewcommand{\basename}{volumeintegral}
%\renewcommand{\dirname}{notes/phy1520/}
\renewcommand{\dirname}{notes/ece1228-electromagnetic-theory/}
%\newcommand{\dateintitle}{}
%\newcommand{\keywords}{}

\input{../latex/peeter_prologue_print2.tex}

\usepackage{peeters_layout_exercise}
\usepackage{peeters_braket}
\usepackage{peeters_figures}
\usepackage{siunitx}
%\usepackage{mhchem} % \ce{}
%\usepackage{macros_bm} % \bcM
%\usepackage{macros_qed} % \qedmarker
\usepackage{txfonts} % \ointclockwise

\beginArtNoToc

\generatetitle{Volume integral.}
%\chapter{Volume integral.}
\label{chap:volumeintegral}

\paragraph{TODO:}

\paragraph{FIXME: Original notes to mine.}

\index{volume parameterization}
\index{area element}
\index{differential form}
An example parameterization with three parameters, and the corresponding differentials with respect to those parameters, and the outwards normals, are sketched in
\cref{fig:normalsOnVolumeAreaElement:normalsOnVolumeAreaElementFig11}.

\imageFigure{../figures/gabook/normalsOnVolumeAreaElementFig11}{Three parameter volume element.}{fig:normalsOnVolumeAreaElement:normalsOnVolumeAreaElementFig11}{0.4}

Given parameters \( a, b, c \), the differentials along each of the parameterization directions are

\begin{dmath}\label{eqn:volumeintegral:1421}
\begin{aligned}
d\Bx_a &= \PD{a}{\Bx} da = \Bx_a da \\
d\Bx_b &= \PD{b}{\Bx} db = \Bx_b db \\
d\Bx_c &= \PD{c}{\Bx} dc = \Bx_c dc.
\end{aligned}
\end{dmath}

The ``volume'' element for this parameterization (a surface area element) is

\begin{equation}\label{eqn:volumeintegral:1441}
d^3 \Bx
=
d\Bx_a
\wedge
d\Bx_b
\wedge
d\Bx_c
=
da db dc (\Bx_a \wedge \Bx_b \wedge \Bx_c).
\end{equation}

The vector derivative, the projection of the gradient onto the surface at the point of integration (also called the tangent space), now has three components

\begin{dmath}\label{eqn:volumeintegral:1461}
\boldpartial
=
\sum_\mu \Bx^\mu (\Bx_\mu \cdot \spacegrad)
=
\Bx^a \PD{a}{}
+
\Bx^b \PD{b}{}
+
\Bx^c \PD{c}{}
\equiv
\Bx^a \partial_a
+
\Bx^b \partial_b
+
\Bx^c \partial_c.
\end{dmath}

The Stokes integral can be evaluated over this volume element for either scalar fields \( \psi \), vector fields \( \Bf \), or bivector fields \( B \) and takes the form

\begin{subequations}
\label{eqn:volumeintegral:1481}
\begin{equation}\label{eqn:volumeintegral:1501}
\int_V d^3 \Bx \cdot (\boldpartial \wedge \psi) =
\int_V (d^3 \Bx \cdot \boldpartial) \psi
=
\int_{\partial V} d^2 \Bx \psi
\end{equation}
\begin{equation}\label{eqn:volumeintegral:1521}
\int_V d^3 \Bx \cdot (\boldpartial \wedge \Bf) =
\int_V (d^3 \Bx \cdot \boldpartial) \cdot \Bf
=
\int_{\partial V} d^2 \Bx \cdot \Bf
\end{equation}
\begin{equation}\label{eqn:volumeintegral:1541}
\int_V d^3 \Bx \cdot (\boldpartial \wedge B) =
\int_V (d^3 \Bx \cdot \boldpartial) \cdot B
=
\int_{\partial V} d^2 \Bx \cdot B.
\end{equation}
\end{subequations}

When working with \R{3} vector spaces, \( \boldpartial = \spacegrad \), but in higher dimensional spaces, the gradient can also be substituted above due using the same arguments about projection onto the tangent space.

An explicit value for the differential form of the boundary integral is desired and can be obtained from the mnemonic \cref{eqn:statement:1561}

\begin{dmath}\label{eqn:volumeintegral:1581}
\sum_i d^3 \Bx \cdot \Bx^i
=
\sum_i da db dc \lr{ \Bx_a \wedge \Bx_b \wedge \Bx_c } \cdot \Bx^i
=
\sum_i da db dc \lr{
\Bx_a \wedge \Bx_b +
\Bx_b \wedge \Bx_c +
\Bx_c \wedge \Bx_a }.
\end{dmath}

The bounding form for the three parameter volume is therefore

\begin{dmath}\label{eqn:volumeintegral:1601}
d^2 \Bx
=
d\Bx_a \wedge d\Bx_b +
d\Bx_b \wedge d\Bx_c +
d\Bx_c \wedge d\Bx_a.
\end{dmath}

%}
%\EndArticle
\EndNoBibArticle
