%
% Copyright � 2018 Peeter Joot.  All Rights Reserved.
% Licenced as described in the file LICENSE under the root directory of this GIT repository.
%
%{
\input{../latex/blogpost.tex}
\renewcommand{\basename}{volumeintegral}
%\renewcommand{\dirname}{notes/phy1520/}
\renewcommand{\dirname}{notes/ece1228-electromagnetic-theory/}
%\newcommand{\dateintitle}{}
%\newcommand{\keywords}{}

\input{../latex/peeter_prologue_print2.tex}

\usepackage{peeters_layout_exercise}
\usepackage{peeters_braket}
\usepackage{peeters_figures}
\usepackage{siunitx}
%\usepackage{mhchem} % \ce{}
%\usepackage{macros_bm} % \bcM
%\usepackage{macros_qed} % \qedmarker
\usepackage{txfonts} % \ointclockwise

\beginArtNoToc

\generatetitle{Volume integral.}
%\chapter{Volume integral.}
\label{chap:volumeintegral}

\paragraph{TODO:}

A three parameter curve, and the corresponding differentials with respect to those parameters, is sketched in
\cref{fig:normalsOnVolumeAreaElement:normalsOnVolumeAreaElementFig11}.

\imageFigure{../figures/gabook/normalsOnVolumeAreaElementFig11}{Three parameter volume element.}{fig:normalsOnVolumeAreaElement:normalsOnVolumeAreaElementFig11}{0.4}

Given parameters \( u_1, u_2, u_3 \), we can denote the differentials along each of the parameterization directions as

\begin{dmath}\label{eqn:volumeintegral:100}
\begin{aligned}
d\Bx_1 &= \PD{u_1}{\Bx} du_1 = \Bx_1 du_1 \\
d\Bx_2 &= \PD{u_2}{\Bx} du_2 = \Bx_2 du_2 \\
d\Bx_3 &= \PD{u_3}{\Bx} du_3 = \Bx_3 du_3 \\
\end{aligned}
\end{dmath}

The trivector valued volume element for this parameterization is

\begin{equation}\label{eqn:volumeintegral:120}
d^3 \Bx
=
d\Bx_1 \wedge
d\Bx_1 \wedge
d\Bx_1
=
d^3 u\, (\Bx_1 \wedge \Bx_2 \wedge \Bx_3),
\end{equation}

where \( d^3 u = du_1 du_2 du_3 \).
The vector derivative, the projection of the gradient onto the volume at the point of integration (also called the tangent space), now has three components

\begin{dmath}\label{eqn:volumeintegral:200}
\boldpartial
=
\sum_i \Bx^i (\Bx_i \cdot \spacegrad)
=
\Bx^1 \PD{u_1}{}
+
\Bx^2 \PD{u_2}{}
+
\Bx^3 \PD{u_3}{}
\equiv
\Bx^1 \partial_1
+
\Bx^2 \partial_2
+
\Bx^3 \partial_3
\end{dmath}

We define a multivector volume integral by

\makedefinition{Multivector volume integral.}{dfn:volumeintegral:100}{
Given an connected volume \( V \) parameterized by two parameters, and multivector functions \( F, G \), we define the volume integral as
\begin{equation*}
\int_V F d^3\Bx \boldpartial G
\equiv
\int_V \lr{ F d^3\Bx \lboldpartial} G
+
\int_V F d^3\Bx \lr{ \rboldpartial G },
\end{equation*}
where the three parameter differential form \( d^3 \Bx = d^3 u\, \Bx_1 \wedge \Bx_2 \wedge \Bx_3, d^3 u = du_1 du_2 du_3 \) varies over the volume.
} % definition

\maketheorem{Multivector volume integral.}{thm:volumeintegral:100}{
Given an connected volume \( V \) parameterized by two parameters, and multivector functions \( F, G \), the volume integral
\begin{equation*}
\int_V F d^3\Bx \boldpartial G
= \ointclockwise_{\partial V} F d^2\Bx G,
\end{equation*}

where \( \partial V \) is the boundary of the volume \( V \).
} % theorem

To see why this works, we would first like to reduce the product of the volume element and the vector derivative

\begin{dmath}\label{eqn:volumeintegral:300}
d^3\Bx \boldpartial
=
d^3 u\, \lr{ \Bx_1 \wedge \Bx_2 \wedge \Bx_3 } \lr{ \Bx^1 \partial_1 + \Bx^2 \partial_2 + \Bx^3 \partial_3 }.
\end{dmath}

Since all \( \Bx^i \) lie within \( \Span \setlr{ \Bx_1, \Bx_2, \Bx_3 } \), this multivector product has only a vector grade.  That is

\begin{dmath}\label{eqn:volumeintegral:320}
\lr{ \Bx_1 \wedge \Bx_2 \wedge \Bx_3 } \Bx^i
=
\lr{ \Bx_1 \wedge \Bx_2 \wedge \Bx_3 } \cdot \Bx^i
+
\cancel{ \lr{ \Bx_1 \wedge \Bx_2 \wedge \Bx_3 } \wedge \Bx^i },
\end{dmath}

for all \( \Bx^i \).  This product reduces to
\begin{dmath}\label{eqn:volumeintegral:n}
\begin{aligned}
\lr{ \Bx_2 \wedge \Bx_3 \wedge \Bx_1 } \Bx^1 &= \Bx_2 \wedge \Bx_3 \\
\lr{ \Bx_3 \wedge \Bx_1 \wedge \Bx_2 } \Bx^2 &= \Bx_3 \wedge \Bx_1 \\
\lr{ \Bx_1 \wedge \Bx_2 \wedge \Bx_3 } \Bx^3 &= \Bx_1 \wedge \Bx_2 \\
\end{aligned}
\end{dmath}

Inserting this into the volume integral, we find

\begin{dmath}\label{eqn:volumeintegral:380}
\int_V F d^3\Bx \boldpartial G
=
\int_V \lr{ F d^3\Bx \lboldpartial} G
+
\int_V F d^3\Bx \lr{ \rboldpartial G }
=
\int_V d^3 u\, \lr{
   (\partial_1 F) \Bx_2 \wedge \Bx_3 G
   +
   (\partial_2 F) \Bx_3 \wedge \Bx_1 G
   +
   (\partial_3 F) \Bx_1 \wedge \Bx_2 G
}
+
\int_V d^3 u\, \lr{
   F \Bx_2 \wedge \Bx_3 (\partial_1 G)
   +
   F \Bx_3 \wedge \Bx_1 (\partial_2 G)
   +
   F \Bx_1 \wedge \Bx_2 (\partial_3 G)
}
=
\int_V d^3 u\, \lr{
   \partial_1 (F \Bx_2 \wedge \Bx_3 G)
   +
   \partial_2 (F \Bx_3 \wedge \Bx_1 G)
   +
   \partial_3 (F \Bx_1 \wedge \Bx_2 G)
}
\end{dmath}

We are able to pull out the partials because \( \Bx_i = \PDi{u_i}{\Bx}, \Bx_j = \PD{u_j}{\Bx}\) are computed with the parameters \( u_k \) fixed for each \( i \ne j \ne k \).

This leaves three perfect differentials, which can integrated separately, giving

\begin{dmath}\label{eqn:volumeintegral:400}
\int_V F d^3\Bx \boldpartial G
=
\int du_2 du_3
\evalbar{ \lr{ F \Bx_2 \wedge \Bx_3 G } }{\Delta u_1}
+
\int du_3 du_1
\evalbar{ \lr{ F \Bx_3 \wedge \Bx_1 G } }{\Delta u_2}
+
\int du_1 du_2
\evalbar{ \lr{ F \Bx_1 \wedge \Bx_2 G } }{\Delta u_3}.
\end{dmath}

With the aid of a geometrical model (FIXME: picture), or in a pinch
\cref{fig:normalsOnVolumeAreaElement:normalsOnVolumeAreaElementFig11}
, it is possible to convince oneself that each of these surface elements match up to a uniformly oriented clockwise oriented surface.

%%%\subsection{Two parameter Stokes' theorem.}
%%%
%%%Two special cases of \cref{thm:volumeintegral:100} when scalar and vector functions are integrated over a volume.  For scalar functions we have
%%%
%%%\maketheorem{Surface integral of scalar function (Stokes').}{thm:volumeintegral:420}{
%%%Given a scalar function \( f(\Bx) \) its volume integrals is given by
%%%\begin{equation*}
%%%\int_V d^3 \Bx \cdot \boldpartial f =
%%%\int_V d^3 \Bx \cdot \spacegrad f = \ointclockwise_{\partial V} d\Bx f.
%%%\end{equation*}
%%%In \R{3} this can be written as
%%%\begin{equation*}
%%%\int_V dA \ncap \cross \spacegrad f = \ointctrclockwise_{\partial V} d\Bx f,
%%%\end{equation*}
%%%where \( \ncap \) is the outwards normal specified by \( d^2 \Bx = I \ncap dA \).
%%%} % theorem
%%%
%%%To show the first part, we can split the (multivector) volume integral into vector and trivector grades
%%%
%%%\begin{dmath}\label{eqn:volumeintegral:440}
%%%\int_V d^3\Bx \boldpartial f
%%%=
%%%\int_V d^3\Bx \cdot \boldpartial f
%%%+
%%%\int_V d^2\Bx \wedge \boldpartial f.
%%%\end{dmath}
%%%
%%%Since \( \Bx^a, \Bx^b \) both lie in the span of \( \setlr{ \Bx_a, \Bx_b } \),
%%%\( d^2\Bx \wedge \boldpartial = 0 \), killing the second integral in \cref{eqn:volumeintegral:440}.
%%%If the gradient is decomposed into its projection along the tangent
%%%space (the vector derivative) and its perpendicular components, only the vector derivative components of the
%%%gradient contribute to its dot product with the volume element.  That is
%%%
%%%\begin{dmath}\label{eqn:volumeintegral:460}
%%%d^2 \Bx \cdot \spacegrad
%%%=
%%%d^2 \Bx \cdot \lr{ \Bx^a \partial_a + \Bx^b \partial_b + \cdots }
%%%=
%%%d^2 \Bx \cdot \lr{ \Bx^a \partial_a + \Bx^b \partial_b }
%%%=
%%%d^2 \Bx \cdot \boldpartial.
%%%\end{dmath}
%%%
%%%This means that for a scalar function
%%%
%%%\begin{dmath}\label{eqn:volumeintegral:480}
%%%\int_V d^2\Bx \boldpartial f
%%%=
%%%\int_V d^2\Bx \cdot \spacegrad f.
%%%\end{dmath}
%%%
%%%The second part of the theorem follows by grade selection, and application of a duality transformation for the volume element
%%%
%%%\begin{dmath}\label{eqn:volumeintegral:500}
%%%d^2 \Bx \cdot \spacegrad f
%%%=
%%%\gpgradeone{ d^2 \Bx \spacegrad f }
%%%=
%%%dA \gpgradeone{ I \ncap \spacegrad f }
%%%=
%%%dA \gpgradeone{ I \lr{ \ncap \cdot \spacegrad f + I \ncap \cross \spacegrad f} }
%%%=
%%%-dA \ncap \cross \spacegrad f.
%%%\end{dmath}
%%%
%%%back substitution of \cref{eqn:volumeintegral:500} completes the proof of \cref{thm:volumeintegral:420}.
%%%
%%%For vector functions we have
%%%
%%%\maketheorem{Surface integral of a vector function (Stokes').}{thm:volumeintegral:500}{
%%%Given a vector function \( \Bf(\Bx) \) the volume integral is given by
%%%\begin{equation*}
%%%\int_V d^2 \Bx \cdot (\spacegrad \wedge \Bf) = \ointclockwise_{\partial V} d\Bx \cdot \Bf.
%%%\end{equation*}
%%%In \R{3} this can be written as
%%%\begin{equation*}
%%%\int_V dA \ncap \cdot \lr{ \spacegrad \cross \Bf} = \ointctrclockwise_{\partial V} d\Bx \cdot \Bf,
%%%\end{equation*}
%%%where \( \ncap \) is the outwards normal specified by \( d^2 \Bx = I \ncap dA \).
%%%} % theorem
%%%
%%%This follows by setting \( F = 1, G = \Bf \) in \cref{thm:volumeintegral:100} and selecting the scalar grade.  In particular we may form the
%%%scalar selection of \( d^2 \Bx \boldpartial \Bf \) in two different ways.  The first is
%%%
%%%\begin{dmath}\label{eqn:volumeintegral:520}
%%%\gpgradezero{ d^2 \Bx \boldpartial \Bf }
%%%=
%%%\gpgradezero{ (d^2 \Bx \cdot \boldpartial + d^2 \Bx \wedge \boldpartial ) \Bf }
%%%\end{dmath}
%%%
%%%%The \( d^2 \Bx \wedge \boldpartial \) product with \( \Bf \) has only trivector and quad-vector components (the latter is zero in \R{3}), so its scalar grade selection is zero, and we are left with
%%%
%%%\begin{dmath}\label{eqn:volumeintegral:540}
%%%\gpgradezero{ d^2 \Bx \boldpartial \Bf }
%%%=
%%%(d^2 \Bx \cdot \boldpartial) \cdot \Bf
%%%=
%%%(d^2 \Bx \cdot \spacegrad) \cdot \Bf,
%%%\end{dmath}
%%%
%%%where we have used \cref{eqn:volumeintegral:460} again.  This product can also be written as
%%%
%%%\begin{dmath}\label{eqn:volumeintegral:560}
%%%(d^2 \Bx \cdot \spacegrad) \cdot \Bf
%%%=
%%%\gpgradezero{ (d^2 \Bx \cdot \spacegrad) \Bf }
%%%=
%%%\gpgradezero{ (d^2 \Bx \spacegrad - d^2 \Bx \wedge \spacegrad) \Bf }
%%%=
%%%\gpgradezero{ d^2 \Bx \spacegrad \Bf }
%%%=
%%%\gpgradezero{ d^2 \Bx \lr{ \cancel{ \spacegrad \cdot \Bf } + \spacegrad \wedge \Bf } }
%%%=
%%%d^2 \Bx \cdot \lr{ \spacegrad \wedge \Bf }.
%%%\end{dmath}
%%%
%%%\begin{dmath}\label{eqn:volumeintegral:580}
%%%\ointclockwise_{\partial V} d\Bx \cdot \Bf
%%%=
%%%\gpgradezero{ \int_V d^2\Bx \boldpartial \Bf }
%%%=
%%%\int_V \lr{ d^2\Bx \cdot \spacegrad } \cdot \Bf
%%%=
%%%\int_V d^2\Bx \cdot \lr{ \spacegrad \wedge \Bf },
%%%\end{dmath}
%%%
%%%as claimed.  In particular in \R{3}, we have
%%%
%%%\begin{dmath}\label{eqn:volumeintegral:600}
%%%d^2\Bx \cdot \lr{ \spacegrad \wedge \Bf }
%%%=
%%%dA \gpgradezero{ I \ncap I \lr{ \spacegrad \cross \Bf } }
%%%=
%%%-dA \ncap \cdot \lr{ \spacegrad \cross \Bf }.
%%%\end{dmath}
%%%
%%%Substitution into \cref{eqn:volumeintegral:580} proves the last part of \cref{thm:volumeintegral:500}.
%%%

\paragraph{FIXME: Original notes to mine.}

\index{volume parameterization}
\index{area element}
\index{differential form}
An example parameterization with three parameters, and the corresponding differentials with respect to those parameters, and the outwards normals, are sketched in
(cut)

Given parameters \( a, b, c \), the differentials along each of the parameterization directions are

\begin{dmath}\label{eqn:volumeintegral:1421}
\begin{aligned}
d\Bx_a &= \PD{a}{\Bx} da = \Bx_a da \\
d\Bx_b &= \PD{b}{\Bx} db = \Bx_b db \\
d\Bx_c &= \PD{c}{\Bx} dc = \Bx_c dc.
\end{aligned}
\end{dmath}

The ``volume'' element for this parameterization (a surface area element) is

\begin{equation}\label{eqn:volumeintegral:1441}
d^3 \Bx
=
d\Bx_a
\wedge
d\Bx_b
\wedge
d\Bx_c
=
da db dc (\Bx_a \wedge \Bx_b \wedge \Bx_c).
\end{equation}

The vector derivative, the projection of the gradient onto the surface at the point of integration (also called the tangent space), now has three components

\begin{dmath}\label{eqn:volumeintegral:1461}
\boldpartial
=
\sum_\mu \Bx^\mu (\Bx_\mu \cdot \spacegrad)
=
\Bx^a \PD{a}{}
+
\Bx^b \PD{b}{}
+
\Bx^c \PD{c}{}
\equiv
\Bx^a \partial_a
+
\Bx^b \partial_b
+
\Bx^c \partial_c.
\end{dmath}

The Stokes integral can be evaluated over this volume element for either scalar fields \( \psi \), vector fields \( \Bf \), or bivector fields \( B \) and takes the form

\begin{subequations}
\label{eqn:volumeintegral:1481}
\begin{equation}\label{eqn:volumeintegral:1501}
\int_V d^3 \Bx \cdot (\boldpartial \wedge \psi) =
\int_V (d^3 \Bx \cdot \boldpartial) \psi
=
\int_{\partial V} d^2 \Bx \psi
\end{equation}
\begin{equation}\label{eqn:volumeintegral:1521}
\int_V d^3 \Bx \cdot (\boldpartial \wedge \Bf) =
\int_V (d^3 \Bx \cdot \boldpartial) \cdot \Bf
=
\int_{\partial V} d^2 \Bx \cdot \Bf
\end{equation}
\begin{equation}\label{eqn:volumeintegral:1541}
\int_V d^3 \Bx \cdot (\boldpartial \wedge B) =
\int_V (d^3 \Bx \cdot \boldpartial) \cdot B
=
\int_{\partial V} d^2 \Bx \cdot B.
\end{equation}
\end{subequations}

When working with \R{3} vector spaces, \( \boldpartial = \spacegrad \), but in higher dimensional spaces, the gradient can also be substituted above due using the same arguments about projection onto the tangent space.

(FIXME: LABEL)
An explicit value for the differential form of the boundary integral is desired and can be obtained from the mnemonic \cref{eqn:statement:1561}

\begin{dmath}\label{eqn:volumeintegral:1581}
\sum_i d^3 \Bx \cdot \Bx^i
=
\sum_i da db dc \lr{ \Bx_a \wedge \Bx_b \wedge \Bx_c } \cdot \Bx^i
=
\sum_i da db dc \lr{
\Bx_a \wedge \Bx_b +
\Bx_b \wedge \Bx_c +
\Bx_c \wedge \Bx_a }.
\end{dmath}

The bounding form for the three parameter volume is therefore

\begin{dmath}\label{eqn:volumeintegral:1601}
d^2 \Bx
=
d\Bx_a \wedge d\Bx_b +
d\Bx_b \wedge d\Bx_c +
d\Bx_c \wedge d\Bx_a.
\end{dmath}

%}
%\EndArticle
\EndNoBibArticle
