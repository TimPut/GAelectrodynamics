%
% Copyright © 2017 Peeter Joot.  All Rights Reserved.
% Licenced as described in the file LICENSE under the root directory of this GIT repository.
%
%
% Copyright � 2016 Peeter Joot.  All Rights Reserved.
% Licenced as described in the file LICENSE under the root directory of this GIT repository.
%
\index{reverse}
\makedefinition{Reverse}{dfn:reverse:1}{

Let \( A \) be a multivector with j multivector factors,
\( A = B_1 B_2 \cdots B_j \),
not necessarily normal.
The reverse \( A^\dagger \), or reversion, of this multivector \( A \) is
\begin{equation*}
A^\dagger = B_j^\dagger B_{j-1}^\dagger \cdots B_1^\dagger.
\end{equation*}
Scalars and vectors are their own reverse, and
the reverse of a sum of multivectors is the sum of the reversions of its summands.
} % definition

Examples:
\begin{dmath}\label{eqn:reverseDefined:21}
\begin{aligned}
\lr{ 1 + 2 \Be_{12} + 3 \Be_{321} }^\dagger &= 1 + 2 \Be_{21} + 3 \Be_{123} \\
\lr{ (1 + \Be_1)(\Be_{23} - \Be_{12} }^\dagger &= (\Be_{32} + \Be_{12})(1 + \Be_1).
\end{aligned}
\end{dmath}


The reverse of a k-blade has useful closed form expression

\maketheorem{Reverse of k-blade.}{thm:reverse:kBlade}{
The reverse of a k-blade \( A_k = \Ba_1 \Ba_2 \cdots \Ba_k \) is given by
\begin{equation*}
A_k^\dagger = (-1)^{k(k-1)/2} A_k.
\end{equation*}
} % theorem

This can be proven by successive interchange of the factors

\begin{dmath}\label{eqn:scalarPermutation:81}
\begin{aligned}
A_k^\dagger
&= \Ba_k \Ba_{k-1} \cdots \Ba_1 \\
&= (-1)^{k-1} \Ba_1 \Ba_k \Ba_{k-1} \cdots \Ba_2 \\
&= (-1)^{k-1} (-1)^{k-2} \Ba_1 \Ba_2 \Ba_k \Ba_{k-1} \cdots \Ba_3 \\
&\qquad \vdots \\
&= (-1)^{k-1} (-1)^{k-2} \cdots (-1)^1 \Ba_1 \Ba_2 \cdots \Ba_k,
&= (-1)^{k(k-1)/2} \Ba_1 \Ba_2 \cdots \Ba_k. \qquad \qedmarker
\end{aligned}
\end{dmath}

A special, but important case, is the reverse of the \R{3} pseudoscalar, which is negated by reversion

\boxedEquation{eqn:reverse:103}{
I^\dagger = -I.
}

