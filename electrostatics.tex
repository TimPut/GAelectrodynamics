%
% Copyright © 2017 Peeter Joot.  All Rights Reserved.
% Licenced as described in the file LICENSE under the root directory of this GIT repository.
%
The study of
field and charge distributions that are independent of time is called electrostatics.
Under these conditions, Maxwell's equations \cref{eqn:freespace:99} are simplified considerably

\begin{subequations}
\label{eqn:electrostatics:99}
\begin{dmath}\label{eqn:electrostatics:100}
\spacegrad \cross \BE = 0
\end{dmath}
\begin{dmath}\label{eqn:electrostatics:120}
\spacegrad \cross \BB = 0
\end{dmath}
\begin{dmath}\label{eqn:electrostatics:140}
\spacegrad \cdot \BE = \frac{\rho}{\epsilon_0}
\end{dmath}
\begin{dmath}\label{eqn:electrostatics:160}
\spacegrad \cdot \BB = 0.
\end{dmath}
\end{subequations}

All the complicated coupling of the electric and magnetic fields is eliminated, and the only source term remaining is a time independent charge density \( \rho = \rho(\Bx) \).

Utilizing \cref{eqn:SimpleProducts2:1640}, the geometric product of the gradient \( \spacegrad \) with a vector \( \Bb \) is

\begin{dmath}\label{eqn:electrostatics:240}
\spacegrad \Bb = \spacegrad \cdot \Bb + I(\spacegrad \cross \Bb).
\end{dmath}

\Cref{eqn:electrostatics:240} can be used to rewrite the electrostatic Maxwell equations (\cref{eqn:electrostatics:99}), as a pair of multivector gradient equations

\begin{subequations}
\label{eqn:electrostatics:360}
\begin{equation}\label{eqn:electrostatics:380}
\spacegrad \BE = \frac{\rho}{\epsilon_0}
\end{equation}
\begin{equation}\label{eqn:electrostatics:400}
\spacegrad \BB = 0.
\end{equation}
\end{subequations}

\subsection{Enclosed charge.}

The charge in a volume can be related to the electric field by integrating \cref{eqn:electrostatics:380}

\begin{dmath}\label{eqn:electrostatics:420}
\int_V d^3 \Bx \spacegrad \BE = \inv{\epsilon_0} \int_V d^3 \Bx \rho(\Bx).
\end{dmath}

This is an oriented integral, where \( d^3 \Bx \) is a pseudoscalar volume element, such as
\( d^3 \Bx = (\Be_1 dx) \wedge (\Be_2 dy) \wedge (\Be_3 dz) = I dx dy dz \).

The LHS integral can be evaluated using the fundamental theorem \cref{thm:fundamentalTheoremOfCalculus:1}

\begin{dmath}\label{eqn:electrostatics:461}
\int_{\partial V} d^2 \Bx \BE = \frac{I}{\epsilon_0} \int_V dV \rho(\Bx).
\end{dmath}

An outward normal \( \ncap \) can be used to
parameterize the bivector surface area element \( d^2 \Bx = I \ncap dA \), which allows the pseudoscalar factors on both
sides to be cancelled

%\begin{dmath}\label{eqn:electrostatics:460}
\boxedEquation{eqn:electrostatics:460}{
\int_{\partial V} dA \ncap \BE = \frac{1}{\epsilon_0} \int_V dV \rho(\Bx).
}
%\end{dmath}

This is a multivector equation which must be simultaneously satisfied by its scalar and bivector components

\begin{subequations}
\label{eqn:electrostatics:481}
\begin{dmath}\label{eqn:electrostatics:501}
\int_{\partial V} dA \ncap \cdot \BE = \frac{1}{\epsilon_0} \int_V dV \rho(\Bx)
\end{dmath}
\begin{dmath}\label{eqn:electrostatics:521}
\int_{\partial V} dA \ncap \wedge \BE = 0.
\end{dmath}
\end{subequations}

The first equation is the familiar relationship between the divergence and the enclosed charge, which could have been derived from \cref{eqn:electrostatics:140} directly.
The second provides a constraint on the tangential components of the field with respect to the enclosed volume, and could have been derived from
\cref{eqn:electrostatics:100} directly.
The multivector equation \cref{eqn:electrostatics:460} encodes both of these relationships, simultaneously incorporating the contributions of the Maxwell divergence and curl equations for the electric field, relating both to the enclosed charge.

\subsection{Electric potential.}

If a gradient representation of the electric field is assumed

\begin{dmath}\label{eqn:electrostatics:561}
\BE(\Bx) = -\spacegrad \phi(\Bx),
\end{dmath}

then
inserting
this assumed representation into \cref{eqn:electrostatics:380} provides the
Poisson equation directly

%\begin{dmath}\label{eqn:electrostatics:581}
\boxedEquation{eqn:electrostatics:581}{
\spacegrad^2 \phi = -\frac{\rho}{\epsilon_0}.
}
%\end{dmath}

%This has solution \cref{eqn:electrostatics:601} but can also be solved using
\subsection{Inverting the gradient equations.}

From \cref{thm:gradientGreensFunctionEuclidean:1}, the
\R{3} Green's function for the gradient (on an infinite spherical bounding surface) is

\begin{dmath}\label{eqn:electrostatics:260}
G(\Bx, \Bx') = \inv{4 \pi} \frac{\Bx - \Bx'}{\Norm{\Bx - \Bx'}^3}.
\end{dmath}

The Green's function convolution that inverts the electric field gradient equation is
\begin{dmath}\label{eqn:electrostatics:621}
\BE(\Bx)
= \int_V dV' G(\Bx, \Bx') \spacegrad' \BE(\Bx')
= \int_V dV' G(\Bx, \Bx') \lr{ \inv{\epsilon_0}\rho(\Bx') }
= \inv{4\pi} \int_V dV' \frac{\Bx - \Bx'}{ \Abs{\Bx - \Bx'}^3 } \lr{ \inv{\epsilon_0}\rho(\Bx') },
\end{dmath}

or
%\begin{dmath}\label{eqn:electrostatics:340}
\boxedEquation{eqn:electrostatics:340}{
\BE(\Bx) =
\inv{4 \pi \epsilon_0} \int dV' \rho(\Bx') \frac{\Bx - \Bx'}{\Norm{\Bx - \Bx'}^3},
}
%\end{dmath}

which is Coulomb's law.

The convolution for the magnetic field is trivial
\begin{dmath}\label{eqn:electrostatics:641}
\BB(\Bx)
= \int_V dV' G(\Bx, \Bx') \spacegrad' \BB(\Bx')
= \int_V dV' G(\Bx, \Bx') (0),
\end{dmath}

so the magnetic field is zero everywhere
\begin{dmath}\label{eqn:electrostatics:320}
\BB(\Bx) = 0.
\end{dmath}

%Question: would a non-zero magnetic field solution be possible if a Green's function for a finite bounded surface were to be used instead?

\subsection{Poisson equation solution.}

Provided \( \Bx \ne \Bx' \), it is simple to show that (\cref{problem:electrostatics:gradrelation})

\begin{dmath}\label{eqn:electrostatics:541}
\spacegrad \Norm{\Bx - \Bx'}^k = k (\Bx - \Bx') \Norm{ \Bx - \Bx' }^{k-2}.
\end{dmath}

In particular

\begin{dmath}\label{eqn:electrostatics:542}
\spacegrad \inv{\Norm{\Bx - \Bx'}} = - \frac{(\Bx - \Bx')}{\Norm{ \Bx - \Bx' }^{3}}.
\end{dmath}

Inserting \cref{eqn:electrostatics:542} into Coulomb's law \cref{eqn:electrostatics:340} gives
\begin{dmath}\label{eqn:electrostatics:661}
\BE(\Bx)
=
-\inv{4 \pi \epsilon_0} \int dV' \rho(\Bx') \spacegrad \inv{\Norm{\Bx - \Bx'}}
=
- \spacegrad \inv{4 \pi \epsilon_0} \int dV' \frac{\rho(\Bx')}{\Norm{\Bx - \Bx'}},
\end{dmath}

which implicitly provides the solution to the Poisson equation \cref{eqn:electrostatics:581}

%\begin{dmath}\label{eqn:electrostatics:601}
\boxedEquation{eqn:electrostatics:601}{
\phi(\Bx) = \inv{4 \pi \epsilon_0} \int dV' \frac{ \rho(\Bx') }{\Norm{\Bx - \Bx'}}.
}
%\end{dmath}

\subsection{Problems.}

\makeproblem{Gradients relation.}{problem:electrostatics:gradrelation}{
Prove \cref{eqn:electrostatics:541}.
} % problem

