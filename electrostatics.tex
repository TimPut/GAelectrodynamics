The study of
field and charge distributions that are independent of time is called electrostatics.
Under these conditions, Maxwell's equations are simplified considerably

\begin{subequations}
\label{eqn:electrostatics:99}
\begin{dmath}\label{eqn:electrostatics:100}
\spacegrad \cross \BE = - \cancel{\PD{t}{\BB}}
\end{dmath}
\begin{dmath}\label{eqn:electrostatics:120}
\spacegrad \cross \BB = \mu_0 \lr{ \cancel{\BJ} + \epsilon_0 \cancel{\PD{t}{\BE}} }
\end{dmath}
\begin{dmath}\label{eqn:electrostatics:140}
\spacegrad \cdot \BE = \frac{\rho}{\epsilon_0}
\end{dmath}
\begin{dmath}\label{eqn:electrostatics:160}
\spacegrad \cdot \BB = 0.
\end{dmath}
\end{subequations}

All the complicated coupling of the electric and magnetic fields is eliminated, and the only source term remaining is a time independent charge density \( \rho = \rho(\Bx) \).

Utilizing \cref{eqn:SimpleProducts2:1640}, the geometric product of the gradient \( \spacegrad \) with a vector \( \Bb \) is

\begin{dmath}\label{eqn:electrostatics:240}
\spacegrad \Bb = \spacegrad \cdot \Bb + I(\spacegrad \cross \Bb).
\end{dmath}

\Cref{eqn:electrostatics:240} can be used to rewrite \cref{eqn:electrostatics:99} as a pair of multivector gradient equations

\begin{subequations}
\label{eqn:electrostatics:180}
\begin{equation}\label{eqn:electrostatics:200}
\spacegrad \BE = \frac{\rho}{\epsilon_0}
\end{equation}
\begin{equation}\label{eqn:electrostatics:220}
\spacegrad \BB = 0.
\end{equation}
\end{subequations}

From \cref{thm:gradientGreensFunctionEuclidean:1}, the
\R{3} Green's function for the gradient (on an infinite spherical bounding surface) is

\begin{dmath}\label{eqn:electrostatics:260}
G(\Bx, \Bx') = \inv{4 \pi} \frac{\Bx - \Bx'}{\Norm{\Bx - \Bx'}^3}.
\end{dmath}

This allows our two electrostatics Maxwell equations to be inverted directly

\boxedEquation{eqn:electrostatics:280}{
\begin{aligned}
\BE(\Bx) &=
\inv{4 \pi} \int dV' \rho(\Bx') \frac{\Bx - \Bx'}{\Norm{\Bx - \Bx'}^3} \\
\BB(\Bx) &= 0.
\end{aligned}
}

The electric field solution of \cref{eqn:electrostatics:280} is Coulomb's law, which is often the starting point of electrostatic analysis.
