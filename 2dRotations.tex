%
% Copyright © 2017 Peeter Joot.  All Rights Reserved.
% Licenced as described in the file LICENSE under the root directory of this GIT repository.
%

\index{rotation}
Plotting \cref{eqn:SimpleProducts2:180}, as in
\cref{fig:rotationOfe1:rotationOfe1Fig1},
 shows that multiplication by \( i \) rotates the \R{2} basis vectors by \( \pi/2 \) radians,
with the
rotation direction dependant on the order of multiplication.

\imageTwoFigures
{../figures/GAelectrodynamics/rotationOfe1Fig1}
{../figures/GAelectrodynamics/rotationOfe2Fig1}
{Multiplication by \( \Be_1 \Be_2 \).}{fig:rotationOfe1:rotationOfe1Fig1}{scale=0.5}

Using a polar vector representation

\begin{dmath}\label{eqn:SimpleProducts2:280}
   \Bx = \rho \lr{ \Be_1 \cos\theta + \Be_2 \sin\theta },
\end{dmath}

it can be
demonstrated directly that
unit pseudoscalar multiplication of an arbitrary vector will induce a \( \pi/2 \) rotation.
\index{pseudoscalar}

Multiplying the vector from the right by the pseudoscalar gives

\begin{dmath}\label{eqn:SimpleProducts2:300}
\Bx i
= \Bx \Be_1 \Be_2
= \rho \lr{ \Be_1 \cos\theta + \Be_2 \sin\theta } \Be_1 \Be_2
= \rho \lr{ \Be_2 \cos\theta - \Be_1 \sin\theta },
\end{dmath}

a counterclockwise rotation of \( \pi/2 \) radians, and
multiplying the vector by the pseudoscalar from the left gives

\begin{dmath}\label{eqn:2dRotations:3}
i \Bx
= \Be_1 \Be_2 \Bx
= \rho \Be_1 \Be_2 \lr{ \Be_1 \cos\theta + \Be_2 \sin\theta } \Be_1 \Be_2
= \rho \lr{ -\Be_2 \cos\theta + \Be_1 \sin\theta },
\end{dmath}

a clockwise rotation by \( \pi/2 \) radians
(\cref{problem:2dRotations:1}).

The transformed vector \( \Bx' = \Bx \Be_1 \Be_2 = \Be_2 \Be_1 \Bx (=\Bx i = -i \Bx) \) has been rotated in the direction that takes \( \Be_1 \) to \( \Be_2 \), as illustrated
in \cref{fig:rotationOfV:rotationOfVFig1}.

\imageFigure{../figures/GAelectrodynamics/rotationOfVFig1}{\( \pi/2\) rotation in the plane using pseudoscalar multiplication.}{fig:rotationOfV:rotationOfVFig1}{0.3}

In complex number theory the complex exponential \( e^{i\theta} \) can be used as a rotation operator.
Geometric algebra puts this rotation operator into the vector algebra toolbox, by utilizing
Euler's formula
\index{Euler's formula}

\begin{dmath}\label{eqn:2dRotations:1140}
e^{i\theta} = \cos\theta + i \sin\theta,
\end{dmath}

which holds for this pseudoscalar imaginary representation too (\cref{problem:2dRotations:Euler}).
\index{complex exponential}
By writing \( \Be_2 = \Be_1 \Be_1 \Be_2 \),
a complex exponential can be factored directly out of the polar vector representation \cref{eqn:SimpleProducts2:280}

\begin{dmath}\label{eqn:SimpleProducts2:940}
\Bx
=
\rho \lr{ \Be_1 \cos\theta + \Be_2 \sin\theta }
=
\rho \lr{ \Be_1 \cos\theta + (\Be_1 \Be_1) \Be_2 \sin\theta }
=
\rho \Be_1 \lr{ \cos\theta + \Be_1 \Be_2 \sin\theta }
=
\rho \Be_1 \lr{ \cos\theta + i \sin\theta }
=
\rho \Be_1 e^{i\theta}.
\end{dmath}

We end up with a complex exponential multivector factor on the right.
Alternatively, since \( \Be_2 = \Be_2 \Be_1 \Be_1 \), a complex exponential can be factored out on the left

\begin{dmath}\label{eqn:SimpleProducts2:960}
\Bx
=
\rho \lr{ \Be_1 \cos\theta + \Be_2 \sin\theta }
=
\rho \lr{ \Be_1 \cos\theta + \Be_2 (\Be_1 \Be_1) \sin\theta }
=
\rho \lr{ \cos\theta - \Be_1 \Be_2 \sin\theta } \Be_1
=
\rho \lr{ \cos\theta - i \sin\theta } \Be_1
=
\rho e^{-i\theta} \Be_1.
\end{dmath}

Left and right exponential expressions have now been found for the polar representation

\begin{equation}\label{eqn:SimpleProducts2:1120}
\rho \lr{ \Be_1 \cos\theta + \Be_2 \sin\theta }
= \rho e^{-i\theta} \Be_1 = \rho \Be_1 e^{i\theta}.
\end{equation}

This is essentially a recipe for rotation of a vector in the x-y plane.
Such rotations are
illustrated in \cref{fig:rotationOfX:rotationOfXFig1}.
\imageFigure{../figures/GAelectrodynamics/rotationOfXFig1}{Rotation in a plane.}{fig:rotationOfX:rotationOfXFig1}{0.3}

This generalizes to rotations of \R{N} vectors constrained to a plane.
Given orthonormal vectors \( \ucap, \vcap \) and any vector in the plane of these two vectors (\( \Bx \in \Span\setlr{\ucap,\vcap} \)), this vector is rotated \( \theta \) radians in the direction of rotation that takes \( \ucap \) to \( \vcap \) by

\begin{equation}\label{eqn:2dRotations:1160}
\Bx' = \Bx e^{ \ucap \vcap \theta } = e^{-\ucap \vcap \theta} \Bx.
\end{equation}

The sense of rotation for the rotation \( e^{ \ucap \vcap \theta} \) is opposite that of \( e^{\vcap \ucap \theta} \), which provides a first hint that bivectors can be characterized as having an orientation, somewhat akin to thinking of a vector as having a head and a tail.
\makeproblem{\R{2} rotations.}{problem:2dRotations:1}{
Using familiar methods, such as rotation matrices, show that the counterclockwise and clockwise rotations of
\cref{eqn:SimpleProducts2:280} are given by
\cref{eqn:SimpleProducts2:300} and
\cref{eqn:2dRotations:3} respectively.
} % problem

\makeproblem{Multivector Euler's formula and trig relations.}{problem:2dRotations:Euler}{
For a multivector \( x \) assume an infinite series representation of the exponential, sine and cosine functions and their hyperbolic analogues
\begin{equation*}
\begin{aligned}
\exp x &= \sum_{k = 0}^\infty \frac{x^k}{k!} \\
\cos x &= \sum_{k = 0}^\infty (-1)^k \frac{x^{2k}}{(2k)!} \qquad \sin x = \sum_{k = 0}^\infty (-1)^k \frac{x^{2k+1}}{(2k+1)!} \\
\cosh x &= \sum_{k = 0}^\infty \frac{x^{2k}}{(2k)!} \qquad \sinh x = \sum_{k = 0}^\infty \frac{x^{2k+1}}{(2k+1)!} \\
\end{aligned}
\end{equation*}

\makesubproblem{}{problem:2dRotations:Euler:a}
Show that for scalar \( \theta \), and any multivector \( J \) that satisfies \( J^2 = -1 \)

\begin{equation*}
\begin{aligned}
\cosh (J \theta ) &= \cos \theta \\
\sinh (J \theta ) &= J \sin \theta.
\end{aligned}
\end{equation*}

\makesubproblem{}{problem:2dRotations:Euler:c}

Show that the trigonometric and hyperbolic Euler formulas

\begin{equation*}
\begin{aligned}
e^{ J \theta } &= \cos \theta + J \sin \theta \\
e^{ K \theta } &= \cosh \theta + K \sinh \theta,
\end{aligned}
\end{equation*}

hold for multivectors \( J, K \) satisfying \( J^2 = -1 \) and \( K^2 = 1 \) respectively.

\makesubproblem{}{problem:2dRotations:Euler:b}
Given multivectors \( X, Y \), show that splitting a multivector exponential into factors of the form
\( e^{ X + Y } = e^{ X } E^{ Y } \) requires \( x \) and \( y \) to commute.
} % problem


