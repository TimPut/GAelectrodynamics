%
% Copyright © 2018 Peeter Joot.  All Rights Reserved.
% Licenced as described in the file LICENSE under the root directory of this GIT repository.
%
%{

\index{gradient!Green's function}
\index{Green's function!gradient representation}

We will see that the GA formulation of the statics equations (no time dependence), all have the form

\begin{dmath}\label{eqn:greensFunctionGradient:420}
\spacegrad F(\Bx) = J(\Bx),
\end{dmath}

where \( F, J \) are multivector fields and sources respectively.  We can assume an unbounded superposition solution

\begin{dmath}\label{eqn:greensFunctionGradient:440}
F(\Bx) = \int G(\Bx, \Bx') J(\Bx') dV' + F_0(\Bx),
\end{dmath}

where \( F_0 \) is any solution to the homogeneous gradient equation \( \spacegrad F_0 = 0 \), and operate on this presumed solution with the gradient to find

\begin{dmath}\label{eqn:greensFunctionGradient:460}
J(\Bx)
= \int \spacegrad G(\Bx, \Bx') J(\Bx') dV',
\end{dmath}

so the Green's function \( G \) for this system must satisfy

\begin{dmath}\label{eqn:greensFunctionGradient:480}
\spacegrad G(\Bx, \Bx') = \delta( \Bx - \Bx' ).
\end{dmath}

We will now show that this Green's function is vector valued as follows

\maketheorem{Green's function for the gradient}{thm:gradientGreensFunctionEuclidean:1}{
A Green's function that satisfies \cref{eqn:greensFunctionGradient:480} is
\begin{equation*}
   G(\Bx, \Bx') = \inv{4 \pi} \frac{\Bx - \Bx'}{\Norm{\Bx-\Bx'}^3}.
\end{equation*}
At points \( \Bx \ne \Bx' \), \( \spacegrad \wedge G = 0 \), or
\( \rspacegrad G = G \lspacegrad \).
} % theorem

To prove this, observe that we know the Laplacian representation of the delta function, so
the Green's function for the gradient can be written as

\begin{equation}\label{eqn:greensFunctionGradient:481}
\spacegrad G(\Bx, \Bx') = \delta( \Bx - \Bx' ) = \spacegrad^2 \lr{ -\inv{4\pi} \inv{ \Norm{\Bx - \Bx'} } }.
\end{equation}

GA provides us the rather beautiful and remarkable ability to factor the Laplacian in to a product of gradients \( \spacegrad^2 = \spacegrad \spacegrad \), so the gradient's Green's function is

\begin{dmath}\label{eqn:greensFunctionGradient:580}
G(\Bx, \Bx')
= \spacegrad \lr{ -\inv{4\pi} \inv{ \Norm{\Bx - \Bx'} } }
= -\inv{4\pi} \rcap \PD{r}{} \inv{r},
\end{dmath}

where \( \Br = \Bx - \Bx', r = \Norm{\Br} \), and \( \rcap = \Br/r \).  Proceeding with the derivatives, we find
\begin{dmath}\label{eqn:greensFunctionGradient:600}
G(\Bx, \Bx')
= -\inv{4\pi} \rcap \lr{ -\inv{r^2} }
= \frac{\rcap}{4 \pi r^2},
\end{dmath}

as claimed.  To show that the Green's function commutes with the gradient at points \( \Bx \ne \Bx' \) we can compute the (bivector) curl

\begin{dmath}\label{eqn:greensFunctionGradient:620}
\spacegrad \wedge \frac{\Br}{r^3}
=
\lr{ \spacegrad \inv{r^3}} \wedge \Br
+
\inv{r^3} \spacegrad \wedge \Br.
\end{dmath}

Since \( \spacegrad \inv{r^m} \propto \rcap \) the first wedge is zero.  The second wedge is also zero, which is easily demonstrated by coordinate expansion
\begin{dmath}\label{eqn:greensFunctionGradient:640}
\spacegrad \wedge \Br
=
\sum_{m,n} (\Be_m \partial_m) \wedge (\Be_n r_n)
=
\sum_{m,n} (\Be_m \wedge \Be_n) \partial_m r_n
=
\sum_{m,n} (\Be_m \wedge \Be_n) \delta_{m n}.
\end{dmath}

This last sum is zero since it is the symmetric sum of an antisymmetric quantity, which completes the proof.

We can determine the structure of an unbounded superposition solution through application of
%the fundamental theorem of geometric calculus
\cref{thm:fundamentalTheoremOfCalculus:1}

\begin{dmath}\label{eqn:greensFunctionGradient:500}
\int_V G(\Bx, \Bx') \lrspacegrad' F(\Bx') dV'
=
\int_V (G(\Bx, \Bx') \lspacegrad') F(\Bx') dV'
+
\int_V G(\Bx, \Bx') (\rspacegrad' F(\Bx')) dV'
=
\int_{\partial V} G(\Bx, \Bx') \ncap' F(\Bx') dA'.
\end{dmath}

As with the Helmholtz equation, we can presuming that the Green's function of
\cref{thm:gradientGreensFunctionEuclidean:1}
for the unbounded superposition solution also applies here.  Since

\begin{dmath}\label{eqn:greensFunctionGradient:520}
G(\Bx, \Bx') \lspacegrad'
=
-G(\Bx', \Bx) \lspacegrad'
=
-
\spacegrad' G(\Bx', \Bx)
= -\delta(\Bx - \Bx'),
\end{dmath}

we have
\begin{dmath}\label{eqn:greensFunctionGradient:540}
-F(\Bx)
+
\int_V G(\Bx, \Bx') J(\Bx') dV'
=
\int_{\partial V} G(\Bx, \Bx') \ncap' F(\Bx') dA',
\end{dmath}

or
\boxedEquation{eqn:gradientGreensFunctionEuclidean:560}{
F(\Bx)
=
\int_V G(\Bx, \Bx') J(\Bx') dV'
-
\int_{\partial V} G(\Bx, \Bx') \ncap' F(\Bx') dA'.
}

We are also free to add any specific solution \( F_0 \) to the gradient equation \( \spacegrad F_0 = 0 \).
Because the Green's function is not unique (we can add any solution \( G_0 \) of the gradient equation \( \spacegrad G_0 = 0 \)),
it may be desirable for bounded problems to construct Green's functions that are zero on the boundary of the integration volume.

%}
