%
% Copyright © 2016 Peeter Joot.  All Rights Reserved.
% Licenced as described in the file LICENSE under the root directory of this GIT repository.
%
\subsection{Complex power}
\index{complex power}

Let \( \BE \) and \( \BH \) be complex phasor (time harmonic) representations of the fields so that

\begin{dmath}\label{eqn:maxwellsEquations:900}
\begin{aligned}
\BE(\Bx, t) &= \Real\lr{ \BE e^{j \omega t} } \\
\BH(\Bx, t) &= \Real\lr{ \BH e^{j \omega t} }
\end{aligned}
\end{dmath}

The time harmonic expansion of the Poynting vector is

\begin{dmath}\label{eqn:maxwellsEquations:820}
\BS
= \lr{ I \BH(\Bx, t)} \cdot \BE(\Bx, t)
=
\lr{ I \Real \lr{ \BH e^{j \omega t}} } \cdot
\Real \lr{ \BE e^{j \omega t} }
= \inv{4} \gpgradeone{
I \lr{
\BH e^{j \omega t}
+ \BH^\conj e^{-j \omega t}
}
\lr{ \BE e^{j \omega t} + \BE^\conj e^{-j \omega t} }
}
=
\inv{4}
\lr{
(I \BH^\conj) \cdot \BE
+(I \BH) \cdot \BE^\conj
+
(I \BH) \cdot \BE e^{ 2 j \omega t }
+
(I \BH^\conj) \cdot \BE^\conj e^{ 2 j \omega t }
}.
\end{dmath}

This shows that the time harmonic representation of the Poynting vector is

\begin{dmath}\label{eqn:maxwellsEquations:840}
\BS = \inv{2} \Real \lr{ (I \BH^\conj) \cdot \BE + (I \BH) \cdot \BE e^{ 2 j \omega t } }.
\end{dmath}

This motivates the definition of a complex Poynting vector

\begin{dmath}\label{eqn:maxwellsEquations:860}
\bcS = \inv{2} (I \BH^\conj) \cdot \BE.
\end{dmath}

Note that this is conventionally written as \( \BS = (\ifrac{1}{2}) \BE \cross \BH^\conj \).
Observe that on average (over a given period), the \( e^{2 j \omega t} \) component of the real Poynting vector has no contribution to the power flux

\begin{dmath}\label{eqn:maxwellsEquations:880}
\inv{T} \int_0^T dt \BS = \Real \bcS.
\end{dmath}
