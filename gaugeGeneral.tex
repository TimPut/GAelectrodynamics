%
% Copyright � 2018 Peeter Joot.  All Rights Reserved.
% Licenced as described in the file LICENSE under the root directory of this GIT repository.
%
%{
%\input{../latex/blogpost.tex}
%\renewcommand{\basename}{gaugeGeneral}
%%\renewcommand{\dirname}{notes/phy1520/}
%\renewcommand{\dirname}{notes/ece1228-electromagnetic-theory/}
%%\newcommand{\dateintitle}{}
%%\newcommand{\keywords}{}
%
%\input{../latex/peeter_prologue_print2.tex}
%
%\usepackage{peeters_layout_exercise}
%\usepackage{peeters_braket}
%\usepackage{peeters_figures}
%\usepackage{siunitx}
%\RequirePackage{amssymb} % relwave.tex uses this?
%%\usepackage{mhchem} % \ce{}
%%\usepackage{macros_bm} % \bcM
%%\usepackage{macros_qed} % \qedmarker
%%\usepackage{txfonts} % \ointclockwise
%
%%\newcommand{\Box}[0]{\square}
%\newcommand{\dLambertian}[0]{\square}
%\newcommand{\stgrad}[0]{\lr{ \spacegrad + \inv{c} \PD{t}{}}}
%\newcommand{\conjstgrad}[0]{\lr{ \spacegrad - \inv{c} \PD{t}{}}}
%
%\beginArtNoToc
%
%\generatetitle{Gauge transformation}
%%\chapter{Gauge transformation}
\label{chap:gaugeGeneral}

\maketheorem{Potential solution of Maxwell's equation.}{thm:gaugeGeneral:20}{
If \( A \) is a multivector solution to \( \dLambertian A = J \),
for which \( \gpgrade{\conjstgrad A}{0,3} \ne 0 \),
%, that is
%\begin{equation*}
%A(\Bx)
%= \int dV' G(\Bx, \Bx') J(\Bx')
%= -\int dV' \frac{J(\Bx')}{\Norm{\Bx - \Bx'} },
%\end{equation*}
then
\( F = \conjstgrad A' \) is a grade 1,2 solution to Maxwell's equation \(\stgrad F = J\), where \( A' = A - \tilde{A} \), and \( \tilde{A} \)
is a solution of the non-homogeneous wave equation
\begin{equation*}
\dLambertian \tilde{A} = \stgrad \gpgrade{\conjstgrad A}{0,3}.
\end{equation*}
} % theorem

To prove \cref{thm:gaugeGeneral:20}, let's first assume that there exists an \( A' \) for which \( \gpgrade{\conjstgrad A}{0,3} = 0 \) that satisifes the wave equation
\begin{dmath}\label{eqn:gaugeGeneral:60}
\dLambertian A' = J.
\end{dmath}
For such a field \( F = \conjstgrad A' \) is a grade 1,2 solution of Maxwell's equation.
Given an arbitrary solution \( A \) of \( \dLambertian A = J \), then \( F = \gpgrade{\conjstgrad A}{1,2} \) is a also solution
of Maxwell's equation.  Let \( A = A' + \tilde{A} \) and insert this into Maxwell's equation
\begin{dmath}\label{eqn:gaugeGeneral:20}
J
= \stgrad F
= \stgrad \gpgrade{\conjstgrad A}{1,2}
= \dLambertian A - \stgrad \gpgrade{\conjstgrad A}{0,3}
= \dLambertian \lr{ A' + \tilde{A} } - \stgrad \gpgrade{\conjstgrad A}{0,3}.
\end{dmath}
Subtracting \cref{eqn:gaugeGeneral:60} from both sides of \cref{eqn:gaugeGeneral:20} and rearranging, we are left with
\begin{dmath}\label{eqn:gaugeGeneral:40}
\dLambertian \tilde{A} = \stgrad \gpgrade{\conjstgrad A}{0,3},
\end{dmath}
which completes the proof.

From \cref{eqn:generalPotential:60}, we see that \cref{eqn:gaugeGeneral:40} means we seek a solution to
\begin{dmath}\label{eqn:gaugeGeneral:80}
\dLambertian \tilde{A} = \stgrad
\lr{
c \spacegrad \cdot \BA + \inv{c} \PD{t}{\phi}
+ I \eta
   \lr{
     c \spacegrad \cdot \BF + \inv{c} \PD{t}{\phi_\txtm}
   }
}
.
\end{dmath}

When the potentials are constrained by
\begin{dmath}\label{eqn:gaugeGeneral:100}
\begin{aligned}
0 &= \spacegrad \cdot \BA + \inv{c^2} \PD{t}{\phi}
0 &= \spacegrad \cdot \BF + \inv{c^2} \PD{t}{\phi_\txtm},
\end{aligned}
\end{dmath}
\index{Lorenz gauge}
then
we are said to be working in the Lorenz gauge.

%}
%\EndNoBibArticle
