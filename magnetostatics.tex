%
% Copyright © 2017 Peeter Joot.  All Rights Reserved.
% Licenced as described in the file LICENSE under the root directory of this GIT repository.
%
\index{magnetostatics}
\index{time independence}
Magnetostatics is the study of Maxwell's equations where
a time independent restriction of the fields is imposed, and
it assumed that there are no static charge distributions.
For such constraints (and no magnetic sources) the free space Maxwell's equations are simply

\begin{subequations}
\label{eqn:magnetostatics:99}
\begin{dmath}\label{eqn:magnetostatics:100}
\spacegrad \cross \BE = 0
\end{dmath}
\begin{dmath}\label{eqn:magnetostatics:120}
\spacegrad \cross \BB = \mu \BJ
\end{dmath}
\begin{dmath}\label{eqn:magnetostatics:140}
\spacegrad \cdot \BE = 0
\end{dmath}
\begin{dmath}\label{eqn:magnetostatics:160}
\spacegrad \cdot \BB = 0.
\end{dmath}
\end{subequations}

\Cref{eqn:electrostatics:240} can be used to rewrite the magnetostatic Maxwell equations (\cref{eqn:magnetostatics:99}), as a pair of multivector gradient equations.
The electric field equation is just

\begin{equation}\label{eqn:magnetostatics:400}
\spacegrad \BE = 0,
\end{equation}
and for the magnetic field, we have

\begin{dmath}\label{eqn:magnetostatics:420}
\spacegrad \BB
=
\spacegrad \cdot \BB
+
I (\spacegrad \cross \BB).
\end{dmath}

As was the case in electrostatics, Maxwell's equations can be reduced to a pair of multivector gradient equations
%\begin{dmath}\label{eqn:magnetostatics:380}
\boxedEquation{eqn:magnetostatics:380}{
\begin{aligned}
\spacegrad \BB &= I \mu \BJ \\
\spacegrad \BE &= 0.
\end{aligned}
}
%\end{dmath}

Constraints must be imposed on the current density for \cref{eqn:magnetostatics:380} to be satisfied.
This can be seen by left multiplying with the gradient

\begin{dmath}\label{eqn:magnetostatics:440}
\spacegrad^2 \BB
= \mu I \spacegrad \BJ
= \mu I \lr{ \spacegrad \cdot \BJ + \spacegrad \wedge \BJ }
= \mu \lr{ I (\spacegrad \cdot \BJ) - \spacegrad \cross \BJ }.
\end{dmath}

The left hand side is a vector, whereas the right hand side has vector and pseudoscalar grades.
This means that magnetostatics conditions require the divergence of the current density to be zero

\begin{dmath}\label{eqn:magnetostatics:460}
\spacegrad \cdot \BJ = 0.
\end{dmath}
