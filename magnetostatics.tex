Magnetostatics is the study of Maxwell's equations where
a time independent restriction of the fields is imposed, and
it assumed that there are no static charge distributions.
These constraints simpify Maxwell's equations considerably

\begin{subequations}
\label{eqn:magnetostatics:99}
\begin{dmath}\label{eqn:magnetostatics:100}
\spacegrad \cross \BE = 0
\end{dmath}
\begin{dmath}\label{eqn:magnetostatics:120}
\spacegrad \cross \BB = \mu_0 \BJ
\end{dmath}
\begin{dmath}\label{eqn:magnetostatics:140}
\spacegrad \cdot \BE = 0
\end{dmath}
\begin{dmath}\label{eqn:magnetostatics:160}
\spacegrad \cdot \BB = 0.
\end{dmath}
\end{subequations}

\Cref{eqn:electrostatics:240} can be used to rewrite the magnetostatic Maxwell equations (\cref{eqn:magnetostatics:99}), as a pair of multivector gradient equations.  The electric field equation is just

\begin{equation}\label{eqn:magnetostatics:400}
\spacegrad \BE = 0,
\end{equation}

and for the magnetic field, we have

\begin{dmath}\label{eqn:magnetostatics:420}
\spacegrad \BB
=
\spacegrad \cdot \BB
I (\spacegrad \cross \BB),
\end{dmath}

so
\begin{dmath}\label{eqn:magnetostatics:380}
\spacegrad \BB
=
I \mu_0 \BJ.
\end{dmath}

Constraints must be imposed on the current density for \cref{eqn:magnetostatics:380} to be satisfied.  This can be seen by left multiplying with the gradient

\begin{dmath}\label{eqn:magnetostatics:440}
\spacegrad^2 \BB
= \mu_0 I \spacegrad \BJ
= \mu_0 I \lr{ \spacegrad \cdot \BJ + \spacegrad \wedge \BJ }
= \mu_0 \lr{ I (\spacegrad \cdot \BJ) - \spacegrad \cross \BJ }
\end{dmath}

The left hand side is a vector, whereas the right hand side has vector and pseudoscalar grades.
This means that the divergence of the current density must be zero

\begin{dmath}\label{eqn:magnetostatics:460}
\spacegrad \cdot \BJ = 0.
\end{dmath}

\subsection{Vector potential.}

Similar to electrostatics where it was assumed that the electric field could be expressed as the gradient of a scalar potential,
a vector potential \( \BA \) solution for the dual of the magnetic field can be assumed

\begin{dmath}\label{eqn:magnetostatics:480}
\spacegrad \BA = I \BB.
\end{dmath}

As the right hand side is a bivector, we must have \( \spacegrad \cdot \BA = 0 \) for this presumed solution to be valid.
Assuming (for now) a zero divergence constraint for the vector potential, then \cref{eqn:magnetostatics:380} is reduced to

%\begin{dmath}\label{eqn:magnetostatics:540}
\boxedEquation{eqn:magnetostatics:540}{
\spacegrad^2 \BA = -\mu_0 \BJ,
}
%\end{dmath}

which can be solved immediately

%\begin{dmath}\label{eqn:magnetostatics:560}
\boxedEquation{eqn:magnetostatics:560}{
\BA(\Bx) = \frac{\mu_0}{4\pi} \int dV' \frac{ \BJ(\Bx') }{\Norm{\Bx - \Bx'}}.
}
%\end{dmath}

The zero divergence constraint for the vector potential is easily dealt with by adding a gradient to the vector potential with the
transformation \( \BA \rightarrow \overbar{\BA} + \spacegrad \chi \).  This gives

\begin{dmath}\label{eqn:magnetostatics:500}
\spacegrad \BA
=
\spacegrad \overbar{\BA} + \spacegrad^2 \chi
=
\spacegrad \cdot \overbar{\BA} + \spacegrad \wedge \overbar{\BA} + \spacegrad^2 \chi,
\end{dmath}

which has the required bivector grade when \( \spacegrad \cdot \overbar{\BA} = -\spacegrad^2 \chi \), or

\begin{dmath}\label{eqn:magnetostatics:520}
\chi(\Bx) = \inv{4\pi} \int dV' \frac{ \spacegrad' \cdot \overbar{\BA}(\Bx') }{\Norm{\Bx - \Bx'}}.
\end{dmath}

\subsection{Enclosed charge density.}

A volume integral of \cref{eqn:magnetostatics:380} provides a relationship between the total enclosed current density and the magnetic field.  The fundamental theorem gives

\begin{dmath}\label{eqn:magnetostatics:580}
-\mu_0 I
\int_V d^3 \Bx \BJ =
\int_V d^3 \Bx \spacegrad \BB =
\int_{\partial V} d^2 \Bx \BB.
\end{dmath}

With a normal parameterization of the oriented surface area element \( d^2 \Bx = I \ncap dA \), and \( d^3 \Bx = I dV \),
\cref{eqn:magnetostatics:580} is reduced to

%\begin{dmath}\label{eqn:magnetostatics:600}
\boxedEquation{eqn:magnetostatics:600}{
\int_{\partial V} dA I \ncap \BB = \mu_0  \int_V dV \BJ.
}
%\end{dmath}

This can be split into two grades

\begin{subequations}
\label{eqn:magnetostatics:620}
\begin{dmath}\label{eqn:magnetostatics:640}
I \int_{\partial V} dA \ncap \cdot \BB = 0
\end{dmath}
\begin{dmath}\label{eqn:magnetostatics:660}
\int_{\partial V} dA \ncap \cross \BB = -\mu_0  \int_V dV \BJ.
\end{dmath}
\end{subequations}

\Cref{eqn:magnetostatics:640} should not be a suprise since it is a direct consequence of Gauss's law \( \spacegrad \cdot \BB = 0 \).  \Cref{eqn:magnetostatics:660} provides a relationship between the tangential components of the magnetic field and the total enclosed current density.

\subsection{Biot-Savart law.}

The magnetostatic Maxwell equation \cref{eqn:magnetostatics:380} can be inverted directly using the Green's function for the gradient

\begin{dmath}\label{eqn:biotSavartGreens:40}
I \BB(\Bx)
= \int_V dV' G(\Bx, \Bx') \spacegrad' I \BB(\Bx')
\end{dmath}

This expansion can be simplified by inserting a no-op grade selection operation

\begin{dmath}\label{eqn:magnetostatics:680}
I \BB(\Bx)
= \gpgradetwo{ \int_V dV' G(\Bx, \Bx') \spacegrad' I \BB(\Bx') }
= \int_V dV' \gpgradetwo{ G(\Bx, \Bx') (-\mu_0 \BJ(\Bx')) }
= \inv{4\pi} \int_V dV' \frac{\Bx - \Bx'}{ \Abs{\Bx - \Bx'}^3 } \wedge (-\mu_0 \BJ(\Bx')),
\end{dmath}
or

\boxedEquation{eqn:magnetostatics:n}{
I \BB(\Bx)
= \frac{\mu_0}{4\pi} \int_V dV' \BJ(\Bx') \wedge \frac{\Bx - \Bx'}{ \Abs{\Bx - \Bx'}^3 }.
}

This is the Biot-Savart law in its GA form.  The traditional expression requires only a duality transformation \( \BJ \wedge \Bf = I ( \BJ \cross \Bf) \), or

\begin{dmath}\label{eqn:magnetostatics:700}
\BB(\Bx)
= \frac{\mu_0}{4\pi} \int_V dV' \BJ(\Bx') \cross \frac{\Bx - \Bx'}{ \Abs{\Bx - \Bx'}^3 }.
\end{dmath}

The freedom to insert a no-op bivector grade selection may seem like a sneaky move.
To remove doubt about the validity of this move, here is a demonstration that
the scalar grade discarded by the grade selection operation on the integrand of \cref{eqn:magnetostatics:680} is explicitly zero,
provided the current density vanishes faster than \( r \) on the infinite sphere.

\begin{dmath}\label{eqn:biotSavartGreens:60}
 \int_V dV' \frac{\Bx - \Bx'}{ \Abs{\Bx - \Bx'}^3 } \cdot \BJ(\Bx')
= \frac{\mu_0}{4\pi} \int_V dV' \lr{ \spacegrad \inv{ \Abs{\Bx - \Bx'} }} \cdot \BJ(\Bx')
=  \int_V dV' \lr{ \spacegrad' \inv{ \Abs{\Bx - \Bx'} }} \cdot \BJ(\Bx')
=  \int_V dV' \lr{
\spacegrad' \cdot \frac{\BJ(\Bx')}{ \Abs{\Bx - \Bx'} }
-
\frac{\spacegrad' \cdot \BJ(\Bx')}{ \Abs{\Bx - \Bx'} }
}.
\end{dmath}

By \cref{eqn:magnetostatics:460}, the divergence of the current density is zero, which kills the second term.  The divergence theorem can be used to express the remaining integral as a surface integral, so

\begin{dmath}\label{eqn:biotSavartGreens:100}
 \int_V dV' \frac{\Bx - \Bx'}{ \Abs{\Bx - \Bx'}^3 } \cdot \BJ(\Bx')
=  \int_V dV' \spacegrad' \cdot \frac{\BJ(\Bx')}{ \Abs{\Bx - \Bx'} }
=  \int_{\partial V} dA' \ncap \cdot \frac{\BJ(\Bx')}{ \Abs{\Bx - \Bx'} }.
\end{dmath}

Provided the normal component of \( \BJ(\Bx')/\Abs{\Bx - \Bx'} \) vanishes on the boundary of an infinite sphere, we see that the
the scalar selection of the convolution integral is zero, justifying the (sneaky) bivector selection operation.

