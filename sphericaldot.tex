%
% Copyright � CCYY Peeter Joot.  All Rights Reserved.
% Licenced as described in the file LICENSE under the root directory of this GIT repository.
%
\makeproblem{Spherical coordinate basis orthogonality.}{problem:sphericaldot:1}{
\index{spherical coordinates}
Using scalar selection, show that the spherical curvilinear basis of \cref{eqn:curvilinearspherical:80} are all mutually orthogonal.
} % problem

\makeanswer{problem:sphericaldot:1}{
Computing the various dot products is made easier by noting that \( \Be_3 \) and \( e^{i \phi } \) commute, whereas \( e^{j\theta } \Be_3 = \Be_3 e^{-j\theta}, \Be_1 e^{i\phi} = e^{-i\phi} \Be_1, \Be_2 e^{i\phi} = e^{-i\phi} \Be_2 \) (since \( \Be_3 j \), \( \Be_1 i \) and \( \Be_2 i \) all anticommute)

\begin{subequations}
\label{eqn:sphericaldot:160}
\begin{dmath}\label{eqn:sphericaldot:180}
\Bx_r \cdot \Bx_\theta
=
\gpgradezero{
\Be_3 e^{j \theta} \Be_1 e^{i\phi} e^{j \theta}
}
=
\gpgradezero{
e^{j \theta} \Be_3 e^{j \theta} \Be_1 e^{i\phi}
}
=
\gpgradezero{
\Be_3 e^{-j \theta} e^{j \theta} \Be_1 e^{i\phi}
}
=
\gpgradezero{
\Be_3 \Be_1 e^{i\phi}
}
= 0
\end{dmath}
\begin{dmath}\label{eqn:sphericaldot:200}
\Bx_r \cdot \Bx_\phi
=
\gpgradezero{
\Be_3 e^{j \theta} r \sin\theta \Be_2 e^{i \phi}
}
=
r \sin\theta
\gpgradezero{
\Be_3 \lr{ \cos\theta + \Be_{31} \sin\theta e^{i\phi} } \Be_2 e^{i \phi}
}
=
r \sin^2\theta
\gpgradezero{
\Be_{1} e^{i\phi} \Be_2 e^{i \phi}
}
=
r \sin^2\theta
\gpgradezero{
\Be_{1} \Be_2
}
=
0
\end{dmath}
\begin{dmath}\label{eqn:sphericaldot:220}
\Bx_\theta \cdot \Bx_\phi
=
r \sin\theta
\gpgradezero{
\Be_1 e^{i\phi} e^{j \theta}
\Be_2 e^{i \phi}
}
=
r \sin\theta
\gpgradezero{
\Be_2 \Be_1 e^{j \theta}
}
=
r \sin\theta
\gpgradezero{
\Be_2 \Be_1 \lr{ \cos\theta + \Be_{31} \sin\theta e^{i \phi} }
}
=
r \sin^2\theta
\gpgradezero{
\Be_{32} e^{i \phi}
}
=
0.
\end{dmath}
\end{subequations}

} % answer
