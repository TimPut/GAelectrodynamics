%
% Copyright © 2017 Peeter Joot.  All Rights Reserved.
% Licenced as described in the file LICENSE under the root directory of this GIT repository.
%
Let's now look at how
\cref{eqn:SimpleProducts2:1440},
the dot and wedge product decomposition of the vector product, can be applied to compute vector projection and rejection, which are defined as

\index{projection}
\index{rejection}
\makedefinition{Projection and rejection}{dfn:projectionAndRejection:projectionandrejection}{
Given a vector \( \Bx \) and vector \( \Bu \) the projection of \( \Bx \) onto the direction of \( \Bu \) is defined as

\begin{equation*}
\Proj_\Bu(\Bx) = (\Bx \cdot \ucap) \ucap,
\end{equation*}

where \( \ucap = \Bu/\Norm{\Bu} \).  The rejection of \( \Bu \) from \( \Bx \) is defined as the component of \( \Bx \) that is
normal to \( \Bu \)

\begin{equation*}
\Rej{\Bu}{\Bx} = \Bx - \Proj_\Bu(\Bx).
\end{equation*}
} % definition

It is possible to factor unity as \( 1 = \ucap \ucap \) to derive a useful GA representation of the projective and rejective components of a vector

\begin{dmath}\label{eqn:SimpleProducts2:680}
\Bx =
\Bx \ucap \ucap
=
\lr{ \Bx \ucap } \ucap
=
\Biglr{ \Bx \cdot \ucap + \Bx \wedge \ucap } \ucap
=
\lr{ \Bx \cdot \ucap } \ucap + \lr{ \Bx \wedge \ucap } \ucap.
\end{dmath}

The vector \( \Bx \) is split nicely into its projection and rejective components in a complementary fashion

\begin{subequations}
\label{eqn:projectionAndRejection:1080}
\begin{dmath}\label{eqn:projectionAndRejection:900}
\Proj_\Bu(\Bx) = \lr{ \Bx \cdot \ucap } \ucap
\end{dmath}
\begin{dmath}\label{eqn:projectionAndRejection:910}
\Rej{\Bu}{\Bx} = \lr{ \Bx \wedge \ucap } \ucap.
\end{dmath}
\end{subequations}

By construction,
\( \lr{ \Bx \wedge \ucap } \ucap \) must be a vector, despite any appearance of a multivector nature.

The utility of this multivector rejection formula is not for hand or computer algebra calculations, where it will generally be faster and simpler to compute \( \Bx - (\Bx \cdot \ucap) \ucap \).  Instead this will come in handy as a new abstract algebraic tool.

When it is desirable to perform this calculation explicitly, it can be done more effieciently using a no-op grade selection operation.
In particular, a vector can be written as it's own grade-1 selection

\begin{dmath}\label{eqn:projectionAndRejection:920}
\Bx = \gpgradeone{ \Bx },
\end{dmath}

so the rejection can be reexpressed
using \cref{dfn:gradeselection:100}
as a generalized bivector-vector dot product

\begin{equation}\label{eqn:projectionAndRejection:940}
\Rej{\Bu}{\Bx}
= \gpgradeone{ \lr{ \Bx \wedge \ucap } \ucap }
= \lr{ \Bx \wedge \ucap } \cdot \ucap.
\end{equation}

To help establish some confidence with these new additions to our toolbox, here are a
pair of illustrative examples using
\cref{eqn:projectionAndRejection:910}, and
\cref{eqn:projectionAndRejection:940} respectively.

\makeexample{\R{2} rejection from the x direction.}{example:projectionAndRejection:1}{
Let \( \Bx = a \Be_1 + b \Be_2 \) and \( \Bu = \Be_1 \) for which the wedge is \( \Bx \wedge \ucap = b \Be_2 \Be_1 \).
Using \cref{eqn:projectionAndRejection:910} the rejection of \( \Bu \) from \( \Bx \) is
\begin{dmath}\label{eqn:projectionAndRejection:1000}
\Rej{\Bu}{\Bx}
=
\lr{ \Bx \wedge \ucap } \ucap
=
(b \Be_2 \Be_1 )\Be_1
=
b \Be_2 (\Be_1 \Be_1)
=
b \Be_2,
\end{dmath}

as expected.
} % example

This example provides some guidance about what is happening geometrically in
\cref{eqn:projectionAndRejection:910}.
The wedge operation produces a pseudoscalar for the plane spanned by \( \setlr{\Bx, \Bu} \) that is scaled as \( \sin\theta \) where \( \theta \) is the angle between \( \Bx \) and \( \Bu \).  When that pseudoscalar is multiplied by \( \ucap \), \( \ucap \) is rotated in the plane by \( \pi/2 \) radians towards \( \Bx \), yielding the normal component of the vector \( \Bx \).

Here's a slightly less trivial \R{3} example

\makeexample{An \R{3} rejection.}{example:projectionAndRejection:r3rejection}{
Let \( \Bx = a \Be_2 + b \Be_3 \) and \( \ucap = ( \Be_1 + \Be_2 )/\sqrt{2} \) for which the
wedge product is

\begin{dmath}\label{eqn:projectionAndRejection:1040}
\Bx \wedge \ucap = \inv{\sqrt{2}}
\begin{vmatrix}
\Be_{23} & \Be_{31} & \Be_{12} \\
0 & a & b \\
1 & 1 & 0
\end{vmatrix}
=
\inv{\sqrt{2}}
\lr{
\Be_{23}(-b) - \Be_{31}(-b) + \Be_{12} (-a)
}
=
\inv{\sqrt{2}}
\lr{
b (\Be_{32} + \Be_{31} ) + a \Be_{21}
}.
\end{dmath}

Using \cref{eqn:projectionAndRejection:940} the rejection of \( \Bu \) from \( \Bx \) is

\begin{dmath}\label{eqn:projectionAndRejection:1060}
(\Bx \wedge \ucap) \cdot \ucap
=
\inv{2}
\lr{ b (\Be_{32} + \Be_{31} ) + a \Be_{21} } \cdot ( \Be_1 + \Be_2 )
\end{dmath}

Each of these bivector-vector dot products has the form \( \Be_{rs} \cdot \Be_t = \gpgradeone{ \Be_{rst} } \) which is zero whenever the indexes \( r,s, t\) are unique, and is a vector whenever one of indexes are repeated (\( r = t \), or \( s = t \)).  This leaves

\begin{dmath}\label{eqn:projectionAndRejection:1100}
(\Bx \wedge \ucap) \cdot \ucap
=
\inv{2}
\lr{ b \Be_3 + a \Be_2 + b \Be_3 - a \Be_1 }
=
b \Be_3 + \frac{a}{2}( \Be_2 - \Be_1 ).
\end{dmath}

The reader should confirm that this equals \( \Bx - (\Bx \cdot \ucap) \ucap \).
} % example

In the GA literature the projection and rejection operations are usually written using the vector inverse

\index{vector inverse}
\makedefinition{Vector inverse.}{dfn:projectionAndRejection:vectorinverse}{
Define the inverse of a vector \( \Bx \), when it exists, as
\begin{equation*}
\Bx^{-1} = \frac{\Bx}{\Norm{\Bx}^2}.
\end{equation*}

This inverse satisfies \( \Bx^{-1} \Bx = \Bx \Bx^{-1} = 1 \).
} % definition

The vector inverse may not exist in a non-Euclidean vector space where \( \Bx^2 \) can be zero for non-zero vectors \( \Bx \).

In terms of the vector inverse, the projection and rejection operations with respect to \( \Bu \) can be written without any reference to the unit vector \( \ucap = \Bu/\Norm{\Bu} \) that lies along that vector

\boxedEquation{eqn:projectionAndRejection:1120}{
\begin{aligned}
\Proj_\Bu(\Bx) &= \lr{ \Bx \cdot \Bu } \inv{\Bu} \\
\Rej{\Bu}{\Bx} &=
\lr{ \Bx \wedge \Bu } \inv{\Bu}
=
\lr{ \Bx \wedge \Bu } \cdot \inv{\Bu}.
\end{aligned}
}

Using this notation, an example of projection and rejection with respect to a direction vector \( \Bu \) is illustrated in
\cref{fig:projectionAndRejection:projectionAndRejectionFig1}.

\imageFigure{../figures/GAelectrodynamics/projectionAndRejectionFig1}{Projection and rejection illustrated.}{fig:projectionAndRejection:projectionAndRejectionFig1}{0.3}

It was claimed in the definition of rejection that the rejection is normal to the projection.  This can be shown trivially without any use of GA (\cref{problem:projectionAndRejection:rejectionnormality}).  This also follows naturally using the grade selection operator, which illustrates another useful new GA technique

\begin{dmath}\label{eqn:SimpleProducts2:720}
\Rej{\Bu}{\Bx} \cdot \Proj_\Bu(\Bx)
=
\gpgradezero{ \Rej{\Bu}{\Bx} \Proj_\Bu(\Bx) }
=
\gpgradezero{ \lr{ \Bx \wedge \ucap } \ucap \lr{ \Bx \cdot \ucap } \ucap }
=
\lr{ \Bx \cdot \ucap } \gpgradezero{ \lr{ \Bx \wedge \ucap } \ucap^2 }
=
\lr{ \Bx \cdot \ucap } \gpgradezero{ \Bx \wedge \ucap }.
\end{dmath}

Since the scalar grade of a wedge product, a (grade-2) bivector, is zero,
this demonstrates that the projection and rejection are normal.

Finally, since the \R{3} wedge product is related to the cross product through \cref{eqn:SimpleProducts2:1620},
it is reasonable to ask how the wedge product form of the rejection operator can be
expressed using the cross product.
Using the grade selection form of \cref{eqn:projectionAndRejection:940}, we find

\begin{dmath}\label{eqn:projectionAndRejection:1160}
\Rej{\Bu}{\Bx}
%=
%\lr{ \Bx \wedge \ucap } \cdot \ucap
=
\gpgradeone{ \lr{ \Bx \wedge \ucap } \ucap }
=
\gpgradeone{ I \lr{ \Bx \cross \ucap } \ucap }
=
\gpgradeone{ I
\biglr{
   \lr{ \Bx \cross \ucap } \cdot \ucap
+
   \lr{ \Bx \cross \ucap } \wedge \ucap
}
}
=
\lr{ \lr{ \Bx \cross \ucap } \cdot \ucap }
\cancel{ \gpgradeone{ I } }
+
\gpgradeone{ I^2
   \lr{ \Bx \cross \ucap } \cross \ucap
}
\end{dmath}

A grade 1 selection from the pseudoscalar is zero, since the pseudoscalar is grade 3.
Application of \cref{eqn:SimpleProducts2:1620} a second time to the remaining vector term produces a (scalar) \( I^2 = -1 \) factor, so after reordering the cross product, the rejection is

\begin{dmath}\label{eqn:projectionAndRejection:1180}
\Rej{\Bu}{\Bx}
=
   \ucap \cross \lr{ \Bx \cross \ucap }.
\end{dmath}

This is a result that may already be familiar.
% implicitly used in the front cover of Jackson
Without GA, one can prove that \cref{eqn:projectionAndRejection:1180} matches the rejection definition by choosing a coordinate system so that \( \ucap \) is aligned with one of the standard basis vectors.
It is also possible to prove this using the identity \( \Ba \cross ( \Bb \cross \Bc ) = (\Ba \cdot \Bc) \Bb - (\Ba \cdot \Bb) \Bc \) (\citep{jackson1975cew}).
A very compact proof using coordinates and using tensor contraction methods is also possible (see: \citep{landau1951classical}).
These methods are all arguably less elegant than this coordinate free GA approach.

\makeproblem{Rejection normality}{problem:projectionAndRejection:rejectionnormality}{
Prove, without any use of GA, that \( \Bx - \Proj_\Bu(\Bx) \) is normal to \( \Bu \), as claimed in
\cref{dfn:projectionAndRejection:projectionandrejection}.
} % problem

%\makeproblem{Prove \ref{dfn:projectionAndRejection:r3pcommutation}.}{problem:projectionAndRejection:1160}{
%} % problem

