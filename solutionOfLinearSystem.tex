%
% Copyright © 2017 Peeter Joot.  All Rights Reserved.
% Licenced as described in the file LICENSE under the root directory of this GIT repository.
%
\index{linear system}
\index{wedge product!linear solution}
Various types of linear systems can be solved using the wedge product.
An illustrative example is that of the intersection of two lines as illustrated in \cref{fig:intersectionOfLines:intersectionOfLinesFig1}.

\imageFigure{../figures/GAelectrodynamics/intersectionOfLinesFig1}{Intersection of two lines.}{fig:intersectionOfLines:intersectionOfLinesFig1}{0.3}

In parametric form, the lines in this problem are

\begin{dmath}\label{eqn:SimpleProducts2:1000}
\begin{aligned}
\Br_1(s) &= \Ba_0 + s( \Ba_1 - \Ba_2 ) \\
\Br_2(t) &= \Bb_0 + t( \Bb_1 - \Bb_2 ),
\end{aligned}
\end{dmath}

so the solution, if it exists, is found at the point satisfying the equality

\begin{dmath}\label{eqn:SimpleProducts2:1020}
\Ba_0 + s( \Ba_1 - \Ba_2 ) = \Bb_0 + t( \Bb_1 - \Bb_2 ).
\end{dmath}

With
\begin{dmath}\label{eqn:SimpleProducts2:1040}
\begin{aligned}
\Bu_1 &= \Ba_1 - \Ba_2 \\
\Bu_2 &= \Bb_1 - \Bb_2 \\
\Bd &= \Ba_0 - \Bb_0,
\end{aligned}
\end{dmath}

so the desired equation to solve is

\begin{dmath}\label{eqn:SimpleProducts2:1060}
\Bd + s \Bu_1 = t \Bu_2.
\end{dmath}

Solving for \( s \) or \( t \) is possible by 
wedging both sides with one of \( \Bu_1 \) or \( \Bu_2 \)

\begin{dmath}\label{eqn:SimpleProducts2:1080}
\begin{aligned}
\Bd \wedge \Bu_1 &= t \Bu_2 \wedge \Bu_1 \\
\Bd \wedge \Bu_2 + s \Bu_1 \wedge \Bu_2 &= 0,
\end{aligned}
\end{dmath}

In \R{2} these equations have a solution if \( \Bu_1 \wedge \Bu_2 \ne 0 \), and in \R{N} these have solutions if the bivectors on each sides of the equations describe the same plane.
Put another way, these have solutions when \( s \) and \( t \) are scalars and

\begin{dmath}\label{eqn:SimpleProducts2:1100}
\begin{aligned}
s &= \frac{\Bu_2 \wedge \Bd}{\Bu_1 \wedge \Bu_2} \\
t &= \frac{\Bu_1 \wedge \Bd}{\Bu_1 \wedge \Bu_2}.
\end{aligned}
\end{dmath}

For \R{2},
where the wedge product can be expressed as a (unit bivector scaled) determinant, this is precisely the Cramer's rule solution of the equivalent matrix equation.
