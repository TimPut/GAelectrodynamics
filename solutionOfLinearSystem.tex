%
% Copyright © 2017 Peeter Joot.  All Rights Reserved.
% Licenced as described in the file LICENSE under the root directory of this GIT repository.
%
\index{linear system}
\index{wedge product!linear solution}
Various types of linear systems can be solved using the wedge product.
An illustrative example is that of the intersection of two lines as illustrated in \cref{fig:intersectionOfLines:intersectionOfLinesFig1}.

\imageFigure{../figures/GAelectrodynamics/intersectionOfLinesFig1}{Intersection of two lines.}{fig:intersectionOfLines:intersectionOfLinesFig1}{0.3}

In parametric form, the lines in this problem are

\begin{dmath}\label{eqn:solutionOfLinearSystem:1000}
\begin{aligned}
\Br_1(s) &= \Ba_0 + s( \Ba_1 - \Ba_2 ) \\
\Br_2(t) &= \Bb_0 + t( \Bb_1 - \Bb_2 ),
\end{aligned}
\end{dmath}

so the solution, if it exists, is found at the point satisfying the equality

\begin{dmath}\label{eqn:solutionOfLinearSystem:1020}
\Ba_0 + s( \Ba_1 - \Ba_2 ) = \Bb_0 + t( \Bb_1 - \Bb_2 ).
\end{dmath}

With
\begin{dmath}\label{eqn:solutionOfLinearSystem:1040}
\begin{aligned}
\Bu_1 &= \Ba_1 - \Ba_2 \\
\Bu_2 &= \Bb_1 - \Bb_2 \\
\Bd &= \Ba_0 - \Bb_0,
\end{aligned}
\end{dmath}

the desired equation to solve is

\begin{dmath}\label{eqn:solutionOfLinearSystem:1060}
\Bd + s \Bu_1 = t \Bu_2.
\end{dmath}

Solving for \( s \) or \( t \) is possible by
wedging both sides with one of \( \Bu_1 \) or \( \Bu_2 \)

\begin{dmath}\label{eqn:solutionOfLinearSystem:1080}
\begin{aligned}
\Bd \wedge \Bu_1 &= t \Bu_2 \wedge \Bu_1 \\
\Bd \wedge \Bu_2 + s \Bu_1 \wedge \Bu_2 &= 0,
\end{aligned}
\end{dmath}

In \R{2} these equations have a solution if \( \Bu_1 \wedge \Bu_2 \ne 0 \), and in \R{N} these have solutions if the bivectors on each sides of the equations describe the same plane.
Put another way, these have solutions when \( s \) and \( t \) are scalars and

\begin{dmath}\label{eqn:solutionOfLinearSystem:1100}
\begin{aligned}
s &= \frac{\Bu_2 \wedge \Bd}{\Bu_1 \wedge \Bu_2} \\
t &= \frac{\Bu_1 \wedge \Bd}{\Bu_1 \wedge \Bu_2}.
\end{aligned}
\end{dmath}

In
\R{2}
with
\begin{dmath}\label{eqn:solutionOfLinearSystem:1120}
\begin{aligned}
\Bu_1 &= u_{11} \Be_1 + u_{12} \Be_2 \\
\Bu_2 &= u_{21} \Be_1 + u_{22} \Be_2 \\
\Bd &= d_{1} \Be_1 + d_{2} \Be_2,
\end{aligned}
\end{dmath}

the wedge products in \cref{eqn:solutionOfLinearSystem:1100}
can be expressed explicitly as a (unit bivector scaled) determinants

\begin{equation}\label{eqn:solutionOfLinearSystem:1140}
%\begin{aligned}
s =
\frac{
\begin{vmatrix}
u_{21} & u_{22} \\
d_1 & d_2
\end{vmatrix}
\Be_{12}
}
{
\begin{vmatrix}
u_{11} & u_{12} \\
u_{21} & u_{22} \\
\end{vmatrix}
\Be_{12}
}
%=
%\frac{
%\begin{vmatrix}
%u_{21} & u_{22} \\
%d_1 & d_2
%\end{vmatrix}
%}
%{
%\begin{vmatrix}
%u_{11} & u_{12} \\
%u_{21} & u_{22} \\
%\end{vmatrix}
%}
\qquad
t =
\frac{
\begin{vmatrix}
u_{11} & u_{12} \\
d_1 & d_2
\end{vmatrix}
\Be_{12}
}
{
\begin{vmatrix}
u_{11} & u_{12} \\
u_{21} & u_{22} \\
\end{vmatrix}
\Be_{12}
}
%=
%\frac{
%\begin{vmatrix}
%u_{11} & u_{12} \\
%d_1 & d_2
%\end{vmatrix}
%}
%{
%\begin{vmatrix}
%u_{11} & u_{12} \\
%u_{21} & u_{22} \\
%\end{vmatrix}
%}
.
%\end{aligned}
\end{equation}

Once the unit bivectors \( \Be_{12} \) are cancelled \cref{eqn:solutionOfLinearSystem:1140} is the Cramer's rule solution of the problem.  Cramer's rule is seen to follow directly from the use of the wedge product to eliminate factors that are not of interest.
In a similar way, the use of the wedge product for a 3D intersection problem with three variables, will lead directly to the Cramer's rule solution.

\makeproblem{Intersection of a line and plane.}{problem:solutionOfLinearSystem:1}{
Let a line be parameterized by
\begin{equation*}
\Br(a) = \Bp + a \Ba,
\end{equation*}
and a plane be parameterized by
\begin{equation*}
\Br(b,c) = \Bq + b \Bb + c \Bc.
\end{equation*}
\makesubproblem{}{problem:solutionOfLinearSystem:1:a}
State the vector equation to be solved, and its solution for \( a \) in terms of a ratio of wedge products.
\makesubproblem{}{problem:solutionOfLinearSystem:1:b}
State the conditions for which the solution exist in \R{3} and \R{N}.
\makesubproblem{}{problem:solutionOfLinearSystem:1:c}
In terms of coordinates in \R{3} write out the ratio of wedge products as determinants and compare to the Cramer's rule solution.
} % problem
