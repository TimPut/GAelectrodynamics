%
% Copyright © 2017 Peeter Joot.  All Rights Reserved.
% Licenced as described in the file LICENSE under the root directory of this GIT repository.
%
\subsection{GA statement.}
The Lorentz force equation \cref{eqn:freespace:200}
can be restated
in terms of \( F = \BE + I \eta \BH = \BE + I c \BB \) as

\begin{dmath}\label{eqn:lorentzForce:20}
%\boxedEquation{eqn:lorentzForce:20}{
\frac{d\Bp}{dt} = q \gpgradeone{ F \lr{ 1 + \frac{\Bv}{c} } },
%}
\end{dmath}

which puts the electric and magnetic fields on equal footing.
This can be demonstrated by splitting the \( F \lr{ 1 + \ifrac{\Bv}{c} } \) multivector into its constituent grades

\begin{dmath}\label{eqn:lorentzForce:40}
q F \lr{ 1 + \frac{\Bv}{c} }
=
q
\lr{ \BE + I c \BB }
\lr{ 1 + \frac{\Bv}{c} }
=
q \BE
+ q I \BB \Bv
+ \frac{q}{c} \BE \Bv
+ q c I \BB
=
  \frac{q}{c} \BE \cdot \Bv
+ q \lr{ \BE + \Bv \cross \BB }
+ q \lr{ c I \BB + \inv{c} \BE \wedge \Bv }
+ q (I \BB) \wedge \Bv.
\end{dmath}

The grade 0 component of this product hints of \cref{eqn:freespace:220}, and substitution into the vector grade selection operation of \cref{eqn:lorentzForce:20} recovers \cref{eqn:freespace:200} as desired.

Looking to the stress tensor for the continuum equivalent of the dual Lorentz force equation \cref{eqn:poyntingLorentzForce:140}, we can introduce a generalized multivector charge

\begin{dmath}\label{eqn:lorentzForce:240}
Q =
q_\txte \lr{ 1 + \Bv_\txte/c }
-I \epsilon q_\txtm \lr{ 1 + \Bv_\txtm/c },
\end{dmath}

where \( q_\txte, q_\txtm \) are the electric and magnetic charges, and \( \Bv_\txte, \Bv_\txtm \) are their respective velocities.
The Lorentz force equation, including both electric and fictious magnetic charges, can now be written as
%\begin{dmath}\label{eqn:lorentzForce:260}
\boxedEquation{eqn:lorentzForce:260}{
\frac{d\Bp}{dt} = \gpgradeone{ F Q }.
}
%\end{dmath}

\subsection{Constant magnetic field.}

The Lorentz force equation that determines the dynamics of a charged particle in an external field \( F \) has been restated as a multivector differential equation, but how to solve such an equation is probably not obvious.
Given a constant external magnetic field, the Lorentz force equation is reduced to

\begin{dmath}\label{eqn:lorentzForce:60}
m \frac{d\Bv}{dt} = q (I \BB) \cdot \Bv,
\end{dmath}

or
\begin{dmath}\label{eqn:lorentzForce:80}
\begin{aligned}
\Omega &= -\frac{q I \BB}{m} \\
\frac{d\Bv}{dt} &= \Bv \cdot \Omega,
\end{aligned}
\end{dmath}

where \( \Omega \) is a bivector containing all the constant factors.

This can be solved by introducing a multivector integration factor \( R \) and its reverse \( R^\dagger \) on the left and right respectively

\begin{dmath}\label{eqn:lorentzForce:100}
R \frac{d\Bv}{dt} R^\dagger
= R \Bv \cdot \Omega R^\dagger
= \inv{2} R \lr{ \Bv \Omega - \Omega \Bv} R^\dagger
= \inv{2} R \Bv \Omega R^\dagger - \inv{2} R \Omega \Bv R^\dagger,
\end{dmath}

or
\begin{dmath}\label{eqn:lorentzForce:120}
0 =
R \frac{d\Bv}{dt} R^\dagger
+ \inv{2} R \Omega \Bv R^\dagger
- \inv{2} R \Bv \Omega R^\dagger
\end{dmath}

Let
\begin{dmath}\label{eqn:lorentzForce:140}
\dot{R} = R \Omega/2.
\end{dmath}

Since \( \Omega \) is a bivector \( \dot{R}^\dagger = -\Omega R^\dagger/2 \), so by chain rule

\begin{dmath}\label{eqn:lorentzForce:160}
0
=
\frac{d}{dt} \lr{
R \Bv R^\dagger
}.
\end{dmath}

The integrating factor has solution

\begin{dmath}\label{eqn:lorentzForce:180}
R = e^{\Omega t/2},
\end{dmath}

a ``complex exponential'', so the solution of \cref{eqn:lorentzForce:60} is

\begin{dmath}\label{eqn:lorentzForce:200}
\Bv(t) = e^{-\Omega t/2} \Bv(0) e^{\Omega t/2}.
\end{dmath}

The velocity of the charged particle traces out a helical path.
Any component of the initial velocity \( \Bv(0)_\perp \) perpendicular to the \( \Omega \) plane is untouched by this rotation operation, whereas components of the initial velocity \( \Bv(0)_\parallel \) that lie in the \( \Omega \) plane will trace out a circular path.
If \( \hat{\Omega} \) is the unit bivector for this plane, that velocity is

\begin{dmath}\label{eqn:lorentzForce:220}
\begin{aligned}
\Bv(0)_\parallel &= \lr{ \Bv(0) \cdot \hat{\Omega} } \hat{\Omega}^{-1} \\
\Bv(0)_\perp &= \lr{ \Bv(0) \wedge \hat{\Omega} } \hat{\Omega}^{-1} \\
\Bv(t) &= \Bv(0)_\parallel e^{\Omega t} + \Bv(0)_\perp.
\end{aligned}
\end{dmath}

A multivector integration factor method for solving the Lorentz force equation in constant external electric and magnetic fields can be found in \citep{hestenes1999nfc}.  Other examples, solved using a relativistic formulation of GA, can be found in \citep{doran2003gap},
\citep{hestenes1974properdynamics}, and
\citep{hestenes1974propermechanics}.
