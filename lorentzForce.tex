%
% Copyright © 2017 Peeter Joot.  All Rights Reserved.
% Licenced as described in the file LICENSE under the root directory of this GIT repository.
%
The Lorentz force equation \cref{eqn:freespace:200}
can be restated
in terms of \( F = \BE + I \eta \BH = \BE + I c \BB \) as

\begin{dmath}\label{eqn:lorentzForce:20}
%\boxedEquation{eqn:lorentzForce:20}{
\frac{d\Bp}{dt} = q \gpgradeone{ F \lr{ 1 + \frac{\Bv}{c} } },
%}
\end{dmath}
which puts the electric and magnetic fields on equal footing.
This can be demonstrated by splitting the \( F \lr{ 1 + \ifrac{\Bv}{c} } \) multivector into its constituent grades

\begin{dmath}\label{eqn:lorentzForce:40}
q F \lr{ 1 + \frac{\Bv}{c} }
=
q
\lr{ \BE + I c \BB }
\lr{ 1 + \frac{\Bv}{c} }
=
q \BE
+ q I \BB \Bv
+ \frac{q}{c} \BE \Bv
+ q c I \BB
=
  \frac{q}{c} \BE \cdot \Bv
+ q \lr{ \BE + \Bv \cross \BB }
+ q \lr{ c I \BB + \inv{c} \BE \wedge \Bv }
+ q (I \BB) \wedge \Bv.
\end{dmath}

The grade 0 component of this product hints of \cref{eqn:freespace:220}, and substitution into the vector grade selection operation of \cref{eqn:lorentzForce:20} recovers \cref{eqn:freespace:200} as desired.

Looking to the energy-momentum tensor for the continuum equivalent of the dual Lorentz force equation \cref{eqn:poyntingLorentzForce:140}, we can introduce a generalized multivector charge

\begin{dmath}\label{eqn:lorentzForce:240}
Q =
q_\txte \lr{ 1 + \Bv_\txte/c }
-I \epsilon q_\txtm \lr{ 1 + \Bv_\txtm/c },
\end{dmath}
where \( q_\txte, q_\txtm \) are the electric and magnetic charges, and \( \Bv_\txte, \Bv_\txtm \) are their respective velocities.
The Lorentz force equation, including both electric and fictious magnetic charges, can now be written as
%\begin{dmath}\label{eqn:lorentzForce:260}
\boxedEquation{eqn:lorentzForce:260}{
\frac{d\Bp}{dt} = \gpgradeone{ F Q }.
}
%\end{dmath}

