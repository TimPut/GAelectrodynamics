%
% Copyright © 2018 Peeter Joot.  All Rights Reserved.
% Licenced as described in the file LICENSE under the root directory of this GIT repository.
%
%{
\index{far field}

Given a
vector potential with a
radial spherical wave representation
\begin{dmath}\label{eqn:potentialSection_farfield:2400}
\BA = \frac{e^{-j k r}}{r} \bcA( \theta, \phi ),
\end{dmath}
we can compute the far field ( \( r \gg 1 \) ) electrodynamic field \( F \).
The spherical representation of the gradient is
\begin{dmath}\label{eqn:potentialSection_farfield:2420}
\begin{aligned}
\spacegrad &= \rcap \partial_r + \spacegrad_\perp \\
\spacegrad_\perp &= \frac{\thetacap}{r} \partial_\theta + \frac{\phicap}{r\sin\theta} \partial_\phi.
\end{aligned}
\end{dmath}

The gradient of the vector potential is
\begin{dmath}\label{eqn:potentialSection_farfield:2440}
\spacegrad \BA
=
\biglr{ \rcap \partial_r + \spacegrad_\perp } \frac{e^{-j k r}}{r} \bcA
=
\rcap \lr{ -j k - \inv{r} } \frac{e^{-j k r}}{r} \bcA
+
\frac{e^{-j k r}}{r}
\spacegrad_\perp
\bcA
= - \lr{ j k + \inv{r} } \rcap \BA + O(1/r^2)
\approx
- j k \rcap \BA.
\end{dmath}

Here, all the \( O(1/r^2) \) terms, including the action of the non-radial component of the gradient on the \( 1/r \) potential, have been neglected.
From \cref{eqn:potentialSection_farfield:2440} the far field divergence and the (bivector) curl of \( \BA \) are
\begin{dmath}\label{eqn:potentialSection_farfield:2460}
\begin{aligned}
\spacegrad \cdot \BA &= - j k \rcap \cdot \BA \\
\spacegrad \wedge \BA &= - j k \rcap \wedge \BA.
\end{aligned}
\end{dmath}

Finally, the far field gradient of the divergence of \( \BA \) is
\begin{dmath}\label{eqn:potentialSection_farfield:2480}
\spacegrad \lr{ \spacegrad \cdot \BA }
=
\biglr{ \rcap \partial_r + \spacegrad_\perp } \lr{ - j k \rcap \cdot \BA }
\approx
-j k \rcap \partial_r \lr{ \rcap \cdot \BA }
=
-j k \rcap \lr{ -j k - \inv{r} } \lr{ \rcap \cdot \BA }
\approx
-k^2 \rcap \lr{ \rcap \cdot \BA },
\end{dmath}
again neglecting any \( O(1/r^2) \) terms.  The field is
\begin{dmath}\label{eqn:potentialSection_farfield:2500}
F
=
- j \omega \BA  -j \frac{c^2}{\omega} \spacegrad \lr{ \spacegrad \cdot \BA } + c \spacegrad \wedge \BA
=
- j \omega \BA  +j \omega \rcap \lr{ \rcap \cdot \BA } - j k c \rcap \wedge \BA
=
- j \omega \lr{ \BA - \rcap \lr{ \rcap \cdot \BA }} - j \omega \rcap \wedge \BA
=
-j \omega \rcap \lr{ \rcap \wedge \BA} - j \omega \rcap \wedge \BA,
\end{dmath}
or
%\begin{dmath}\label{eqn:potentialSection_farfield:2520}
\boxedEquation{eqn:potentialSection_farfield:2520}{
F = -j \omega \lr{ \rcap + 1 } \lr{ \rcap \wedge \BA}.
}
%\end{dmath}

One interpretation of this is that the (bivector) magnetic field is represented by the plane perpendicular to the direction of propagation, and the electric field by a vector in that plane.
These electric and magnetic fields can be extracted by inspection.
Let \( \BA_\perp = \rcap \lr{ \rcap \wedge \BA} \) represent the
non-radial component of the potential, so these respective fields are
\begin{dmath}\label{eqn:potentialSection_farfield:2560}
\begin{aligned}
\BE &= -j \omega \BA_\perp \\
\BH &= \inv{\eta} \rcap \cross \BE.
\end{aligned}
\end{dmath}

Having calculated the far field approximation for the electrodynamic field for a vector potential \( \BA \) the field for the magnetic source vector potential \( \BF \) is
%\begin{dmath}\label{eqn:potentialSection_farfield:2580}
\boxedEquation{eqn:potentialSection_farfield:2580}{
F = -j \omega \eta I \lr{ \rcap + 1 } \lr{ \rcap \wedge \BF },
}
%\end{dmath}
for which the electric and magnetic field components are
\begin{dmath}\label{eqn:potentialSection_farfield:2600}
\begin{aligned}
\BE &= j \omega \eta \rcap \cross \BF \\
\BH &= -j \omega \BF_\perp.
\end{aligned}
\end{dmath}

\makeexample{Vertical dipole potential.}{example:potentialSection:1}{
We will calculate the far field along the propagation direction vector \( \kcap \) in the z-y plane
\begin{dmath}\label{eqn:potentialSection_farfield:2620}
\begin{aligned}
\kcap &= \Be_3 e^{i \theta} \\
i &= \Be_{32},
\end{aligned}
\end{dmath}
for the infinitesimal dipole potential
\begin{dmath}\label{eqn:potentialSection_farfield:2640}
\BA = \frac{\Be_3 \mu I_0 l}{4 \pi r} e^{-j k r},
\end{dmath}
as illustrated in \cref{fig:vectorPotential:vectorPotentialFig1}.

\imageFigure{../figures/GAelectrodynamics/vectorPotentialFig1}{Vertical infinitesimal dipole and selected propagation direction.}{fig:vectorPotential:vectorPotentialFig1}{0.3}

The wedge of \( \kcap \) with \( \BA \) is proportional to
\begin{dmath}\label{eqn:potentialSection_farfield:2660}
\kcap \wedge \Be_3
=
\gpgradetwo{
\kcap \Be_3
}
=
\gpgradetwo{
\Be_3 e^{i \theta}
\Be_3
}
=
\gpgradetwo{
\Be_3^2 e^{-i \theta}
}
=
-i \sin\theta,
\end{dmath}
so from \cref{eqn:potentialSection_farfield:2520} the field is
\begin{dmath}\label{eqn:potentialSection_farfield:2680}
F = j \omega \lr{ 1 + \Be_3 e^{i\theta} } i \sin\theta \frac{\mu I_0 l}{4 \pi r} e^{-j k r}.
\end{dmath}

The electric and magnetic fields can be found from the respective vector and bivector grades of \cref{eqn:potentialSection_farfield:2680}
\begin{dmath}\label{eqn:potentialSection_farfield:2700}
\BE
=
\frac{j \omega \mu I_0 l}{4 \pi r} e^{-j k r} \Be_3 e^{i\theta} i \sin\theta
=
\frac{j \omega \mu I_0 l}{4 \pi r} e^{-j k r} \Be_2 e^{i\theta} \sin\theta
=
\frac{j k \eta I_0 l \sin\theta}{4 \pi r} e^{-j k r} \lr{ \Be_2 \cos\theta - \Be_3 \sin\theta },
\end{dmath}
and
\begin{dmath}\label{eqn:potentialSection_farfield:2720}
\BH
=
\inv{I \eta}
j \omega i \sin\theta_0 \frac{\mu I_0 l}{4 \pi r} e^{-j k r}
=
\inv{\eta} \Be_{321} \Be_{32}
j \omega \sin\theta_0 \frac{\mu I_0 l}{4 \pi r} e^{-j k r}
=
-\Be_1 \frac{ j k \sin\theta_0 I_0 l}{4 \pi r} e^{-j k r}.
\end{dmath}

The multivector electrodynamic field expression
\cref{eqn:potentialSection_farfield:2680} for
\( F \) is more algebraically compact than the separate electric and magnetic field expressions, but this comes with the complexity of dealing with different types of imaginaries.
There are two explicit unit imaginaries in \cref{eqn:potentialSection_farfield:2680}, the scalar imaginary \( j \) used to encode the time harmonic nature of the field, and \( i = \Be_{32} \) used to represent the plane that the far field propagation direction vector lay in.
Additionally, when the magnetic field component was extracted, the pseudoscalar \( I = \Be_{123} \) entered into the mix.
Care is required to keep these all separate, especially since \( I, j \) commute with all grades, but \( i \) does not.
} % example

%}
