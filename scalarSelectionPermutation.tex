%
% Copyright © 2017 Peeter Joot.  All Rights Reserved.
% Licenced as described in the file LICENSE under the root directory of this GIT repository.
%
As scalar selection is at the heart of the
generalized dot product, it is worth knowing
some of the ways that such a selection operation can be manipulated.

\maketheorem{Permutation of multivector products in scalar selection.}{theorem:scalarPermutation:1}{
The factors within a scalar grade selection of a pair of multivector products may be permuted or may be cyclically permuted
\begin{equation*}
\begin{aligned}
\gpgradezero{A B} &= \gpgradezero{B A} \\
\gpgradezero{A B \cdots Y Z} &= \gpgradezero{Z A B \cdots Y}.
\end{aligned}
\end{equation*}
} % problem

It is sufficient to prove just the two multivector permutation case.
One simple, but inelegant method, is to first expand the pair of multivectors in coordinates.
Let

\begin{equation}\label{eqn:scalarSelectionPermutation:40}
\begin{aligned}
A &= a_0 + \sum_i a_i \Be_i + \sum_{i < j} a_{ij} \Be_{ij} + \cdots \\
B &= b_0 + \sum_i b_i \Be_i + \sum_{i < j} b_{ij} \Be_{ij} + \cdots
\end{aligned}
\end{equation}

Only the products of like unit blades can contribute scalar components to the sum, so the
the scalar selection of the products must have the form

\begin{dmath}\label{eqn:scalarSelectionPermutation:60}
\gpgradezero{ A B }
=
a_0 b_0 + \sum_i a_i b_i \Be_i^2 + \sum_{i < j} a_{ij} b_{ij} \Be_{ij}^2 + \cdots
\end{dmath}

This sum is also clearly equal to \( \gpgradezero{ B A } \), completing the proof.
