%
% Copyright © 2018 Peeter Joot.  All Rights Reserved.
% Licenced as described in the file LICENSE under the root directory of this GIT repository.
%
%{
An elliptical polarized electric field can be parameterized as
\begin{dmath}\label{eqn:polarization_elliptical:340}
\BE
=
E_a \Be_1 \cos\theta + E_b \Be_2 \sin\theta,
\end{dmath}

which corresponds to a Jones vector \( (E_a, -i E_b) \), or circular polarization coefficients with values

\begin{dmath}\label{eqn:polarization_elliptical:400}
\begin{aligned}
\alpha_\txtL &= \inv{2}\lr{ E_a - E_b } \\
\alpha_\txtR &= \inv{2}\lr{ E_a + E_b }.
\end{aligned}
\end{dmath}

Therefore an elliptically polarized field can be represented as

\begin{dmath}\label{eqn:polarization_elliptical:420}
F = \inv{2} (1 + \Be_3) \Be_1 \lr{ (E_a + E_b) e^{i\phi} + (E_a - E_b) e^{-i\phi} }.
\end{dmath}

An interesting variation of the elliptical polarization uses a hyperbolic parameterization.
If \( a, b \) are the semi-major/minor axes of the ellipse (i.e. \( a > b \)),
and \( \Ba = a \Be_1 e^{i\psi} \) is the vectoral representation of the semi-major axis (not necessarily placed along \( \Be_1 \)),
and \( e = \sqrt{1 - (b/a)^2} \) is the eccentricity of the ellipse,
then it can be shown (\citep{hestenes1999nfc})
that an elliptic parameterization can be written
in the compact form

\begin{dmath}\label{eqn:polarization_elliptical:360}
\Br(\phi)
=
e \Ba \cosh( \tanh^{-1}(b/a) + i \phi).
\end{dmath}

When the bivector imaginary \( i = \Be_{12} \) is used then
this parameterization is real and has only vector grades, so the electromagnetic field for a general elliptic wave has the form

\begin{dmath}\label{eqn:polarization_elliptical:380}
\begin{aligned}
F &= e E_a \lr{ 1 + \Be_3 } \Be_1 e^{ i \psi } \cosh\lr{ m + i \phi} \\
m &= \tanh^{-1}\lr{ E_b/E_a } \\
e &= \sqrt{1 - {(E_b/E_a)}^2 },
\end{aligned}
\end{dmath}

where \( E_a(E_b) \) are the magnitudes of the electric field components lying along the semi-major(minor) axes, and the propagation direction \( \Be_3 \) is normal to both the major and minor axis directions.
An elliptic electric field polarization is illustrated in \cref{fig:ellipticalPolarization:ellipticalPolarizationFig1}, where the vectors representing the major and minor axes are \( \BE_a = E_a \Be_1 e^{i\psi}, \BE_b = E_b \Be_1 e^{i\psi} \).
Observe that setting \( E_b = 0 \) results in the linearly polarized field of \cref{eqn:polarization_linearPolarization:300}.
\imageFigure{../figures/GAelectrodynamics/ellipticalPolarizationFig1}{Electric field with elliptical polarization.}{fig:ellipticalPolarization:ellipticalPolarizationFig1}{0.3}

Following the procedure of \cref{eqn:polarization_phaseAndEnergyMomentum:560}, the energy-momentum of an elliptically polarized field is
\begin{dmath}\label{eqn:polarization_elliptical:600}
\calE + \frac{\BS}{v}
=
\inv{2} \epsilon
F F^\dagger
=
\inv{2} \epsilon
e^2 E_a^2 \lr{ 1 + \Be_3 } \Be_1 \cancel{e^{ i \psi }} \cosh\lr{ m + i \phi}
\cosh\lr{ m - i \phi}
\cancel{e^{ -i \psi } }
\Be_1
\lr{ 1 + \Be_3 }
=
\inv{2} \epsilon
e^2 E_a^2 \lr{ 1 + \Be_3 }
\lr{ \cosh(2m) + \cos(2 \phi) }
=
\inv{2} \epsilon
\lr{ 1 + \Be_3 }
\lr{ E_b^2 + 2 \lr{
E_a^2 - E_b^2
 } \cos^2 \phi }
.
\end{dmath}

The simplification above made use of the identity \( (1 - (b/a)^2) \cosh(2 \Atanh(b/a)) = 1 + (b/a)^2 \).
%
% $Assumptions = b > 0 && b < 1 && a > 0 && a > b;
% (1 - (b/a)^2) Cosh[2 ArcTanh[b/a]] // FullSimplify

%}
