%
% Copyright © 2017 Peeter Joot.  All Rights Reserved.
% Licenced as described in the file LICENSE under the root directory of this GIT repository.
%
The dot product with scalar grade selection allows
identification,
in general we define the dot product of multivectors in terms of a similar grade selection

\index{multivector dot product}
\makedefinition{Multivector dot product}{dfn:gradeselection:100}{
The dot (or inner) product of two multivectors
\( A = \sum_{i = 0}^N \gpgrade{A}{i}, B = \sum_{i = 0}^N \gpgrade{B}{i} \)
is defined as
\begin{equation*}
A \cdot B \equiv
\sum_{i,j = 0}^N \gpgrade{ A_i B_j }{\Abs{i - j}},
\end{equation*}

where \( A_i = \gpgrade{ A }{i} \)
and \( B_i = \gpgrade{ B }{i} \) are the respective grades of the multivectors \( A, B \).
} % definition

Like the multivector dot product, we define a multivector wedge product as a grade selection operation

\index{multivector wedge product}
\makedefinition{Multivector wedge product.}{dfn:gradeselection:480}{
For the multivectors \( A, B \) defined in \cref{dfn:gradeselection:100}, the wedge (or outer) product is defined as
\begin{equation*}
A \wedge B
\equiv
\sum_{i,j = 0}^N \gpgrade{ A_i B_j }{i + j}.
\end{equation*}
} % definition
