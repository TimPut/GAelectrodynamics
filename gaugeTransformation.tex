%
% Copyright � 2018 Peeter Joot.  All Rights Reserved.
% Licenced as described in the file LICENSE under the root directory of this GIT repository.
%
%{
\input{../latex/blogpost.tex}
\renewcommand{\basename}{gaugeTransformation}
%\renewcommand{\dirname}{notes/phy1520/}
\renewcommand{\dirname}{notes/ece1228-electromagnetic-theory/}
%\newcommand{\dateintitle}{}
%\newcommand{\keywords}{}

\input{../latex/peeter_prologue_print2.tex}

\usepackage{peeters_layout_exercise}
\usepackage{peeters_braket}
\usepackage{peeters_figures}
\usepackage{siunitx}
%\usepackage{mhchem} % \ce{}
%\usepackage{macros_bm} % \bcM
\usepackage{macros_qed} % \qedmarker
%\usepackage{txfonts} % \ointclockwise

%\newcommand{\dLambertian}[0]{\Box}
\newcommand{\dLambertian}[0]{\square}

\newcommand{\stgrad}[0]{\lr{ \spacegrad + \inv{c} \PD{t}{}}}
\newcommand{\conjstgrad}[0]{\lr{ \spacegrad - \inv{c} \PD{t}{}}}

\beginArtNoToc

\generatetitle{Multivector potentials.}
%\chapter{Multivector potentials.}
%\label{chap:gaugeTransformation}

Conventional electromagnetism utilizes scalar and vector potentials, so it is reasonable to expect that
the desired multivector representation of the potential is a grade 0,1 multivector.
A potential representation with grades 2,3 works for (fictitous) magnetic sources, so we may generally
allow a multivector potential to have any grades.  Such a potential is related to the field as follows.

\makedefinition{Multivector potential.}{thm:generalPotential:80}{
The electromagnetic field strength \( F \) for a \textit{multivector potential} \( A \) is
\begin{equation*}
F = \gpgrade{\conjstgrad A}{1,2}.
\end{equation*}
} % definition

Before unpacking \( \conjstgrad A \), we want to label the 
different grades of the multivector potential, and do so in a way that is consisent with the conventional 
potential representation of the electric and magnetic fields.
\makedefinition{Multivector potential representation.}{dfn:unpackStaticPotential:80}{
Let
%\label{eqn:potentialSection:40}
\begin{equation*}
A =
      - \phi
      + c \BA
      + \eta I \lr{ -\phi_m + c \BF },
\end{equation*}
where
\begin{enumerate}
\item \( \phi \) is the scalar potential \si{V} (Volts).
\item \( \BA \) is the vector potential \si{W/m} (Webers/meter).
\item \( \phi_m \) is the scalar potential for (fictitious) magnetic sources \si{A} (Amperes).
\item \( \BF \) is the vector potential for (fictitious) magnetic sources \si{C} (Coulombs).
\end{enumerate}
} % definition
This specific breakdown of \( A \) into scalar and vector potentials, and dual (pseudoscalar and bivector) potentials has been chosen to match SI conventions, specifically those of \citep{balanis2005antenna} (which includes fictitious magnetic sources.)

We can now express the fields in terms of the potentials.

\maketheorem{Fields and the potential wave equations.}{thm:generalPotential:40}{
In terms of the potential components, the electric field vector and the magnetic field bivector are
\begin{equation*}
\begin{aligned}
\BE &=
\gpgrade{\conjstgrad A}{1}
=
   - \spacegrad \phi
   - \PD{t}{\BA}
   - \inv{\epsilon} \spacegrad \cross \BF \\
I \eta \BH &=
\gpgrade{\conjstgrad A}{2}
=
   I \eta
   \lr{
      - \spacegrad \phi_\txtm
      - \PD{t}{\BF}
      + \inv{\mu} \spacegrad \cross \BA
   }
.
\end{aligned}
\end{equation*}
The potentials are related to the sources by
\begin{equation*}
\begin{aligned}
\dLambertian
\phi &= -\frac{\rho}{\epsilon} - \PD{t}{} \lr{ \spacegrad \cdot \BA + \inv{c^2} \PD{t}{\phi} } \\
\dLambertian
\BA &= -\mu \BJ + \spacegrad \lr{ \spacegrad \cdot \BA + \inv{c^2} \PD{t}{\phi} } \\
\dLambertian
\BF &= - \epsilon \BM + \spacegrad \lr{ \spacegrad \cdot \BF + \inv{c^2} \PD{t}{\phi_\txtm} } \\
\dLambertian
\phi_\txtm &= -\frac{\rho_\txtm}{\mu} - \PD{t}{} \lr{ \spacegrad \cdot \BF + \inv{c^2} \PD{t}{\phi_\txtm} }
\end{aligned}
\end{equation*}
} % theorem

To prove \cref{thm:generalPotential:40} we start by expanding \( (\spacegrad - (1/c)\partial_t) A \) using
\cref{dfn:unpackStaticPotential:80} and then group by grade to find
\begin{dmath}\label{eqn:generalPotential:60}
\begin{aligned}
\conjstgrad A
&=
\conjstgrad \lr{  - \phi
      + c \BA
      + \eta I \lr{ -\phi_m + c \BF } } \\
&=
- \spacegrad \phi + c \spacegrad \cdot \BA + c \spacegrad \wedge \BA + \inv{c} \PD{t}{\phi} - \PD{t}{\BA} \\
&\quad + I \eta
\lr{
- \spacegrad \phi_\txtm + c \spacegrad \cdot \BF + c \spacegrad \wedge \BF + \inv{c} \PD{t}{\phi_\txtm} - \PD{t}{\BF}
} \\
&=
c \spacegrad \cdot \BA
+ \inv{c} \PD{t}{\phi}
\\
&
+
\mathLabelBox[ labelstyle={below of=m\themathLableNode, below of=m\themathLableNode} ]
{
   - \spacegrad \phi
   - \PD{t}{\BA}
   - \inv{\epsilon} \spacegrad \cross \BF
}
{
\(\BE\)
}
+
\mathLabelBox[ labelstyle={below of=m\themathLableNode, below of=m\themathLableNode} ]
{
   I \eta
   \lr{
      - \spacegrad \phi_\txtm
      - \PD{t}{\BF}
      + \inv{\mu} \spacegrad \cross \BA
   }
}
{\(I \eta \BH\)
} \\
&
+ I \eta\lr{
  c \spacegrad \cdot \BF
+ \inv{c} \PD{t}{\phi_\txtm}
},
\end{aligned}
\end{dmath}
which shows the claimed field split.

In terms of the potentials Maxwell's equation \( \stgrad F = J \) is
\begin{dmath}\label{eqn:gaugeTransformation:20}
\stgrad \gpgrade{\conjstgrad A}{1,2} = J,
\end{dmath}
or
\begin{dmath}\label{eqn:gaugeTransformation:40}
\dLambertian A = J + \stgrad \gpgrade{\conjstgrad A}{0,3}.
\end{dmath}
This is almost a wave equation.  Inserting \cref{eqn:generalPotential:60} into \cref{eqn:gaugeTransformation:40} and selecting each grade gives four almost-wave equations
\begin{equation*}
\begin{aligned}
-
\dLambertian
\phi &= \frac{\rho}{\epsilon} + \inv{c} \PD{t}{} \lr{ c \spacegrad \cdot \BA + \inv{c} \PD{t}{\phi} } \\
c
\dLambertian
\BA &= -\eta \BJ + \spacegrad \lr{ c \spacegrad \cdot \BA + \inv{c} \PD{t}{\phi} } \\
\eta c I
\dLambertian
\BF &= - I \BM + \spacegrad \cdot \lr{ I \eta\lr{ c \spacegrad \cdot \BF + \inv{c} \PD{t}{\phi_\txtm} } } \\
-I \eta
\dLambertian
\phi_\txtm &= I c \rho_\txtm + \inv{c} \PD{t}{} I \eta\lr{ c \spacegrad \cdot \BF + \inv{c} \PD{t}{\phi_\txtm} }
\end{aligned}
\end{equation*}
Using \( \eta = \mu c, \eta c \epsilon = 1 \), and
\( \spacegrad \cdot (I \psi) = I \spacegrad \psi \) for scalar \(\psi\), a bit
of rearrangement completes the proof.

Clearly it is desirable if potentials can be found for which \( \spacegrad \cdot \BA + (1/c^2) \partial_t \phi = \spacegrad \cdot \BF + (1/c^2) \partial_t \phi_\txtm = 0 \).  Finding such potentials relies on the fact that the potential representation is not unique.  In particular, 
we have the freedom to add any spacetime gradient to a potential without changing the field.
\maketheorem{Gauge invariance.}{thm:gaugeTransformation:60}{
Adding the spacetime gradient to a potential
\begin{equation*}
A' = A + \stgrad \psi,
\end{equation*}
leaves the field unchanged.  That is
\begin{equation*}
F
= \gpgrade{\conjstgrad A}{1,2}
= \gpgrade{\conjstgrad A'}{1,2}.
\end{equation*}
} % theorem

%}
\EndArticle
