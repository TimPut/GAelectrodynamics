%
% Copyright � 2017 Peeter Joot.  All Rights Reserved.
% Licenced as described in the file LICENSE under the root directory of this GIT repository.
%
%{
\subsection{General potential representation.}

For both electrostatics and magnetostatics, where Maxwell's equations are both a pair of gradients, we were able to require that the respective scalar and vector potentials were both gradients.  For electrodynamics where Maxwell's equation is

\begin{dmath}\label{eqn:potentialSection:1800}
\lr{ \spacegrad + \inv{v} \PD{t}{} } F = J,
\end{dmath}

it seems more reasonable to demand a different structure of the potential, say

\begin{dmath}\label{eqn:potentialSection:1820}
F = \lr{ \spacegrad - \inv{v} \PD{t}{} } A,
\end{dmath}

where \( A \) is a multivector potential that may contain all grades, with structure to be determined.
If such a multivector potential can be found, then Maxwell's equation is reduced to a single wave equation

\begin{dmath}\label{eqn:potentialSection:1840}
\lr{ \spacegrad^2 - \inv{v^2} \PDSq{t}{} } A = J,
\end{dmath}

which can be thought of as one wave equation for each multivector grade of the multivector source \( J \).

Some thought shows that the guess \cref{eqn:potentialSection:1820} is not quite right, as it allows for the invalid possibility that \( F \) has scalar or pseudoscalar grades.
While it is possible to impose constraints (a gauge choice) on potential \( A \) that ensure
\( F \) has only the vector and bivector grades,
in general,
a grade selection filter must be imposed

\boxedEquation{eqn:potentialSection:1860}{
F
=
\gpgrade{ \lr{ \spacegrad - \inv{v} \PD{t}{} } A }{1,2}.
}

We will find that the desired representation of the multivector potential is

\begin{dmath}\label{eqn:potentialSection:40}
A =
      - \phi
      + v \BA
      + \eta I \lr{ -\phi_m + v \BF }.
\end{dmath}

Here

\begin{enumerate}
\item \( \phi \) is the scalar potential \si{V} (Volts).
\item \( \BA \) is the vector potential \si{W/m} (Webers/meter).
\item \( \phi_m \) is the scalar potential for (fictitious) magnetic current sources \si{A} (Amperes).
\item \( \BF \) is the vector potential for (fictitious) magnetic current sources \si{C} (Coulombs).
\end{enumerate}

This specific breakdown of \( A \) into scalar and vector potentials, and dual (pseudoscalar and bivector) potentials has been chosen to match existing SI conventions, specifically those of \citep{balanis2005antenna}.

\subsection{Electric sources.}

For a multivector current with only electric sources

\begin{dmath}\label{eqn:potentialSection:1880}
J = \eta \lr{ v \rho - \BJ },
\end{dmath}

we can construct a multivector potential with only scalar and vector grades
\begin{dmath}\label{eqn:potentialSection:1900}
A = - \phi + v \BA.
\end{dmath}

The resulting field is

\begin{dmath}\label{eqn:potentialSection:80}
F
=
\BE + I \eta \BH
=
\gpgrade{ \lr{ \spacegrad - \inv{v} \PD{t}{} }
\lr{
      - \phi
      + v \BA
}
}{1,2},
\end{dmath}

which expands to

\boxedEquation{eqn:potentialSection:2240}{
F =
-\spacegrad \phi
-\PD{t}{\BA}
+ v \spacegrad \wedge \BA.
}

The respective electric and magnetic fields can be extracted using a duality transformation for the bivector curl

\begin{dmath}\label{eqn:potentialSection:1920}
F
=
-\spacegrad \phi
-\PD{t}{\BA}
+ I v \spacegrad \cross \BA,
\end{dmath}

from which we can read off the field components

\begin{dmath}\label{eqn:potentialSection:100}
\begin{aligned}
\BE &= -\spacegrad \phi -\PD{t}{\BA} \\
\mu \BH &= \spacegrad \cross \BA.
\end{aligned}
\end{dmath}

Observe that the grade selection encodes the precise reciepe required to produce the desired combination of gradients, curls and time partials.

The potential representation of the field \cref{eqn:potentialSection:80} is only a solution if Maxwell's equation is also satisfied, or

\begin{dmath}\label{eqn:potentialSection:1960}
\lr{ \spacegrad^2 - \inv{v^2} \PDSq{t}{} } \lr{ -\phi + v\BA }
= \eta \lr{ v \rho - \BJ } +
\lr{ \spacegrad + \inv{v} \PD{t}{} } \gpgrade{ \lr{ \spacegrad - \inv{v} \PD{t}{} } \lr{ -\phi + v\BA } }{0,3}
= \eta \lr{ v \rho - \BJ } +
\lr{ \spacegrad + \inv{v} \PD{t}{} } \lr{ v \spacegrad \cdot \BA + \inv{v} \PD{t}{\phi} }.
\end{dmath}

Imposing a constraint on the potential grades

\begin{dmath}\label{eqn:potentialSection:2020}
\spacegrad \cdot \BA + \inv{v^2} \PD{t}{\phi} = 0,
\end{dmath}

the Lorenz gauge condition, is clearly an expediant way to simplify this relationship.  In particular,
in the frequency domain \( \PDi{t}{} \leftrightarrow j \omega = j k v \), this gauge choice allows the scalar potential to be entirely eliminated, since

\begin{dmath}\label{eqn:potentialSection:2040}
\phi = \frac{j v^2}{\omega} \spacegrad \cdot \BA.
\end{dmath}

so the multivector potential is completely determined by a single vector potential

\begin{dmath}\label{eqn:potentialSection:2060}
A =
-\frac{j v^2}{\omega} \spacegrad \cdot \BA + v \BA,
\end{dmath}

Maxwell's equation is reduced to a Helmholtz equation

\begin{dmath}\label{eqn:potentialSection:2080}
\lr{ \spacegrad^2 + k^2} A = J,
\end{dmath}

and the field is simply
\begin{dmath}\label{eqn:potentialSection:2100}
F = \lr{ \spacegrad - j k } A.
\end{dmath}

To compare this to conventional results, let's substitute \cref{eqn:potentialSection:2060} into \cref{eqn:potentialSection:2100},

\begin{dmath}\label{eqn:potentialSection:2300}
F
=
\lr{ \spacegrad - j k }
\lr{
-\frac{j v^2}{\omega} \spacegrad \cdot \BA + v \BA
}
=
-\frac{k v^2}{\omega} \spacegrad \cdot \BA -j k v \BA
-\frac{j v^2}{\omega} \spacegrad \lr{ \spacegrad \cdot \BA } + v \spacegrad \BA
=
- \cancel{ v \spacegrad \cdot \BA }
-j \omega \BA
-\frac{j v}{k} \spacegrad \lr{ \spacegrad \cdot \BA }
+ \cancel{ v \spacegrad \cdot \BA }
+ v \spacegrad \wedge \BA
=
-j \omega \BA -\frac{j v}{k} \spacegrad \lr{ \spacegrad \cdot \BA }
+ v \spacegrad \wedge \BA.
\end{dmath}

From this the electric and magnetic fields can be read off

\begin{dmath}\label{eqn:potentialSection:2320}
\begin{aligned}
\BE &= -j \omega \BA -\frac{j v}{k} \spacegrad \lr{ \spacegrad \cdot \BA } \\
\mu \BH &= \spacegrad \cross \BA.
\end{aligned}
\end{dmath}

\subsection{Magnetic sources.}

For a multivector current with only magnetic sources

\begin{dmath}\label{eqn:potentialSection:2140}
J = I \lr{ v \rho_m - \BM },
\end{dmath}

we can construct a multivector potential with only pseudoscalar and bivector grades
\begin{dmath}\label{eqn:potentialSection:2160}
A = \eta I\lr{ - \phi_m + v \BF}.
\end{dmath}

The resulting field is

\begin{dmath}\label{eqn:potentialSection:120}
F
=
\BE + I \eta \BH
=
\gpgrade{ \lr{ \spacegrad - \inv{v}\PD{t}{} }
\lr{
      - I \eta \phi_m
      + I \eta v \BF
}
}{1,2},
\end{dmath}

which simplifies to

\boxedEquation{eqn:potentialSection:2260}{
F
=
I \eta \lr{ v \spacegrad \wedge \BF
-\PD{t}{\BF}
- \spacegrad \phi_m
}.
}

The separate electric and magnetic field contributions can be read off from

\begin{dmath}\label{eqn:potentialSection:2280}
F
=
- \eta v \spacegrad \cross \BF
+ \eta I \lr
{
-\spacegrad \phi_m
- \PD{t}{\BF}
},
\end{dmath}

yielding

\begin{dmath}\label{eqn:potentialSection:140}
\begin{aligned}
\BE &= -\inv{\epsilon} \spacegrad \cross \BF \\
\BH &= -\spacegrad \phi_m
- \PD{t}{\BF}.
\end{aligned}
\end{dmath}

The potential representation of the field \cref{eqn:potentialSection:140} is only a solution if Maxwell's equation is also satisfied, or

\begin{dmath}\label{eqn:potentialSection:2120}
\lr{ \spacegrad^2 - \inv{v^2} \PDSq{t}{} }
\eta I \lr{ - \phi_m + v \BF}
=
I \lr{ v \rho_m - \BM }
+
\lr{ \spacegrad + \inv{v} \PD{t}{} } \gpgrade{ \lr{ \spacegrad - \inv{v} \PD{t}{} } \eta I \lr{ -\phi_m + v\BF } }{0,3}
=
I \lr{ v \rho_m - \BM }
+
\lr{ \spacegrad + \inv{v} \PD{t}{} }
\lr{
\frac{\eta I}{v} \PD{t}{\phi_m} + \eta v I \spacegrad \cdot \BF
}
\end{dmath}

Again, imposing a constraint on the potential grades

\begin{dmath}\label{eqn:potentialSection:2180}
\spacegrad \cdot \BF
+ \inv{v^2}
\PD{t}{\phi_m}
= 0,
\end{dmath}

the Lorenz gauge condition for the magnetic potentials, is clearly an expediant way to simplify this relationship.
As before, in the frequency domain the scalar potential can be entirely eliminated

\begin{dmath}\label{eqn:potentialSection:2200}
\phi_m = \frac{j v^2}{\omega} \spacegrad \cdot \BF.
\end{dmath}

In this case the
multivector potential is

\begin{dmath}\label{eqn:potentialSection:2220}
A =
\eta I \lr{
-\frac{j v^2}{\omega} \spacegrad \cdot \BF + v \BF
},
\end{dmath}

and Maxwell's equation and the field are given by
\cref{eqn:potentialSection:2080} and
\cref{eqn:potentialSection:2100} respectively.

Again, as a check, let's compute the electric and magnetic fields explicitly from \cref{eqn:potentialSection:2220}

\begin{dmath}\label{eqn:potentialSection:2340}
F
=
\lr{ \spacegrad - j k }
\eta I \lr{
-\frac{j v^2}{\omega} \spacegrad \cdot \BF + v \BF
}
=
\eta I
\lr{
- v \spacegrad \cdot \BF -j \omega \BF
-\frac{j v}{k} \spacegrad \lr{ \spacegrad \cdot \BF } + v \spacegrad \BF
}
=
\eta I
\lr{
-j \omega \BF
-\frac{j v}{k} \spacegrad \lr{ \spacegrad \cdot \BF } + v \spacegrad \wedge \BF
}
=
-\inv{\epsilon} \spacegrad \cross \BF
- j \eta I \lr{
\omega \BF + \frac{v}{k} \spacegrad \lr{ \spacegrad \cdot \BF }
},
\end{dmath}

so

\begin{dmath}\label{eqn:potentialSection:2360}
\begin{aligned}
\BE &= -\inv{\epsilon} \spacegrad \cross \BF  \\
\BH &= -j \omega \BF -j \frac{v}{k} \spacegrad \lr{ \spacegrad \cdot \BF }.
\end{aligned}
\end{dmath}

\subsection{Gauge transformations}

Because the potential representation of the field is expressed as a grade 1,2 selection, the addition of scalar or pseudoscalar components to the grade selection will not alter the field.  In particular, it is possible to alter the multivector potential

\begin{dmath}\label{eqn:potentialSection:160}
A \rightarrow A + \lr{ \spacegrad + \inv{v} \PD{t}{}} \psi,
\end{dmath}

where \( \psi \) is any multivector field with scalar and pseudoscalar grades, without changing the field

\begin{dmath}\label{eqn:potentialSection:180}
F
\rightarrow
\gpgrade{
   \lr{ \spacegrad - \inv{v} \PD{t}{} }
   \lr{ A + \lr{ \spacegrad + \inv{v} \PD{t}{}} \psi }
}{1,2}
=
F +
\gpgrade{
   \lr{ \spacegrad^2 - \inv{v^2} \PDSq{t}{}} \psi
}{1,2}
.
\end{dmath}

That last grade selection is zero, since \( \psi \) has no vector or bivector grades, demonstrating that the electromagnetic field is invariant with respect to this multivector potential transformation.

It is worth looking how such a transformation impacts each grade of the potential.  Let \( \psi = v \psi^\e + \eta v I \psi^\m \), where \( \psi^\e \) and \( \psi^\m \) are both scalar fields.  The gauge transformation provides the mapping

\begin{subequations}
\label{eqn:potentialSection:220}
\begin{dmath}\label{eqn:potentialSection:200}
- \phi \rightarrow - \phi + \PD{t}{} \psi^\e
\end{dmath}
\begin{dmath}\label{eqn:potentialSection:240}
v \BA \rightarrow v \BA + v \spacegrad \psi^\e
\end{dmath}
\begin{dmath}\label{eqn:potentialSection:260}
I v \BF \rightarrow I v \BF + I v \spacegrad \psi^\m
\end{dmath}
\begin{dmath}\label{eqn:potentialSection:280}
- I \eta \phi_m \rightarrow -I \eta \phi_m + I \eta \PD{t}{} \psi^\m,
\end{dmath}
\end{subequations}

or

\begin{subequations}
\label{eqn:potentialSection:400}
\begin{dmath}\label{eqn:potentialSection:420}
\phi \rightarrow \phi - \PD{t}{} \psi^\e
\end{dmath}
\begin{dmath}\label{eqn:potentialSection:440}
\BA \rightarrow \BA + \spacegrad \psi^\e
\end{dmath}
\begin{dmath}\label{eqn:potentialSection:460}
\BF \rightarrow \BF + \spacegrad \psi^\m
\end{dmath}
\begin{dmath}\label{eqn:potentialSection:480}
\phi_m \rightarrow \phi_m - \PD{t}{} \psi^\m.
\end{dmath}
\end{subequations}

These have the alternation of sign that is found in the usual recipe for gauge transformation of the scalar and vector potentials.  In conventional electromagnetism, the first two relations are usually found by observing it is possible to add any gradient to the vector potential, and then finding the transformation consequences that that choice imposes on the electric field.  With the grade selection formulation of the electromagnetic field, this special coupling of the field potentials comes for free without having to consider the curl of a specific field component.

Note that the latter two dual transformation relationships are for magnetic sources, and are usually expressed in the frequency domain, where the gauge transformations take the form

\begin{subequations}
\label{eqn:potentialSection:300}
\begin{dmath}\label{eqn:potentialSection:320}
\phi \rightarrow \phi - j \omega \psi^\e
\end{dmath}
\begin{dmath}\label{eqn:potentialSection:340}
\BA \rightarrow \BA + \spacegrad \psi^\e
\end{dmath}
\begin{dmath}\label{eqn:potentialSection:360}
\BF \rightarrow \BF + \spacegrad \psi^\m
\end{dmath}
\begin{dmath}\label{eqn:potentialSection:380}
\phi_m \rightarrow \phi_m -j \omega \psi^\m.
\end{dmath}
\end{subequations}

\subsection{Lorenz gauge}

With the flexibility to alter make a gauge transformation of the potential, it is useful to examine the conditions for which it is possible to express the electromagnetic field without any grade selection operation.  That is

\begin{dmath}\label{eqn:potentialSection:1720}
F
=
\lr{ \spacegrad - \inv{v} \PD{t}{} }
\lr{
      - \phi
      + v \BA
      + \eta I \lr{ -\phi_m + v \BF }
}.
\end{dmath}

There should be no a-priori assumption that such a field representation has no scalar, nor no pseudoscalar components, which can be seen by the explicit expansion in grades

\begin{dmath}\label{eqn:potentialSection:1640}
\begin{aligned}
F
&=
\lr{ \spacegrad - \inv{v} \PD{t}{} } A \\
&=
\lr{ \spacegrad - \inv{v} \PD{t}{} } \lr{ -\phi + v \BA + \eta I \lr{ -\phi_m + v \BF } } \\
&=
\inv{v} \partial_t \phi
+ v \spacegrad \cdot \BA  \\
&-\spacegrad \phi
+ I \eta v \spacegrad \wedge \BF
- \partial_t \BA  \\
&+ v \spacegrad \wedge \BA
- \eta I \spacegrad \phi_m
- I \eta \partial_t \BF \\
&+ \eta I \inv{v} \partial_t \phi_m
+ I \eta v \spacegrad \cdot \BF,
\end{aligned}
\end{dmath}

so if this potential representation has only vector and bivector grades, it must be true that

\begin{dmath}\label{eqn:potentialSection:1660}
\begin{aligned}
\inv{v} \partial_t \phi + v \spacegrad \cdot \BA &= 0 \\
\inv{v} \partial_t \phi_m + v \spacegrad \cdot \BF &= 0.
\end{aligned}
\end{dmath}

The first is the well known Lorenz gauge condition, whereas the second is the dual of that condition for magnetic sources.

Should one of these conditions, say the Lorenz condition for the electric source potentials, be non-zero, then it is possible to make a potential transformation for which this condition is zero

\begin{dmath}\label{eqn:potentialSection:1680}
0 \ne
\inv{v} \partial_t \phi + v \spacegrad \cdot \BA
=
\inv{v} \partial_t (\phi' - \partial_t \psi) + v \spacegrad \cdot (\BA' + \spacegrad \psi)
=
\inv{v} \partial_t \phi' + v \spacegrad \BA'
+ v \lr{ \spacegrad^2 - \inv{v^2} \partial_{tt} } \psi,
\end{dmath}

so if \( \inv{v} \partial_t \phi' + v \spacegrad \BA' \) is zero, \( \psi \) must be found such that
\begin{dmath}\label{eqn:potentialSection:1700}
\inv{v} \partial_t \phi + v \spacegrad \cdot \BA
= v \lr{ \spacegrad^2 - \inv{v^2} \partial_{tt} } \psi.
\end{dmath}

Such a gauge transformation requires a non-homogeneous wave equation solution, or equivalently in the frequency domain requires the solution of a Helmholtz equation

\begin{dmath}\label{eqn:potentialSection:1740}
\inv{v} j \omega \phi + v \spacegrad \cdot \BA
= v \lr{ \spacegrad^2 + k^2 } \psi.
\end{dmath}

A similar transformation is also clearly possible to eliminate any pseudoscalar grades in \cref{eqn:potentialSection:1720}.  Such a potential representation is desirable since
Maxwell's equations for such a potential are completely decoupled

\begin{dmath}\label{eqn:potentialSection:1760}
\lr{ \spacegrad^2 - \inv{v^2} \PDSq{t}{} } A = J,
\end{dmath}

which is equivalent to precisely one non-homogenious wave equation for each grade source and potential

\begin{dmath}\label{eqn:potentialSection:1600}
\begin{aligned}
\lr{ \spacegrad^2 - \inv{v^2} \PDSq{t}{} } \phi &= - \inv{\epsilon} \rho \\
\lr{ \spacegrad^2 - \inv{v^2} \PDSq{t}{} } \BA &= - \mu \BJ \\
\lr{ \spacegrad^2 - \inv{v^2} \PDSq{t}{} } \phi_m &= - \frac{I}{\mu} \rho_m \\
\lr{ \spacegrad^2 - \inv{v^2} \PDSq{t}{} } \BF &= - I \epsilon \BM,
\end{aligned}
\end{dmath}

or equivalently, in the frequency domain, a forced Helmholtz equation for each grade

\begin{dmath}\label{eqn:potentialSection:1780}
\begin{aligned}
\lr{ \spacegrad^2 + k^2 } \phi &= - \inv{\epsilon} \rho \\
\lr{ \spacegrad^2 + k^2 } \BA &= - \mu \BJ \\
\lr{ \spacegrad^2 + k^2 } \phi_m &= - \frac{1}{\mu} \rho_m \\
\lr{ \spacegrad^2 + k^2 } \BF &= - \epsilon \BM.
\end{aligned}
\end{dmath}

%}
