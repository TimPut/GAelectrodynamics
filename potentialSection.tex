%
% Copyright � 2017 Peeter Joot.  All Rights Reserved.
% Licenced as described in the file LICENSE under the root directory of this GIT repository.
%
%{
\index{potential}
\index{multivector potential}

For both electrostatics and magnetostatics, where Maxwell's equations are both a pair of gradients, we were able to require that the respective scalar and vector potentials were both gradients.
For electrodynamics where Maxwell's equation is
\begin{dmath}\label{eqn:potentialSection:1800}
\lr{ \spacegrad + \inv{c} \PD{t}{} } F = J,
\end{dmath}
it seems more reasonable to demand a different structure of the potential, say
\begin{dmath}\label{eqn:potentialSection:1820}
F = \lr{ \spacegrad - \inv{c} \PD{t}{} } A,
\end{dmath}
where \( A \) is a multivector potential that may contain all grades, with structure to be determined.
If such a multivector potential can be found, then Maxwell's equation is reduced to a single wave equation
\begin{dmath}\label{eqn:potentialSection:1840}
\lr{ \spacegrad^2 - \inv{c^2} \PDSq{t}{} } A = J,
\end{dmath}
which can be thought of as one wave equation for each multivector grade of the multivector source \( J \).

Some thought shows that the guess \cref{eqn:potentialSection:1820} is not quite right, as it allows for the invalid possibility that \( F \) has scalar or pseudoscalar grades.
While it is possible to impose constraints (a gauge choice) on potential \( A \) that ensure
\( F \) has only the vector and bivector grades,
in general,
a grade selection filter must be imposed
\boxedEquation{eqn:potentialSection:1860}{
F
=
\gpgrade{ \lr{ \spacegrad - \inv{c} \PD{t}{} } A }{1,2}.
}

We will find that the desired representation of the multivector potential is
\begin{dmath}\label{eqn:potentialSection:40}
A =
      - \phi
      + c \BA
      + \eta I \lr{ -\phi_m + c \BF }.
\end{dmath}

Here

\begin{enumerate}
\item \( \phi \) is the scalar potential \si{V} (Volts).
\item \( \BA \) is the vector potential \si{W/m} (Webers/meter).
\item \( \phi_m \) is the scalar potential for (fictitious) magnetic current sources \si{A} (Amperes).
\item \( \BF \) is the vector potential for (fictitious) magnetic current sources \si{C} (Coulombs).
\end{enumerate}

This specific breakdown of \( A \) into scalar and vector potentials, and dual (pseudoscalar and bivector) potentials has been chosen to match existing SI conventions, specifically those of \citep{balanis2005antenna}.

