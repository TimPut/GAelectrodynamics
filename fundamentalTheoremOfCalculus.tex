%
% Copyright � 2016 Peeter Joot.  All Rights Reserved.
% Licenced as described in the file LICENSE under the root directory of this GIT repository.
%
%{
%\input{../blogpost.tex}
%\renewcommand{\basename}{fundamentalTheoremOfCalculus}
%\renewcommand{\dirname}{notes/phy1520/}
%%\newcommand{\dateintitle}{}
%%\newcommand{\keywords}{}
%
%\input{../peeter_prologue_print2.tex}
%
%\usepackage{peeters_layout_exercise}
%\usepackage{peeters_braket}
%\usepackage{peeters_figures}
%\usepackage{siunitx}
%
%\beginArtNoToc
%
%\generatetitle{Fundamental theorem of geometric calculus}
%\label{chap:fundamentalTheoremOfCalculus}

\subsection{Hypervolume integral}
We wish to generalize the concepts of line, surface and volume integrals to hypervolumes and multivector functions, and define a hypervolume integral as

\makedefinition{Multivector integral.}{dfn:fundamentalTheoremOfCalculus:240}{
Given a hypervolume parameterized by \( k \) parameters, k-volume volume element \( d^k \Bx \), and
multivector functions \( F, G \), we define the k-volume integral as
\begin{equation*}
\int F d^k\Bx \lrboldpartial G
\equiv
\int \lr{ F d^k\Bx \lboldpartial} G
+
\int F d^k\Bx \lr{ \rboldpartial G }.
\end{equation*}
} % definition

Because multivectors may not commute with the vector derivative or the differential, we allow the vector derivative to act bidirectionally using the chain rule.
Here \( \rboldpartial, \lboldpartial,\lrboldpartial \) are used to indicate that the vector derivative is acting only to the right on \( G\), to the left on \( F\), or in both directions respectively.
The meaning of these can be made explicit by expansion in coordinates
\begin{dmath}\label{eqn:lineintegral:220}
F \lrboldpartial G
=
(\partial_i F) \sum_i \Bx^i G
+
F \sum_i \Bx^i (\partial_i G)
=
(F \lboldpartial) G
+
F (\rboldpartial G).
\end{dmath}
This bidirectional notation is also used for the gradient.
%The scope of the action of the vector derivative when acting only to the left or right is indicated using braces above.
Should we wish to only integrate single functions, we can set either of the other to \( 1 \), yielding integrals of the form
\( \int F d^k\Bx \lboldpartial, \) or \( \int d^k\Bx \boldpartial G \).

%The basic problem is that we have to indicate the scope of the partials of the derivative operator, but may not be able to commute the multivector components of the operator.
Some authors use overdots or ticks to indicate the exact scope of a multivector derivative operator.
FIXME: perhaps I should too?  That would emphasizes the fact that the action is on the functions \( F, G \), and not on the hypervolume element \( d^k \Bx \).

\subsection{Fundamental theorem.}
\index{fundamental theorem of geometric calculus}

The fundamental theorem of geometric calculus is a generalization of many conventional scalar and vector integral theorems.
It is a powerful theorem, which we will use with Green's functions to solve Maxwell's equation, and to derive the geometric algebra form of Stokes' theorem.

\maketheorem{Fundamental theorem of geometric calculus}{thm:fundamentalTheoremOfCalculus:1}{
For multivectors \(F, G \), and a hypervolume element \(d^k \Bx\),
\begin{equation*}
\int_V F d^k \Bx \boldpartial G = \oint_{\partial V} F d^{k-1} \Bx G.
\end{equation*}
}

This theorem relates the hypervolume integral to the integral over the bounding surface of hypervolume.
Additional work is required to describe the precise meaning of the boundary differential \( d^{k-1} \Bx \).  We will do so for line, surface, and volume integrals, proving the theorem in a limited fashion for each of those cases as we go.

For a full proof of \cref{thm:fundamentalTheoremOfCalculus:1}, additional mathematical sublties must be considered.
For full proofs and additional details, the reader is referred to \citep{hestenes1985clifford}, \citep{doran2003gap}, \citep{aMacdonaldVAGC} and \citep{sobczyk2011fundamental}, which all
which all tackle different aspects of general geometric calculus.

Before considering multivector line, surface and volume integral specializations of
\cref{thm:fundamentalTheoremOfCalculus:1},
we will state Stokes' theorem in its geometric algebra form.

%}
%\EndArticle
