%
% Copyright © 2018 Peeter Joot.  All Rights Reserved.
% Licenced as described in the file LICENSE under the root directory of this GIT repository.
%
%{

Given a static line charge density and current density along the z-axis
\begin{dmath}\label{eqn:statics_infiniteLineCharge:120}
\begin{aligned}
\rho(\Bx) &= \lambda \delta(x) \delta(y) \\
\BJ(\Bx) &= \Bv \rho(\Bx) = v \lambda \Be_3 \delta(x) \delta(y),
\end{aligned}
\end{dmath}
the total multivector current is
\begin{dmath}\label{eqn:statics_infiniteLineCharge:140}
J
= \eta ( c \rho - \BJ )
= \eta ( c - v \Be_3 ) \lambda \delta(x) \delta(y)
= \frac{\lambda}{\epsilon} \lr{ 1 - \frac{v}{c} \Be_3 } \delta(x) \delta(y).
\end{dmath}

We can find the field for this current using
%\cref{eqn:statics:80}.
\cref{thm:statics:100}.
To do so, let the field observation point be \( \Bx = \Bx_\perp + z \Be_3 \), so the total field is
\begin{dmath}\label{eqn:statics_infiniteLineCharge:160}
F(\Bx)
= \frac{\lambda}{4\pi \epsilon} \int_V dx'dy'dz' \frac{\gpgrade{(\Bx - \Bx') (1 - (v/c) \Be_3 )}{1,2}}{\Norm{\Bx - \Bx'}^3} \delta(x') \delta(y')
= \frac{\lambda}{4\pi \epsilon} \int_{-\infty}^\infty dz' \frac{\gpgrade{(\Bx_\perp + (z - z') \Be_3) (1 - (v/c) \Be_3 )}{1,2}}{\lr{\Bx_\perp^2 + (z-z')^2}^{3/2}}
=
\frac{\lambda \lr{ \Bx_\perp - (v/c) \Bx_\perp \Be_3}}{4\pi \epsilon} \int_{-\infty}^\infty \frac{dz'}{\lr{\Bx_\perp^2 + (z-z')^2}^{3/2}}
+
\frac{\lambda \Be_3}{4\pi \epsilon} \int_{-\infty}^\infty \frac{(z - z') dz'}{\lr{\Bx_\perp^2 + (z-z')^2}^{3/2}}.
\end{dmath}

The first integral is \( 2/\Bx_\perp^2 \), whereas the second is zero (odd function, over even interval).
The bivector term of the grade selection above had a \( \Bx_\perp \wedge \Be_3 = \Bx_\perp \Be_3 \) factor, which can be further reduced using cylindrical coordinates \( \Bx = R \rhocap + z \Be_3 \), since \( \Bx_\perp = R \rhocap \), which leaves
\begin{equation}\label{eqn:statics_infiniteLineCharge:180}
F(\Bx)
=
\frac{\lambda}{2\pi \epsilon R} \rhocap \lr{ 1 - \Bv/c} = \BE \lr{ 1 - \Bv/c }
= \BE + I \lr{ \frac{\Bv}{c} \cross \BE },
\end{equation}
where \( \Bv = v \Be_3 \).
The vector component of this is the electric field, which is therefore directed radially, whereas the (dual) magnetic field \( \eta I \BH \)
is a set of oriented planes spanning the radial and z-axis directions.
We can also see that there is a constant proportionality factor that relates the electric and magnetic field components, namely
\begin{dmath}\label{eqn:statics_infiniteLineCharge:200}
I \eta \BH = -\BE \Bv/c,
\end{dmath}
or
\begin{dmath}\label{eqn:statics_infiniteLineCharge:220}
\BH = \Bv \cross \BD.
\end{dmath}

\makeproblem{Linear magnetic density and currents.}{problem:statics:240}{
Given magnetic charge density \( \rho_m = \lambda_m \delta(x) \delta(y) \), and current density \( \BM = v \Be_3 \rho_m = \Bv \rho_m \), show that the field is given by
\begin{equation*}
F(\Bx) = \frac{\lambda_m c}{4 \pi R} I \rhocap \lr{ 1 - \frac{\Bv}{c} },
\end{equation*}
or with \( \BB = \lambda_m \rhocap/(4 \pi R) \),
\begin{equation*}
F = \BB \cross \Bv + c I \BB.
\end{equation*}
} % problem

%}
