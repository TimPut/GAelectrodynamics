%
% Copyright © 2017 Peeter Joot.  All Rights Reserved.
% Licenced as described in the file LICENSE under the root directory of this GIT repository.
%
The multiplication table for the \R{2} geometric algebra can be computed with relative ease.
Many of the interesting products involve \( i = \Be_1 \Be_2 \), the unit pseudoscalar.
Using \cref{eqn:multiplication:140} the imaginary nature of the pseudoscalar, mentioned early, can now be demonstrated explicitly

\begin{dmath}\label{eqn:SimpleProducts2:220}
   \lr{ \Be_1 \Be_2 }^2
   =
   (\Be_1 \Be_2)(\Be_1 \Be_2)
   =
   -(\Be_1 \Be_2)(\Be_2 \Be_1)
   =
   -\Be_1 (\Be_2^2 ) \Be_1
   =
   -\Be_1^2
   = -1.
\end{dmath}

\index{complex imaginary}
Like the (scalar) complex imaginary, this bivector also squares to \( -1 \).
The only non-trivial products left to fill in the \R{2} multiplication table are those of the unit vectors with \( i \), products that are order dependent

\begin{dmath}\label{eqn:SimpleProducts2:180}
\begin{aligned}
   \Be_1 i &= \Be_1 \lr{ \Be_1 \Be_2 } \\
           &= \lr{ \Be_1 \Be_1 } \Be_2 \\
           &= \Be_2 \\
   i \Be_1 &= \lr{ \Be_1 \Be_2 } \Be_1 \\
           &= \lr{ -\Be_2 \Be_1 } \Be_1 \\
           &= -\Be_2 \lr{ \Be_1 \Be_1 } \\
           &= -\Be_2 \\
   \Be_2 i &= \Be_2 \lr{ \Be_1 \Be_2 } \\
           &= \Be_2 \lr{ -\Be_2 \Be_1 } \\
           &= -\lr{ \Be_2 \Be_2 }\Be_1 \\
           &= -\Be_1 \\
   i \Be_2 &= \lr{ \Be_1 \Be_2 } \Be_2 \\
           &= \Be_1 \lr{ \Be_2 \Be_2 } \\
           &= \Be_1.
\end{aligned}
\end{dmath}

The multiplication table for the \R{2} multivector basis can now be tabulated

%FIXME: how to reference a tcolorbox table?
% examples in http://ctan.mirrors.hoobly.com/macros/latex/contrib/tcolorbox/tcolorbox.pdf section 5.1
% requires setting up a counter variable like some of the others (theorem environments)

% various options for prettier than default table:
% https://tex.stackexchange.com/a/135421/15
% https://tex.stackexchange.com/a/298109/15
% https://tex.stackexchange.com/a/112359/15
%\captionedTable{2D Multiplication table.}{tab:SimpleProducts2:10}{
%\begin{tabular}{|l||l|l|l|l|}
%\hline
%        & \( 1 \) & \( \Be_1 \) & \( \Be_2 \) & \( \Be_1 \Be_2 \) \\ \hline
%\( 1 \) & \( 1 \) & \( \Be_1 \) & \( \Be_2 \) & \( \Be_1 \Be_2 \) \\ \hline
%\( \Be_1\) & \( \Be_1 \) & \( 1 \) & \( \Be_1 \Be_2 \) & \( \Be_2 \)\\ \hline
%\( \Be_2\) & \( \Be_2 \) & \( -\Be_1 \Be_2 \) & \( 1 \) & \( -\Be_1 \)\\ \hline
%\( \Be_1 \Be_2\) & \( \Be_1 \Be_2 \) & \( -\Be_2 \) & \( \Be_1 \) & \( -1 \) \\ \hline
%\end{tabular}
%}

%\label{tab:SimpleProducts2:10}
\begin{tcolorbox}[tab2,tabularx={X||Y|Y|Y|Y},title=2D Multiplication table.,boxrule=0.5pt]
        & \( 1 \) & \( \Be_1 \) & \( \Be_2 \) & \( \Be_1 \Be_2 \) \\ \hline
\( 1 \) & \( 1 \) & \( \Be_1 \) & \( \Be_2 \) & \( \Be_1 \Be_2 \) \\ \hline
\( \Be_1\) & \( \Be_1 \) & \( 1 \) & \( \Be_1 \Be_2 \) & \( \Be_2 \)\\ \hline
\( \Be_2\) & \( \Be_2 \) & \( -\Be_1 \Be_2 \) & \( 1 \) & \( -\Be_1 \)\\ \hline
\( \Be_1 \Be_2\) & \( \Be_1 \Be_2 \) & \( -\Be_2 \) & \( \Be_1 \) & \( -1 \) \\ \hline
\end{tcolorbox}

\index{pseudoscalar}
It is important to point out that the
pseudoscalar \( i \) does not commute with either basis vector, but anticommutes with both, since \( i \Be_1 = - \Be_1 i \), and \( i \Be_2 = - \Be_2 i \).
By superposition \( i \) anticommutes with any vector in the x-y plane.

More generally, if \( \ucap \) and \( \vcap \) are orthonormal, and \( \Bx \in \Span\setlr{\ucap, \vcap} \) then the bivector \( \ucap \vcap \) anticommutes with \( \Bx \), or any other vector in this plane.

%\ref{tab:SimpleProducts2:10}.

%Complex numbers may be represented as multivectors of the form \( z = x + i y \) where \( x, y \) are real numbers (scalars) and \( i \) is any multivector that squares to \( -1 \).  \Cref{eqn:SimpleProducts2:220} showed that \( i = \Be_{12} \) was one such imaginary, but others, such as \( i \in \setlr{ \Be_{23}, \Be_{31}, \Be_{123}} \) will also work, provided only one such imaginary basis element is allowed.
\makeproblem{Quaternions.}{problem:2dMultiplication:quaternions}{
The quaternion algebra is the set \( q \in \setlr{ a + x \Bi + y \Bj + z \Bk \mid a, x, y, z \in \bbR} \) where
\begin{equation}\label{eqn:2dMultiplication:240}
\begin{aligned}
\Bi^2 &= \Bj^2 = \Bk^2 = -1 \\
\Bi \Bj &= \Bk = -\Bj \Bi \\
\Bj \Bk &= \Bi = -\Bk \Bj \\
\Bk \Bi &= \Bj = -\Bi \Bk.
\end{aligned}
\end{equation}
Show that the relations \cref{eqn:2dMultiplication:240} are satisfied by the unit bivectors \( \Bi = \Be_{32}, \Bj = \Be_{13}, \Bk = \Be_{21} \), demonstrating that quaternions, like complex numbers, may be represented as multivector subspaces.
} % problem
