%
% Copyright © 2016 Peeter Joot.  All Rights Reserved.
% Licenced as described in the file LICENSE under the root directory of this GIT repository.
%

\index{complex imaginary}
\index{pseudoscalar}

FIXME: introduce reverse here, and simplfy the squaring steps.  coincidence reference is now redundant.

As with the use of the symbol \( i \) for the \R{2} pseudoscalar, it is not a coincidence that \( I \) was used 
for the 
\R{3} pseudoscalar.  It is also true that 
\( I = \Be_1 \Be_2 \Be_3 \) behaves like a complex imaginary with \( I^2 = -1 \).  This follows 
directly from repeated anticommutation

\begin{dmath}\label{eqn:projectionAndRejection:1140}
I^2
=
(\Be_1 \Be_2 \Be_3)(\Be_1 \Be_2 \Be_3)
=
\Be_1 \Be_2 (\Be_3 \Be_1) \Be_2 \Be_3
=
\Be_1 \Be_2 (-\Be_1 \Be_3) \Be_2 \Be_3
=
-\Be_1 \Be_2 \Be_1 (\Be_3 \Be_2) \Be_3
=
-\Be_1 \Be_2 \Be_1 (-\Be_2 \Be_3) \Be_3
=
+\Be_1 \Be_2 \Be_1 \Be_2 (\Be_3 \Be_3)
=
(\Be_1 \Be_2)^2
=
-1.
\end{dmath}

%\makeproblem{\R{3} pseudoscalar square}{problem:gradeselection:R3PseudoscalarSquare}{
%With the \R{3} pseudoscalar of \cref{eqn:definitions:340} show that \( I^2 = -1 \).
%} % problem
%
%\makeanswer{problem:gradeselection:R3PseudoscalarSquare}{
%
%\begin{dmath}\label{eqn:gaTutorial:160}
%I^2
%=
%...
%=
%-1,
%\end{dmath}
%
%as expected, showing that this quantity also has characteristics of an imaginary number.
%} % answer
