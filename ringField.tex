%
% Copyright © 2017 Peeter Joot.  All Rights Reserved.
% Licenced as described in the file LICENSE under the root directory of this GIT repository.
%
%{

\subsection{Example.  Field of a ring of charge or current density.}

Let's now compute the field due to a static charge or current density on a ring of radius \( R \) as illustrated in
\cref{fig:chargeAndCurrentOnRing:chargeAndCurrentOnRingFig1}.

\imageFigure{../figures/GAelectrodynamics/chargeAndCurrentOnRingFig1}{Field due to a circular distribution.}{fig:chargeAndCurrentOnRing:chargeAndCurrentOnRingFig1}{0.3}

A static charge distribution on a ring at \( z = 0 \) has the form

\begin{dmath}\label{eqn:ringField:20}
\rho(\Bx) = \lambda \delta(z) \delta(r - R).
\end{dmath}

As always the current distribution is of the form \( \BJ = \Bv \rho \), and in this case the velocity is azimuthal \( \Bv = \Be_2 e^{i\phi}, i = \Be_{12} \).
The total multivector current is

\begin{dmath}\label{eqn:ringField:40}
J = \inv{\epsilon_0} \lambda \delta(z) \delta(r - R) \lr{ 1 - \frac{\Bv}{c} }.
\end{dmath}

Let the point that we observe the field, and the integration variables be

\begin{dmath}\label{eqn:ringField:60}
\begin{aligned}
\Bx &= z \Be_3 + r \rhocap \\
\Bx' &= z' \Be_3 + r' \rhocap'.
\end{aligned}
\end{dmath}

The field is

\begin{dmath}\label{eqn:ringField:80}
F(\Bx)
= \frac{\lambda}{4 \pi \epsilon_0} \iiint dz' r' dr' d\phi' \delta(z') \delta(r' - R) \frac{\gpgrade{ \lr{ (z - z') \Be_3 + r \rhocap - r' \rhocap' } \lr{ 1 - \frac{v}{c} \Be_2 e^{i\phi'} } }{1,2} } { \lr{ (z-z')^2 + (r \rhocap - r' \rhocap')^2}^{3/2} }
= \frac{\lambda}{4 \pi \epsilon_0} \int R d\phi' \frac{\gpgrade{ \lr{ z \Be_3 + r \rhocap - R \rhocap' } \lr{ 1 - \frac{v}{c} \Be_2 e^{i\phi'} } }{1,2} } { \lr{ z^2 + (r \rhocap - R \rhocap')^2}^{3/2} }
\end{dmath}

Without loss of generality, we can align the axes so that \( \rhocap = \Be_1 \), and
introduce dimensionless variables

\begin{dmath}\label{eqn:ringField:100}
\begin{aligned}
\alpha &= z/R \\
\beta &= r/R,
\end{aligned}
\end{dmath}

which gives
\begin{dmath}\label{eqn:ringField:120}
F
= \frac{\lambda}{4 \pi \epsilon_0 R} \int_0^{2\pi} d\phi' \frac{\gpgrade{ \lr{ \alpha \Be_3 + \beta \Be_1 - \Be_1 e^{i\phi'} } \lr{ 1 - \frac{v}{c} \Be_2 e^{i\phi'} } }{1,2} } { \lr{ \alpha^2 + (\beta \Be_1 - \Be_1 e^{i\phi'})^2}^{3/2} }.
\end{dmath}

In the denominator, the vector square expands as
\begin{dmath}\label{eqn:ringField:140}
(\beta \Be_1 - \Be_1 e^{i\phi'})^2
=
(\beta - e^{-i\phi'}) \Be_1^2 (\beta - e^{i\phi'})
=
\beta^2 + 1 - 2 \beta \cos(\phi'),
\end{dmath}

and the grade selection in the numerator is

\begin{dmath}\label{eqn:ringField:160}
\gpgrade{ \lr{ \alpha \Be_3 + \beta \Be_1 - \Be_1 e^{i\phi'} } \lr{ 1 - \frac{v}{c} \Be_2 e^{i\phi'}}}{1,2}
=
\alpha \Be_3 + \beta \Be_1 - \Be_1 e^{i\phi'}
-\frac{v}{c}\lr{ \alpha \Be_{31} e^{i\phi'} + \beta i \cos(\phi') + i }.
\end{dmath}

Any of the exponential integrals terms
are of the form

\begin{dmath}\label{eqn:ringField:180}
\int_0^{2\pi} d\phi' e^{i\phi'} f(\cos(\phi')) = \int_0^{2\pi} d\phi' \cos(\phi') f(\cos(\phi')),
\end{dmath}

since
the \( i \sin\phi' f(\cos(\phi') \) contributions are odd functions around \( \phi' = \pi \).

In general, for all \( z, r \) we can't symbolically evaluate the integrals.
Let
\begin{subequations}
\label{eqn:ringField:260}
\begin{equation}\label{eqn:ringField:200}
A = A(\alpha, \beta) = \int_0^{2\pi} d\phi' \frac{1}{\lr{ 1 + \alpha^2 + \beta^2 - 2 \beta \cos(\phi') }^{3/2}}
\end{equation}
\begin{equation}\label{eqn:ringField:280}
B = B(\alpha, \beta) = \int_0^{2\pi} d\phi' \frac{\cos(\phi')}{\lr{ 1 + \alpha^2 + \beta^2 - 2 \beta \cos(\phi') }^{3/2}}.
\end{equation}
\end{subequations}

The solutions of \cref{eqn:ringField:260} both require elliptic integrals, but are also amenable to direct numeric evaluation.  Taking these as given, the field is

\begin{dmath}\label{eqn:ringField:220}
F
=
\frac{\lambda}{4 \pi \epsilon_0 R}
\lr{
\lr{ \alpha \Be_3 + \beta \Be_1 -\frac{v }{c}\Be_{12} } A
- \lr{
\Be_1 + \frac{v}{c} \lr{ \alpha \Be_{31} + \beta \Be_{12} } } B
}.
\end{dmath}

The field directions are nicely parameterized as multivector expresssions, with the relative weightings in different directions scaled by the position dependent integral coefficients of \cref{eqn:ringField:260}.
The multivector field can be separated into its respective electric and magnetic components by inspection

\begin{dmath}\label{eqn:ringField:240}
\begin{aligned}
\BE &=
\gpgradeone{F}
=
\frac{\lambda}{4 \pi R \epsilon_0} \lr{ \alpha A \Be_3 + \Be_1( \beta A - B) } \\
\BH &=
\inv{\eta_0} \gpgradeone{-I F}
=
\frac{\lambda v}{4 \pi R } \lr{ -\Be_3 \lr{ A + \beta B } - \Be_2 \alpha },
\end{aligned}
\end{dmath}

which shows that the static charge distribution \( \rho \propto \lambda \) only contributes to the electric field, and the static current distribution \( \BJ \propto v \lambda \) only contributes to the magnetic field.

Finally, to restore generality,
should we wish to compute the field or its electric or magnetic components at an arbitrary observation azimuthal angle \( \phi \), we need only transform \( \Be_1 \rightarrow \Be_1 e^{i \phi} = \rhocap, \Be_2 \rightarrow \Be_2 e^{i\phi} = \phicap \).

%}
