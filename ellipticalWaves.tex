%
% Copyright � 2017 Peeter Joot.  All Rights Reserved.
% Licenced as described in the file LICENSE under the root directory of this GIT repository.
%
%{
\input{../latex/blogpost.tex}
\renewcommand{\basename}{ellipticalWaves}
%\renewcommand{\dirname}{notes/phy1520/}
\renewcommand{\dirname}{notes/ece1228-electromagnetic-theory/}
%\newcommand{\dateintitle}{}
%\newcommand{\keywords}{}

\input{../latex/peeter_prologue_print2.tex}

\usepackage{peeters_layout_exercise}
\usepackage{peeters_braket}
\usepackage{peeters_figures}
\usepackage{siunitx}
%\usepackage{mhchem} % \ce{}
\usepackage{macros_bm} % \bcM
%\usepackage{macros_qed} % \qedmarker
%\usepackage{txfonts} % \ointclockwise

\beginArtNoToc

\generatetitle{Plane, circular, and elliptical waves in Geometric Algebra.}
%\chapter{Elliptical waves}
%\label{chap:ellipticalWaves}
% \citep{sakurai2014modern} pr X.Y
% \citep{pozar2009microwave}
% \citep{qftLectureNotes}
% \citep{doran2003gap}
% \citep{jackson1975cew}
% \citep{griffiths1999introduction}

\paragraph{Real Phasor representation.}

A real time dependent field, represented in terms of a complex vector valued phasor \( \tilde{\BA} \), is formed by taking the real part of the product of that phasor with the phase exponential

\begin{dmath}\label{eqn:ellipticalWaves:20}
\bcA
= \Real\lr{ \tilde{\BA} e^{j \Bk \cdot \Bx -j \omega t} }
=
\BA_r \cos\lr{ \Bk \cdot \Bx - \omega t }
- \BA_i \sin\lr{ \Bk \cdot \Bx - \omega t }.
\end{dmath}

In the complex representation above, the imaginary \( j \) is not interpreted geometrically, but like the unit pseudoscalar \( I = \Be_1 \Be_2 \Be_3 \), squares to \( -1 \) and commutes with all grades.  It is therefore possible to express the field using the pseudoscalar as the imaginary.  With \( \phi = \Bk \cdot \Bx - \omega t \), that is

\begin{dmath}\label{eqn:ellipticalWaves:40}
\bcA
=
\inv{2} \BA_r
\lr{ e^{I \phi} + e^{-I\phi} }
   - \inv{2 I} \BA_i
   \lr{ e^{I \phi} - e^{-I\phi} }
=
\inv{2}\lr{ \BA_r + I \BA_i } e^{I \phi}
+
\inv{2}\lr{ \BA_r - I \BA_i } e^{-I \phi}
=
\inv{2} \lr{ \BA e^{I \phi} + \lr{ \BA e^{I \phi} }^\dagger }
,
\end{dmath}

where the field magnitude and orientation has been specified by a ``complex'' (grade-1,3) multivector

\begin{dmath}\label{eqn:ellipticalWaves:120}
\BA = \BA_r + I \BA_i,
\end{dmath}

and its reverse \( \BA^\dagger \).
This has the structure of a real-part operation, where the real part is represented by half the multivector plus its reverse.  This is in fact one way of expressing the vector grade selection operation for the grade-1,3 multivector \( \BA e^{I\phi} \), which can also be considered a phasor representation

\begin{dmath}\label{eqn:ellipticalWaves:80}
\BA e^{I \phi}
=
\lr{ \BA_r + I \BA_i }
\lr{ \cos\phi + I \sin\phi }
=
\lr{ \BA_r + I \BA_i }
\lr{ \cos\phi + I \sin\phi }
=
\BA_r \cos\phi - \BA_i \sin\phi
+ I \BA_i \cos\phi + I \BA_r \sin\phi.
\end{dmath}

Adding this to its reverse (which negates the sign of the pseudoscalar, but not the vector), eliminates all the bivector components of this multivector phasor representation.  It is now possible to represent the field completely in terms of real vectors and a vector grade selection operation

\begin{dmath}\label{eqn:ellipticalWaves:100}
   \bcA = \gpgradeone{ \BA e^{I \lr{ \omega t - \Bk \cdot \Bx }}}.
\end{dmath}

\paragraph{Electromagnetic plane wave.}

Recall that the electromagnetic field, with \( \BE = \BE_r + j \BE_i \), for a plane wave is

\begin{dmath}\label{eqn:ellipticalWaves:140}
   F = \Real \lr{ \lr{ 1 + \kcap } \BE e^{j \phi} },
\end{dmath}

so the real representation, with multivector phasor \( \BE = \BE_r + I \BE_i \), is

\begin{dmath}\label{eqn:ellipticalWaves:160}
F
=
\lr{ 1 + \kcap } \gpgradeone{ \BE e^{I \phi } }
=
\inv{2} \lr{ 1 + \kcap } \lr{ \BE e^{I \phi } + \BE^\dagger e^{-I \phi } }.
\end{dmath}

Note that this is not equal to \(
\inv{2} \lr{
   \lr{ 1 + \kcap } \BE e^{I \phi }
   +
\lr{ \lr{ 1 + \kcap } \BE e^{I \phi } }^\dagger } \), since \( \lr{ \kcap \BE }^\dagger = -\kcap \BE^\dagger \).
% because \( \kcap \) is normal to both the \( \BE_r \) and \( \BE_i \) vectors.

Should the electric and magnetic fields be desired explictly, they can be obtained by the grade selection, with

\begin{dmath}\label{eqn:ellipticalWaves:220}
F =
\gpgradeone{ \BE e^{I \phi } } +
\gpgradetwo{ \kcap \BE e^{I \phi } },
\end{dmath}

where this split into electric (vector) and magnetic (bivector) field components was facilitated by
the fact that
\( \kcap \gpgradeone{ \BE e^{I \phi } } = \gpgradetwo{ \kcap \BE e^{I \phi } } \) [exersize].

\makeproblem{}{problem:ellipticalWaves:1}{
Given \( \BE = \BE_r + I \BE_i \), and \( \kcap \cdot \BE_r = \kcap \cdot \BE_i = 0 \), show that
\( \kcap \gpgradeone{ \BE e^{I \phi } } = \gpgradetwo{ \kcap \BE e^{I \phi } } \).
Also show that \( \gpgradetwo{ \kcap \BE e^{I \phi } } \) can be expanded as an antisymmetric sum of the multivector \( \kcap \BE e^{I\phi} \) and its reverse.
} % problem

\makeanswer{problem:ellipticalWaves:1}{
\begin{dmath}\label{eqn:ellipticalWaves:180}
\gpgradetwo{ \kcap \BE e^{I \phi } }
=
\gpgradetwo{ \kcap \lr{ \BE_r + I \BE_i} e^{I \phi } }
=
\gpgradetwo{ \kcap \lr{ \BE_r \cos\phi - \BE_i \sin\phi + I \BE_i \cos\phi + I \BE_r \sin\phi } }
=
\kcap \wedge \BE_r \cos\phi - \kcap \wedge \BE_i \sin\phi
=
\kcap \BE_r \cos\phi - \kcap \BE_i \sin\phi
=
\kcap \lr{ \BE_r \cos\phi - \kcap \BE_i \sin\phi }
=
\kcap \gpgradeone{ \BE e^{I\phi} }.
\end{dmath}

For the second part, we have

\begin{dmath}\label{eqn:ellipticalWaves:200}
\inv{2} \lr{ \kcap \BE e^{I \phi } - \lr{ \kcap \BE e^{I \phi } }^\dagger }
=
\inv{2} \lr{ \kcap \BE e^{I \phi } - e^{-I \phi } \BE^\dagger \kcap }
=
\inv{2} \lr{ \kcap \BE e^{I \phi } + e^{-I \phi } \kcap \BE^\dagger }
=
\frac{\kcap}{2} \lr{ \BE e^{I \phi } + \BE^\dagger e^{-I \phi } }
=
\frac{\kcap}{2} \lr{ \lr{\BE_r + I \BE_i} \lr{ \cos\phi + I \sin\phi } + \lr{\BE_r - I \BE_i} \lr{ \cos\phi - I \sin\phi } }
=
\frac{\kcap}{2} \lr{
   \BE_r \cos\phi - \BE_i \sin\phi + I \lr{ \BE_i \cos\phi + \BE_r \sin\phi }
+  \BE_r \cos\phi - \BE_i \sin\phi - I \lr{ \BE_i \cos\phi + \BE_r \sin\phi }
}
=
\kcap \lr{ \BE_r \cos\phi - \BE_i \sin\phi }
=
\kcap \gpgradeone{ \BE e^{I \phi} }
=
\gpgradetwo{ \kcap \BE e^{I \phi} }
.
\end{dmath}

} % answer

%}
%\EndArticle
\EndNoBibArticle
