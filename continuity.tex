%
% Copyright � 2018 Peeter Joot.  All Rights Reserved.
% Licenced as described in the file LICENSE under the root directory of this GIT repository.
%
%{
FIXME: integrate this and the wave equation stuff just previous.

As the multivector solutions \( F \) to Maxwell's equation \cref{dfn:isotropicMaxwells:680}
has only grades 1 and 2, this imposes constraints on the components of the multivctor current \( J \).

This can be seen by pre-multiplying Maxwell's equation by \( \spacegrad - (1/c) \partial_t \), forming the wave equation that including sources
\begin{dmath}\label{eqn:continuity:20}
\Biglr{ \spacegrad^2 - \inv{c^2} \partial_{tt} } F =
\Biglr{ \spacegrad - \inv{c} \partial_{t} }
\biglr{
   \eta \lr{ c \rho - \BJ } + I \lr{ c \rho_\txtm - \BM }
}.
\end{dmath}

The LHS has only grades one and two, which means that all grades zero and three components must be zero.  That zero equality for the grade 0 component is
\begin{dmath}\label{eqn:continuity:40}
0
=
\gpgradezero{
\Biglr{ \spacegrad - \inv{c} \partial_{t} }
\biglr{
   \eta \lr{ c \rho - \BJ } + I \lr{ c \rho_\txtm - \BM }
}
}
=
-\eta
\biglr{ \spacegrad \cdot \BJ + \PD{t}{\rho} }
,
\end{dmath}
so
\boxedEquation{eqn:continuity:70}{
\spacegrad \cdot \BJ + \PD{t}{\rho} = 0.
}

Similarly, the grade three zero equality means
\begin{dmath}\label{eqn:continuity:60}
0
=
\gpgradethree{
\Biglr{ \spacegrad - \inv{c} \partial_{t} }
\biglr{
   \eta \lr{ c \rho - \BJ } + I \lr{ c \rho_\txtm - \BM }
}
}
=
-I \biglr{
   \spacegrad \cdot \BM + \PD{t}{\rho_\txtm}
},
\end{dmath}
so
\boxedEquation{eqn:continuity:90}{
\spacegrad \cdot \BM + \PD{t}{\rho_\txtm} = 0.
}

%}
