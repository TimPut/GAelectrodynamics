%
% Copyright � 2018 Peeter Joot.  All Rights Reserved.
% Licenced as described in the file LICENSE under the root directory of this GIT repository.
%
%{
As the multivector solutions \( F \) to Maxwell's equation \cref{eqn:maxwellsEquations:460}
has only grades 1 and 2, this imposes constraints on the components of the multivctor current \( J \).

This can be seen by pre-multiplying Maxwell's equation by \( \spacegrad - (1/v) \partial_t \), forming the wave equation that including sources

\begin{dmath}\label{eqn:continuity:20}
\lr{ \spacegrad^2 - \inv{v^2} \partial_{tt} } F =
\lr{ \spacegrad - \inv{v} \partial_{t} }
\lr{
   \eta \lr{ v \rho - \BJ } + I \lr{ v \rho_\txtm - \BM }
}.
\end{dmath}

The LHS has only grades one and two, which means that all grades zero and three components must be zero.  That zero equality for the grade 0 component is

\begin{dmath}\label{eqn:continuity:40}
0
=
\gpgradezero{
\lr{ \spacegrad - \inv{v} \partial_{t} }
\biglr{
   \eta \lr{ v \rho - \BJ } + I \lr{ v \rho_\txtm - \BM }
}
}
=
-\eta 
\lr{ \spacegrad \cdot \BJ + \PD{t}{\rho} }
,
\end{dmath}

so
\boxedEquation{eqn:continuity:70}{
\spacegrad \cdot \BJ + \PD{t}{\rho} = 0.
}

Similarly, the grade three zero equality means

\begin{dmath}\label{eqn:continuity:60}
0
=
\gpgradethree{
\biglr{ \spacegrad - \inv{v} \partial_{t} }
\lr{
   \eta \lr{ v \rho - \BJ } + I \lr{ v \rho_\txtm - \BM }
}
}
=
-I \lr{
   \spacegrad \cdot \BM + \PD{t}{\rho_\txtm} 
},
\end{dmath}

so

\boxedEquation{eqn:continuity:90}{
\spacegrad \cdot \BM + \PD{t}{\rho_\txtm} = 0.
}

%}
