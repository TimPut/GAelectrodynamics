%
% Copyright � 2018 Peeter Joot.  All Rights Reserved.
% Licenced as described in the file LICENSE under the root directory of this GIT repository.
%
%{
Some would argue that the conventional form \cref{eqn:freespace:3100} of Maxwell's equations have built in redundancy since continuity equations on the charge and current densities couple some of these equations.  We will take an opposing view, and show that such continuity equations are neccessary consequences of Maxwell's equation, and derive those conditions.  This amounts to a statement that the
multivector current \( J \) is not completely unconstrained.

To show this, we operate on \cref{dfn:isotropicMaxwells:680} with \( \spacegrad - (1/c) \partial_t \), one of the factors, along with the
spacetime gradient, of the
d'Alembertian (wave equation) operator, which gives
\begin{dmath}\label{eqn:continuity:110}
\lr{ \spacegrad^2 - \inv{c^2} \PDSq{t}{} } F =
\lr{ \spacegrad - \inv{c} \PD{t}{} } J
=
\gpgrade{\lr{ \spacegrad - \inv{c} \PD{t}{} } J}{1,2},
\end{dmath}
where a no-op grade 1,2 selection operation has been introduced since the grades of the left and right hand sides must match.

There is, however, information in the requirement that any grade 0,3 terms in the right hand side must be zero.
In particular, grade 0 selection gives
\begin{dmath}\label{eqn:continuity:40}
0
=
\gpgradezero{ ( \spacegrad - (1/c) \partial_t ) J }
=
\gpgradezero{
\Biglr{ \spacegrad - \inv{c} \PD{t}{} }
\biglr{
   \eta \lr{ c \rho - \BJ } + I \lr{ c \rho_\txtm - \BM }
}
}
=
-\eta
\lr{ \spacegrad \cdot \BJ + \PD{t}{\rho} }
,
\end{dmath}
so
\boxedEquation{eqn:continuity:70}{
\spacegrad \cdot \BJ + \PD{t}{\rho} = 0.
}
Similarly, grade three selection gives
\begin{dmath}\label{eqn:continuity:60}
0
=
\gpgradethree{  (\spacegrad - (1/c) \partial_t ) J }
=
\gpgradethree{
\Biglr{ \spacegrad - \inv{c} \PD{t}{} }
\lr{
   \eta \lr{ c \rho - \BJ } + I \lr{ c \rho_\txtm - \BM }
}
}
=
-I \lr{
   \spacegrad \cdot \BM + \PD{t}{\rho_\txtm}
},
\end{dmath}
so
\boxedEquation{eqn:continuity:90}{
\spacegrad \cdot \BM + \PD{t}{\rho_\txtm} = 0.
}

\index{wave equation}
%Having assembled all of Maxwell's equations into \cref{dfn:isotropicMaxwells:680}, some results now follow almost trivially.
%One such result is the wave equation in space free of sources.
%In such a region, Maxwell's equation is just
%\begin{dmath}\label{eqn:waveequation:480}
%\lr{ \spacegrad + \inv{c} \PD{t}{} } F = 0.
%\end{dmath}
%
%This can be multiplied from the left with the multivector operator \( \spacegrad - \inv{c} \PD{t}{} \), to give
%\begin{dmath}\label{eqn:waveequation:500}
%0 =
%\lr{ \spacegrad - \inv{c} \PD{t}{} }
%\lr{ \spacegrad + \inv{c} \PD{t}{} } F
%=
%\lr{ \spacegrad^2 - \inv{c^2} \PDSq{t}{} } F,
%\end{dmath}
%or
In source free conditions \cref{eqn:continuity:110} reduces to an homogeneous wave equation
%\begin{dmath}\label{eqn:waveequation:520}
\boxedEquation{eqn:continuity:520}{
\spacegrad^2 F = \inv{c^2} \PDSq{t}{F}.
}
%\end{dmath}

The solutions of the wave equation are well known, allowing us to immediately state that the solution is
\begin{dmath}\label{eqn:continuity:560}
F(\Bx, t) = f(\Norm{\Bx} \pm c t),
\end{dmath}
where \( f \) is any grade 1,2 multivector, provided that \( F \) also satisfies the constraints imposed by
Maxwell's equation \cref{dfn:isotropicMaxwells:680}.

In conventional electromagnetism, we have independent wave equations for
each of the electric and magnetic fields.
We can obtain those by applying vector and bivector grade selection operations to
\cref{eqn:continuity:520} to find
\begin{dmath}\label{eqn:waveequation:540}
\begin{aligned}
\spacegrad^2 \BE &= \inv{c^2} \PDSq{t}{\BE} \\
\spacegrad^2 (I \BH) &= \inv{c^2} \PDSq{t}{(I \BH)}.
\end{aligned}
\end{dmath}
The pseudoscalar factors in the magnetic field wave equation can be eliminated by multiplying with
\( -I \), which yields the conventional pair of vector wave equations for the electric and magnetic fields.  Any solutions
to
\cref{eqn:waveequation:540} are also constrained by Maxwell's equation.
%}
