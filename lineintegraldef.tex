%
% Copyright � 2018 Peeter Joot.  All Rights Reserved.
% Licenced as described in the file LICENSE under the root directory of this GIT repository.
%
%{
\index{differential form}
In geometric algebra, the integrand of a multivector line integral contains product of multivector(s) and a single parameter differential
\makedefinition{Multivector line integral.}{dfn:lineintegraldef:multivectorlineintegral}{
Given a continuous and differentiable curve described by a vector function \( \Bx(a) \), parameterized by single value \( a \) with differential
\begin{equation*}
d^1 \Bx \equiv d\Bx_a = \PD{a}{\Bx} da = \Bx_a da,
\end{equation*}
and multivector functions \( F, G \), the integral
\begin{equation*}
\int F d^1 \Bx G
\end{equation*}
is called a multivector line integral.
} % definition

An illustration of a single parameter curve and its
differential with respect to that parameter, is given in
\cref{fig:oneParameterDifferential:oneParameterDifferentialFig1}.
Observe that the differential is tangent to the curve at all points.
Possible physical realizations of the parameter describing the curve include
time, arclength, or angle.

\imageFigure{../figures/GAelectrodynamics/oneParameterDifferentialFig1}{One parameter manifold.}{fig:oneParameterDifferential:oneParameterDifferentialFig1}{0.2}

Suppose that \( \Bf(\Bx(a)) \) is a vector valued function defined along the curve.
The conventional line integral from vector calculus, a dot product of a differential and the function \( \Bf \)
may be obtained by the sum of two multivector line integrals one with \( F,G = \Bf/2,1 \), and the other with \( F,G = 1,\Bf/2 \)
\begin{dmath}\label{eqn:lineintegraldef:20}
\int d \Bx \frac{\Bf}{2}
+\int
\frac{\Bf}{2}
d \Bx
=
\int d \Bx \cdot \Bf.
\end{dmath}
Unlike the conventional dot product line integral, the multivector line integral of a vector function such as \( \int d \Bx \Bf \) is generally multivector valued, with both a scalar and a bivector component.  Let's consider some examples of multivector line integrals.

\paragraph{Example: Circular path.}
Let \( f(t) = \Ba t + \Bb t^2 \), where \( \Ba, \Bb \) are constant vectors, \( t \) is a scalar parameter, and the integration path is circular \( \Bx(t) = \Be_1 e^{i t} \), where \( i = \Be_1 \Be_2 \).
The line integral of \( \Bf\, d\Bx \) is
\begin{dmath}\label{eqn:lineintegraldef:40}
\int \Bf(t)\, d\Bx
=
\Ba \Be_2 \int t e^{i t} dt +
\Bb \Be_2 \int t^2 e^{i t} dt
=
\lr{\Ba \Be_2 \lr{ 1 - i t } + \Bb \Be_2 \lr{ 2 i + 2 t - i t^2 } } e^{i t}
=
\lr{ \Ba + 2 \Bb t} \Be_2 e^{i t}
+
\lr{ \Ba t - 2 \Bb  + \Bb t^2 } \Be_1 e^{i t},
\end{dmath}
and the
line integral of \( d\Bx\, \Bf \) is
\begin{dmath}\label{eqn:lineintegraldef:60}
\int d\Bx\, \Bf
=
\Be_2 \int t e^{i t} dt \Ba
+
\Be_2 \int t^2 e^{i t} dt \Bb
=
\Be_2 e^{i t} \lr{ \lr{ 1 - i t } \Ba + \lr{ 2 i + 2 t - i t^2 } \Bb }
=
\Be_2 e^{i t}
\lr{ \Ba + 2 \Bb t}
+
\Be_1 e^{i t}
\lr{ \Ba t - 2 \Bb  + \Bb t^2 }.
\end{dmath}
Unless the vector constants \( \Ba, \Bb \) have only components along the z-axis, \cref{eqn:lineintegraldef:40} and \cref{eqn:lineintegraldef:60} are not generally equal.

\paragraph{Example: Circular bivector.}
Given a bivector valued function \( F(t) = \Be_2 \wedge \lr{ \Be_3 e^{it} } \), where \( i = \Be_{3} \Be_1 \), and a curve
\( \Bx(t) = \Be_3 + \Be_2 t + \Be_1 t^2/2 \), we can compute the line integral with the differential on the right
\begin{dmath}\label{eqn:lineintegraldef:80}
\int F\, d\Bx
=
\Be_{23} \int e^{i t} \lr{ \Be_2 + \Be_1 t} dt
=
-\Be_3 \int e^{it} dt + \Be_{123} \int t e^{-i t} dt
=
\Be_{1} e^{it} + \Be_{123} \lr{ 1 + i t} e^{-it}
=
\Be_1 e^{it} + \Be_2 t e^{-it} + \Be_{123} e^{-it},
\end{dmath}
or the line integral with the differential on the left
\begin{dmath}\label{eqn:lineintegraldef:100}
\int d\Bx\, F
=
\int \lr{ \Be_2 + \Be_1 t} \Be_{31} e^{it} dt
=
\Be_{123} \int e^{it} dt - \Be_3 \int t e^{it}
=
\Be_{2} e^{it} - \Be_3 \lr{ 1 - i t } e^{it}
=
\Be_{2} e^{it} - \Be_3 e^{it} + \Be_{1} t e^{it}.
\end{dmath}
In both \cref{eqn:lineintegraldef:80} and \cref{eqn:lineintegraldef:100} the end result has both vector and trivector grades.  While both integrals are equal (zero) when the angular velocity parameter \( t \) is a multiple of \( 2 \pi \), this shows that the order of the products in the integrand makes a difference once again.

\paragraph{Example: Function with only scalar and pseudoscalar grades.}
In \R{3}, given any function with only scalar and pseudoscalar grades, say \( F(t) = f(t) + I g(t) \), where \( f, g \) are both scalar functions, then the order of the products in a line integrand do not matter.  For any such function we have
\begin{dmath}\label{eqn:lineintegraldef:120}
\int F d\Bx = \int d\Bx F,
\end{dmath}
since both the scalar and pseudoscalar grades commute with any vector differential.

%}
