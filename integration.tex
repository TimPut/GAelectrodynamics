%
% Copyright � 2018 Peeter Joot.  All Rights Reserved.
% Licenced as described in the file LICENSE under the root directory of this GIT repository.
%
%{
\input{../latex/blogpost.tex}
\renewcommand{\basename}{integration}
%\renewcommand{\dirname}{notes/phy1520/}
\renewcommand{\dirname}{notes/ece1228-electromagnetic-theory/}
%\newcommand{\dateintitle}{}
%\newcommand{\keywords}{}

\input{../latex/peeter_prologue_print2.tex}

\usepackage{peeters_layout_exercise}
\usepackage{peeters_braket}
\usepackage{peeters_figures}
\usepackage{siunitx}
%\usepackage{mhchem} % \ce{}
%\usepackage{macros_bm} % \bcM
%\usepackage{macros_qed} % \qedmarker
\usepackage{txfonts} % \ointclockwise

\beginArtNoToc

\generatetitle{XXX}
%\chapter{XXX}
%\label{chap:integration}

   \section{Integration theory.}
      \subsection{Line integral}
         %
% Copyright � 2018 Peeter Joot.  All Rights Reserved.
% Licenced as described in the file LICENSE under the root directory of this GIT repository.
%
%{
\index{differential form}
In geometric algebra, the integrand of a multivector line integral contains product of multivector(s) and a single parameter differential
\makedefinition{Multivector line integral.}{dfn:lineintegraldef:multivectorlineintegral}{
Given a continuous and differentiable curve described by a vector function \( \Bx(a) \), parameterized by single value \( a \) with differential
\begin{equation*}
d^1 \Bx \equiv d\Bx_a = \PD{a}{\Bx} da = \Bx_a da,
\end{equation*}
and multivector functions \( F, G \), the integral
\begin{equation*}
\int F d^1 \Bx G
\end{equation*}
is called a line integral.
} % definition

An illustration of a single parameter curve and its
differential with respect to that parameter, is given in
\cref{fig:oneParameterDifferential:oneParameterDifferentialFig1}.
Observe that the differential is tangent to the curve at all points.
Possible physical realizations of the parameter describing the curve include
time, arclength, and angle.

\imageFigure{../figures/GAelectrodynamics/oneParameterDifferentialFig1}{One parameter manifold.}{fig:oneParameterDifferential:oneParameterDifferentialFig1}{0.2}

Suppose that \( \Bf(\Bx(a)) \) is a vector valued function defined along the curve.
The conventional line integral from vector calculus, a dot product of a differential and the function \( \Bf \) 
may be obtained by the sum of two multivector line integrals one with \( F,G = \Bf/2,1 \), and the other with \( F,G = 1,\Bf/2 \)
\begin{dmath}\label{eqn:lineintegraldef:20}
\int d^1 \Bx \frac{\Bf}{2}
+\int
\frac{\Bf}{2}
d^1 \Bx
=
\int d^1 \Bx \cdot \Bf.
\end{dmath}
Unlike the conventional dot product line integral, the multivector line integral of a vector function such as \( \int d^1 \Bx \Bf \) is generally multivector valued, with both a scalar and a bivector component.  Let's consider some examples of multivector line integrals.

%}

      \subsection{Surface integral}
         %
% Copyright � 2018 Peeter Joot.  All Rights Reserved.
% Licenced as described in the file LICENSE under the root directory of this GIT repository.
%
%{
\index{area element}
\index{differential form}
A two parameter curve, and the corresponding differentials with respect to those parameters, is plotted in
\cref{fig:twoParameterDifferential:twoParameterDifferentialFig1}.

\imageFigure{../figures/GAelectrodynamics/twoParameterDifferentialFig1}{Two parameter manifold differentials.}{fig:twoParameterDifferential:twoParameterDifferentialFig1}{0.4}

Given parameters \( a, b \), the differentials along each of the parameterization directions are
\begin{dmath}\label{eqn:surfaceintegral:100}
\begin{aligned}
d\Bx_a &= \PD{a}{\Bx} da = \Bx_a da \\
d\Bx_b &= \PD{b}{\Bx} db = \Bx_b db.
\end{aligned}
\end{dmath}

The bivector valued surface area element for this parameterization is
\begin{equation}\label{eqn:surfaceintegral:120}
d^2 \Bx
=
d\Bx_a \wedge
d\Bx_b
=
da db (\Bx_a \wedge \Bx_b).
\end{equation}

\makedefinition{Multivector surface integral.}{dfn:lineintegraldef:multivectorsurfaceintegral}{
Given a continuous and differentiable surface described by a vector function \( \Bx(a, b) \), parameterized by two scalars \( a, b \) with differential
\begin{equation*}
d^2 \Bx \equiv d\Bx_a \wedge d\Bx_b =
\PD{a}{\Bx} \wedge \PD{b}{\Bx}
da db = {\Bx_a \wedge \Bx_b } da db,
\end{equation*}
and multivector functions \( F, G \), the integral
\begin{equation*}
\int F d^2 \Bx G
\end{equation*}
is called an surface integral.
} % definition

Unless \( F, G \) are both scalars, such a surface integral is not generally bivector valued like the area element.

In \R{3} it will often be convenient to utilize a dual representation of the area element \( d^2 \Bx = I \ncap dA \), where \( dA \) is a scalar area element, and \( \ncap \) is a normal vector to the surface.  With such an area element representation we will call \( I \int dA\, F \ncap G \) a surface integral.

\paragraph{Spherical surface integral}

From \cref{eqn:curvilinearspherical:300}, we know that
\begin{dmath}\label{eqn:surfaceintegraldef:140}
\Bx_r \Bx_\theta \Bx_\phi = I r^2 \sin\theta,
\end{dmath}
so
\begin{dmath}\label{eqn:surfaceintegraldef:160}
\Bx_\theta \wedge \Bx_\phi
=
\Bx_\theta \Bx_\phi
=
\Bx_r I r^2 \sin\theta,
\end{dmath}
so the (bivector-valued) area element for a spherical surface is
\begin{dmath}\label{eqn:surfaceintegraldef:180}
d^2 \Bx =
I \Bx_r r^2 \sin\theta d\theta d\phi.
\end{dmath}

Suppose we integrate a vector valued function \( F(\theta, \phi) = \alpha \Bx^r + \beta \Bx^\theta + \gamma \Bx^\phi \), where \( \alpha, \beta, \gamma\) are constants, over the surface of a sphere of radius \( r \), then the surface integral (with the area element on the right) is
\begin{dmath}\label{eqn:surfaceintegraldef:200}
\int F d^2\Bx
=
\alpha I r^2 \int \Bx^r \Bx_r \sin\theta d\theta d\phi
+
\beta I r^2 \int \Bx^\theta \Bx_r \sin\theta d\theta d\phi
+
\gamma I r^2 \int \Bx^\phi \Bx_r \sin\theta d\theta d\phi.
\end{dmath}
This can be simplified using \( \rcap \thetacap \phicap = I \), and \cref{eqn:curvilinearspherical:260}, to find
\begin{dmath}\label{eqn:surfaceintegraldef:220}
\begin{aligned}
\Bx^r \Bx_r &= 1 \\
I \Bx^\theta \Bx_r &= \inv{r} I \thetacap \rcap = \inv{r} \phicap \\
I \Bx^\phi \Bx_r &= \inv{r \sin\theta} I \phicap \rcap = -\inv{r \sin\theta} \thetacap,
\end{aligned}
\end{dmath}
so
\begin{dmath}\label{eqn:surfaceintegraldef:240}
\int F d^2\Bx =
\alpha I 4 \pi r^2
+
\beta r \int \phicap \sin\theta d\theta d\phi
-
\gamma r \int \thetacap d\theta d\phi
=
\alpha I 4 \pi r^2,
\end{dmath}
where the integrands containing \( \thetacap, \phicap \) are killed by the integral over \( \phi \in [0, 2\pi] \).  If integrated over a subset of the spherical surface, where such perfect cancellation does not occur, this surface integral may have both vector and trivector components.

\paragraph{Bivector function.}
Given a bivector valued function \( F(a,b) = (a + b) \Be_2 \Be_1 + 2 (a \Be_1 - b \Be_2) \Be_3 \) defined over the unit square \( a,b \in [0, 1] \), and a surface \( \Bx(a,b) = a \Be_1 + b \Be_2 \), the multivector surface integral (with the area element on the right) is
\begin{dmath}\label{eqn:surfaceintegraldef:260}
\int F d^2 \Bx
=
\int_0^1\int_0^1 (a + b) \,da db
+
2 \int_0^1\int_0^1 (a \Be_1 - b \Be_2) \Be_3 \Be_1 \Be_2\, da db
=
1+
I \int_0^1 \evalrange{a^2}{0}{1} \Be_1 db
-
I \int_0^1 \evalrange{b^2}{0}{1} \Be_2 da
=
1+
I \lr{ \Be_1 - \Be_2 }
=
1+
\lr{ \Be_{1} + \Be_{2} } \Be_3.
\end{dmath}
In this example, the integral of a bivector valued function over a (bivector-valued) surface area element results in a multivector with a scalar and bivector grade.  In higher dimensional spaces, such an integral may also have grade-4 components.

%}

      \subsection{Volume integral}
      \subsection{Perfect derivatives}
         % Fixme: Badly named.  This shouldn't be called the multivector integral, but
         % something related to perfect derivative.
         %
% Copyright � 2016 Peeter Joot.  All Rights Reserved.
% Licenced as described in the file LICENSE under the root directory of this GIT repository.
%
%{
%\input{../blogpost.tex}
%\renewcommand{\basename}{fundamentalTheoremOfCalculus}
%\renewcommand{\dirname}{notes/phy1520/}
%%\newcommand{\dateintitle}{}
%%\newcommand{\keywords}{}
%
%\input{../peeter_prologue_print2.tex}
%
%\usepackage{peeters_layout_exercise}
%\usepackage{peeters_braket}
%\usepackage{peeters_figures}
%\usepackage{siunitx}
%
%\beginArtNoToc
%
%\generatetitle{Fundamental theorem of geometric calculus}
%\label{chap:fundamentalTheoremOfCalculus}

\subsection{Hypervolume integral}
We wish to generalize the concepts of line, surface and volume integrals to hypervolumes and multivector functions, and define a hypervolume integral as

\makedefinition{Multivector integral.}{dfn:fundamentalTheoremOfCalculus:240}{
Given a hypervolume parameterized by \( k \) parameters, k-volume volume element \( d^k \Bx \), and
multivector functions \( F, G \), we define k-volume integral with the vector derivative acting to the right on \( F \) as
\begin{equation*}
\int d^k\Bx \lr{ \rboldpartial F },
\end{equation*}
a k-volume integral with the vector derivative acting to the left \( F \) as
\begin{equation*}
\int F d^k\Bx \lboldpartial,
\end{equation*}
and a k-volume integral with the vector derivative acting bidirectionally on \( F, G \) as
\begin{equation*}
\int F d^k\Bx \lrboldpartial G
\equiv
\int \lr{ F d^k\Bx \lboldpartial} G
+
\int F d^k\Bx \lr{ \rboldpartial G },
\end{equation*}
where the meaning given to these directionally acting derivative operations is
\begin{equation*}
F d^k \Bx \lrboldpartial G
=
F d^k \Bx \lr{ \sum_i \Bx^i {\stackrel{ \leftrightarrow }{\partial_i}} } G
=
(\partial_i F) d^k \Bx \sum_i \Bx^i G
+
F d^k \Bx \sum_i \Bx^i (\partial_i G)
\equiv
(F d^k \Bx \lboldpartial) G
+
F d^k \Bx (\rboldpartial G),
\end{equation*}
with \( \boldpartial \) acting on \( F \) and \( G \), but not the volume element \( d^k \Bx \), which may also be a function of the implied parameterization.
} % definition

The vector derivative (and gradient)
may not commute with \( F, G \) nor the volume element \( d^k \Bx \), so we are forced to use some notation to indicate what the vector derivative (or gradient) acts on.
In conventional right acting cases, where there is no ambiguity, arrows will usually be omitted, but braces may also be used to indicate the scope of derivative operators.
This bidirectional notation will also be used for the gradient, especially for volume integrals in \R{3} where the vector derivative is identitical to the gradient.

Some authors use overdots or ticks are used to indicate the exact scope of multivector derivative operators, as in
\begin{dmath}\label{eqn:fundamentalTheoremOfCalculus:260}
F d^k \Bx \boldpartial G =
\dot{F} d^k \Bx \dot{\boldpartial} G
+
F d^k \Bx \dot{\boldpartial} \dot{G}.
\end{dmath}
Here the (Hestenes) dot notation would have the advantage of emphasizing that the action of the vector derivative (or gradient) is on the functions \( F, G \), and not on the hypervolume element \( d^k \Bx \).
In this book, where we will use ticks to indicate whether gradients are with respect to primed \( \Bx' \) or unprimed \( \Bx \) variables, over arrows seemed like a better choice than dots to indicate operator scope, and have the advantage of being visually conspicuous.

\subsection{Fundamental theorem.}
\index{fundamental theorem of geometric calculus}

The fundamental theorem of geometric calculus is a generalization of many conventional scalar and vector integral theorems.
It is a powerful theorem, which we will use with Green's functions to solve Maxwell's equation, and to derive the geometric algebra form of Stokes' theorem.

\maketheorem{Fundamental theorem of geometric calculus}{thm:fundamentalTheoremOfCalculus:1}{
For multivectors \(F, G \), and a hypervolume element \(d^k \Bx\),
\begin{equation*}
\int_V F d^k \Bx \boldpartial G = \oint_{\partial V} F d^{k-1} \Bx G.
\end{equation*}
}

This theorem relates the hypervolume integral to the integral over the bounding surface of hypervolume.
Additional work is required to describe the precise meaning of the boundary differential \( d^{k-1} \Bx \).  We will do so for line, surface, and volume integrals, proving the theorem in a limited fashion for each of those cases as we go.

For a full proof of \cref{thm:fundamentalTheoremOfCalculus:1}, additional mathematical sublties must be considered.
For full proofs and additional details, the reader is referred to \citep{hestenes1985clifford}, \citep{doran2003gap}, \citep{aMacdonaldVAGC} and \citep{sobczyk2011fundamental}, which all
which all tackle different aspects of general geometric calculus.

Before considering multivector line, surface and volume integral specializations of
\cref{thm:fundamentalTheoremOfCalculus:1},
we will state Stokes' theorem in its geometric algebra form.

%}
%\EndArticle

      \subsection{Fundamental theorem.}
         %
% Copyright © 2018 Peeter Joot.  All Rights Reserved.
% Licenced as described in the file LICENSE under the root directory of this GIT repository.
%
%{
\index{fundamental theorem of geometric calculus}

The fundamental theorem of geometric calculus is a generalization of many conventional scalar and vector integral theorems, and relates a hypervolume integral to its boundary.
This is a a powerful theorem, which we will use with Green's functions to solve Maxwell's equation, but also to derive the geometric algebra form of Stokes' theorem, from which most of the familiar integral calculus results follow.
%
% Copyright � 2018 Peeter Joot.  All Rights Reserved.
% Licenced as described in the file LICENSE under the root directory of this GIT repository.
%
\maketheorem{Fundamental theorem of geometric calculus}{thm:fundamentalTheoremOfCalculus:1}{
Given
multivectors \(F, G \),
a parameterization \( \Bx = \Bx(u_1, u_2, \cdots) \), with hypervolume element \( d^k \Bx = d^k u I_k \), where
\( I_k = \Bx_1 \wedge \Bx_2 \wedge \cdots \wedge \Bx_k \), the hypervolume integral is related to the boundary integral by
\begin{equation*}
\int_V F d^k \Bx \lrboldpartial G = \int_{\partial V} F d^{k-1} \Bx G,
\end{equation*}
where \( \partial V \) represents the boundary of the volume, and \( d^{k-1} \Bx \) is the hypersurface element.
This is called the \textit{Fundamental theorem of geometric calculus}.

The hypersurface element and boundary integral is defined for \( k > 1 \) as
\begin{equation*}
\int_{\partial V} F d^{k-1} \Bx G
\equiv
\sum_i \int d^{k-1} u_i \evalbar{ \lr{ F \lr{ I_k \cdot \Bx^i} G }}{\Delta u_i},
\end{equation*}
where \( d^{k-1} u_i \) is the product of all \( du_j \) except for \( du_i \).
For
\( k = 1 \) the hypersurface element and associated
boundary ``integral''
is really just convenient general shorthand, and
should be taken to mean the evaluation of the \( F G \) multivector product over the range of the parameter
\begin{equation*}
\int_{\partial V} F d^{0} \Bx G
\equiv
\evalbar{ F G }{\Delta u_1}.
\end{equation*}
} % theorem

The geometry of the hypersurface element \( d^{k-1} \Bx \) will be made more clear when we
consider the specific cases of \( k = 1, 2, 3 \), representing generalized line, surface, and volume integrals respectively.
Instead of terrorizing the reader with a general proof
\cref{thm:fundamentalTheoremOfCalculus:1},
which requires some unpleasant index gymnastics,
this book
will separately state and prove the fundamental theorem of calculus
for each of the \( k = 1, 2, 3 \) cases that are of interest for problems in \R{2} and \R{3}.
For the interested reader, a sketch of the general proof
of \cref{thm:fundamentalTheoremOfCalculus:1}
is available in \cref{chap:gagcProof}.

Before moving on to the line, surface, and volume integral cases, we will state and prove the
general Stokes' theorem in its geometric algebra form.

%}

   \subsection{Stokes' theorem.}
      %
% Copyright © 2013 Peeter Joot.  All Rights Reserved.
% Licenced as described in the file LICENSE under the root directory of this GIT repository.
%
\maketheorem{Stokes' Theorem}{thm:stokesTheoremGeometricAlgebra:1740}{

For blades \(F \in \bigwedge^{s}\), and \(m\) volume element \(d^k \Bx, s < k\),

\begin{equation*}%\label{eqn:stokesTheoremTheStatement:120}
\int_V d^k \Bx \cdot (\boldpartial \wedge F) = \int_{\partial V} d^{k-1} \Bx \cdot F.
\end{equation*}

Here the volume integral is over a \(m\) dimensional surface (manifold).  The derivative operator \(\boldpartial\) is called the vector derviative and is the projection of the gradient onto the tangent space of the manifold.  Integration over the boundary of \(V\) is indicated by \( \partial V \).
}

The vector derivative is defined by

\begin{equation}\label{eqn:stokesTheoremTheStatement:1400}
\boldpartial = \Bx^i \partial_i = \sum_i \Bx_i \PD{u^i}{}.
\end{equation}

where \( \Bx^i \) are reciprocal frame vectors dual to the tangent vector basis \( \Bx_i \) associated with the parameters \( u^1, u^2, \cdots \).
%These will be defined in more detail in the next section.
Once the volume element, vector product and the other concepts are defined, the proof of
Stokes theorem is really just a statement that

\boxedEquation{eqn:stokesTheoremGeometricAlgebra:2840}{
\int_V d^k \Bx \cdot (\Bx^i \partial_i \wedge F) =
\int_V \lr{ d^k \Bx \cdot \Bx^i } \cdot \partial_i F.
}

This dot product expansion applies to any degree blade and volume element provided the degree of the blade is less than that of the volume element (i.e. \(s < k\)).  That magic follows directly from \cref{thm:stokesTheoremGeometricAlgebra:1420}.

   \subsection{Line integral.}
      %
% Copyright � 2018 Peeter Joot.  All Rights Reserved.
% Licenced as described in the file LICENSE under the root directory of this GIT repository.
%
%{
%%%\input{../latex/blogpost.tex}
%%%\renewcommand{\basename}{lineintegral}
%%%%\renewcommand{\dirname}{notes/phy1520/}
%%%\renewcommand{\dirname}{notes/ece1228-electromagnetic-theory/}
%%%%\newcommand{\dateintitle}{}
%%%%\newcommand{\keywords}{}
%%%
%%%\input{../latex/peeter_prologue_print2.tex}
%%%
%%%\usepackage{peeters_layout_exercise}
%%%\usepackage{peeters_braket}
%%%\usepackage{peeters_figures}
%%%\usepackage{siunitx}
%%%%\usepackage{mhchem} % \ce{}
%%%%\usepackage{macros_bm} % \bcM
%%%%\usepackage{macros_qed} % \qedmarker
%%%%\usepackage{txfonts} % \ointclockwise
%%%
%%%\beginArtNoToc
%%%
%%%\generatetitle{Multivector line integral.}
%\section{Line integral.}
\label{chap:lineintegral}

\index{differential form}
A single parameter curve, and the corresponding differential with respect to that parameter, is plotted in
\cref{fig:oneParameterDifferential:oneParameterDifferentialFig1}.
%, for a parameterization over \( [a, b] \in [0,1]\otimes[0,1] \).

\imageFigure{../figures/GAelectrodynamics/oneParameterDifferentialFig1}{One parameter manifold.}{fig:oneParameterDifferential:oneParameterDifferentialFig1}{0.3}

The differential with respect to the parameter \( a \) is
\begin{equation}\label{eqn:lineintegral:20}
d\Bx_a = \PD{a}{\Bx} da = \Bx_a da.
\end{equation}

The vector derivative has just one component
\begin{dmath}\label{eqn:lineintegral:40}
\boldpartial
=
\sum_i \Bx^i (\Bx_i \cdot \spacegrad)
=
\Bx^a \PD{a}{}
\equiv
\Bx^a \partial_a.
\end{dmath}

(cut)
\makedefinition{Multivector line integral.}{dfn:lineintegral:100}{
Given an connected curve \( C \) parameterized by a single parameter, and multivector functions \( F, G \), we define the line integral as
\begin{equation*}
\int_C F d^1\Bx \lrboldpartial G
\equiv
\int_C \lr{ F d^1\Bx \lboldpartial} G
+
\int_C F d^1\Bx \lr{ \rboldpartial G },
\end{equation*}
where the one parameter differential form \( d^1 \Bx = da \Bx_a \) varies over the curve.
} % definition


The fundamental theorem of calculus for a mulitvector line integral is just

\maketheorem{Multivector line integral.}{thm:lineintegral:100}{
Given an connected curve \( C \) parameterized by a single parameter, and multivector functions \( F, G \), the line integral
\begin{equation*}
\int_C F d^1\Bx \boldpartial G
= \evalbar{F G}{\Delta a}.
\end{equation*}
} % theorem

Using the (single variable) parameterization \( a \) above, the proof follows directly by expansion
\begin{dmath}\label{eqn:lineintegral:120}
\int_C F d^1\Bx \boldpartial G
=
\int_C \lr{ F d^1\Bx \lboldpartial} G
+
\int_C F d^1\Bx \lr{ \rboldpartial G }
=
\int_C \PD{a}{F} da \Bx_a \Bx^a G
+
\int_C F da \Bx_a \Bx^a \PD{a}{G}
=
\int_C da \PD{a}{F} G
+
\int_C da F \PD{a}{G}
=
\int_C da \PD{a}{} \lr{ F G }
=
F(a_1) G(a_1) -
F(a_0) G(a_0).
.
\end{dmath}

We have a perfect cancellation of the reciprocal frame \( \Bx^a \) with the vector \( \Bx_a \) that lies along the curve, since \( \Bx^a \Bx_a = 1 \).  This leaves a perfect derivative of the product of \( F G \), which can be integrated over the length of the curve, yielding the difference of the product with respect to the parameterization of the end points of the curve (assumed to be \( [a_0, a_1] \) in the expansion above.)

For a single parameter subspace
the reciprocal frame vector \( \Bx^a \)
is trivial to calculate, as it is just the inverse of \( \Bx_a \), that is \( \Bx^a = \Bx_a/\Norm{\Bx_a}^2 \).
Observe that we did not actually have to calculate it, but instead only require that the vector is invertible.

An important (and familiar) special case of \cref{thm:lineintegral:100} is the fundamental theorem of calculus for line integrals, which can be obtained by using a
single scalar function \( f \)

\maketheorem{Line integral of a scalar function (Stokes').}{thm:lineintegral:180}{
Given a scalar function \( f \), its line integral is given by
\begin{equation*}
\int_C d^1\Bx \cdot \boldpartial f =
\int_C d^1\Bx \cdot \spacegrad f = \evalbar{F}{\Delta a}.
\end{equation*}
} % theorem

Writing out \cref{thm:lineintegral:100} with \( F = 1, G = f(\Bx(a)) \), we have
\begin{dmath}\label{eqn:lineintegral:140}
\int_C d^1\Bx \boldpartial f = \evalbar{f}{\Delta a}.
\end{dmath}

This is a multivector equation with scalar and bivector grades on the left hand side, but only scalar grades on the right.  Equating grades yields two equations

\begin{subequations}
\label{eqn:lineintegral:180}
\begin{dmath}\label{eqn:lineintegral:160}
\int_C d^1\Bx \cdot \boldpartial f = \evalbar{f}{\Delta a}
\end{dmath}
\begin{dmath}\label{eqn:lineintegral:200}
\int_C d^1\Bx \wedge \boldpartial f = 0
\end{dmath}
\end{subequations}

Because \( d^1\Bx \cdot \boldpartial = d^1\Bx \cdot \spacegrad \), we can replace the vector derivative with the gradient in \cref{eqn:lineintegral:160}, which yields the conventional line integral result, proving the theorem.

%}
%\EndNoBibArticle

   \subsection{Surface integral.}
      %
% Copyright � 2018 Peeter Joot.  All Rights Reserved.
% Licenced as described in the file LICENSE under the root directory of this GIT repository.
%
%{
%%%\input{../latex/blogpost.tex}
%%%\renewcommand{\basename}{surfaceintegral}
%%%%\renewcommand{\dirname}{notes/phy1520/}
%%%\renewcommand{\dirname}{notes/ece1228-electromagnetic-theory/}
%%%%\newcommand{\dateintitle}{}
%%%%\newcommand{\keywords}{}
%%%
%%%\input{../latex/peeter_prologue_print2.tex}
%%%
%%%\usepackage{peeters_layout_exercise}
%%%\usepackage{peeters_braket}
%%%\usepackage{peeters_figures}
%%%\usepackage{siunitx}
%%%%\usepackage{mhchem} % \ce{}
%%%%\usepackage{macros_bm} % \bcM
%%%%\usepackage{macros_qed} % \qedmarker
%%%\usepackage{txfonts} % \ointclockwise
%%%
%%%\beginArtNoToc
%%%
%%%\generatetitle{Multivector surface integral.}
%\section{Surface integral.}
%\label{chap:surfaceintegral}

\index{area element}
\index{differential form}
A two parameter curve, and the corresponding differentials with respect to those parameters, is plotted in
\cref{fig:twoParameterDifferential:twoParameterDifferentialFig1}.

\imageFigure{../figures/GAelectrodynamics/twoParameterDifferentialFig1}{Two parameter manifold differentials.}{fig:twoParameterDifferential:twoParameterDifferentialFig1}{0.4}

Given parameters \( a, b \), the differentials along each of the parameterization directions are

\begin{dmath}\label{eqn:surfaceintegral:100}
\begin{aligned}
d\Bx_a &= \PD{a}{\Bx} da = \Bx_a da \\
d\Bx_b &= \PD{b}{\Bx} db = \Bx_b db.
\end{aligned}
\end{dmath}

The bivector valued surface area element for this parameterization is

\begin{equation}\label{eqn:surfaceintegral:120}
d^2 \Bx
=
d\Bx_a \wedge
d\Bx_b
=
da db (\Bx_a \wedge \Bx_b).
\end{equation}

The vector derivative, the projection of the gradient onto the surface at the point of integration (also called the tangent space), now has two components

\begin{dmath}\label{eqn:surfaceintegral:200}
\boldpartial
=
\sum_i \Bx^i (\Bx_i \cdot \spacegrad)
=
\Bx^a \PD{a}{}
+
\Bx^b \PD{b}{}
\equiv
\Bx^a \partial_a
+
\Bx^b \partial_b.
\end{dmath}

We define a multivector surface integral by

\makedefinition{Multivector surface integral.}{dfn:surfaceintegral:100}{
Given an connected surface \( S \) parameterized by two parameters, and multivector functions \( F, G \), we define the surface integral as
\begin{equation*}
\int_S F d^2\Bx \boldpartial G
\equiv
\int_S \lr{ F d^2\Bx \lboldpartial} G
+
\int_S F d^2\Bx \lr{ \rboldpartial G },
\end{equation*}
where the two parameter differential form \( d^2 \Bx = da db\, \Bx_a \wedge \Bx_b \) varies over the surface.
} % definition

As mentioned in a line integral context,
multivectors may not commute with the vector derivative or the differential, so we allow the vector derivative to act bidirectionally using the chain rule.
The scope of the action of the vector derivative when acting only to the left or right is indicated using braces above.
Should we wish to only integrate single functions, we can set either of the other to \( 1 \), yielding integrals of the form
\( \int_S F d^2\Bx \lboldpartial, \) or \( \int_S d^2\Bx \boldpartial G \).

\maketheorem{Multivector surface integral.}{thm:surfaceintegral:100}{
Given an connected surface \( S \) parameterized by two parameters, and multivector functions \( F, G \), the surface integral
\begin{equation*}
\int_S F d^2\Bx \boldpartial G
= \ointclockwise_{\partial S} F d\Bx G,
\end{equation*}
where \( \partial S \) is the boundary of the surface \( S \).
} % theorem

To see why this works, we would first like to reduce the product of the area element and the vector derivative

\begin{dmath}\label{eqn:surfaceintegral:300}
d^2\Bx \boldpartial
=
da db\, \lr{ \Bx_a \wedge \Bx_b } \lr{ \Bx^a \partial_a + \Bx^b \partial_b }.
\end{dmath}

Since \( \Bx^a \in \Span \setlr{ \Bx_a, \Bx_b } \), this multivector product has only a vector grade.  That is

\begin{dmath}\label{eqn:surfaceintegral:320}
\lr{ \Bx_a \wedge \Bx_b } \Bx^a
=
\lr{ \Bx_a \wedge \Bx_b } \cdot \Bx^a
+
\cancel{ \lr{ \Bx_a \wedge \Bx_b } \wedge \Bx^a }
=
\lr{ \Bx_a \wedge \Bx_b } \cdot \Bx^a
=
\Bx_a \lr{ \Bx_b \cdot \Bx^a } -
\Bx_b \lr{ \Bx_a \cdot \Bx^a }
=
-\Bx_b.
\end{dmath}

Similarly
\begin{dmath}\label{eqn:surfaceintegral:340}
\lr{ \Bx_a \wedge \Bx_b } \Bx^b
=
\lr{ \Bx_a \wedge \Bx_b } \cdot \Bx^b
+
\cancel{ \lr{ \Bx_a \wedge \Bx_b } \wedge \Bx^b }
=
\lr{ \Bx_a \wedge \Bx_b } \cdot \Bx^b
=
\Bx_a \lr{ \Bx_b \cdot \Bx^b } -
\Bx_b \lr{ \Bx_a \cdot \Bx^b }
=
\Bx_a,
\end{dmath}
so
\begin{dmath}\label{eqn:surfaceintegral:360}
d^2\Bx \boldpartial
=
\Bx_a \partial_b
-\Bx_b \partial_a.
\end{dmath}

Inserting this into the surface integral, we find

\begin{dmath}\label{eqn:surfaceintegral:380}
\int_S F d^2\Bx \boldpartial G
=
\int_S \lr{ F d^2\Bx \lboldpartial} G
+
\int_S F d^2\Bx \lr{ \rboldpartial G }
=
\int_S da db\, \lr{ \partial_b F \Bx_a - \partial_a F \Bx_b } G
+
\int_S da db\, F \lr{ \Bx_a \partial_b G - \Bx_b \partial_a G }
=
\int_S da db\, \lr{ \PD{b}{F} \PD{a}{\Bx} - \PD{a}{F} \PD{b}{\Bx} } G
+
\int_S da db\, F \lr{ \PD{a}{\Bx} \PD{b}{G} - \PD{b}{\Bx} \PD{a}{G} }
=
\int_S da db\, \PD{b}{} \lr{ F \PD{a}{\Bx} G } - \int_S da db\, \PD{a}{} \lr{ F \PD{b}{\Bx} G }.
\end{dmath}

We are able to pull out the partials because \( \Bx_a \) is computed with the parameter \( b \) fixed,
and \( \Bx_b \) is computed with the parameter \( a \) fixed.

This leaves two perfect differentials, which can both be integrated separately.  That gives

\begin{dmath}\label{eqn:surfaceintegral:400}
\int_S F d^2\Bx \boldpartial G
=
\int_{\Delta a} da\, \evalbar{\lr{ F \PD{a}{\Bx} G }}{\Delta b} - \int_{\Delta b} db\, \evalbar{\lr{ F \PD{b}{\Bx} G }}{\Delta a}
=
\int_{\Delta a} \evalbar{\lr{ F d\Bx_a G }}{\Delta b} - \int_{\Delta b} \evalbar{\lr{ F d\Bx_b G }}{\Delta a}.
\end{dmath}

Suppose we are integrating over the unit parameter volume space \( [a, b] \in [0,1] \otimes [0,1] \) as illustrated in
\cref{fig:twoParameterDifferentialBoundary:twoParameterDifferentialBoundaryFig2}.
\imageFigure{../figures/GAelectrodynamics/twoParameterDifferentialBoundaryFig2}{Contour for two parameter surface boundary.}{fig:twoParameterDifferentialBoundary:twoParameterDifferentialBoundaryFig2}{0.4}

Comparing to the figure we see that we've ended up with a clockwise line integral around the boundary of the surface.
For a given subset of the surface, the bivector area element can be chosen small enough that it lies in the tangent space
to the surface at the point of integration.
In that case, a larger bounding loop can be concepualized as the sum of a number of smaller ones, as sketched
in \cref{fig:loopIntegralInfinitesimalSum:loopIntegralInfinitesimalSumFig2},
in which case the
contributions of the interior loop segments cancel out.

\imageFigure{../figures/gabook/loopIntegralInfinitesimalSumFig2}{Sum of infinitesimal loops.}{fig:loopIntegralInfinitesimalSum:loopIntegralInfinitesimalSumFig2}{0.35}

There are some subtlies related to general triangulation of the surface and the smoothness of the surface that really need to be considered for a full proof
of \cref{thm:surfaceintegral:100}, but the sketch above gives the rough ideas.
For some of that detail, the reader is referred to \citep{hestenes1985clifford}, and \citep{doran2003gap}, which tackle general geometric calculus in more detail.

\subsection{Two parameter Stokes' theorem.}

Two special cases of \cref{thm:surfaceintegral:100} when scalar and vector functions are integrated over a surface.  For scalar functions we have

\maketheorem{Surface integral of scalar function (Stokes').}{thm:surfaceintegral:420}{
Given a scalar function \( f(\Bx) \) its surface integrals is given by
\begin{equation*}
\int_S d^2 \Bx \cdot \boldpartial f =
\int_S d^2 \Bx \cdot \spacegrad f = \ointclockwise_{\partial S} d\Bx f.
\end{equation*}
In \R{3} this can be written as
\begin{equation*}
\int_S dA \ncap \cross \spacegrad f = \ointctrclockwise_{\partial S} d\Bx f,
\end{equation*}
where \( \ncap \) is the outwards normal specified by \( d^2 \Bx = I \ncap dA \).
} % theorem

To show the first part, we can split the (multivector) surface integral into vector and trivector grades

\begin{dmath}\label{eqn:surfaceintegral:440}
\int_S d^2\Bx \boldpartial f
=
\int_S d^2\Bx \cdot \boldpartial f
+
\int_S d^2\Bx \wedge \boldpartial f.
\end{dmath}

Since \( \Bx^a, \Bx^b \) both lie in the span of \( \setlr{ \Bx_a, \Bx_b } \),
\( d^2\Bx \wedge \boldpartial = 0 \), killing the second integral in \cref{eqn:surfaceintegral:440}.
If the gradient is decomposed into its projection along the tangent
space (the vector derivative) and its perpendicular components, only the vector derivative components of the
gradient contribute to its dot product with the area element.  That is

\begin{dmath}\label{eqn:surfaceintegral:460}
d^2 \Bx \cdot \spacegrad
=
d^2 \Bx \cdot \lr{ \Bx^a \partial_a + \Bx^b \partial_b + \cdots }
=
d^2 \Bx \cdot \lr{ \Bx^a \partial_a + \Bx^b \partial_b }
=
d^2 \Bx \cdot \boldpartial.
\end{dmath}

This means that for a scalar function

\begin{dmath}\label{eqn:surfaceintegral:480}
\int_S d^2\Bx \boldpartial f
=
\int_S d^2\Bx \cdot \spacegrad f.
\end{dmath}

The second part of the theorem follows by grade selection, and application of a duality transformation for the area element

\begin{dmath}\label{eqn:surfaceintegral:500}
d^2 \Bx \cdot \spacegrad f
=
\gpgradeone{ d^2 \Bx \spacegrad f }
=
dA \gpgradeone{ I \ncap \spacegrad f }
=
dA \gpgradeone{ I \lr{ \ncap \cdot \spacegrad f + I \ncap \cross \spacegrad f} }
=
-dA \ncap \cross \spacegrad f.
\end{dmath}

back substitution of \cref{eqn:surfaceintegral:500} completes the proof of \cref{thm:surfaceintegral:420}.

For vector functions we have

\maketheorem{Surface integral of a vector function (Stokes').}{thm:surfaceintegral:500}{
Given a vector function \( \Bf(\Bx) \) the surface integral is given by
\begin{equation*}
\int_S d^2 \Bx \cdot (\spacegrad \wedge \Bf) = \ointclockwise_{\partial S} d\Bx \cdot \Bf.
\end{equation*}
In \R{3} this can be written as
\begin{equation*}
\int_S dA \ncap \cdot \lr{ \spacegrad \cross \Bf} = \ointctrclockwise_{\partial S} d\Bx \cdot \Bf,
\end{equation*}
where \( \ncap \) is the outwards normal specified by \( d^2 \Bx = I \ncap dA \).
} % theorem

This follows by setting \( F = 1, G = \Bf \) in \cref{thm:surfaceintegral:100} and selecting the scalar grade.  In particular we may form the
scalar selection of \( d^2 \Bx \boldpartial \Bf \) in two different ways.  The first is

\begin{dmath}\label{eqn:surfaceintegral:520}
\gpgradezero{ d^2 \Bx \boldpartial \Bf }
=
\gpgradezero{ (d^2 \Bx \cdot \boldpartial + d^2 \Bx \wedge \boldpartial ) \Bf }.
\end{dmath}

The \( d^2 \Bx \wedge \boldpartial \) product with \( \Bf \) has only bivector and quad-vector components (the latter is zero in \R{3}), so its scalar grade selection is zero, and we are left with

\begin{dmath}\label{eqn:surfaceintegral:540}
\gpgradezero{ d^2 \Bx \boldpartial \Bf }
=
(d^2 \Bx \cdot \boldpartial) \cdot \Bf
=
(d^2 \Bx \cdot \spacegrad) \cdot \Bf,
\end{dmath}
where we have used \cref{eqn:surfaceintegral:460} again.  This product can also be written as

\begin{dmath}\label{eqn:surfaceintegral:560}
(d^2 \Bx \cdot \spacegrad) \cdot \Bf
=
\gpgradezero{ (d^2 \Bx \cdot \spacegrad) \Bf }
=
\gpgradezero{ (d^2 \Bx \spacegrad - d^2 \Bx \wedge \spacegrad) \Bf }
=
\gpgradezero{ d^2 \Bx \spacegrad \Bf }
=
\gpgradezero{ d^2 \Bx \lr{ \cancel{ \spacegrad \cdot \Bf } + \spacegrad \wedge \Bf } }
=
d^2 \Bx \cdot \lr{ \spacegrad \wedge \Bf }.
\end{dmath}

\begin{dmath}\label{eqn:surfaceintegral:580}
\ointclockwise_{\partial S} d\Bx \cdot \Bf
=
\gpgradezero{ \int_S d^2\Bx \boldpartial \Bf }
=
\int_S \lr{ d^2\Bx \cdot \spacegrad } \cdot \Bf
=
\int_S d^2\Bx \cdot \lr{ \spacegrad \wedge \Bf },
\end{dmath}
as claimed.  In particular in \R{3}, we have

\begin{dmath}\label{eqn:surfaceintegral:600}
d^2\Bx \cdot \lr{ \spacegrad \wedge \Bf }
=
dA \gpgradezero{ I \ncap I \lr{ \spacegrad \cross \Bf } }
=
-dA \ncap \cdot \lr{ \spacegrad \cross \Bf }.
\end{dmath}

Substitution into \cref{eqn:surfaceintegral:580} proves the last part of \cref{thm:surfaceintegral:500}.

\subsection{Green's theorem.}

\Cref{thm:surfaceintegral:500}, when stated in terms of coordinates, is another well known result

\maketheorem{Green's theorem.}{thm:surfaceintegral:620}{
Given a vector \( \Bf = \sum_i f_i \Bx^i \) in \R{N}, and a surface parameterized by \( \Bx = \Bx(u_1, u_2) \)
\begin{equation*}
\int_S du_1 du_2 \lr{ \PD{u_2}{f_1} - \PD{u_1}{f_2} }
=
\ointclockwise_{\partial S} du_1 f_1 + du_2 f_2.
\end{equation*}
This is
often stated for \R{2} with a Cartesian \(x,y\) parameterization, such as \( \Bf = P \Be_1 + Q \Be_2 \).  In that case
\begin{equation*}
\int_S dx dy \lr{ \PD{y}{P} - \PD{x}{Q} }
=
\ointclockwise_{\partial S} P dx + Q dy.
\end{equation*}
} % theorem

FIXME: Add an example (lots to pick from in any 3rd term calc text)

The first equality in \cref{thm:surfaceintegral:620} holds in \R{N} for vectors expressed in terms of an arbitrary curvilinear basis.
Only the coordinates of the vector \( \Bf \) contribute to this integral, and only those that lie in the tangent space.
The reciprocal basis vectors \( \Bx^i \) are also nowhere to be seen.  This is because they are either obliterated in dot products with \( \Bx_j \), or cancel due to mixed partial equality.

To see how this occurs let's look at the
area integrand of \cref{thm:surfaceintegral:500}
\begin{dmath}\label{eqn:surfaceintegral:660}
d^2 \Bx \cdot \lr{ \spacegrad \wedge \Bf }
=
du_1 du_2\, \lr{ \Bx_1 \wedge \Bx_2 } \cdot \lr{ \sum_{ij} \lr{ \Bx^i \partial_i } \wedge \lr{ f_j \Bx^j } }
=
du_1 du_2\, \sum_{ij} \lr{ \lr{ \Bx_1 \wedge \Bx_2 } \cdot \Bx^i } \cdot \lr{ \partial_i (f_j \Bx^j) }
=
du_1 du_2\, \sum_{ij} \lr{ \lr{ \Bx_1 \wedge \Bx_2 } \cdot \Bx^i } \cdot \Bx^j \partial_i f_j
+
du_1 du_2\, \sum_{ij} f_j \lr{ \lr{ \Bx_1 \wedge \Bx_2 } \cdot \Bx^i } \cdot (\partial_i \Bx^j).
\end{dmath}

With a bit of trouble, we will see that the second integrand is zero.  On the other hand, the first integrand
simplifies
without too much trouble

\begin{dmath}\label{eqn:surfaceintegral:680}
\sum_{ij} \lr{ \lr{ \Bx_1 \wedge \Bx_2 } \cdot \Bx^i } \cdot \Bx^j \partial_i f_j
=
\sum_{ij} \lr{ \Bx_1 \delta_{2i} - \Bx_2 \delta_{1i} } \cdot \Bx^j \partial_i f_j
=
\sum_{j} \Bx_1 \cdot \Bx^j \partial_2 f_j -\Bx_2 \cdot \Bx^j \partial_1 f_j
=
\partial_2 f_1 - \partial_1 f_2.
\end{dmath}

For the second integrand, we have
\begin{dmath}\label{eqn:surfaceintegral:700}
\sum_{ij} f_j \lr{ \lr{ \Bx_1 \wedge \Bx_2 } \cdot \Bx^i } \cdot (\partial_i \Bx^j)
=
\sum_{j} f_j \sum_i \lr{ \Bx_1 \delta_{2i} - \Bx_2 \delta_{1i} } \cdot (\partial_i \Bx_j)
=
\sum_{j} f_j
\lr{
\Bx_1 \cdot (\partial_2 \Bx^j)
-
\Bx_2 \cdot (\partial_1 \Bx^j)
}
\end{dmath}

We can apply the chain rule (backwards) to the portion in brackets to find

\begin{dmath}\label{eqn:surfaceintegral:720}
\Bx_1 \cdot (\partial_2 \Bx^j)
-
\Bx_2 \cdot (\partial_1 \Bx^j)
=
\cancel{\partial_2 \lr{ \Bx_1 \cdot \Bx^j }}
-
(\partial_2 \Bx_1) \cdot \Bx^j
-
\cancel{\partial_1 \lr{ \Bx_2 \cdot \Bx^j }}
+
(\partial_1 \Bx_2) \cdot \Bx^j
=
\Bx_j \cdot \lr{ \partial_1 \Bx_2 - \partial_2 \Bx_1 }
=
\Bx_j \cdot \lr{ \PD{u_1}{} \PD{u_2}{\Bx} - \PD{u_2}{} \PD{u_1}{\Bx} }
= 0.
\end{dmath}

In this reduction the derivatives of \( \Bx_i \cdot \Bx^j = \delta_{ij} \) were killed since those are constants (either zero or one).  The final step relies on the fact that we assume our vector parameterization is well behaved enought that the mixed partials are zero.

Substituting these results into
\cref{thm:surfaceintegral:500} we find

\begin{dmath}\label{eqn:surfaceintegral:740}
\ointclockwise_{\partial S} d\Bx \cdot \Bf
=
\ointclockwise_{\partial S} \lr{ du_1 \Bx_1 + du_2 \Bx_2 } \cdot \lr{ \sum_i f_i \Bx^i }
=
\ointclockwise_{\partial S} du_1 f_1 + du_2 f_2
=
\int_S du_1 du_2\, \lr{ \partial_2 f_1 - \partial_1 f_2 },
\end{dmath}
which completes the proof.

%}
%%%\EndArticle

   \subsection{Volume integral.}
      %
% Copyright � 2018 Peeter Joot.  All Rights Reserved.
% Licenced as described in the file LICENSE under the root directory of this GIT repository.
%
%{
%%%\input{../latex/blogpost.tex}
%%%\renewcommand{\basename}{volumeintegral}
%%%%\renewcommand{\dirname}{notes/phy1520/}
%%%\renewcommand{\dirname}{notes/ece1228-electromagnetic-theory/}
%%%%\newcommand{\dateintitle}{}
%%%%\newcommand{\keywords}{}
%%%
%%%\input{../latex/peeter_prologue_print2.tex}
%%%
%%%\usepackage{peeters_layout_exercise}
%%%\usepackage{peeters_braket}
%%%\usepackage{peeters_figures}
%%%\usepackage{siunitx}
%%%%\usepackage{mhchem} % \ce{}
%%%%\usepackage{macros_bm} % \bcM
%%%%\usepackage{macros_qed} % \qedmarker
%%%\usepackage{txfonts} % \ointclockwise
%%%
%%%\beginArtNoToc
%%%
%%%\generatetitle{Volume integral.}
%%%%\chapter{Volume integral.}
\label{chap:volumeintegral}

%\subsection{Volume integral.}
\index{volume parameterization}
\index{volume element}
\index{differential form}
A three parameter curve, and the corresponding differentials with respect to those parameters, is sketched in
\cref{fig:normalsOnVolumeAreaElement:normalsOnVolumeAreaElementFig11}.

\imageFigure{../figures/gabook/normalsOnVolumeAreaElementFig11}{Three parameter volume element.}{fig:normalsOnVolumeAreaElement:normalsOnVolumeAreaElementFig11}{0.4}

Given parameters \( u_1, u_2, u_3 \), we can denote the differentials along each of the parameterization directions as
\begin{dmath}\label{eqn:volumeintegral:100}
\begin{aligned}
d\Bx_1 &= \PD{u_1}{\Bx} du_1 = \Bx_1 du_1 \\
d\Bx_2 &= \PD{u_2}{\Bx} du_2 = \Bx_2 du_2 \\
d\Bx_3 &= \PD{u_3}{\Bx} du_3 = \Bx_3 du_3.
\end{aligned}
\end{dmath}

The trivector valued volume element for this parameterization is
\begin{equation}\label{eqn:volumeintegral:120}
d^3 \Bx
=
d\Bx_1 \wedge
d\Bx_1 \wedge
d\Bx_1
=
d^3 u\, (\Bx_1 \wedge \Bx_2 \wedge \Bx_3),
\end{equation}
where \( d^3 u = du_1 du_2 du_3 \).
The vector derivative, the projection of the gradient onto the volume at the point of integration (also called the tangent space), now has three components
\begin{dmath}\label{eqn:volumeintegral:200}
\boldpartial
=
\sum_i \Bx^i (\Bx_i \cdot \spacegrad)
=
\Bx^1 \PD{u_1}{}
+
\Bx^2 \PD{u_2}{}
+
\Bx^3 \PD{u_3}{}
\equiv
\Bx^1 \partial_1
+
\Bx^2 \partial_2
+
\Bx^3 \partial_3.
\end{dmath}

The volume integral specialization of \cref{dfn:fundamentalTheoremOfCalculus:240} can now be stated

\makedefinition{Multivector volume integral.}{dfn:volumeintegral:100}{
Given an connected volume \( V \) parameterized by two parameters, and multivector functions \( F, G \), we define the volume integral as
\begin{equation*}
\int_V F d^3\Bx \lrboldpartial G
\equiv
\int_V \lr{ F d^3\Bx \lboldpartial} G
+
\int_V F d^3\Bx \lr{ \rboldpartial G },
\end{equation*}
where the three parameter differential form \( d^3 \Bx = d^3 u\, \Bx_1 \wedge \Bx_2 \wedge \Bx_3, d^3 u = du_1 du_2 du_3 \) varies over the volume, and \( \lrboldpartial \) acts on \( F, G \), but not the volume element \( d^2 \Bx \).
} % definition

The volume integral specialization of \cref{thm:fundamentalTheoremOfCalculus:1} is

\maketheorem{Multivector volume integral.}{thm:volumeintegral:100}{
Given an connected volume \( V \) parameterized by three parameters for which \( d\Bx_1, d\Bx_2, d\Bx_3 \) is a right handed triple, and multivector functions \( F, G \), a volume integral can be reduced to a surface integral as follows
\begin{equation*}
\int_V F d^3\Bx \lrboldpartial G
= \ointctrclockwise_{\partial V} F d^2\Bx G,
\end{equation*}
where \( \partial V \) is the boundary of the volume \( V \), and \( d^2 \Bx \) is the counterclockwise oriented area element on the boundary of the volume.
In \R{3} with \( d^3 \Bx = I dV \), \( d^2 \Bx = I \ncap dA \), this can be written
\begin{equation*}
\int_V dV F \lrboldpartial G
= \int_{\partial V} dA F \ncap G,
\end{equation*}
} % theorem
To see why this works, and define \( d^2 \Bx \) more precisely, we would first like to reduce the product of the volume element and the vector derivative
\begin{dmath}\label{eqn:volumeintegral:300}
d^3\Bx \boldpartial
=
d^3 u\, \lr{ \Bx_1 \wedge \Bx_2 \wedge \Bx_3 } \lr{ \Bx^1 \partial_1 + \Bx^2 \partial_2 + \Bx^3 \partial_3 }.
\end{dmath}

Since all \( \Bx^i \) lie within \( \Span \setlr{ \Bx_1, \Bx_2, \Bx_3 } \), this multivector product has only a vector grade.  That is
\begin{dmath}\label{eqn:volumeintegral:320}
\lr{ \Bx_1 \wedge \Bx_2 \wedge \Bx_3 } \Bx^i
=
\lr{ \Bx_1 \wedge \Bx_2 \wedge \Bx_3 } \cdot \Bx^i
+
\cancel{ \lr{ \Bx_1 \wedge \Bx_2 \wedge \Bx_3 } \wedge \Bx^i },
\end{dmath}
for all \( \Bx^i \).  These products reduces to
\begin{dmath}\label{eqn:volumeintegral:1621}
\begin{aligned}
\lr{ \Bx_2 \wedge \Bx_3 \wedge \Bx_1 } \Bx^1 &= \Bx_2 \wedge \Bx_3 \\
\lr{ \Bx_3 \wedge \Bx_1 \wedge \Bx_2 } \Bx^2 &= \Bx_3 \wedge \Bx_1 \\
\lr{ \Bx_1 \wedge \Bx_2 \wedge \Bx_3 } \Bx^3 &= \Bx_1 \wedge \Bx_2.
\end{aligned}
\end{dmath}

Inserting \cref{eqn:volumeintegral:1621}
into the volume integral, we find
\begin{dmath}\label{eqn:volumeintegral:380}
\int_V F d^3\Bx \boldpartial G
=
\int_V \lr{ F d^3\Bx \lboldpartial} G
+
\int_V F d^3\Bx \lr{ \rboldpartial G }
=
\int_V d^3 u\, \lr{
   (\partial_1 F) \Bx_2 \wedge \Bx_3 G
   +
   (\partial_2 F) \Bx_3 \wedge \Bx_1 G
   +
   (\partial_3 F) \Bx_1 \wedge \Bx_2 G
}
+
\int_V d^3 u\, \lr{
   F \Bx_2 \wedge \Bx_3 (\partial_1 G)
   +
   F \Bx_3 \wedge \Bx_1 (\partial_2 G)
   +
   F \Bx_1 \wedge \Bx_2 (\partial_3 G)
}
=
\int_V d^3 u\, \lr{
   \partial_1 (F \Bx_2 \wedge \Bx_3 G)
   +
   \partial_2 (F \Bx_3 \wedge \Bx_1 G)
   +
   \partial_3 (F \Bx_1 \wedge \Bx_2 G)
}
-
\int_V d^3 u\, \lr{
   F (\partial_1 (\Bx_2 \wedge \Bx_3)) G
   +
   F (\partial_2 (\Bx_3 \wedge \Bx_1)) G
   +
   F (\partial_3 (\Bx_1 \wedge \Bx_2)) G
}
=
\int_V d^3 u\, \lr{
   \partial_1 (F \Bx_2 \wedge \Bx_3 G)
   +
   \partial_2 (F \Bx_3 \wedge \Bx_1 G)
   +
   \partial_3 (F \Bx_1 \wedge \Bx_2 G)
}
-
\int_V d^3 u\, F
\lr{
   \partial_1 (\Bx_2 \wedge \Bx_3)
   +
   \partial_2 (\Bx_3 \wedge \Bx_1)
   +
   \partial_3 (\Bx_1 \wedge \Bx_2)
}
G
.
\end{dmath}

The sum within the second integral is
\begin{dmath}\label{eqn:volumeintegral:1111}
\begin{aligned}
\sum_{i = 1}^3 \partial_i \lr{ I_k \cdot \Bx^i }
&=
\partial_3 \lr{ (\Bx_1 \wedge \Bx_2 \wedge \Bx_3) \cdot \Bx^3 }
+
\partial_1 \lr{ (\Bx_2 \wedge \Bx_3 \wedge \Bx_1) \cdot \Bx^1 }
+
\partial_2 \lr{ (\Bx_3 \wedge \Bx_1 \wedge \Bx_2) \cdot \Bx^2 } \\
&=
\partial_3 \lr{ \Bx_1 \wedge \Bx_2 }
+
\partial_1 \lr{ \Bx_2 \wedge \Bx_3 }
+
\partial_2 \lr{ \Bx_3 \wedge \Bx_1 } \\
&=
         (\partial_3 \Bx_1) \wedge \Bx_2 + \Bx_1 \wedge (\partial_3 \Bx_2) \\
&\quad + (\partial_1 \Bx_2) \wedge \Bx_3 + \Bx_2 \wedge (\partial_1 \Bx_3) \\
&\quad + (\partial_2 \Bx_3) \wedge \Bx_1 + \Bx_3 \wedge (\partial_2 \Bx_1) \\
&=
\Bx_2 \wedge \lr{ - \partial_3 \Bx_1 + \partial_1 \Bx_3 }
+
\Bx_3 \wedge \lr{ - \partial_1 \Bx_2 + \partial_2 \Bx_1 }
+
\Bx_1 \wedge \lr{ - \partial_2 \Bx_3 + \partial_3 \Bx_2 } \\
&=
\Bx_2 \wedge \lr{ - \frac{\partial^2 \Bx}{\partial_3 \partial_1} + \frac{\partial^2 \Bx}{\partial_1 \partial_3} }
+
\Bx_3 \wedge \lr{ - \frac{\partial^2 \Bx}{\partial_1 \partial_2} + \frac{\partial^2 \Bx}{\partial_2 \partial_1} }
+
\Bx_1 \wedge \lr{ - \frac{\partial^2 \Bx}{\partial_2 \partial_3} + \frac{\partial^2 \Bx}{\partial_3 \partial_2} },
\end{aligned}
\end{dmath}
which is zero by equality of mixed partials.
This leaves three perfect differentials, which can integrated separately, giving
\begin{dmath}\label{eqn:volumeintegral:400}
\int_V F d^3\Bx \boldpartial G
=
\int du_2 du_3
\evalbar{ \lr{ F \Bx_2 \wedge \Bx_3 G } }{\Delta u_1}
+
\int du_3 du_1
\evalbar{ \lr{ F \Bx_3 \wedge \Bx_1 G } }{\Delta u_2}
+
\int du_1 du_2
\evalbar{ \lr{ F \Bx_1 \wedge \Bx_2 G } }{\Delta u_3}
=
\int
\evalbar{ \lr{ F d\Bx_2 \wedge d\Bx_3 G } }{\Delta u_1}
+
\int
\evalbar{ \lr{ F d\Bx_3 \wedge d\Bx_1 G } }{\Delta u_2}
+
\int
\evalbar{ \lr{ F d\Bx_1 \wedge d\Bx_2 G } }{\Delta u_3}.
\end{dmath}

This proves the theorem from an algebraic point of view.
With the aid of a geometrical model, such as that of \cref{fig:differentialVolume:differentialVolumeFig}, if
assuming that \( d\Bx_1, d\Bx_2, d\Bx_3 \) is a right handed triple).
it is possible to convince oneself that the two parameter integrands describe an integral over a counterclockwise oriented surface (
\imageTwoFigures{../figures/GAelectrodynamics/differentialVolumeFig1}{../figures/GAelectrodynamics/differentialVolumeFig2}{Differential surface of a volume.}{fig:differentialVolume:differentialVolumeFig}{scale=0.05}

We obtain the RHS of \cref{thm:volumeintegral:100} if we
introduce a mnemonic for the bounding oriented surface of the volume
\begin{dmath}\label{eqn:volumeintegral:1641}
d^2 \Bx \equiv d\Bx_1 \wedge d\Bx_2 + d\Bx_2 \wedge d\Bx_3 + d\Bx_3 \wedge d\Bx_1,
\end{dmath}
where it is implied that each component of this area element and anything that it is multiplied with is evaluated on the boundaries of the integration volume (for the parameter omitted) as detailed explicitly in
\cref{eqn:volumeintegral:400}.

\subsection{Three parameter Stokes' theorem.}

Three special cases of \cref{thm:volumeintegral:100} can be obtained by integrating scalar, vector or bivector functions over the volume, as follows

\maketheorem{Volume integral of scalar function (Stokes').}{thm:volumeintegral:420}{
Given a scalar function \( f(\Bx) \) its volume integral is given by
\begin{equation*}
\int_V d^3 \Bx \cdot \boldpartial f =
\int_V d^3 \Bx \cdot \spacegrad f = \ointctrclockwise_{\partial V} d^2\Bx f.
\end{equation*}
In \R{3} this can be written as
\begin{equation*}
\int_V dV \spacegrad f = \int_{\partial V} dA \ncap f
\end{equation*}
where \( \ncap \) is the outwards normal specified by \( d^2 \Bx = I \ncap dA, \) and \( d^3 \Bx = I dV \).
} % theorem

\maketheorem{Volume integral of vector function (Stokes').}{thm:volumeintegral:1661}{
Given a vector function \( \Bf(\Bx) \) the volume
integral of the (bivector) curl is related to a surface integral by
\begin{equation*}
\int_V d^3 \Bx \cdot \lr{ \boldpartial \wedge \Bf } =
\int_V d^3 \Bx \cdot \lr{ \spacegrad \wedge \Bf } = \ointctrclockwise_{\partial V} d^2\Bx \cdot \Bf.
\end{equation*}
In \R{3} this can be written as
\begin{equation*}
\int_V dV \spacegrad \cross \Bf = \int_{\partial V} dA \ncap \cross \Bf,
\end{equation*}
or with a duality transformation \( \Bf = I B \), where \( B \) is a bivector
\begin{equation*}
\int_V dV \spacegrad \cdot B = \int_{\partial V} dA \ncap \cdot \Bf,
\end{equation*}
where \( \ncap \) is the outwards normal specified by \( d^2 \Bx = I \ncap dA, \) and \( d^3 \Bx = I dV \).
} % theorem

\maketheorem{Volume integral of bivector function (Stokes', divergence).}{thm:volumeintegral:1681}{
Given a bivector function \( B(\Bx) \), the volume
integral of the (trivector) curl is related to a surface integral by
\begin{equation*}
\int_V d^3 \Bx \cdot \lr{ \boldpartial \wedge B } =
\int_V d^3 \Bx \cdot \lr{ \spacegrad \wedge B } = \ointctrclockwise_{\partial V} d^2\Bx \cdot B.
\end{equation*}
In \R{3} this can be written as
\begin{equation*}
\int_V dV \spacegrad \wedge B = \int_{\partial V} dA \ncap \wedge B,
\end{equation*}
or, making a duality transformation \( B(\Bx) = I \Bf(\Bx) \), where \( \Bf \) is a vector, by
\begin{equation*}
\int_V dV \spacegrad \cdot \Bf = \int_{\partial V} dA \ncap \cdot \Bf,
\end{equation*}
where \( \ncap \) is the outwards normal specified by \( d^2 \Bx = I \ncap dA, \) and \( d^3 \Bx = I dV \).
} % theorem

\subsection{Divergence theorem.}

Observe that for \R{3} we there are dot product relations in each of
\cref{thm:volumeintegral:420},
\cref{thm:volumeintegral:1661} and
\cref{thm:volumeintegral:1681} which can be summarized as
\index{divergence theorem}
\maketheorem{Divergence theorem.}{thm:volumeintegral:2661}{
Given an \R{3} multivector \( M \) containing only grades 0,1, or 2
\begin{equation*}
\int_V dV \spacegrad \cdot M = \int_{\partial V} dA \ncap \cdot M,
\end{equation*}
where \( \ncap \) is the outwards normal to the surface bounding \( V \).
} % theorem

%}
%\EndNoBibArticle



%}
\EndArticle
%\EndNoBibArticle
