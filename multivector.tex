\section{Notation and prerequisite definitions.}

\makedefinition{Vector space.}{def:multiplication:vectorspace}{
A vector space is a set \( V \), the elements of which are called vectors.
For all vectors \( \Bx, \By, \Bz \in V \) and scalars \( a, b \in \bbR \),
the addition and multiplication operations must satisfy the following axioms

\begin{tcolorbox}[tab2,tabularx={X|Y},title=Vector space axioms.,boxrule=0.5pt]
    Addition is closed. & \( \Bx + \By \in V \) \\ \hline
    (Scalar) multiplication is closed. & \( a \Bx \in V \) \\ \hline
    Addition is associative. & \( (\Bx + \By) + \Bz = \Bx + (\By + \Bz) \) \\ \hline
    Addition is commutative. & \( \By + \Bx = \Bx + \By \) \\ \hline
    There exists a zero element \( 0 \in V \).  & \( \Bx + 0 = \Bx \) \\ \hline
    For any \( \Bx \in V \) there exists a negative additive inverse \( -\Bx \in V \). & \( \Bx + (-\Bx) = 0 \) \\ \hline
    (Scalar) multiplication is distributive.  & \( a( \Bx + \By ) = a \Bx + a \By \), \( (a + b)\Bx = a \Bx + b\Bx \) \\ \hline
    (Scalar) multiplication is associative. & \( (a b) \Bx = a ( b \Bx ) \) \\ \hline
    There exists a multiplicative identity \( 1 \). & \( 1 \Bx = \Bx \) \\ \hline
\end{tcolorbox}
}

\makedefinition{Dimension}{dfn:multivector:dimension}{
   The dimension of a vector space is a count of the number of normal directions in that space.
} % definition

\makedefinition{Basis and coordinates}{dfn:multivector:basis}{
   If \( N \) is the dimension of a vector space \( V \), a set of \( N \) vectors \( B = \setlr{ \Ba_1, \Ba_2, \cdots \Ba_N } \) is a basis for that vector space, if it is possible to form any vector \( \Bx \in V \) as a linear combination of those vectors \( \Ba_k \).  That is, there exists scalars \( c_k \) such that for any \( \Bx \in V \)

\begin{equation*}
   \Bx = \sum_{k = 1}^N c_k \Ba_k.
\end{equation*}

The numbers \( (c_1, c_2, \cdots, c_N ) \) are referred to as the coordinates of the vector \( \Bx \) with respect to the basis \( B \).
}

It is common to represent a vector \( \Bx \) strictly in terms of that vectors coordinates with respect to a basis, writing
\begin{dmath}\label{eqn:multivector:180}
   \Bx =
\begin{bmatrix}
   c_1 \\
   c_2 \\
   \hdots \\
   c_N \\
\end{bmatrix}.
\end{dmath}

In this book, if coordinates are ever used, they will be used explicitly along with their associated basis.

\makedefinition{Quadratic form}{dfn:multivector:quadraticform}{
   Given a basis \( B \) and coordinates \( c_i \) of a vector \( \Bx \) with respect to that basis, a quadratic form over those coordinates is a scalar mapping

\begin{equation*}
   q(c_1, \cdots, c_N) = \sum_{i,j = 1}^N c_i a_{ij} c_j.
\end{equation*}
}

It is possible to show (Jacobi) that any quadratic form can be reduced to a diagonal form

\begin{equation}\label{eqn:multivector:200}
\sum_{i=1}^N \tilde{c}_i^2 \lambda_i.
\end{equation}

\makedefinition{Norm}{dfn:multivector:norm}{
   For the purposes of this book a norm for elements
   \( \Bx \in V \) of a vector space is a length operations \( \Norm{\Bx} \) that can be reduced to a diagonal
   quadradic form as in \cref{eqn:multivector:200} where all the constants \( \lambda_i \) are \( \pm 1 \).
   Such a norm is called Euclidean if all such \( \lambda_i \) values equal one, and non-Euclidean if that is not the case.

   The only non-Euclidean norm of interest in electromagnetism is the norm for ``spacetime'' vectors, for which
   \( \lambda_1 = \lambda_2 = \lambda_3 = 1 = -\lambda_0 \), or
   \( \lambda_1 = \lambda_2 = \lambda_3 = -1 = -\lambda_0 \) depending on the sign convention in effect.
}

%\begin{enumerate}
%   \item \( \Norm{\Bx} \ge 0 \), and \( \Norm{\Bx} = 0 \iff \Bx = 0 \).
%   \item \( \Norm{a \Bx} = \Abs{a} \Norm{\Bx} \).
%   \item \( \Norm{\Bx + \By} \le \Norm{\Bx} + \Norm{\By} \).
%\end{enumerate}

\makedefinition{Unit vector}{dfn:multivector:unitvector}{
   A vector \( \Bu \), in a Euclidean vector space is called a unit vector if its length is one, \( \Norm{\Bu} = 1 \).
   A vector in a non-Euclidean vector space \( \Bu \) is called a unit vector if its absolute length is one, \( \Norm{\Bu} = \pm 1 \).
} % definition

\makedefinition{Normal}{dfn:multivector:normal}{
   Two Euclidean\footnote{The general definition of normal, also applying to non-Euclidean spaces, is best deferred to the point where the dot product is defined.} vectors \( \Bx, \By \) are normal, or perpendicular, with respect to a length operation \( \Norm{\cdot} \), if the length of a sum or difference has a Pythagorean relationship

\begin{equation*}
   \Norm{\Bx \pm \By}^2 = \Norm{\Bx}^2 + \Norm{\By}^2.
\end{equation*}
}

\makedefinition{Orthonormal}{dfn:multivector:orthonormal}{
   A set of vectors \( \setlr{ \Bx, \By, \cdots \Bz } \) is a an orthonormal set if all pairs of vectors in that set are perpendicular and are also unit vectors.
} % definition

\makedefinition{Standard basis}{dfn:multivector:standardbases}{
   A basis \( \setlr{ \Be_1, \Be_2, \cdots, \Be_N } \), where \( N \) is the dimension of the space is referred to as a standard basis if all pairs \( \Be_i, \Be_j \), \( i \ne j \) are orthonormal.
} % definition

There are many possible standard bases sets.  In \R{3}, it is conventional to refer to \( \Be_1, \Be_2, \Be_3 \) as the standard bases elements if these represent the directions of the x, y, and z directions respectively.  Unless otherwise noted \( \Be_k \) refers to the direction vector for the k-th direction in a standard basis for that space.

\subsection{Problems}
\makeproblem{One dimensional multivector space.}{problem:multivector:30}{
   Given \( f(x) = e^x \) and \( g(x) = x^2 \), and scalars \( a,b \in \bbR \) determine whether the set
   \( V = \setlr{ a f(x) + b g(x) } \) is a vector space.
} % problem

\makeproblem{Pauli matrix vector space.}{problem:multivector:20}{
The Pauli matrices are defined as

\begin{dmath}\label{eqn:multivector:160}
\begin{aligned}
   \sigma_1 &= \PauliX \\
   \sigma_2 &= \PauliY \\
   \sigma_3 &= \PauliZ.
\end{aligned}
\end{dmath}

\makesubproblem{}{problem:multivector:20:a}
Given any scalars \( a, b, c \in \bbR \), show that the set \( V = \setlr{ a \sigma_1 + b \sigma_2 + c \sigma_3 } \) is a vector space, and determine the required form of the unit multiplicative identity element.

\makesubproblem{}{problem:multivector:20:b}

Show that \( \sigma_k^2 = I \), where \( I \) is the 2x2 identity matrix, and that \( \sigma_k \sigma_j = -\sigma_k \sigma_j \) for all \( k \ne j \).

\makesubproblem{}{problem:multivector:20:c}

Using the results of \partref{problem:multivector:20:b}, show that
\( \lr{ a \sigma_x + b \sigma_y + c \sigma_z }^2 = (a^2 + b^2 + c^2) I \), where \( I \) is the 2x2 identity matrix.
%This shows that the Pauli matrices are an example \R{3} basis for which the contraction axiom is built right into the representation.
} % problem

\section{Multivector}
Geometric Algebra (\textAndIndex{GA}) generalizes the concept of vector and a normed vector space, introducing a vector multiplication operation into the mix, and a vector generalization called a \textAndIndex{multivector}.

The multivector is a hybrid object that may contain any sum of all or some of

\begin{enumerate}
   \item scalars, numeric quantities with magnitude and no direction,
   \item vectors, quantities with magnitude and direction,
   \item k-vectors, generalizatized line, area or volume elements, representing subspaces with orientation and magnitude.
\end{enumerate}

Scalars and vectors (1-vectors) are assumed to be familiar, however, a sum of a scalar and vector is a new and arguable strange idea.  The k-vectors with \( k=2 \) and \( k = 3 \) are also called bivectors and trivectors, and represent oriented planes and volumes in space respectively.
Bivectors, trivectors, and k-vectors in general, will be defined later in a more precise fashion, as will orientation.  For now, orientation can be thought of algebraically as a sign, but physically may have an interpretation of sidedness, direction of a normal to the surface\footnote{In three dimensional spaces where a normal can be defined.}, or a rotational sense.

The vector multiplication operation is a new type of vector product.  The vector product is distinct from, but relatied to, the familiar dot or cross products in a way that will become clear.

The algebraic description of a multivector space is very similar to that of the vector space, the definition of which is

Multivectors are built from normed vector spaces as follows

\makedefinition{Multivector space.}{def:multiplication:multivectorspace}{
   Given a normed vector space \( V \), with elements \( \setlr{ \Bx, \By, \cdots, \Bz} \in V \),
   a multivector can be formed from any product of one or more of these vectors \( \Bx \By \cdots \Bz \), a scalar multiple of such a product, or the sum thereof.  These vector products are constrained by the contraction axiom, a requirement that,
   for any vector \( \Bx \in V \) the square of that vector is the squared length of that vector

\begin{equation*}
    \Bx^2 = \Norm{\Bx}^2.
\end{equation*}

A multivector space \( M \) is the set of all possible multivectors that can be formed from the generating vector space \( V \).  All multivectors \( x, y, z \in M \) must satisfy the following additional axioms
\begin{tcolorbox}[tab2,tabularx={X|Y},title=Multivector space axioms.,boxrule=0.5pt]
    Addition is closed. & \( x + y \in M \) \\ \hline
    Multiplication is closed. & \( x y \in M \) \\ \hline
    Addition is associative. & \( (x + y) + z = x + (y + z) \) \\ \hline
    Addition is commutative. & \( y + x = x + y \) \\ \hline
    There exists a zero element \( 0 \in M \).  & \( x + 0 = x \) \\ \hline
    There exists a negative additive inverse \( -x \in M \). & \( x + (-x) = 0 \) \\ \hline
    Multiplication is distributive.  & \( z( x + y ) = z x + z y \), \( (z + w)x = z x + w x \) \\ \hline
    Multiplication is associative. & \( (x y) z = x ( y z ) \) \\ \hline
    There exists a multiplicative identity \( 1 \). & \( 1 x = x \) \\ \hline
\end{tcolorbox}
}

Some work is required to systematically examine the consequences of this abstract definition.

Observe first that since a scalar multiple of the square of a vector is as scalar by the definition above,
any scalar is also a multivector.  For example, if \( \Be_1 \) is the unit vector along the x-axis and \( s \) is a scalar, then

\begin{equation}\label{eqn:multivector:20}
   x = s \Be_1^2 = s,
\end{equation}

is a multivector.  Since vectors (a product of one vector, or a scalar multiple thereof) is also a multivector, this
means that vectors are multivectors, and that ``wierd'' sums of scalars and vectors, such as

\begin{dmath}\label{eqn:multivector:40}
   x = 1 + \Be_1,
\end{dmath}

are also multivectors!  A quantity like

\begin{dmath}\label{eqn:multivector:45}
   x = 1 + \Be_1 + \Be_1 \Be_2 - \Be_1 \Be_2 \Be_3,
\end{dmath}

where \( \Be_k \) are the standard orthonormal basis vectors for \R{3} (unit vectors that are mutually perperpendicular), is also a multivector.  The product \( \Be_1 \Be_2 \) is a bivector, and represents a positively oriented unit magnitude area in the x-y plane, whereas \( - \Be_1 \Be_2 \Be_3 \) is a trivector, representing a negatively oriented unit volume (inwards normals).

\section{Unpacking the axioms}
Now consider some consequences of the contraction axiom.  In particular,
requires the square of a unit (Euclidean) vector \( \Be_i \) to be unity

%\begin{equation}\label{eqn:multiplication:60}
\boxedEquation{eqn:multiplication:60}{
\Be_i^2 = 1.
}
%\end{equation}

With this implication noted, now consider the square of a simple two dimensional vector

\begin{dmath}\label{eqn:gaTutorial:80}
2
=
(\Be_1 + \Be_2)^2
= (\Be_1 + \Be_2)(\Be_1 + \Be_2)
= \Be_1^2 + \Be_2 \Be_1 + \Be_1 \Be_2 + \Be_2^2
= 2 + \Be_2 \Be_1 + \Be_1 \Be_2.
\end{dmath}

The sum above with both scalar terms and terms that are composed of products of vectors is called a multivector.
A product of two perpendicular vectors (or a sum of such products) is called a bivector, and can be used to represent an oriented plane.
Geometric Algebra allows for sums of scalars, vectors, bivectors, and higher degree products.

Observe that for this identity to hold, the bivector terms must sum to zero.  That is

%\begin{dmath}\label{eqn:multiplication:140}
\boxedEquation{eqn:multiplication:140}{
\Be_1 \Be_2 = -\Be_1 \Be_2.
}
%\end{dmath}

This implies that the product of two orthonormal vectors anticommutes.  In general it is also true that

\maketheorem{Normal anticommutation}{thm:multiplication:anticommutationNormal}{
The product of any two normal vectors \(\Bu\), and \(\Bv\) anticommute.
\begin{equation*}
\Bu \Bv = -\Bv \Bu.
\end{equation*}
} % theorem

%


\subsection{Problems}
\makeproblem{One dimensional multivector space.}{problem:multivector:40}{
   Verify that for \( c, d \in \bbR \) the set \( M = \setlr{ c + d \Be_1 } \) satisifies all the multivector axioms.
} % problem

%%\makedefinition{Scalar}{def:multiplication:scalar}{
%%   A (real) number with no implied direction.
%%}
%%
%%Examples of scalars are \( \pi, 3, -4 \), and \( 0 \).
%%
%%\makedefinition{Vector}{def:multiplication:vector}{
%%%\href{https://www.youtube.com/watch?v=bOIe0DIMbI8}{A quantity with direction and magnitude.}
%%\href{https://youtu.be/bOIe0DIMbI8?t=19}{A quantity with direction and magnitude.}
%%}
%%
%%In this book,
%%In order to express
%%\begin{dmath}\label{eqn:multivector:60}
%%\Bx = c_1 \Be_1 + c_2 \Be_2,
%%\end{dmath}
%%
%%where \( \Be_1 \) and \( \Be_2 \) are a pair of perpendicular vectors of length one along the x and y axis respectively, as illustrated in
%%
%%FIXME: figure.
%%These
%%
%%, as represented pictorially as an arrow
%%
%%
%%
%%\section{Vector space}
%%\section{Vector multiplication}
%%\section{Multivector}
%%
%%Geometric Algebra, or \boldTextAndIndex{GA} defines a multiplication operation for vectors.
%%GA also
%%generalizes the concept of a vector, introducing a new type of mathematical object, the multivector.
%%
%%
%%
%%In traditional vector algebra, a sum of a scalar and a vector, such as
%%
%%\begin{dmath}\label{eqn:multivector:80}
%%M = 1 + 2 \Be_1,
%%\end{dmath}
%%
%%is not considered meaningful.  This is
%%
%%Vectors and scalars, or their sums
%%It is assumed here that the student is familiar with coordinate representations of vectors, the concepts of
%%The reader should be familiar with
%%Vectors will be represented algebraically as scaled sums
%%A vector may also be represented algebraically in terms  as the sum of directed
%%or in its algebraic sense, such as the \R{2} vector
%%
%%Vectors, and abstraction representing quantities with magnitude and orientation are types of multivectors.  Scalars (numbers), which have magnitude but no orientation, are also multivectors.  Multivectors that represented oriented areas, oriented volumes, and oriented higher dimensional spaces will also be defined.  The
%%
%%.  Sets of multivectors can be assembled
%%
%%s and multivector spaces, and introduces a multiplication operation for vectors
%%
%%, is built around the \boldTextAndIndex{multivector}.  The purpose of this section is to provide a definition of the multivector, and
