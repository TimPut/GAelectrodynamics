%\chapter{Multivector spaces}
\section{Multivectors}

\subsection{Multivector space.}

The first lesson that must be learned in the study in GA, is to unlearn claims
%\footnote{I heard such claims from high school math and physics teachers.}
that vectors cannot be multiplied.
Instead we start by assuming that a multiplication operation between any number of vectors can be defined,
and that sums, called multivectors, of scalars, vectors, and any products of vectors are also well defined.
These rules are formalized in the definition of a multivector space.

\makedefinition{Multivector space.}{def:multiplication:multivectorspace}{
   Given a (generating) vector space \( V \) with a basis \( \setlr{ \Bx_1, \Bx_2, \cdots } \), a multivector is any sum

   \( a_0 + \sum_i a_i \Bx_i + \sum_{ij} a_{ij} \Bx_i \Bx_j + \sum_{ijk} \Bx_i \Bx_j \Bx_k + \cdots \), where \( a_0, a_i, a_{ij}, \cdots \) are scalars,
and vector multiplication is represented by juxtaposition.

   A multivector space is a set \( M = \setlr{ x, y, z, \cdots } \) of multivectors, where the following axioms are satisfied

\begin{tcolorbox}[tab2,tabularx={X|Y},title=Multivector space axioms.,boxrule=0.5pt]
    Contraction. & \( \Bx^2 = \Bx \cdot \Bx, \forall \Bx \in V \) \\ \hline
    Addition is closed. & \( x + y \in M \) \\ \hline
    Multiplication is closed. & \( x y \in M \) \\ \hline
    Addition is associative. & \( (x + y) + z = x + (y + z) \) \\ \hline
    Addition is commutative. & \( y + x = x + y \) \\ \hline
    There exists a zero element \( 0 \in M \).  & \( x + 0 = x \) \\ \hline
    There exists a negative additive inverse \( -x \in M \). & \( x + (-x) = 0 \) \\ \hline
    Multiplication is distributive.  & \( z( x + y ) = z x + z y \), \( (z + w)x = z x + w x \) \\ \hline
    Multiplication is associative. & \( (x y) z = x ( y z ) \) \\ \hline
    There exists a multiplicative identity \( 1 \). & \( 1 x = x \) \\ \hline
\end{tcolorbox}
}

The definition of a multivector space is strikingly similar to that of the vector space, def'n. \ref{def:prerequisites:vectorspace}, with the following changes

\begin{itemize}
\item In the axioms, any references to vectors are changed to multivectors.
\item Scalar multiplication is generalized to multivector multiplication (of which scalar multiplication is a special case).
\item The vector addition operation is generalized to multivector addition.
\item A vector multiplication operation is presumed.
\item A rule that specifies the meaning of a product of a vector with itself is provided (the contraction axiom).
\end{itemize}

These axioms may seem simple enough, especially since they are not that different from the familiar axioms of the vector space,
but it will take considerable work to extract all their consequences.

The subject of Geometric Algebra can be viewed as the study of the impliciations of the axioms
of the multivector space.

\subsection{Nomenclature: grade, k-vector, bivector, trivector.}

To facilitate discussion, it is useful to first introduce some nomenclature.

\makedefinition{Blade and grade}{def:multiplication:blade}{
A product of \( k \) perpendicular vectors is called a k-blade, or a blade of grade \( k \).
A grade zero blade defined as a scalar.

The shorthand \( F \in \bigwedge^k \), often seen in GA literature, which means that \( F \) is a blade of grade \( k \).
}

The multivector
\begin{dmath}\label{eqn:multiplication:220}
\Be_1 \Be_2 + 3 \Be_1 \Be_3
=
\Be_1 (\Be_2 + 3 \Be_3),
\end{dmath}

which can be factored into two perpendicular vectors is a blade, whereas the grade two multivectors

\begin{dmath}\label{eqn:multiplication:240}
\Be_1 \Be_2 + 3 \Be_3 \Be_4,
\end{dmath}

and
\begin{dmath}\label{eqn:multiplication:260}
\Be_1 \Be_2 + \Be_2 \Be_3 + \Be_3 \Be_1
\end{dmath}

are not.

\makedefinition{K-vector.}{dfn:multivector:kvector}{
A sum of k-blades is called a k-vector.
} % definition

Multivectors are therefore sums of k-vectors with different grades.

The maximum grade of a multivector is equal to the dimension of the generating vector space.
For example, for a multivector space generated by \R{3}, no k-vector can have grade greater than 3.

Examples of k-vectors with grades 0, 1, 2, and 3 respectively are

\begin{dmath}\label{eqn:multivector:180}
\begin{aligned}
&1 \\
&\Be_1, \Be_2, \Be_3 \\
&\Be_1 \Be_2, \Be_2 \Be_1, \Be_1 \Be_2 + \Be_2 \Be_3, \Be_1 \Be_2 + \Be_3 \Be_4 \\
&\Be_1 \Be_2 \Be_3, \Be_1 \Be_3 \Be_2, \Be_1 \Be_4 \Be_2
\end{aligned}
\end{dmath}

K-vectors with grades 2 and 3 are so pervasive that we give them special names.

\makedefinition{Bivector.}{dfn:multivector:bivector}{
A bivector, or 2-vector, is a k-vector with grade 2.
} % definition

The product \( \Be_1 \Be_2 \) is a bivector, as is \( \Be_2 \Be_3 + 3 \Be_4 \Be_1 \)
%Each of \( \Be_1 \Be_2, \Be_2 \Be_1, \Be_1 \Be_2 + \Be_2 \Be_3 \), and \( \Be_1 \Be_2 + \Be_3 \Be_4 \) are bivectors.
%All but the last of these represents an oriented plane segment.

\makedefinition{Trivector.}{dfn:multivector:trivector}{
A trivector, or 3-vector, is a k-vector with grade 3.
} % definition

%Quantities with higher grades than 3 are not generally given explicit names.
The multivector \( \Be_3 \Be_1 \Be_2 \) is a trivector, as is \( \Be_1 \Be_2 \Be_3 + 3 \Be_5 \Be_4 \Be_1 \).  The latter is not a blade.
%Each of \( \Be_1 \Be_2 \Be_3, \Be_1 \Be_3 \Be_2, \Be_1 \Be_4 \Be_2 \) are trivectors.
% , and represent oriented volumes.

\paragraph{Some justification for the contraction axiom.}

%
% Copyright � 2016 Peeter Joot.  All Rights Reserved.
% Licenced as described in the file LICENSE under the root directory of this GIT repository.
%
%{
%%\input{../blogpost.tex}
%%\renewcommand{\basename}{multiplication}
%%%\renewcommand{\dirname}{notes/phy1520/}
%%\renewcommand{\dirname}{notes/ece1228-electromagnetic-theory/}
%%%\newcommand{\dateintitle}{}
%%%\newcommand{\keywords}{}
%%
%%\input{../peeter_prologue_print2.tex}
%%
%%\usepackage{peeters_layout_exercise}
%%\usepackage{peeters_braket}
%%\usepackage{peeters_figures}
%%\usepackage{siunitx}
%%%\usepackage{mhchem} % \ce{}
%%%\usepackage{macros_bm} % \bcM
%%%\usepackage{txfonts} % \ointclockwise
%%
%%\beginArtNoToc
%%
%%\generatetitle{Vector multiplication}
%%%\chapter{Vector multiplication}
%%%\label{chap:multiplication}
%%
Geometric Algebra defines a multiplication operation for vectors, forming a vector space spanned by all the possible vector products.  This algebra is described by the following small set of axioms

\makeaxiom{Associative multiplication.}{axiom:multiplication:associative}{

The product of any three vectors \(\Ba,\Bb,\Bc\) is associative.

\begin{equation*}\label{eqn:multiplication:160}
\Ba (\Bb \Bc)
= (\Ba \Bb) \Bc
= \Ba \Bb \Bc.
\end{equation*}
}

\makeaxiom{Linearity.}{axiom:multiplication:linear}{
Vector products are linear with respect to addition and subtraction.

\begin{dmath*}\label{eqn:multiplication:180}
\begin{aligned}
(\Ba + 3 \Bb \Bd) \Bc &= \Ba \Bb + 3 \Bb \Bd \Bc \\
\Ba (\Bb \Bd - 2 \Bc) &= \Ba \Bb \Bd - 2 \Ba \Bc.
\end{aligned}
\end{dmath*}
}

\makeaxiom{Contraction.}{axiom:multiplication:contraction}{

The square of a vector is the squared length of the vector.

\begin{dmath*}\label{eqn:multiplication:200}
\Ba^2 = \Abs{\Ba}^2.
\end{dmath*}

The notion of length here is metric dependent.  For the problems considered in these notes
it can be assumed that there is an orthonormal Euclidean basis, where the vector length is always positive.
For special relativistic calculations, also of interest in electrodynamics, but not the focus of these notes, the length of a (four-)vector may generally be negative or positive.
}

These axioms are simple enough, but have a rich set of consequences\footnote{Similar to Feynman on gravitation \citep{feynman1963flp} ``... have shall said everything required, for a sufficiently talented mathematician could then deduce all the consequences of these principles.  However, since you are not assumed to be sufficiently talented yet, we shall discuss the consequences in more detail''.}.

The linearity and associativity axioms need little comment, but the contraction property might be surprising.  For one justification of this rule, consider a one dimensional vector space spanned by a single unit vector \( \setlr{ \Be } \).  That span, for real \( x \) is all the values

\begin{dmath}\label{eqn:multiplication:20}
\Bx = x \Be.
\end{dmath}

FIXME: picture to demonstrate the number line isomorphism.

This vector space is isomorphic with a number line, all the possible real values \( x \).
Given a positive number \( x \), the multiplication rules for real numbers require that \( (\pm x)^2 = x^2 \).
The square of a number provides the (squared) length of the number, its distance from the origin.
The same rule can be imposed for one dimensional vectors,
a requirement that the (squared) distance from the origin equals the square of the vector itself.   Such a rule is consistent with the rules of scalar multiplication, and for the
one dimensional vectors of \cref{eqn:multiplication:20} can be stated as

\begin{equation}\label{eqn:multiplication:40}
\Bx^2 = x^2.
\end{equation}

This contraction axiom, justified or not, has additional implications

\begin{dmath}\label{eqn:multiplication:80}
x^2
= \Bx^2
= (x \Be)(x \Be)
= x^2 \Be^2.
\end{dmath}

This rule requires the square of a unit (Euclidean) vector to be unity

%\begin{equation}\label{eqn:multiplication:60}
\boxedEquation{eqn:multiplication:60}{
\Be^2 = 1.
}
%\end{equation}

With this implication noted, now consider the square of a simple two dimensional vector

\begin{dmath}\label{eqn:gaTutorial:80}
2
=
(\Be_1 + \Be_2)^2
= (\Be_1 + \Be_2)(\Be_1 + \Be_2)
= \Be_1^2 + \Be_2 \Be_1 + \Be_1 \Be_2 + \Be_2^2
= 2 + \Be_2 \Be_1 + \Be_1 \Be_2.
\end{dmath}

The sum above with both scalar terms and terms that are composed of products of vectors is called a multivector.
A product of two perpendicular vectors (or a sum of such products) is called a bivector, and can be used to represent an oriented plane.
Geometric Algebra allows for sums of scalars, vectors, bivectors, and higher degree products.

Observe that for this identity to hold, the bivector terms must sum to zero.  That is

%\begin{dmath}\label{eqn:multiplication:140}
\boxedEquation{eqn:multiplication:140}{
\Be_1 \Be_2 = -\Be_1 \Be_2.
}
%\end{dmath}

This implies that the product of two orthonormal vectors anticommutes.  In general it is also true that

\maketheorem{Normal anticommutation}{thm:multiplication:anticommutationNormal}{
The product of any two normal vectors \(\Bu\), and \(\Bv\) anticommute.
\begin{equation*}
\Bu \Bv = -\Bv \Bu.
\end{equation*}
} % theorem

%%%}
%%%\EndArticle
%%\EndNoBibArticle


\subsection{Unpacking the axioms.}

Unless otherwise stated, a Euclidean vector space with an orthonormal basis \( \setlr{\Be_1, \Be_2, \cdots } \) is assumed for the remainder of this chapter.
Generalizations required for non-Euclidean spaces will be discussed when spacetime vectors are introduced.
%  At that point, it is a good exersize for the reader to come back to this, and determine where any result

\paragraph{Effects of commutation of normal vectors.}

One of the consequences of the contraction axiom is that
the square of any unit vector, such as any \( \Be_i \) is also unity

%\begin{equation}\label{eqn:multiplication:300}
\boxedEquation{eqn:multiplication:320}{
\Be_i^2 = 1.
}
%\end{equation}

Now consider the square of the vector with its head at \( (1,1) \), as sketched in \cref{fig:unitSum:unitSumFig1}.
\imageFigure{../figures/GAelectrodynamics/unitSumFig1}{\( \Be_1 + \Be_2 \).}{fig:unitSum:unitSumFig1}{0.3}
By the contraction axiom, the square of this vector is \( 2\), but can also be computed by expansion, and a second application of the contraction axiom

\begin{dmath}\label{eqn:gaTutorial:80}
2
=
(\Be_1 + \Be_2)^2
= (\Be_1 + \Be_2)(\Be_1 + \Be_2)
= \Be_1^2 + \Be_2 \Be_1 + \Be_1 \Be_2 + \Be_2^2
= 2 + \Be_2 \Be_1 + \Be_1 \Be_2.
\end{dmath}

This equation has a scalar component on the left hand side, and a mixed grade multivector with grades zero and two on the right hand side.
The only possible solution requires that the grade two components of this equation are zero.  That is

\begin{dmath}\label{eqn:multivector:280}
\Be_2 \Be_1 + \Be_1 \Be_2 = 0,
\end{dmath}

or
%\begin{dmath}\label{eqn:multiplication:140}
\boxedEquation{eqn:multiplication:140}{
\Be_1 \Be_2 = -\Be_1 \Be_2.
}
%\end{dmath}

We see that this product of two orthonormal vectors anticommutes (changing the order, changes the sign).  In general this is true of any normal vectors.

\maketheorem{Normal anticommutation}{thm:multiplication:anticommutationNormal}{
The product of any two normal vectors \(\Bu\), and \(\Bv\) anticommute.
\begin{equation*}
\Bu \Bv = -\Bv \Bu.
\end{equation*}
} % theorem

\paragraph{Mixed grade sums}
In traditional vector algebra, the
``weird'' sum of a scalar and vector is forbidden and undefined, but is explicitly allowed in GA.  For example,

\begin{dmath}\label{eqn:multivector:240}
1 + \Be_1,
\end{dmath}

is a simple mixed grade multivector.
Such mixed grade mathematical objects are not only well defined in GA, but are required to represent some vector products.  One of the simplest examples is the following vector product

\begin{dmath}\label{eqn:multivector:260}
\Be_1 ( \Be_1 + \Be_2 )
=
\Be_1 \Be_1 + \Be_1 \Be_2
=
\Be_1 \cdot \Be_1 + \Be_1 \Be_2
=
1 + \Be_1 \Be_2,
\end{dmath}

where the last step assumes the vector space is Euclidean.

\subsection{Problems}

\makeproblem{One dimensional multivector space.}{problem:multivector:40}{
   Verify that for \( c, d \in \bbR \) the set \( M = \setlr{ c + d \Be_1 } \) satisifies all the multivector axioms.
} % problem

\makeproblem{Normal anticommutation.}{problem:multiplication:anticommutationNormal}{
Prove \cref{thm:multiplication:anticommutationNormal}.
}

