%
% Copyright © 2017 Peeter Joot.  All Rights Reserved.
% Licenced as described in the file LICENSE under the root directory of this GIT repository.
%
\subsection{Multivector space.}

The first lesson that must be learned in the study in GA, is to unlearn claims
%\footnote{I heard such claims from high school math and physics teachers.}
that vectors cannot be multiplied.
Instead we start by assuming that a multiplication operation between any number of vectors can be defined,
and that sums, called multivectors, of scalars, vectors, and any products of vectors are also well defined.
These rules are formalized in the definition of a multivector space.

\makedefinition{Multivector space.}{def:multiplication:multivectorspace}{
   Given a (generating) vector space \( V \) with a basis \( \setlr{ \Bx_1, \Bx_2, \cdots } \), a multivector is any sum

   \( a_0 + \sum_i a_i \Bx_i + \sum_{ij} a_{ij} \Bx_i \Bx_j + \sum_{ijk} a_{ijk} \Bx_i \Bx_j \Bx_k + \cdots \), where \( a_0, a_i, a_{ij}, \cdots \) are scalars,
and vector multiplication is represented by juxtaposition.

   A multivector space is a set \( M = \setlr{ x, y, z, \cdots } \) of multivectors, where the following axioms are satisfied

\begin{tcolorbox}[tab2,tabularx={X|Y},title=Multivector space axioms.,boxrule=0.5pt]
    Contraction. & \( \Bx^2 = \Bx \cdot \Bx, \forall \Bx \in V \) \\ \hline
    Addition is closed. & \( x + y \in M \) \\ \hline
    Multiplication is closed. & \( x y \in M \) \\ \hline
    Addition is associative. & \( (x + y) + z = x + (y + z) \) \\ \hline
    Addition is commutative. & \( y + x = x + y \) \\ \hline
    There exists a zero element \( 0 \in M \).  & \( x + 0 = x \) \\ \hline
    There exists a negative additive inverse \( -x \in M \). & \( x + (-x) = 0 \) \\ \hline
    Multiplication is distributive.  & \( x( y + z ) = x y + x z \), \( (x + y)z = x z + y z \) \\ \hline
    Multiplication is associative. & \( (x y) z = x ( y z ) \) \\ \hline
    There exists a multiplicative identity \( 1 \). & \( 1 x = x \) \\ \hline
\end{tcolorbox}
}

Compared to the vector space, def'n. \ref{def:prerequisites:vectorspace}, the multivector space

\begin{itemize}
\item presumes a vector multiplication operation, which is not assumed to be commutative (order matters),
\item generalizes vector addition to multivector addition,
\item generalizes scalar multiplication to multivector multiplication (of which scalar multiplication is a special case),
\item and most importantly, specifies a rule providing the meaning of a squared vector (the contraction axiom).
\end{itemize}

The contraction axiom is arguably the most important of the multivector space axioms, since multiplicative closure would not be possible without it.
The remaining set of non-contraction axioms of a multivector space are almost that of a field
\footnote{A mathematician would call a multivector space a non-commutative ring with identity \citep{van1943modern}, and could state the multivector space definition much more compactly without listing all those properties explicitly.}
(as encountered in the study of complex inner products),
as they describe most of the properties one
would expect of a ``well behaved'' set of number-like quantities.
However, a field also requires a multiplicitive inverse element for all elements of the space, which exists for some multivector subspaces, but not in general.

%These axioms may seem simple enough, especially since they are not that different from the familiar axioms of the vector space,
%but it will take considerable work to extract all their consequences.
%The subject of Geometric Algebra can be viewed as the study of the impliciations of the axioms
%of the multivector space.

\subsection{Nomenclature.}

A fair amount of nomenclature and notation is unfortunately required before systematically examining the implications of the multivector space axioms that define geometric algebra.

\makedefinition{Blade and grade}{def:multiplication:blade}{
A product of \( k \) perpendicular vectors is called a k-blade, or a blade of grade \( k \).
A grade zero blade is a scalar.

The notation \( F \in \bigwedge^k \) is used in the literature to indicate that \( F \) is a blade of grade \( k \).
}

The maximum grade of a multivector is equal to the dimension of the generating vector space.
For example, for a multivector space generated by \R{3}, no k-vector can have grade greater than 3.

Examples of blades with grades 0, 1, 2, and 3 respectively are

\begin{dmath}\label{eqn:multivector:180}
\begin{aligned}
&1 \\
&\Be_1,\quad \Be_2,\quad \Be_3 \\
&\Be_1 \Be_2,\quad \Be_2 \Be_1,\quad \Be_1 \Be_2 + \Be_2 \Be_3 \\
&\Be_1 \Be_2 \Be_3,\quad \Be_1 \Be_3 \Be_2,\quad \Be_1 \Be_4 \Be_2
\end{aligned}
\end{dmath}

Multivectors which can be factored into perpendicular vector products, such as
\begin{dmath}\label{eqn:multiplication:220}
\Be_1 \Be_2 + 3 \Be_1 \Be_3
=
\Be_1 (\Be_2 + 3 \Be_3),
\end{dmath}

are blades.  In contrast, the following grade 2 multivectors

\begin{dmath}\label{eqn:multiplication:240}
\Be_1 \Be_2 + \Be_3 \Be_4,
\end{dmath}

and
\begin{dmath}\label{eqn:multiplication:260}
\Be_1 \Be_2 + \Be_2 \Be_3 + \Be_3 \Be_1,
\end{dmath}

which cannot be factored into two vector products, are not blades.

\makedefinition{k-vector.}{dfn:multivector:kvector}{
A sum of k-blades is called a k-vector.
} % definition

Multivectors are therefore sums of k-vectors with different grades.

All the k-blade examples above are also k-vectors.
K-vectors with grades 2 and 3 are so pervasive that they are given special names.

\makedefinition{Bivector.}{dfn:multivector:bivector}{
A bivector, or 2-vector, is a k-vector with grade 2.
} % definition

The product \( \Be_1 \Be_2 \) is a bivector, as is \( \Be_2 \Be_3 + 3 \Be_4 \Be_1 \)
%Each of \( \Be_1 \Be_2, \Be_2 \Be_1, \Be_1 \Be_2 + \Be_2 \Be_3 \), and \( \Be_1 \Be_2 + \Be_3 \Be_4 \) are bivectors.
%All but the last of these represents an oriented plane segment.

\makedefinition{Trivector.}{dfn:multivector:trivector}{
A trivector, or 3-vector, is a k-vector with grade 3.
} % definition

%Quantities with higher grades than 3 are not generally given explicit names.
The multivector \( \Be_3 \Be_1 \Be_2 \) is a trivector, as is \( \Be_1 \Be_2 \Be_3 + 3 \Be_5 \Be_4 \Be_1 \).  The latter is not a blade.
%Each of \( \Be_1 \Be_2 \Be_3, \Be_1 \Be_3 \Be_2, \Be_1 \Be_4 \Be_2 \) are trivectors.
% , and represent oriented volumes.

\makedefinition{Grade selection operator}{dfn:gradeselection:gradeselection}{
Given a set of k-vectors \( M_k, k \in [0,N] \), and any multivector of their sum

\begin{equation*}
M = \sum_{i = 0}^N M_i,
\end{equation*}

the grade selection operator is defined as

\begin{equation*}\label{eqn:gradeselection:40}
\gpgrade{M}{k} \equiv M_k.
\end{equation*}

Due to its importance, selection of the (scalar) zero grade is given the shorthand
\begin{equation*}
\gpgradezero{M} \equiv \gpgrade{M}{0} = M_0.
\end{equation*}
}

For example, if \( M = 3 - \Be_3 + 2 \Be_1 \Be_2 \), then
\begin{equation}\label{eqn:gradeselection:80}
\begin{aligned}
\gpgradezero{M} &= 3 \\
\gpgrade{M}{1} &= - \Be_3 \\
\gpgrade{M}{2} &= 2 \Be_1 \Be_2 \\
\gpgrade{M}{3} &= 0.
\end{aligned}
\end{equation}

\makedefinition{Orthonormal product shorthand.}{dfn:multivector:shorthand}{
Given an orthonormal basis \( \setlr{ \Be_1, \Be_2, \cdots } \), a multiple indexed quantity \( \Be_{ij\cdots k} \) should be interpretted as the product (in the same order) of the basis elements with those indexes

\begin{equation*}
\Be_{ij\cdots k} = \Be_i \Be_j \cdots \Be_k.
\end{equation*}
} % definition

For example,

\begin{equation}\label{eqn:multivector:360}
\begin{aligned}
\Be_{12} &= \Be_1 \Be_2 \\
\Be_{123} &= \Be_1 \Be_2 \Be_3 \\
\Be_{23121} &= \Be_2 \Be_3 \Be_1 \Be_2 \Be_1.
\end{aligned}
\end{equation}

\makedefinition{Pseudoscalar.}{def:multiplication:pseudoscalar}{
A blade with grade that matches the dimension of the space.
}

In a two dimensional space \( \Be_1 \Be_2 \) is a pseudoscalar, as is \( 3 \Be_2 \Be_1 \).  In a three dimensional space
\( \Be_3 \Be_1 \Be_2 \) is a pseudoscalar, as is \( - 7 \Be_3 \Be_1 \Be_2 \).
%A pseudoscalar has an implied orientation, which can be
%associated with the handedness of the underlying basis.
It is conventional to refer to

\begin{dmath}\label{eqn:definitions:320}
i = \Be_1 \Be_2,
\end{dmath}

as ``the pseudoscalar'' for a two dimensional space, and to

\begin{dmath}\label{eqn:definitions:340}
I = \Be_1 \Be_2 \Be_3,
\end{dmath}

as ``the pseudoscalar'' for a three dimensional space.


