%
% Copyright © 2017 Peeter Joot.  All Rights Reserved.
% Licenced as described in the file LICENSE under the root directory of this GIT repository.
%
%{
Geometric algebra takes a vector space and adds two additional operations, a vector multiplication operation, and a generalized addition operation that extends vector addition to include addition of scalars and products of vectors.
Multiplication of vectors is indicated by juxtaposition, for example, if \( \Bx, \By, \Be_1, \Be_2, \Be_3, \cdots \) are vectors, then some vector products are

\begin{dmath}\label{eqn:multivector:20}
\begin{aligned}
&\Bx \By, \Bx \By \Bx, \Bx \By \Bx \By, \\
&\Be_1 \Be_2, \Be_2 \Be_1, \Be_2 \Be_3, \Be_3 \Be_2, \Be_3 \Be_1, \Be_1 \Be_3, \\
&\Be_1 \Be_2 \Be_3, \Be_3 \Be_1 \Be_2, \Be_2 \Be_3 \Be_1, \Be_3 \Be_2 \Be_1, \Be_2 \Be_1 \Be_3, \Be_1 \Be_3 \Be_2, \\
&\Be_1 \Be_2 \Be_3 \Be_1, \Be_1 \Be_2 \Be_3 \Be_1 \Be_3 \Be_2, \cdots
\end{aligned}
\end{dmath}

Vector multiplication is constrained by a rule, the contraction axiom, which specifies that the square of vector is the squared length of that vector (i.e. a scalar).

In a sum of scalars, vectors, and vector products, such as
\begin{dmath}\label{eqn:multivector:40}
1 + 2 \Be_1 + 3 \Be_1 \Be_2 + 4 \Be_1 \Be_2 \Be_3,
\end{dmath}
the value \( 1 \) is a scalar (or 0-vector), \( 2 \Be_1 \) is a vector (or 1-vector),
\( 3 \Be_1 \Be_2 \) is called a bivector (or 2-vector), \( 4 \Be_1 \Be_2 \Be_3 \) is called a trivector (or 3-vector), and the sum itself is called a multivector.
Geometric algebra uses vector multiplication to build up a hierarchy of geometrical objects, representing oriented points, lines, planes, volumes and in higher dimensional spaces oriented hypervolumes.

\index{scalar}
\index{0-vector}
\paragraph{Scalar.}
A scalar, which we will also call a 0-vector, is a zero-dimensional object with sign, and a magnitude.
The sign of a scalar can be represented graphically as an arrow with a head and a tail pointing into the paper (or chalkboard),
as illustrated in
\cref{fig:scalarOrientation:scalarOrientationFig1} where a crossed circle represents the tail, and a solid dot represents the head.
%\footnote{We don't usually try to represent the magnitude of a scalar graphically, but could do so by scaling the size of the cross or dot.}
\imageFigure{../figures/GAelectrodynamics/scalarOrientationFig1}{Scalar illustration.}{fig:scalarOrientation:scalarOrientationFig1}{0.05}

\index{vector}
\index{1-vector}
\paragraph{Vector.}
A vector, which we will also call a 1-vector, is a one-dimensional object with a sign, a magnitude, and possible orientation, as illustrated in
\cref{fig:VectorsWithOppositeOrientation:VectorsWithOppositeOrientationFig1}, where the length of the line segment represents the magnitude, and
the sign of the vector can be represented graphically using the relative placement of the head vs. tail of the vector.
\imageFigure{../figures/GAelectrodynamics/VectorsWithOppositeOrientationFig1}{Vector illustration.}{fig:VectorsWithOppositeOrientation:VectorsWithOppositeOrientationFig1}{0.3}
One vector in isolation is a one dimensional object, however, when embedded in a higher order space, such as a plane or a volume, it also has an orientation in that space.

Vectors can be added graphically by connecting them head to tail in sequence, and joining the first tail point to the final head, as
illustrated in
\cref{fig:vectorAddition:vectorAdditionFig1}.
\imageFigure{../figures/GAelectrodynamics/vectorAdditionFig1}{Graphical vector addition.}{fig:vectorAddition:vectorAdditionFig1}{0.3}

\index{bivector}
\index{2-vector}
\paragraph{Bivector.}

We now wish to define a bivector, or 2-vector, as a 2 dimensional object representing a signed plane segment with magnitude and orientation.
%\footnote{This physical (plane segment) representation can break down when the underlying vector space has greater than three dimensions.}.
Values such as torque and angular momentum, which are usually represented as cross products, can be represented as bivectors lying in the plane of the torque or angular momentum instead of normal to that plane.  A bivector representation can eliminate the requirement utilize a third (normal) dimension, which may not be relevant in the problem, and can allow some concepts to be generalized to dimensions higher than three when desirable.

Assuming a vector product, with properties to be defined, we can algebraically define a bivector as

\makedefinition{Bivector.}{dfn:multivector:60}{
A bivector, or 2-vector, is a sum of products of pairs of normal vectors.
Given an \( N \) dimensional vector space \( V \) with an orthonormal basis \( \setlr{ \Be_1, \Be_2, \cdots } \),
a general bivector can be expressed as
\begin{equation*}
\sum_{1 \le i \ne j \le N} B_{ij} \Be_i \Be_j,
\end{equation*}
where \( B_{ij} \) is a scalar.
The vector basis \( V \) is said to be a generator of a bivector space.
} % definition

In the sum above, the rationale for excluding products for which \( i = j \) must be deferred until the properties of the vector product are detailed (coming soon).
An oriented plane segment can always be represented by such an algebraic object in any number of dimensions\footnote{When the generating vector space has dimension \( N \ge 4 \) not all bivectors defined by \cref{dfn:multivector:60} necessarily represent oriented plane segments.
The restrictions required for a bivector to have an associated oriented plane segment interpretation in higher dimensional spaces will be defined later.}.
%We will see later that the set \( \Span \setlr{ \Be_i \Be_j \mid i < j } \), can be considered a linearly independent basis for a bivector space, and that
In three dimensions, the sign, or ``sided-ness'' of a bivector can be represented using a vector normal to the plane.
In 2 or \( N > 3 \) dimensions, where a normal
direction may not be meaningful, we can represent the bivector sign graphically using a cyclic direction on the surface of the plane segment as illustrated in
\cref{fig:orientedAreas:orientedAreasFig1}.
Other than having a boundary that defines the total area, a graphical bivector representation as a plane segment need not have any specific geometry.
Two bivectors represented graphically as oriented circles in three dimensional space can be found in
\cref{fig:circularBivectorsIn3D:circularBivectorsIn3DFig1}.
\imageFigure{../figures/GAelectrodynamics/orientedAreasFig1}{Oriented unit areas in the x-y plane.}{fig:orientedAreas:orientedAreasFig1}{0.2}
\imageFigure{../figures/GAelectrodynamics/circularBivectorsIn3DFig1}{Circular representation of two bivectors.}{fig:circularBivectorsIn3D:circularBivectorsIn3DFig1}{0.3}

With vectors, addition is performed by connecting vectors head to tail, which maintains the orientation.
The same can be done with bivectors, where the bivectors are also connected with compatible orientation to construct a sum.
This is illustrated graphically in \cref{fig:AdditionOfBivectors:AdditionOfBivectorsFig1}, where a blue bivector with a right handed orientation is added to a red bivector with right handed orientation, to form a green bivector also with right handed orientation, where all orientations are with respect to the exterior of the bounding surface formed by the three bivectors.
\imageFigure{../figures/GAelectrodynamics/AdditionOfBivectorsFig1}{Bivector addition.}{fig:AdditionOfBivectors:AdditionOfBivectorsFig1}{0.3}

%We will see that the products of normal vectors, like \( \Be_1 \Be_2 \) anticommute\footnote{Quantities that anticommute are unchanged if both the order and the sign are toggled.},
%for example \( \Be_2 \Be_1 = -\Be_1 \Be_2 \).
%This means that many of the products in \cref{eqn:multivector:20} are not independent, and that the definition of a bivector could be a more restrictive sum, such as \( \sum_{1 \le i < j \le N} b_{ij} \Be_i \Be_j \), where \( b_{ij} \) is an antisymmetric
%\footnote{An indexed quantity such as \( b_{ij} \) is antisymmetric if toggling the order of indexes changes the sign, that is \( b_{ji} = -b_{ij} \).}
%scalar.
%

\index{trivector}
\index{3-vector}
\paragraph{Trivector.}

Again, assuming a vector product

\makedefinition{Trivector.}{dfn:multivector:80}{
A trivector, or 3-vector, is a sum of products of triplets of mutually normal vectors.
Given an \( N \) dimensional vector space \( V \) with an orthonormal basis \( \setlr{ \Be_1, \Be_2, \cdots } \),
a general trivector can be expressed as
\begin{equation*}
\sum_{1 \le i \ne j \ne k \le N} T_{ijk} \Be_i \Be_j \Be_k,
\end{equation*}
where \( \T_{ijk} \) is a scalar.
The vector space \( V \) is said to generate a trivector space.
} % definition

In three dimensional space, we will see that all trivectors are scalar multiples of \( \Be_1 \Be_2 \Be_3 \), and
can represent an oriented volume segment such as the differential form in a volume integral.
This orientation can be visualized with a normal pointing into or out of the volume, or like bivectors, with a cyclic direction on the surface of the volume as in illustrated with the spherical volume of \cref{fig:orientedVolume:orientedVolumeFig1}.
\imageFigure{../figures/GAelectrodynamics/orientedVolumeFig1}{Oriented Volume}{fig:orientedVolume:orientedVolumeFig1}{0.3}
%In greater than three dimensions, a trivector can have a ``direction'' in the higher dimensional space, as well as a sidedness.
%As was the case with the bivector, because not all the products \( \Be_i \Be_j \Be_k \) for any set of indexes \( i, j, k \) are independent, it is possible to form a trivector as a sum over a more restricted set, such as \( \sum_{1 \le i < j < k \le N} T_{ijk} \Be_i \Be_j \Be_k \).
%In particular, in three dimensions, all trivectors can be expressed as scalar multiples of \( \Be_1 \Be_2 \Be_3 \).
%
\index{k-vector}
\index{grade}
\paragraph{K-vector.}
\makedefinition{K-vector and grade.}{dfn:multivector:100}{
A k-vector is a sum of products of \( k \) mutually normal vectors.
Given an \( N \) dimensional vector space with an orthonormal basis \( \setlr{ \Be_1, \Be_2, \cdots } \),
a general k-vector can be expressed as
\begin{equation*}
\sum_{1 \le i_1 \ne i_2 \cdots \ne i_k \le N} K_{i_1 i_2 \cdots i_k} \Be_{i_1} \Be_{i_2} \cdots \Be_{i_k},
\end{equation*}
where \( K_{i_1 i_2 \cdots i_k} \) is a scalar.

The number \( k \) of normal vectors that generate a k-vector is called the grade.

A 1-vector is defined as a vector, and a 0-vector is defined as a scalar.

The vector space \( V \) is said to generate the k-vector space.
} % definition

We will see that the highest grade for a k-vector in an N dimensional vector space is \( N \).

\index{multivector}
\index{multivector space}
\paragraph{Multivector space.}
\makedefinition{Multivector space.}{def:multiplication:multivectorspace}{
   Given an N dimensional (generating) vector space \( V \) with an orthonormal basis \( \setlr{ \Be_1, \Be_2, \cdots, \Be_N } \),
%a basis \( \setlr{ \Bx_1, \Bx_2, \cdots } \),
and a vector multiplication operation represented by juxtaposition,
a multivector is a sum of k-vectors, \( k \in [ 1, N ] \), such as
   \( a_0 + \sum_i a_i \Be_i + \sum_{i \ne j} a_{ij} \Be_i \Be_j + \sum_{i \ne j \ne k} a_{ijk} \Be_i \Be_j \Be_k + \cdots \), where \( a_0, a_i, a_{ij}, \cdots \) are scalars.

The multivector space generated by \( V \) is a set \( M = \setlr{ x, y, z, \cdots } \) of multivectors, where the following axioms are satisfied

\begin{tcolorbox}[tab2,tabularx={X|Y},title=Multivector space axioms.,boxrule=0.5pt]
    Contraction. & \( \Bx^2 = \Bx \cdot \Bx, \,\forall \Bx \in V \) \\ \hline
    Addition is closed. & \( x + y \in M \) \\ \hline
    Multiplication is closed. & \( x y \in M \) \\ \hline
    Addition is associative. & \( (x + y) + z = x + (y + z) \) \\ \hline
    Addition is commutative. & \( y + x = x + y \) \\ \hline
    There exists a zero element \( 0 \in M \).  & \( x + 0 = x \) \\ \hline
    There exists a negative additive inverse \( -x \in M \). & \( x + (-x) = 0 \) \\ \hline
    Multiplication is distributive.  & \( x( y + z ) = x y + x z \), \( (x + y)z = x z + y z \) \\ \hline
    Multiplication is associative. & \( (x y) z = x ( y z ) \) \\ \hline
    There exists a multiplicative identity \( 1 \). & \( 1 x = x \) \\ \hline
\end{tcolorbox}
}

Compared to the vector space, def'n. \ref{def:prerequisites:vectorspace}, the multivector space

\begin{itemize}
\item presumes a vector multiplication operation, which is not assumed to be commutative (order matters),
\item generalizes vector addition to multivector addition,
\item generalizes scalar multiplication to multivector multiplication (of which scalar multiplication and vector multiplication are special cases),
\item and most importantly, specifies a rule providing the meaning of a squared vector (the contraction axiom).
\end{itemize}

The contraction axiom is arguably the most important of the multivector space axioms, as it allows for multiplicative closure without an infinite dimensional multivector space.
The remaining set of non-contraction axioms of a multivector space are almost that of a field
\footnote{A mathematician would call a multivector space a non-commutative ring with identity \citep{van1943modern}, and could state the multivector space definition much more compactly without listing all the properties of a ring explicitly as done above.}
(as encountered in the study of complex inner products),
as they describe most of the properties one
would expect of a ``well behaved'' set of number-like quantities.
However, a field also requires a multiplicative inverse element for all elements of the space, which exists for some multivector subspaces, but not in general.

%These axioms may seem simple enough, especially since they are not that different from the familiar axioms of the vector space,
%but it will take considerable work to extract all their consequences.
%The subject of Geometric Algebra can be viewed as the study of the impliciations of the axioms
%of the multivector space.

%}
