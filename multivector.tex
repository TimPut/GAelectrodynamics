%
% Copyright © 2017 Peeter Joot.  All Rights Reserved.
% Licenced as described in the file LICENSE under the root directory of this GIT repository.
%
The first lesson that must be learned in the study in GA, is to unlearn claims
%\footnote{I heard such claims from high school math and physics teachers.}
that vectors cannot be multiplied.
Instead we start by assuming that a multiplication operation between any number of vectors can be defined,
and that sums, called multivectors, of scalars, vectors, and any products of vectors are also well defined.
These rules are formalized in the definition of a multivector space.

\index{multivector space}
\makedefinition{Multivector space.}{def:multiplication:multivectorspace}{
   Given a (generating) vector space \( V \) with a basis \( \setlr{ \Bx_1, \Bx_2, \cdots } \), a multivector is any sum

   \( a_0 + \sum_i a_i \Bx_i + \sum_{ij} a_{ij} \Bx_i \Bx_j + \sum_{ijk} a_{ijk} \Bx_i \Bx_j \Bx_k + \cdots \), where \( a_0, a_i, a_{ij}, \cdots \) are scalars,
and vector multiplication is represented by juxtaposition.

   A multivector space is a set \( M = \setlr{ x, y, z, \cdots } \) of multivectors, where the following axioms are satisfied

\begin{tcolorbox}[tab2,tabularx={X|Y},title=Multivector space axioms.,boxrule=0.5pt]
    Contraction. & \( \Bx^2 = \Bx \cdot \Bx, \forall \Bx \in V \) \\ \hline
    Addition is closed. & \( x + y \in M \) \\ \hline
    Multiplication is closed. & \( x y \in M \) \\ \hline
    Addition is associative. & \( (x + y) + z = x + (y + z) \) \\ \hline
    Addition is commutative. & \( y + x = x + y \) \\ \hline
    There exists a zero element \( 0 \in M \).  & \( x + 0 = x \) \\ \hline
    There exists a negative additive inverse \( -x \in M \). & \( x + (-x) = 0 \) \\ \hline
    Multiplication is distributive.  & \( x( y + z ) = x y + x z \), \( (x + y)z = x z + y z \) \\ \hline
    Multiplication is associative. & \( (x y) z = x ( y z ) \) \\ \hline
    There exists a multiplicative identity \( 1 \). & \( 1 x = x \) \\ \hline
\end{tcolorbox}
}

Compared to the vector space, def'n. \ref{def:prerequisites:vectorspace}, the multivector space

\begin{itemize}
\item presumes a vector multiplication operation, which is not assumed to be commutative (order matters),
\item generalizes vector addition to multivector addition,
\item generalizes scalar multiplication to multivector multiplication (of which scalar multiplication and vector multiplication are special cases),
\item and most importantly, specifies a rule providing the meaning of a squared vector (the contraction axiom).
\end{itemize}

The contraction axiom is arguably the most important of the multivector space axioms, since multiplicative closure would not be possible without it.
The remaining set of non-contraction axioms of a multivector space are almost that of a field
\footnote{A mathematician would call a multivector space a non-commutative ring with identity \citep{van1943modern}, and could state the multivector space definition much more compactly without listing all the properties of a ring explicitly as done above.}
(as encountered in the study of complex inner products),
as they describe most of the properties one
would expect of a ``well behaved'' set of number-like quantities.
However, a field also requires a multiplicitive inverse element for all elements of the space, which exists for some multivector subspaces, but not in general.

%These axioms may seem simple enough, especially since they are not that different from the familiar axioms of the vector space,
%but it will take considerable work to extract all their consequences.
%The subject of Geometric Algebra can be viewed as the study of the impliciations of the axioms
%of the multivector space.

