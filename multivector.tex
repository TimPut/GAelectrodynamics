\section{Multivectors}
Geometric Algebra (\textAndIndex{GA}) generalizes the concept of vector and a normed vector space.  This is done by introducing a vector multiplication operation into the mix, and a vector generalization called a \textAndIndex{multivector}.

The multivector is a hybrid object that may contain any sum of all or some of

\begin{enumerate}
   \item scalars, numeric quantities with magnitude and no direction,
   \item vectors (1-vectors), quantities with magnitude and direction,
   \item k-vectors, which are generalizatized line, area, volume and hypervolume elements, represent subspaces with orientation and magnitude.
\end{enumerate}

Scalars and vectors are assumed to be familiar, however, a sum of a scalar and vector is a new and arguably strange idea.  The k-vectors with \( k=2 \) and \( k = 3 \) are called bivectors and trivectors, and represent oriented planes and volumes in space respectively.
Bivectors, trivectors, and k-vectors will be defined later in a more precise fashion, as will orientation.  For now, orientation can be thought of algebraically as a sign, but physically may have an interpretation of sidedness, direction of a normal to the surface\footnote{In three dimensional spaces where a normal can be defined.}, or a rotational sense.

FIXME: orientation pictures here.

The vector multiplication operation is a new type of vector product.  The vector product is distinct from, but relatied to, the familiar dot or cross products in a way that will become clear.

The algebraic description of a multivector space is very similar to that of the vector space, the definition of which is

Multivectors are built from normed vector spaces as follows

\makedefinition{Multivector space.}{def:multiplication:multivectorspace}{
   Given a normed vector space \( V \), with elements \( \setlr{ \Bx, \By, \cdots, \Bz} \in V \),
   a multivector can be formed from any product of one or more of these vectors \( \Bx \By \cdots \Bz \), a scalar multiple of such a product, or the sum thereof.  These vector products are constrained by the contraction axiom, a requirement that,
   for any vector \( \Bx \in V \) the square of that vector is the squared length of that vector

\begin{equation*}
    \Bx^2 = \Norm{\Bx}^2.
\end{equation*}

A multivector space \( M \) is the set of all possible multivectors that can be formed from the generating vector space \( V \).  All multivectors \( x, y, z \in M \) must satisfy the following additional axioms
\begin{tcolorbox}[tab2,tabularx={X|Y},title=Multivector space axioms.,boxrule=0.5pt]
    Addition is closed. & \( x + y \in M \) \\ \hline
    Multiplication is closed. & \( x y \in M \) \\ \hline
    Addition is associative. & \( (x + y) + z = x + (y + z) \) \\ \hline
    Addition is commutative. & \( y + x = x + y \) \\ \hline
    There exists a zero element \( 0 \in M \).  & \( x + 0 = x \) \\ \hline
    There exists a negative additive inverse \( -x \in M \). & \( x + (-x) = 0 \) \\ \hline
    Multiplication is distributive.  & \( z( x + y ) = z x + z y \), \( (z + w)x = z x + w x \) \\ \hline
    Multiplication is associative. & \( (x y) z = x ( y z ) \) \\ \hline
    There exists a multiplicative identity \( 1 \). & \( 1 x = x \) \\ \hline
\end{tcolorbox}
}

Some work is required to systematically examine the consequences of this abstract definition.

Observe first that since a scalar multiple of the square of a vector is as scalar by the definition above,
any scalar is also a multivector.  For example, if \( \Be_1 \) is the unit vector along the x-axis and \( s \) is a scalar, then

\begin{equation}\label{eqn:multivector:20}
   x = s \Be_1^2 = s,
\end{equation}

is a multivector.  Since vectors (a product of one vector, or a scalar multiple thereof) is also a multivector, this
means that vectors are multivectors, and that ``wierd'' sums of scalars and vectors, such as

\begin{dmath}\label{eqn:multivector:40}
   x = 1 + \Be_1,
\end{dmath}

are also multivectors!  A quantity like

\begin{dmath}\label{eqn:multivector:45}
   x = 1 + \Be_1 + \Be_1 \Be_2 - \Be_1 \Be_2 \Be_3,
\end{dmath}

where \( \Be_k \) are the standard orthonormal basis vectors for \R{3} (unit vectors that are mutually perperpendicular), is also a multivector.  The product \( \Be_1 \Be_2 \) is a bivector, and represents a positively oriented unit magnitude area in the x-y plane, whereas \( - \Be_1 \Be_2 \Be_3 \) is a trivector, representing a negatively oriented unit volume (inwards normals).

\section{Unpacking the axioms}
Now consider some consequences of the contraction axiom.  In particular,
requires the square of a unit (Euclidean) vector \( \Be_i \) to be unity

%\begin{equation}\label{eqn:multiplication:60}
\boxedEquation{eqn:multiplication:60}{
\Be_i^2 = 1.
}
%\end{equation}

With this implication noted, now consider the square of a simple two dimensional vector

\begin{dmath}\label{eqn:gaTutorial:80}
2
=
(\Be_1 + \Be_2)^2
= (\Be_1 + \Be_2)(\Be_1 + \Be_2)
= \Be_1^2 + \Be_2 \Be_1 + \Be_1 \Be_2 + \Be_2^2
= 2 + \Be_2 \Be_1 + \Be_1 \Be_2.
\end{dmath}

The sum above with both scalar terms and terms that are composed of products of vectors is called a multivector.
A product of two perpendicular vectors (or a sum of such products) is called a bivector, and can be used to represent an oriented plane.
Geometric Algebra allows for sums of scalars, vectors, bivectors, and higher degree products.

Observe that for this identity to hold, the bivector terms must sum to zero.  That is

%\begin{dmath}\label{eqn:multiplication:140}
\boxedEquation{eqn:multiplication:140}{
\Be_1 \Be_2 = -\Be_1 \Be_2.
}
%\end{dmath}

This implies that the product of two orthonormal vectors anticommutes.  In general it is also true that

\maketheorem{Normal anticommutation}{thm:multiplication:anticommutationNormal}{
The product of any two normal vectors \(\Bu\), and \(\Bv\) anticommute.
\begin{equation*}
\Bu \Bv = -\Bv \Bu.
\end{equation*}
} % theorem

%


\subsection{Problems}
\makeproblem{One dimensional multivector space.}{problem:multivector:40}{
   Verify that for \( c, d \in \bbR \) the set \( M = \setlr{ c + d \Be_1 } \) satisifies all the multivector axioms.
} % problem

%%\makedefinition{Scalar}{def:multiplication:scalar}{
%%   A (real) number with no implied direction.
%%}
%%
%%Examples of scalars are \( \pi, 3, -4 \), and \( 0 \).
%%
%%\makedefinition{Vector}{def:multiplication:vector}{
%%%\href{https://www.youtube.com/watch?v=bOIe0DIMbI8}{A quantity with direction and magnitude.}
%%\href{https://youtu.be/bOIe0DIMbI8?t=19}{A quantity with direction and magnitude.}
%%}
%%
%%In this book,
%%In order to express
%%\begin{dmath}\label{eqn:multivector:60}
%%\Bx = c_1 \Be_1 + c_2 \Be_2,
%%\end{dmath}
%%
%%where \( \Be_1 \) and \( \Be_2 \) are a pair of perpendicular vectors of length one along the x and y axis respectively, as illustrated in
%%
%%FIXME: figure.
%%These
%%
%%, as represented pictorially as an arrow
%%
%%
%%
%%\section{Vector space}
%%\section{Vector multiplication}
%%\section{Multivector}
%%
%%Geometric Algebra, or \boldTextAndIndex{GA} defines a multiplication operation for vectors.
%%GA also
%%generalizes the concept of a vector, introducing a new type of mathematical object, the multivector.
%%
%%
%%
%%In traditional vector algebra, a sum of a scalar and a vector, such as
%%
%%\begin{dmath}\label{eqn:multivector:80}
%%M = 1 + 2 \Be_1,
%%\end{dmath}
%%
%%is not considered meaningful.  This is
%%
