%
% Copyright © 2017 Peeter Joot.  All Rights Reserved.
% Licenced as described in the file LICENSE under the root directory of this GIT repository.
%
\makeexample{Electrostatic and magnetostatics.}{example:vectorproduct:electrostatics}{

With no magnetic current, no magnetic sources, and no time derivatives, Maxwell's equations in simple media take the form

\begin{dmath}\label{eqn:vectorproduct:120}
\begin{aligned}
\spacegrad \cdot \BB &= 0 \\
\spacegrad \cross \BB &= \mu \BJ \\
\spacegrad \cross \BE &= 0 \\
\spacegrad \cdot \BE &= \frac{\rho}{\epsilon}.
\end{aligned}
\end{dmath}

For electrostatic conditions \( \BJ = 0 \), so using \cref{eqn:vectorproduct:100} the first and last equations can be combined into a single first order homogeneous multivector gradient equation

\begin{equation}\label{eqn:vectorproduct:140}
\spacegrad \BB
=
\spacegrad \cdot \BB +I (\spacegrad \cross \BB )
=
0.
\end{equation}

The electric gradient equation is

\begin{equation}\label{eqn:vectorproduct:160}
\spacegrad \BE
=
\spacegrad \cdot \BE +I (\spacegrad \cross \BE )
=
\frac{\rho}{\epsilon}.
\end{equation}

Maxwell's equations are reduced to two multivector equations with this transformation
\begin{dmath}\label{eqn:vectorproduct:180}
\begin{aligned}
\spacegrad \BE &= \frac{\rho}{\epsilon} \\
\spacegrad \BB &= 0.
\end{aligned}
\end{dmath}

For magnetostatics \( \rho = 0 \), and the same assembly of Maxwell's equations gives

\begin{dmath}\label{eqn:vectorproduct:220}
\begin{aligned}
\spacegrad \BB &= I \mu \BJ \\
\spacegrad \BE &= 0.
\end{aligned}
\end{dmath}

It will be seen later that it is actually more natural to express magnetic fields as a bivector \( I \BB \).  Using \( I^2 = -1 \) (\cref{problem:gradeselection:R3PseudoscalarSquare}) the magnetostatic equation takes the form

\begin{dmath}\label{eqn:vectorproduct:240}
\spacegrad (I \BB) = - \mu \BJ.
\end{dmath}

Both the electrostatic and magnetostatic equations can be solved directly using the Green's function for the gradient, producing the Coulomb integral for the electric field and Biot-Savart's law for the magnetic field.
Before demonstrating this, the concepts required to attack multivector integrals must be formulated.
} % example
