%
% Copyright © 2016 Peeter Joot.  All Rights Reserved.
% Licenced as described in the file LICENSE under the root directory of this GIT repository.
%
\subsection{Poynting theorem}
\index{Poynting theorem}

Poynting's theorem describes the relationship between the flux of energy through a surface bounding a volume.
The theorem follows from computing the divergence of the Poynting vector \( \BS = \BE \cross \BH \).
In terms of \( \BE \) and \( \BH \) the Poynting vector can be written in dual form as a dot product

\begin{equation}\label{eqn:maxwellsEquations:780}
\BE \cross \BH
=
\gpgradeone{ I (\BH \wedge \BE) }
=
\gpgradeone{ I \BH \BE }
=
(I \BH) \cdot \BE.
\end{equation}

Similarly, the Poynting divergence is most compactly expressed as a scalar selection operation

\begin{equation}\label{eqn:maxwellsEquations:640}
\spacegrad \cdot \lr{ \BE \cross \BH }
=
\gpgradezero{ \spacegrad I \lr{ \BH \wedge \BE } }
=
\gpgradezero{ \spacegrad I \BH \BE }.
\end{equation}

Here the gradient is acting on everything to the right, however, allowing the gradient to act bidirectionally, and employing the
the flexibility to use cyclic permutation within a scalar selection
(i.e. \(\gpgradezero{ABC} = \gpgradezero{CAB}\))
, allows for the easy application of the chain rule

\begin{dmath}\label{eqn:maxwellsEquations:760}
\gpgradezero{ \spacegrad I \BH \BE }
=
\gpgradezero{ \BE \lrspacegrad I \BH }
=
\gpgradezero{ (\BE \lspacegrad) I \BH }
+\gpgradezero{ \BE (\rspacegrad I \BH) }
\end{dmath}

Explicit left and right acting gradients are required because the gradient operator does not commute with the vector fields.

The gradient action on \( I \BH \) (from the left) is given by
\cref{eqn:maxwellsEquations:380}.
The right acting gradient action on \( \BE \) is given by reversing all the products in
%\spacegrad \BE = \inv{\epsilon} \rho - I \BM - \mu \PD{t}{(I\BH)}
\cref{eqn:maxwellsEquations:360} (in particular noting that \( I^\dagger = -I \) )

\begin{dmath}\label{eqn:maxwellsEquations:660}
%I \BH \lspacegrad = \frac{I \rho_\txtm}{\mu} + \BJ + \epsilon \PD{t}{\BE}.
\BE \lspacegrad = \inv{\epsilon} \rho + I \BM + \mu \PD{t}{(I\BH)}.
\end{dmath}

This gives
\begin{dmath}\label{eqn:maxwellsEquations:680}
\spacegrad \cdot \BS
=
\gpgradezero
{
\lr{ \inv{\epsilon} \rho + I \BM + \mu \PD{t}{(I\BH)} } I \BH
+
\BE
\lr
{
\frac{I \rho_\txtm}{\mu} - \BJ - \epsilon \PD{t}{\BE}
}
}
,
\end{dmath}

or
%\begin{dmath}\label{eqn:maxwellsEquations:700}
\boxedEquation{eqn:maxwellsEquations:720}{
0 =
\spacegrad \cdot \BS
+
\BH \cdot \BM + \BJ \cdot \BE
+ \PD{t}{\BB} \cdot \BH
+ \PD{t}{\BD} \cdot \BE.
}
%\end{dmath}

The sum of the last two terms is the time rate of change of the energy density.
To illustrate this consider the change of energy density through a volume with neither electric nor magnetic current sources in that region of space

\begin{dmath}\label{eqn:maxwellsEquations:740}
\PD{t}{} \int_V
\inv{2} dV \lr{
\BB \cdot \BH
+ \BD \cdot \BE
}
=
-\int_{\partial V} dA \ncap \cdot \BS.
\end{dmath}

Here \( \ncap \) is the outward normal, so if the energy contained in the volume is decreasing, then \( \BS \) must represent the energy per unit area that leaves the volume.
The direction of the Poynting vector is the direction that the energy is leaving the volume.
Only the components of the Poynting vector that are colinear with the surface normal will result in energy leaving or entering the volume.

