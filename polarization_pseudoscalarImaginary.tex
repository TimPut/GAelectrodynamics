%
% Copyright © 2018 Peeter Joot.  All Rights Reserved.
% Licenced as described in the file LICENSE under the root directory of this GIT repository.
%
%{

The multivector \( 1 + \Be_3 \) acts as a projector, consuming any factors of \( \Be_3 \)

\begin{dmath}\label{eqn:polarization_pseudoscalarImaginary:440}
(1 + \Be_3) \Be_3
=
\Be_3 + \Be_3^2
=
1 + \Be_3.
\end{dmath}

This property allows all the bivector imaginaries \( i = \Be_{12} = \Be_3 I \) in \cref{eqn:polarization_circular:200} to be re-expressed in terms of the \R{3} pseudoscalar \( I = \Be_{123} \).  To illustrate this consider just the left circular polarized wave

\begin{dmath}\label{eqn:polarization_pseudoscalarImaginary:460}
F_\txtL
=
\lr{ 1 + \Be_3 } \Be_1 \alpha_\txtL e^{i\phi}
=
\lr{ 1 + \Be_3 } \Be_1 \alpha_\txtL \lr{ \cos\phi + \Be_3 I \sin\phi }
=
\lr{ 1 + \Be_3 } \Be_1 \alpha_\txtL \cos\phi
-\lr{ 1 + \Be_3 } \Be_3 \Be_1 \alpha_\txtL I \sin\phi
=
\lr{ 1 + \Be_3 } \Be_1 \alpha_\txtL e^{-I\phi}
=
\lr{ 1 + \Be_3 } \Be_1 \lr{ \alpha_{\txtL 1} + \Be_3 I\alpha_{\txtL 2}  } e^{-I\phi}
=
\lr{ 1 + \Be_3 } \Be_1 \lr{ \alpha_{\txtL 1} - I \alpha_{\txtL 2} } e^{-I\phi}.
\end{dmath}

This shows that the coefficients for the circular polarized states can be redefined using the pseudoscalar as an imaginary (in contrast to the bivector imaginary used in \cref{eqn:polarization_circular:220})
\begin{dmath}\label{eqn:polarization_pseudoscalarImaginary:480}
\begin{aligned}
\alpha_\txtL' &= \alpha_{\txtL 1} - I \alpha_{\txtL 2} \\
\alpha_\txtR' &= \alpha_{\txtR 1} - I \alpha_{\txtR 2},
\end{aligned}
\end{dmath}
so that the plane wave is
\begin{dmath}\label{eqn:polarization_pseudoscalarImaginary:500}
F = \lr{ 1 + \Be_3 } \Be_1 \lr{ \alpha_\txtL' e^{-I\phi} + \alpha_\txtR' e^{I\phi} }.
\end{dmath}

Like \cref{eqn:polarization_circular:200} this plane wave representation does not require taking any real parts.  The transverse plane in which the electric and magnetic fields lie is defined by the duality relation \( i = I \Be_3 \).

The energy momentum multivector for a wave described in terms of the pseudoscalar circular polarization states of \cref{eqn:polarization_pseudoscalarImaginary:500} is just

\begin{dmath}\label{eqn:polarization_pseudoscalarImaginary:620}
\calE + \frac{\BS}{v} =
\epsilon \lr{ 1 + \Be_3 } \lr{ \Abs{\alpha_\txtL'}^2 + \Abs{\alpha_\txtR'}^2 },
\end{dmath}
where the absolute value is computed using the reverse as the conjugation operation \( \Abs{z}^2 = z z^\dagger \).
%}
