%
% Copyright � 2016 Peeter Joot.  All Rights Reserved.
% Licenced as described in the file LICENSE under the root directory of this GIT repository.
%
%{
\index{Green's function}

\subsection{Motivation.}

We will now introduce Green's functions, which provide a general method of solving many of the linear differential equations that will be encountered in electromagnetism.

\subsubsection{Time domain problems in electromagnetism}

Examples of the PDEs that we can apply Green's function techniques to include

\begin{subequations}
\label{eqn:greensFunctionOverview:200}
\begin{equation}\label{eqn:greensFunctionOverview:220}
\lr{ \spacegrad + \inv{c} \PD{t}{} } F(\Bx, t) = J(\Bx, t)
\end{equation}
\begin{equation}\label{eqn:greensFunctionOverview:162}
\lr{ \spacegrad^2 - \inv{c^2} \PDSq{t}{} } F(\Bx, t) =
\lr{ \spacegrad - \inv{c} \PD{t}{} }
\lr{ \spacegrad + \inv{c} \PD{t}{} } F(\Bx, t) =
B(\Bx, t).
\end{equation}
\end{subequations}

The reader is no doubt familiar with the wave equation (\cref{eqn:greensFunctionOverview:162}), where \( F \) is the waving function, and \( B \) is the forcing function.
Scalar and vector valued wave equations are
encountered in scalar and vector forms in conventional electromagnetism.
We will see multivector variations of the wave equation, so it should be assumed that \( F \) and \( B \) are multivector valued.

\Cref{eqn:greensFunctionOverview:220} is actually the geometric algebra form of Maxwell's equation (singular),
where \( F \) is a 1,2 multivector, and \( J \) is a multivector containing all the charge and current density contributions.
We will call the operator in \cref{eqn:greensFunctionOverview:220} the spacetime gradient\footnote{A slightly different operator is also called the spacetime gradient in STA (Space Time Algebra) \citep{doran2003gap}, which employs a non-Euclidean basis to generate a four dimensional relativistic geometric algebra.
Our spacetime gradient is related to the STA spacetime gradient by a constant factor.}.

\subsubsection{Frequency domain problems in electromagnetism.}
It will often be convient to utilize a time harmonic (frequency domain) representation of \cref{eqn:greensFunctionOverview:200}.
This can be achieved by utilizing Fourier transform pairs or with a phasor representation.

We may define Fourier transform pairs of multivector fields and sources in the conventional fashion

\index{Fourier transform}
\makedefinition{Multivector Fourier transform pairs}{dfn:greensFunctionOverview:280}{
The Fourier transform pair for a multivector valued function \( f(\Bx, t) \) will be written as
\begin{equation*}
\begin{aligned}
f(\Bx, t) &= \int f_\omega(\Bx) e^{j \omega t} d\omega \\
f_\omega(\Bx) &= \inv{2 \pi} \int f(\Bx, t) e^{-j \omega t} dt,
\end{aligned}
\end{equation*}
where \( j \) is an arbitrary scalar imaginary that commutes with all multivectors.
} % definition

In these transform pairs, the imaginary \( j \) need not be represented by any geometrical imaginary such as \( \Be_{12} \).
In particular, we need not assume that the represention of \( j \) is the
\R{3} pseudoscalar \( I \), despite the fact that \( I \) does commute with all \R{3} multivectors.
We wish to have the freedom to
assume that non-geometric real and imaginary operations can be performed without picking or leaving out any specific grade pseudoscalar components of the multivector fields or sources, so we won't impose any a-priori restrictions on the representations of \( j \).
In particular, this provides the freedom to utilize phasor (fixed frequency) representions of our multivector functions.
%Introduction of yet another imaginary quantity in a geometric algebra context where we have so many to pick it somewhat unfortunate, but it allows us to apply Fourier transform techniques without worry about the non-commutative effects that might have to be considered should we choose to use a geometric imaginary to represent the frequency dependency.
We will use the engineering convention for our
phasor representations, where assuming a complex exponential time dependence of the following form is assumed

\index{time harmonic}
\index{frequency domain}
\makedefinition{Multivector phasor representation.}{dfn:greensFunctionOverview:300}{
The phasor representation \( f(\Bx) \) of a multivector valued (real) function \( f(\Bx, t) \) is defined implicitly as
\begin{equation*}
f(\Bx, t) = \Real\lr{ f(\Bx) e^{j \omega t} },
\end{equation*}
where \( j \) is an arbitrary scalar imaginary that commutes with all multivectors.
} % definition

The complex valued multivector \( f(\Bx) \) is still generated from the real Euclidean basis for \R{3}.
We will have
no reason to introduce complex inner products spaces into the mix.
The reader must take care when reading any literature that utilizes Fourier transforms or phasor representation, since the conventions vary.
In particular the physics representation of a phasor typically uses the opposite sign convention
\( f(\Bx, t) = \Real\lr{ f(\Bx) e^{-i \omega t }} \).

Armed with Fourier transform or phasor representations, the frequency domain representations of
\cref{eqn:greensFunctionOverview:200} are found to be

\index{Helmholtz operator}
\begin{subequations}
\label{eqn:greensFunctionOverview:320}
\begin{equation}\label{eqn:greensFunctionOverview:240}
\lr{ \spacegrad + j k } F(\Bx) = J(\Bx)
\end{equation}
\begin{equation}\label{eqn:greensFunctionOverview:260}
\lr{ \spacegrad^2 + k^2 } F(\Bx)
=
\lr{ \spacegrad - j k } \lr{ \spacegrad + j k } F(\Bx)
= B(\Bx),
\end{equation}
\end{subequations}

where \( k = \omega/c \), and any explicit frequency dependence in our transform pairs has been suppressed.
We will call these equations the first and second order Helmholtz equations respectively.
The first order equation applies a multivector differential operator to a multivector field, which must equal the multivector forcing function (the sources).

For statics problems (\( k = 0 \)), we may work with real fields and sources, dispensing with any need to take real parts once the equations are solved.

\subsection{Green's function solutions.}

\subsubsection{Unbounded.}

As the operators in \cref{eqn:greensFunctionOverview:200}, and \cref{eqn:greensFunctionOverview:320} all have a similar linear structure, which we can write in a more abstract notation as

\begin{dmath}\label{eqn:greensFunctionOverview:340}
\LL F(\Bx) = J(\Bx).
\end{dmath}

Given the linear structure of the PDE that we wish to solve, it makes sense to assume that the solutions also have a linear structure.
The most general such solution we can assume has the form

\index{Green's function}
\begin{dmath}\label{eqn:greensFunctionOverview:360}
F(\Bx) = \int G(\Bx, \Bx') J(\Bx') dV' + F_0(\Bx),
\end{dmath}

where \( F_0(\Bx) \) is any solution to the equivalent homogeneous equation \( \LL F_0 = 0 \), and \( G(\Bx, \Bx') \) is the Green's function associated with \cref{eqn:greensFunctionOverview:340}.
Operating on the presumed solution
\cref{eqn:greensFunctionOverview:360} with \( \LL \) yields

\begin{dmath}\label{eqn:greensFunctionOverview:380}
J(\Bx) = \LL F(\Bx) = \LL\lr{
\int G(\Bx, \Bx') J(\Bx') dV' + F_0(\Bx) }
=
\int \lr{ \LL G(\Bx, \Bx') } J(\Bx') dV',
\end{dmath}

which shows that we require the Green's function to have delta function semantics satisfying

\begin{dmath}\label{eqn:greensFunctionOverview:400}
\LL G(\Bx, \Bx') = \delta(\Bx - \Bx').
\end{dmath}

The scalar valued Green's functions for the Laplacian and the (2nd order) Helmholtz equations are well known.
The Green's functions for the spacetime gradient and the 1st order Helmholtz equation (which is just the gradient when \( k = 0 \)) are multivector valued and will be derived here.

\subsubsection{Green's theorem.}

When the presumed solution is a superposition of only states in a bounded region, such as

\begin{dmath}\label{eqn:greensFunctionHelmholtz:200}
F(\Bx) = \int_V dV' B(\Bx') G(\Bx, \Bx') + F_0(\Bx),
\end{dmath}

then life gets a bit more interesting, and requires different treatment for first and second order (in the gradient) problems.

For the second order problems, we require Green's theorem, which must be generalized slightly for use with multivector fields.

The basic idea is that we can relate the Laplacian's of the Green's function and the field
\( F(\Bx') (\spacegrad')^2 G(\Bx, \Bx') = G(\Bx, \Bx') (\spacegrad')^2 F(\Bx') + \cdots \).
That relation is usually expressed in terms of the difference of the two in the integral domain

\maketheorem{Green's theorem}{thm:gradientGreensFunctionEuclidean:220}{
Given a multivector function \( F \) and a scalar function \( G \)
\begin{equation*}
\int_V \lr{ F \spacegrad^2 G - G \spacegrad^2 F } dV = \int_{\partial V} \lr{ F \ncap \cdot \spacegrad G - G \ncap \cdot \spacegrad F },
\end{equation*}
where \( \partial V \) is the boundary of the volume \( V \).
} % theorem

A straightforward, but perhaps inelligant way of proving this theorem is to expand the sandwich difference in coordinates

\begin{dmath}\label{eqn:greensFunctionHelmholtz:260}
F \spacegrad^2 G - G \spacegrad^2 F
=
\sum_k F \partial_k \partial_k G - G \partial_k \partial_k F
=
\sum_k \partial_k \lr{
F \partial_k G - G \partial_k F
}
-
(\partial_k F)(\partial_k G) + (\partial_k G)(\partial_k F).
\end{dmath}

Since \( G \) is a scalar, the last two terms cancel, and we can integrate

\begin{dmath}\label{eqn:greensFunctionHelmholtz:280}
\int_V \lr{ F \spacegrad^2 G - G \spacegrad^2 F } dV
=
\sum_k \int_V \partial_k \lr{ F \partial_k G - G \partial_k F }.
\end{dmath}

Each integral above involves one component of the gradient.
From
%the fundamental theorem of geometric calculus
\cref{thm:fundamentalTheoremOfCalculus:1}
we know that
\begin{dmath}\label{eqn:greensFunctionHelmholtz:300}
\int_V \spacegrad Q dV = \int_{\partial V} \ncap Q dA,
\end{dmath}

for any multivector \( Q \).
Equating components gives

\begin{dmath}\label{eqn:greensFunctionHelmholtz:340}
\int_V \partial_k Q dV = \int_{\partial V} \ncap \cdot \Be_k Q dA,
\end{dmath}

which can be substituted into \cref{eqn:greensFunctionHelmholtz:280} to find

\begin{dmath}\label{eqn:greensFunctionHelmholtz:360}
\int_V \lr{ F \spacegrad^2 G - G \spacegrad^2 F } dV
=
\sum_k \int_{\partial V} \ncap \cdot \Be_k \lr{ F \partial_k G - G \partial_k F } dA
=
\int_{\partial V} \lr{ F (\ncap \cdot \spacegrad) G - G (\ncap \cdot \spacegrad) F } dA,
\end{dmath}

which proves the theorem.

\subsubsection{Bounded solutions to first order problems.}

For first order problems we will need an intermediate result similar to Green's theorem.

\makelemma{Normal relations for a gradient sandwich.}{lemma:greensFunctionOverview:420}{
Given a multivector functions \( F(\Bx'), G(\Bx, \Bx') \), and a gradient \( \spacegrad' \) acting bidirectionally on functions \( \Bx' \), we have
\begin{equation*}
- \int_V \lr{ G(\Bx, \Bx') \lspacegrad' } F(\Bx') dV'
=
\int_V G(\Bx, \Bx') \lr{ \lspacegrad' F(\Bx') } dV'
-
\int_{\partial V} G(\Bx, \Bx') \ncap' F(\Bx') dA'.
\end{equation*}
} % lemma

This follows directly from \cref{thm:fundamentalTheoremOfCalculus:1}

\begin{dmath}\label{eqn:greensFunctionOverview:440}
\int_{\partial V} G(\Bx, \Bx') \ncap' F(\Bx') dA'
=
\int_V G(\Bx, \Bx') \lrspacegrad' F(\Bx') dV'
=
\int_V \lr{ G(\Bx, \Bx') \lspacegrad' } F(\Bx') dV'
+
\int_V G(\Bx, \Bx') \lr{ \lspacegrad' F(\Bx') } dV',
\end{dmath}

which can be rearranged to prove \cref{lemma:greensFunctionOverview:420}.

%}
