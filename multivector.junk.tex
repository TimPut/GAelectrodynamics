
%%\makedefinition{Scalar}{def:multiplication:scalar}{
%%   A (real) number with no implied direction.
%%}
%%
%%Examples of scalars are \( \pi, 3, -4 \), and \( 0 \).
%%
%%\makedefinition{Vector}{def:multiplication:vector}{
%%%\href{https://www.youtube.com/watch?v=bOIe0DIMbI8}{A quantity with direction and magnitude.}
%%\href{https://youtu.be/bOIe0DIMbI8?t=19}{A quantity with direction and magnitude.}
%%}
%%
%%In this book,
%%In order to express
%%\begin{dmath}\label{eqn:multivector:60}
%%\Bx = c_1 \Be_1 + c_2 \Be_2,
%%\end{dmath}
%%
%%where \( \Be_1 \) and \( \Be_2 \) are a pair of perpendicular vectors of length one along the x and y axis respectively, as illustrated in
%%
%%FIXME: figure.
%%These
%%
%%, as represented pictorially as an arrow
%%
%%
%%
%%\section{Vector space}
%%\section{Vector multiplication}
%%\section{Multivector}
%%
%%Geometric Algebra, or \boldTextAndIndex{GA} defines a multiplication operation for vectors.
%%GA also
%%generalizes the concept of a vector, introducing a new type of mathematical object, the multivector.
%%
%%
%%
%%In traditional vector algebra, a sum of a scalar and a vector, such as
%%
%%\begin{dmath}\label{eqn:multivector:80}
%%M = 1 + 2 \Be_1,
%%\end{dmath}
%%
%%is not considered meaningful.  This is
%%


\section{Junk}
Geometric Algebra (\textAndIndex{GA}) generalizes the concept of vector and a normed vector space.  This is done by introducing a vector multiplication operation into the mix, and a vector generalization called a \textAndIndex{multivector}.

The multivector is a hybrid object that may contain any sum of all or some of

\begin{enumerate}
   \item scalars, numeric quantities with magnitude and no direction,
   \item vectors (1-vectors), quantities with magnitude and direction,
   \item k-vectors, which are generalizatized line, area, volume and hypervolume elements, represent subspaces with orientation and magnitude.
\end{enumerate}

Scalars and vectors are assumed to be familiar, however, a sum of a scalar and vector is a new and arguably strange idea.  The k-vectors with \( k=2 \) and \( k = 3 \) are called bivectors and trivectors, and represent oriented planes and volumes in space respectively.
Bivectors, trivectors, and k-vectors will be defined later in a more precise fashion, as will orientation.  For now, orientation can be thought of algebraically as a sign, but physically may have an interpretation of sidedness, direction of a normal to the surface\footnote{In three dimensional spaces where a normal can be defined.}, or a rotational sense.

FIXME: orientation pictures here.

The vector multiplication operation is a new type of vector product.  The vector product is distinct from, but relatied to, the familiar dot or cross products in a way that will become clear.


