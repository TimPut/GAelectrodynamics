%
% Copyright © 2016 Peeter Joot.  All Rights Reserved.
% Licenced as described in the file LICENSE under the root directory of this GIT repository.
%

\makeproblem{\R{3} dual forms of Stokes' theorem.}{problem:stokesTheoremCoreProblems:1}{
Prove
\makesubproblem{}{problem:stokesTheoremCoreProblems:1:a}
\cref{eqn:scalarVolumeElement:1681},
\makesubproblem{}{problem:stokesTheoremCoreProblems:1:b}
\cref{eqn:scalarVolumeElement:1801},
\makesubproblem{}{problem:stokesTheoremCoreProblems:1:c}
and \cref{eqn:scalarVolumeElement:1801c}.
} % problem

\makeanswer{problem:stokesTheoremCoreProblems:1}{

The volume elements are
\makeSubAnswer{}{problem:stokesTheoremCoreProblems:1:a}
\begin{subequations}
\label{eqn:stokesTheoremCoreProblems:20}
\begin{dmath}\label{eqn:stokesTheoremCoreProblems:40}
d^2 \Bx \cdot \spacegrad
=
dA \gpgradeone{ I \ncap \spacegrad }
=
dA I \ncap \wedge \spacegrad
\end{dmath}
\begin{dmath}\label{eqn:stokesTheoremCoreProblems:60}
d^2 \Bx \cdot (\spacegrad \wedge \BA)
=
dA \gpgradezero{ I \ncap \spacegrad \BA }
=
dA I \ncap \wedge \spacegrad \wedge \BA
\end{dmath}
\begin{dmath}\label{eqn:stokesTheoremCoreProblems:80}
d^3 \Bx \cdot \spacegrad \phi
=
dV \gpgradetwo{ I \spacegrad \phi }
=
dV I \spacegrad \phi
\end{dmath}
\begin{dmath}\label{eqn:stokesTheoremCoreProblems:100}
d^3 \Bx \cdot (\spacegrad \wedge \BA)
=
dV \gpgradeone{ I (\spacegrad \wedge \BA) }
=
dV I \spacegrad \wedge \BA
\end{dmath}
\begin{dmath}\label{eqn:stokesTheoremCoreProblems:120}
d^3 \Bx \cdot (\spacegrad \wedge B)
=
dV \gpgradezero{ I (\spacegrad \wedge B) }
=
dV I \spacegrad \wedge B.
\end{dmath}
\end{subequations}

The corresponding boundary forms are
\begin{subequations}
\label{eqn:stokesTheoremCoreProblems:140}
\begin{equation}\label{eqn:stokesTheoremCoreProblems:160}
d\Bx \psi
\end{equation}
\begin{dmath}\label{eqn:stokesTheoremCoreProblems:180}
d\Bx \cdot \BA
\end{dmath}
\begin{dmath}\label{eqn:stokesTheoremCoreProblems:200}
d^2 \Bx \psi
=
dA I \ncap \psi
\end{dmath}
\begin{dmath}\label{eqn:stokesTheoremCoreProblems:220}
d^2 \Bx \cdot \BA
=
dA \gpgradeone{ I \ncap \BA }
=
dA I \ncap \wedge \BA
\end{dmath}
\begin{dmath}\label{eqn:stokesTheoremCoreProblems:240}
d^2 \Bx \cdot B
=
dA \gpgradezero{ I \ncap B }
=
dA I \ncap \wedge B.
\end{dmath}
\end{subequations}

Assembling these pieces back into the integrals proves the relationships.

\makeSubAnswer{}{problem:stokesTheoremCoreProblems:1:b}

To show \cref{eqn:scalarVolumeElement:1841} note that
\begin{dmath}\label{eqn:stokesTheoremCoreProblems:260}
I (\Ba \wedge \Bb \wedge \Bc)
=
\gpgradezero{ I \Ba \wedge \Bb \wedge \Bc }
=
\gpgradezero{ I \Ba (\Bb \wedge \Bc) -
I \Ba \cdot (\Bb \wedge \Bc)
}
=
\gpgradezero{ I \Ba I(\Bb \cross \Bc) }
=
- \Ba \cdot (\Bb \cross \Bc).
\end{dmath}

To show \cref{eqn:scalarVolumeElement:1901} note that
\begin{dmath}\label{eqn:stokesTheoremCoreProblems:280}
\Ba \wedge (I \BA)
=
\Ba \wedge (I \BA)
=
\gpgradethree{ \Ba I \BA }
=
\gpgradethree{ I \Ba \cdot \BA }
=
I (\Ba \cdot \BA).
\end{dmath}

\makeSubAnswer{}{problem:stokesTheoremCoreProblems:1:c}

For vector \( \Ba \), these transformations all follow from
\begin{dmath}\label{eqn:stokesTheoremCoreProblems:300}
\Ba \cross \Bf
=
\gpgradeone{ -I \Ba \wedge \Bf}
=
\gpgradeone{ -I \Ba \Bf}
=
-\gpgradeone{ \Ba I \Bf}
=
-\Ba \cdot (I \Bf)
=
\Ba \cdot B.
\end{dmath}

} % answer
