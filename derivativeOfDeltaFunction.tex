%
% Copyright � 2018 Peeter Joot.  All Rights Reserved.
% Licenced as described in the file LICENSE under the root directory of this GIT repository.
%
%{
%%%\input{../latex/blogpost.tex}
%%%\renewcommand{\basename}{derivativeOfDeltaFunction}
%%%%\renewcommand{\dirname}{notes/phy1520/}
%%%\renewcommand{\dirname}{notes/ece1228-electromagnetic-theory/}
%%%%\newcommand{\dateintitle}{}
%%%%\newcommand{\keywords}{}
%%%
%%%\input{../latex/peeter_prologue_print2.tex}
%%%
%%%\usepackage{peeters_layout_exercise}
%%%\usepackage{peeters_braket}
%%%\usepackage{peeters_figures}
%%%\usepackage{siunitx}
%%%%\usepackage{mhchem} % \ce{}
%%%%\usepackage{macros_bm} % \bcM
%%%%\usepackage{macros_qed} % \qedmarker
%%%%\usepackage{txfonts} % \ointclockwise
%%%
%%%\beginArtNoToc
%%%
%%%\generatetitle{Derivative of a delta function}
%\chapter{Derivative of a delta function}
\label{chap:derivativeOfDeltaFunction}

%In the geometric algebra formulation of Maxwell's equation (singular in GA), the Green's function for the spacetime gradient ends up with terms like
%In the geometric algebra formulation of Maxwell's equation (singular in GA),
The Green's function for the spacetime gradient ends up with terms like
\begin{dmath}\label{eqn:derivativeOfDeltaFunction:20}
\begin{aligned}
\frac{d}{dr} &\delta( -r/c + t - t' ) \\
\frac{d}{dt} &\delta( -r/c + t - t' ),
\end{aligned}
\end{dmath}
where \( t' \) is the integration variable of the test function that the delta function will be applied to.
If these were derivatives with respect to the integration variable, then we could use
\begin{dmath}\label{eqn:derivativeOfDeltaFunction:60}
\int_{-\infty}^\infty
\lr{ \frac{d}{dt'} \delta(t') } \phi(t') = -\phi'(0),
\end{dmath}
which follows by chain rule, and an assumption that \( \phi(t') \) is well behaved at the points at infinity.
It is not clear that how, if at all, this could be applied to either of \cref{eqn:derivativeOfDeltaFunction:20}.
% or \cref{eqn:derivativeOfDeltaFunction:40}.

Let's go back to square one, and figure out the meaning of these delta functions by their action on a test function.
We wish to compute
\begin{dmath}\label{eqn:derivativeOfDeltaFunction:80}
\int_{-\infty}^\infty \frac{d}{du} \delta( a u + b - t' ) f(t') dt'.
\end{dmath}

Let's start with a change of variables \( t'' = a u + b - t' \), for which we find
\begin{dmath}\label{eqn:derivativeOfDeltaFunction:100}
\begin{aligned}
t' &= a u + b - t'' \\
dt'' &= - dt' \\
\frac{d}{du} &= \frac{dt''}{du} \frac{d}{dt''} = a \frac{d}{dt''}.
\end{aligned}
\end{dmath}

Substitution back into \cref{eqn:derivativeOfDeltaFunction:80} gives
\begin{dmath}\label{eqn:derivativeOfDeltaFunction:120}
\begin{aligned}
a &\int_{\infty}^{-\infty} \lr{ \frac{d}{dt''} \delta( t'' ) } f( a u + b - t'' ) (-dt'') \\
&=
a \int_{-\infty}^{\infty} \lr{ \frac{d}{dt''} \delta( t'' ) } f( a u + b - t'' ) dt'' \\
&=
\evalrange{a \delta(t'') f( a u + b - t'')}{-\infty}{\infty} \\
&\qquad -
a \int_{-\infty}^{\infty} \delta( t'' )  \frac{d}{dt''} f( a u + b - t'' ) dt'' \\
&=
- \evalbar{ a \frac{d}{dt''} f( a u + b - t'' ) }{t'' = 0} \\
&=
\evalbar{ a \frac{d}{ds} f( s ) }{s = a u + b}.
\end{aligned}
\end{dmath}

This shows that the action of the derivative of the delta function (with respect to a non-integration variable parameter \( u \)) is
%\begin{dmath}\label{eqn:derivativeOfDeltaFunction:140}
\boxedEquation{eqn:derivativeOfDeltaFunction:140}{
\frac{d}{du} \delta( a u + b - t' )
\sim
\evalbar{a \frac{d}{ds}}{s = a u + b}.
}
%\end{dmath}
%%%%}
%%%%\EndArticle
%%%\EndNoBibArticle
