%
% Copyright � 2018 Peeter Joot.  All Rights Reserved.
% Licenced as described in the file LICENSE under the root directory of this GIT repository.
%
%{
%%%\input{../latex/blogpost.tex}
%%%\renewcommand{\basename}{planewavesMultivector}
%%%%\renewcommand{\dirname}{notes/phy1520/}
%%%\renewcommand{\dirname}{notes/ece1228-electromagnetic-theory/}
%%%%\newcommand{\dateintitle}{}
%%%%\newcommand{\keywords}{}
%%%
%%%\input{../latex/peeter_prologue_print2.tex}
%%%
%%%\usepackage{peeters_layout_exercise}
%%%\usepackage{peeters_braket}
%%%\usepackage{peeters_figures}
%%%\usepackage{siunitx}
%%%%\usepackage{mhchem} % \ce{}
%%%%\usepackage{macros_bm} % \bcM
%%%%\usepackage{macros_qed} % \qedmarker
%%%%\usepackage{txfonts} % \ointclockwise
%%%
%%%\beginArtNoToc
%%%
%%%\generatetitle{Multivector plane wave representation}
%\chapter{Multivector plane wave representation}
%\label{chap:planewavesMultivector}

The geometric algebra form of Maxwell's equations in free space (or source free isotopic media with group velocity \( c \)) is the multivector equation
\begin{dmath}\label{eqn:planewavesMultivector:20}
\lr{ \spacegrad + \inv{c}\PD{t}{} } F(\Bx, t) = 0.
\end{dmath}
Here \( F = \BE + I c \BB \) is a multivector with grades 1 and 2 (vector and bivector components).
The velcoty \( c \) is called the group velocity since \( F \), or its components \( \BE, \BH \) satisfy the wave equation, which can be seen by pre-multiplying with \( \spacegrad - (1/c)\PDi{t}{} \) to find
\begin{dmath}\label{eqn:planewavesMultivector:300}
\lr{ \spacegrad^2 - \inv{c^2}\PDSq{t}{} } F(\Bx, t) = 0.
\end{dmath}

Let's look at the frequency domain solution of this equation with a presumed phasor representation
\begin{dmath}\label{eqn:planewavesMultivector:40}
F(\Bx, t) = \Real \lr{ F(\Bk) e^{-j \Bk \cdot \Bx + j \omega t} },
\end{dmath}
where \( j \) is a scalar imaginary, not necessarily with any geometric interpretation.

Maxwell's equation reduces to just
\begin{dmath}\label{eqn:planewavesMultivector:60}
0
=
-j \lr{ \Bk - \frac{\omega}{c} } F(\Bk).
\end{dmath}

If \( F(\Bk) \) has a left multivector factor
\begin{dmath}\label{eqn:planewavesMultivector:80}
F(\Bk) =
\lr{ \Bk + \frac{\omega}{c} } \tilde{F},
\end{dmath}
where \( \tilde{F} \) is a multivector to be determined, then
\begin{dmath}\label{eqn:planewavesMultivector:100}
\lr{ \Bk - \frac{\omega}{c} }
F(\Bk)
=
\lr{ \Bk - \frac{\omega}{c} }
\lr{ \Bk + \frac{\omega}{c} } \tilde{F}
=
\lr{ \Bk^2 - \lr{\frac{\omega}{c}}^2 } \tilde{F},
\end{dmath}
which is zero if
if \( \Norm{\Bk} = \ifrac{\omega}{c} \).

Let \( \kcap = \ifrac{\Bk}{\Norm{\Bk}} \), and \( \Norm{\Bk} \tilde{F} = F_0 + F_1 + F_2 + F_3 \), where
\( F_0, F_1, F_2, \) and \( F_3 \) are respectively have grades 0,1,2,3.  Then
\begin{dmath}\label{eqn:planewavesMultivector:120}
F(\Bk)
= \lr{ 1 + \kcap } \lr{ F_0 + F_1 + F_2 + F_3 }
=
F_0 + F_1 + F_2 + F_3
+
\kcap F_0 + \kcap F_1 + \kcap F_2 + \kcap F_3
=
F_0 + F_1 + F_2 + F_3
+
\kcap F_0 + \kcap \cdot F_1 + \kcap \cdot F_2 + \kcap \cdot F_3
+
\kcap \wedge F_1 + \kcap \wedge F_2
=
\lr{
   F_0 + \kcap \cdot F_1
}
+
\lr{
   F_1 + \kcap F_0 + \kcap \cdot F_2
}
+
\lr{
   F_2 + \kcap \cdot F_3 + \kcap \wedge F_1
}
+
\lr{
   F_3 + \kcap \wedge F_2
}.
\end{dmath}
Since the field \( F \) has only vector and bivector grades, the grades zero and three components of the expansion above must be zero, or
\begin{dmath}\label{eqn:planewavesMultivector:140}
\begin{aligned}
   F_0 &= - \kcap \cdot F_1 \\
   F_3 &= - \kcap \wedge F_2,
\end{aligned}
\end{dmath}
so
\begin{dmath}\label{eqn:planewavesMultivector:160}
F(\Bk)
=
\lr{ 1 + \kcap } \lr{
   F_1 - \kcap \cdot F_1 +
   F_2 - \kcap \wedge F_2
}
=
\lr{ 1 + \kcap } \lr{
   F_1 - \kcap F_1 + \kcap \wedge F_1 +
   F_2 - \kcap F_2 + \kcap \cdot F_2
}.
\end{dmath}
The multivector \( 1 + \kcap \) has the projective property of gobbling any leading factors of \( \kcap \)
\begin{equation}\label{eqn:planewavesMultivector:180}
(1 + \kcap)\kcap
= \kcap + 1
= 1 + \kcap,
\end{equation}
so for \( F_i \in F_1, F_2 \)
\begin{equation}\label{eqn:planewavesMultivector:200}
(1 + \kcap) ( F_i - \kcap F_i )
=
(1 + \kcap) ( F_i - F_i )
= 0,
\end{equation}
leaving
\begin{dmath}\label{eqn:planewavesMultivector:220}
F(\Bk)
=
\lr{ 1 + \kcap } \lr{
   \kcap \cdot F_2 +
   \kcap \wedge F_1
}.
\end{dmath}

For \( \kcap \cdot F_2 \) to be non-zero \( F_2 \) must be a bivector that lies in a plane containing \( \kcap \), and
\( \kcap \cdot F_2 \) is a vector in that plane that is perpendicular to \( \kcap \).
On the other hand \( \kcap \wedge F_1 \) is non-zero only if \( F_1 \) has a non-zero component that does not lie in along the \( \kcap \) direction, but \( \kcap \wedge F_1 \), like \( F_2 \) describes a plane that containing \( \kcap \).
This means that having both bivector and vector free variables \( F_2 \) and \( F_1 \) provide more degrees of freedom than required.
For example, if \( \BE \) is any vector, and \( F_2 = \kcap \wedge \BE \), then
\begin{dmath}\label{eqn:planewavesMultivector:240}
\lr{ 1 + \kcap }
   \kcap \cdot F_2
=
\lr{ 1 + \kcap }
   \kcap \cdot \lr{ \kcap \wedge \BE }
=
\lr{ 1 + \kcap }
\lr{
   \BE
-
\kcap \lr{ \kcap \cdot \BE }
}
=
\lr{ 1 + \kcap }
\kcap \lr{ \kcap \wedge \BE }
=
\lr{ 1 + \kcap }
\kcap \wedge \BE,
\end{dmath}
which has the form \( \lr{ 1 + \kcap } \lr{ \kcap \wedge F_1 } \), so the solution of the free space Maxwell's equation can be written
%\begin{boxed}\label{eqn:planewavesMultivector:260}
\boxedEquation{eqn:planewavesMultivector:280}{
F(\Bx, t)
=
\Real \lr{
\lr{ 1 + \kcap }
\BE\,
e^{-j \Bk \cdot \Bx + j \omega t}
}
,
}
%\end{boxed}
where \( \BE \) is any vector for which \( \BE \cdot \Bk = 0 \).

\section{OLD. Retain any}
%%
% Copyright © 2016 Peeter Joot.  All Rights Reserved.
% Licenced as described in the file LICENSE under the root directory of this GIT repository.
%
%\section{Plane waves.}
\index{plane wave}
\index{time harmonic}
\index{source free}
In the time harmonic representation for source free conditions Maxwell's equation \cref{dfn:isotropicMaxwells:680} is just
\begin{dmath}\label{eqn:planewaves:560}
\begin{aligned}
F &= \BE + \eta I \BH \\
\lr{ \spacegrad + j k } F &= 0,
\end{aligned}
\end{dmath}
where \( k = \omega/v \) is the wave number.
It is now possible to examine what constraints Maxwell's equation imposes on plane waves of the form
%\begin{dmath}\label{eqn:planewaves:580}
%\begin{aligned}
%\BE(\Bk) &= \BE_0 e^{-j \Bk \cdot \Bx} \\
%\BH(\Bk) &= \BH_0 e^{-j \Bk \cdot \Bx},
%\end{aligned}
%\end{dmath}
\begin{dmath}\label{eqn:planewaves:600}
F(\Bk) = F_0 e^{-j \Bk \cdot \Bx}.
\end{dmath}

where \( F = \Real( F(\Bk) e^{j \omega t} ) \).
Let's seek a phasor solution \( F(\Bk) \) for the total electromagnetic field, but drop the explicit frequency dependence for simplicity

%
% Copyright © 2016 Peeter Joot.  All Rights Reserved.
% Licenced as described in the file LICENSE under the root directory of this GIT repository.
%
%\section{Plane waves}
\index{plane wave}
The gradient action on the electromagnetic field is

\begin{dmath}\label{eqn:frequencydomainPlaneWaves:160}
\spacegrad F_0 e^{-j \Bk \cdot \Bx}
=
\sum_{m = 1}^3 \Be_m \partial_m
F_0 e^{-j \Bk \cdot \Bx}
=
\sum_{m = 1}^3 \Be_m
F_0
\lr{ -j k_m }
e^{-j \Bk \cdot \Bx}
=
-j \Bk F_0,
\end{dmath}
so

\begin{dmath}\label{eqn:frequencydomainPlaneWaves:180}
j k (1 - \kcap) F_0 = 0.
\end{dmath}

This means that the field must be of the form

%\begin{dmath}\label{eqn:frequencydomainPlaneWaves:200}
\boxedEquation
{eqn:frequencydomainPlaneWaves:200}
{
F = (1 + \kcap) \BE_0 e^{-j \Bk \cdot \Bx},
}
%\end{dmath}
where \( \BE_0 \) is a vector valued complex constant, and \( \kcap \cdot \BE_0 = 0 \).
The dot product constraint follows from the requirement that the \( I \BH \propto \kcap \BE_0 \) portion of the electromagnetic field is a bivector.
The time domain representation of the field is
\begin{dmath}\label{eqn:frequencydomainPlaneWaves:460}
F = (1 + \kcap) \Real{ \BE_0 e^{-j \Bk \cdot \Bx} },
\end{dmath}
but we will see later
instead of using a scalar imaginary \( j \), it is possible to use either the unit bivector for the transverse plane or the \R{3} unit pseudoscalar as the imaginary, and that a plane wave of any polarization can be encoded without any requirement to take real parts.

From \cref{eqn:frequencydomainPlaneWaves:200} the interdependence of the electric and magnetic field portions of the field can be read off immediately.
Those are

\begin{subequations}
\label{eqn:frequencydomainPlaneWaves:220}
\begin{dmath}\label{eqn:frequencydomainPlaneWaves:221}
\BE = \BE_0 e^{-j \Bk \cdot \Bx}
\end{dmath}
\begin{dmath}\label{eqn:frequencydomainPlaneWaves:222}
I \BH = \inv{\eta} \kcap \BE_0 e^{-j \Bk \cdot \Bx},
\end{dmath}
\end{subequations}

or
\begin{dmath}\label{eqn:frequencydomainPlaneWaves:380}
I \BH = \inv{\eta} \kcap \BE.
\end{dmath}

\index{pseudoscalar!spherical}
Since the \R{3} pseudoscalar can be written as

\begin{dmath}\label{eqn:frequencydomainPlaneWaves:400}
I = \kcap \Ecap \Hcap,
\end{dmath}
the directions \( \kcap, \Ecap, \Hcap \) must form a right handed triple.
It is thus expected that the magnetic field is perpendicular to the propagation direction, and that the electric and magnetic fields are explicitly perpendicular, facts that are easily verified

\begin{subequations}
\label{eqn:frequencydomainPlaneWaves:440}
\begin{dmath}\label{eqn:frequencydomainPlaneWaves:260}
\kcap \cdot \BH
= \gpgradezero{ \kcap (-I \kcap \BE_0) } e^{-j \Bk \cdot \Bx}
= -\gpgradezero{ I \BE_0 } e^{-j \Bk \cdot \Bx}
= 0
\end{dmath}
\begin{dmath}\label{eqn:frequencydomainPlaneWaves:280}
\BE \cdot \BH
=
\gpgradezero{ \BE \lr{ -\frac{I}{\eta}} \kcap \BE }
=
-\inv{\eta} \BE^2
\gpgradezero{ \kcap I }
=
0.
\end{dmath}
\end{subequations}

In conventional vector treatments of electromagnetic field theory the field relationships of \cref{eqn:frequencydomainPlaneWaves:220} and the propagation directions are written out explicitly as cross products, instead of multivector equations.
Those cross product relations are obtained easily

\begin{subequations}
\label{eqn:frequencydomainPlaneWaves:420}
\begin{dmath}\label{eqn:frequencydomainPlaneWaves:240}
\BH
= -I \inv{\eta} \kcap \BE
= -I \inv{\eta} (\kcap \wedge \BE)
= -I \inv{\eta} I (\kcap \cross \BE)
= \inv{\eta} \kcap \cross \BE
\end{dmath}
\begin{dmath}\label{eqn:frequencydomainPlaneWaves:300}
\BE
= \eta \kcap I \BH
= \eta I \kcap \wedge \BH
= \eta I^2 \kcap \cross \BH
= \eta \BH \cross \kcap
\end{dmath}
\begin{dmath}\label{eqn:frequencydomainPlaneWaves:340}
\kcap
= I \Hcap \Ecap
= I (\Hcap \wedge \Ecap)
= I^2 (\Hcap \cross \Ecap)
= \Ecap \cross \Hcap.
\end{dmath}
\end{subequations}



%
% Copyright � 2016 Peeter Joot.  All Rights Reserved.
% Licenced as described in the file LICENSE under the root directory of this GIT repository.
%
%\section{Plane waves.}
\index{plane wave}
\index{time harmonic}
\index{source free}
In the time harmonic representation for source free conditions Maxwell's equation \cref{dfn:isotropicMaxwells:680} is just
\begin{dmath}\label{eqn:planewaves:560}
\begin{aligned}
F &= \BE + \eta I \BH \\
\lr{ \spacegrad + j k } F &= 0,
\end{aligned}
\end{dmath}
where \( k = \omega/v \) is the wave number.
It is now possible to examine what constraints Maxwell's equation imposes on plane waves of the form
%\begin{dmath}\label{eqn:planewaves:580}
%\begin{aligned}
%\BE(\Bk) &= \BE_0 e^{-j \Bk \cdot \Bx} \\
%\BH(\Bk) &= \BH_0 e^{-j \Bk \cdot \Bx},
%\end{aligned}
%\end{dmath}
\begin{dmath}\label{eqn:planewaves:600}
F(\Bk) = F_0 e^{-j \Bk \cdot \Bx}.
\end{dmath}

where \( F = \Real( F(\Bk) e^{j \omega t} ) \).
Let's seek a phasor solution \( F(\Bk) \) for the total electromagnetic field, but drop the explicit frequency dependence for simplicity

%%
% Copyright © 2016 Peeter Joot.  All Rights Reserved.
% Licenced as described in the file LICENSE under the root directory of this GIT repository.
%
%\section{Plane waves}
\index{plane wave}
The gradient action on the electromagnetic field is

\begin{dmath}\label{eqn:frequencydomainPlaneWaves:160}
\spacegrad F_0 e^{-j \Bk \cdot \Bx}
=
\sum_{m = 1}^3 \Be_m \partial_m
F_0 e^{-j \Bk \cdot \Bx}
=
\sum_{m = 1}^3 \Be_m
F_0
\lr{ -j k_m }
e^{-j \Bk \cdot \Bx}
=
-j \Bk F_0,
\end{dmath}
so

\begin{dmath}\label{eqn:frequencydomainPlaneWaves:180}
j k (1 - \kcap) F_0 = 0.
\end{dmath}

This means that the field must be of the form

%\begin{dmath}\label{eqn:frequencydomainPlaneWaves:200}
\boxedEquation
{eqn:frequencydomainPlaneWaves:200}
{
F = (1 + \kcap) \BE_0 e^{-j \Bk \cdot \Bx},
}
%\end{dmath}
where \( \BE_0 \) is a vector valued complex constant, and \( \kcap \cdot \BE_0 = 0 \).
The dot product constraint follows from the requirement that the \( I \BH \propto \kcap \BE_0 \) portion of the electromagnetic field is a bivector.
The time domain representation of the field is
\begin{dmath}\label{eqn:frequencydomainPlaneWaves:460}
F = (1 + \kcap) \Real{ \BE_0 e^{-j \Bk \cdot \Bx} },
\end{dmath}
but we will see later
instead of using a scalar imaginary \( j \), it is possible to use either the unit bivector for the transverse plane or the \R{3} unit pseudoscalar as the imaginary, and that a plane wave of any polarization can be encoded without any requirement to take real parts.

From \cref{eqn:frequencydomainPlaneWaves:200} the interdependence of the electric and magnetic field portions of the field can be read off immediately.
Those are

\begin{subequations}
\label{eqn:frequencydomainPlaneWaves:220}
\begin{dmath}\label{eqn:frequencydomainPlaneWaves:221}
\BE = \BE_0 e^{-j \Bk \cdot \Bx}
\end{dmath}
\begin{dmath}\label{eqn:frequencydomainPlaneWaves:222}
I \BH = \inv{\eta} \kcap \BE_0 e^{-j \Bk \cdot \Bx},
\end{dmath}
\end{subequations}

or
\begin{dmath}\label{eqn:frequencydomainPlaneWaves:380}
I \BH = \inv{\eta} \kcap \BE.
\end{dmath}

\index{pseudoscalar!spherical}
Since the \R{3} pseudoscalar can be written as

\begin{dmath}\label{eqn:frequencydomainPlaneWaves:400}
I = \kcap \Ecap \Hcap,
\end{dmath}
the directions \( \kcap, \Ecap, \Hcap \) must form a right handed triple.
It is thus expected that the magnetic field is perpendicular to the propagation direction, and that the electric and magnetic fields are explicitly perpendicular, facts that are easily verified

\begin{subequations}
\label{eqn:frequencydomainPlaneWaves:440}
\begin{dmath}\label{eqn:frequencydomainPlaneWaves:260}
\kcap \cdot \BH
= \gpgradezero{ \kcap (-I \kcap \BE_0) } e^{-j \Bk \cdot \Bx}
= -\gpgradezero{ I \BE_0 } e^{-j \Bk \cdot \Bx}
= 0
\end{dmath}
\begin{dmath}\label{eqn:frequencydomainPlaneWaves:280}
\BE \cdot \BH
=
\gpgradezero{ \BE \lr{ -\frac{I}{\eta}} \kcap \BE }
=
-\inv{\eta} \BE^2
\gpgradezero{ \kcap I }
=
0.
\end{dmath}
\end{subequations}

In conventional vector treatments of electromagnetic field theory the field relationships of \cref{eqn:frequencydomainPlaneWaves:220} and the propagation directions are written out explicitly as cross products, instead of multivector equations.
Those cross product relations are obtained easily

\begin{subequations}
\label{eqn:frequencydomainPlaneWaves:420}
\begin{dmath}\label{eqn:frequencydomainPlaneWaves:240}
\BH
= -I \inv{\eta} \kcap \BE
= -I \inv{\eta} (\kcap \wedge \BE)
= -I \inv{\eta} I (\kcap \cross \BE)
= \inv{\eta} \kcap \cross \BE
\end{dmath}
\begin{dmath}\label{eqn:frequencydomainPlaneWaves:300}
\BE
= \eta \kcap I \BH
= \eta I \kcap \wedge \BH
= \eta I^2 \kcap \cross \BH
= \eta \BH \cross \kcap
\end{dmath}
\begin{dmath}\label{eqn:frequencydomainPlaneWaves:340}
\kcap
= I \Hcap \Ecap
= I (\Hcap \wedge \Ecap)
= I^2 (\Hcap \cross \Ecap)
= \Ecap \cross \Hcap.
\end{dmath}
\end{subequations}

%
% Copyright � 2016 Peeter Joot.  All Rights Reserved.
% Licenced as described in the file LICENSE under the root directory of this GIT repository.
%
%\section{Plane waves}
\index{plane wave}
The gradient action on the electromagnetic field is
\begin{dmath}\label{eqn:frequencydomainPlaneWaves:160}
\spacegrad F_0 e^{-j \Bk \cdot \Bx}
=
\sum_{m = 1}^3 \Be_m \partial_m
F_0 e^{-j \Bk \cdot \Bx}
=
\sum_{m = 1}^3 \Be_m
F_0
\lr{ -j k_m }
e^{-j \Bk \cdot \Bx}
=
-j \Bk F_0,
\end{dmath}
so
\begin{dmath}\label{eqn:frequencydomainPlaneWaves:180}
j k (1 - \kcap) F_0 = 0.
\end{dmath}

This means that the field must be of the form
%\begin{dmath}\label{eqn:frequencydomainPlaneWaves:200}
\boxedEquation
{eqn:frequencydomainPlaneWaves:200}
{
F = (1 + \kcap) \BE_0 e^{-j \Bk \cdot \Bx},
}
%\end{dmath}
where \( \BE_0 \) is a vector valued complex constant, and \( \kcap \cdot \BE_0 = 0 \).
The dot product constraint follows from the requirement that the \( I \BH \propto \kcap \BE_0 \) portion of the electromagnetic field is a bivector.
The time domain representation of the field is
\begin{dmath}\label{eqn:frequencydomainPlaneWaves:460}
F = (1 + \kcap) \Real{ \BE_0 e^{-j \Bk \cdot \Bx} },
\end{dmath}
but we will see later
instead of using a scalar imaginary \( j \), it is possible to use either the unit bivector for the transverse plane or the \R{3} unit pseudoscalar as the imaginary, and that a plane wave of any polarization can be encoded without any requirement to take real parts.

From \cref{eqn:frequencydomainPlaneWaves:200} the interdependence of the electric and magnetic field portions of the field can be read off immediately.
Those are

\begin{subequations}
\label{eqn:frequencydomainPlaneWaves:220}
\begin{dmath}\label{eqn:frequencydomainPlaneWaves:221}
\BE = \BE_0 e^{-j \Bk \cdot \Bx}
\end{dmath}
\begin{dmath}\label{eqn:frequencydomainPlaneWaves:222}
I \BH = \inv{\eta} \kcap \BE_0 e^{-j \Bk \cdot \Bx},
\end{dmath}
\end{subequations}

or
\begin{dmath}\label{eqn:frequencydomainPlaneWaves:380}
I \BH = \inv{\eta} \kcap \BE.
\end{dmath}

\index{pseudoscalar!spherical}
Since the \R{3} pseudoscalar can be written as
\begin{dmath}\label{eqn:frequencydomainPlaneWaves:400}
I = \kcap \Ecap \Hcap,
\end{dmath}
the directions \( \kcap, \Ecap, \Hcap \) must form a right handed triple.
It is thus expected that the magnetic field is perpendicular to the propagation direction, and that the electric and magnetic fields are explicitly perpendicular, facts that are easily verified

\begin{subequations}
\label{eqn:frequencydomainPlaneWaves:440}
\begin{dmath}\label{eqn:frequencydomainPlaneWaves:260}
\kcap \cdot \BH
= \gpgradezero{ \kcap (-I \kcap \BE_0) } e^{-j \Bk \cdot \Bx}
= -\gpgradezero{ I \BE_0 } e^{-j \Bk \cdot \Bx}
= 0
\end{dmath}
\begin{dmath}\label{eqn:frequencydomainPlaneWaves:280}
\BE \cdot \BH
=
\gpgradezero{ \BE \lr{ -\frac{I}{\eta}} \kcap \BE }
=
-\inv{\eta} \BE^2
\gpgradezero{ \kcap I }
=
0.
\end{dmath}
\end{subequations}

In conventional vector treatments of electromagnetic field theory the field relationships of \cref{eqn:frequencydomainPlaneWaves:220} and the propagation directions are written out explicitly as cross products, instead of multivector equations.
Those cross product relations are obtained easily

\begin{subequations}
\label{eqn:frequencydomainPlaneWaves:420}
\begin{dmath}\label{eqn:frequencydomainPlaneWaves:240}
\BH
= -I \inv{\eta} \kcap \BE
= -I \inv{\eta} (\kcap \wedge \BE)
= -I \inv{\eta} I (\kcap \cross \BE)
= \inv{\eta} \kcap \cross \BE
\end{dmath}
\begin{dmath}\label{eqn:frequencydomainPlaneWaves:300}
\BE
= \eta \kcap I \BH
= \eta I \kcap \wedge \BH
= \eta I^2 \kcap \cross \BH
= \eta \BH \cross \kcap
\end{dmath}
\begin{dmath}\label{eqn:frequencydomainPlaneWaves:340}
\kcap
= I \Hcap \Ecap
= I (\Hcap \wedge \Ecap)
= I^2 (\Hcap \cross \Ecap)
= \Ecap \cross \Hcap.
\end{dmath}
\end{subequations}

%}
%\EndNoBibArticle
