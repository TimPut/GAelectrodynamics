%
% Copyright � 2018 Peeter Joot.  All Rights Reserved.
% Licenced as described in the file LICENSE under the root directory of this GIT repository.
%
%{
\input{../latex/blogpost.tex}
\renewcommand{\basename}{ga20}
%\renewcommand{\dirname}{notes/phy1520/}
\renewcommand{\dirname}{notes/ece1228-electromagnetic-theory/}
%\newcommand{\dateintitle}{}
%\newcommand{\keywords}{}

\input{../latex/peeter_prologue_print2.tex}

\usepackage{peeters_layout_exercise}
\usepackage{peeters_braket}
\usepackage{peeters_figures}
\usepackage{siunitx}
%\usepackage{mhchem} % \ce{}
%\usepackage{macros_bm} % \bcM
%\usepackage{macros_qed} % \qedmarker
%\usepackage{txfonts} % \ointclockwise

\beginArtNoToc

\generatetitle{Geometric Algebra for \R{2}}
%\chapter{XXX}
%\label{chap:ga20}

\section{Preliminaries}

\makedefinition{Reverse of vector products.}{dfn:ga20:260}{
The reverse of a multivector product of \( k \) vectors
\( M = \Bu_1 \Bu_2 \cdots \Bu_k \) is a multivector with the order of all vector factors reversed
\( M^\dagger = \Bu_k \cdots \Bu_2 \Bu_1 \).
} % definition

Examples
\begin{itemize}
\item \( (\Be_1 \Be_2)^\dagger = \Be_2 \Be_1, (\Be_2 \Be_4)^\dagger = \Be_4 \Be_2, \)
\item \( (\Be_1 \Be_2 \Be_3)^\dagger = \Be_3 \Be_2 \Be_1. \)
\end{itemize}

\Cref{dfn:ga20:260} also applies indirectly to scalars and vectors.  We can show easily that scalars and vectors are unchanged under reversion operations.

\maketheorem{Reverse of a scalar.}{thm:ga20:scalarReverse}{
A scalar is its own reverse.  That is \( s^\dagger = s, \forall s \in R \).
} % theorem

To prove \cref{thm:ga20:scalarReverse}, factor \( s \) into a product of colinear vectors \( \Bx = a \Bu, \By = b \Bu \), where \( s = \Bx \By = a b \Bu^2 \).  The reverse of \( s \) is
\begin{dmath}\label{eqn:ga20:400}
s^\dagger = \By \Bx = b \Bu a \Bu = a b \Bu^2 = s.
\end{dmath}

\maketheorem{Reverse of a vector.}{thm:ga20:vectorReverse}{
A vector is its own reverse.  That is \( \Bu^\dagger = \Bu, \forall \Bu \in \bbR \).
} % theorem

To prove \cref{thm:ga20:vectorReverse}, let \( \Bu, \Bv \) be a pair of vectors, where \( \Bv \) is a unit vector.  Then
\begin{dmath}\label{eqn:ga20:420}
\Bu^\dagger
= \lr{ \Bu \Bv \Bv }^\dagger
= \Bv \Bv \Bu
= \Bu.
\end{dmath}

%\item \( 3^\dagger = 3, 0^\dagger = 0, \)
%\item \( \Be_1^\dagger = \Be_1, \Be_3^\dagger = \Be_3, \)

%\makedefinition{Reverse of a k-vector.}{dfn:ga20:360}{
%The reverse of a sum of (same-k) k-blades \( f = k_1 + k_2 + \cdots \) is the sum of the reverses of each such k-blade, that is
%\( f^\dagger = k_1^\dagger + k_2^\dagger + \cdots \).
%} % definition
%Examples
%\begin{itemize}
%\item \( (\Be_1 + \Be_2 - \Be_3)^\dagger = \Be_1^\dagger + \Be_2^\dagger - \Be_3^\dagger = \Be_1 + \Be_2 - \Be_3, \)
%\item \( (\Be_1 \Be_2 + \Be_2 \Be_3)^\dagger = \Be_2 \Be_1 + \Be_3 \Be_2, (\Be_1 \Be_2 + \Be_3 \Be_4)^\dagger = \Be_2 \Be_1 + \Be_4 \Be_3, \)
%\item \( (3 \Be_1 \Be_2 + 7 \Be_3 \Be_2 \Be_4)^\dagger = 3 \Be_2 \Be_1 + 7 \Be_4 \Be_2 \Be_3.\)
%\end{itemize}
%
\makedefinition{Multivector reverse}{dfn:ga20:280}{
The reverse of a sum is the sum of the reverse of the summands.
} % definition
In particular, the reverse of a multivector 
\( M = \sum_{k = 1}^N \gpgrade{M}{k} \)
is the sum of the reverse of the constituent grades
\( M^\dagger = \sum_{k = 1}^N \lr{\gpgrade{M}{k}}^\dagger \).
Examples
\begin{itemize}
\item \( (3 + \Be_1 \Be_2)^\dagger = 3^\dagger + (\Be_1 \Be_2)^\dagger = 3 + \Be_2 \Be_1 \)
\item \( (\Be_2 + \Be_3 \Be_2 \Be_1)^\dagger = \Be_2^\dagger + (\Be_3 \Be_2 \Be_1)^\dagger = \Be_2 + \Be_1 \Be_2 \Be_3 \).
\end{itemize}

%\maketheorem{Reverse of a product.}{thm:ga20:440}{
%} % theorem

\section{Geometric Algebra for \R{2}}
The geometric algebra of \R{2} is strongly associated with the algebra of complex numbers.  Here we will explore some of those connections, and use this simplest geometric algebra to develop our geometric algebra skills.

\maketheorem{Basis for the \R{2} geometric algebra.}{thm:ga20:1}{
Given an orthonormal basis \( \setlr{\Be_1, \Be_2} \) for \R{2},
the basis of the \R{2} geometric algebra (often called the geometric algebra generated by \R{2}), has the following elements
\begin{center}
\begin{tabular}{| l | c c c |}
\hline
scalar   &              & 1                 &               \\ \hline
vector   & \( \Be_1 \)  &                   & \( \Be_2 \)   \\ \hline
bivector &              & \( \Be_1 \Be_2 \) &               \\ \hline
\end{tabular}
\end{center}
That is, any \R{2} multivector can be put into the form \( M = a + b \Be_1 + c \Be_2 + d \Be_1 \Be_2 \), where \( a, b, c, d \) are scalars.
} % theorem
This theorem is essentially an assertion that
products of the \R{2} basis vectors \( \Be_1, \Be_2\) may be used to form scalars, vectors, or bivectors, but not trivectors.  For example
\begin{itemize}
\item \( \Be_1 \Be_1 = 1, \Be_2 \Be_2 = 1, \Be_1 \Be_2 \Be_1 \Be_2 = -1, \cdots \)
\item \( \Be_1 \Be_2 \Be_1 = -\Be_2, \Be_2 \Be_1 \Be_2 = -\Be_1, \Be_1 \Be_2 \Be_2 \Be_1 \Be_2 = \Be_2, \cdots \)
\item \( \Be_1 \Be_2, \Be_2 \Be_1 \Be_2 \Be_1 \Be_2 \Be_1 = \Be_1 \Be_2, \cdots \).
\end{itemize}
The reader should attempt to prove this to themselves.  We will also walk through portions of the proof below, with the help of a few auxillary theorems.

\maketheorem{Imaginary nature of the unit bivector.}{thm:ga20:imaginary}{
If \( \Bu, \Bv \) are two orthonormal bivectors, their products \( \Bu\Bv \), \(\Bv \Bu\), called unit bivectors, have a complex imaginary nature.
That is,
\begin{equation*}
(\Bu\Bv)^2 = -1.
\end{equation*}
} % theorem
The proof is simple, and is independent of the dimension of the underlying vector space
\begin{dmath}\label{eqn:ga20:100}
(\Bu\Bv)^2 = (\Bu\Bv)(\Bu\Bv)
           = -(\Bv\Bu)(\Bu\Bv)
           = -\Bv(\Bu\Bu)\Bv
           = -\Bv^2
           = -1.
\end{dmath}
It happens that all \R{2} unit bivectors are equal to \( \Be_1 \Be_2 \), up to a sign.
To demonstrate, construct a general pair of orthonormal \R{2} vectors, letting
\( \Bu = \Be_1 \cos\theta + \Be_2 \sin\theta \), and \( \Bv = \pm\lr{ \Be_2 \cos\theta - \Be_1 \sin\theta } \), which have the product
\begin{dmath}\label{eqn:ga20:120}
\Bu \Bv
=
\pm \lr{ \Be_1 \cos\theta + \Be_2 \sin\theta } \lr{ \Be_2 \cos\theta - \Be_1 \sin\theta }
=
\pm \lr{ \Be_1 \Be_2 \cos^2\theta - \Be_2 \Be_1 \sin^2 \theta - \Be_1^2 \cos\theta \sin\theta + \Be_2^2 \sin\theta \cos\theta }
=
\pm \lr{ \Be_1 \Be_2 } \lr{ \cos^2\theta + \sin^2 \theta }
=
\pm \lr{ \Be_1 \Be_2 }.
\end{dmath}
\index{i}
\index{imaginary}
In this book, we will use the symbol \( i \) to represent one such ``imaginary'' bivector.
\makedefinition{\R{2} imaginary.}{dfn:ga20:imaginary}{
For the \R{2} geometric algebra, we define
\begin{equation*}
i = \Be_1 \Be_2.
\end{equation*}
} % definition
This choice is arbitrary, and there is no reason, other than aesthetic sensibility, that we could not have defined \( i = \Be_2 \Be_1 \) instead\footnote{Later we will see that this choice effects the sense of our rotational operators.}.
In this book, we will also use the symbol \( i \) to designate \R{3} unit bivectors representing planes of interest (not neccessarily the x-y plane), but will always provide a local definition of \( i \) when this is done.

\makedefinition{Reverse.}{dfn:ga20:300}{
} % definition


It is important to point out that \R{2} vectors do not commute with \( i \).
\maketheorem{Vector anticommutation with the imaginary.}{thm:ga20:imaginaryanticommutation}{
Given any \R{2} vector \( \Bx \)
\begin{equation*}
\Bx i = -i \Bx.
\end{equation*}
} % theorem
This anticommutation property can be first proven for each of the unit vectors
\begin{dmath}\label{eqn:ga20:140}
\Be_1 i
= \Be_1 (\Be_1 \Be_2)
= -\Be_1 (\Be_2 \Be_1)
= -(\Be_1 \Be_2) \Be_1
= - i \Be_1,
\end{dmath}
and
\begin{dmath}\label{eqn:ga20:160}
\Be_2 i
= \Be_2 \Be_1 \Be_2
= (\Be_2 \Be_1) \Be_2
= -(\Be_1 \Be_2) \Be_2
= - i \Be_2.
\end{dmath}
The general result follow by superposition.

Returning to \cref{thm:ga20:1}, let's consider an example vector product
\begin{dmath}\label{eqn:ga20:20}
A = \Be_1 \Be_2 \Be_2 \Be_1 \Be_2 \Be_1 \Be_2 \Be_1 \Be_2 \Be_2 \Be_2 .
\end{dmath}
Utilizing the contraction axiom (\( \Be_1 \Be_1 = \Be_2 \Be_2 = 1 \)), all the pairs of adjacent like vectors can be eliminated iteratively
\begin{dmath}\label{eqn:ga20:40}
A
= \Be_1 \cancel{(\Be_2 \Be_2)} \Be_1 \Be_2 \Be_1 \Be_2 \Be_1 \cancel{(\Be_2 \Be_2)} \Be_2
= \cancel{(\Be_1 \Be_1)} \Be_2 \Be_1 \Be_2 \Be_1 \Be_2
= \Be_2 \Be_1 \Be_2 \Be_1 \Be_2 ,
\end{dmath}
until we are left with \( \pm 1 \) or the remaining products are all alternating products of \( \Be_1, \Be_2 \).
When such product has an even number of factors, it must be a scalar or bivector, since by
\cref{thm:ga20:imaginary}
\( (\Be_1 \Be_2)^{2k} = -1^k = (\Be_2 \Be_1)^{2k} \).
When the product \cref{eqn:ga20:40} has an odd number of factors, it can be similarily reduced to product of one or three factors, but the case of three such factors is really just a vector in disguise
\begin{dmath}\label{eqn:ga20:80}
\begin{aligned}
\Be_1 \Be_2 \Be_1 &= -(\Be_2 \Be_1) \Be_1 = -\Be_2 \\
\Be_2 \Be_1 \Be_2 &= -(\Be_1 \Be_2) \Be_2 = -\Be_1.
\end{aligned}
\end{dmath}

\makedefinition{Complex number.}{dfn:ga20:180}{
An even grade multivector of the form \( z = x + i y \) where \( x, y \in \bbR \), \( i = \Bu \Bv \), and \( \Bu, \Bv \) are orthonormal vectors, will be called a complex number for the plane spanned by \( \Bu, \Bv \).  When \( \Bu, \Bv \) are the unit vectors \( \Be_1, \Be_2 \in \text{\R{2}} \) respectively, \( z \) will be called a complex number.
} % definition

\makedefinition{Conjugate.}{dfn:ga20:460}{
The geometric reverse operator is used to denote conjugation.  With \( z = x + i y \), that is, \( z^\dagger = x - i y \).
} % definition

\maketheorem{Complex representation of vectors in a plane.}{thm:ga20:180}{
Any vector may be expressed as a product of a complex number and a unit vector.
} % theorem
Given an \R{2} vector \( \Bx = x \Be_1 + y \Be_2 \) we may factor either unit vector to the left or right
\begin{dmath}\label{eqn:ga20:200}
\begin{aligned}
\Bx &= \Be_1 \lr{ x + \Be_1 \Be_2 y } = \Be_2 \lr{ y - \Be_1 \Be_2 x } \\
    &= \lr{ x - \Be_1 \Be_2 y } \Be_1 = \lr{ y + \Be_1 \Be_2 x } \Be_2,
\end{aligned}
\end{dmath}
or
\begin{dmath}\label{eqn:ga20:220}
\begin{aligned}
\Bx &= \Be_1 \lr{ x + i y } = \Be_2 \lr{ y - i x } \\
    &= \lr{ x - i y } \Be_1 = \lr{ y + i x } \Be_2,
\end{aligned}
\end{dmath}
More generally, given a vector \( \Bx = x \Bu + y \Bv \) lying in the plane spanned by orthonormal vectors \( \Bu, \Bv \), then we may write \( \Bx \) in terms of various complex factors
\begin{dmath}\label{eqn:ga20:240}
\begin{aligned}
\Bx &= \Bu \lr{ x + \Bu \Bv y } = \Bv \lr{ y - \Bu \Bv x } \\
    &= \lr{ x - \Bu \Bv y } \Bu = \lr{ y + \Bu \Bv x } \Bv.
\end{aligned}
\end{dmath}
By setting \( i = \Bu \Bv \) \cref{eqn:ga20:240} can be put into the form \cref{eqn:ga20:220}, showing that complex numbers for a plane may be factored from any vectors that are constrained to that plane.  In this book, we will make extensive use of the ability to introduce ad-hoc complex numbers for any plane of convienence.

\maketheorem{Product of a vector and complex number.}{thm:ga20:320}{
The product \( \Bx' = \Bx z = z^\dagger \Bx \), where
\( \Bx = a \Be_1 + b \Be_2 \in \text{R{2}}\), and \( z = x + i y \) is a complex number, is another vector \( \Bx' \in \text{\R{2}} \).
} % theorem
%FIXME: define reverse or reference definition.
To demonstrate that the product is a vector, we can compute one the products
\begin{dmath}\label{eqn:ga20:340}
\Bx z
=
\lr{  a \Be_1 + b \Be_2 } \lr{ x + i y }
=
a x \Be_1 + b x \Be_2
+ a y \Be_2 - b y \Be_1
=
\Be_1 \lr{ a x - b y } + \Be_2 \lr{ b x + a y }.
\end{dmath}
The conjugation relationship follows from \cref{thm:ga20:imaginaryanticommutation}
\begin{dmath}\label{eqn:ga20:380}
\Bx \lr{ x + i y }
=
x \Bx + y \Bx i
=
x \Bx - y i \Bx
=
\lr{ x + i y }^\dagger \Bx
= z^\dagger \Bx.
\end{dmath}



%}
\EndArticle
%\EndNoBibArticle
