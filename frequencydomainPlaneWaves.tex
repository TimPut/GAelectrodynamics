%
% Copyright © 2016 Peeter Joot.  All Rights Reserved.
% Licenced as described in the file LICENSE under the root directory of this GIT repository.
%
%\section{Plane waves}

The gradient action on the electromagnetic field is

\begin{dmath}\label{eqn:frequencydomainCore:160}
\spacegrad F_0 e^{-j \Bk \cdot \Bx}
=
\sum_{m = 1}^3 \Be_m \partial_m
F_0 e^{-j \Bk \cdot \Bx}
=
\sum_{m = 1}^3 \Be_m
F_0
\lr{ -j k_m }
e^{-j \Bk \cdot \Bx}
=
-j \Bk F_0,
\end{dmath}

so

\begin{dmath}\label{eqn:frequencydomainCore:180}
j k (1 - \kcap) F_0 = 0.
\end{dmath}

This means that the field must be of the form

%\begin{dmath}\label{eqn:frequencydomainCore:200}
\boxedEquation
{eqn:frequencydomainCore:200}
{
F = (1 + \kcap) \BE_0 e^{-j \Bk \cdot \Bx},
}
%\end{dmath}

where \( \BE_0 \) is a vector valued complex constant, and \( \kcap \cdot \BE_0 = 0 \).
The dot product constraint follows from the requirement that the \( I \BH \propto \kcap \BE_0 \) portion of the electromagnetic field is a bivector.

From \cref{eqn:frequencydomainCore:200} the interdependence of the electric and magnetic field portions of the field can be read off immediately.
Those are

\begin{subequations}
\label{eqn:frequencydomainCore:220}
\begin{dmath}\label{eqn:frequencydomainCore:221}
\BE = \BE_0 e^{-j \Bk \cdot \Bx} 
\end{dmath}
\begin{dmath}\label{eqn:frequencydomainCore:222}
I \BH = \inv{\eta} \kcap \BE_0 e^{-j \Bk \cdot \Bx},
\end{dmath}
\end{subequations}

or
\begin{dmath}\label{eqn:frequencydomainCore:380}
I \BH = \inv{\eta} \kcap \BE.
\end{dmath}

Since the \R{3} pseudoscalar can be written as

\begin{dmath}\label{eqn:frequencydomainCore:400}
I = \kcap \Ecap \Hcap,
\end{dmath}

the directions \( \kcap, \Ecap, \Hcap \) must form a right handed triple.
It is thus expected that the magnetic field is perpendicular to the propagation direction, and that the electric and magnetic fields are explicitly perpendicular, facts that are easily verified

\begin{subequations}
\label{eqn:frequencydomainCore:440}
\begin{dmath}\label{eqn:frequencydomainCore:260}
\kcap \cdot \BH
= \gpgradezero{ \kcap (-I \kcap \BE_0) } e^{-j \Bk \cdot \Bx}
= -\gpgradezero{ I \BE_0 } e^{-j \Bk \cdot \Bx}
= 0
\end{dmath}
\begin{dmath}\label{eqn:frequencydomainCore:280}
\BE \cdot \BH
=
\gpgradezero{ \BE \lr{ -\frac{I}{\eta}} \kcap \BE }
=
-\inv{\eta} \BE^2
\gpgradezero{ \kcap I }
=
0.
\end{dmath}
\end{subequations}

In conventional vector treatments of electromagnetic field theory the field relationships of \cref{eqn:frequencydomainCore:220} and the propagation directions are written out explicitly as cross products, instead of multivector equations.
Those cross product relations are obtained easily

\begin{subequations}
\label{eqn:frequencydomainCore:420}
\begin{dmath}\label{eqn:frequencydomainCore:240}
\BH
= -I \inv{\eta} \kcap \BE
= -I \inv{\eta} (\kcap \wedge \BE)
= -I \inv{\eta} I (\kcap \cross \BE)
= \inv{\eta} \kcap \cross \BE
\end{dmath}
\begin{dmath}\label{eqn:frequencydomainCore:300}
\BE
= \eta \kcap I \BH
= \eta I \kcap \wedge \BH
= \eta I^2 \kcap \cross \BH
= \eta \BH \cross \kcap
\end{dmath}
\begin{dmath}\label{eqn:frequencydomainCore:340}
\kcap
= I \Hcap \Ecap
= I (\Hcap \wedge \Ecap)
= I^2 (\Hcap \cross \Ecap)
= \Ecap \cross \Hcap.
\end{dmath}
\end{subequations}
