%
% Copyright � 2018 Peeter Joot.  All Rights Reserved.
% Licenced as described in the file LICENSE under the root directory of this GIT repository.
%
%{
%%%\input{../latex/blogpost.tex}
%%%\renewcommand{\basename}{lineintegral}
%%%%\renewcommand{\dirname}{notes/phy1520/}
%%%\renewcommand{\dirname}{notes/ece1228-electromagnetic-theory/}
%%%%\newcommand{\dateintitle}{}
%%%%\newcommand{\keywords}{}
%%%
%%%\input{../latex/peeter_prologue_print2.tex}
%%%
%%%\usepackage{peeters_layout_exercise}
%%%\usepackage{peeters_braket}
%%%\usepackage{peeters_figures}
%%%\usepackage{siunitx}
%%%%\usepackage{mhchem} % \ce{}
%%%%\usepackage{macros_bm} % \bcM
%%%%\usepackage{macros_qed} % \qedmarker
%%%%\usepackage{txfonts} % \ointclockwise
%%%
%%%\beginArtNoToc
%%%
%%%\generatetitle{Multivector line integral.}
%\section{Line integral.}
\label{chap:lineintegral}

The line integral specialization of \cref{thm:fundamentalTheoremOfCalculus:1} is

%
% Copyright � 2018 Peeter Joot.  All Rights Reserved.
% Licenced as described in the file LICENSE under the root directory of this GIT repository.
%
\maketheorem{Multivector line integral.}{thm:lineintegral:100}{
Given an connected curve \( C \) parameterized by a single parameter, and multivector functions \( F, G \), the line integral is related to the boundary by
\begin{equation*}
\int_C F d^1\Bx \boldpartial G
= \evalbar{F G}{\Delta a}.
\end{equation*}
} % theorem


The differential form \( d\Bx = d^1 \Bx = du\, \Bx_u = du \PDi{u}{\Bx} \) varies over the curve,
and the vector derivative is just \( \boldpartial = \Bx^u \partial_u \) (no sum).

The proof follows by expansion.
For
\begin{dmath}\label{eqn:lineintegral:120}
\int_C F d\Bx\, \lrboldpartial G
=
\int_C \lr{ F d\Bx \lboldpartial} G
+
\int_C F d\Bx \lr{ \rboldpartial G }
=
\int_C \PD{u}{F} du\, \Bx_u \Bx^u G
+
\int_C F du\, \Bx_u \Bx^u \PD{u}{G}
=
\int_C du\, \PD{u}{F} G
+
\int_C du\, F \PD{u}{G}
=
\int_C du\, \PD{u}{} \lr{ F G }
=
F(u_1) G(u_1) -
F(u_0) G(u_0),
\end{dmath}
%where the boundaries of the parameterization have been assumed to be
%\( C: u \in [u_0, u_1] \).
We have a perfect cancellation of the reciprocal frame \( \Bx^u \) with the vector \( \Bx_u \) that lies along the curve, since \( \Bx^u \Bx_u = 1 \).  This leaves a perfect derivative of the product of \( F G \), which can be integrated over the length of the curve, yielding the difference of the product with respect to the parameterization of the end points of the curve.

For a single parameter subspace
the reciprocal frame vector \( \Bx^u \)
is trivial to calculate, as it is just the inverse of \( \Bx_u \), that is \( \Bx^u = \Bx_u/\Norm{\Bx_u}^2 \).
Observe that we did not actually have to calculate it, but instead only require that the vector is invertible.

An important (and familiar) special case of \cref{thm:lineintegral:100} is the fundamental theorem of calculus for line integrals, which can be obtained by using a
single scalar function \( f \)

%
% Copyright � 2018 Peeter Joot.  All Rights Reserved.
% Licenced as described in the file LICENSE under the root directory of this GIT repository.
%
\maketheorem{Line integral of a scalar function (Stokes').}{thm:lineintegral:180}{
Given a scalar function \( f \), its \textit{line integral} is given by
\begin{equation*}
\int_C d^1\Bx \cdot \boldpartial f =
\int_C d^1\Bx \cdot \spacegrad f = \evalbar{f}{\Delta a}.
\end{equation*}
} % theorem


\Cref{thm:lineintegral:180} is no doubt familiar in its gradient form.  Our proof starts with
\cref{thm:lineintegral:100} setting \( F = 1, G = f(\Bx(u)) \)
\begin{dmath}\label{eqn:lineintegral:140}
\int_C d\Bx\, \boldpartial f = \evalbar{f}{\Delta u},
\end{dmath}
which is a multivector equation with scalar and bivector grades on the left hand side, but only scalar grades on the right.  Equating grades yields two equations
\begin{subequations}
\label{eqn:lineintegral:180}
\begin{dmath}\label{eqn:lineintegral:160}
\int_C d\Bx \cdot \boldpartial f = \evalbar{f}{\Delta u}
\end{dmath}
\begin{dmath}\label{eqn:lineintegral:200}
\int_C d\Bx \wedge \boldpartial f = 0.
\end{dmath}
\end{subequations}
\Cref{eqn:lineintegral:160}, the scalar grade of \cref{eqn:lineintegral:140}, proves part of \cref{thm:lineintegral:180}.
To complete the proof, consider the specific case of \R{3} which is representitive.
Suppose, that we have an \R{3} volume parameterization \( \Bx(u, v, w) \) sharing an edge with the curve \( C = \Bx(u,0,0) \).
The curvilinear representation of the \R{3} gradient is
\begin{dmath}\label{eqn:lineintegral:220}
\spacegrad = \Bx^u \partial_u + \Bx^v \partial_v + \Bx^w \partial_w = \boldpartial + \Bx^v \partial_v + \Bx^w \partial_w,
\end{dmath}
Over the curve \( C \)
\begin{dmath}\label{eqn:lineintegral:240}
d\Bx \cdot \spacegrad
=
du\, \Bx_u \cdot \lr{ \boldpartial + \Bx^v \partial_v + \Bx^w \partial_w },
\end{dmath}
but \( \Bx_u \cdot \Bx^v = \Bx_u \cdot \Bx^w = 0 \), so \( d\Bx \cdot \spacegrad = d\Bx \cdot \boldpartial \) over the curve, completing the proof.

%}
%\EndNoBibArticle
