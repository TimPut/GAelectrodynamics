%
% Copyright � 2018 Peeter Joot.  All Rights Reserved.
% Licenced as described in the file LICENSE under the root directory of this GIT repository.
%
%{
%%%\input{../latex/blogpost.tex}
%%%\renewcommand{\basename}{lineintegral}
%%%%\renewcommand{\dirname}{notes/phy1520/}
%%%\renewcommand{\dirname}{notes/ece1228-electromagnetic-theory/}
%%%%\newcommand{\dateintitle}{}
%%%%\newcommand{\keywords}{}
%%%
%%%\input{../latex/peeter_prologue_print2.tex}
%%%
%%%\usepackage{peeters_layout_exercise}
%%%\usepackage{peeters_braket}
%%%\usepackage{peeters_figures}
%%%\usepackage{siunitx}
%%%%\usepackage{mhchem} % \ce{}
%%%%\usepackage{macros_bm} % \bcM
%%%%\usepackage{macros_qed} % \qedmarker
%%%%\usepackage{txfonts} % \ointclockwise
%%%
%%%\beginArtNoToc
%%%
%%%\generatetitle{Multivector line integral.}
%\section{Line integral.}
\label{chap:lineintegral}

\index{differential form}
A single parameter curve, and the corresponding differential with respect to that parameter, is plotted in
\cref{fig:oneParameterDifferential:oneParameterDifferentialFig1}.
%, for a parameterization over \( [a, b] \in [0,1]\otimes[0,1] \).

\imageFigure{../figures/GAelectrodynamics/oneParameterDifferentialFig1}{One parameter manifold.}{fig:oneParameterDifferential:oneParameterDifferentialFig1}{0.2}

The differential with respect to the parameter \( a \) is
\begin{equation}\label{eqn:lineintegral:20}
d\Bx_a = \PD{a}{\Bx} da = \Bx_a da.
\end{equation}

The vector derivative has just one component
\begin{dmath}\label{eqn:lineintegral:40}
\boldpartial
=
\sum_i \Bx^i (\Bx_i \cdot \spacegrad)
=
\Bx^a \PD{a}{}
\equiv
\Bx^a \partial_a.
\end{dmath}

The line integral specialization of \cref{dfn:fundamentalTheoremOfCalculus:240} can now be stated

\makedefinition{Multivector line integral.}{dfn:lineintegral:100}{
Given an connected curve \( C \) parameterized by a single parameter, and multivector functions \( F, G \), we define the line integral as
\begin{equation*}
\int_C F d^1\Bx \lrboldpartial G
\equiv
\int_C \lr{ F d^1\Bx \lboldpartial} G
+
\int_C F d^1\Bx \lr{ \rboldpartial G },
\end{equation*}
where the one parameter differential form \( d^1 \Bx = da\, \Bx_a \) varies over the curve.
} % definition

The line integral specialization of \cref{thm:fundamentalTheoremOfCalculus:1} is

%
% Copyright � 2018 Peeter Joot.  All Rights Reserved.
% Licenced as described in the file LICENSE under the root directory of this GIT repository.
%
\maketheorem{Multivector line integral.}{thm:lineintegral:100}{
Given an connected curve \( C \) parameterized by a single parameter, and multivector functions \( F, G \), the line integral is related to the boundary by
\begin{equation*}
\int_C F d^1\Bx \boldpartial G
= \evalbar{F G}{\Delta a}.
\end{equation*}
} % theorem


The proof follows by expansion.
For
the (single variable) parameterization \( a \) above
\begin{dmath}\label{eqn:lineintegral:120}
\int_C F d\Bx\, \lrboldpartial G
=
\int_C \lr{ F d\Bx \lboldpartial} G
+
\int_C F d\Bx \lr{ \rboldpartial G }
=
\int_C \PD{a}{F} da\, \Bx_a \Bx^a G
+
\int_C F da\, \Bx_a \Bx^a \PD{a}{G}
=
\int_C da\, \PD{a}{F} G
+
\int_C da\, F \PD{a}{G}
=
\int_C da\, \PD{a}{} \lr{ F G }
=
F(a_1) G(a_1) -
F(a_0) G(a_0),
\end{dmath}
where the boundaries of the parameterization have been assumed to be
%\( C\lcolon a \in [a_0, a_1] \).
\( C: a \in [a_0, a_1] \).
We have a perfect cancellation of the reciprocal frame \( \Bx^a \) with the vector \( \Bx_a \) that lies along the curve, since \( \Bx^a \Bx_a = 1 \).  This leaves a perfect derivative of the product of \( F G \), which can be integrated over the length of the curve, yielding the difference of the product with respect to the parameterization of the end points of the curve.

For a single parameter subspace
the reciprocal frame vector \( \Bx^a \)
is trivial to calculate, as it is just the inverse of \( \Bx_a \), that is \( \Bx^a = \Bx_a/\Norm{\Bx_a}^2 \).
Observe that we did not actually have to calculate it, but instead only require that the vector is invertible.

An important (and familiar) special case of \cref{thm:lineintegral:100} is the fundamental theorem of calculus for line integrals, which can be obtained by using a
single scalar function \( f \)

%
% Copyright � 2018 Peeter Joot.  All Rights Reserved.
% Licenced as described in the file LICENSE under the root directory of this GIT repository.
%
\maketheorem{Line integral of a scalar function (Stokes').}{thm:lineintegral:180}{
Given a scalar function \( f \), its \textit{line integral} is given by
\begin{equation*}
\int_C d^1\Bx \cdot \boldpartial f =
\int_C d^1\Bx \cdot \spacegrad f = \evalbar{f}{\Delta a}.
\end{equation*}
} % theorem


Writing out \cref{thm:lineintegral:100} with \( F = 1, G = f(\Bx(a)) \), we have
\begin{dmath}\label{eqn:lineintegral:140}
\int_C d\Bx\, \boldpartial f = \evalbar{f}{\Delta a}.
\end{dmath}

This is a multivector equation with scalar and bivector grades on the left hand side, but only scalar grades on the right.  Equating grades yields two equations

\begin{subequations}
\label{eqn:lineintegral:180}
\begin{dmath}\label{eqn:lineintegral:160}
\int_C d\Bx \cdot \boldpartial f = \evalbar{f}{\Delta a}
\end{dmath}
\begin{dmath}\label{eqn:lineintegral:200}
\int_C d\Bx \wedge \boldpartial f = 0
\end{dmath}
\end{subequations}

Because \( d\Bx \cdot \boldpartial = d\Bx \cdot \spacegrad \), we can replace the vector derivative with the gradient in \cref{eqn:lineintegral:160}, which yields the conventional line integral result, proving the theorem.

%}
%\EndNoBibArticle
