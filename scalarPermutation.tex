%
% Copyright � CCYY Peeter Joot.  All Rights Reserved.
% Licenced as described in the file LICENSE under the root directory of this GIT repository.
%
\makeproblem{Cyclic permutation within scalar selection.}{problem:scalarPermutation:1}{
Show that, for multivectors \( A \), \( B \) a cyclic permutation of the multivectors within a grade zero selection is possible

\begin{equation}\label{eqn:scalarPermutation:1}
\gpgradezero{A B C}
=
\gpgradezero{B C A}.
\end{equation}
} % problem

\makeanswer{problem:scalarPermutation:1}{

It is sufficient to show that

\begin{equation}\label{eqn:scalarPermutation:21}
\gpgradezero{A B}
=
\gpgradezero{B A}.
\end{equation}

If the maximum grade of \( A \) and \( B \) are \( a \) and \( b \) respectively, then

\begin{equation}\label{eqn:scalarPermutation:41}
\gpgradezero{A B}
=
\sum_{r = 0}^a \sum_{s = 0}^b \gpgradezero{A_r B_s}
=
\sum_{k = 0}^{\min(a,b)} \gpgradezero{A_k B_k}
\end{equation}

Because \( \gpgradezero{M}^\dagger = \gpgradezero{M^\dagger} = \gpgradezero{M} \), reversing all the factors in each of these grade zero selections leaves the result unchanged.  That is

\begin{equation}\label{eqn:scalarPermutation:61}
\gpgradezero{A B}
=
\sum_{k = 0}^{\min(a,b)} \gpgradezero{B_k^\dagger A_k^\dagger}.
\end{equation}

Using \cref{eqn:scalarPermutation:101}, this is

\begin{dmath}\label{eqn:scalarPermutation:121}
\gpgradezero{A B}
=
\sum_{k = 0}^{\min(a,b)} \lr{(-1)^{k(k-1)/2} }^2 \gpgradezero{B_k A_k}
=
\sum_{k = 0}^{\min(a,b)} \gpgradezero{B_k A_k}
=
\gpgradezero{ B A }. \qedmarker
\end{dmath}

} % answer
