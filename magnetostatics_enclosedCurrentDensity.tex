%
% Copyright © 2018 Peeter Joot.  All Rights Reserved.
% Licenced as described in the file LICENSE under the root directory of this GIT repository.
%
%{
\index{enclosed current density}

A volume integral of \cref{eqn:magnetostatics:380} provides a relationship between the total enclosed current density and the magnetic field.
The fundamental theorem gives

\begin{dmath}\label{eqn:magnetostatics_enclosedCurrentDensity:580}
-\mu I
\int_V d^3 \Bx \BJ =
\int_V d^3 \Bx \spacegrad \BB =
\int_{\partial V} d^2 \Bx \BB.
\end{dmath}

With a normal parameterization of the oriented surface area element \( d^2 \Bx = I \ncap dA \), and \( d^3 \Bx = I dV \),
\cref{eqn:magnetostatics_enclosedCurrentDensity:580} is reduced to

%\begin{dmath}\label{eqn:magnetostatics_enclosedCurrentDensity:600}
\boxedEquation{eqn:magnetostatics_enclosedCurrentDensity:600}{
\int_{\partial V} dA I \ncap \BB = \mu  \int_V dV \BJ.
}
%\end{dmath}

This can be split into two grades

\begin{subequations}
\label{eqn:magnetostatics_enclosedCurrentDensity:620}
\begin{dmath}\label{eqn:magnetostatics_enclosedCurrentDensity:640}
I \int_{\partial V} dA \ncap \cdot \BB = 0
\end{dmath}
\begin{dmath}\label{eqn:magnetostatics_enclosedCurrentDensity:660}
\int_{\partial V} dA \ncap \cross \BB = -\mu  \int_V dV \BJ.
\end{dmath}
\end{subequations}

\Cref{eqn:magnetostatics_enclosedCurrentDensity:640} states that the magnetic flux through a closed surface is zero, which is not be a surprise since it is a direct consequence of Gauss's law \( \spacegrad \cdot \BB = 0 \).
\Cref{eqn:magnetostatics_enclosedCurrentDensity:660} provides a relationship between the tangential components of the magnetic field and the total enclosed current density.

%}
