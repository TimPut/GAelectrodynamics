%
% Copyright � 2016 Peeter Joot.  All Rights Reserved.
% Licenced as described in the file LICENSE under the root directory of this GIT repository.
%
%{

%\footnote{Similar to Feynman on gravitation \citep{feynman1963flp} ``... have shall said everything required, for a sufficiently talented mathematician could then deduce all the consequences of these principles.  However, since you are not assumed to be sufficiently talented yet, we shall discuss the consequences in more detail''.}.

The contraction axiom is arguably the most important of the multivector space axioms, but may be unintuitive.

For one justification of this rule, consider a one dimensional vector space spanned by a single unit vector \( \setlr{ \Be } \).  That span, for real \( x \) is all the values

\begin{dmath}\label{eqn:multiplication:20}
\Bx = x \Be.
\end{dmath}

A one dimensional vector space is isomorphic with a number line \( \setlr{x}, x \in \bbR \), so can we use the rules for numeric multiplication to justify a rule for vector multiplication in a one dimensional vector space?

Consider, for example, two vectors \( -3\Be \) and \( 7 \Be \), in \R{1} and in the number line space as plotted in
\cref{fig:1Darrows:1DarrowsFig2}.
\imageTwoFigures
{../figures/GAelectrodynamics/1DarrowsFig2}
{../figures/GAelectrodynamics/1DnumberlineFig1}
{Equivalent vectors in \R{1} and on a number line.}{fig:1Darrows:1DarrowsFig2}{scale=0.5}
%\imageFigure{../figures/GAelectrodynamics/1DarrowsFig2}{Vectors in 1D space.}{fig:1Darrows:1DarrowsFig2}{0.03}
%\imageFigure{../figures/GAelectrodynamics/1DnumberlineFig1}{Points on a number line.}{fig:1Dnumberline:1DnumberlineFig1}{0.045}
%\cref{fig:1Dnumberline:1DnumberlineFig1}.

The multiplication rules for real numbers require that for any point \( x \) distant from the origin, we have

\begin{equation}\label{eqn:multiplication:60}
(\pm x)^2 = \Abs{x}^2 = x^2.
\end{equation}

Requiring this of the equivalent one dimensional vector space requires

\begin{equation}\label{eqn:multiplication:40}
(\pm \Bx)^2 = \Abs{x}^2 = x^2,
\end{equation}

which is a statement of the contraction axiom for a one dimensional Euclidean vector space.
In this sense the contraction axiom is just taking the rule for real multiplication, and applying it to vector spaces.

%%}
%%%\EndArticle
%%\EndNoBibArticle
