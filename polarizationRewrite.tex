%
% Copyright � 2018 Peeter Joot.  All Rights Reserved.
% Licenced as described in the file LICENSE under the root directory of this GIT repository.
%
%{
%\input{../latex/blogpost.tex}
%\renewcommand{\basename}{polarizationRewrite}
%%\renewcommand{\dirname}{notes/phy1520/}
%\renewcommand{\dirname}{notes/ece1228-electromagnetic-theory/}
%%\newcommand{\dateintitle}{}
%%\newcommand{\keywords}{}
%
%\input{../latex/peeter_prologue_print2.tex}
%
%\usepackage{peeters_layout_exercise}
%\usepackage{peeters_braket}
%\usepackage{peeters_figures}
%\usepackage{siunitx}
%%\usepackage{mhchem} % \ce{}
%%\usepackage{macros_bm} % \bcM
%\usepackage{macros_cal} % \calE
%%\usepackage{macros_qed} % \qedmarker
%%\usepackage{txfonts} % \ointclockwise
%
%\beginArtNoToc
%
%\generatetitle{Polarization}
%\chapter{Polarization}
\label{chap:polarizationRewrite}

%\section{Polarization.}
\subsection{Phasor representation.}
            %%
% Copyright © 2017 Peeter Joot.  All Rights Reserved.
% Licenced as described in the file LICENSE under the root directory of this GIT repository.
%
%{
\index{plane wave}
\index{polarization}
In a discussion of polarization, it is convenient to align the propagation direction along a fixed direction, usually the z-axis.
Setting \( \kcap = \Be_3, \beta z = \Bk \cdot \Bx \) in \cref{eqn:frequencydomainPlaneWaves:200} the plane wave representation of the field is

\begin{dmath}\label{eqn:polarization:20}
\begin{aligned}
F(\Bx, \omega) &= (1 + \Be_3) \BE e^{-j \beta z} \\
F(\Bx, t) &= \Real\lr{ F(\Bx, \omega) e^{j \omega t} }.
\end{aligned}
\end{dmath}

Here the imaginary \( j \) has no intrinsic geometrical interpretation, \( \BE = \BE_\txtr + j \BE_\txti \) is allowed to have complex values, and all components of \( \BE \) is perpendicular to the propagation direction (\( \Be_\txtr \cdot \Be_3 = \BE_\txti \cdot \Be_3 = 0 \)).
\index{Jones vector}
A common representation of the electric field components is the Jones vector \( (c_1, c_2) \), which specifies complex coefficients for the electric field phasor in each of the possible directions

\begin{dmath}\label{eqn:polarization:120}
\BE = c_1 \Be_1 + c_2 \Be_2,
\end{dmath}

where \( c_1, c_2 \) are complex valued, say

\begin{dmath}\label{eqn:polarization:140}
\begin{aligned}
c_1 &= \alpha_1 + j \beta_1 \\
c_2 &= \alpha_2 + j \beta_2.
\end{aligned}
\end{dmath}

The tuple \( (c_1, c_2) \) is called the Jones vector, and compactly encodes the geometry of the pattern that the electric field traces out in the transverse plane.


\index{plane wave}
\index{polarization}
In a discussion of polarization, it is convenient to align the propagation direction along a fixed direction, usually the z-axis.
%%\begin{dmath}\label{eqn:frequencydomainPlaneWaves:200}
%\boxedEquation
%{eqn:frequencydomainPlaneWaves:200}
%{
%F = (1 + \kcap) \BE_0 e^{-j \Bk \cdot \Bx},
%}
%%\end{dmath}
Setting \( \kcap = \Be_3, \beta z = \Bk \cdot \Bx \)
in a
plane wave representation from
\cref{thm:planewavesMultivector:620}
%\cref{eqn:frequencydomainPlaneWaves:200} the plane wave representation of
the field is
\begin{dmath}\label{eqn:polarization:20}
\begin{aligned}
F(\Bx, \omega) &= (1 + \Be_3) \BE e^{-j \beta z} \\
F(\Bx, t) &= \Real\lr{ F(\Bx, \omega) e^{j \omega t} },
\end{aligned}
\end{dmath}
where \( \BE \cdot \Be_3 = 0 \)
, (i.e. \( \BE \) is an electric field, and not just a free parameter).

Here the imaginary \( j \) has no intrinsic geometrical interpretation, \( \BE = \BE_\txtr + j \BE_\txti \) is allowed to have complex values, and all components of \( \BE \) is perpendicular to the propagation direction (\( \BE_\txtr \cdot \Be_3 = \BE_\txti \cdot \Be_3 = 0 \)).  Stated explicitly, this means that the electric field phasor may have real or complex components in either of the transverse plane basis directions, as in
\begin{dmath}\label{eqn:polarization:640}
\BE =
\lr{ \alpha_1 + j \beta_1 } \Be_1 +
\lr{ \alpha_2 + j \beta_2 } \Be_2.
\end{dmath}
The total time domain field for this general phasor field is easily found to be
\begin{dmath}\label{eqn:polarization_circular:160}
F(\Bx, t) = (1 + \Be_3) \lr{
\lr{ \alpha_1 \Be_1 + \alpha_2 \Be_2 } \cos\lr{ \omega t - \beta z }
-\lr{ \beta_1 \Be_1 + \beta_2 \Be_2 } \sin\lr{ \omega t - \beta z }
}.
\end{dmath}

Different combinations of \( \alpha_1, \alpha_2, \beta_1, \beta_2 \) lead to linear, circular, or elliptically polarized plane wave states to be discussed shortly.
Before doing so, we want to find natural multivector representations of \cref{eqn:polarization_circular:160}.
Such representations are possible using
either the pseudoscalar for the transverse plane \( \Be_{12} \), or the \R{3} pseudoscalar \( I \).

\subsection{Transverse plane pseudoscalar.}
\subsubsection{Statement.}

In this section the pseudoscalar of the transverse plane, written \( i = \Be_{12} \), is used as an imaginary.

\index{\(\phi\)}
\makedefinition{Phase angle.}{dfn:polarizationRewrite:660}{
Define the total phase as
\begin{equation*}
\phi(z,t) = \omega t - \beta z.
\end{equation*}
} % definition

We seek a representation of the field utilizing complex exponentials of the phase, instead of signs and cosines.  It will be helpful to define the coordinates of the Jones vector to state that representation.

\index{Jones vector}
\index{\(c_1, c_2\)}
\index{\(\alpha_\txtR, \alpha_\txtL\)}
\makedefinition{Jones vectors.}{dfn:polarizationRewrite:680}{
The coordinates of the Jones vector, conventionally defined as a tuple of complex values \((c_1, c_2)\), are
\begin{equation*}
\begin{aligned}
c_1 &= \alpha_1 + i \beta_1 \\
c_2 &= \alpha_2 + i \beta_2.
\end{aligned}
\end{equation*}
In this definition we have used \( i = \Be_{12} \), the pseudoscalar of the transverse plane, as the imaginary.
} % definition
We will not use the Jones vector as a tuple, but will use \( c_1, c_2 \) as stated above.
%
% Copyright � 2018 Peeter Joot.  All Rights Reserved.
% Licenced as described in the file LICENSE under the root directory of this GIT repository.
%
\maketheorem{Circular polarization coefficients.}{thm:polarizationRewrite:700}{
The time domain representation of the field in \cref{eqn:polarization_circular:160} can be stated in terms of the total phase as
\begin{equation*}
F = \lr{ 1 + \Be_3 } \Be_1 \lr{ \alpha_\txtR e^{i\phi} + \alpha_\txtL e^{-i\phi} },
\end{equation*}
where
\begin{equation*}
\begin{aligned}
\alpha_\txtR &= \inv{2}\lr{ c_1 + i c_2 } \\
\alpha_\txtL &= \inv{2}\lr{ c_1 - i c_2 }^\dagger,
\end{aligned}
\end{equation*}
where \( c_1, c_2 \) are the 0,2 grade multivector representation of the Jones vector coordinates from \cref{dfn:polarizationRewrite:680}.
} % theorem

To prove \cref{thm:polarizationRewrite:700}, we have only to factor \( \Be_1 \) out of \cref{eqn:polarization_circular:160} and then substitute complex exponentials for the sine and cosine
\begin{equation}\label{eqn:polarizationRewrite:700}
\begin{aligned}
\biglr{ \alpha_1 \Be_1 &+ \alpha_2 \Be_2 } \cos\lr{ \phi }
-\biglr{ \beta_1 \Be_1 + \beta_2 \Be_2 } \sin\lr{ \phi } \\
&=
\Be_1 \lr{
 \lr{ \alpha_1  + \alpha_2 i } \cos\lr{ \phi }
-\lr{ \beta_1  + \beta_2 i } \sin\lr{ \phi }
} \\
&=
\frac{\Be_1}{2} \Biglr{
 \lr{ \alpha_1  + \alpha_2 i } \lr{ e^{i\phi} + e^{-i\phi} } \\
&\qquad +\lr{ \beta_1  + \beta_2 i } i \lr{ e^{i\phi} - e^{-i\phi} }
} \\
&=
\frac{\Be_1}{2} \Biglr{
 \lr{ \alpha_1 + i \beta_1  + i (\alpha_2 + i \beta_2) } e^{i\phi} \\
&\qquad +
 \lr{ (\alpha_1 + i \beta_1 )^\dagger + i( \alpha_2 + i \beta_2 )^\dagger } e^{-i\phi}
} \\
&=
\frac{\Be_1}{2} \lr{
 \lr{ c_1 + i c_2 } e^{i\phi}
+
 \lr{ c_1 - i c_2 }^\dagger e^{-i\phi}
},
\end{aligned}
\end{equation}
which completes the proof.

\subsubsection{Linear polarization.}

Linear polarization is described by
\begin{dmath}\label{eqn:polarization_linearPolarization:280}
\begin{aligned}
\alpha_\txtR &= \inv{2}\Norm{\BE} e^{i(\psi + \theta)} \\
\alpha_\txtL &= \inv{2}\Norm{\BE} e^{i(\psi - \theta)},
\end{aligned}
\end{dmath}
so the field is
\begin{dmath}\label{eqn:polarization_linearPolarization:300}
F = \lr{ 1 + \Be_3 } \Norm{\BE} \Be_1 e^{i\psi} \cos( \omega t - \beta z + \theta ).
\end{dmath}
Here \( \theta \) is an arbitrary initial phase.
The electric field \( \BE \) traces out all the points along the line spanning the points between \( \pm \Be_1 e^{i\psi} \Norm{\BE} \), whereas the magnetic field \( \BH \) traces
out all the points along \( \pm \Be_2 e^{i\psi} \Norm{\BE}/\eta \) as illustrated (with \( \eta = 1 \)) in
\cref{fig:linearPolarization:linearPolarizationFig1}.
\mathImageFigure{../figures/GAelectrodynamics/linearPolarizationFig1}{Linear polarization.}{fig:linearPolarization:linearPolarizationFig1}{0.3}{linearPolarizationFig1.nb}

\subsubsection{Circular polarization.}
\index{circular polarization}
\index{left circular polarization}
\index{right circular polarization}

A field for which the change in phase
results in the electric field tracing out a (clockwise,counterclockwise) circle
\begin{dmath}\label{eqn:polarization_circular:180}
\begin{aligned}
\BE_\txtR &= \Norm{\BE} \lr{ \Be_1 \cos\phi + \Be_2 \sin\phi } = \Norm{\BE} \Be_1 \exp\lr{  \Be_{12} \phi } \\
\BE_\txtL &= \Norm{\BE} \lr{ \Be_1 \cos\phi - \Be_2 \sin\phi } = \Norm{\BE} \Be_1 \exp\lr{ -\Be_{12} \phi },
\end{aligned}
\end{dmath}
is referred to as having
(right,left) circular polarization, so the choice
\(\alpha_\txtR = \Norm{\BE}, \alpha_\txtL = 0 \) results
in a right polarized wave
\begin{dmath}\label{eqn:polarizationRewrite:720}
F = (1 + \Be_3) \Norm{\BE} \Be_1 e^{i(\omega t - k z)},
\end{dmath}
and
\(\alpha_\txtL = \Norm{\BE}, \alpha_\txtR = 0 \) results
in a left polarized wave
\begin{dmath}\label{eqn:polarizationRewrite:740}
F = (1 + \Be_3) \Norm{\BE} \Be_1 e^{-i(\omega t - k z)},
\end{dmath}
There are different conventions for the polarization orientation, and here the IEEE antenna convention discussed in \citep{balanis1989advanced} are used.

\subsubsection{Elliptical parameterization.}
An elliptical polarized electric field can be parameterized as
\begin{dmath}\label{eqn:polarization_elliptical:340}
\BE
=
E_a \Be_1 \cos\phi + E_b \Be_2 \sin\phi,
\end{dmath}
which corresponds to
%a Jones vector \( (E_a, -i E_b) \), or
circular polarization coefficients with values
\begin{dmath}\label{eqn:polarization_elliptical:400}
\begin{aligned}
\alpha_\txtR &= \inv{2}\lr{ E_a - E_b } \\
\alpha_\txtL &= \inv{2}\lr{ E_a + E_b }.
\end{aligned}
\end{dmath}

Therefore an elliptically polarized field can be represented as
\begin{dmath}\label{eqn:polarization_elliptical:420}
F = \inv{2} (1 + \Be_3) \Be_1 \lr{ (E_a + E_b) e^{i\phi} + (E_a - E_b) e^{-i\phi} }.
\end{dmath}

An interesting variation of the elliptical polarization uses a hyperbolic parameterization.
If \( a, b \) are the semi-major/minor axes of the ellipse (i.e. \( a > b \)),
and \( \Ba = a \Be_1 e^{i\psi} \) is the vectoral representation of the semi-major axis (not necessarily placed along \( \Be_1 \)),
and \( e = \sqrt{1 - (b/a)^2} \) is the eccentricity of the ellipse,
then it can be shown (\citep{hestenes1999nfc})
that an elliptic parameterization can be written
in the compact form
\begin{dmath}\label{eqn:polarization_elliptical:360}
\Br(\phi)
=
e \Ba \cosh( \tanh^{-1}(b/a) + i \phi).
\end{dmath}

When the bivector imaginary \( i = \Be_{12} \) is used then
this parameterization is real and has only vector grades, so the electromagnetic field for a general elliptic wave has the form
\begin{dmath}\label{eqn:polarization_elliptical:380}
\begin{aligned}
F &= e E_a \lr{ 1 + \Be_3 } \Be_1 e^{ i \psi } \cosh\lr{ m + i \phi} \\
m &= \tanh^{-1}\lr{ E_b/E_a } \\
e &= \sqrt{1 - {(E_b/E_a)}^2 },
\end{aligned}
\end{dmath}
where \( E_a(E_b) \) are the magnitudes of the electric field components lying along the semi-major(minor) axes, and the propagation direction \( \Be_3 \) is orthogonal to both the major and minor axis directions.
An elliptic electric field polarization is illustrated in \cref{fig:ellipticalPolarization:ellipticalPolarizationFig1}, where the vectors representing the major and minor axes are \( \BE_a = E_a \Be_1 e^{i\psi}, \BE_b = E_b \Be_1 e^{i\psi} \).
Observe that setting \( E_b = 0 \) results in the linearly polarized field of \cref{eqn:polarization_linearPolarization:300}.
\mathImageFigure{../figures/GAelectrodynamics/ellipticalPolarizationFig1}{Electric field with elliptical polarization.}{fig:ellipticalPolarization:ellipticalPolarizationFig1}{0.3}{ellipticalPolarizationFig1.nb}

\subsubsection{Energy and momentum.}

Each polarization considered above (linear, circular, elliptical) have the same general form
\begin{dmath}\label{eqn:polarizationRewrite:760}
F = \lr{ 1 + \Be_3 } \Be_1 e^{i\psi} f(\phi),
\end{dmath}
where \( f(\phi) \) is a complex valued function (i.e. grade 0,2).  The structure of \cref{eqn:polarizationRewrite:760} could be more general than considered so far.  For example, a Gaussian modulation could be added into the mix with \( f(\phi) = e^{i \phi - (\phi/\sigma)^2/2 } \).  Independent of the form of \( f \), we may compute the energy, momentum and Maxwell stress tensor for the plane wave given by \cref{eqn:polarizationRewrite:760}.
%
% Copyright � 2018 Peeter Joot.  All Rights Reserved.
% Licenced as described in the file LICENSE under the root directory of this GIT repository.
%
\maketheorem{Plane wave energy momentum tensor components.}{thm:polarizationRewrite:780}{
The \textit{energy momentum tensor components} for the plane wave given by \cref{eqn:polarizationRewrite:760} are
\begin{equation*}
\begin{aligned}
T(1) &= -T(\Be_3) = \epsilon \lr{ 1 + \Be_3 } f f^\dagger \quad \lr{ = \calE + \frac{\BS}{c} } \\
T(\Be_1) &= T(\Be_2) = 0.
\end{aligned}
\end{equation*}
} % theorem

Only the propagation direction of a plane wave, regardless of its polarization (or even whether or not there are Gaussian or other damping factors), carries any energy or momentum, and only the propagation direction component of the Maxwell stress tensor \( \BT(\Ba) \) is non-zero.

To prove \cref{thm:polarizationRewrite:780}, we may compute \( T(a) \) separately for each of \( a = 1,\Be_1, \Be_2, \Be_3 \).  Key to all of these computations is the fact that \( \Be_3 \) commutes with scalars and \( i \), and \( \Be_1, \Be_2 \) both anticommute with \( i \), and more generally \(
\begin{bmatrix}
\Be_1 \\
\Be_2
\end{bmatrix}
(a + i b) = (a - i b)
\begin{bmatrix}
\Be_1 \\
\Be_2
\end{bmatrix}
\).  For \( T(1) \) we need the product of the field and its reverse
\begin{dmath}\label{eqn:polarizationRewrite:780}
F F^\dagger
=
\lr{ 1 + \Be_3 } \cancel{\Be_1 e^{i\psi}} \
\mathLabelBox[ labelstyle={below of=m\themathLableNode, below of=m\themathLableNode} ]
{ f f^\dagger
}
{
scalar
}
\cancel{ e^{-i\psi} \Be_1} \lr{ 1 + \Be_3 }
=
\lr{ 1 + \Be_3 }^2 f f^\dagger
= 2 \lr{ 1 + \Be_3 } f f^\dagger,
\end{dmath}
so \( T(1) = \epsilon \lr{ 1 + \Be_3 } f f^\dagger \).  For \( T(\Be_3) \) we have
\begin{dmath}\label{eqn:polarizationRewrite:800}
F \Be_3 F^\dagger
=
\lr{ 1 + \Be_3 } \Be_1 e^{i\psi} f \Be_3 f^\dagger e^{-i\psi} \Be_1 \lr{ 1 + \Be_3 }
=
-\lr{ 1 + \Be_3 } \Be_3 \Be_1 e^{i\psi} f f^\dagger e^{-i\psi} \Be_1 \lr{ 1 + \Be_3 }
=
-\lr{ 1 + \Be_3 } \Be_1 e^{i\psi} f f^\dagger e^{-i\psi} \Be_1 \lr{ 1 + \Be_3 }
=
-2 \lr{ 1 + \Be_3 } f f^\dagger,
\end{dmath}
so \( T(\Be_3) = -T(1) \).
For \( T(\Be_1) \), we have
\begin{dmath}\label{eqn:polarizationRewrite:820}
F \Be_1 F^\dagger
=
\lr{ 1 + \Be_3 } \Be_1 e^{i\psi} f \Be_1 f^\dagger e^{-i\psi} \Be_1 \lr{ 1 + \Be_3 }
=
\lr{ 1 + \Be_3 } \Be_1 e^{i\psi} f^2 e^{i\psi} \Be_1^2 \lr{ 1 + \Be_3 }
=
\lr{ 1 + \Be_3 } \Be_1 f^2 e^{2 i\psi} \lr{ 1 + \Be_3 }
=
\lr{ 1 + \Be_3 } \Be_1 \lr{ 1 + \Be_3 } f^2 e^{2 i\psi}
=
\lr{ 1 + \Be_3 } \lr{ 1 - \Be_3 } \Be_1 f^2 e^{2 i\psi}
=
\lr{ 1 - \Be_3^2 } \Be_1 f^2 e^{2 i\psi}
=
0.
\end{dmath}
Clearly \( F \Be_2 F^\dagger = 0 \) as well, so \( T(\Be_1) = T(\Be_2) = 0 \), completing the proof.

Using \cref{thm:polarizationRewrite:780} the energy momentum vector for the linearly polarized wave of \cref{eqn:polarization_linearPolarization:300} is
\begin{dmath}\label{eqn:polarizationRewrite:840}
T(1) = \frac{\epsilon}{2} \lr{ 1 + \Be_3 } \Norm{\BE}^2 \cos^2( \phi + \theta ),
\end{dmath}
and for the circularly polarized wave of \cref{eqn:polarizationRewrite:720}, or
\cref{eqn:polarizationRewrite:740} is
\begin{dmath}\label{eqn:polarizationRewrite:860}
T(1) = \frac{\epsilon}{2} (1 + \Be_3) \Norm{\BE}^2.
\end{dmath}
A circularly polarized wave carries maximum energy and momentum, whereas the energy and momentum of a linearly polarized wave
oscillates with the phase angle.

For the elliptically polarized wave of \cref{eqn:polarization_elliptical:380} we have
\begin{dmath}\label{eqn:polarizationRewrite:1040}
f(\phi) = e E_a \cosh\lr{ m + i \phi}.
\end{dmath}
The absolute value of \( f \) is
\begin{dmath}\label{eqn:polarizationRewrite:880}
f f^\dagger
= e^2 E_a^2 \cosh\lr{ m + i \phi} \lr{\cosh\lr{ m + i \phi}}^\dagger
= e^2 E_a^2 \lr{ \cosh(2m) + \cos(2 \phi) }
= e^2 E_a^2 \lr{ \frac{E_b^2}{E_a^2} + 2 \lr{ 1 - \frac{E_b^2}{E_a^2} } \cos^2 \phi }
\end{dmath}
The simplification above made use of the identity
\begin{dmath}\label{eqn:polarizationRewrite:1060}
(1 - (b/a)^2) \cosh(2 \Atanh(b/a)) = 1 + (b/a)^2.
\end{dmath}
The energy momentum for an elliptically polarized wave is therefore
\begin{dmath}\label{eqn:polarizationRewrite:900}
T(1)
= \frac{\epsilon}{2} \lr{ 1 + \Be_3 } e^2 E_a^2 \lr{ \frac{E_b^2}{E_a^2} + 2 \lr{ 1 - \frac{E_b^2}{E_a^2} } \cos^2 \phi }.
\end{dmath}
As expected, the phase dependent portion of the energy momentum tensor vanishes as the wave function approaches circular polarization.

\subsection{Pseudoscalar imaginary.}

In this section we use the \R{3} pseudoscalar as an imaginary.
As before, we seek a representation of the field utilizing complex exponentials of the phase, instead of signs and cosines, and as before the we wish to define Jones vector coordinates as a go-between.

\index{Jones vector}
\index{\(c_1, c_2\)}
\index{\(\alpha_\txtR, \alpha_\txtL\)}
\makedefinition{Jones vectors.}{dfn:polarizationRewrite:920}{
The coordinates of the Jones vector, conventionally defined as a tuple of complex values \((c_1, c_2)\), are
\begin{equation*}
\begin{aligned}
c_1 &= \alpha_1 + I \beta_1 \\
c_2 &= \alpha_2 + I \beta_2.
\end{aligned}
\end{equation*}
In this definition we have used the \R{3} pseudoscalar \( I \) as the imaginary.
} % definition
We will not use the Jones vector as a tuple, but will use \( c_1, c_2 \) as stated above.
%
% Copyright � 2018 Peeter Joot.  All Rights Reserved.
% Licenced as described in the file LICENSE under the root directory of this GIT repository.
%
\maketheorem{Circular polarization coefficients.}{thm:polarizationRewrite:940}{
The time domain representation of the field in \cref{eqn:polarization_circular:160} can be stated in terms of the total phase as
\begin{equation*}
F = \lr{ 1 + \Be_3 } \Be_1 \lr{ \alpha_\txtR e^{-I\phi} + \alpha_\txtL e^{I\phi} },
\end{equation*}
where
\begin{equation*}
\begin{aligned}
\alpha_\txtR &= \inv{2}\lr{ c_1 + I c_2 }^\dagger \\
\alpha_\txtL &= \inv{2}\lr{ c_1 - I c_2 },
\end{aligned}
\end{equation*}
where \( c_1, c_2 \) are the 0,2 grade multivector representation of the Jones vector coordinates
\begin{equation*}
\begin{aligned}
c_1 &= \alpha_1 + I \beta_1 \\
c_2 &= \alpha_2 + I \beta_2,
\end{aligned}
\end{equation*}
defined here as 0,3 complex numbers, using \( I \) as the imaginary.
} % theorem

Notice that the signs of the exponentials have flipped for the left and right handed circular polarizations.  It may not obvious that the electric and magnetic fields in this representation have the desired transverse properties.  To see why that is still the case, and to understand the conjugation in the complex exponentials, consider the right circular polarization case with \( \alpha_\txtR = \Norm{\BE}, \alpha_\txtL = 0 \)
\begin{dmath}\label{eqn:polarizationRewrite:940}
F
= \lr{ 1 + \Be_3 } \Be_1 \Norm{\BE} e^{-I\phi}
= \lr{ 1 + \Be_3 } \Norm{\BE} \lr{ \Be_1 \cos\phi - \Be_{23} \sin\phi }
= \lr{ 1 + \Be_3 } \Norm{\BE} \lr{ \Be_1 \cos\phi + \Be_{32} \sin\phi },
\end{dmath}
but since \( \lr{ 1 + \Be_3 } \Be_3 = 1 + \Be_3 \), we have
\begin{dmath}\label{eqn:polarizationRewrite:960}
F
= \lr{ 1 + \Be_3 } \Norm{\BE} \lr{ \Be_1 \cos\phi + \Be_{2} \sin\phi },
\end{dmath}
which has the claimed right circular polarization.

To prove \cref{thm:polarizationRewrite:940} itself, the sine and cosine in \cref{eqn:polarization_circular:160} can be expanded in complex exponentials
\begin{dmath}\label{eqn:polarizationRewrite:980}
\begin{aligned}
2 &\lr{ \alpha_1 \Be_1 + \alpha_2 \Be_2 } \cos \phi
-
2 \lr{ \beta_1 \Be_1 + \beta_2 \Be_2 } \sin \phi \\
&\quad=
\lr{ \alpha_1 \Be_1 + \alpha_2 \Be_2 }
\lr{ e^{I \phi} + e^{-I\phi} }
+
\lr{ \beta_1 \Be_1 + \beta_2 \Be_2 } I
\lr{ e^{I \phi} - e^{-I\phi} } \\
&\quad =
\lr{ \alpha_1 \Be_1 - I \alpha_2 (I \Be_2) }
\lr{ e^{I \phi} + e^{-I\phi} }
+
\lr{ \beta_1 \Be_1 - I \beta_2 (I \Be_2) } I
\lr{ e^{I \phi} - e^{-I\phi} }.
\end{aligned}
\end{dmath}
Since the leading \( 1 + \Be_3 \) gobbles any \( \Be_3 \) factors, its action on the dual of \( \Be_2 \) is
\begin{dmath}\label{eqn:polarizationRewrite:1000}
\lr{ 1 + \Be_3 } I \Be_2
=
\lr{ 1 + \Be_3 } \Be_{31}
=
\lr{ 1 + \Be_3 } \Be_{1}.
\end{dmath}
This allows us to unconditionally factor out \( \Be_1 \) from
\cref{eqn:polarizationRewrite:980}, so the field is
\begin{dmath}\label{eqn:polarizationRewrite:1020}
F
= \inv{2}\lr{ 1 + \Be_3 } \Be_1
\lr{
   \lr{ \alpha_1  - I \alpha_2  } \lr{ e^{I \phi} + e^{-I\phi} }
   +
   \lr{ \beta_1  - I \beta_2  } I \lr{ e^{I \phi} - e^{-I\phi} }
}
= \inv{2}\lr{ 1 + \Be_3 } \Be_1
\lr{
   \lr{
      \alpha_1 + I \beta_1
    - I \lr{ \alpha_2 + I \beta_2 }
   } e^{I \phi}
   +
   \lr{
      \alpha_1 - I \beta_1
      -I \lr{ \alpha_2 - I \beta_2 }
   } e^{-I\phi}
}
= \inv{2}
\lr{ 1 + \Be_3 } \Be_1
\lr{
   \lr{ c_1 - I c_2 } e^{I \phi}
   +
   \lr{
      c_1^\dagger -I c_2^\dagger
   } e^{-I\phi}
}
=
\lr{ 1 + \Be_3 } \Be_1
\lr{
   \alpha_\txtR e^{-I\phi}
+
   \alpha_\txtL e^{-I\phi}
},
\end{dmath}
which completes the proof.

Observe that there are some advantages to the pseudoscalar plane wave form, especially for computing energy momentum tensor components since \( I \) commutes with all grades.  For example, we can see practically by inspection that
\begin{equation}\label{eqn:polarization_pseudoscalarImaginary:620}
T(1) = \calE + \frac{\BS}{v} =
\epsilon \lr{ 1 + \Be_3 } \lr{ \Abs{\alpha_\txtR}^2 + \Abs{\alpha_\txtL}^2 },
\end{equation}
where the absolute value is computed using the reverse as the conjugation operation \( \Abs{z}^2 = z z^\dagger \).

%}
%\EndArticle
