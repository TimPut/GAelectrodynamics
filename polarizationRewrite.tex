%
% Copyright � 2018 Peeter Joot.  All Rights Reserved.
% Licenced as described in the file LICENSE under the root directory of this GIT repository.
%
%{
\input{../latex/blogpost.tex}
\renewcommand{\basename}{polarizationRewrite}
%\renewcommand{\dirname}{notes/phy1520/}
\renewcommand{\dirname}{notes/ece1228-electromagnetic-theory/}
%\newcommand{\dateintitle}{}
%\newcommand{\keywords}{}

\input{../latex/peeter_prologue_print2.tex}

\usepackage{peeters_layout_exercise}
\usepackage{peeters_braket}
\usepackage{peeters_figures}
\usepackage{siunitx}
%\usepackage{mhchem} % \ce{}
%\usepackage{macros_bm} % \bcM
\usepackage{macros_cal} % \calE
%\usepackage{macros_qed} % \qedmarker
%\usepackage{txfonts} % \ointclockwise

\beginArtNoToc

\generatetitle{Polarization}
%\chapter{Polarization}
%\label{chap:polarizationRewrite}

\section{Polarization.}
\subsection{Phasor representation.}
            %%
% Copyright © 2017 Peeter Joot.  All Rights Reserved.
% Licenced as described in the file LICENSE under the root directory of this GIT repository.
%
%{
\index{plane wave}
\index{polarization}
In a discussion of polarization, it is convenient to align the propagation direction along a fixed direction, usually the z-axis.
Setting \( \kcap = \Be_3, \beta z = \Bk \cdot \Bx \) in \cref{eqn:frequencydomainPlaneWaves:200} the plane wave representation of the field is

\begin{dmath}\label{eqn:polarization:20}
\begin{aligned}
F(\Bx, \omega) &= (1 + \Be_3) \BE e^{-j \beta z} \\
F(\Bx, t) &= \Real\lr{ F(\Bx, \omega) e^{j \omega t} }.
\end{aligned}
\end{dmath}

Here the imaginary \( j \) has no intrinsic geometrical interpretation, \( \BE = \BE_\txtr + j \BE_\txti \) is allowed to have complex values, and all components of \( \BE \) is perpendicular to the propagation direction (\( \Be_\txtr \cdot \Be_3 = \BE_\txti \cdot \Be_3 = 0 \)).
\index{Jones vector}
A common representation of the electric field components is the Jones vector \( (c_1, c_2) \), which specifies complex coefficients for the electric field phasor in each of the possible directions

\begin{dmath}\label{eqn:polarization:120}
\BE = c_1 \Be_1 + c_2 \Be_2,
\end{dmath}

where \( c_1, c_2 \) are complex valued, say

\begin{dmath}\label{eqn:polarization:140}
\begin{aligned}
c_1 &= \alpha_1 + j \beta_1 \\
c_2 &= \alpha_2 + j \beta_2.
\end{aligned}
\end{dmath}

The tuple \( (c_1, c_2) \) is called the Jones vector, and compactly encodes the geometry of the pattern that the electric field traces out in the transverse plane.


\index{plane wave}
\index{polarization}
In a discussion of polarization, it is convenient to align the propagation direction along a fixed direction, usually the z-axis.
%%\begin{dmath}\label{eqn:frequencydomainPlaneWaves:200}
%\boxedEquation
%{eqn:frequencydomainPlaneWaves:200}
%{
%F = (1 + \kcap) \BE_0 e^{-j \Bk \cdot \Bx},
%}
%%\end{dmath}
Setting \( \kcap = \Be_3, \beta z = \Bk \cdot \Bx \)
in a
plane wave representation from
\cref{thm:planewavesMultivector:620}
%\cref{eqn:frequencydomainPlaneWaves:200} the plane wave representation of
the field is
\begin{dmath}\label{eqn:polarization:20}
\begin{aligned}
F(\Bx, \omega) &= (1 + \Be_3) \BE e^{-j \beta z} \\
F(\Bx, t) &= \Real\lr{ F(\Bx, \omega) e^{j \omega t} },
\end{aligned}
\end{dmath}
where \( \BE \cdot \Be_3 = 0 \)
, (i.e. \( \BE \) is an electric field, and not just a free parameter).

Here the imaginary \( j \) has no intrinsic geometrical interpretation, \( \BE = \BE_\txtr + j \BE_\txti \) is allowed to have complex values, and all components of \( \BE \) is perpendicular to the propagation direction (\( \BE_\txtr \cdot \Be_3 = \BE_\txti \cdot \Be_3 = 0 \)).  Stated explicitly, this means that the electric field phasor may have real or complex components in either of the transverse plane basis directions, as in
\begin{dmath}\label{eqn:polarization:640}
\BE =
\lr{ \alpha_1 + j \beta_1 } \Be_1 +
\lr{ \alpha_2 + j \beta_2 } \Be_2.
\end{dmath}
The total time domain field for this general phasor field is easily found to be
\begin{dmath}\label{eqn:polarization_circular:160}
F(\Bx, t) = (1 + \Be_3) \lr{
\lr{ \alpha_1 \Be_1 + \alpha_2 \Be_2 } \cos\lr{ \omega t - \beta z }
-\lr{ \beta_1 \Be_1 + \beta_2 \Be_2 } \sin\lr{ \omega t - \beta z }
}.
\end{dmath}

Different combinations of \( \alpha_1, \alpha_2, \beta_1, \beta_2 \) lead to linear, circular, or elliptically polarized plane wave states to be discussed shortly.
Before doing so, we want to find natural multivector representations of \cref{eqn:polarization_circular:160}.
Such representations are possible using
either the pseudoscalar for the transverse plane \( \Be_{12} \), or the \R{3} pseudoscalar \( I \).

\section{Transverse plane pseudoscalar.}
\subsection{Statement.}

In this section the pseudoscalar of the transverse plane, written \( i = \Be_{12} \), is used as an imaginary.

\makedefinition{Phase angle.}{dfn:polarizationRewrite:660}{
Define the total phase as
\begin{equation*}
\phi(z,t) = \omega t - \beta z.
\end{equation*}
} % definition

We seek a representation of the field utilizing complex exponentials of the phase, instead of signs and cosines.  It will be helpful to define the coordinates of the Jones vector to state that reprepresentation.

\index{Jones vector}
\makedefinition{Jones vectors.}{dfn:polarizationRewrite:680}{
The coordinates of the Jones vector, conventionally defined as a tuple of complex values \((c_1, c_2)\), are
\begin{equation*}
\begin{aligned}
c_1 &= \alpha_1 + i \beta_1 \\
c_2 &= \alpha_2 + i \beta_2.
\end{aligned}
\end{equation*}
In this definition we have used \( i = \Be_{12} \), the pseudoscalar of the transverse plane, as the imaginary.
} % definition
We will not use the Jones vector as a tuple, but will use \( c_1, c_2 \) as stated above.

\maketheorem{Circular polarization coefficients.}{thm:polarizationRewrite:700}{
The time domain representation of the field in \cref{eqn:polarization_circular:160} can be stated in terms of the total phase as
\begin{equation*}
F = \lr{ 1 + \Be_3 } \Be_1 \lr{ \alpha_\txtL e^{i\phi} + \alpha_\txtR e^{-i\phi} },
\end{equation*}
where
\begin{equation*}
\begin{aligned}
\alpha_\txtL &= \inv{2}\lr{ c_1 + i c_2 } \\
\alpha_\txtR &= \inv{2}\lr{ c_1 - i c_2 }^\dagger,
\end{aligned}
\end{equation*}
where \( c_1, c_2 \) are the 0,2 grade multivector representation of the Jones vector coordinates from \cref{dfn:polarizationRewrite:680}.
} % definition

To prove \cref{thm:polarizationRewrite:700}, we have only to factor \( \Be_1 \) out of \cref{eqn:polarization_circular:160} and then substitute complex exponentials for the sine and cosine
\begin{dmath}\label{eqn:polarizationRewrite:700}
\begin{aligned}
\biglr{ \alpha_1 \Be_1 &+ \alpha_2 \Be_2 } \cos\lr{ \phi }
-\biglr{ \beta_1 \Be_1 + \beta_2 \Be_2 } \sin\lr{ \phi } \\
&=
\Be_1 \lr{
 \lr{ \alpha_1  + \alpha_2 i } \cos\lr{ \phi }
-\lr{ \beta_1  + \beta_2 i } \sin\lr{ \phi }
} \\
&=
\frac{\Be_1}{2} \lr{
 \lr{ \alpha_1  + \alpha_2 i } \lr{ e^{i\phi} + e^{-i\phi} }
+\lr{ \beta_1  + \beta_2 i } i \lr{ e^{i\phi} - e^{-i\phi} }
} \\
&=
\frac{\Be_1}{2} \lr{
 \lr{ \alpha_1 + i \beta_1  + i (\alpha_2 + i \beta_2) } e^{i\phi}
+
 \lr{ (\alpha_1 + i \beta_1 )^\dagger + i( \alpha_2 + i \beta_2 )^\dagger } e^{-i\phi}
} \\
&=
\frac{\Be_1}{2} \lr{
 \lr{ c_1 + i c_2 } e^{i\phi}
+
 \lr{ c_1 - i c_2 }^\dagger e^{-i\phi}
},
\end{aligned}
\end{dmath}
which completes the proof.

\subsection{Linear polarization.}

Linear polarization is described by
\begin{dmath}\label{eqn:polarization_linearPolarization:280}
\begin{aligned}
\alpha_\txtL &= \inv{2}\Norm{\BE} e^{i(\psi + \theta)} \\
\alpha_\txtR &= \inv{2}\Norm{\BE} e^{i(\psi - \theta)},
\end{aligned}
\end{dmath}
so the field is
\begin{dmath}\label{eqn:polarization_linearPolarization:300}
F = \lr{ 1 + \Be_3 } \Norm{\BE} \Be_1 e^{i\psi} \cos( \omega t - \beta z + \theta ).
\end{dmath}
Here \( \theta \) is an arbitrary initial phase.
The electric field \( \BE \) traces out all the points along the line spanning the points between \( \pm \Be_1 e^{i\psi} \Norm{\BE} \), whereas the magnetic field \( \BH \) traces
out all the points along \( \pm \Be_2 e^{i\psi} \Norm{\BE}/\eta \) as illustrated (with \( \eta = 1 \)) in
\cref{fig:linearPolarization:linearPolarizationFig1}.
\imageFigure{../figures/GAelectrodynamics/linearPolarizationFig1}{Linear polarization.}{fig:linearPolarization:linearPolarizationFig1}{0.3}

\subsection{Circular polarization.}
\index{circular polarization}
\index{left circular polarization}
\index{right circular polarization}

A field for which the change in phase
results in the electric field tracing out a (clockwise,counterclockwise) circle
\begin{dmath}\label{eqn:polarization_circular:180}
\begin{aligned}
\BE_\txtL &= \Norm{\BE} \lr{ \Be_1 \cos\phi + \Be_2 \sin\phi } = \Norm{\BE} \Be_1 \exp\lr{  \Be_{12} \phi } \\
\BE_\txtR &= \Norm{\BE} \lr{ \Be_1 \cos\phi - \Be_2 \sin\phi } = \Norm{\BE} \Be_1 \exp\lr{ -\Be_{12} \phi },
\end{aligned}
\end{dmath}
is referred to as having
(right,left) circular polarization, so the choice
\(\alpha_\txtL = \Norm{\BE}, \alpha_\txtR = 0 \) results
in a right polarized wave
\begin{dmath}\label{eqn:polarizationRewrite:720}
F = (1 + \Be_3) \Norm{\BE} \Be_1 e^{i(\omega t - k z)},
\end{dmath}
and
\(\alpha_\txtR = \Norm{\BE}, \alpha_\txtL = 0 \) results
in a left polarized wave
\begin{dmath}\label{eqn:polarizationRewrite:740}
F = (1 + \Be_3) \Norm{\BE} \Be_1 e^{-i(\omega t - k z)},
\end{dmath}
There are different conventions for the polarization orientation, and here the IEEE antenna convention discussed in \citep{balanis1989advanced} are used.

\subsection{Elliptical parameterization.}
An elliptical polarized electric field can be parameterized as
\begin{dmath}\label{eqn:polarization_elliptical:340}
\BE
=
E_a \Be_1 \cos\phi + E_b \Be_2 \sin\phi,
\end{dmath}
which corresponds to 
%a Jones vector \( (E_a, -i E_b) \), or 
circular polarization coefficients with values
\begin{dmath}\label{eqn:polarization_elliptical:400}
\begin{aligned}
\alpha_\txtL &= \inv{2}\lr{ E_a - E_b } \\
\alpha_\txtR &= \inv{2}\lr{ E_a + E_b }.
\end{aligned}
\end{dmath}

Therefore an elliptically polarized field can be represented as
\begin{dmath}\label{eqn:polarization_elliptical:420}
F = \inv{2} (1 + \Be_3) \Be_1 \lr{ (E_a + E_b) e^{i\phi} + (E_a - E_b) e^{-i\phi} }.
\end{dmath}

An interesting variation of the elliptical polarization uses a hyperbolic parameterization.
If \( a, b \) are the semi-major/minor axes of the ellipse (i.e. \( a > b \)),
and \( \Ba = a \Be_1 e^{i\psi} \) is the vectoral representation of the semi-major axis (not necessarily placed along \( \Be_1 \)),
and \( e = \sqrt{1 - (b/a)^2} \) is the eccentricity of the ellipse,
then it can be shown (\citep{hestenes1999nfc})
that an elliptic parameterization can be written
in the compact form
\begin{dmath}\label{eqn:polarization_elliptical:360}
\Br(\phi)
=
e \Ba \cosh( \tanh^{-1}(b/a) + i \phi).
\end{dmath}

When the bivector imaginary \( i = \Be_{12} \) is used then
this parameterization is real and has only vector grades, so the electromagnetic field for a general elliptic wave has the form
\begin{dmath}\label{eqn:polarization_elliptical:380}
\begin{aligned}
F &= e E_a \lr{ 1 + \Be_3 } \Be_1 e^{ i \psi } \cosh\lr{ m + i \phi} \\
m &= \tanh^{-1}\lr{ E_b/E_a } \\
e &= \sqrt{1 - {(E_b/E_a)}^2 },
\end{aligned}
\end{dmath}
where \( E_a(E_b) \) are the magnitudes of the electric field components lying along the semi-major(minor) axes, and the propagation direction \( \Be_3 \) is normal to both the major and minor axis directions.
An elliptic electric field polarization is illustrated in \cref{fig:ellipticalPolarization:ellipticalPolarizationFig1}, where the vectors representing the major and minor axes are \( \BE_a = E_a \Be_1 e^{i\psi}, \BE_b = E_b \Be_1 e^{i\psi} \).
Observe that setting \( E_b = 0 \) results in the linearly polarized field of \cref{eqn:polarization_linearPolarization:300}.
\imageFigure{../figures/GAelectrodynamics/ellipticalPolarizationFig1}{Electric field with elliptical polarization.}{fig:ellipticalPolarization:ellipticalPolarizationFig1}{0.3}

\section{FIXME:} rewrite marker.

Following the procedure of \cref{eqn:polarization_phaseAndEnergyMomentum:560}, the energy momentum of an elliptically polarized field is
\begin{dmath}\label{eqn:polarization_elliptical:600}
\calE + \frac{\BS}{v}
=
\inv{2} \epsilon
F F^\dagger
=
\inv{2} \epsilon
e^2 E_a^2 \lr{ 1 + \Be_3 } \Be_1 \cancel{e^{ i \psi }} \cosh\lr{ m + i \phi}
\cosh\lr{ m - i \phi}
\cancel{e^{ -i \psi } }
\Be_1
\lr{ 1 + \Be_3 }
=
\inv{2} \epsilon
e^2 E_a^2 \lr{ 1 + \Be_3 }
\lr{ \cosh(2m) + \cos(2 \phi) }
=
\inv{2} \epsilon
\lr{ 1 + \Be_3 }
\lr{ E_b^2 + 2 \lr{
E_a^2 - E_b^2
 } \cos^2 \phi }
.
\end{dmath}

The simplification above made use of the identity \( (1 - (b/a)^2) \cosh(2 \Atanh(b/a)) = 1 + (b/a)^2 \).
%
% $Assumptions = b > 0 && b < 1 && a > 0 && a > b;
% (1 - (b/a)^2) Cosh[2 ArcTanh[b/a]] // FullSimplify

\subsection{Other phase dependence and energy momentum.}

The linear polarization of \cref{eqn:polarization_linearPolarization:300} can be generalized from sinosoidal functions of the phase angle \cref{eqn:polarization_circular:520}, to arbitrary functions, as in
\begin{dmath}\label{eqn:polarization_phaseAndEnergyMomentum:540}
F = \lr{ 1 + \Be_3 } \Norm{\BE} \Be_1 e^{i\psi} f(\phi).
\end{dmath}

For example, \( f(\phi) = e^{i\phi} \) would result in a circularly polarized state, and
a Gaussian modulation could be added into the mix with \( f(\phi) = e^{i \phi - (\phi/\sigma)^2/2 } \).

If the phase dependence of \cref{eqn:polarization_phaseAndEnergyMomentum:540} is a scalar function, then
the energy momentum multivector for the field can be calculated simply
\begin{dmath}\label{eqn:polarization_phaseAndEnergyMomentum:560}
\calE + \frac{\BS}{v}
=
\inv{2} \epsilon
F F^\dagger
=
\inv{2} \epsilon
\lr{ 1 + \Be_3 } \Norm{\BE}^2 \Be_1 \cancel{e^{i\psi}} f^2(\phi)
\cancel{e^{-i\psi} }
\Be_1
\lr{ 1 + \Be_3 }
=
\inv{2} \epsilon
\lr{ 1 + \Be_3 } \Norm{\BE}^2 \cancel{\Be_1} f^2(\phi)
\cancel{\Be_1 }
\lr{ 1 + \Be_3 }
=
\epsilon \lr{ 1 + \Be_3 } \Norm{\BE}^2 f^2(\phi),
\end{dmath}
where the projective property \( \lr{ 1 + \Be_3 }^2 = 2 \lr{ 1 + \Be_3 } \) was used in the final simplification.
The energy, and Poynting vectors are
\begin{dmath}\label{eqn:polarization_phaseAndEnergyMomentum:580}
\begin{aligned}
\calE &= \epsilon \Norm{\BE}^2 f^2(\phi) \\
\BS &= \inv{\eta} \Be_3 \Norm{\BE}^2 f^2(\phi).
\end{aligned}
\end{dmath}
% v epsilon = sqrt( epsilon^2/ (epsilon mu) ) = 1/eta

More care for this calculation is required if the phase function \( f(\phi) \) is multivector valued, since it may not commute with the \( \Be_1 \) and \( e^{i\psi} \) factors of \( F \).
         \subsection{Pseudoscalar imaginary.}
            %

The multivector \( 1 + \Be_3 \) acts as a projector, consuming any factors of \( \Be_3 \)

\begin{dmath}\label{eqn:polarization_pseudoscalarImaginary:440}
(1 + \Be_3) \Be_3
=
\Be_3 + \Be_3^2
=
1 + \Be_3.
\end{dmath}

This property allows all the bivector imaginaries \( i = \Be_{12} = \Be_3 I \) in \cref{eqn:polarization_circular:200} to be re-expressed in terms of the \R{3} pseudoscalar \( I = \Be_{123} \).  To illustrate this consider just the left circular polarized wave

\begin{dmath}\label{eqn:polarization_pseudoscalarImaginary:460}
F_\txtL
=
\lr{ 1 + \Be_3 } \Be_1 \alpha_\txtL e^{i\phi}
=
\lr{ 1 + \Be_3 } \Be_1 \alpha_\txtL \lr{ \cos\phi + \Be_3 I \sin\phi }
=
\lr{ 1 + \Be_3 } \Be_1 \alpha_\txtL \cos\phi
-\lr{ 1 + \Be_3 } \Be_3 \Be_1 \alpha_\txtL I \sin\phi
=
\lr{ 1 + \Be_3 } \Be_1 \alpha_\txtL e^{-I\phi}
=
\lr{ 1 + \Be_3 } \Be_1 \lr{ \alpha_{\txtL 1} + \Be_3 I\alpha_{\txtL 2}  } e^{-I\phi}
=
\lr{ 1 + \Be_3 } \Be_1 \lr{ \alpha_{\txtL 1} - I \alpha_{\txtL 2} } e^{-I\phi}.
\end{dmath}

This shows that the coefficients for the circular polarized states can be redefined using the pseudoscalar as an imaginary (in contrast to the bivector imaginary used in \cref{eqn:polarization_circular:220})
\begin{dmath}\label{eqn:polarization_pseudoscalarImaginary:480}
\begin{aligned}
\alpha_\txtL' &= \alpha_{\txtL 1} - I \alpha_{\txtL 2} \\
\alpha_\txtR' &= \alpha_{\txtR 1} - I \alpha_{\txtR 2},
\end{aligned}
\end{dmath}

so that the plane wave is
\begin{dmath}\label{eqn:polarization_pseudoscalarImaginary:500}
F = \lr{ 1 + \Be_3 } \Be_1 \lr{ \alpha_\txtL' e^{-I\phi} + \alpha_\txtR' e^{I\phi} }.
\end{dmath}

Like \cref{eqn:polarization_circular:200} this plane wave representation does not require taking any real parts.  The transverse plane in which the electric and magnetic fields lie is defined by the duality relation \( i = I \Be_3 \).

The energy momentum multivector for a wave described in terms of the pseudoscalar circular polarization states of \cref{eqn:polarization_pseudoscalarImaginary:500} is just

\begin{dmath}\label{eqn:polarization_pseudoscalarImaginary:620}
\calE + \frac{\BS}{v} =
\epsilon \lr{ 1 + \Be_3 } \lr{ \Abs{\alpha_\txtL'}^2 + \Abs{\alpha_\txtR'}^2 },
\end{dmath}

where the absolute value is computed using the reverse as the conjugation operation \( \Abs{z}^2 = z z^\dagger \).



The multivector \( 1 + \Be_3 \) acts as a projector, consuming any factors of \( \Be_3 \)
\begin{dmath}\label{eqn:polarization_pseudoscalarImaginary:440}
(1 + \Be_3) \Be_3
=
\Be_3 + \Be_3^2
=
1 + \Be_3.
\end{dmath}

This property allows all the bivector imaginaries \( i = \Be_{12} = \Be_3 I \) in \cref{eqn:polarization_circular:200} to be re-expressed in terms of the \R{3} pseudoscalar \( I = \Be_{123} \).  To illustrate this consider just the left circular polarized wave
\begin{dmath}\label{eqn:polarization_pseudoscalarImaginary:460}
F_\txtL
=
\lr{ 1 + \Be_3 } \Be_1 \alpha_\txtL e^{i\phi}
=
\lr{ 1 + \Be_3 } \Be_1 \alpha_\txtL \lr{ \cos\phi + \Be_3 I \sin\phi }
=
\lr{ 1 + \Be_3 } \Be_1 \alpha_\txtL \cos\phi
-\lr{ 1 + \Be_3 } \Be_3 \Be_1 \alpha_\txtL I \sin\phi
=
\lr{ 1 + \Be_3 } \Be_1 \alpha_\txtL e^{-I\phi}
=
\lr{ 1 + \Be_3 } \Be_1 \lr{ \alpha_{\txtL 1} + \Be_3 I\alpha_{\txtL 2}  } e^{-I\phi}
=
\lr{ 1 + \Be_3 } \Be_1 \lr{ \alpha_{\txtL 1} - I \alpha_{\txtL 2} } e^{-I\phi}.
\end{dmath}

This shows that the coefficients for the circular polarized states can be redefined using the pseudoscalar as an imaginary (in contrast to the bivector imaginary used in \cref{eqn:polarization_circular:220})
\begin{dmath}\label{eqn:polarization_pseudoscalarImaginary:480}
\begin{aligned}
\alpha_\txtL' &= \alpha_{\txtL 1} - I \alpha_{\txtL 2} \\
\alpha_\txtR' &= \alpha_{\txtR 1} - I \alpha_{\txtR 2},
\end{aligned}
\end{dmath}
so that the plane wave is
\begin{dmath}\label{eqn:polarization_pseudoscalarImaginary:500}
F = \lr{ 1 + \Be_3 } \Be_1 \lr{ \alpha_\txtL' e^{-I\phi} + \alpha_\txtR' e^{I\phi} }.
\end{dmath}

Like \cref{eqn:polarization_circular:200} this plane wave representation does not require taking any real parts.  The transverse plane in which the electric and magnetic fields lie is defined by the duality relation \( i = I \Be_3 \).

The energy momentum multivector for a wave described in terms of the pseudoscalar circular polarization states of \cref{eqn:polarization_pseudoscalarImaginary:500} is just
\begin{dmath}\label{eqn:polarization_pseudoscalarImaginary:620}
\calE + \frac{\BS}{v} =
\epsilon \lr{ 1 + \Be_3 } \lr{ \Abs{\alpha_\txtL'}^2 + \Abs{\alpha_\txtR'}^2 },
\end{dmath}
where the absolute value is computed using the reverse as the conjugation operation \( \Abs{z}^2 = z z^\dagger \).

         \subsection{Problems.}
            %%
% Copyright © 2018 Peeter Joot.  All Rights Reserved.
% Licenced as described in the file LICENSE under the root directory of this GIT repository.
%
%{

\makeproblem{Circular polarization coefficients relationship to the Jones vector.}{problem:polarization:1}{
By substituting \cref{eqn:polarization_circular:220} into \cref{eqn:polarization_circular:200}, and comparing to \cref{eqn:polarization_circular:160},
show that the circular state coefficients have the following relationship to the Jones vector coordinates
\begin{equation*}
\begin{aligned}
\alpha_\txtL &= \lr{ \alpha_1 + \beta_2 }/2 + i \lr{ -\alpha_2 + \beta_1 }/2 \\
\alpha_\txtR &= \lr{ \alpha_1 - \beta_2 }/2 + i \lr{ -\alpha_2 - \beta_1 }/2,
\end{aligned}
\end{equation*}
and use this to prove \cref{eqn:polarization_circular:260}.
} % problem

\makeproblem{Pseudoscalar Jones vector.}{problem:polarization:2}{
With the Jones vector defined in terms of the \R{3} pseudoscalar
\begin{equation*}
\begin{aligned}
c_1 &= \alpha_1 + I \beta_1 \\
c_2 &= \alpha_2 + I \beta_2,
\end{aligned}
\end{equation*}
calculate the values \( \alpha_\txtL', \alpha_\txtR' \) of \cref{eqn:polarization_pseudoscalarImaginary:480} in terms of this Jones vector.
} % problem
%}


\makeproblem{Circular polarization coefficients relationship to the Jones vector.}{problem:polarization:1}{
By substituting \cref{eqn:polarization_circular:220} into \cref{eqn:polarization_circular:200}, and comparing to \cref{eqn:polarization_circular:160},
show that the circular state coefficients have the following relationship to the Jones vector coordinates
\begin{equation*}
\begin{aligned}
\alpha_\txtL &= \lr{ \alpha_1 + \beta_2 }/2 + i \lr{ -\alpha_2 + \beta_1 }/2 \\
\alpha_\txtR &= \lr{ \alpha_1 - \beta_2 }/2 + i \lr{ -\alpha_2 - \beta_1 }/2,
\end{aligned}
\end{equation*}
and use this to prove \cref{eqn:polarization_circular:260}.
} % problem

\makeproblem{Pseudoscalar Jones vector.}{problem:polarization:2}{
With the Jones vector defined in terms of the \R{3} pseudoscalar
\begin{equation*}
\begin{aligned}
c_1 &= \alpha_1 + I \beta_1 \\
c_2 &= \alpha_2 + I \beta_2,
\end{aligned}
\end{equation*}
calculate the values \( \alpha_\txtL', \alpha_\txtR' \) of \cref{eqn:polarization_pseudoscalarImaginary:480} in terms of this Jones vector.
} % problem

%}
\EndArticle
