%
% Copyright © 2018 Peeter Joot.  All Rights Reserved.
% Licenced as described in the file LICENSE under the root directory of this GIT repository.
%
%{
\subsubsection{Unbounded superposition solutions for the Helmholtz equation.}

We can utilize \cref{eqn:gradientGreensFunctionEuclidean:160} to illustrate the Green's function technique.
As this equation is a linear differential operator relating the wave and the driving sources,
it is reasonable to assume that the solution also has a general linear structure, such as

\begin{dmath}\label{eqn:gradientGreensFunctionEuclidean:100}
F(\Bx) = \int dV' B(\Bx') G(\Bx, \Bx') + F_0(\Bx),
\end{dmath}

where the function \( G(\Bx, \Bx') \) is called the Green's function for the Helmholtz operator, and \( F_0 \) is any particular solution to the inhomogeneous Helmholtz equation \( \lr{ \spacegrad^2 + k^2 } F_0 = 0 \).
Operating on \cref{eqn:gradientGreensFunctionEuclidean:100} with the Helmholtz operator \( \spacegrad + k^2 \) we find that the Green's function must
satisfy

\begin{dmath}\label{eqn:gradientGreensFunctionEuclidean:140}
\lr{ \spacegrad^2 + k^2 } G(\Bx, \Bx') = \delta(\Bx - \Bx').
\end{dmath}

While it is possible \citep{schwinger1998classical} to derive the Green's function using Fourier transform techniques, we will state the result instead, which is well known

\index{Helmholtz!Green's function}
\index{Green's function!Helmholtz}
\maketheorem{Green's function for the Helmholtz operator.}{thm:gradientGreensFunctionEuclidean:3}{
The advancing (causal), and the receding (acausal) Green's functions satisfying
\cref{eqn:gradientGreensFunctionEuclidean:140} are respectively

\begin{equation*}
\begin{aligned}
G_{\textrm{adv}}(\Bx, \Bx') &= -\frac{e^{-j k \Norm{ \Bx - \Bx' } }}{ 4 \pi \Norm{\Bx - \Bx'}} \\
G_{\textrm{rec}}(\Bx, \Bx') &= -\frac{e^{j k \Norm{ \Bx - \Bx' } }}{ 4 \pi \Norm{\Bx - \Bx'}}.
\end{aligned}
\end{equation*}
} % theorem

We will use the advancing Green's function, and refer to this function as \( G(\Bx, \Bx') \) without any subscript.
A demonstration that these Green's function representations are valid can be found in \cref{chap:helmholtzGreens}.

\index{Laplacian!Green's function}
\index{Green's function!Laplacian}
Observe that as a special case, the Helmholtz Green's function reduces to the Green's function for the Laplacian when \( k = 0 \)

\begin{dmath}\label{eqn:gradientGreensFunctionEuclidean:80}
G(\Bx, \Bx') = -\inv{ 4 \pi \Norm{\Bx - \Bx'}}.
\end{dmath}

\subsubsection{Bounded superposition solutions for the Helmholtz equation.}

When the presumed solution is a superposition of only states in a bounded region, such as

\begin{dmath}\label{eqn:gradientGreensFunctionEuclidean:200}
F(\Bx) = \int_V dV' B(\Bx') G(\Bx, \Bx') + F_0(\Bx),
\end{dmath}

then life gets a bit more interesting.
For such problems, we require Green's theorem, which must be generalized slightly for use with multivector fields.
The basic idea is that we can relate the Laplacian's of the Green's function and the field
\( F(\Bx') (\spacegrad')^2 G(\Bx, \Bx') = G(\Bx, \Bx') (\spacegrad')^2 F(\Bx') + \cdots \).
That relation is usually expressed in terms of the difference of the two in the integral domain

\maketheorem{Green's theorem}{thm:gradientGreensFunctionEuclidean:220}{
Given a multivector function \( F \) and a scalar function \( G \)
\begin{equation*}
\int_V \lr{ F \spacegrad^2 G - G \spacegrad^2 F } dV = \int_{\partial V} \lr{ F \ncap \cdot \spacegrad G - G \ncap \cdot \spacegrad F },
\end{equation*}
where \( \partial V \) is the boundary of the volume \( V \).
} % theorem

A straightforward, but perhaps inelligant way of proving this theorem is to expand the sandwich difference in coordinates

\begin{dmath}\label{eqn:gradientGreensFunctionEuclidean:260}
F \spacegrad^2 G - G \spacegrad^2 F
=
\sum_k F \partial_k \partial_k G - G \partial_k \partial_k F
=
\sum_k \partial_k \lr{
F \partial_k G - G \partial_k F
}
-
(\partial_k F)(\partial_k G) + (\partial_k G)(\partial_k F).
\end{dmath}

Since \( G \) is a scalar, the last two terms cancel, and we can integrate

\begin{dmath}\label{eqn:gradientGreensFunctionEuclidean:280}
\int_V \lr{ F \spacegrad^2 G - G \spacegrad^2 F } dV
=
\sum_k \int_V \partial_k \lr{ F \partial_k G - G \partial_k F }.
\end{dmath}

Each integral above involves one component of the gradient.
From
%the fundamental theorem of geometric calculus
\cref{thm:fundamentalTheoremOfCalculus:1}
we know that
\begin{dmath}\label{eqn:gradientGreensFunctionEuclidean:300}
\int_V \spacegrad Q dV = \int_{\partial V} \ncap Q dA,
\end{dmath}

for any multivector \( Q \).
Equating components gives

\begin{dmath}\label{eqn:gradientGreensFunctionEuclidean:340}
\int_V \partial_k Q dV = \int_{\partial V} \ncap \cdot \Be_k Q dA,
\end{dmath}

which can be substituted into \cref{eqn:gradientGreensFunctionEuclidean:280} to find

\begin{dmath}\label{eqn:gradientGreensFunctionEuclidean:360}
\int_V \lr{ F \spacegrad^2 G - G \spacegrad^2 F } dV
=
\sum_k \int_{\partial V} \ncap \cdot \Be_k \lr{ F \partial_k G - G \partial_k F } dA
=
\int_{\partial V} \lr{ F (\ncap \cdot \spacegrad) G - G (\ncap \cdot \spacegrad) F } dA,
\end{dmath}

which proves the theorem.
For our application of
{thm:gradientGreensFunctionEuclidean:3} to the Helmholtz problem, we
are actually interested in a antisymmetric sandwich of the Helmholtz operator by the function \( F \) and the scalar (Green's) function \( G \), but
that reduces to a sandwich of Laplacian's

\begin{dmath}\label{eqn:gradientGreensFunctionEuclidean:240}
F \lr{ \spacegrad^2 + k^2 } G - G \lr{ \spacegrad^2 + k^2 } F
=
F \spacegrad^2 G + \cancel{F k^2 G} - G \spacegrad^2 F - \cancel{G k^2 F}
=
F \spacegrad^2 G - G \spacegrad^2 F,
\end{dmath}

so

\begin{dmath}\label{eqn:gradientGreensFunctionEuclidean:380}
\int_V F(\Bx') \lr{ (\spacegrad')^2 + k^2 } G(\Bx, \Bx')
=
\int_V G(\Bx, \Bx') \lr{ (\spacegrad')^2 + k^2} F(\Bx') dV'
+
\int_{\partial V} \lr{ F(\Bx') (\ncap' \cdot \spacegrad') G(\Bx, \Bx') - G(\Bx, \Bx') (\ncap' \cdot \spacegrad') F(\Bx') } dA'
\end{dmath}

This shows that if we assume the Green's function satisfies
the delta function condition
\cref{eqn:gradientGreensFunctionEuclidean:140}
%that was also true for the unbounded case
, then the general solution to \cref{eqn:gradientGreensFunctionEuclidean:160} is

\boxedEquation{eqn:gradientGreensFunctionEuclidean:400}{
\begin{aligned}
F(\Bx) &=
\int_V G(\Bx, \Bx') B( \Bx' ) dV' \\
&+
\int_{\partial V} \lr{
 G(\Bx, \Bx') (\ncap' \cdot \spacegrad') F(\Bx')
-F(\Bx') (\ncap' \cdot \spacegrad') G(\Bx, \Bx')
} dA'.
\end{aligned}
}

We are also free to add in any specific solution \( F_0(\Bx) \) that satisfies the
homogeneous Helmholtz equation.
There is also freedom to add any solution of the homogeneous Helmholtz equation to the Green's function itself, so it is not unique.
For a bounded superposition we generally desire that the solution \( F \) and its normal derivative, or the Green's function \( G \) (and it's normal derivative) or an appropriate combination of the two are zero on the boundary, so that the surface integral is killed.

%}
