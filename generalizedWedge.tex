%
% Copyright © 2017 Peeter Joot.  All Rights Reserved.
% Licenced as described in the file LICENSE under the root directory of this GIT repository.
%
Like the multivector dot product, we define a multivector wedge product as a grade selection operation

\index{multivector wedge product}
\makedefinition{Multivector wedge product.}{dfn:generalizedWedge:480}{
For the multivectors \( A, B \) defined in \cref{dfn:generalizedDot:100}, the wedge (or outer) product is defined as
\begin{equation*}
A \wedge B
\equiv
\sum_{i,j = 0}^N \gpgrade{ A_i B_j }{i + j}.
\end{equation*}
} % definition

\paragraph{examples}

%For example, the wedge product of the 2D vectors 
%\cref{eqn:generalizedWedge:140} is
%
%\begin{dmath}\label{eqn:generalizedWedge:500}
%\Ba \wedge \Bb
%=
%\gpgradetwo{
%\lr{ x \Be_1 + y \Be_2 }
%\lr{ x' \Be_1 + y' \Be_2 }
%}
%=
%\gpgradetwo{
%(x x' + y y') + (x y' - x' y) \Be_1 \Be_2
%}
%=
%(x y' - x' y) \Be_1 \Be_2.
%\end{dmath}
%
%The wedge product of two vectors in a plane contains an antisymmetrized sum of the vector coefficients, but is weighted by a ``unit'' bivector, the pseudoscalar for the plane.
%
%As another example consider
%
%\begin{dmath}\label{eqn:generalizedWedge:520}
%\Be_1 \wedge \lr{ 2\Be_1 + 3 \Be_2 }
%=
%\gpgradetwo{
%\Be_1 \lr{ 2\Be_1 + 3 \Be_2 }
%}
%=
%\gpgradetwo{
%2 \Be_1^2 + 3 \Be_1 \Be_2
%}
%=
%3 \Be_1 \Be_2.
%\end{dmath}
%
%Components of the vectors are that colinear are filtered out.  In this case that is the \( \Be_1 \) component of the second vector \( \Be_1 + 3 \Be_2 \).  It is not coincidence that this is also a property of the cross product.  That relationship will be explored in (\cref{problem:generalizedWedge:WedgeRelationshipToCrossProduct}).
%
As an example, consider a specific vector-bivector wedge product

\begin{dmath}\label{eqn:generalizedWedge:540}
\Be_1 \wedge \lr{ \Be_1 \Be_2 - 7 \Be_2 \Be_3 }
=
\gpgradethree{
\Be_1 \lr{ \Be_1 \Be_2 - 7 \Be_2 \Be_3 }
}
=
\gpgradethree{
\Be_1^2 \Be_2 - 7 \Be_1 \Be_2 \Be_3
}
=
- 7 \Be_1 \Be_2 \Be_3.
\end{dmath}

Because \( \Be_1 \Be_2 \) has a common factor with \( \Be_1 \) it is filtered out of the resulting wedge product.  The end result, in this case, is a 3D pseudoscalar.

The wedge product of two bivectors in \R{3}, by this definition, is always zero, since there can be no grade 4 term in such a product.  It is also the case that the components of any \R{3} bivectors wedged together will also have a common factor, which necessarily kills the wedge product of any two \R{3} bivectors.  This is not the case for arbitrary \R{N} bivectors, an example of which is \( \Be_1 \Be_2 + \Be_3 \Be_4 \).  There is no common factor in this bivector, so it can be wedged with itself and still produce a non-zero result (i.e. this bivector is not a blade).
