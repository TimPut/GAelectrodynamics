%
% Copyright © 2017 Peeter Joot.  All Rights Reserved.
% Licenced as described in the file LICENSE under the root directory of this GIT repository.
%
We've identified the vector wedge product with scalar grade selection of their vector product, the selection of the highest grade of their product.
Looking back to the multivector products of \cref{eqn:generalizedDot:601}, and \cref{eqn:generalizedDot:621} as motivation, 
a generalized wedge product can be defined that selects the highest grade terms of a given multivector product

\index{multivector wedge product}
\makedefinition{Multivector wedge product.}{dfn:generalizedWedge:480}{
For the multivectors \( A, B \) defined in \cref{dfn:generalizedDot:100}, the wedge (or outer) product is defined as
\begin{equation*}
A \wedge B
\equiv
\sum_{i,j = 0}^N \gpgrade{ A_i B_j }{i + j}.
\end{equation*}
} % definition

If \( A, B \) are a k-vectors with grades \( r, s \) respectively, then their wedge product is a single grade selection

\begin{dmath}\label{eqn:generalizedDot:581}
A \wedge B = \gpgrade{ A B }{ r + s}.
\end{dmath}

\paragraph{examples}

As an example, consider a specific vector-bivector wedge product

\begin{dmath}\label{eqn:generalizedWedge:540}
\Be_1 \wedge \lr{ \Be_2 \wedge (-\Be_1 - 7 \Be_3) }
=
\Be_1 \wedge \lr{ \Be_1 \Be_2 - 7 \Be_2 \Be_3 }
=
\gpgradethree{
\Be_1 \lr{ \Be_1 \Be_2 - 7 \Be_2 \Be_3 }
}
=
\gpgradethree{
\Be_1^2 \Be_2 - 7 \Be_1 \Be_2 \Be_3
}
=
- 7 \Be_1 \Be_2 \Be_3.
\end{dmath}

Because \( \Be_1 \Be_2 \) has a common factor with \( \Be_1 \) it is filtered out of the resulting wedge product.  The end result, in this case, is a 3D pseudoscalar.

The wedge product of two bivectors in \R{3}, by this definition, is always zero, since there can be no grade 4 term in such a product.  It is also the case that the components of any \R{3} bivectors wedged together will also have a common factor, which necessarily kills the wedge product of any two \R{3} bivectors.  This is not the case for arbitrary \R{N} bivectors, an example of which is \( \Be_1 \Be_2 + \Be_3 \Be_4 \).  There is no common factor in this bivector, so it can be wedged with itself and still produce a non-zero result (i.e. this bivector is not a blade).
