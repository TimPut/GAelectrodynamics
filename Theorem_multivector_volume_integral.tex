%
% Copyright � 2018 Peeter Joot.  All Rights Reserved.
% Licenced as described in the file LICENSE under the root directory of this GIT repository.
%
\maketheorem{Fundamental theorem for volume integrals.}{thm:volumeintegral:100}{
Given a
% piecewise-smooth
continuous and connected volume
 \( V = \Bx(u, v, w) \)
parameterized by \( u \in [u_0, u_1], v \in [v_0, v_1], w \in [w_0, w_1] \), multivector functions \( F(\Bx), G(\Bx) \) that are differentable over \( V \), and
an (trivector-valued) volume element \( d^3 \Bx = d\Bx_1 \wedge d\Bx_2 \wedge d\Bx_3 = du dv dw\, \Bx_u \wedge \Bx_v \wedge \Bx_w \)
\begin{equation*}
\int_V F d^3\Bx \lrboldpartial G
= \ointclockwise_{\partial V} F d^2 \Bx G,
\end{equation*}
where \( \partial V \) is the boundary of the volume \( V \),
and \( d^2 \Bx \) is the counterclockwise oriented area element on the boundary of the volume, that is
\begin{equation*}
\ointclockwise_{\partial V} F d^2 \Bx G
=
\int \evalbar{\lr{F d\Bx_1 \wedge d\Bx_2 G}}{\Delta w}
+\int \evalbar{\lr{F d\Bx_2 \wedge d\Bx_3 G}}{\Delta u}
+\int \evalbar{\lr{F d\Bx_3 \wedge d\Bx_1 G}}{\Delta v}.
\end{equation*}
In \R{3} with \( d^3 \Bx = I dV \), \( d^2 \Bx = I \ncap dA \), this integral can be written using a scalar volume element, as
\begin{equation*}
\int_V dV\, F \lrboldpartial G
= \int_{\partial V} dA\, F \ncap G.
\end{equation*}
} % theorem

