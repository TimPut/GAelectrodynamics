%
% Copyright � 2016 Peeter Joot.  All Rights Reserved.
% Licenced as described in the file LICENSE under the root directory of this GIT repository.
%
%{
\input{../latex/blogpost.tex}
\renewcommand{\basename}{maxwells}
%\renewcommand{\dirname}{notes/phy1520/}
\renewcommand{\dirname}{notes/ece1228-electromagnetic-theory/}
%\newcommand{\dateintitle}{}
%\newcommand{\keywords}{}

\input{../latex/peeter_prologue_print2.tex}

\usepackage{peeters_layout_exercise}
%\usepackage{peeters_layout}
\usepackage{peeters_braket}
\usepackage{peeters_figures}
\usepackage{siunitx}
\usepackage{macros_qed}
\usepackage{macros_cal}
%\usepackage{mhchem} % \ce{}
\usepackage{macros_bm} % \bcM
%\usepackage{txfonts} % \ointclockwise

\usepackage{tcolorbox}
\usepackage{tabularx}
\usepackage{array}
\usepackage{colortbl}
\tcbuselibrary{skins}

%\newcommand{\lrspacetimegrad}[2]{\lr{\stackrel{ \leftrightarrow }{#2 \spacegrad + #1 \PD{t}{}}}}
%\newcommand{\lspacetimegrad}[2]{\lr{\stackrel{ \leftarrow }{#2 \spacegrad + #1 \PD{t}{}}}}
%\newcommand{\rspacetimegrad}[2]{\lr{\stackrel{ \rightarrow }{#2 \spacegrad + #1 \PD{t}{}}}}

\newcolumntype{Y}{>{\raggedleft\arraybackslash}X}

\tcbset{tab1/.style={fonttitle=\bfseries\large,fontupper=\normalsize\sffamily,
colback=yellow!10!white,colframe=red!75!black,colbacktitle=Salmon!40!white,
coltitle=black,center title,freelance,frame code={
\foreach \n in {north east,north west,south east,south west}
{\path [fill=red!75!black] (interior.\n) circle (3mm); };},}}

\tcbset{tab2/.style={enhanced,fonttitle=\bfseries,fontupper=\normalsize\sffamily,
colback=yellow!10!white,colframe=red!50!black,colbacktitle=Salmon!40!white,
coltitle=black,center title}}

\newcommand{\e}[0]{{(\mathrm{e})}}
\newcommand{\m}[0]{{(\mathrm{m})}}

%
% https://tex.stackexchange.com/a/28212/15
%
%For bold and upright, you could use the regular text-version of \imath and \jmath, which are \i and \j:
%
%\newcommand{\ihat}{\hat{\textbf{\i}}}
%\newcommand{\jhat}{\hat{\textbf{\j}}}
%bold & upright i & j in math
%
%Additionally, if you want the \hat to be bold as well, then use \boldsymbol.
%
%\newcommand{\ihat}{\boldsymbol{\hat{\textbf{\i}}}}
%\newcommand{\jhat}{\boldsymbol{\hat{\textbf{\j}}}}

\beginArtNoToc

%\generatetitle{Bivector, Trivector, Multivector, and Multivector space}
\generatetitle{scratch notes.}

% this is the E, H extraction example (digression)
%%
% Copyright © 2016 Peeter Joot.  All Rights Reserved.
% Licenced as described in the file LICENSE under the root directory of this GIT repository.
%
\section{Conventional differential form}

The differential form of Maxwell's equations, with extensions for magnetic sources, is the starting point for all the analysis in these notes.  Those equations are

\input{../ece1229-antenna/MaxwellsStatement.tex}

The magnetic sources can be considered fictional, but are useful for modelling real phenomina such as infinitesimal current loops, especially in antenna theory.

\input{../ece1229-antenna/MaxwellsFieldAndSourceDescription.tex}

The fields and sources are all real valued functions of both space and time.  In many situations it will be desirable to work with a time harmonic (frequency-domain phasor) form of Maxwell's equations.  In engineering, a time harmonic representation presumes that all sources and fields have a frequency dependence of the form
\index{time harmonic}

\begin{dmath}\label{eqn:maxwellsEquations:20}
\bcY(\Bx, t) = \Real( \BY(\Bx, \omega) e^{j\omega t} ),
\end{dmath}

where the field (or source) \( \BY(\Bx, \Bomega) \) is allowed to be complex valued.  Given this frequency dependence Maxwell's equations take the form

\input{../ece1229-antenna/MaxwellsTimeHarmonic.tex}

Note that the time harmonic convention typically used in physics literature presumes a frequency dependence of the form

\begin{dmath}\label{eqn:maxwellsEquations:40}
\bcY(\Bx, t) = \Real( \BY(\Bx, \omega) e^{-i\omega t} ),
\end{dmath}

which alters the sign of any imaginary originating from a time derivative.  Care is required by the reader to understand which form of frequency dependence has been assumed.

\section{GA differential form}

Geometric Algebra admits a number of alternative representations of Maxwell's equations.  The first follows from expressing the cross products all as wedge products, leaving a pair of bivector and a pair of scalar equations

\begin{subequations}
\begin{dmath}\label{eqn:maxwellsEquations:60}
\spacegrad \wedge \bcE = - I \bcM - \PD{t}{I\bcB}
\end{dmath}
\begin{dmath}\label{eqn:maxwellsEquations:80}
\spacegrad \wedge \bcH = I \bcJ + I \PD{t}{\bcD}
\end{dmath}
\begin{dmath}\label{eqn:maxwellsEquations:100}
\spacegrad \cdot \bcD = q_\txte
\end{dmath}
\begin{dmath}\label{eqn:maxwellsEquations:120}
\spacegrad \cdot \bcB = q_\txtm.
\end{dmath}
\end{subequations}

Alternatively, the duality transformation \( \Ba \wedge \Bb = -I \Ba \cdot (I \Bb) \) allows Maxwell's equations to be all written as dot products

\begin{subequations}
\begin{dmath}\label{eqn:maxwellsEquations:140}
\spacegrad \cdot (I \bcE) = \bcM + \PD{t}{\bcB}
\end{dmath}
\begin{dmath}\label{eqn:maxwellsEquations:160}
\spacegrad \cdot (I \bcH) = -\bcJ - \PD{t}{\bcD}
\end{dmath}
\begin{dmath}\label{eqn:maxwellsEquations:180}
\spacegrad \cdot \bcD = q_\txte
\end{dmath}
\begin{dmath}\label{eqn:maxwellsEquations:200}
\spacegrad \cdot \bcB = q_\txtm,
\end{dmath}
\end{subequations}

or, using the duality transformation \( \Ba \cdot \Bb = -I (\Ba \wedge (I \Bb) \), Maxwell's equations can all be written as wedge products

\begin{subequations}
\begin{dmath}\label{eqn:maxwellsEquations:220}
\spacegrad \wedge \bcE = - I \bcM - \PD{t}{I\bcB}
\end{dmath}
\begin{dmath}\label{eqn:maxwellsEquations:240}
\spacegrad \wedge \bcH = I \bcJ + I \PD{t}{\bcD}
\end{dmath}
\begin{dmath}\label{eqn:maxwellsEquations:260}
\spacegrad \wedge (I\bcD) = I q_\txte
\end{dmath}
\begin{dmath}\label{eqn:maxwellsEquations:280}
\spacegrad \wedge (I\bcB) = I q_\txtm.
\end{dmath}
\end{subequations}

Each of these forms can be useful in different circumstances, however the real power of GA in electromagnetism follows from presuming constituative relationships between the pairs of electric and magnetic fields

\begin{subequations}
\label{eqn:maxwellsEquations:300}
\begin{dmath}\label{eqn:maxwellsEquations:320}
\bcB = \mu \bcH
\end{dmath}
\begin{dmath}\label{eqn:maxwellsEquations:340}
\bcD = \epsilon \bcE,
\end{dmath}
\end{subequations}

where \( \epsilon \) is the permitivitity of the medium [\si{F/m}] (Farads/meter), and \( \mu \) is the permeability of the medium [\si{H/m}] (Henries/meter).
The permitivitity and permeability may be functions of both time and position, and model the materials that the fields are propagating through.  In general, the these may be non-isotropic tensor operators, however, unless otherwise specified, isotropic media will be assumed in these notes.

With this constitutative relationship assumed (and a bit of rescaling), the dot and wedge products of \cref{eqn:maxwellsEquations:60}, \cref{eqn:maxwellsEquations:100} can be added, as can those of \cref{eqn:maxwellsEquations:80}, \cref{eqn:maxwellsEquations:120}.  This reduces Maxwell's equations to a pair of first order coupled gradient equations

\begin{subequations}
\label{eqn:maxwellsEquations:361}
\begin{dmath}\label{eqn:maxwellsEquations:360}
\spacegrad \bcE = \inv{\epsilon} q_\txte - I \bcM - \mu \PD{t}{(I\bcH)}
\end{dmath}
\begin{dmath}\label{eqn:maxwellsEquations:380}
\spacegrad (I \bcH) = \frac{I q_\txtm}{\mu} - \bcJ - \epsilon \PD{t}{\bcE}.
\end{dmath}
\end{subequations}

Note that it is more natural to work with a bivector magnetic field \( I \bcH \) in GA, than it is to work with a vector field \( \bcH \).  Observe that, when magnetic sources are included, this pair of coupled equations have sources of each grade (scalar, vector, bivector, and pseudoscalar).

The multivector equation \cref{eqn:maxwellsEquations:360} has grades 0,2 (scalar and bivector), whereas the multivector equation \cref{eqn:maxwellsEquations:380} has grades 1,3 (vector, pseudoscalar).  This means that arbitrary linear combinations of these equations, such as \( \spacegrad (a \bcE + b I \bcH ) \), are possible without any loss of information, since the original equations can then be recovered by grade selection.  To determine a desirable scaling of such a sum, these equations can be non-dimensionalized by expressing the fields as \( \sqrt{\epsilon} \bcE, \sqrt{\mu} \bcH \)

\begin{subequations}
\begin{dmath}\label{eqn:maxwellsEquations:400}
\spacegrad \sqrt{\epsilon} \bcE = \inv{\sqrt{\epsilon}} q_\txte - I \sqrt{\epsilon} \bcM - \sqrt{\epsilon \mu} \PD{t}{(I\sqrt{\mu} \bcH)}
\end{dmath}
\begin{dmath}\label{eqn:maxwellsEquations:420}
\spacegrad (I \sqrt{\mu} \bcH) = \frac{I q_\txtm}{\sqrt{\mu}} - \sqrt{\mu} \bcJ - \sqrt{\epsilon\mu} \PD{t}{\sqrt{\epsilon} \bcE}.
\end{dmath}
\end{subequations}

The dimensions of both differential operators are now equal \( [\spacegrad] = [\sqrt{\epsilon\mu} \PDi{t}{}] = 1/L \), allowing the remaining two multivector Maxwell equations to be decoupled into a single first order equation to solve for the multivector field \( \sqrt{\epsilon} \bcE + I \sqrt{\mu} \bcH \)

\begin{dmath}\label{eqn:maxwellsEquations:440}
\lr{ \spacegrad + \sqrt{\epsilon\mu} \PDi{t}{} }
\lr{ \sqrt{\epsilon} \bcE
\pm
I \sqrt{\mu} \bcH
}
=
\inv{\sqrt{\epsilon}} q_\txte
- I \sqrt{\epsilon} \bcM
\pm \lr{
+ \frac{I q_\txtm}{\sqrt{\mu}}
- \sqrt{\mu} \bcJ
}
.
\end{dmath}

Whether or not to add or subtract is essentially a phase choice for the electric field relative to the magnetic field.  It is conventional to pick the sum rather than the difference.  In engineering, with \( \bcE \) and \( \bcH \) as the primary fields, Maxwell's equation can now be expressed in its multivector form

\boxedEquation{eqn:maxwellsEquations:460}{
\begin{aligned}
\bcF &= \bcE + \eta I \bcH \\
\lr{ \spacegrad + \inv{v} \PD{t}{} } \bcF
&=
\inv{\epsilon v} \lr{ v q_\txte - \bcJ }
+ I \lr{ v q_\txtm - \bcM }
,
\end{aligned}
}

where \( \eta = \sqrt{\mu/\epsilon} \) (\( [\Omega] \) Ohms)
is the impedance of the media
, and \( v = 1/\sqrt{\epsilon\mu} \)
([\si{m/s}] meters/second)
is the group velocity of a wave in the media.

In the time harmonic representation the electromagnetic field will be of the form

\begin{dmath}\label{eqn:maxwellsEquations:620}
F = \BE + \eta I \BH,
\end{dmath}

where \( \BE \) and \( \BH \) are complex.

\makedigression{
\input{../frequencydomain/frequencydomainMaxwellsExtraction.tex}
}

\section{Wave equation.}

Having assembled all of Maxwell's equations into \cref{eqn:maxwellsEquations:460}, some results now follow almost trivially.  One such result is the wave equation in space free of sources.  In such a region, Maxwell's equation is just

\begin{dmath}\label{eqn:maxwellsEquations:480}
\lr{ \spacegrad + \inv{v} \PD{t}{} } \bcF = 0.
\end{dmath}

This can be multiplied with \( \spacegrad - \inv{v} \PD{t}{} \), to give

\begin{dmath}\label{eqn:maxwellsEquations:500}
0 =
\lr{ \spacegrad - \inv{v} \PD{t}{} }
\lr{ \spacegrad + \inv{v} \PD{t}{} } \bcF
=
\lr{ \spacegrad^2 - \inv{v^2} \PDSq{t}{} } \bcF,
\end{dmath}

or

\begin{dmath}\label{eqn:maxwellsEquations:520}
\spacegrad^2 \bcF = \inv{v^2} \PDSq{t}{\bcF}.
\end{dmath}

Since \( \spacegrad^2 \) is a scalar operator, selection of the vector and bivector components of \cref{eqn:maxwellsEquations:520} gives

\begin{dmath}\label{eqn:maxwellsEquations:540}
\begin{aligned}
\spacegrad^2 \bcE &= \inv{v^2} \PDSq{t}{\bcE} \\
\spacegrad^2 (I \bcH) &= \inv{v^2} \PDSq{t}{(I \bcH)} \\
\end{aligned}
\end{dmath}

These equations can be solved independently, provided the solutions are also constrained by Maxwell's equation \cref{eqn:maxwellsEquations:480}.

\section{Plane waves.}

In the time harmonic representation for source free conditions Maxwell's equation \cref{eqn:maxwellsEquations:460} is just
\begin{dmath}\label{eqn:maxwellsEquations:560}
\begin{aligned}
F &= \BE + \eta I \BH \\
\lr{ \spacegrad + j k } F &= 0,
\end{aligned}
\end{dmath}

where \( k = \omega/v \) is the wave number.  It is now possible to examine what constraints Maxwell's equation imposes on plane waves of the form

\begin{dmath}\label{eqn:maxwellsEquations:580}
\begin{aligned}
\BE &= \BE_0 e^{-j \Bk \cdot \Bx} \\
\BH &= \BH_0 e^{-j \Bk \cdot \Bx},
\end{aligned}
\end{dmath}

or
\begin{dmath}\label{eqn:maxwellsEquations:600}
F = F_0 e^{-j \Bk \cdot \Bx}.
\end{dmath}

%
% Copyright © 2016 Peeter Joot.  All Rights Reserved.
% Licenced as described in the file LICENSE under the root directory of this GIT repository.
%
%\section{Plane waves}
\index{plane wave}
The gradient action on the electromagnetic field is

\begin{dmath}\label{eqn:frequencydomainPlaneWaves:160}
\spacegrad F_0 e^{-j \Bk \cdot \Bx}
=
\sum_{m = 1}^3 \Be_m \partial_m
F_0 e^{-j \Bk \cdot \Bx}
=
\sum_{m = 1}^3 \Be_m
F_0
\lr{ -j k_m }
e^{-j \Bk \cdot \Bx}
=
-j \Bk F_0,
\end{dmath}
so

\begin{dmath}\label{eqn:frequencydomainPlaneWaves:180}
j k (1 - \kcap) F_0 = 0.
\end{dmath}

This means that the field must be of the form

%\begin{dmath}\label{eqn:frequencydomainPlaneWaves:200}
\boxedEquation
{eqn:frequencydomainPlaneWaves:200}
{
F = (1 + \kcap) \BE_0 e^{-j \Bk \cdot \Bx},
}
%\end{dmath}
where \( \BE_0 \) is a vector valued complex constant, and \( \kcap \cdot \BE_0 = 0 \).
The dot product constraint follows from the requirement that the \( I \BH \propto \kcap \BE_0 \) portion of the electromagnetic field is a bivector.
The time domain representation of the field is
\begin{dmath}\label{eqn:frequencydomainPlaneWaves:460}
F = (1 + \kcap) \Real{ \BE_0 e^{-j \Bk \cdot \Bx} },
\end{dmath}
but we will see later
instead of using a scalar imaginary \( j \), it is possible to use either the unit bivector for the transverse plane or the \R{3} unit pseudoscalar as the imaginary, and that a plane wave of any polarization can be encoded without any requirement to take real parts.

From \cref{eqn:frequencydomainPlaneWaves:200} the interdependence of the electric and magnetic field portions of the field can be read off immediately.
Those are

\begin{subequations}
\label{eqn:frequencydomainPlaneWaves:220}
\begin{dmath}\label{eqn:frequencydomainPlaneWaves:221}
\BE = \BE_0 e^{-j \Bk \cdot \Bx}
\end{dmath}
\begin{dmath}\label{eqn:frequencydomainPlaneWaves:222}
I \BH = \inv{\eta} \kcap \BE_0 e^{-j \Bk \cdot \Bx},
\end{dmath}
\end{subequations}

or
\begin{dmath}\label{eqn:frequencydomainPlaneWaves:380}
I \BH = \inv{\eta} \kcap \BE.
\end{dmath}

\index{pseudoscalar!spherical}
Since the \R{3} pseudoscalar can be written as

\begin{dmath}\label{eqn:frequencydomainPlaneWaves:400}
I = \kcap \Ecap \Hcap,
\end{dmath}
the directions \( \kcap, \Ecap, \Hcap \) must form a right handed triple.
It is thus expected that the magnetic field is perpendicular to the propagation direction, and that the electric and magnetic fields are explicitly perpendicular, facts that are easily verified

\begin{subequations}
\label{eqn:frequencydomainPlaneWaves:440}
\begin{dmath}\label{eqn:frequencydomainPlaneWaves:260}
\kcap \cdot \BH
= \gpgradezero{ \kcap (-I \kcap \BE_0) } e^{-j \Bk \cdot \Bx}
= -\gpgradezero{ I \BE_0 } e^{-j \Bk \cdot \Bx}
= 0
\end{dmath}
\begin{dmath}\label{eqn:frequencydomainPlaneWaves:280}
\BE \cdot \BH
=
\gpgradezero{ \BE \lr{ -\frac{I}{\eta}} \kcap \BE }
=
-\inv{\eta} \BE^2
\gpgradezero{ \kcap I }
=
0.
\end{dmath}
\end{subequations}

In conventional vector treatments of electromagnetic field theory the field relationships of \cref{eqn:frequencydomainPlaneWaves:220} and the propagation directions are written out explicitly as cross products, instead of multivector equations.
Those cross product relations are obtained easily

\begin{subequations}
\label{eqn:frequencydomainPlaneWaves:420}
\begin{dmath}\label{eqn:frequencydomainPlaneWaves:240}
\BH
= -I \inv{\eta} \kcap \BE
= -I \inv{\eta} (\kcap \wedge \BE)
= -I \inv{\eta} I (\kcap \cross \BE)
= \inv{\eta} \kcap \cross \BE
\end{dmath}
\begin{dmath}\label{eqn:frequencydomainPlaneWaves:300}
\BE
= \eta \kcap I \BH
= \eta I \kcap \wedge \BH
= \eta I^2 \kcap \cross \BH
= \eta \BH \cross \kcap
\end{dmath}
\begin{dmath}\label{eqn:frequencydomainPlaneWaves:340}
\kcap
= I \Hcap \Ecap
= I (\Hcap \wedge \Ecap)
= I^2 (\Hcap \cross \Ecap)
= \Ecap \cross \Hcap.
\end{dmath}
\end{subequations}


\section{Poynting theorem}

Poynting's theorem describes the relationship between the flux of energy through a surface bounding a volume.
The theorem follows from computing the divergence of the Poynting vector \( \bcS = \bcE \cross \bcH \).  With the GA toolbox at hand, this divergence can be written as a scalar selection

\begin{equation}\label{eqn:maxwellsEquations:640}
\spacegrad \cdot \lr{ \bcE \cross \bcH }
=
\gpgradezero{ \spacegrad (-I) \lr{ \bcE \wedge \bcH } }
=
-\gpgradezero{ \spacegrad \bcE I \bcH }.
\end{equation}

Here the gradient is acting on everything to the right, however, allowing the gradient to act bidirectionally, and employing the
the flexibility to use cyclic permutation within a scalar selection
(\(\gpgradezero{AB} = \gpgradezero{BA}\))
, allows for the easy application of the chain rule

\begin{dmath}\label{eqn:maxwellsEquations:760}
-\gpgradezero{ \spacegrad \bcE I \bcH }
=
-\gpgradezero{ I \bcH \lrspacegrad \bcE }
=
-\gpgradezero{ I \bcH (\rspacegrad \bcE) }
-\gpgradezero{ (I \bcH \lspacegrad) \bcE }.
\end{dmath}

Explicit left and right acting gradients are required because the gradient operator does not commute with the vector fields.

The gradient action on \( \bcE \) (from the left) is given by \cref{eqn:maxwellsEquations:360}.  The right acting gradient action on \( I \bcH \) is given by reversing
all the products in
%\spacegrad \bcE = \inv{\epsilon} q_\txte - I \bcM - \mu \PD{t}{(I\bcH)}
\cref{eqn:maxwellsEquations:380}

\begin{dmath}\label{eqn:maxwellsEquations:660}
I \bcH \lspacegrad = \frac{I q_\txtm}{\mu} + \bcJ + \epsilon \PD{t}{\bcE}.
\end{dmath}

This gives
\begin{dmath}\label{eqn:maxwellsEquations:680}
0
=
\spacegrad \cdot \lr{ \bcE \cross \bcH }
+
\gpgradezero
{
I \bcH
\lr{ \inv{\epsilon} q_\txte - I \bcM - \mu \PD{t}{(I\bcH)} }
+
\lr{ \frac{I q_\txtm}{\mu} + \bcJ + \epsilon \PD{t}{\bcE} } \bcE
},
%=
%\spacegrad \cdot \lr{ \bcE \cross \bcH }
%+
%\bcM \cdot \bcH + \PD{t}{\bcB} \cdot \bcH
%+ \bcJ \cdot \bcE + \PD{t}{\bcD} \cdot \bcE,
\end{dmath}

or
%\begin{dmath}\label{eqn:maxwellsEquations:700}
\boxedEquation{eqn:maxwellsEquations:720}{
0 =
\spacegrad \cdot \lr{ \bcE \cross \bcH }
+
\bcH \cdot \bcM + \bcJ \cdot \bcE
+ \PD{t}{\bcB} \cdot \bcH
+ \PD{t}{\bcD} \cdot \bcE.
}
%\end{dmath}

The last two terms is the time rate of change of the energy density.  Consider the change of energy density through a volume with neither electric nor magnetic current sources in that region of space

\begin{dmath}\label{eqn:maxwellsEquations:740}
\PD{t}{} \int_V
\inv{2} dV \lr{
\bcB \cdot \bcH
+ \bcD \cdot \bcE
}
=
-\int_{\partial V} dA \ncap \cdot \bcS.
\end{dmath}

Here \( \ncap \) is the outward normal, so if the energy contained in the volume is decreasing, then \( \bcS \) must represent the energy per unit area that leaves the volume.  The direction of the Poynting vector is the direction that the energy is leaving the volume.  Only the components of the Poynting vector that are colinear with the surface normal will result in energy leaving or entering the volume.



%
% Copyright © 2017 Peeter Joot.  All Rights Reserved.
% Licenced as described in the file LICENSE under the root directory of this GIT repository.
%
\subsection{Complex power.}
TODO.
%\index{complex power}

%%
% Copyright © 2017 Peeter Joot.  All Rights Reserved.
% Licenced as described in the file LICENSE under the root directory of this GIT repository.
%

A fair amount of nomenclature and notation is unfortunately required before systematically examining the implications of the multivector space axioms that define geometric algebra.

Multivectors which can be factored into normal vector products, such as
\begin{dmath}\label{eqn:multiplication:220}
\Be_1 \Be_2 + 3 \Be_1 \Be_3
=
\Be_1 (\Be_2 + 3 \Be_3),
\end{dmath}

are blades.
In contrast, the following grade 2 multivectors

\begin{dmath}\label{eqn:multiplication:240}
\Be_1 \Be_2 + \Be_3 \Be_4,
\end{dmath}

and
\begin{dmath}\label{eqn:multiplication:260}
\Be_1 \Be_2 + \Be_2 \Be_3 + \Be_3 \Be_1,
\end{dmath}

which cannot be factored into two vector products, are not blades.

\index{k-vector}
\makedefinition{k-vector.}{dfn:multivector:kvector}{
A sum of k-blades is called a k-vector.
} % definition

Multivectors are therefore sums of k-vectors with different grades.

All the k-blade examples in 
\cref{eqn:multivector:180}
 are also k-vectors.
K-vectors with grades 2 and 3 are so pervasive that they are given special names.

\index{bivector}
\makedefinition{Bivector.}{dfn:multivector:bivector}{
A bivector, or 2-vector, is a k-vector with grade 2.
} % definition

Any 2-blade, such as the product \( \Be_1 \Be_2 \) is a bivector.
Any sum of 2-blades, such as \( \Be_2 \Be_3 + 3 \Be_4 \Be_1 \), is also a bivector.
%Each of \( \Be_1 \Be_2, \Be_2 \Be_1, \Be_1 \Be_2 + \Be_2 \Be_3 \), and \( \Be_1 \Be_2 + \Be_3 \Be_4 \) are bivectors.
%All but the last of these represents an oriented plane segment.

\index{trivector}
\makedefinition{Trivector.}{dfn:multivector:trivector}{
A trivector, or 3-vector, is a k-vector with grade 3.
} % definition

%Quantities with higher grades than 3 are not generally given explicit names.
The multivector \( \Be_3 \Be_1 \Be_2 \) is a trivector, as is \( \Be_1 \Be_2 \Be_3 + 3 \Be_5 \Be_4 \Be_1 \).
The latter is not a blade.
%Each of \( \Be_1 \Be_2 \Be_3, \Be_1 \Be_3 \Be_2, \Be_1 \Be_4 \Be_2 \) are trivectors.
% , and represent oriented volumes.



%\section{Polarization.}
%%
% Copyright © 2017 Peeter Joot.  All Rights Reserved.
% Licenced as described in the file LICENSE under the root directory of this GIT repository.
%
%{
\index{plane wave}
\index{polarization}
In a discussion of polarization, it is convenient to align the propagation direction along a fixed direction, usually the z-axis.
Setting \( \kcap = \Be_3, \beta z = \Bk \cdot \Bx \) in \cref{eqn:frequencydomainPlaneWaves:200} the plane wave representation of the field is

\begin{dmath}\label{eqn:polarization:20}
\begin{aligned}
F(\Bx, \omega) &= (1 + \Be_3) \BE e^{-j \beta z} \\
F(\Bx, t) &= \Real\lr{ F(\Bx, \omega) e^{j \omega t} }.
\end{aligned}
\end{dmath}

Here the imaginary \( j \) has no intrinsic geometrical interpretation, \( \BE = \BE_\txtr + j \BE_\txti \) is allowed to have complex values, and all components of \( \BE \) is perpendicular to the propagation direction (\( \Be_\txtr \cdot \Be_3 = \BE_\txti \cdot \Be_3 = 0 \)).
\index{Jones vector}
A common representation of the electric field components is the Jones vector \( (c_1, c_2) \), which specifies complex coefficients for the electric field phasor in each of the possible directions

\begin{dmath}\label{eqn:polarization:120}
\BE = c_1 \Be_1 + c_2 \Be_2,
\end{dmath}

where \( c_1, c_2 \) are complex valued, say

\begin{dmath}\label{eqn:polarization:140}
\begin{aligned}
c_1 &= \alpha_1 + j \beta_1 \\
c_2 &= \alpha_2 + j \beta_2.
\end{aligned}
\end{dmath}

The tuple \( (c_1, c_2) \) is called the Jones vector, and compactly encodes the geometry of the pattern that the electric field traces out in the transverse plane.


%%
% Copyright © 2017 Peeter Joot.  All Rights Reserved.
% Licenced as described in the file LICENSE under the root directory of this GIT repository.
%
%{
Geometric algebra takes a vector space and adds two additional operations, a vector multiplication operation, and a generalized addition operation that extends vector addition to include addition of scalars and products of vectors.
Multiplication of vectors is indicated by juxtaposition, for example, if \( \Bx, \By, \Be_1, \Be_2, \Be_3, \cdots \) are vectors, then some vector products are

\begin{dmath}\label{eqn:multivector:20}
\begin{aligned}
&\Bx \By, \Bx \By \Bx, \Bx \By \Bx \By, \\
&\Be_1 \Be_2, \Be_2 \Be_1, \Be_2 \Be_3, \Be_3 \Be_2, \Be_3 \Be_1, \Be_1 \Be_3, \\
&\Be_1 \Be_2 \Be_3, \Be_3 \Be_1 \Be_2, \Be_2 \Be_3 \Be_1, \Be_3 \Be_2 \Be_1, \Be_2 \Be_1 \Be_3, \Be_1 \Be_3 \Be_2, \\
&\Be_1 \Be_2 \Be_3 \Be_1, \Be_1 \Be_2 \Be_3 \Be_1 \Be_3 \Be_2, \cdots
\end{aligned}
\end{dmath}

Vector multiplication is constrained by a rule, the contraction axiom, which specifies that the square of vector is the squared length of that vector (i.e. a scalar).

In a sum of scalars, vectors, and vector products, such as
\begin{dmath}\label{eqn:multivector:40}
1 + 2 \Be_1 + 3 \Be_1 \Be_2 + 4 \Be_1 \Be_2 \Be_3,
\end{dmath}
\( \Be_1 \Be_2 \) is called a bivector, \( \Be_1 \Be_2 \Be_3 \) is called a trivector, and the sum itself is a multivector.

Put more formally

\index{bivector}
\index{2-vector}
\makedefinition{Bivector.}{dfn:multivector:60}{
A bivector, or 2-vector, is a sum of products of pairs of normal vectors.
Given an \( N \) dimensional vector space with an orthonormal basis \( \setlr{ \Be_1, \Be_2, \cdots } \),
a general bivector can be expressed as
\begin{equation*}
\sum_{1 \le i \ne j \le N} B_{ij} \Be_i \Be_j,
\end{equation*}
where \( B_{ij} \) is a scalar.
} % definition

Just as vectors can represent line segments with direction and magnitude, a bivector can represent an plane segment with an orientation (both the orientation of the plane in space, and a sidedness) and magnitude.
Fixme: pictures.
We will see that the products of normal vectors, like \( \Be_1 \Be_2 \) anticommute\footnote{Quantities that anticommute are unchanged if both the order and the sign are toggled.},
for example \( \Be_2 \Be_1 = -\Be_1 \Be_2 \).
This means that many of the products in \cref{eqn:multivector:20} are not independent, and that the definition of a bivector could be a more restrictive sum, such as \( \sum_{1 \le i < j \le N} b_{ij} \Be_i \Be_j \), where \( b_{ij} \) is an antisymmetric
\footnote{An indexed quantity such as \( b_{ij} \) is antisymmetric if toggling the order of indexes changes the sign, that is \( b_{ji} = -b_{ij} \).}
scalar.

\index{trivector}
\index{3-vector}
\makedefinition{Trivector.}{dfn:multivector:80}{
A trivector, or 3-vector, is a sum of products of triplets of mutually normal vectors.
Given an \( N \) dimensional vector space with an orthonormal basis \( \setlr{ \Be_1, \Be_2, \cdots } \),
a general trivector can be expressed as
\begin{equation*}
\sum_{1 \le i \ne j \ne k \le N} T_{ijk} \Be_i \Be_j \Be_k,
\end{equation*}
where \( \T_{ijk} \) is a scalar.
} % definition

A trivector can represent an oriented volume segment.
In a three dimensional space, this orientation describes a ``sidedness'' of the volume, perhaps represented with an outwards or inwards facing normal, or with an oriented cyclic direction on the surface.
In greater than three dimensions, a trivector can have a ``direction'' in the higher dimensional space, as well as a sidedness.
As was the case with the bivector, because not all the products \( \Be_i \Be_j \Be_k \) for any set of indexes \( i, j, k \) are independent, it is possible to form a trivector as a sum over a more restricted set, such as \( \sum_{1 \le i < j < k \le N} T_{ijk} \Be_i \Be_j \Be_k \).
In particular, in three dimensions, all trivectors can be expressed as scalar multiples of \( \Be_1 \Be_2 \Be_3 \).

\index{k-vector}
\index{grade}
\makedefinition{K-vector and grade.}{dfn:multivector:100}{
A k-vector is a sum of products of \( k \) mutually normal vectors.
Given an \( N \) dimensional vector space with an orthonormal basis \( \setlr{ \Be_1, \Be_2, \cdots } \),
a general k-vector can be expressed as
\begin{equation*}
\sum_{1 \le i_1 \ne i_2 \cdots \ne i_k \le N} K_{i_1 i_2 \cdots i_k} \Be_{i_1} \Be_{i_2} \cdots \Be_{i_k},
\end{equation*}
where \( K_{i_1 i_2 \cdots i_k} \) is a scalar.

The number \( k \) of normal vectors that generate a k-vector is called the grade.

A 1-vector is defined as a vector, and a 0-vector is defined as a scalar.
} % definition

We will see that the highest grade for a k-vector in an N dimensional vector space is \( N \).

\index{multivector}
\index{multivector space}
\makedefinition{Multivector space.}{def:multiplication:multivectorspace}{
   Given an N dimensional (generating) vector space \( V \) with an orthonormal basis \( \setlr{ \Be_1, \Be_2, \cdots, \Be_N } \),
%a basis \( \setlr{ \Bx_1, \Bx_2, \cdots } \), 
and a vector multiplication operation represented by juxtaposition,
a multivector is a sum of k-vectors, \( k \in [ 1, N ] \), such as
   \( a_0 + \sum_i a_i \Be_i + \sum_{i \ne j} a_{ij} \Be_i \Be_j + \sum_{i \ne j \ne k} a_{ijk} \Be_i \Be_j \Be_k + \cdots \), where \( a_0, a_i, a_{ij}, \cdots \) are scalars.

A multivector space is a set \( M = \setlr{ x, y, z, \cdots } \) of multivectors, where the following axioms are satisfied

\begin{tcolorbox}[tab2,tabularx={X|Y},title=Multivector space axioms.,boxrule=0.5pt]
    Contraction. & \( \Bx^2 = \Bx \cdot \Bx, \,\forall \Bx \in V \) \\ \hline
    Addition is closed. & \( x + y \in M \) \\ \hline
    Multiplication is closed. & \( x y \in M \) \\ \hline
    Addition is associative. & \( (x + y) + z = x + (y + z) \) \\ \hline
    Addition is commutative. & \( y + x = x + y \) \\ \hline
    There exists a zero element \( 0 \in M \).  & \( x + 0 = x \) \\ \hline
    There exists a negative additive inverse \( -x \in M \). & \( x + (-x) = 0 \) \\ \hline
    Multiplication is distributive.  & \( x( y + z ) = x y + x z \), \( (x + y)z = x z + y z \) \\ \hline
    Multiplication is associative. & \( (x y) z = x ( y z ) \) \\ \hline
    There exists a multiplicative identity \( 1 \). & \( 1 x = x \) \\ \hline
\end{tcolorbox}
}

Compared to the vector space, def'n. \ref{def:prerequisites:vectorspace}, the multivector space

\begin{itemize}
\item presumes a vector multiplication operation, which is not assumed to be commutative (order matters),
\item generalizes vector addition to multivector addition,
\item generalizes scalar multiplication to multivector multiplication (of which scalar multiplication and vector multiplication are special cases),
\item and most importantly, specifies a rule providing the meaning of a squared vector (the contraction axiom).
\end{itemize}

The contraction axiom is arguably the most important of the multivector space axioms, as it allows for multiplicative closure without an infinite dimensional multivector space.
The remaining set of non-contraction axioms of a multivector space are almost that of a field
\footnote{A mathematician would call a multivector space a non-commutative ring with identity \citep{van1943modern}, and could state the multivector space definition much more compactly without listing all the properties of a ring explicitly as done above.}
(as encountered in the study of complex inner products),
as they describe most of the properties one
would expect of a ``well behaved'' set of number-like quantities.
However, a field also requires a multiplicative inverse element for all elements of the space, which exists for some multivector subspaces, but not in general.

%These axioms may seem simple enough, especially since they are not that different from the familiar axioms of the vector space,
%but it will take considerable work to extract all their consequences.
%The subject of Geometric Algebra can be viewed as the study of the impliciations of the axioms
%of the multivector space.

%}


%}
\EndArticle
%\EndNoBibArticle
