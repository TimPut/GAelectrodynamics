%
% Copyright © 2016 Peeter Joot.  All Rights Reserved.
% Licenced as described in the file LICENSE under the root directory of this GIT repository.
%
\makeproblem{\R{3} pseudoscalar commutation.}{problem:gradeselection:R3PseudoscalarCommutation}{
Show that \( I \) given by \cref{eqn:definitions:340}
commutes with any grade \R{3} multivector.
} % problem

\makeanswer{problem:gradeselection:R3PseudoscalarCommutation}{

Showing that \( I \) commutes with each of the basis vectors is sufficient

\begin{dmath}\label{eqn:gradeselectionProblems:620}
\Be_1 I
=
\Be_1 (\Be_1 \Be_2 \Be_3)
=
\Be_1 (-\Be_2 \Be_1) \Be_3
=
-\Be_1 \Be_2 (-\Be_3 \Be_1)
=
I \Be_1
\end{dmath}
\begin{dmath}\label{eqn:gradeselectionProblems:640}
\Be_2 I
=
\Be_2 (\Be_1 \Be_2 \Be_3)
=
\Be_2 \Be_1 (-\Be_3 \Be_2)
=
-(-\Be_1 \Be_2) \Be_3 \Be_2
=
I \Be_2.
\end{dmath}
\begin{dmath}\label{eqn:gradeselectionProblems:660}
\Be_3 I
=
\Be_3 (\Be_1 \Be_2 \Be_3)
=
(\Be_3 \Be_1 \Be_2) \Be_3
=
-(\Be_1 \Be_3) \Be_2 \Be_3
=
-\Be_1 (-\Be_2 \Be_3) \Be_3
=
I \Be_3. \qedmarker
\end{dmath}
} % answer
