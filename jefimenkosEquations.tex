%
% Copyright � 2018 Peeter Joot.  All Rights Reserved.
% Licenced as described in the file LICENSE under the root directory of this GIT repository.
%
%{
%%%\input{../latex/blogpost.tex}
%%%\renewcommand{\basename}{jefimenkosEquations}
%%%%\renewcommand{\dirname}{notes/phy1520/}
%%%\renewcommand{\dirname}{notes/ece1228-electromagnetic-theory/}
%%%%\newcommand{\dateintitle}{}
%%%%\newcommand{\keywords}{}
%%%
%%%\input{../latex/peeter_prologue_print2.tex}
%%%
%%%\usepackage{peeters_layout_exercise}
%%%\usepackage{peeters_braket}
%%%\usepackage{peeters_figures}
%%%\usepackage{siunitx}
%%%%\usepackage{mhchem} % \ce{}
%%%%\usepackage{macros_bm} % \bcM
%%%%\usepackage{macros_qed} % \qedmarker
%%%%\usepackage{txfonts} % \ointclockwise
%%%
%%%\newcommand{\dotBJ}[0]{\mathbf{\dot{J}}}
%%%
%%%\beginArtNoToc
%%%
%%%\generatetitle{Inverting Maxwell's equation}
%%%%\chapter{Inverting Maxwell's equation}
\label{chap:jefimenkosEquations}

Maxwell's equation (\cref{dfn:isotropicMaxwells:680}) is invertable using the Green's function for the spacetime gradient \cref{thm:greensFunctionSpacetimeGradient:120}.  That solution is

\maketheorem{Jefimenkos solution.}{thm:jefimenkosEquations:120}{
The solution of Maxwell's equation is given by
\begin{equation*}
F(\Bx, t)
=
F_0(\Bx, t)
+
\inv{4 \pi}
\int dV'
\lr{
   \frac{\rcap}{r^2} J(\Bx', t_r)
   +
   \inv{c r} \lr{ 1 + \rcap } \dispdot{J}(\Bx', t_r)
},
\end{equation*}
where \( F_0(\Bx, t) \) is any specific solution of the homogeneous equation \( \lr{ \spacegrad + (1/c) \partial_t } F_0 = 0 \),
time derivatives are denoted by overdots, and all times are evaluated at the retarded time \( t_r = t - r/c \).
When expanded in terms of the electric and magnetic fields (ignoring magnetic sources), the non-homogeneous portion of this solution is known as the
Jefimenkos' equations \citep{griffiths1999introduction}.
\begin{dmath}\label{eqn:jefimenkosEquations:100}
\begin{aligned}
\BE &=
\inv{4 \pi}
\int dV'
\lr{
\frac{\rcap}{\epsilon r} \lr{
\frac{\rho(\Bx', t_r)}{r} + \frac{\dispdot{\rho}(\Bx', t_r) }{c} }
   - \frac{\eta }{ c r } \dotBJ(\Bx', t_r)
} \\
\BH &=
\inv{4 \pi}
\int dV'
\lr{
   \frac{1}{c r} \dotBJ(\Bx', t_r)
+
   \frac{1}{r^2} \BJ(\Bx', t_r)
} \cross \rcap,
\end{aligned}
\end{dmath}
%which checks against Griffiths.
} % theorem

The full solution is
\begin{dmath}\label{eqn:jefimenkosEquations:20}
F(\Bx, t)
= F_0(\Bx, t)
+ \int dV' dt' G(\Bx - \Bx', t - t') J(\Bx', t')
= F_0(\Bx, t)
+
\inv{4\pi}
\int dV' dt'
\lr{
   \lr{
   - \frac{\rcap}{r^2} \PD{r}{}
   + \frac{\rcap}{r}
   + \inv{c r} \PD{t}{}
   }
   \delta( -r/c + t - t' )
}
J(\Bx', t')
\end{dmath}
where \( \Br = \Bx - \Bx', r = \Norm{\Br} \) and \( \rcap = \Br/r \).
With the help of \cref{eqn:derivativeOfDeltaFunction:140}, the derivatives in the Green's function \cref{eqn:jefimenkosEquations:20} can be evaluated, and
the convolution reduces to
\begin{dmath}\label{eqn:jefimenkosEquations:40}
\int dt' G(\Bx - \Bx', t - t') J(\Bx', t')
=
\inv{4 \pi}
\evalbar{
\lr{
\frac{\rcap}{r^2} J(\Bx', t_r)
-
\frac{\rcap}{r} \lr{ -\inv{c} } \frac{d}{dt_r} J(\Bx', t_r)
+
\inv{c r} \frac{d}{dt_r} J(\Bx', t_r)
}
}{t_r = t - r/c}.
\end{dmath}

There have been lots of opportunities to mess up with signs and factors of \( c \), so let's expand this out explicitly for a non-magnetic current source \( J = \rho/\epsilon - \eta \BJ \).
%, and check the results against Jefimenko's equations found in
%\citep{griffiths1999introduction}.
Neglect the contribution of the homogeneous solution \( F_0 \), and utilizing our freedom to
insert a no-op grade 1,2 selection operation, that
removes any scalar and pseudoscalar terms that are necessarily killed over the full integration range, we find
\begin{dmath}\label{eqn:jefimenkosEquations:80}
F =
\inv{4 \pi}
\int dV'
\gpgrade{
   \frac{\rcap}{r^2}
\lr{ \frac{\rho}{\epsilon} - \eta \BJ }
   +
   \inv{c r} \lr{ 1 + \rcap } \lr{ \frac{\dispdot{\rho}}{\epsilon} - \eta \dotBJ }
}{1,2}
=
\inv{4 \pi}
\int dV'
\lr{
   \frac{\rcap}{\epsilon r^2} \rho
   - \eta \frac{\rcap}{r^2} \wedge \BJ
   - \frac{\eta}{ c r } \dotBJ
   + \inv{c r} \rcap \frac{\dispdot{\rho}}{\epsilon}
   - \frac{\eta}{c r} \rcap \wedge \dotBJ
}
=
\inv{4 \pi}
\int dV'
\lr{
   \frac{\rcap}{\epsilon r^2} \rho
   + \frac{\rcap \dispdot{\rho}}{\epsilon c r}
   - \frac{\eta \dotBJ}{ c r }
   - I \frac{\eta }{c r} \rcap \cross \dotBJ
   - I \frac{\eta}{r^2} \rcap \cross \BJ
}.
\end{dmath}

As \( F = \BE + I \eta \BH \), the respective electric and magnetic fields by inspection.  After re-inserting the space and time parameters that we suppressed temporarily, the proof is complete.

The disadvantages of separating the field and current components into their constituent components is also made painfully obvious by the complexity of the conventional statement of the solution compared to the equivalent multivector form.

%}
%%%\EndArticle
