%
% Copyright � 2018 Peeter Joot.  All Rights Reserved.
% Licenced as described in the file LICENSE under the root directory of this GIT repository.
%
%{
%%%\input{../latex/blogpost.tex}
%%%\renewcommand{\basename}{jefimenkosEquations}
%%%%\renewcommand{\dirname}{notes/phy1520/}
%%%\renewcommand{\dirname}{notes/ece1228-electromagnetic-theory/}
%%%%\newcommand{\dateintitle}{}
%%%%\newcommand{\keywords}{}
%%%
%%%\input{../latex/peeter_prologue_print2.tex}
%%%
%%%\usepackage{peeters_layout_exercise}
%%%\usepackage{peeters_braket}
%%%\usepackage{peeters_figures}
%%%\usepackage{siunitx}
%%%%\usepackage{mhchem} % \ce{}
%%%%\usepackage{macros_bm} % \bcM
%%%%\usepackage{macros_qed} % \qedmarker
%%%%\usepackage{txfonts} % \ointclockwise
%%%
%%%\newcommand{\dotBJ}[0]{\mathbf{\dot{J}}}
%%%
%%%\beginArtNoToc
%%%
%%%\generatetitle{Inverting Maxwell's equation}
%%%%\chapter{Inverting Maxwell's equation}
\label{chap:jefimenkosEquations}

Maxwell's equation (\cref{dfn:isotropicMaxwells:680}) is invertable using the Green's function for the spacetime gradient \cref{thm:greensFunctionSpacetimeGradient:120}.

The full solution is
\begin{dmath}\label{eqn:jefimenkosEquations:20}
F(\Bx, t)
= F_0(\Bx, t)
+ \int dV' dt' G(\Bx - \Bx', t - t') J(\Bx', t'),
\end{dmath}
where \( F_0(\Bx, t) \) is any specific solution of the homogenoous equation \( \lr{ \spacegrad + (1/c) \partial_t } F_0 = 0 \).  With the help of \cref{eqn:derivativeOfDeltaFunction:140} we find for the time integral of \cref{eqn:jefimenkosEquations:20}
\begin{dmath}\label{eqn:jefimenkosEquations:40}
\int dt' G(\Bx - \Bx', t - t') J(\Bx', t')
=
\inv{4 \pi}
\evalbar{
\lr{
\frac{\rcap}{r^2} J(\Bx', t_r)
-
\frac{\rcap}{r} \lr{ -\inv{c} } \frac{d}{dt_r} J(\Bx', t_r)
+
\inv{c r} \frac{d}{dt_r} J(\Bx', t_r)
}
}{t_r = t - r/c}.
\end{dmath}

Denoting the time derivatives with overdots, and evaluating all terms at the retarded time \( t_r = t - r/c \), the full solution of Maxwell's equation is
given by
\boxedEquation{eqn:jefimenkosEquations:60}{
F(\Bx, t)
=
F_0(\Bx, t)
+
\inv{4 \pi}
\int dV'
\lr{
   \frac{\rcap}{r^2} J(\Bx', t_r)
   +
   \inv{c r} \lr{ 1 + \rcap } \dot{J}(\Bx', t_r)
}.
}

There have been lots of opportuntites to mess up with signs and factors of \( c \), so let's expand this out explicitly for a non-magnetic current source \( J = \rho/\epsilon - \eta \BJ \), and check the results against Jefimenko's equations found in \citep{griffiths1999introduction}.
Let's neglect the contribution of the homogeneous solution \( F_0 \), and also utilize our freedom to
insert a no-op grade 1,2 selection operation.
Such a grade selection removes any scalar and pseudoscalar terms that are neccessarily killed over the full integration range, giving
\begin{dmath}\label{eqn:jefimenkosEquations:80}
F =
\inv{4 \pi}
\int dV'
\gpgrade{
   \frac{\rcap}{r^2}
\lr{ \frac{\rho}{\epsilon} - \eta \BJ }
   +
   \inv{c r} \lr{ 1 + \rcap } \lr{ \frac{\dot{\rho}}{\epsilon} - \eta \dotBJ }
}{1,2}
=
\inv{4 \pi}
\int dV'
\lr{
   \frac{\rcap}{\epsilon r^2} \rho
   - \eta \frac{\rcap}{r^2} \wedge \BJ
   - \frac{\eta}{ c r } \dotBJ
   + \inv{c r} \rcap \frac{\dot{\rho}}{\epsilon}
   - \frac{\eta}{c r} \rcap \wedge \dotBJ
}
=
\inv{4 \pi}
\int dV'
\lr{
   \frac{\rcap}{\epsilon r^2} \rho
   + \frac{\rcap \dot{\rho}}{\epsilon c r}
   - \frac{\eta \dotBJ}{ c r }
   - I \frac{\eta }{c r} \rcap \cross \dotBJ
   - I \frac{\eta}{r^2} \rcap \cross \BJ
}.
\end{dmath}

As \( F = \BE + I \eta \BH \), we can read of the respective electric and magnetic fields by inspection.  Inserting the space and time parameters that we suppressed temporarily, we find that Jefimenko's equations for the electric and magnetic fields are
\begin{dmath}\label{eqn:jefimenkosEquations:100}
\begin{aligned}
\BE &=
\inv{4 \pi}
\int dV'
\lr{
\frac{\rcap}{\epsilon r} \lr{
\frac{\rho(\Bx', t_r)}{r} + \frac{\dot{\rho}(\Bx', t_r) }{c} }
   - \frac{\eta }{ c r } \dotBJ(\Bx', t_r)
} \\
\BH &=
\inv{4 \pi}
\int dV'
\lr{
   \frac{1}{c r} \dotBJ(\Bx', t_r)
+
   \frac{1}{r^2} \BJ(\Bx', t_r)
} \cross \rcap,
\end{aligned}
\end{dmath}
which checks against Griffiths.  The disadvantages of separating the field and current components into their constituent components is also made painfully obvious by the complexity of \cref{eqn:jefimenkosEquations:100} compared to the equivalent multivector form \cref{eqn:jefimenkosEquations:60}.

%}
%%%\EndArticle
