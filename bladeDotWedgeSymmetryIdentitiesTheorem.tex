%
% Copyright © 2016 Peeter Joot.  All Rights Reserved.
% Licenced as described in the file LICENSE under the root directory of this GIT repository.
%

\maketheorem{Vector Blade dot and wedge product relations.}{thm:bladeDotWedgeSymmetryIdentities:180}{
Given a k-blade \( B \) and a vector \( \Ba \), the dot and wedge products have the following commutation relationships
\boxedEquation{eqn:bladeDotWedgeSymmetryIdentities:200}{
\begin{aligned}
B \cdot \Ba  &= (-1)^{k-1} \Ba \cdot B \\
B \wedge \Ba &= (-1)^k \Ba \wedge B,
\end{aligned}
}
and can be expressed as symmetric and antisymmetric sums depending on the grade of the blade
\boxedEquation{eqn:bladeDotWedgeSymmetryIdentities:220}{
\begin{aligned}
\Ba \wedge B &= \inv{2}\lr{ \Ba B + (-1)^k B \Ba } \\
\Ba \cdot B &= \inv{2}\lr{ \Ba B - (-1)^k B \Ba }.
\end{aligned}
}
} % theorem

To prove these, split the blade into components that intersect with and are disjoint from \( \Ba \) as follows
\begin{dmath}\label{eqn:bladeDotWedgeSymmetryIdentitiesTheorem:240}
B
=
\inv{\Ba} \Bn_1 \Bn_2 \cdots \Bn_{k-1} + \Bm_1 \Bm_2 \cdots \Bm_k,
\end{dmath}
where \( \Bn_i \) orthogonal to \( \Ba \) and each other, and where \( \Bm_i \) are all orthogonal.  The products of \( B \) with \( \Ba \) are
\begin{dmath}\label{eqn:bladeDotWedgeSymmetryIdentitiesTheorem:340}
\Ba B
=
\Ba \inv{\Ba} \Bn_1 \Bn_2 \cdots \Bn_{k-1} + \Ba \Bm_1 \Bm_2 \cdots \Bm_k
=
\Bn_1 \Bn_2 \cdots \Bn_{k-1} + \Ba \Bm_1 \Bm_2 \cdots \Bm_k,
\end{dmath}
and
\begin{dmath}\label{eqn:bladeDotWedgeSymmetryIdentitiesTheorem:360}
B \Ba
=
\inv{\Ba} \Bn_1 \Bn_2 \cdots \Bn_{k-1} \Ba + \Bm_1 \Bm_2 \cdots \Bm_k \Ba
=
(-1)^{k-1} \Bn_1 \Bn_2 \cdots \Bn_{k-1} + (-1)^k \Ba \Bm_1 \Bm_2 \cdots \Bm_k
=
(-1)^k \lr{ - \Bn_1 \Bn_2 \cdots \Bn_{k-1} + \Ba \Bm_1 \Bm_2 \cdots \Bm_k },
\end{dmath}
or
\begin{dmath}\label{eqn:bladeDotWedgeSymmetryIdentitiesTheorem:380}
(-1)^k B \Ba
=
- \Bn_1 \Bn_2 \cdots \Bn_{k-1} + \Ba \Bm_1 \Bm_2 \cdots \Bm_k.
\end{dmath}

Respective addition and subtraction of \cref{eqn:bladeDotWedgeSymmetryIdentitiesTheorem:340} and \cref{eqn:bladeDotWedgeSymmetryIdentitiesTheorem:380} gives
\begin{dmath}\label{eqn:bladeDotWedgeSymmetryIdentitiesTheorem:400}
\Ba B + (-1)^k B \Ba
= 2 \Ba \Bm_1 \Bm_2 \cdots \Bm_k
= 2 \Ba \wedge B,
\end{dmath}
and
\begin{dmath}\label{eqn:bladeDotWedgeSymmetryIdentitiesTheorem:420}
\Ba B - (-1)^k B \Ba
=
2
\Bn_1 \Bn_2 \cdots \Bn_{k-1}
= 2 \Ba \cdot B,
\end{dmath}
proving \cref{eqn:bladeDotWedgeSymmetryIdentities:220}.  Grade selection from \cref{eqn:bladeDotWedgeSymmetryIdentitiesTheorem:380} gives
\begin{dmath}\label{eqn:bladeDotWedgeSymmetryIdentitiesTheorem:440}
(-1)^k B \cdot \Ba
=
- \Bn_1 \Bn_2 \cdots \Bn_{k-1}
= - \Ba \cdot B,
\end{dmath}
and
\begin{dmath}\label{eqn:bladeDotWedgeSymmetryIdentitiesTheorem:460}
(-1)^k B \wedge \Ba
=
\Ba \Bm_1 \Bm_2 \cdots \Bm_k
= \Ba \wedge B,
\end{dmath}
which proves \cref{eqn:bladeDotWedgeSymmetryIdentities:200}.

