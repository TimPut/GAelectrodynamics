
Unless otherwise stated, a Euclidean vector space with an orthonormal basis \( \setlr{\Be_1, \Be_2, \cdots } \) is assumed for the remainder of this chapter.
Generalizations required for non-Euclidean spaces will be discussed when spacetime vectors are introduced.
%  At that point, it is a good exersize for the reader to come back to this, and determine where any result

\subsection{Colinear vectors.}

It was pointed out that the vector multiplication operation was not assumed to be commutative (order matters).  The only condition for which the product of two vectors is order independent, is when those vectors are colinear.

\maketheorem{Commutation}{thm:multiplication:commutation}{
Let \(\Bu\), and \(\Bv\) be two non-zero colinear vectors.  The product of these vectors commute (is unchanged by interchange).

\begin{equation*}
\Bu \Bv = \Bv \Bu.
\end{equation*}
} % theorem

The proof is simple.  Because these vectors are colinear there exists some \( \alpha \) for which \( \Bv = \alpha \Bu \), so

\begin{dmath}\label{eqn:multivector:380}
\Bv \Bu
=
(\alpha \Bu) \Bu
=
\alpha \Bu \Bu
=
\Bu \alpha \Bu
=
\Bu (\alpha \Bu)
=
\Bu \Bv.
\end{dmath}

Also observe that, because of the contraction axiom, the product of two colinear vectors is a scalar.  In particular, the square of a unit vector is unity, something that should be highlighted explicitly, since this property will be used again and again
%\begin{equation}\label{eqn:multiplication:300}
\boxedEquation{eqn:multiplication:320}{
\xcap^2 = 1.
}
%\end{equation}

For example, the squares of any orthonormal basis vectors are unity \( \Be_1^2 = \Be_2^2 = \Be^3 = 1 \).

A corollory of this is that any scalar can also be factored into a product of two colinear vectors.
In particular, the square \( 1 \) can be factored into the square of any unit vector

\boxedEquation{eqn:multiplication:400}{
1 = \xcap \xcap.
}

This has been highlighted explicitly, because this factorization trick will be used repeatedly.

\subsection{Normal vectors.}

The simplest bivectors are products of two different orthonormal vectors.  For \R{3} all such bivectors that can be formed from the basis elements are \( \Be_1 \Be_2, \Be_2 \Be_1, \Be_2 \Be_3, \Be_3 \Be_2, \Be_3 \Be_1, \Be_1 \Be_3 \).

These are not all independent, which an be demonstrated simply by consider the square of the vector \( \Be_1 + \Be_2 \),
as sketched in \cref{fig:unitSum:unitSumFig1}.
\imageFigure{../figures/GAelectrodynamics/unitSumFig1}{\( \Be_1 + \Be_2 \).}{fig:unitSum:unitSumFig1}{0.3}
By the contraction axiom, the square of this vector is \( 2 \)

\begin{dmath}\label{eqn:multivector:420}
\lr{ \Be_1 + \Be_2 }^2 =
\lr{ \Be_1 + \Be_2 } \cdot
\lr{ \Be_1 + \Be_2 }
=
\Be_1 \cdot \Be_1
+
\cancel{\Be_1 \cdot \Be_2}
+
\cancel{\Be_2 \cdot \Be_1}
+
\Be_2 \cdot \Be_2
=
2.
\end{dmath}

Computing this same square by expansion in vector products must give the same result, but for that we get

\begin{dmath}\label{eqn:gaTutorial:80}
(\Be_1 + \Be_2)^2
= (\Be_1 + \Be_2)(\Be_1 + \Be_2)
= \Be_1^2 + \Be_2 \Be_1 + \Be_1 \Be_2 + \Be_2^2
= 1 + \Be_2 \Be_1 + \Be_1 \Be_2 + 1
= 2 + \Be_2 \Be_1 + \Be_1 \Be_2.
\end{dmath}

The right hand side is a mixed grade multivector with grades zero and two, however, it must also equal \( 2 \).
The only possible solution requires that the sum of all the grade two components of this equation are zero

\begin{dmath}\label{eqn:multivector:280}
\Be_2 \Be_1 + \Be_1 \Be_2 = 0,
\end{dmath}

or
%\begin{dmath}\label{eqn:multiplication:140}
\boxedEquation{eqn:multiplication:140}{
\Be_1 \Be_2 = -\Be_1 \Be_2.
}
%\end{dmath}

The same computation could have been performed for any two orthonormal vectors, so we conclude that any interchange of two orthonormal vectors changes the sign.  In general this is true of any normal vectors.

\maketheorem{Anticommutation}{thm:multiplication:anticommutationNormal}{
Let \(\Bu\), and \(\Bv\) be two normal vectors, the product of which \( \Bu \Bv \) is a bivector.
Changing the order of these products toggles the sign of the bivector.

\begin{equation*}
\Bu \Bv = -\Bv \Bu.
\end{equation*}

This sign change on interchange is called anticommutation.
} % theorem

