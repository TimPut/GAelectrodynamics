%
% Copyright © 2017 Peeter Joot.  All Rights Reserved.
% Licenced as described in the file LICENSE under the root directory of this GIT repository.
%
%{
\subsection{Inverting the Maxwell statics equation.}

Similar to electrostatics and magnetostatics, we can restrict attention to time invariant fields (\( \partial_t F = 0\)) and time invariant sources (\(\partial_t J = 0\)), but consider both electric and magnetic sources.  In that case Maxwell's equation is reduced to a first order gradient equation

\begin{dmath}\label{eqn:statics:20}
\spacegrad F(\Bx) = J(\Bx)
\end{dmath}

Using the Green's function for the (first order) gradient \cref{eqn:electrostatics:260}, this can be inverted as

\begin{dmath}\label{eqn:statics:40}
F(\Bx)
= \int_V dV' G(\Bx, \Bx') \spacegrad' J(\Bx')
= \gpgrade{\int_V dV' G(\Bx, \Bx') \spacegrad' J(\Bx')}{1,2}
= \inv{4\pi} \int_V dV' \gpgrade{\frac{(\Bx - \Bx') J(\Bx')}{\Norm{\Bx - \Bx'}^3} }{1,2}.
\end{dmath}

Like we did in magnetostatics, a no-op grade selection has been inserted to simplify subsequent manipulation
\footnote{If this grade selection filter is omitted, it is possible to show that the scalar and pseudoscalar contributions to the \( (\Bx -\Bx') J \) product are zero on the boundary of the Green's integration volume. \citep{jancewicz1988multivectors:appendixI}},
yielding a compact solution to the
Maxwell statics equation

\boxedEquation{eqn:statics:80}{
F(\Bx)
= \inv{4\pi} \int_V dV' \frac{\gpgrade{(\Bx - \Bx') J(\Bx')}{1,2}}{\Norm{\Bx - \Bx'}^3}.
}

It will be preferable and simpler to work with combined fields and densities, however, some insight can be gained by
explicit expansion of \cref{eqn:statics:80} in terms of charge and current densities.
The grade selection, in terms of terms of \( \Bs = \Bx -\Bx' \) expands as

\begin{dmath}\label{eqn:statics:60}
\gpgrade{\Bs J}{1,2}
=
\eta \gpgrade{\Bs (v \rho - \BJ)}{1,2}
+
\gpgrade{\Bs I(v \rho_m - \BM)}{1,2}
=
\inv{\epsilon} \Bs \rho + \eta \BJ \wedge \Bs + \Bs v \rho_m I + \Bs \cross \BM,
\end{dmath}

so the solution \( F = \BE + \eta I \BH \) to the Maxwell statics equation in terms of the charge and current densities, and electric and magnetic fields is

\begin{dmath}\label{eqn:statics:100}
\begin{aligned}
\BE(\Bx)
&=
\inv{4\pi} \int_V dV' \inv{\Norm{\Bx - \Bx'}^3}
\lr{
   \inv{\epsilon}(\Bx - \Bx') \rho(\Bx')
   +
   (\Bx - \Bx') \cross \BM(\Bx')
} \\
\BH(\Bx)
&=
\inv{4\pi} \int_V dV' \inv{\Norm{\Bx - \Bx'}^3}
\lr{
  \BJ(\Bx') \cross (\Bx - \Bx')
+ \inv{\mu} (\Bx - \Bx') \rho_m(\Bx')
}.
\end{aligned}
\end{dmath}

We see that the combined solution \cref{eqn:statics:80} incorporates both a Coulomb's law contribution and a Biot-Savart law contribution.
When desired, any fictitious magnetic sources, also contribute to the electric and magnetic fields.

\makeproblem{}{problem:statics:81}{
Fill in any steps left out of the derivations of \cref{eqn:statics:60} and \cref{eqn:statics:100}.
} % problem

\subsection{Example.  Infinite line charge and current.}

Given a static line charge density and current density along the z-axis in free space

\begin{dmath}\label{eqn:statics:120}
\begin{aligned}
\rho(\Bx) &= \lambda \delta(x) \delta(y) \\
\BJ(\Bx) &= \Bv \rho(\Bx) = v \lambda \Be_3 \delta(x) \delta(y),
\end{aligned}
\end{dmath}

the total multivector current is
\begin{dmath}\label{eqn:statics:140}
J
= \eta_0 ( c \rho - \BJ )
= \eta_0 ( c - v \Be_3 ) \lambda \delta(x) \delta(y)
= \frac{\lambda}{\epsilon_0} \lr{ 1 - \frac{v}{c} \Be_3 } \delta(x) \delta(y)
\end{dmath}

We can find the field for this current by substitution into \cref{eqn:statics:80}.
To do so, let the field observation point be \( \Bx = \Bx_\perp + z \Be_3 \), so the total field is
\begin{dmath}\label{eqn:statics:160}
F(\Bx)
= \frac{\lambda}{4\pi \epsilon_0} \int_V dx'dy'dz' \frac{\gpgrade{(\Bx - \Bx') (1 - (v/c) \Be_3 )}{1,2}}{\Norm{\Bx - \Bx'}^3} \delta(x') \delta(y')
= \frac{\lambda}{4\pi \epsilon_0} \int_{-\infty}^\infty dz' \frac{\gpgrade{(\Bx_\perp + (z - z') \Be_3) (1 - (v/c) \Be_3 )}{1,2}}{\lr{\Bx_\perp^2 + (z-z')^2}^{3/2}}
=
\frac{\lambda \lr{ \Bx_\perp - (v/c) \Bx_\perp \Be_3}}{4\pi \epsilon_0} \int_{-\infty}^\infty \frac{dz'}{\lr{\Bx_\perp^2 + (z-z')^2}^{3/2}}
+
\frac{\lambda \Be_3}{4\pi \epsilon_0} \int_{-\infty}^\infty \frac{(z - z') dz'}{\lr{\Bx_\perp^2 + (z-z')^2}^{3/2}}.
\end{dmath}

The first integral is \( 2/\Bx_\perp^2 \), whereas the second is zero (odd function, over even interval).
Using cylindrical coordinates \( \Bx = R \rhocap + z \Be_3 \), and since
\( \Bx_\perp \cdot \Be_3 = 0 \), the wedge can be replaced with the vector product \( \Bx_\perp \Be_3 = R \rhocap \Be_3 \), leaving

\begin{equation}\label{eqn:statics:180}
F(\Bx)
=
\frac{\lambda}{2\pi \epsilon_0 R} \rhocap \lr{ 1 - \Bv/c} = \BE \lr{ 1 - \Bv/c }
= \BE + I \lr{ \frac{\Bv}{c} \cross \BE },
\end{equation}

where \( \Bv = v \Be_3 \).
The vector component of this is the electric field, which is therefore directed radially, whereas the (dual) magnetic field \( \eta_0 I \BH \)
is a set of oriented planes spanning the radial and z-axis directions.
We can also see that there is a constant proportionality factor that relates the electric and magnetic field components, namely

\begin{dmath}\label{eqn:statics:200}
I \eta_0 \BH = -\BE \Bv/c,
\end{dmath}

or

\begin{dmath}\label{eqn:statics:220}
\BH = \Bv \cross \BD.
\end{dmath}

\makeproblem{Statics solution for magnetic density and currents.}{problem:statics:240}{
Given magnetic charge density \( \rho_m = \lambda_m \delta(x) \delta(y) \), and current density \( \BM = v \Be_3 \rho_m = \Bv \rho_m \), show that the field is given by
\begin{equation*}
F(\Bx) = \frac{\lambda_m c}{4 \pi R} I \rhocap \lr{ 1 - \frac{\Bv}{c} },
\end{equation*}
or with \( \BB = \lambda_m \rhocap/(4 \pi R) \),
\begin{equation*}
F = \BB \cross \Bv + c I \BB.
\end{equation*}
} % problem

%}
