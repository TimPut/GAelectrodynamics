%
% Copyright © 2018 Peeter Joot.  All Rights Reserved.
% Licenced as described in the file LICENSE under the root directory of this GIT repository.
%
%{
\chapter{Multivector calculus.}
   \section{Reciprocal frames.}
      \subsection{Motivation and definition.}
         %
% Copyright � 2016 Peeter Joot.  All Rights Reserved.
% Licenced as described in the file LICENSE under the root directory of this GIT repository.
%
%{
%\input{../blogpost.tex}
%\renewcommand{\basename}{reciprocal}
%%\renewcommand{\dirname}{notes/phy1520/}
%\renewcommand{\dirname}{notes/ece1228-electromagnetic-theory/}
%%\newcommand{\dateintitle}{}
%%\newcommand{\keywords}{}
%
%\input{../peeter_prologue_print2.tex}
%
%\usepackage{peeters_layout_exercise}
%\usepackage{peeters_braket}
%\usepackage{peeters_figures}
%\usepackage{siunitx}
%%\usepackage{mhchem} % \ce{}
%%\usepackage{macros_bm} % \bcM
%%\usepackage{macros_qed} % \qedmarker
%%\usepackage{txfonts} % \ointclockwise
%
%\beginArtNoToc
%
%\generatetitle{Reciprocal frame vectors}
%%\chapter{reciprocal frame vectors}
%%\label{chap:reciprocal}
%
The end goal of this chapter is to be able to integrate multivector functions along curves and surfaces, known collectively as manifolds.
For our purposes, a manifold is defined by a parameterization, such as the vector valued function \( \Bx(a,b) \) where \( a, b\) are scalar parameters.  With one parameter the vector traces out a curve, with two a surface, three a volume, and so forth.
The respective partial derivatives of such a parameterized vector define a local basis for the surface at the point at which the partials are evaluated.
The span of such a basis is called the tangent space, and the partials that constitute it are not necessarily orthonormal, or even normal.

Unfortunately, in order to work with the curvilinear non-orthonormal bases that will be encountered in general integration theory, some
additional tools are required.

\begin{itemize}
\item
We introduce a reciprocal frame which partially generalizes the notion of normal to non-orthonormal bases.
\item
We will borrow the upper and lower index (tensor) notation from relativistic physics that is useful for the intrinsically non-orthonormal spaces encountered in that study, as this notation works well to define the reciprocal frame.
\end{itemize}

\index{reciprocal frame}
\makedefinition{Reciprocal frame}{dfn:reciprocal:frame}{
Given a basis for a subspace \( \setlr{ \Bx_1, \Bx_2, \cdots \Bx_n } \), where the vectors \( \Bx_i \) are not necessarily orthonormal, the reciprocal frame is defined as the set of vectors \( \setlr{ \Bx^1, \Bx^2, \cdots \Bx^n } \) satisfying

\begin{dmath*}
\Bx_i \cdot \Bx^j = {\delta_i}^j,
\end{dmath*}

where the vector \( \Bx^j \) is not the j-th power of \( \Bx \), but is a superscript index, the conventional way of denoting a reciprocal frame vector, and \( {\delta_i}^j \) is the Kronecker delta.
} % definition

This definition introduces mixed index variables for the first time in this text, which may be unfamiliar.  These are most often used in tensor algebra, where any expression that has pairs of upper and lower indexes implies a sum, and is called the summation (or Einstein) convention.  For example:

\begin{dmath}\label{eqn:reciprocal:400}
\begin{aligned}
a_i b^i &\equiv \sum_i a_i b^i \\
{A^{i}}_j B_i C^j &\equiv \sum_{i,j} {A^{i}}_j B_i C^j.
\end{aligned}
\end{dmath}

Summation convention will not be used explicitly in this text, as it deviates from normal practises in electrical engineering\footnote{Generally, when summation convention is used, explicit summation is only used explicitly when upper and lower indexes are not perfectly matched, but summation is still implied.  Readers of texts that use summation convention can check for proper matching of upper and lower indexes to ensure that the expressions make sense.  Such matching is the reason a mixed index Kronecker delta has been used in the definition of the reciprocal frame.}.

The most important application of a reciprocal frame is for the computation of the coordinates of a vector with respect to a non-orthonormal frame.
Let a vector \( \Ba \) have coordinates \( a^i \) with respect to a basis \( \setlr{ \Bx_i } \)

\begin{dmath}\label{eqn:reciprocal:20}
\Ba = \sum_j a^j \Bx_j,
\end{dmath}

where \( j \) is an index not a power\footnote{In tensor algebra, any index that is found in matched upper and lower index pairs, is known as a dummy summation index, whereas an index that is unmatched is known as a free index.  For example, in \( a^j b_{ij} \) (summation implied) \( j \) is a summation index, and \( i \) is a free index.  We are free to make a change of variables of any summation index, so for the same example we can write
\( a^k b_{ik} \).  These index tracking conventions are obvious when summation symbols are included explicitly, as we will do.}.

Dotting with the reciprocal frame vectors \( \Bx^i \) provides these coordinates \( a^i \)

\begin{dmath}\label{eqn:reciprocal:40}
\Ba \cdot \Bx^i
= \lr{\sum_j a^j \Bx_j} \cdot \Bx^i
= \sum_j a^j {\delta_j}^i
= a^i.
\end{dmath}

The vector can also be expressed with coordinates taken with respect to the reciprocal frame.  Let those coordinates be \( a_i \), so that

\begin{dmath}\label{eqn:reciprocal:60}
\Ba = \sum_i a_i \Bx^i.
\end{dmath}

Dotting with the basis vectors \( \Bx_i \) provides the reciprocal frame relative coordinates \( a_i \)

\begin{dmath}\label{eqn:reciprocal:80}
\Ba \cdot \Bx_i
= \lr{\sum_j a_j \Bx^j} \cdot \Bx_i
= \sum_j a_j {\delta^j}_i
= a_i.
\end{dmath}

We can summarize \cref{eqn:reciprocal:40} and \cref{eqn:reciprocal:80} by stating that a vector can be expressed in terms of coordinates relative to either the original or reciprocal basis as follows

\begin{equation}\label{eqn:reciprocal:420}
\Ba
= \sum_j \lr{ \Ba \cdot \Bx^j } \Bx_j
= \sum_j \lr{ \Ba \cdot \Bx_j } \Bx^j.
\end{equation}

In tensor algebra the basis is generally implied\footnote{
In tensor algebra, a vector, identified by the coordinates \( a^i \) is called a contravariant vector.
When that vector is identified by the coordinates \( a_i \) it is called a covariant vector.  These labels relate to how the coordinates transform with respect to norm preserving transformations.
We have no need of this nomenclature, since we never transform coordinates in isolation, but will always transform the coordinates along with their associated basis vectors.}.

%When doing tensor algebra manipulations, you'll generally have the freedom to swap any pairs of upper and lower indexes as done above.

An example of a 2D oblique Euclidean basis and a corresponding reciprocal basis is plotted in \cref{fig:obliqueReciprocal:obliqueReciprocalFig2}.
Also plotted are the superposition of the projections required to arrive at a given point \( (4,2) \)) along the \( \Be_1, \Be_2 \) directions or the \( \Be^1, \Be^2 \) directions.
In this plot, neither of the reciprocal frame vectors \( \Be^i \) are normal to the corresponding basis vectors \( \Be_i \).
When one of \( \Be_i \) is increased(decreased) in magnitude, there will be a corresponding decrease(increase) in the magnitude of \( \Be^i \), but if the orientation is remained fixed, the corresponding direction of the reciprocal frame vector stays the same.

\imageFigure{../figures/GAelectrodynamics/obliqueReciprocalFig2}{Oblique and reciprocal bases.}{fig:obliqueReciprocal:obliqueReciprocalFig2}{0.5}

How are the reciprocal frame vectors computed?  While these vectors have a natural GA representation, this is not intrinsically a GA problem, and can be solved with standard linear algebra, using a matrix inversion.
For example, given a 2D basis \( \setlr{ \Bx_1, \Bx_2 } \), the reciprocal basis can be assumed to have a coordinate representation in the original basis

\begin{dmath}\label{eqn:reciprocal:100}
\begin{aligned}
\Bx^1 &= a \Bx_1 + b \Bx_2 \\
\Bx^2 &= c \Bx_1 + d \Bx_2.
\end{aligned}
\end{dmath}

Imposing the constraints of \cref{dfn:reciprocal:frame} leads to a pair of 2x2 linear systems that are easily solved to find
\begin{dmath}\label{eqn:reciprocal:120}
\begin{aligned}
\Bx^1 &= \inv{ (\Bx_1)^2 (\Bx_2)^2 - \lr{ \Bx_1 \cdot \Bx_2}^2 } \lr{ (\Bx_2)^2 \Bx_1 - \lr{ \Bx_1 \cdot \Bx_2 } \Bx_2 } \\
\Bx^2 &= \inv{ (\Bx_1)^2 (\Bx_2)^2 - \lr{ \Bx_1 \cdot \Bx_2}^2 } \lr{ (\Bx_1)^2 \Bx_2 - \lr{ \Bx_1 \cdot \Bx_2 } \Bx_1 } \\
\end{aligned}
\end{dmath}

The reader may notice that for \R{3} the denominator is related to the norm of the cross product \( \Bx_1 \cross \Bx_2 \).
More generally, this can be expressed as the square of the bivector \( \Bx_1 \wedge \Bx_2 \)

\begin{dmath}\label{eqn:reciprocal:140}
-\lr{\Bx_1 \wedge \Bx_2 }^2
=
-\lr{\Bx_1 \wedge \Bx_2 } \cdot \lr{\Bx_1 \wedge \Bx_2 }
=
-\lr{ \lr{\Bx_1 \wedge \Bx_2 } \cdot \Bx_1 } \cdot \Bx_2
=
(\Bx_1)^2 (\Bx_2)^2 - \lr{\Bx_1 \cdot \Bx_2}^2.
\end{dmath}

Additionally, the numerators are each dot products of \( \Bx_1, \Bx_2 \) with that same bivector

\begin{dmath}\label{eqn:reciprocal:160}
\begin{aligned}
\Bx^1 &= \frac{\Bx_2 \cdot \lr{ \Bx_1 \wedge \Bx_2 } }{ \lr{ \Bx_1 \wedge \Bx_2}^2 } \\
\Bx^2 &= \frac{\Bx_1 \cdot \lr{ \Bx_2 \wedge \Bx_1 } }{ \lr{ \Bx_1 \wedge \Bx_2}^2 },
\end{aligned}
\end{dmath}

or

%\begin{dmath}\label{eqn:reciprocal:180}
\boxedEquation{eqn:reciprocal:180}{
\begin{aligned}
\Bx^1 &= \Bx_2 \cdot \inv{ \Bx_1 \wedge \Bx_2 } \\
\Bx^2 &= \Bx_1 \cdot \inv{ \Bx_2 \wedge \Bx_1 }.
\end{aligned}
}
%\end{dmath}

Geometrically, dotting with the bivector of the plane is a hybrid rotation and scaling operation.
For example, for \R{2} with \( \Bx_1 = a_1 \Be_1 + a_2 \Be_2, \Bx_2 = b_1 \Be_1 + b_2 \Be_2 \), that pseudoscalar for this basis is

\begin{dmath}\label{eqn:reciprocal:260}
\Bx_1 \wedge \Bx_2
=
\lr{ a_1 \Be_1 + a_2 \Be_2 } \wedge \lr{ b_1 \Be_1 + b_2 \Be_2 }
=
\lr{ a_1 b_2 - a_2 b_1 } \Be_{12}.
\end{dmath}

This has inverse
\begin{dmath}\label{eqn:reciprocal:280}
\inv{\Bx_1 \wedge \Bx_2 }
=
\inv{ a_1 b_2 - a_2 b_1 } \Be_{21}.
\end{dmath}

So for the \R{2} the reciprocal frame is just

\begin{dmath}\label{eqn:reciprocal:300}
\begin{aligned}
\Bx^1 &= \Bx_2 \frac{\Be_{21}}{ a_1 b_2 - a_2 b_1 } \\
\Bx^2 &= \Bx_1 \frac{\Be_{12}}{ a_1 b_2 - a_2 b_1 }
\end{aligned}
\end{dmath}

The vector \( \Bx^1 \) is obtained by rotating \( \Bx_2 \) by \( -\pi/2 \), and rescaling it.
The vector \( \Bx^2 \) is similarly obtained by a scaling and a rotation of \( \Bx_1 \) by \( \pi/2 \).

Generalizing \cref{eqn:reciprocal:180} is almost possible by inspection.
Given
a subspace spanned by a three vector basis \( \setlr{ \Bx_1, \Bx_2, \Bx_3 } \) the reciprocal frame vectors can be written as dot products

\begin{dmath}\label{eqn:reciprocal:320}
\begin{aligned}
\Bx^1 &= \lr{ \Bx_2 \wedge \Bx_3 } \cdot \lr{ \Bx^3 \wedge \Bx^2 \wedge \Bx^1 } \\
\Bx^2 &= \lr{ \Bx_3 \wedge \Bx_1 } \cdot \lr{ \Bx^1 \wedge \Bx^3 \wedge \Bx^2 } \\
\Bx^3 &= \lr{ \Bx_1 \wedge \Bx_2 } \cdot \lr{ \Bx^2 \wedge \Bx^1 \wedge \Bx^3 } \\
\end{aligned}
\end{dmath}

Each of those trivector terms equals \( - \Bx^1 \wedge \Bx^2 \wedge \Bx^3 \) and can be related to the (known) pseudoscalar \( \Bx_1 \wedge \Bx_2 \wedge \Bx_3 \) by observing that

\begin{dmath}\label{eqn:reciprocal:340}
\lr{ \Bx^1 \wedge \Bx^2 \wedge \Bx^3 } \cdot \lr{ \Bx_3 \wedge \Bx_2 \wedge \Bx_1 }
=
\Bx^1 \cdot \lr{ \Bx^2 \cdot \lr{ \Bx^3 \cdot \lr{ \Bx_3 \wedge \Bx_2 \wedge \Bx_1 } }}
=
\Bx^1 \cdot \lr{ \Bx^2 \cdot \lr{ \Bx_2 \wedge \Bx_1 } }
=
\Bx^1 \cdot \Bx_1
=
1,
\end{dmath}

which means that

\begin{dmath}\label{eqn:reciprocal:360}
-\Bx^1 \wedge \Bx^2 \wedge \Bx^3
= -\inv{ \Bx_3 \wedge \Bx_2 \wedge \Bx_1 }
= \inv{ \Bx_1 \wedge \Bx_2 \wedge \Bx_3 },
\end{dmath}

and

\boxedEquation{eqn:reciprocal:380}{
\begin{aligned}
\Bx^1 &= \lr{ \Bx_2 \wedge \Bx_3 } \cdot \inv{ \Bx_1 \wedge \Bx_2 \wedge \Bx_3 } \\
\Bx^2 &= \lr{ \Bx_3 \wedge \Bx_1 } \cdot \inv{ \Bx_1 \wedge \Bx_2 \wedge \Bx_3 } \\
\Bx^3 &= \lr{ \Bx_1 \wedge \Bx_2 } \cdot \inv{ \Bx_1 \wedge \Bx_2 \wedge \Bx_3 }
\end{aligned}
}

Geometrically, this trivector division is a duality transformation within the subspace spanned by the three vectors \( \Bx_1, \Bx_2, \Bx_3 \), also scaling the result so that the \( \Bx_i \cdot \Bx^j = {\delta_i}^j \) condition is satisfied.

It should be clear how to generalize the reciprocal basis calculation formulas of
\cref{eqn:reciprocal:180} and \cref{eqn:reciprocal:380} to higher dimensions if desired.
%}
%\EndNoBibArticle

         %
% Copyright © 2018 Peeter Joot.  All Rights Reserved.
% Licenced as described in the file LICENSE under the root directory of this GIT repository.
%
%{
%%%Any multivector can be expressed in terms of the curvilinear basis \( \setlr{ \Bx_{u_1}, \Bx_{u_2}, \cdots, \Bx_k} \), but computation of the curvilinear coordinates requires the reciprocal basis.
%%%
%%%For example, a vector \( \Bf \) constrained to the tangent space admits a representation
%%%
%%%\begin{dmath}\label{eqn:curvilinearDefinedCoordinates:380}
%%%\Bf = \sum_i a_i \Bx_{u_i}.
%%%\end{dmath}
%%%
%%%Dotting with \( \Bx^{u_j} \) gives
%%%
%%%\begin{dmath}\label{eqn:curvilinearDefinedCoordinates:280}
%%%\Bf \cdot \Bx^{u_j}
%%%= \sum_i a_i \Bx_{u_i} \cdot \Bx^{u_j}
%%%= \sum_i a_i {\delta^i}_j
%%%= a_j,
%%%\end{dmath}
%%%
%%%so
%%%\begin{dmath}\label{eqn:curvilinearDefinedCoordinates:300}
%%%\Bf = \sum_i \lr{ \Bf \cdot \Bx^{u_i} } \Bx_{u_i}.
%%%\end{dmath}
%%%
\subsubsection{Bivector coordinates.}
Higher grade multivector objects may also be represented in curvilinear coordinates.  Illustrating by example, we will calculate the coordinates of a
bivector constrained to a three parameter manifold \( \Span \setlr{ \Bx_1, \Bx_2, \Bx_3 } \) which can be represented as

\begin{equation}\label{eqn:curvilinearDefinedCoordinates:320}
B
= \inv{2} \sum_{i, j} B^{ij} \Bx_{i} \wedge \Bx_{j}
= \sum_{i < j} B^{ij} \Bx_{i} \wedge \Bx_{j}
\end{equation}

The coordinates \( B^{ij} \) can be determined by dotting \( B \) with \( \Bx^{j} \wedge \Bx^{i} \), where \( i \ne j \), yielding

\begin{dmath}\label{eqn:curvilinearDefinedCoordinates:340}
B \cdot \lr{ \Bx^{j} \wedge \Bx^{i} }
=
\inv{2} \sum_{r, s} B^{rs} \lr{ \Bx_{r} \wedge \Bx_{s} } \cdot \lr{ \Bx^{j} \wedge \Bx^{i} }
=
\inv{2} \sum_{r, s} B^{rs} \lr{ \lr{ \Bx_{r} \wedge \Bx_{s} } \cdot \Bx^{j} } \cdot \Bx^{i}
=
\inv{2} \sum_{r, s} B^{rs} \lr{ \Bx_{r} {\delta_s}^j - \Bx_{s} {\delta_r}^j } \cdot \Bx^{i}
=
\inv{2} \sum_{r, s} B^{rs} \lr{ {\delta_r}^i {\delta_s}^j - {\delta_s}^i {\delta_r}^j }
=
\inv{2} \lr{ B^{i j} - B^{j i} }.
\end{dmath}

We see that the coordinates of a bivector, even with respect to a non-orthonormal basis, are antisymmetric, so
\cref{eqn:curvilinearDefinedCoordinates:340} is just \( B^{ij} \) as claimed.  That is

\begin{dmath}\label{eqn:curvilinearDefinedCoordinates:401}
B^{ij} = B \cdot \lr{ \Bx^{j} \wedge \Bx^{i} }.
\end{dmath}

Just as the reciprocal frame was instrumental for computation of the coordinates of a vector with respect to an arbitrary (i.e. non-orthonormal frame), we use the reciprocal frame to calculate the coordinates of a bivector, and could do the same for higher grade k-vectors as well.
%}

      \subsection{\R{2} reciprocal frame.}
         %
% Copyright © 2018 Peeter Joot.  All Rights Reserved.
% Licenced as described in the file LICENSE under the root directory of this GIT repository.
%
%{
How are the reciprocal frame vectors computed?  While these vectors have a natural GA representation, this is not intrinsically a GA problem, and can be solved with standard linear algebra, using a matrix inversion.
For example, given a 2D basis \( \setlr{ \Bx_1, \Bx_2 } \), the reciprocal basis can be assumed to have a coordinate representation in the original basis

\begin{dmath}\label{eqn:reciprocal_R2:100}
\begin{aligned}
\Bx^1 &= a \Bx_1 + b \Bx_2 \\
\Bx^2 &= c \Bx_1 + d \Bx_2.
\end{aligned}
\end{dmath}

Imposing the constraints of \cref{dfn:reciprocal:frame} leads to a pair of 2x2 linear systems that are easily solved to find
\begin{dmath}\label{eqn:reciprocal_R2:120}
\begin{aligned}
\Bx^1 &= \inv{ (\Bx_1)^2 (\Bx_2)^2 - \lr{ \Bx_1 \cdot \Bx_2}^2 } \lr{ (\Bx_2)^2 \Bx_1 - \lr{ \Bx_1 \cdot \Bx_2 } \Bx_2 } \\
\Bx^2 &= \inv{ (\Bx_1)^2 (\Bx_2)^2 - \lr{ \Bx_1 \cdot \Bx_2}^2 } \lr{ (\Bx_1)^2 \Bx_2 - \lr{ \Bx_1 \cdot \Bx_2 } \Bx_1 } \\
\end{aligned}
\end{dmath}

The reader may notice that for \R{3} the denominator is related to the norm of the cross product \( \Bx_1 \cross \Bx_2 \).
More generally, this can be expressed as the square of the bivector \( \Bx_1 \wedge \Bx_2 \)

\begin{dmath}\label{eqn:reciprocal_R2:140}
-\lr{\Bx_1 \wedge \Bx_2 }^2
=
-\lr{\Bx_1 \wedge \Bx_2 } \cdot \lr{\Bx_1 \wedge \Bx_2 }
=
-\lr{ \lr{\Bx_1 \wedge \Bx_2 } \cdot \Bx_1 } \cdot \Bx_2
=
(\Bx_1)^2 (\Bx_2)^2 - \lr{\Bx_1 \cdot \Bx_2}^2.
\end{dmath}

Additionally, the numerators are each dot products of \( \Bx_1, \Bx_2 \) with that same bivector

\begin{dmath}\label{eqn:reciprocal_R2:160}
\begin{aligned}
\Bx^1 &= \frac{\Bx_2 \cdot \lr{ \Bx_1 \wedge \Bx_2 } }{ \lr{ \Bx_1 \wedge \Bx_2}^2 } \\
\Bx^2 &= \frac{\Bx_1 \cdot \lr{ \Bx_2 \wedge \Bx_1 } }{ \lr{ \Bx_1 \wedge \Bx_2}^2 },
\end{aligned}
\end{dmath}

or

%\begin{dmath}\label{eqn:reciprocal_R2:180}
\boxedEquation{eqn:reciprocal_R2:180}{
\begin{aligned}
\Bx^1 &= \Bx_2 \cdot \inv{ \Bx_1 \wedge \Bx_2 } \\
\Bx^2 &= \Bx_1 \cdot \inv{ \Bx_2 \wedge \Bx_1 }.
\end{aligned}
}
%\end{dmath}

Recall that dotting with the unit bivector of a plane (or its inverse) rotates a vector in that plane by \( \pi/2 \).
In a plane subspace, such a rotation is exactly the transformation to ensure that \( \Bx_1 \cdot \Bx^2 = \Bx_2 \cdot \Bx^1 = 0 \).
This shows that the reciprocal frame for the basis of a two dimensional subspace is found by a duality transformation of each of the curvilinear coordinates, plus a subsequent scaling operation.
As \( \Bx_1 \wedge \Bx_2 \), the pseudoscalar for the subspace spanned by \( \setlr{ \Bx_1, \Bx_2 } \), is not generally a unit bivector, the dot product with its inverse also has a scaling effect.

%\makeexample{Numerical example}{example:reciprocal_R2:n}{
\paragraph{Numerical example:}
Here is a Mathematica calculation of the reciprocal frame depicted in \cref{fig:obliqueReciprocal:obliqueReciprocalFig2}

\begin{mmaCell}[moredefined={x1, x2, inverse, e, x12, OuterProduct, GeometricProduct, x12inverse, s1, InnerProduct, s2, dots},morepattern={a_, a, b_, b}]{Input}
  ClearAll[x1, x2, inverse]
  x1 = e[1] + e[2]; x2 = e[1] + 2 e[2];
  x12 = OuterProduct[x1, x2];
  inverse[a_] := a / GeometricProduct[a, a] ;
  x12inverse = inverse[x12];
  s1 = InnerProduct[x2, x12inverse];
  s2 = InnerProduct[x1, -x12inverse];
  s1
  s2
  dots[a_,b_] := \{a , "\(\pmb{\cdot}\)", b, " = ",
                   InnerProduct[a // ReleaseHold, b // ReleaseHold]\};
  MapThread[dots, \{\{x1 // HoldForm, x2 // HoldForm,
                      x1 // HoldForm, x2 // HoldForm\},
                  \{s1 // HoldForm, s1 // HoldForm,
                    s2 // HoldForm, s2 // HoldForm\}\}] // Grid
\end{mmaCell}
\begin{mmaCell}{Output}
  2 e[1] - e[2]
\end{mmaCell}
\begin{mmaCell}{Output}
  -e[1] + e[2]
\end{mmaCell}
\begin{mmaCell}{Output}
  x1	\(\cdot\)	s1	 = 	1
  x2	\(\cdot\)	s1	 = 	0
  x1	\(\cdot\)	s2	 = 	0
  x2	\(\cdot\)	s2	 = 	1
\end{mmaCell}

This shows the reciprocal vector calculations using \cref{eqn:reciprocal_R2:180} and that the
%This gives \( \Bx^1 = 2 \Be_1 - \Be_2, \Bx^2 = \Be_2 - \Be_1 \), which
%satisfies the
defining property
\( \Bx_i \cdot \Bx^j = {\delta_i}^j \)
of the reciprocal frame vectors
is satisfied.
%} % example

\paragraph{Example: \R{2}:}

Given a pair of arbitrary oriented vectors in \R{2}, \( \Bx_1 = a_1 \Be_1 + a_2 \Be_2, \Bx_2 = b_1 \Be_1 + b_2 \Be_2 \), the pseudoscalar associated with the basis \( \setlr{ \Bx_1, \Bx_2} \) is

\begin{dmath}\label{eqn:reciprocal_R2:260}
\Bx_1 \wedge \Bx_2
=
\lr{ a_1 \Be_1 + a_2 \Be_2 } \wedge \lr{ b_1 \Be_1 + b_2 \Be_2 }
=
\lr{ a_1 b_2 - a_2 b_1 } \Be_{12}.
\end{dmath}

The inverse of this pseudoscalar is

\begin{dmath}\label{eqn:reciprocal_R2:280}
\inv{\Bx_1 \wedge \Bx_2 }
=
\inv{ a_1 b_2 - a_2 b_1 } \Be_{21}.
\end{dmath}

So for this fixed oblique \R{2} basis, the reciprocal frame is just

\begin{dmath}\label{eqn:reciprocal_R2:300}
\begin{aligned}
\Bx^1 &= \Bx_2 \frac{\Be_{21}}{ a_1 b_2 - a_2 b_1 } \\
\Bx^2 &= \Bx_1 \frac{\Be_{12}}{ a_1 b_2 - a_2 b_1 }
\end{aligned}
\end{dmath}

The vector \( \Bx^1 \) is obtained by rotating \( \Bx_2 \) by \( -\pi/2 \), and rescaling it by the area of the parallelogram spanned by \( \Bx_1, \Bx_2 \).
The vector \( \Bx^2 \) is obtained with the same scaling plus a rotation of \( \Bx_1 \) by \( \pi/2 \).

%}

      \subsection{\R{3} reciprocal frame.}
         %
% Copyright © 2018 Peeter Joot.  All Rights Reserved.
% Licenced as described in the file LICENSE under the root directory of this GIT repository.
%
%{

In this section we generalize \cref{eqn:reciprocal_R2:180} to \R{3} vectors, which will illustrate the general case by example.

Given
a subspace spanned by a three vector basis \( \setlr{ \Bx_1, \Bx_2, \Bx_3 } \) the reciprocal frame vectors can be written as dot products

\begin{dmath}\label{eqn:reciprocal_R3:320}
\begin{aligned}
\Bx^1 &= \lr{ \Bx_2 \wedge \Bx_3 } \cdot \lr{ \Bx^3 \wedge \Bx^2 \wedge \Bx^1 } \\
\Bx^2 &= \lr{ \Bx_3 \wedge \Bx_1 } \cdot \lr{ \Bx^1 \wedge \Bx^3 \wedge \Bx^2 } \\
\Bx^3 &= \lr{ \Bx_1 \wedge \Bx_2 } \cdot \lr{ \Bx^2 \wedge \Bx^1 \wedge \Bx^3 }.
\end{aligned}
\end{dmath}

Each of those trivector terms equals \( - \Bx^1 \wedge \Bx^2 \wedge \Bx^3 \) and can be related to the (known) pseudoscalar \( \Bx_1 \wedge \Bx_2 \wedge \Bx_3 \) by observing that

\begin{dmath}\label{eqn:reciprocal_R3:340}
\lr{ \Bx^1 \wedge \Bx^2 \wedge \Bx^3 } \cdot \lr{ \Bx_3 \wedge \Bx_2 \wedge \Bx_1 }
=
\Bx^1 \cdot \lr{ \Bx^2 \cdot \lr{ \Bx^3 \cdot \lr{ \Bx_3 \wedge \Bx_2 \wedge \Bx_1 } }}
=
\Bx^1 \cdot \lr{ \Bx^2 \cdot \lr{ \Bx_2 \wedge \Bx_1 } }
=
\Bx^1 \cdot \Bx_1
=
1,
\end{dmath}
which means that

\begin{dmath}\label{eqn:reciprocal_R3:360}
-\Bx^1 \wedge \Bx^2 \wedge \Bx^3
= -\inv{ \Bx_3 \wedge \Bx_2 \wedge \Bx_1 }
= \inv{ \Bx_1 \wedge \Bx_2 \wedge \Bx_3 },
\end{dmath}
and

\boxedEquation{eqn:reciprocal:380}{
\begin{aligned}
\Bx^1 &= \lr{ \Bx_2 \wedge \Bx_3 } \cdot \inv{ \Bx_1 \wedge \Bx_2 \wedge \Bx_3 } \\
\Bx^2 &= \lr{ \Bx_3 \wedge \Bx_1 } \cdot \inv{ \Bx_1 \wedge \Bx_2 \wedge \Bx_3 } \\
\Bx^3 &= \lr{ \Bx_1 \wedge \Bx_2 } \cdot \inv{ \Bx_1 \wedge \Bx_2 \wedge \Bx_3 }
\end{aligned}
}

Geometrically, dotting with this trivector is a duality transformation within the subspace spanned by the three vectors \( \Bx_1, \Bx_2, \Bx_3 \), also scaling the result so that the \( \Bx_i \cdot \Bx^j = {\delta_i}^j \) condition is satisfied.  The scaling factor is the volume of the parallelopiped spanned by \( \Bx_1, \Bx_2, \Bx_3 \).

%}

         %\subsection{General reciprocal basis calculation.}
         %   
It should be clear how to generalize the reciprocal basis calculation formulas of
\cref{eqn:reciprocal:180} and \cref{eqn:reciprocal:380} to higher dimensions if desired.

FIXME: the sign required for each term may not be obvious.  State this explicitly.



      \subsection{Problems.}
         %
% Copyright © 2016 Peeter Joot.  All Rights Reserved.
% Licenced as described in the file LICENSE under the root directory of this GIT repository.
%
\makeproblem{Reciprocal frame for two dimensional subspace.}{problem:reciprocal:2dsubspaceRecip}{
Prove \cref{eqn:reciprocal:120}.
} % problem

\makeanswer{problem:reciprocal:2dsubspaceRecip}{

Assuming the representation of \cref{eqn:reciprocal:100}, the dot products are

\begin{dmath}\label{eqn:2dreciprocalMatrixCalculation:200}
\begin{aligned}
1 &= \Bx_1 \cdot \Bx^1 = a \Bx_1^2 + b \Bx_1 \cdot \Bx_2 \\
0 &= \Bx_2 \cdot \Bx^1 = a \Bx_2 \cdot \Bx_1 + b \Bx_2^2 \\
0 &= \Bx_1 \cdot \Bx^2 = c \Bx_1^2 + d \Bx_1 \cdot \Bx_2 \\
1 &= \Bx_2 \cdot \Bx^2 = c \Bx_2 \cdot \Bx_1 + d \Bx_2^2
\end{aligned}
\end{dmath}

This can be written out as a pair of matrix equations

\begin{dmath}\label{eqn:2dreciprocalMatrixCalculation:220}
\begin{aligned}
\begin{bmatrix}
1 \\
0
\end{bmatrix}
&=
\begin{bmatrix}
\Bx_1^2 & \Bx_1 \cdot \Bx_2 \\
\Bx_2 \cdot \Bx_1 & \Bx_2^2 \\
\end{bmatrix}
\begin{bmatrix}
a \\
b
\end{bmatrix} \\
\begin{bmatrix}
0 \\
1
\end{bmatrix}
&=
\begin{bmatrix}
\Bx_1^2 & \Bx_1 \cdot \Bx_2 \\
\Bx_2 \cdot \Bx_1 & \Bx_2^2 \\
\end{bmatrix}
\begin{bmatrix}
c \\
d
\end{bmatrix}.
\end{aligned}
\end{dmath}

The matrix inverse is
\begin{dmath}\label{eqn:2dreciprocalMatrixCalculation:240}
{
\begin{bmatrix}
\Bx_1^2 & \Bx_1 \cdot \Bx_2 \\
\Bx_2 \cdot \Bx_1 & \Bx_2^2 \\
\end{bmatrix}
}^{-1}
=
\inv{ \Bx_1^2 \Bx_2^2 - \lr{\Bx_1 \cdot \Bx_2}^2 }
\begin{bmatrix}
\Bx_2^2 & -\Bx_1 \cdot \Bx_2 \\
-\Bx_2 \cdot \Bx_1 & \Bx_1^2 \\
\end{bmatrix}
\end{dmath}

Multiplying by the \( (1,0) \), and \( (0,1) \) vectors picks out the respective columns, and gives \cref{eqn:reciprocal:120}.
} % answer

         %
% Copyright © 2016 Peeter Joot.  All Rights Reserved.
% Licenced as described in the file LICENSE under the root directory of this GIT repository.
%

\index{reciprocal frame}
\makeproblem{Two vector reciprocal frame}{problem:2subspaceR3reciprocalExample:2subspaceR3reciprocalExample}{
Calculate the reciprocal frame for the \R{3} subspace spanned by \( \setlr{ \Bx_1, \Bx_2 } \) where

\begin{dmath}\label{eqn:2subspaceR3reciprocalExample:20}
\begin{aligned}
\Bx_1 &= \Be_1 + 2 \Be_2 \\
\Bx_2 &= \Be_2 - \Be_3.
\end{aligned}
\end{dmath}
} % problem

\makeanswer{problem:2subspaceR3reciprocalExample:2subspaceR3reciprocalExample}{
The bivector for the plane spanned by this basis is

\begin{dmath}\label{eqn:2subspaceR3reciprocalExample:40}
\Bx_1 \wedge \Bx_2
=
\lr{ \Be_1 + 2 \Be_2 } \wedge
\lr{ \Be_2 - \Be_3 }
=
\Be_{12} - \Be_{13} - 2 \Be_{23}
=
\Be_{12} + \Be_{31} + 2 \Be_{32}.
\end{dmath}

This has the square
\begin{dmath}\label{eqn:2subspaceR3reciprocalExample:60}
\lr{ \Bx_1 \wedge \Bx_2 }^2
=
\lr{ \Be_{12} + \Be_{31} + 2 \Be_{32} }
\cdot
\lr{ \Be_{12} + \Be_{31} + 2 \Be_{32} }
=
-1 -1 -4
=
-6.
\end{dmath}

Dotting \( -\Bx_1 \) with the bivector is
\begin{dmath}\label{eqn:2subspaceR3reciprocalExample:80}
\Bx_1 \cdot \lr{ \Bx_2 \wedge \Bx_1 }
=
-\lr{ \Be_1 + 2 \Be_2 } \cdot \lr{\Be_{12} + \Be_{31} + 2 \Be_{32} }
=
-\lr{ \Be_2 - \Be_3 - 2 \Be_1 - 4 \Be_3 }
= 2 \Be_1 - \Be_2 + 5 \Be_3.
\end{dmath}

For \( \Bx_2 \) the dot product with the bivector is

\begin{dmath}\label{eqn:2subspaceR3reciprocalExample:100}
\Bx_2 \cdot \lr{ \Bx_1 \wedge \Bx_2 }
=
\lr{ \Be_2 - \Be_3 } \cdot \lr{\Be_{12} + \Be_{31} + 2 \Be_{32} }
=
- \Be_1 - 2 \Be_3 - \Be_1 - 2 \Be_2
=
- 2 \Be_1 - 2 \Be_2 - 2 \Be_3,
\end{dmath}
so
\begin{dmath}\label{eqn:2subspaceR3reciprocalExample:120}
\begin{aligned}
\Bx^1 &= \inv{3} \lr{ \Be_1 + \Be_2 + \Be_3 } \\
\Bx^2 &= \inv{6} \lr{ -2 \Be_1 + \Be_2 - 5 \Be_3 }.
\end{aligned}
\end{dmath}
It is easy to verify that this has the desired semantics.
} % answer

   \section{Curvilinear coordinates.}
      \subsection{Two parameters.}
         %
% Copyright © 2017 Peeter Joot.  All Rights Reserved.
% Licenced as described in the file LICENSE under the root directory of this GIT repository.
%
Curvilinear coordinates can be defined for any subspace spanned by a parameterized vector into that space.  Given, for example, a vector into a subspace parameterized by parameters \(u,v\)

\begin{dmath}\label{eqn:curvilinear:20}
\Bx = \Bx(u, v),
\end{dmath}

the partials with respect to these parameters

\begin{dmath}\label{eqn:curvilinear:40}
\begin{aligned}
d\Bx_u &= \PD{u}{\Bx} du \\
d\Bx_v &= \PD{v}{\Bx} dv
\end{aligned}
\end{dmath}

span the space at the point that these partials are evaluated.  In the language of differential forms, this localized subspace is called the tangent space.  It is generally desirable to consider parameterizations for which the tangent space volume element is non-zero.  In this case, that is

\begin{dmath}\label{eqn:curvilinear:60}
d\Bx_u \wedge d\Bx_v \ne 0.
\end{dmath}

The differentials form a basis for the tangent space, as do the partials themselves

\begin{dmath}\label{eqn:curvilinear:80}
\begin{aligned}
\Bx_u &= \PD{u}{\Bx} \\
\Bx_v &= \PD{v}{\Bx}.
\end{aligned}
\end{dmath}

There is no reason to presume that there is any orthonormality constraint on the basis \( \setlr{ \Bx_u, \Bx_v } \) for this two parameter subspace, so a reciprocal basis \( \setlr{ \Bx^u, \Bx^v } \)
must be used to compute coordinates.
%, defined by \( \Bx^i \cdot \Bx_j = {\delta^i}_j \),  must be used to compute coordinates.   [SEE: GAelectrodynamics: 3.1].

More generally, given a parameterization of \( \Bx(u_1, u_2, \cdots, u_k) \), a curvilinear basis defined on the tangent space is induced by the partials

\begin{dmath}\label{eqn:curvilinear:240}
\Bx_{u_i} = \PD{u_i}{\Bx}.
\end{dmath}

The volume element for the subspace is

\begin{dmath}\label{eqn:curvilinear:260}
d^k \Bx = du_1 du_2 \cdots du_k\,
\Bx_{u_1} \wedge
\Bx_{u_2} \wedge \cdots \wedge
\Bx_{u_k}.
\end{dmath}

Unlike a scalar volume, this volume element is oriented.  Any multivector can be expressed in terms of the curvilinear basis \( \setlr{ \Bx_{u_1}, \Bx_{u_2}, \cdots, \Bx_k} \), but computation of the curvilinear coordinates requires the reciprocal basis.  For example, a vector \( \Bf \) constrained to the tangent space admits a representation

\begin{dmath}\label{eqn:curvilinear:380}
\Bf = \sum_i a_i \Bx_{u_i}.
\end{dmath}

Dotting with \( \Bx^{u_j} \) gives

\begin{dmath}\label{eqn:curvilinear:280}
\Bf \cdot \Bx^{u_j}
= \sum_i a_i \Bx_{u_i} \cdot \Bx^{u_j}
= \sum_i a_i {\delta^i}_j
= a_j,
\end{dmath}

so
\begin{dmath}\label{eqn:curvilinear:300}
\Bf = \sum_i \lr{ \Bf \cdot \Bx^{u_i} } \Bx_{u_i}.
\end{dmath}

Higher grade multivector objects may also be represented in curvilinear coordinates.  For example, given a bivector constrained to the tangent space

\begin{dmath}\label{eqn:curvilinear:320}
B = \inv{2} \sum_{i, j} b_{ij} \Bx_{u_i} \wedge \Bx_{u_j},
\end{dmath}

the coordinates \( b_{ij} \) can be determined by dotting \( B \) with \( \Bx^{u_j} \wedge \Bx^{u_i} \), yielding

\begin{dmath}\label{eqn:curvilinear:340}
B \cdot \lr{ \Bx^{u_j} \wedge \Bx^{u_i} }
=
\inv{2} \sum_{i' , j'} b_{i'j'} \lr{ \Bx_{u_i'} \wedge \Bx_{u_j'} } \cdot \lr{ \Bx^{u_j} \wedge \Bx^{u_i} }
=
\inv{2} \sum_{i' , j'} b_{i'j'} \lr{ \lr{ \Bx_{u_i'} \wedge \Bx_{u_j'} } \cdot \Bx^{u_j} } \cdot \Bx^{u_i}
=
\inv{2} \sum_{i' , j'} b_{i'j'} \lr{  \Bx_{u_i'} {\delta_j'}^j - \Bx_{u_j'} {\delta_i'}^j } \cdot \Bx^{u_i}
=
\inv{2} \sum_{i' , j'} b_{i'j'} \lr{  {\delta_i'}^i {\delta_j'}^j - {\delta_j'}^i {\delta_i'}^j }
=
\inv{2} \lr{ b_{i j} - b_{ji} }.
\end{dmath}

When \( i \ne j \) this is \( b_{ij} \) and is zero otherwise.  The curvilinear representation of the bivector is therefore

\begin{dmath}\label{eqn:curvilinear:400}
B = \sum_{i < j} \lr{ B \cdot \lr{ \Bx^{u_j} \wedge \Bx^{u_i} }} \Bx_{u_i} \wedge \Bx_{u_j}.
\end{dmath}


      \subsection{Three (or more) parameters.}
         %
% Copyright © 2018 Peeter Joot.  All Rights Reserved.
% Licenced as described in the file LICENSE under the root directory of this GIT repository.
%
%{

We can extend the previous two parameter subspace ideas to higher dimensional (or one dimensional) subspaces associated with a parameterization

\index{tangent space}
\index{curvilinear coordinates}
\index{oriented volume element}
\index{volume element}
\index{\(d^k \Bx\)}
\index{\(\Bx_i\)}
\makedefinition{Curvilinear coordinates and volume element}{dfn:curvilinearThree:280}{
Given a parameterization \( \Bx(u_1, u_2, \cdots, u_k) \) with \( k \) degrees of freedom, we define the curvilinear basis elements \( \Bx_i \) by the partials
\begin{equation*}
\Bx_{i} = \PD{u_i}{\Bx}.
\end{equation*}
The span of \( \setlr{ \Bx_{i} } \) at the point of evaluation is called the tangent space.
A subspace associated with a parameterization of this sort is also called a manifold.
The volume element for the subspace is
\begin{equation*}
d^k \Bx = du_1 du_2 \cdots du_k\,
\Bx_{1} \wedge
\Bx_{2} \wedge \cdots \wedge
\Bx_{k}.
\end{equation*}
Such a volume element is a k-vector.  The volume of the (hyper-) parallelepiped bounded by \( \setlr{ \Bx_{i} } \)  is \( \sqrt{\Abs{(d^k \Bx)^2}} \).
} % definition

We will assume that the parameterization is non-generate.
This means that the
volume element \( d^k \Bx \) is non-zero in the region of interest.
Note that a zero volume element implies a linear dependency in the curvilinear basis elements \( \Bx_i \).

Given a parameterization \( \Bx = \Bx(u,v,\cdots, w) \), write
\( \Bx_u, \Bx_v, \cdots, \Bx_w \) for the curvilinear basis elements, and
\( \Bx^u, \Bx^v, \cdots, \Bx^w \) for the reciprocal frame.
When doing so, sums over numeric indexes like \( \sum_i \Bx^i \Bx_i \) should be interpreted as a sum over all the parameter labels, i.e. \( \Bx^u \Bx_u + \Bx^v \Bx_v + \cdots \).

%}

      \subsection{Gradient.}
         %
% Copyright � 2018 Peeter Joot.  All Rights Reserved.
% Licenced as described in the file LICENSE under the root directory of this GIT repository.
%
%{
%%%\input{../latex/blogpost.tex}
%%%\renewcommand{\basename}{gradient}
%%%%\renewcommand{\dirname}{notes/phy1520/}
%%%\renewcommand{\dirname}{notes/ece1228-electromagnetic-theory/}
%%%%\newcommand{\dateintitle}{}
%%%%\newcommand{\keywords}{}
%%%
%%%\input{../latex/peeter_prologue_print2.tex}
%%%
%%%\usepackage{peeters_layout_exercise}
%%%\usepackage{peeters_braket}
%%%\usepackage{peeters_figures}
%%%\usepackage{siunitx}
%%%%\usepackage{mhchem} % \ce{}
%%%%\usepackage{macros_bm} % \bcM
%%%%\usepackage{macros_qed} % \qedmarker
%%%%\usepackage{txfonts} % \ointclockwise
%%%
%%%\beginArtNoToc
%%%
%%%\generatetitle{Gradient and vector derivative.}
%%%%\chapter{Gradient.}
%%%\label{chap:gradient}
%%%
%%%\paragraph{definition}
%%%%
% Copyright © 2018 Peeter Joot.  All Rights Reserved.
% Licenced as described in the file LICENSE under the root directory of this GIT repository.
%
%{

We can extend the previous two parameter subspace ideas to higher dimensional (or one dimensional) subspaces associated with a parameterization

\index{tangent space}
\index{curvilinear coordinates}
\index{oriented volume element}
\index{volume element}
\index{\(d^k \Bx\)}
\index{\(\Bx_i\)}
\makedefinition{Curvilinear coordinates and volume element}{dfn:curvilinearThree:280}{
Given a parameterization \( \Bx(u_1, u_2, \cdots, u_k) \) with \( k \) degrees of freedom, we define the curvilinear basis elements \( \Bx_i \) by the partials
\begin{equation*}
\Bx_{i} = \PD{u_i}{\Bx}.
\end{equation*}
The span of \( \setlr{ \Bx_{i} } \) at the point of evaluation is called the tangent space.
A subspace associated with a parameterization of this sort is also called a manifold.
The volume element for the subspace is
\begin{equation*}
d^k \Bx = du_1 du_2 \cdots du_k\,
\Bx_{1} \wedge
\Bx_{2} \wedge \cdots \wedge
\Bx_{k}.
\end{equation*}
Such a volume element is a k-vector.  The volume of the (hyper-) parallelepiped bounded by \( \setlr{ \Bx_{i} } \)  is \( \sqrt{\Abs{(d^k \Bx)^2}} \).
} % definition

We will assume that the parameterization is non-generate.
This means that the
volume element \( d^k \Bx \) is non-zero in the region of interest.
Note that a zero volume element implies a linear dependency in the curvilinear basis elements \( \Bx_i \).

Given a parameterization \( \Bx = \Bx(u,v,\cdots, w) \), write
\( \Bx_u, \Bx_v, \cdots, \Bx_w \) for the curvilinear basis elements, and
\( \Bx^u, \Bx^v, \cdots, \Bx^w \) for the reciprocal frame.
When doing so, sums over numeric indexes like \( \sum_i \Bx^i \Bx_i \) should be interpreted as a sum over all the parameter labels, i.e. \( \Bx^u \Bx_u + \Bx^v \Bx_v + \cdots \).

%}

%%%
%%%\paragraph{Gradient.}
%%%
With the introduction of the ideas of reciprocal frame and curvilinear coordinates, we are getting closer to be able to formulate the geometric algebra generalizations of vector calculus.

The next step in the required mathematical preliminaries for geometric calculus is to determine the form of the gradient with respect to curvilinear coordinates and the
parameters associated with those coordinates.

Suppose we have a vector parameterization of \R{N}

\begin{dmath}\label{eqn:gradient:60}
\Bx = \Bx(u_1, u_2, \cdots, u_N).
\end{dmath}

We can employ the chain rule to express the gradient in terms of derivatives with respect to \( u_i \)

\begin{dmath}\label{eqn:gradient:80}
\spacegrad
=
\sum_i \Be_i \PD{x_i}{}
=
\sum_{i,j} \Be_i
\PD{x_i}{u_j}
\PD{u_j}{}
=
\sum_j \lr{ \sum_i \Be_i \PD{x_i}{u_j} } \PD{u_j}{}
=
\sum_j \lr{ \spacegrad u_j } \PD{u_j}{}.
\end{dmath}

It turns out that the gradients of the parameters are in fact the reciprocal frame vectors

\maketheorem{Reciprocal frame vectors}{thm:curvilinearGradient:1}{
Given a curvilinear basis with elements \( \Bx_i = \PDi{u_i}{\Bx} \), the reciprocal frame vectors are given by
\begin{dmath*}
\Bx^i = \spacegrad u_i.
\end{dmath*}
} % theorem

This can be proven by direct computation

\begin{dmath}\label{eqn:gradient:20}
\Bx^i \cdot \Bx_j
=
(\spacegrad u_i) \cdot \PD{u_j}{\Bx}
=
\sum_{r,s=1}^n
\lr{ \Be_r \PD{x_r}{u_i} } \cdot \lr{ \Be_s \PD{u_j}{x_s} }
=
\sum_{r,s=1}^n (\Be_r \cdot \Be_s)
\PD{x_r}{u_i} \PD{u_j}{x_s}
=
\sum_{r,s=1}^n \delta_{rs}
\PD{x_r}{u_i} \PD{u_j}{x_s}
=
\sum_{r=1}^n
\PD{x_r}{u_i} \PD{u_j}{x_r}
=
\PD{u_i}{u_j}
=
\delta_{ij}.
\end{dmath}

This shows that \( \Bx^i = \spacegrad u_i \) has the properties required of the reciprocal frame, proving the theorem.  We are now able to define the gradient with respect to an arbitrary set of parameters

\maketheorem{Curvilinear representation of the gradient}{thm:curvilinearGradient:2}{
Given an N-parameter vector parameterization
\( \Bx = \Bx(u_1, u_2, \cdots, u_N) \)
of \R{N},
with curvilinear basis elements \( \Bx_i = \PDi{u_i}{\Bx} \), the gradient can be expressed as
\begin{dmath*}
\spacegrad = \sum_i \Bx^i \PD{u_i}{}.
\end{dmath*}
It is often convenient to define \( \partial_i \equiv \PDi{u_i}{} \), so that the gradient can be expressed in mixed index representation
\begin{dmath*}
\spacegrad = \sum_i \Bx^i \partial_i.
\end{dmath*}
%or the same with sums over mixed indexes implied.
} % theorem


%}
%\EndArticle

      \subsection{Vector derivative.}
         %
% Copyright © 2018 Peeter Joot.  All Rights Reserved.
% Licenced as described in the file LICENSE under the root directory of this GIT repository.
%
%{

Given curvilinear coordinates defined on a subspace \cref{dfn:curvilinearThree:280}, we don't have enough parameters to define the gradient.  For calculus on the k-dimensional subspace, we define the vector derivative

\index{vector derivative}
\makedefinition{Vector derivative}{dfn:gradient:100}{
Given an k-parameter vector parameterization
\( \Bx = \Bx(u_1, u_2, \cdots, u_k) \) of \R{N} with \( k \le N \),
and curvilinear basis elements \( \Bx_i = \PDi{u_i}{\Bx} \), the vector derivative \( \boldpartial \) is defined as
\begin{dmath*}
\boldpartial = \sum_{i=1}^k \Bx^i \partial_i.
\end{dmath*}
} % theorem

When the dimension of the subspace (number of parameters) equals the dimension of the underlying vector space, the vector derivative equals the gradient.  Otherwise we can write
\begin{dmath}\label{eqn:vectorDerivative:101}
\spacegrad = \boldpartial + \spacegrad_\perp,
\end{dmath}
and can think of the vector derivative as the projection of the gradient onto the tangent space at the point of evaluation.

Please see \citep{aMacdonaldVAGC} for an excellent introduction of the reciprocal frame, the gradient, and the vector derivative, and for
details about the connectivity of the manifold ignored here.

%}


The mathematical preliminaries required to formulate geometric calculus are now finally complete.
Before we do that, let's work through some example parameterizations.
%      Many of the concepts are illuminated nicely by considering some examples.

      \subsubsection{Polar coordinates.}
         %
% Copyright © 2017 Peeter Joot.  All Rights Reserved.
% Licenced as described in the file LICENSE under the root directory of this GIT repository.
%
%\index{cylindrical coordinates}
\index{polar coordinates}
\index{curvilinear coordinates}

We will now consider a simple concrete example of a vector parameterization, that of polar coordinates in \R{2}, as illustrated in
\cref{fig:curvilinearPolar:curvilinearPolarFig1}.

\imageFigure{../figures/GAelectrodynamics/curvilinearPolarFig1}{Polar coordinates.}{fig:curvilinearPolar:curvilinearPolarFig1}{0.3}

The parameterization is simplest in complex exponential form

\begin{dmath}\label{eqn:2Dcylindrical:100}
\Bx(\rho, \phi) = \rho \Be_1 \exp\lr{ \Be_{12} \phi }.
\end{dmath}

The curvilinear coordinate basis is therefore

\begin{subequations}
\label{eqn:2Dcylindrical:120}
\begin{dmath}\label{eqn:2Dcylindrical:140}
\Bx_\rho
= \PD{\rho}{} \lr{ \rho \Be_1 \exp\lr{ \Be_{12} \phi } }
= \Be_1 \exp\lr{ \Be_{12} \phi }
\end{dmath}
\begin{dmath}\label{eqn:2Dcylindrical:160}
\Bx_\phi
= \PD{\phi}{} \lr{ \rho \Be_1 \exp\lr{ \Be_{12} \phi } }
= \rho
\Be_1 \Be_{12} \exp\lr{ \Be_{12} \phi }
= \rho
\Be_2 \exp\lr{ \Be_{12} \phi }.
\end{dmath}
\end{subequations}

To show that these vectors are 
perpendicular, we can
select the scalar grade of their product, and
use \cref{thm:SimpleProducts2:1780}, property (c) to swap the vector and complex exponential, conjugating the exponential

\begin{dmath}\label{eqn:2Dcylindrical:640}
\Bx_\rho \cdot \Bx_\phi
=
\gpgradezero{
   \lr{ \Be_1 \exp\lr{ \Be_{12} \phi } }
   \lr{ \rho \Be_2 \exp\lr{ \Be_{12} \phi } }
}
=
\rho
\gpgradezero{
   \Be_1 \exp\lr{ \Be_{12} \phi }
   \exp\lr{ -\Be_{12} \phi } \Be_2
}
=
\rho
\gpgradezero{
   \Be_1
\Be_2
}
=
0.
\end{dmath}

We can use the same method to find the (squared) length of the vectors

\begin{dmath}\label{eqn:2Dcylindrical:680}
\Bx_\rho^2
=
\gpgradezero{
   \Be_1 \exp\lr{ \Be_{12} \phi }
   \Be_1 \exp\lr{ \Be_{12} \phi }
}
=
\gpgradezero{
   \Be_1 \exp\lr{ \Be_{12} \phi }
   \exp\lr{ -\Be_{12} \phi } \Be_1
}
=
\gpgradezero{
   \Be_1^2
}
= 1,
\end{dmath}

and

\begin{dmath}\label{eqn:2Dcylindrical:700}
\Bx_\phi^2
=
\gpgradezero{
   \lr{ \rho \Be_2 \exp\lr{ \Be_{12} \phi } }
   \lr{ \rho \Be_2 \exp\lr{ \Be_{12} \phi } }
}
=
\rho^2
\gpgradezero{
   \Be_2 \exp\lr{ \Be_{12} \phi }
   \exp\lr{ -\Be_{12} \phi } \Be_2
}
=
\rho^2
\gpgradezero{
\Be_2^2
}
= \rho^2.
\end{dmath}

The volume element for this subspace is
\begin{dmath}\label{eqn:2Dcylindrical:220}
d\Bx_\rho \wedge d\Bx_\phi
=
d\rho d\phi
\Bx_\rho \wedge \Bx_\phi
=
d\rho d\phi
\gpgradetwo{
\Bx_\rho \Bx_\phi
}
=
d\rho d\phi
\gpgradetwo{
\Be_1 \exp\lr{ \Be_{12} \phi } \rho
\Be_2 \exp\lr{ \Be_{12} \phi }
}
=
\rho d\rho d\phi
\gpgradetwo{
\Be_1 \Be_2 \exp\lr{ -\Be_{12} \phi }
\exp\lr{ \Be_{12} \phi }
}
=
\rho d\rho d\phi \Be_{12}.
\end{dmath}

Observe that the (oriented) volume of a circular region of radius \( r \) in this space has the expected result

\begin{dmath}\label{eqn:2Dcylindrical:360}
\int d\Bx_\rho \wedge d\Bx_\phi
=
\int_0^r \rho d\rho \int_0^{2\pi} d\phi \Be_{12}
= \pi r^2 \Be_{12}.
\end{dmath}

\index{reciprocal basis}
Noting that this is a normal set of vectors, the reciprocal basis can be found by inspection

\begin{dmath}\label{eqn:2Dcylindrical:180}
\begin{aligned}
\Bx^\rho &= \Be_1 \exp\lr{ \Be_{12} \phi } \\
\Bx^\phi &= \inv{\rho} \Be_2 \exp\lr{ \Be_{12} \phi }.
\end{aligned}
\end{dmath}

\index{gradient}
For completeness, it's worth verifying that the gradient representation of the reciprocal frame provides this same result.
The \( x, y \) variables are related to \( \rho, \phi \) through

\begin{dmath}\label{eqn:2Dcylindrical:620}
\begin{aligned}
x &= r \cos\phi \\
y &= r \sin\phi.
\end{aligned}
\end{dmath}

Rearranging slightly to facilitate evaluation of the \( x, y \) partials

\begin{dmath}\label{eqn:2Dcylindrical:500}
\begin{aligned}
\rho^2 &= x^2 + y^2 \\
\tan\phi &= y/x,
\end{aligned}
\end{dmath}

we can evaluate the components of the gradients by implicit differentiation

\begin{dmath}\label{eqn:2Dcylindrical:520}
\begin{aligned}
2 \rho \PD{x}{\rho} &= 2 x \\
2 \rho \PD{y}{\rho} &= 2 y \\
\inv{\cos^2\phi} \PD{x}{\phi} &= -\frac{y}{x^2} \\
\inv{\cos^2\phi} \PD{y}{\phi} &= \inv{x},
\end{aligned}
\end{dmath}

The gradients are
\begin{subequations}
\label{eqn:2Dcylindrical:540}
\begin{dmath}\label{eqn:2Dcylindrical:560}
\spacegrad \rho
= \inv{\rho} (\cos\phi, \sin\phi)
= \Be_1 e^{\Be_{12} \phi}
= \Bx^\rho
\end{dmath}
\begin{dmath}\label{eqn:2Dcylindrical:580}
\spacegrad \phi
=
\cos^2 \phi \lr{ -\frac{y}{x^2}, \inv{x} }
=
\inv{\rho} ( -\sin\phi, \cos\phi )
=
\frac{\Be_2}{\rho} ( \cos\phi + \Be_{12} \sin\phi )
=
\frac{\Be_2}{\rho} e^{ \Be_{12} \phi }
=
\Bx^\phi,
\end{dmath}
\end{subequations}

which is consistent with the result found by inspection as desired.

In this particular parameterization, it is convenient to define a locally orthonormal coordinate basis \( \setlr{ \rhocap, \phicap } \)

\begin{dmath}\label{eqn:2Dcylindrical:200}
\begin{aligned}
\rhocap &= \Bx_\rho = \Be_1 \exp\lr{ \Be_{12} \phi } \\
\phicap &= \inv{\rho} \Bx_\phi = \Be_2 \exp\lr{ \Be_{12} \phi },
\end{aligned}
\end{dmath}

so that \( \Bx^\rho = \Bx_\rho = \rhocap \), \( \Bx_\phi = \rho \rhocap \), and \( \Bx^\phi = \rhocap/\rho \), and the gradient is

\begin{dmath}\label{eqn:2Dcylindrical:600}
\spacegrad
=
\Bx^\rho \PD{\rho}{}
+ \Bx^\phi \PD{\phi}{}
=
\rhocap \PD{\rho}{}
+\inv{\rho} \phicap \PD{\phi}{}.
\end{dmath}

%%Given a vector \( \Bv = \Be_1 f(\rho, \phi) + \Be_2 g(\rho, \phi) \), the cylindrical representation \( \Bv = \Bv_\rho \rhocap + \Bv_\phi \phicap \) can be found by computing the dot products
%%
%%\begin{subequations}
%%\label{eqn:2Dcylindrical:420}
%%\begin{dmath}\label{eqn:2Dcylindrical:440}
%%\Bv \cdot \rhocap
%%=
%%\gpgradezero{ (\Be_1 f + \Be_2 g) \Be_1 e^{\Be_{12} \phi} }
%%=
%%f \cos\phi + g \sin\phi
%%\end{dmath}
%%\begin{dmath}\label{eqn:2Dcylindrical:460}
%%\Bv \cdot \phicap
%%=
%%\gpgradezero{ (\Be_1 f + \Be_2 g) \Be_2 e^{\Be_{12} \phi} }
%%=
%%g \cos\phi - f \sin\phi,
%%\end{dmath}
%%\end{subequations}
%%
%%so
%%\begin{dmath}\label{eqn:2Dcylindrical:480}
%%\Bv = \lr{ f \cos\phi + g \sin\phi } \rhocap + \lr{ g \cos\phi - f \sin\phi } \phicap.
%%\end{dmath}


      \subsubsection{Spherical coordinates.}
         %
% Copyright � 2017 Peeter Joot.  All Rights Reserved.
% Licenced as described in the file LICENSE under the root directory of this GIT repository.
%
\index{curvilinear coordinates}
\index{spherical coordinates}
The spherical vector parameterization admits a compact GA representation.
From the coordinate representation, some factoring gives

\begin{dmath}\label{eqn:curvilinearspherical:20}
\Bx
= r \lr{ \Be_1 \sin\theta \cos\phi + \Be_2 \sin\theta \sin\phi + \Be_3 \cos\theta }
= r \lr{ \sin\theta \Be_1 (\cos\phi + \Be_{12} \sin\phi ) + \Be_3 \cos\theta }
= r \lr{ \sin\theta \Be_1 e^{\Be_{12} \phi } + \Be_3 \cos\theta }
= r \Be_3 \lr{ \cos\theta + \sin\theta \Be_3 \Be_1 e^{\Be_{12} \phi } }.
\end{dmath}

With
\begin{dmath}\label{eqn:curvilinearspherical:40}
\begin{aligned}
i &= \Be_{12} \\
j &= \Be_{31} e^{i \phi},
\end{aligned}
\end{dmath}

this is

\begin{dmath}\label{eqn:curvilinearspherical:60}
\Bx = r \Be_3 e^{j \theta}.
\end{dmath}

The curvilinear basis vectors can now be computed

\begin{subequations}
\label{eqn:curvilinearspherical:80}
\begin{dmath}\label{eqn:curvilinearspherical:100}
\Bx_r = \Be_3 e^{j \theta}
\end{dmath}
\begin{dmath}\label{eqn:curvilinearspherical:120}
\Bx_\theta
= r \Be_3 j e^{j \theta}
= r \Be_3 \Be_{31} e^{i\phi} e^{j \theta}
= r \Be_1 e^{i\phi} e^{j \theta}
\end{dmath}
\begin{dmath}\label{eqn:curvilinearspherical:140}
\Bx_\phi
=
\PD{\phi}{} \lr{
r \Be_3 \sin\theta \Be_{31} e^{i \phi}
}
=
r \sin\theta \Be_1 \Be_{12} e^{i \phi}
=
r \sin\theta \Be_2 e^{i \phi}.
\end{dmath}
\end{subequations}

These are all mutually normal, which can be verified by computing dot products.
With that asserted, orthonormalization of the curvilinear basis is now possible by inspection

\begin{dmath}\label{eqn:curvilinearspherical:240}
\begin{aligned}
\rcap &= \Bx_r = \Be_3 e^{j \theta} \\
\thetacap &= \inv{r} \Bx_\theta = \Be_1 e^{i\phi} e^{j \theta} \\
\phicap &= \inv{r \sin\theta} \Bx_\phi = \Be_2 e^{i \phi},
\end{aligned}
\end{dmath}

so

\begin{dmath}\label{eqn:curvilinearspherical:260}
\begin{aligned}
\Bx^r &= \rcap = \Be_3 e^{j \theta} \\
\Bx^\theta &= \inv{r} \thetacap = \inv{r} \Be_1 e^{i\phi} e^{j \theta} \\
\Bx^\phi &= \inv{r \sin\theta} \phicap = \inv{r \sin\theta} \Be_2 e^{i \phi}.
\end{aligned}
\end{dmath}

\index{gradient!spherical}
In particular, this shows that the spherical representation of the gradient is
\begin{dmath}\label{eqn:curvilinearspherical:280}
\spacegrad
=
\Bx^r \PD{r}{}
+ \Bx^\theta \PD{\theta}{}
+ \Bx^\phi \PD{\phi}{}
=
\rcap \PD{r}{}
+\inv{r} \thetacap \PD{\theta}{}
+\inv{r \sin\theta} \phicap \PD{\phi}{}.
\end{dmath}

The spherical (oriented) volume element can also be computed in a compact fashion, without having to evaluate a very messy Jacobian determinant

\begin{dmath}\label{eqn:curvilinearspherical:300}
\Bx_r \wedge \Bx_\theta \wedge \Bx_\phi
=
\gpgradethree{
\Bx_r \Bx_\theta \Bx_\phi
}
=
\gpgradethree{
\Be_3 e^{j \theta}
r \Be_1 e^{i\phi} e^{j \theta}
r \sin\theta \Be_2 e^{i \phi}
}
=
r^2 \sin\theta
\gpgradethree{
\Be_3 e^{j \theta}
\Be_1 e^{i\phi} e^{j \theta}
\Be_2 e^{i \phi}
}
=
r^2 \sin\theta\, \Be_{123}
.
\end{dmath}

The final reduction is left as a problem for the student.
It is left to the student to evaluate whether this method is easier or more difficult than the conventional volume element Jacobean determinant expansion

\begin{dmath}\label{eqn:curvilinearspherical:320}
dV =
dr d\theta d\phi\,
\frac{\partial( x_1, x_2, x_3)}{\partial(r, \theta, \phi)}
=
dr d\theta d\phi\,
\begin{vmatrix}
\sin\theta \cos\phi & \sin\theta \sin\phi & \cos\theta \\
r \cos\theta \cos\phi & r \cos\theta \sin\phi & -r \sin\theta \\
-r \sin\theta \sin\phi & r \sin\theta \cos\phi & 0 \\
\end{vmatrix}.
\end{dmath}

Expanding this determinant (\cref{problem:curvilinearspherical:1}) gives
\begin{dmath}\label{eqn:curvilinearspherical:n}
dV =
dr d\theta d\phi\, r^2 \sin\theta,
\end{dmath}

consistent with \cref{eqn:curvilinearspherical:300}.

\makeproblem{Spherical volume Jacobean.}{problem:curvilinearspherical:1}{
Expand the determinant in \cref{eqn:curvilinearspherical:320}.
} % problem

%It is easily argued that both volume element calculation methods are best performed by a computer algebra system.
%FIXME: Wolfgang: ``please show this as an exercise!''
%  In EVA I had to use xu = r *(e1*sin(t)*cos(p)+e2*sin(t)*sin(p)+e3*cos(p)).
%  But the result deserves interpretation ..)


      \subsubsection{Toroidal coordinates.}
         %
% Copyright � 2012 Peeter Joot.  All Rights Reserved.
% Licenced as described in the file LICENSE under the root directory of this GIT repository.
%
\index{toroid}
\index{differential form}
%\imageFigure{../figures/gabook/toriodalSegment}{Toroidal parameterization.}{fig:toriodalSegment}{0.5}
\imageFigure{../figures/GAelectrodynamics/toroidFig1}{Toroidal parameterization.}{fig:toriodalSegment}{0.3}

Here is a 3D example of a parameterization with a non-orthogonal curvilinear basis, that of a
toroidal subspace specified by two angles and a radial distance to the center of the toroid, as illustrated in \cref{fig:toriodalSegment}.

The position vector on the surface of a toroid of radius \( \rho \) within the torus can be stated directly

\begin{subequations}
\begin{align}\label{eqn:torusCenterOfMassParameterization:1}
\Bx(\rho, \theta, \phi) &= e^{-j\theta/2} \left( \rho \Be_1 e^{ i \phi } + R \Be_3 \right) e^{j \theta/2} \\
i &= \Be_1 \Be_3 \\
j &= \Be_3 \Be_2
\end{align}
\end{subequations}

It happens that the unit bivectors \(i\) and \(j\) used in this construction happen
to have the
quaternion-ic properties \(i j = -j i\), and \(i^2 = j^2 = -1\) which can be verified easily.

After some regrouping the curvilinear basis is found to be

\begin{subequations}
\begin{align}\label{eqn:torusCenterOfMassParameterization:3}
\Bx_\rho &= \PD{\rho}{\Bx} = e^{-j\theta/2} \Be_1 e^{ i \phi } e^{j \theta/2} \\
\Bx_\theta &= \PD{\theta}{\Bx}
%&= e^{-j\theta/2} \left( \rho \inv{2} \left( -\Be_3 \Be_2 \Be_1 e^{ i \phi } + \Be_1 e^{ i \phi } \Be_3 \Be_2 \right) + R \Be_2 \right) e^{j \theta/2} \\
= e^{-j\theta/2} \left( R + \rho \sin\phi \right) \Be_2 e^{j \theta/2} \\
\Bx_\phi &= \PD{\phi}{\Bx} = e^{-j\theta/2} \rho \Be_3 e^{ i \phi } e^{j \theta/2}.
\end{align}
\end{subequations}

The oriented
volume element can be computed using a trivector selection operation, which conveniently wipes out a number of the interior exponentials
%\begin{align}\label{eqn:torusCenterOfMassParameterization:4}
\begin{dmath}\label{eqn:torusCenterOfMassParameterization:4}
\PD{\rho}{\Bx} \wedge \PD{\theta}{\Bx} \wedge \PD{\phi}{\Bx}
=
\rho \left( R + \rho \sin\phi \right) \gpgradethree{ e^{-j\theta/2} \Be_1 e^{ i \phi } \Be_2 \Be_3 e^{ i \phi } e^{j \theta/2} }.
%\end{align}
\end{dmath}

Note that \(\Be_1\) commutes with \(j = \Be_3 \Be_2\), so also with \(e^{-j\theta/2}\).
Also \(\Be_2 \Be_3 = -j\) anticommutes with \(i\), so
there is a conjugate commutation effect \(e^{i\phi} j = j e^{-i\phi}\).  This gives
\begin{dmath}\label{eqn:torusCenterOfMassParameterization:28}
\begin{aligned}
\gpgradethree{ e^{-j\theta/2} \Be_1 e^{ i \phi } \Be_2 \Be_3 e^{ i \phi } e^{j \theta/2} }
&=
-\gpgradethree{ \Be_1 e^{-j\theta/2} j e^{ -i \phi } e^{ i \phi } e^{j \theta/2} } \\
&=
-\gpgradethree{ \Be_1 e^{-j\theta/2} j e^{j \theta/2} } \\
&=
-\gpgradethree{ \Be_1 j } \\
&=
I.
\end{aligned}
\end{dmath}

Together the trivector grade selection reduces almost magically to just
\begin{equation}\label{eqn:torusCenterOfMassParameterization:5}
\PD{\rho}{\Bx} \wedge \PD{\theta}{\Bx} \wedge \PD{\phi}{\Bx}
=
\rho \left( R + \rho \sin\phi \right) I.
\end{equation}

\todo{Show this with Mathematica too.}

Thus the (scalar) volume element is
\begin{align}\label{eqn:torusCenterOfMassParameterization:6}
dV = \rho \left( R + \rho \sin\phi \right) d\rho d\theta d\phi.
\end{align}

As a check, it should be the case that the
volume of the complete torus using this volume element has the
expected \(V = (2 \pi R) (\pi r^2)\) value.

That volume is
\begin{align}\label{eqn:torusCenterOfMassParameterization:7}
V = \int_{\rho=0}^r \int_{\theta=0}^{2\pi} \int_{\phi=0}^{2\pi} \rho \left( R + \rho \sin\phi \right) d\rho d\theta d\phi.
\end{align}

The sine term conveniently vanishes over the \(2\pi\) interval, leaving just
\begin{align}\label{eqn:torusCenterOfMassParameterization:8}
V = \inv{2} r^2 R (2 \pi)(2 \pi),
\end{align}

as expected.


      \subsection{Problems.}
         %
% Copyright � CCYY Peeter Joot.  All Rights Reserved.
% Licenced as described in the file LICENSE under the root directory of this GIT repository.
%
\makeproblem{Spherical coordinate basis normality.}{problem:sphericaldot:1}{
\index{spherical coordinates}
Show that the spherical curvilinear basis of \cref{eqn:curvilinearspherical:80} are all mutually normal.
} % problem

\makeanswer{problem:sphericaldot:1}{
Computing the various dot products is made easier by noting that \( \Be_3 \) and \( e^{i \phi } \) commute, whereas \( e^{j\theta } \Be_3 = \Be_3 e^{-j\theta}, \Be_1 e^{i\phi} = e^{-i\phi} \Be_1, \Be_2 e^{i\phi} = e^{-i\phi} \Be_2 \) (since \( \Be_3 j \), \( \Be_1 i \) and \( \Be_2 i \) all anticommute)

\begin{subequations}
\label{eqn:sphericaldot:160}
\begin{dmath}\label{eqn:sphericaldot:180}
\Bx_r \cdot \Bx_\theta
=
\gpgradezero{
\Be_3 e^{j \theta} \Be_1 e^{i\phi} e^{j \theta}
}
=
\gpgradezero{
e^{j \theta} \Be_3 e^{j \theta} \Be_1 e^{i\phi}
}
=
\gpgradezero{
\Be_3 e^{-j \theta} e^{j \theta} \Be_1 e^{i\phi}
}
=
\gpgradezero{
\Be_3 \Be_1 e^{i\phi}
}
= 0
\end{dmath}
\begin{dmath}\label{eqn:sphericaldot:200}
\Bx_r \cdot \Bx_\phi
=
\gpgradezero{
\Be_3 e^{j \theta} r \sin\theta \Be_2 e^{i \phi}
}
=
r \sin\theta
\gpgradezero{
\Be_3 \lr{ \cos\theta + \Be_{31} \sin\theta e^{i\phi} } \Be_2 e^{i \phi}
}
=
r \sin^2\theta
\gpgradezero{
\Be_{1} e^{i\phi} \Be_2 e^{i \phi}
}
=
r \sin^2\theta
\gpgradezero{
\Be_{1} \Be_2
}
=
0
\end{dmath}
\begin{dmath}\label{eqn:sphericaldot:220}
\Bx_\theta \cdot \Bx_\phi
=
r \sin\theta
\gpgradezero{
\Be_1 e^{i\phi} e^{j \theta}
\Be_2 e^{i \phi}
}
=
r \sin\theta
\gpgradezero{
\Be_2 \Be_1 e^{j \theta}
}
=
r \sin\theta
\gpgradezero{
\Be_2 \Be_1 \lr{ \cos\theta + \Be_{31} \sin\theta e^{i \phi} }
}
=
r \sin^2\theta
\gpgradezero{
\Be_{32} e^{i \phi}
}
=
0.
\end{dmath}
\end{subequations}

} % answer

         %
% Copyright � CCYY Peeter Joot.  All Rights Reserved.
% Licenced as described in the file LICENSE under the root directory of this GIT repository.
%
\makeproblem{Spherical volume element pseudoscalar.}{problem:volumeselection:1}{
Confirm the grade three selection claim made in \cref{eqn:curvilinearspherical:300}.
} % problem

\makeanswer{problem:volumeselection:1}{

Using the commutation relations from last problem, first note that

\begin{dmath}\label{eqn:volumeselection:20}
\Be_1 e^{i\phi} e^{j \theta}
=
\Be_1 \lr{ \cos\theta e^{i\phi} + \sin\theta \Be_{31} e^{-i\phi} e^{i\phi} }
=
\Be_1 \lr{ \cos\theta + \sin\theta \Be_{31} e^{-i\phi} } e^{i\phi}
=
\lr{ \cos\theta - \sin\theta \Be_{31} e^{i\phi} } \Be_1 e^{i\phi}
=
e^{-j\theta}
\Be_1 e^{i\phi}.
\end{dmath}

This gives

\begin{dmath}\label{eqn:volumeselection:40}
\gpgradethree{
\Be_3 e^{j \theta}
\Be_1 e^{i\phi} e^{j \theta}
\Be_2 e^{i \phi}
}
=
\gpgradethree{
\Be_3
\Be_1 e^{i\phi}
\Be_2 e^{i \phi}
}
=
\gpgradethree{
\Be_3
\Be_1
\Be_2
}
=
\Be_{123}.
\end{dmath}

} % answer



%gabook: 31.1
%Also: Stokes chapter.
% Lots of examples there that should really be separated out from the stokes core content
%(now included here).
   \section{Multivector Fourier transform and phasors.}
      %
% Copyright © 2018 Peeter Joot.  All Rights Reserved.
% Licenced as described in the file LICENSE under the root directory of this GIT repository.
%
%{
It will often be convient to utilize time harmonic (frequency domain) representations.
%of \cref{eqn:greensFunctionOverview:200}.
This can be achieved by utilizing Fourier transform pairs or with a phasor representation.

We may define Fourier transform pairs of multivector fields and sources in the conventional fashion

\index{Fourier transform}
\makedefinition{Multivector Fourier transform pairs}{dfn:greensFunctionOverview:280}{
The Fourier transform pair for a multivector valued function \( f(\Bx, t) \) will be written as
\begin{equation*}
\begin{aligned}
f(\Bx, t) &= \int f_\omega(\Bx) e^{j \omega t} d\omega \\
f_\omega(\Bx) &= \inv{2 \pi} \int f(\Bx, t) e^{-j \omega t} dt,
\end{aligned}
\end{equation*}
where \( j \) is an arbitrary scalar imaginary that commutes with all multivectors.
} % definition

In these transform pairs, the imaginary \( j \) need not be represented by any geometrical imaginary such as \( \Be_{12} \).
In particular, we need not assume that the represention of \( j \) is the
\R{3} pseudoscalar \( I \), despite the fact that \( I \) does commute with all \R{3} multivectors.
We wish to have the freedom to
assume that non-geometric real and imaginary operations can be performed without picking or leaving out any specific grade pseudoscalar components of the multivector fields or sources, so we won't impose any a-priori restrictions on the representations of \( j \).
In particular, this provides the freedom to utilize phasor (fixed frequency) representions of our multivector functions.
%Introduction of yet another imaginary quantity in a geometric algebra context where we have so many to pick it somewhat unfortunate, but it allows us to apply Fourier transform techniques without worry about the non-commutative effects that might have to be considered should we choose to use a geometric imaginary to represent the frequency dependency.
We will use the engineering convention for our
phasor representations, where assuming a complex exponential time dependence of the following form is assumed

\index{time harmonic}
\index{frequency domain}
\makedefinition{Multivector phasor representation.}{dfn:greensFunctionOverview:300}{
The phasor representation \( f(\Bx) \) of a multivector valued (real) function \( f(\Bx, t) \) is defined implicitly as
\begin{equation*}
f(\Bx, t) = \Real\lr{ f(\Bx) e^{j \omega t} },
\end{equation*}
where \( j \) is an arbitrary scalar imaginary that commutes with all multivectors.
} % definition

The complex valued multivector \( f(\Bx) \) is still generated from the real Euclidean basis for \R{3}, so
there will be
no reason to introduce complex inner products spaces into the mix.

The reader must take care when reading any literature that utilizes Fourier transforms or phasor representation, since the conventions vary.
In particular the physics representation of a phasor typically uses the opposite sign convention
\( f(\Bx, t) = \Real\lr{ f(\Bx) e^{-i \omega t }} \), which toggles the sign of all the imaginaries in derived results.
%}

   \section{Green's theorem.}
      %
% Copyright © 2013 Peeter Joot.  All Rights Reserved.
% Licenced as described in the file LICENSE under the root directory of this GIT repository.
%
Given a two parameter (\(u,v\)) surface parameterization, the curvilinear coordinate representation of a vector \(\Bf\) has the form

\begin{dmath}\label{eqn:stokesTheoremGeometricAlgebra:1640}
\Bf = f_u \Bx^u + f_v \Bx^v + f_\perp \Bx^\perp.
\end{dmath}

We assume that the vector space is of dimension two or greater but otherwise unrestricted, and need not have an Euclidean basis.
Here \(f_\perp \Bx^\perp\) denotes the rejection of \(\Bf\) from the tangent space at the point of evaluation.
Green's theorem relates the integral around a closed curve to an ``area'' integral on that surface

\maketheorem{Green's Theorem}{thm:stokesTheoremGeometricAlgebra:1660}{
\index{Green's theorem}
\begin{equation*}
\ointctrclockwise \Bf \cdot d\Bl
=
\iint \lr{
-\PD{v}{f_u}
+\PD{u}{f_v}
}
du dv
\end{equation*}
}

Following the arguments used in \citep{schwartz1987pe} for Stokes' theorem in three dimensions, we first evaluate the loop integral along the differential element of the surface at the point \(\Bx(u_0, v_0)\) evaluated over the range \((du, dv)\), as shown in the infinitesimal loop of \cref{fig:loopIntegralInfinitesimal:loopIntegralInfinitesimalFig1}.

\imageFigure{../figures/gabook/loopIntegralInfinitesimalFig1}{Infinitesimal loop integral.}{fig:loopIntegralInfinitesimal:loopIntegralInfinitesimalFig1}{0.35}

Over the infinitesimal area, the loop integral decomposes into

\begin{dmath}\label{eqn:stokesTheoremGeometricAlgebra:1700}
\ointctrclockwise \Bf \cdot d\Bl
=
\int \Bf \cdot d\Bx_1
+\int \Bf \cdot d\Bx_2
+\int \Bf \cdot d\Bx_3
+\int \Bf \cdot d\Bx_4,
\end{dmath}

where the differentials along the curve are

\begin{dmath}\label{eqn:stokesTheoremGeometricAlgebra:1600}
\begin{aligned}
d\Bx_1 &= \evalbar{ \PD{u}{\Bx} }{v = v_0} du \\
d\Bx_2 &= \evalbar{ \PD{v}{\Bx} }{u = u_0 + du} dv \\
d\Bx_3 &= -\evalbar{ \PD{u}{\Bx} }{v = v_0 + dv} du \\
d\Bx_4 &= -\evalbar{ \PD{v}{\Bx} }{u = u_0} dv.
\end{aligned}
\end{dmath}

It is assumed that the parameterization change \((du, dv)\) is small enough that this loop integral can be considered planar (regardless of the dimension of the vector space).
Making use of the fact that \(\Bx^\perp \cdot \Bx_\alpha = 0\) for \(\alpha \in \setlr{u,v}\), the loop integral is

\begin{dmath}\label{eqn:stokesTheoremGeometricAlgebra:1620}
\ointctrclockwise \Bf \cdot d\Bl
=
\int
\lr{
f_u \Bx^u + f_v \Bx^v + f_\perp \Bx^\perp
}
\cdot
\Bigl(
\Bx_u(u, v_0) du - \Bx_u(u, v_0 + dv) du
+\Bx_v(u_0 + du, v) dv - \Bx_v(u_0, v) dv
\Bigr)
=
\int
f_u(u, v_0) du - f_u(u, v_0 + dv) du
+
f_v(u_0 + du, v) dv - f_v(u_0, v) dv
\end{dmath}

With the distances being infinitesimal, these differences can be rewritten as partial differentials

\begin{dmath}\label{eqn:stokesTheoremGeometricAlgebra:1860}
\ointctrclockwise \Bf \cdot d\Bl
=
\iint \lr{
-\PD{v}{f_u}
+\PD{u}{f_v}
}
du dv.
\end{dmath}

We can now sum over a larger area as in \cref{fig:loopIntegralInfinitesimalSum:loopIntegralInfinitesimalSumFig2}

\imageFigure{../figures/gabook/loopIntegralInfinitesimalSumFig2}{Sum of infinitesimal loops.}{fig:loopIntegralInfinitesimalSum:loopIntegralInfinitesimalSumFig2}{0.35}

All the opposing oriented loop elements cancel, so the integral around the complete boundary of the surface \(\Bx(u, v)\) is given by the \(u,v\) area integral of the partials difference.

We will see that Green's theorem is a special case of the Stokes' theorem.
This observation will also provide a geometric interpretation of the right hand side area integral of \cref{thm:stokesTheoremGeometricAlgebra:1660}, and allow for a coordinate free representation.

\paragraph{Special case:}

An important special case of Green's theorem is for a Euclidean two dimensional space where the vector function is

\begin{dmath}\label{eqn:stokesTheoremGeometricAlgebra:1720}
\Bf = P \Be_1 + Q \Be_2.
\end{dmath}

Here Green's theorem takes the form

\boxedEquation{eqn:stokesTheoremGeometricAlgebra:1710}{
\ointctrclockwise P dx + Q dy
=
\iint \lr{
\PD{x}{Q}
-\PD{y}{P}
}
dx dy.
}

      %\subsection{Problems}
   \section{Stokes' theorem.}
      \subsection{Statement.}
         %
% Copyright © 2016 Peeter Joot.  All Rights Reserved.
% Licenced as described in the file LICENSE under the root directory of this GIT repository.
%
\index{Stokes' theorem}
Stokes' theorem is fairly easy to state, but takes a fair amount of work to understand and unpack its implications.

%
% Copyright © 2013 Peeter Joot.  All Rights Reserved.
% Licenced as described in the file LICENSE under the root directory of this GIT repository.
%
\maketheorem{Stokes' Theorem}{thm:stokesTheoremGeometricAlgebra:1740}{

For blades \(F \in \bigwedge^{s}\), and \(m\) volume element \(d^k \Bx, s < k\),

\begin{equation*}%\label{eqn:stokesTheoremTheStatement:120}
\int_V d^k \Bx \cdot (\boldpartial \wedge F) = \int_{\partial V} d^{k-1} \Bx \cdot F.
\end{equation*}

Here the volume integral is over a \(m\) dimensional surface (manifold).  The derivative operator \(\boldpartial\) is called the vector derviative and is the projection of the gradient onto the tangent space of the manifold.  Integration over the boundary of \(V\) is indicated by \( \partial V \).
}

The vector derivative is defined by

\begin{equation}\label{eqn:stokesTheoremTheStatement:1400}
\boldpartial = \Bx^i \partial_i = \sum_i \Bx_i \PD{u^i}{}.
\end{equation}

where \( \Bx^i \) are reciprocal frame vectors dual to the tangent vector basis \( \Bx_i \) associated with the parameters \( u^1, u^2, \cdots \).
%These will be defined in more detail in the next section.
Once the volume element, vector product and the other concepts are defined, the proof of
Stokes theorem is really just a statement that

\boxedEquation{eqn:stokesTheoremGeometricAlgebra:2840}{
\int_V d^k \Bx \cdot (\Bx^i \partial_i \wedge F) =
\int_V \lr{ d^k \Bx \cdot \Bx^i } \cdot \partial_i F.
}

This dot product expansion applies to any degree blade and volume element provided the degree of the blade is less than that of the volume element (i.e. \(s < k\)).  That magic follows directly from \cref{thm:stokesTheoremGeometricAlgebra:1420}.


\index{oriented volume element}
This dot product defines the oriented surface ``area'' elements associated with the ``volume'' element \( d^k \Bx \).
That area element can be obtained from the mnemonic

\begin{dmath}\label{eqn:statement:1561}
\sum_i d^k \Bx \cdot \Bx^i,
\end{dmath}

with each of the i-th differentials evaluated.
This will be made clear by example.


      \subsection{One parameter specialization of Stokes' theorem.}
         %
% Copyright © 2016 Peeter Joot.  All Rights Reserved.
% Licenced as described in the file LICENSE under the root directory of this GIT repository.
%

An example parameterization with one parameter, and the corresponding differential with respect to that parameter, is plotted in
\cref{fig:oneParameterDifferential:oneParameterDifferentialFig1}, for a parameterization over \( [a, b] \in [0,1]\otimes[0,1] \).

\imageFigure{../figures/GAelectrodynamics/oneParameterDifferentialFig1}{One parameter manifold.}{fig:oneParameterDifferential:oneParameterDifferentialFig1}{0.3}

The differential with respect to the parameter \( a \) is

\begin{equation}\label{eqn:stokesTheoremCore:20}
d\Bx_a = \PD{a}{\Bx} da = \Bx_a da.
\end{equation}

On this curve the projection of the gradient has just one component

\begin{dmath}\label{eqn:stokesTheoremCore:40}
\boldpartial
=
\sum_\mu \Bx^\mu (\Bx_\mu \cdot \spacegrad)
=
\Bx^a \PD{a}{}
\equiv
\Bx^a \partial_a.
\end{dmath}

Please see \citep{aMacdonaldVAGC} for a full justification of the curvilinear coordinate representation of the vector derivative (or the gradient).
That text also discusses pertinent issues with the connectivity of the manifold ignored here.

Stokes' theorem for a one parameter manifold can only be expressed for scalar fields.
That is

\begin{dmath}\label{eqn:stokesTheoremCore:60}
\int d\Bx \cdot (\boldpartial \wedge \psi)
=
\int d\Bx \cdot \boldpartial \psi
=
\int da \PD{a}{ \psi }
= \evalbar{\psi}{\Delta a}.
\end{dmath}

Observe that the vector derivative can be replaced by the gradient since \( d\Bx \cdot \boldpartial = d\Bx \cdot \spacegrad \).
This is the case since dotting the
gradient with a differential element \( d\Bx \) on this curve, no component of the gradient that isn't colinear to the curve makes no contribution.

That means that Stokes' theorem for a one parameter curve is exactly the fundamental theorem of calculus for line integrals

%\begin{dmath}\label{eqn:stokesTheoremCore:80}
\boxedEquation{eqn:stokesTheoremCore:80}{
\int_{\Ba}^{\Bb} d\Bx \cdot \spacegrad \psi = \psi(\Bb) - \psi(\Ba).
}
%\end{dmath}

      \subsection{Two parameter specialization of Stokes' theorem.}
         %
% Copyright © 2016 Peeter Joot.  All Rights Reserved.
% Licenced as described in the file LICENSE under the root directory of this GIT repository.
%

An example parameterization with two parameters, and the corresponding differentials with respect to those parameters, is plotted in
\cref{fig:twoParameterDifferential:twoParameterDifferentialFig1}.

\imageFigure{../figures/GAelectrodynamics/twoParameterDifferentialFig1}{Two parameter manifold differentials.}{fig:twoParameterDifferential:twoParameterDifferentialFig1}{0.4}

Given parameters \( a, b \), the differentials along each of the parameterization directions are

\begin{dmath}\label{eqn:stokesTheoremCore:100}
\begin{aligned}
d\Bx_a &= \PD{a}{\Bx} da = \Bx_a da \\
d\Bx_b &= \PD{b}{\Bx} db = \Bx_b db.
\end{aligned}
\end{dmath}

The ``volume'' element for this parameterization (a surface area element) is

\begin{equation}\label{eqn:stokesTheoremCore:120}
d^2 \Bx
=
d\Bx_a \wedge
d\Bx_b
=
da db (\Bx_a \wedge \Bx_b).
\end{equation}

The vector derivative, the projection of the gradient onto the surface at the point of integration (also called the tangent space), now has two components

\begin{dmath}\label{eqn:stokesTheoremCore:200}
\boldpartial
=
\sum_\mu \Bx^\mu (\Bx_\mu \cdot \spacegrad)
=
\Bx^a \PD{a}{}
+
\Bx^b \PD{b}{}
\equiv
\Bx^a \partial_a
+
\Bx^b \partial_b.
\end{dmath}

The Stokes integral can be evaluated over this volume element for either scalar fields \( \psi \) or vector fields \( \Bf \), and takes the form

\begin{subequations}
\label{eqn:stokesTheoremCore:140}
\begin{equation}\label{eqn:stokesTheoremCore:160}
\int_A d^2 \Bx \cdot (\boldpartial \wedge \psi) =
\int_A (d^2 \Bx \cdot \boldpartial) \psi
=
\int_{\partial A} d^1 \Bx \psi
\end{equation}
\begin{equation}\label{eqn:stokesTheoremCore:180}
\int_A d^2 \Bx \cdot (\boldpartial \wedge \Bf) =
\int_A (d^2 \Bx \cdot \boldpartial) \cdot \Bf
=
\int_{\partial A} d^1 \Bx \cdot \Bf.
\end{equation}
\end{subequations}

To extract the full meaning of this the boundary differential \( d^1 \Bx \) must be computed.  This has the same structure for a vector or scalar field

\begin{dmath}\label{eqn:stokesTheoremCore:220}
\begin{aligned}
\int_A d^2 \Bx \cdot (\boldpartial \wedge \Bf)
&=
\int_A (d^2 \Bx \cdot \boldpartial) \cdot \Bf \\
&=
\sum_\mu \int_A (d^2 \Bx \cdot \Bx^\mu) \cdot \partial_\mu \Bf \\
&=
\sum_\mu \int_A da db  \lr{ \Bx_a \wedge \Bx_b ) \cdot \Bx^\mu } \cdot \partial_\mu \Bf \\
&=
\sum_\mu \int_A da db  \lr{ \Bx_a {\delta_b}^\mu - \Bx_b {\delta_a}^\mu } \cdot \partial_\mu \Bf \\
&=
\int_A da db  \lr{ \Bx_a \cdot \PD{b}{ \Bf} - \Bx_b \cdot \PD{a}{\Bf} }
\end{aligned}
\end{dmath}

While \( \Bx_a, \Bx_b \) both depend on both parameters \( a, b \), the differential form immediately above is still a perfect integral in the variables of the partials since \( \Bx_a \) is computed with \( b \) held fixed, and \( \Bx_b \) is computed with \( a \) held fixed.  Proceeding with the integrals that match the respective partials, this gives

\begin{dmath}\label{eqn:stokesTheoremCore:240}
\int_A d^2 \Bx \cdot (\boldpartial \wedge \Bf)
=
\int
da \Bx_a \cdot \evalbar{\Bf}{\Delta b}
-\int
db \Bx_b \cdot \evalbar{\Bf}{\Delta a}
=
\int
d\Bx_a \cdot \evalbar{\Bf}{\Delta b}
-\int
d\Bx_b \cdot \evalbar{\Bf}{\Delta a}.
\end{dmath}

This shows that the boundary differential \( d^1 \Bx \) in \cref{eqn:stokesTheoremCore:140} is given by

\begin{dmath}\label{eqn:stokesTheoremCore:260}
d^1 \Bx = d\Bx_a - d\Bx_b,
\end{dmath}

where it is implied that the field in question is evaluated at the boundaries of the parameter that has been eliminated by this first integration.  This boundary integral can be interpreted as the integral around a contour, as indicated in
\cref{fig:twoParameterDifferentialBoundary:twoParameterDifferentialBoundaryFig2}.

\imageFigure{../figures/GAelectrodynamics/twoParameterDifferentialBoundaryFig2}{Contour for two parameter surface boundary.}{fig:twoParameterDifferentialBoundary:twoParameterDifferentialBoundaryFig2}{0.4}

Additionally, as with the single parameter case, a substitution of the gradient does not change the result, since any component of the gradient that lies outside of the tangent space on the surface has a zero dot product with the surface volume element \( d^2 \Bx \).
This allows the two parameter Stokes integrals to be written as

%\begin{dmath}\label{eqn:stokesTheoremCore:280}
\boxedEquation{eqn:stokesTheoremCore:280}{
\begin{aligned}
\int_A d^2 \Bx \cdot \spacegrad \psi &= \ointclockwise d\Bx \psi \\
\int_A d^2 \Bx \cdot (\spacegrad \wedge \Bf) &= \ointclockwise d\Bx \cdot \Bf.
\end{aligned}
}
%\end{dmath}

It can be shown that this two parameter Stokes integral is equivalent to Green's theorem.

      \subsection{Three parameter specialization of Stokes' theorem.}
         %
% Copyright © 2016 Peeter Joot.  All Rights Reserved.
% Licenced as described in the file LICENSE under the root directory of this GIT repository.
%

An example parameterization with three parameters, and the corresponding differentials with respect to those parameters, and the outwards normals, are sketched in
\cref{fig:normalsOnVolumeAreaElement:normalsOnVolumeAreaElementFig11}.

\imageFigure{../figures/gabook/normalsOnVolumeAreaElementFig11}{Three parameter volume element.}{fig:normalsOnVolumeAreaElement:normalsOnVolumeAreaElementFig11}{0.4}

Given parameters \( a, b, c \), the differentials along each of the parameterization directions are

\begin{dmath}\label{eqn:stokesTheoremCore:1421}
\begin{aligned}
d\Bx_a &= \PD{a}{\Bx} da = \Bx_a da \\
d\Bx_b &= \PD{b}{\Bx} db = \Bx_b db \\
d\Bx_c &= \PD{c}{\Bx} dc = \Bx_c dc.
\end{aligned}
\end{dmath}

The ``volume'' element for this parameterization (a surface area element) is

\begin{equation}\label{eqn:stokesTheoremCore:1441}
d^3 \Bx
=
d\Bx_a
\wedge
d\Bx_b
\wedge
d\Bx_c
=
da db dc (\Bx_a \wedge \Bx_b \wedge \Bx_c).
\end{equation}

The vector derivative, the projection of the gradient onto the surface at the point of integration (also called the tangent space), now has three components

\begin{dmath}\label{eqn:stokesTheoremCore:1461}
\boldpartial
=
\sum_\mu \Bx^\mu (\Bx_\mu \cdot \spacegrad)
=
\Bx^a \PD{a}{}
+
\Bx^b \PD{b}{}
+
\Bx^c \PD{c}{}
\equiv
\Bx^a \partial_a
+
\Bx^b \partial_b
+
\Bx^c \partial_c.
\end{dmath}

The Stokes integral can be evaluated over this volume element for either scalar fields \( \psi \), vector fields \( \Bf \), or bivector fields \( B \) and takes the form

\begin{subequations}
\label{eqn:stokesTheoremCore:1481}
\begin{equation}\label{eqn:stokesTheoremCore:1501}
\int_V d^3 \Bx \cdot (\boldpartial \wedge \psi) =
\int_V (d^3 \Bx \cdot \boldpartial) \psi
=
\int_{\partial V} d^2 \Bx \psi
\end{equation}
\begin{equation}\label{eqn:stokesTheoremCore:1521}
\int_V d^3 \Bx \cdot (\boldpartial \wedge \Bf) =
\int_V (d^3 \Bx \cdot \boldpartial) \cdot \Bf
=
\int_{\partial V} d^2 \Bx \cdot \Bf
\end{equation}
\begin{equation}\label{eqn:stokesTheoremCore:1541}
\int_V d^3 \Bx \cdot (\boldpartial \wedge B) =
\int_V (d^3 \Bx \cdot \boldpartial) \cdot B
=
\int_{\partial V} d^2 \Bx \cdot B.
\end{equation}
\end{subequations}

When working with \R{3} vector spaces, \( \boldpartial = \spacegrad \), but in higher dimensional spaces, the gradient can also be substituted above due using the same arguments about projection onto the tangent space.

An explicit value for the differential form of the boundary integral is desired and can be obtained from the mnemonic \cref{eqn:stokesTheoremCore:1561}

\begin{dmath}\label{eqn:stokesTheoremCore:1581}
\sum_i d^3 \Bx \cdot \Bx^i
=
\sum_i da db dc \lr{ \Bx_a \wedge \Bx_b \wedge \Bx_c } \cdot \Bx^i
=
\sum_i da db dc \lr{
\Bx_a \wedge \Bx_b +
\Bx_b \wedge \Bx_c +
\Bx_c \wedge \Bx_a }.
\end{dmath}

The bounding form for the three parameter volume is therefore

\begin{dmath}\label{eqn:stokesTheoremCore:1601}
d^2 \Bx
=
d\Bx_a \wedge d\Bx_b +
d\Bx_b \wedge d\Bx_c +
d\Bx_c \wedge d\Bx_a.
\end{dmath}

      \subsection{Using scalar volume elements}
         %
% Copyright © 2016 Peeter Joot.  All Rights Reserved.
% Licenced as described in the file LICENSE under the root directory of this GIT repository.
%

FIXME: remove most of this and introduce inline with the oriented area and volume results.  This is already done for the \( d^2 \Bx \) integrals.

In \R{3} the area elements of
(FIXME: equation reference dead with rewrite)
%\cref{eqn:twoparameter:140}
, and volume elements of 
\cref{eqn:threeparameter:1481}
can be reexpressed as scalars, recovering a number of the integral calculus identities that are more familiar than the wedge product variants above.

The pseudoscalar volume element can be written

\begin{dmath}\label{eqn:scalarVolumeElement:1621}
d^3 \Bx = I dV,
\end{dmath}
and the (oriented) area elements can be written as

\begin{dmath}\label{eqn:scalarVolumeElement:1641}
d^2 \Bx \ncap = I dA,
\end{dmath}
or
\begin{dmath}\label{eqn:scalarVolumeElement:1661}
d^2 \Bx = I \ncap dA.
\end{dmath}

For \( \psi \in \bigwedge^0, \Bf \in \bigwedge^1, B \in \bigwedge^2 \), this gives

\begin{subequations}
\label{eqn:scalarVolumeElement:1681}
\begin{equation}\label{eqn:scalarVolumeElement:1701}
I \int_A dA \ncap \wedge \spacegrad \psi = \ointclockwise d\Bx \psi
\end{equation}
\begin{equation}\label{eqn:scalarVolumeElement:1721}
I \int_A dA \ncap \wedge \spacegrad \wedge \Bf = \ointclockwise d\Bx \cdot \Bf
\end{equation}
\begin{equation}\label{eqn:scalarVolumeElement:1741}
\int_V dV \spacegrad \psi = \int_{\partial V} dA \ncap \psi
\end{equation}
\begin{equation}\label{eqn:scalarVolumeElement:1761}
\int_V dV \spacegrad \wedge \Bf = \int_{\partial V} dA \ncap \wedge \Bf
\end{equation}
\begin{equation}\label{eqn:scalarVolumeElement:1781}
\int dV \spacegrad \wedge B = \int_{\partial V} dA \ncap \wedge B
\end{equation}
\end{subequations}

It is straightforward to re-express all the wedge products above in their dual forms.
With \( B = I \Bf \), that is

\begin{subequations}
\label{eqn:scalarVolumeElement:1801}
\begin{equation}\label{eqn:scalarVolumeElement:1821}
\int_A dA \ncap \cross \spacegrad \psi = \ointctrclockwise d\Bx \psi
\end{equation}
\begin{equation}\label{eqn:scalarVolumeElement:1841}
\int_A dA \ncap \cdot (\spacegrad \cross \Bf) = \ointctrclockwise d\Bx \cdot \Bf
\end{equation}
\begin{equation}\label{eqn:scalarVolumeElement:1861}
\int_V dV \spacegrad \psi = \int_{\partial V} dA \ncap \psi
\end{equation}
\begin{equation}\label{eqn:scalarVolumeElement:1881}
\int_V dV \spacegrad \cross \Bf = \int_{\partial V} dA \ncap \cross \Bf
\end{equation}
\begin{equation}\label{eqn:scalarVolumeElement:1901}
\int dV \spacegrad \cdot \Bf = \int_{\partial V} dA \ncap \cdot \Bf.
\end{equation}
\end{subequations}

Each of the cross product terms above can also be put into dual forms, giving

\begin{subequations}
\label{eqn:scalarVolumeElement:1801c}
\begin{equation}\label{eqn:scalarVolumeElement:1821c}
\int_A dA \ncap \cdot \lr{ I \spacegrad \psi } = \ointclockwise d\Bx \psi
\end{equation}
\begin{equation}\label{eqn:scalarVolumeElement:1841c}
\int_A dA \ncap \cdot (\spacegrad \cdot B) = \ointctrclockwise d\Bx \cdot (I B)
\end{equation}
\begin{equation}\label{eqn:scalarVolumeElement:1881c}
\int_V dV \spacegrad \cdot B = \int_{\partial V} dA \ncap \cdot B.
\end{equation}
\end{subequations}

Note that all of
\cref{eqn:scalarVolumeElement:1861}, \cref{eqn:scalarVolumeElement:1901}, and \cref{eqn:scalarVolumeElement:1881c} all have the same form

%\begin{equation}\label{eqn:scalarVolumeElement:1881d}
\boxedEquation{eqn:scalarVolumeElement:1881d}{
\int_V dV \spacegrad \cdot A = \int_{\partial V} dA \ncap \cdot A.
}
%\end{equation}
\index{divergence theorem}

This is also true for pseudoscalar grades, which can be demonstrated by multiplying both sides of \cref{eqn:scalarVolumeElement:1741} with \( I \).
This implies that \cref{eqn:scalarVolumeElement:1881d} is valid for any \R{3} multivector, generalizing the conventional divergence theorem over a 3D volume to all spatial grades.

      \subsection{Problems}
         %
% Copyright � CCYY Peeter Joot.  All Rights Reserved.
% Licenced as described in the file LICENSE under the root directory of this GIT repository.
%
\makeproblem{Stokes' theorem relation to Green's theorem}{problem:stokesAndGreens:1}{
Show that Stokes' theorem, in its two parameter form, applied to a vector field recovers Green's theorem.
\index{Green's theorem}
\index{Stokes' theorem}
} % problem

\makeanswer{problem:stokesAndGreens:1}{

To demonstrate this, expand the LHS of the Stokes identity

\begin{dmath}\label{eqn:stokesAndGreens:20}
\int_A d^2 \Bx \cdot (\boldpartial \wedge \Bf) = \ointclockwise d\Bx \cdot \Bf.
\end{dmath}

Assuming \( u, v\) parameterization

\begin{dmath}\label{eqn:stokesAndGreens:40}
\int_A d^2 \Bx \cdot (\boldpartial \wedge \Bf)
=
\int_A (d\Bx_u \wedge d\Bx_v) \cdot (\boldpartial \wedge \Bf)
=
\int_A ((d\Bx_u \wedge d\Bx_v) \cdot \Bx^u) \cdot \partial_u \Bf
+
\int_A ((d\Bx_u \wedge d\Bx_v) \cdot \Bx^v) \cdot \partial_v \Bf
=
-\int_A du dv \Bx_v \cdot \partial_u \Bf
+
\int_A du dv \Bx_u \cdot \partial_v \Bf
=
-\int_A du dv \Bx_v \cdot \partial_u \Bf
+
\int_A du dv \lr{
-\Bx_v \cdot \partial_u \Bf
+
\Bx_u \cdot \partial_v \Bf
}.
\end{dmath}

The coordinate expansion of \( \Bf \) with respect to the tangent space coordinates is

\begin{dmath}\label{eqn:stokesAndGreens:60}
\Bf = \Bx^u f_u + \Bx^v f_v + \Bf_\perp
\end{dmath}

where \( \Bf_\perp \) lies in normal to the tangent space at the point in question.
Because \( \Bx_v \) is computed with \( u \) held fixed and \( \Bx_u \) computed with \( v \) held fixed, the area integrand can be written

\begin{dmath}\label{eqn:stokesAndGreens:80}
-\Bx_v \cdot \partial_u \Bf
+
\Bx_u \cdot \partial_v \Bf
=
-\PD{u}{}\lr{ \Bx_v \cdot \Bf }
+\PD{v}{}\lr{ \Bx_u \cdot \Bf }
=
-\PD{u}{f_v}
+\PD{v}{f_u},
\end{dmath}

which gives
\begin{dmath}\label{eqn:stokesAndGreens:100}
\int_A du dv \lr{ -\PD{u}{f_v}
+\PD{v}{f_u}
}
=
\ointclockwise d\Bx \cdot \Bf,
\end{dmath}

which recovers \cref{thm:stokesTheoremGeometricAlgebra:1660} as desired.
} % answer

         %
% Copyright © 2016 Peeter Joot.  All Rights Reserved.
% Licenced as described in the file LICENSE under the root directory of this GIT repository.
%

\makeproblem{\R{3} dual forms of Stokes' theorem.}{problem:stokesTheoremCoreProblems:1}{
Prove
\makesubproblem{}{problem:stokesTheoremCoreProblems:1:a}
\cref{eqn:scalarVolumeElement:1681},
\makesubproblem{}{problem:stokesTheoremCoreProblems:1:b}
\cref{eqn:scalarVolumeElement:1801},
\makesubproblem{}{problem:stokesTheoremCoreProblems:1:c}
and \cref{eqn:scalarVolumeElement:1801c}.
} % problem

\makeanswer{problem:stokesTheoremCoreProblems:1}{

The volume elements are
\makeSubAnswer{}{problem:stokesTheoremCoreProblems:1:a}
\begin{subequations}
\label{eqn:stokesTheoremCoreProblems:20}
\begin{dmath}\label{eqn:stokesTheoremCoreProblems:40}
d^2 \Bx \cdot \spacegrad
=
dA \gpgradeone{ I \ncap \spacegrad }
=
dA I \ncap \wedge \spacegrad
\end{dmath}
\begin{dmath}\label{eqn:stokesTheoremCoreProblems:60}
d^2 \Bx \cdot (\spacegrad \wedge \BA)
=
dA \gpgradezero{ I \ncap \spacegrad \BA }
=
dA I \ncap \wedge \spacegrad \wedge \BA
\end{dmath}
\begin{dmath}\label{eqn:stokesTheoremCoreProblems:80}
d^3 \Bx \cdot \spacegrad \phi
=
dV \gpgradetwo{ I \spacegrad \phi }
=
dV I \spacegrad \phi
\end{dmath}
\begin{dmath}\label{eqn:stokesTheoremCoreProblems:100}
d^3 \Bx \cdot (\spacegrad \wedge \BA)
=
dV \gpgradeone{ I (\spacegrad \wedge \BA) }
=
dV I \spacegrad \wedge \BA
\end{dmath}
\begin{dmath}\label{eqn:stokesTheoremCoreProblems:120}
d^3 \Bx \cdot (\spacegrad \wedge B)
=
dV \gpgradezero{ I (\spacegrad \wedge B) }
=
dV I \spacegrad \wedge B.
\end{dmath}
\end{subequations}

The corresponding boundary forms are
\begin{subequations}
\label{eqn:stokesTheoremCoreProblems:140}
\begin{equation}\label{eqn:stokesTheoremCoreProblems:160}
d\Bx \psi
\end{equation}
\begin{dmath}\label{eqn:stokesTheoremCoreProblems:180}
d\Bx \cdot \BA
\end{dmath}
\begin{dmath}\label{eqn:stokesTheoremCoreProblems:200}
d^2 \Bx \psi
=
dA I \ncap \psi
\end{dmath}
\begin{dmath}\label{eqn:stokesTheoremCoreProblems:220}
d^2 \Bx \cdot \BA
=
dA \gpgradeone{ I \ncap \BA }
=
dA I \ncap \wedge \BA
\end{dmath}
\begin{dmath}\label{eqn:stokesTheoremCoreProblems:240}
d^2 \Bx \cdot B
=
dA \gpgradezero{ I \ncap B }
=
dA I \ncap \wedge B.
\end{dmath}
\end{subequations}

Assembling these pieces back into the integrals proves the relationships.

\makeSubAnswer{}{problem:stokesTheoremCoreProblems:1:b}

To show \cref{eqn:scalarVolumeElement:1841} note that

\begin{dmath}\label{eqn:stokesTheoremCoreProblems:260}
I (\Ba \wedge \Bb \wedge \Bc)
=
\gpgradezero{ I \Ba \wedge \Bb \wedge \Bc }
=
\gpgradezero{ I \Ba (\Bb \wedge \Bc) -
I \Ba \cdot (\Bb \wedge \Bc)
}
=
\gpgradezero{ I \Ba I(\Bb \cross \Bc) }
=
- \Ba \cdot (\Bb \cross \Bc).
\end{dmath}

To show \cref{eqn:scalarVolumeElement:1901} note that

\begin{dmath}\label{eqn:stokesTheoremCoreProblems:280}
\Ba \wedge (I \BA)
=
\Ba \wedge (I \BA)
=
\gpgradethree{ \Ba I \BA }
=
\gpgradethree{ I \Ba \cdot \BA }
=
I (\Ba \cdot \BA).
\end{dmath}

\makeSubAnswer{}{problem:stokesTheoremCoreProblems:1:c}

For vector \( \Ba \), these transformations all follow from

\begin{dmath}\label{eqn:stokesTheoremCoreProblems:300}
\Ba \cross \Bf
=
\gpgradeone{ -I \Ba \wedge \Bf}
=
\gpgradeone{ -I \Ba \Bf}
=
-\gpgradeone{ \Ba I \Bf}
=
-\Ba \cdot (I \Bf)
=
\Ba \cdot B.
\end{dmath}

} % answer


   \section{Fundamental theorem of geometric calculus.}
      %\subsection{Fundamental theorem of geometric calculus.}
      %
% Copyright � 2016 Peeter Joot.  All Rights Reserved.
% Licenced as described in the file LICENSE under the root directory of this GIT repository.
%
%{
%\input{../blogpost.tex}
%\renewcommand{\basename}{fundamentalTheoremOfCalculus}
%\renewcommand{\dirname}{notes/phy1520/}
%%\newcommand{\dateintitle}{}
%%\newcommand{\keywords}{}
%
%\input{../peeter_prologue_print2.tex}
%
%\usepackage{peeters_layout_exercise}
%\usepackage{peeters_braket}
%\usepackage{peeters_figures}
%\usepackage{siunitx}
%
%\beginArtNoToc
%
%\generatetitle{Fundamental theorem of geometric calculus}
%\label{chap:fundamentalTheoremOfCalculus}

\subsection{Hypervolume integral}
We wish to generalize the concepts of line, surface and volume integrals to hypervolumes and multivector functions, and define a hypervolume integral as

\makedefinition{Multivector integral.}{dfn:fundamentalTheoremOfCalculus:240}{
Given a hypervolume parameterized by \( k \) parameters, k-volume volume element \( d^k \Bx \), and
multivector functions \( F, G \), we define k-volume integral with the vector derivative acting to the right on \( F \) as
\begin{equation*}
\int d^k\Bx \lr{ \rboldpartial F },
\end{equation*}
a k-volume integral with the vector derivative acting to the left \( F \) as
\begin{equation*}
\int F d^k\Bx \lboldpartial,
\end{equation*}
and a k-volume integral with the vector derivative acting bidirectionally on \( F, G \) as
\begin{equation*}
\int F d^k\Bx \lrboldpartial G
\equiv
\int \lr{ F d^k\Bx \lboldpartial} G
+
\int F d^k\Bx \lr{ \rboldpartial G },
\end{equation*}
where the meaning given to these directionally acting derivative operations is
\begin{equation*}
F d^k \Bx \lrboldpartial G
=
F d^k \Bx \lr{ \sum_i \Bx^i {\stackrel{ \leftrightarrow }{\partial_i}} } G
=
(\partial_i F) d^k \Bx \sum_i \Bx^i G
+
F d^k \Bx \sum_i \Bx^i (\partial_i G)
\equiv
(F d^k \Bx \lboldpartial) G
+
F d^k \Bx (\rboldpartial G),
\end{equation*}
with \( \boldpartial \) acting on \( F \) and \( G \), but not the volume element \( d^k \Bx \), which may also be a function of the implied parameterization.
} % definition

The vector derivative (and gradient)
may not commute with \( F, G \) nor the volume element \( d^k \Bx \), so we are forced to use some notation to indicate what the vector derivative (or gradient) acts on.
In conventional right acting cases, where there is no ambiguity, arrows will usually be omitted, but braces may also be used to indicate the scope of derivative operators.
This bidirectional notation will also be used for the gradient, especially for volume integrals in \R{3} where the vector derivative is identitical to the gradient.

Some authors use overdots or ticks are used to indicate the exact scope of multivector derivative operators, as in
\begin{dmath}\label{eqn:fundamentalTheoremOfCalculus:260}
F d^k \Bx \boldpartial G =
\dot{F} d^k \Bx \dot{\boldpartial} G
+
F d^k \Bx \dot{\boldpartial} \dot{G}.
\end{dmath}
Here the (Hestenes) dot notation would have the advantage of emphasizing that the action of the vector derivative (or gradient) is on the functions \( F, G \), and not on the hypervolume element \( d^k \Bx \).
In this book, where we will use ticks to indicate whether gradients are with respect to primed \( \Bx' \) or unprimed \( \Bx \) variables, over arrows seemed like a better choice than dots to indicate operator scope, and have the advantage of being visually conspicuous.

\subsection{Fundamental theorem.}
\index{fundamental theorem of geometric calculus}

The fundamental theorem of geometric calculus is a generalization of many conventional scalar and vector integral theorems.
It is a powerful theorem, which we will use with Green's functions to solve Maxwell's equation, and to derive the geometric algebra form of Stokes' theorem.

\maketheorem{Fundamental theorem of geometric calculus}{thm:fundamentalTheoremOfCalculus:1}{
For multivectors \(F, G \), and a hypervolume element \(d^k \Bx\),
\begin{equation*}
\int_V F d^k \Bx \boldpartial G = \oint_{\partial V} F d^{k-1} \Bx G.
\end{equation*}
}

This theorem relates the hypervolume integral to the integral over the bounding surface of hypervolume.
Additional work is required to describe the precise meaning of the boundary differential \( d^{k-1} \Bx \).  We will do so for line, surface, and volume integrals, proving the theorem in a limited fashion for each of those cases as we go.

For a full proof of \cref{thm:fundamentalTheoremOfCalculus:1}, additional mathematical sublties must be considered.
For full proofs and additional details, the reader is referred to \citep{hestenes1985clifford}, \citep{doran2003gap}, \citep{aMacdonaldVAGC} and \citep{sobczyk2011fundamental}, which all
which all tackle different aspects of general geometric calculus.

Before considering multivector line, surface and volume integral specializations of
\cref{thm:fundamentalTheoremOfCalculus:1},
we will state Stokes' theorem in its geometric algebra form.

%}
%\EndArticle

   \section{Green's functions}
      %
% Copyright � 2016 Peeter Joot.  All Rights Reserved.
% Licenced as described in the file LICENSE under the root directory of this GIT repository.
%
%{
\index{Green's function}

\subsection{Motivation.}

We will now introduce Green's functions, which provide a general method of solving many of the linear differential equations that will be encountered in electromagnetism.

One such linear differential equation is the inhomogeneous wave equation

\begin{dmath}\label{eqn:gradientGreensFunctionEuclidean:162}
\lr{ \spacegrad^2 - \inv{c^2} \PDSq{t}{} } F(\Bx, t) = B(\Bx, t)
\end{dmath}

The time harmonic (frequency domain) representation of the wave equation can be found by assuming a that fixed frequency solution exists.
In complex notation, that means that we can assume that all sources and fields have a complex exponential time dependence of the form
\footnote{This is the engineering convention for the time dependence.
The reader must take care when reading the literature, since some authors (notably in physics) use the opposite sign convention
\( F(\Bx, t) = \Real\lr{ F(\Bx) e^{-i \omega t }} \).}

\index{time harmonic}
\index{frequency domain}
\begin{dmath}\label{eqn:gradientGreensFunctionEuclidean:60}
F(\Bx, t) = \Real\lr{ F(\Bx) e^{j \omega t} },
\end{dmath}

where \( j \) is a scalar imaginary that need not be represented by any geometrical imaginary such as \( \Be_{123}, \Be_{12}, \cdots \).
After substitution of the time harmonic representation into \cref{eqn:gradientGreensFunctionEuclidean:162}, the problem is reduced to finding a solution that is a function of space and time to one that is purely spatial

\begin{dmath}\label{eqn:gradientGreensFunctionEuclidean:159}
\lr{ \spacegrad^2 + \frac{\omega^2}{c^2} } F(\Bx) = B(\Bx).
\end{dmath}

Superposition of discrete or continuous combinations of fixed frequency solutions, once found, can be used to determine more general solutions to the original wave equation \cref{eqn:gradientGreensFunctionEuclidean:162}.

We will writing \( \omega^2/c^2 = k^2 \), to obtain the standard form of the Helmholtz equation we wish to solve

\index{Helmholtz equation}
\index{second order Helmholtz equation}
\begin{dmath}\label{eqn:gradientGreensFunctionEuclidean:160}
\lr{ \spacegrad^2 + k^2 } F(\Bx) = B(\Bx).
\end{dmath}

This is a linear differential equation that is second order with respect to the gradient.
Despite employing a complex representations of the fields and sources, our vector basis is still a real valued Euclidean basis, and we will have no reason to introduce complex inner products spaces into the mix.
We will also encounter statics problems that have no time dependence in electromagnetism.
Some of these problems have the structure of \cref{eqn:gradientGreensFunctionEuclidean:160} with \( k = 0 \), and for those problems the fields and sources are real.

\index{Helmholtz operator}
Observe that the Helmholtz operator can be factored into operators that are first order in the gradient

\begin{dmath}\label{eqn:gradientGreensFunctionEuclidean:161}
\lr{ \spacegrad - j k }\lr{ \spacegrad + j k } F(\Bx) = B(\Bx).
\end{dmath}

We will see that the time harmonic Maxwell's equation, in its GA form, is a first order equation in the gradient of the form

\index{first order Helmholtz equation}
\begin{dmath}\label{eqn:gradientGreensFunctionEuclidean:180}
\lr{ \spacegrad + j k } F(\Bx) = J(\Bx),
\end{dmath}

where \( F \) is a (complex) 1,2 multivector, and \( J \) is a (complex) multivector containing all the charge and current density contributions.
Our initial goal is to develop the Green's function toolbox that can be used to solve first and second order Helmholtz equations of the form
\cref{eqn:gradientGreensFunctionEuclidean:180} and
\cref{eqn:gradientGreensFunctionEuclidean:160} respectively.

%}

      \subsection{Green's function solutions.}
         %
% Copyright © 2018 Peeter Joot.  All Rights Reserved.
% Licenced as described in the file LICENSE under the root directory of this GIT repository.
%
%{
\subsubsection{Unbounded.}

The operators in \cref{eqn:greensFunctionOverview:200}, and \cref{eqn:greensFunctionOverview:320} all have a similar linear structure.
Abstracting that structure, all these problems have the form
\begin{dmath}\label{eqn:greensFunctionSolutions:340}
\LL F(\Bx) = J(\Bx),
\end{dmath}
where \( \LL \) is an operator formed from a linear combination of linear operators \( 1, \spacegrad, \spacegrad^2, \partial_t, \partial_{tt} \).

Given the linear structure of the PDE that we wish to solve, it makes sense to assume that the solutions also have a linear structure.
The most general such solution we can assume has the form

\index{Green's function}
\begin{dmath}\label{eqn:greensFunctionSolutions:360}
F(\Bx, t) = \int G(\Bx, \Bx' ; t, t') J(\Bx', t') dV' dt' + F_0(\Bx, t),
\end{dmath}
where \( F_0(\Bx, t) \) is any solution to the equivalent homogeneous equation \( \LL F_0 = 0 \), and \( G(\Bx, \Bx' ; t, t') \) is the Green's function (to be determined) associated with \cref{eqn:greensFunctionSolutions:340}.
Operating on the presumed solution
\cref{eqn:greensFunctionSolutions:360} with \( \LL \) yields
\begin{dmath}\label{eqn:greensFunctionSolutions:380}
J(\Bx, t) = \LL F(\Bx, t) = \LL\lr{
\int G(\Bx, \Bx' ; t, t') J(\Bx', t') dV' dt' + F_0(\Bx, t) }
=
\int \lr{ \LL G(\Bx, \Bx'; t, t') } J(\Bx', t') dV' dt',
\end{dmath}
which shows that we require the Green's function to have delta function semantics satisfying
\begin{dmath}\label{eqn:greensFunctionSolutions:400}
\LL G(\Bx, \Bx' ; t, t') = \delta(\Bx - \Bx') \delta(t - t').
\end{dmath}

The scalar valued Green's functions for the Laplacian and the (2nd order) Helmholtz equations are well known.
The Green's functions for the spacetime gradient and the 1st order Helmholtz equation (which is just the gradient when \( k = 0 \)) are multivector valued and will be derived here.

%%%For multivector functions, it can be helpful to assume that the assumed solution
%%%\cref{eqn:greensFunctionSolutions:360} includes a grade selection operation.  In particular, for the electromagnetic field, which has only grades 1,2, we may start by demanding that our solution is of the form
%%%
%%%\begin{dmath}\label{eqn:greensFunctionSolutions:360}
%%%F(\Bx) = \gpgrade{ \int G(\Bx, \Bx') J(\Bx') dV'}{1,2} + F_0(\Bx),
%%%\end{dmath}
%%%
\subsubsection{Green's theorem.}

When the presumed solution is a superposition of only states in a bounded region
then life gets a bit more interesting.  For instance, consider a problem for which the differential operator is a function of space only, with a presumed solution such as
\begin{dmath}\label{eqn:greensFunctionSolutions:200}
F(\Bx) = \int_V dV' B(\Bx') G(\Bx, \Bx') + F_0(\Bx),
\end{dmath}
then life gets a bit more interesting.
This sort of problem requires different treatment for operators that are first and second order in the gradient.

For the second order problems, we require Green's theorem, which must be generalized slightly for use with multivector fields.

The basic idea is that we can relate the Laplacian of the Green's function and the field
\( F(\Bx') \lr{ (\spacegrad')^2 G(\Bx, \Bx') } = G(\Bx, \Bx') \lr{ (\spacegrad')^2 F(\Bx')} + \cdots \).
That relationship can be expressed as the integral of an antisymmetric sandwich of the two functions

\maketheorem{Green's theorem}{thm:gradientGreensFunctionEuclidean:220}{
Given a multivector function \( F \) and a scalar function \( G \)
\begin{equation*}
\int_V \lr{ F \spacegrad^2 G - G \spacegrad^2 F } dV = \int_{\partial V} \lr{ F \ncap \cdot \spacegrad G - G \ncap \cdot \spacegrad F },
\end{equation*}
where \( \partial V \) is the boundary of the volume \( V \).
} % theorem

A straightforward, but perhaps inelegant way of proving this theorem is to expand the antisymmetric product in coordinates
\begin{dmath}\label{eqn:greensFunctionSolutions:260}
F \spacegrad^2 G - G \spacegrad^2 F
=
\sum_k F \partial_k \partial_k G - G \partial_k \partial_k F
=
\sum_k \partial_k \lr{
F \partial_k G - G \partial_k F
}
-
(\partial_k F)(\partial_k G) + (\partial_k G)(\partial_k F).
\end{dmath}

Since \( G \) is a scalar, the last two terms cancel, and we can integrate
\begin{dmath}\label{eqn:greensFunctionSolutions:280}
\int_V \lr{ F \spacegrad^2 G - G \spacegrad^2 F } dV
=
\sum_k \int_V \partial_k \lr{ F \partial_k G - G \partial_k F }.
\end{dmath}

Each integral above involves one component of the gradient.
From
%the fundamental theorem of geometric calculus
\cref{thm:fundamentalTheoremOfCalculus:1}
we know that
\begin{dmath}\label{eqn:greensFunctionSolutions:300}
\int_V \spacegrad Q dV = \int_{\partial V} \ncap Q dA,
\end{dmath}
for any multivector \( Q \).
Equating components gives
\begin{dmath}\label{eqn:greensFunctionSolutions:460}
\int_V \partial_k Q dV = \int_{\partial V} \ncap \cdot \Be_k Q dA,
\end{dmath}
which can be substituted into \cref{eqn:greensFunctionSolutions:280} to find
\begin{dmath}\label{eqn:greensFunctionSolutions:480}
\int_V \lr{ F \spacegrad^2 G - G \spacegrad^2 F } dV
=
\sum_k \int_{\partial V} \ncap \cdot \Be_k \lr{ F \partial_k G - G \partial_k F } dA
=
\int_{\partial V} \lr{ F (\ncap \cdot \spacegrad) G - G (\ncap \cdot \spacegrad) F } dA,
\end{dmath}
which proves the theorem.

\subsubsection{Bounded solutions to first order problems.}

For first order problems we will need an intermediate result similar to Green's theorem.

\index{\(\spacegrad'\)}
\index{\(\rspacegrad'\)}
\index{\(\lspacegrad'\)}
\makelemma{Normal relations for a gradient sandwich.}{lemma:greensFunctionOverview:420}{
Given multivector functions \( F(\Bx'), G(\Bx, \Bx') \), and a gradient \( \spacegrad' \) acting bidirectionally on functions of \( \Bx' \), we have
\begin{equation*}
- \int_V \lr{ G(\Bx, \Bx') \lspacegrad' } F(\Bx') dV'
=
\int_V G(\Bx, \Bx') \lr{ \rspacegrad' F(\Bx') } dV'
-
\int_{\partial V} G(\Bx, \Bx') \ncap' F(\Bx') dA'.
\end{equation*}
} % lemma

% caused mysterious latex error in the .ind file:
%\index{\(\lrspacegrad'\)}
\index{\(\lroverarrow{\spacegrad}'\)}
This follows directly from \cref{thm:fundamentalTheoremOfCalculus:1}
\begin{dmath}\label{eqn:greensFunctionSolutions:440}
\int_{\partial V} G(\Bx, \Bx') \ncap' F(\Bx') dA'
=
\int_V G(\Bx, \Bx') \lrspacegrad' F(\Bx') dV'
=
\int_V \lr{ G(\Bx, \Bx') \lspacegrad' } F(\Bx') dV'
+
\int_V G(\Bx, \Bx') \lr{ \rspacegrad' F(\Bx') } dV',
\end{dmath}
which can be rearranged to prove \cref{lemma:greensFunctionOverview:420}.

%}

      \subsection{Helmholtz equation.}
         %
% Copyright © 2018 Peeter Joot.  All Rights Reserved.
% Licenced as described in the file LICENSE under the root directory of this GIT repository.
%
%{
\subsubsection{Unbounded superposition solutions for the Helmholtz equation.}

(FIXME: reference)
We can utilize \cref{eqn:greensFunctionOverview:160} to illustrate the Green's function technique.
As this equation is a linear differential operator relating the wave and the driving sources,
it is reasonable to assume that the solution also has a general linear structure, such as

\begin{dmath}\label{eqn:greensFunctionHelmholtz:100}
F(\Bx) = \int dV' B(\Bx') G(\Bx, \Bx') + F_0(\Bx),
\end{dmath}

where the function \( G(\Bx, \Bx') \) is called the Green's function for the Helmholtz operator, and \( F_0 \) is any particular solution to the inhomogeneous Helmholtz equation \( \lr{ \spacegrad^2 + k^2 } F_0 = 0 \).
Operating on \cref{eqn:greensFunctionHelmholtz:100} with the Helmholtz operator \( \spacegrad + k^2 \) we find that the Green's function must
satisfy

\begin{dmath}\label{eqn:greensFunctionHelmholtz:140}
\lr{ \spacegrad^2 + k^2 } G(\Bx, \Bx') = \delta(\Bx - \Bx').
\end{dmath}

While it is possible \citep{schwinger1998classical} to derive the Green's function using Fourier transform techniques, we will state the result instead, which is well known

\index{Helmholtz!Green's function}
\index{Green's function!Helmholtz}
\maketheorem{Green's function for the Helmholtz operator.}{thm:gradientGreensFunctionEuclidean:3}{
The advancing (causal), and the receding (acausal) Green's functions satisfying
\cref{eqn:greensFunctionHelmholtz:140} are respectively

\begin{equation*}
\begin{aligned}
G_{\textrm{adv}}(\Bx, \Bx') &= -\frac{e^{-j k \Norm{ \Bx - \Bx' } }}{ 4 \pi \Norm{\Bx - \Bx'}} \\
G_{\textrm{rec}}(\Bx, \Bx') &= -\frac{e^{j k \Norm{ \Bx - \Bx' } }}{ 4 \pi \Norm{\Bx - \Bx'}}.
\end{aligned}
\end{equation*}
} % theorem

We will use the advancing Green's function, and refer to this function as \( G(\Bx, \Bx') \) without any subscript.
A demonstration that these Green's function representations are valid can be found in \cref{chap:helmholtzGreens}.

\index{Laplacian!Green's function}
\index{Green's function!Laplacian}
Observe that as a special case, the Helmholtz Green's function reduces to the Green's function for the Laplacian when \( k = 0 \)

\begin{dmath}\label{eqn:greensFunctionHelmholtz:80}
G(\Bx, \Bx') = -\inv{ 4 \pi \Norm{\Bx - \Bx'}}.
\end{dmath}

\subsubsection{Bounded superposition solutions for the Helmholtz equation.}

For our application of
{thm:gradientGreensFunctionEuclidean:3} to the Helmholtz problem, we
are actually interested in a antisymmetric sandwich of the Helmholtz operator by the function \( F \) and the scalar (Green's) function \( G \), but
that reduces to a sandwich of Laplacian's

\begin{dmath}\label{eqn:greensFunctionHelmholtz:240}
F \lr{ \spacegrad^2 + k^2 } G - G \lr{ \spacegrad^2 + k^2 } F
=
F \spacegrad^2 G + \cancel{F k^2 G} - G \spacegrad^2 F - \cancel{G k^2 F}
=
F \spacegrad^2 G - G \spacegrad^2 F,
\end{dmath}

so

\begin{dmath}\label{eqn:greensFunctionHelmholtz:380}
\int_V F(\Bx') \lr{ (\spacegrad')^2 + k^2 } G(\Bx, \Bx')
=
\int_V G(\Bx, \Bx') \lr{ (\spacegrad')^2 + k^2} F(\Bx') dV'
+
\int_{\partial V} \lr{ F(\Bx') (\ncap' \cdot \spacegrad') G(\Bx, \Bx') - G(\Bx, \Bx') (\ncap' \cdot \spacegrad') F(\Bx') } dA'
\end{dmath}

This shows that if we assume the Green's function satisfies
the delta function condition
\cref{eqn:greensFunctionHelmholtz:140}
%that was also true for the unbounded case
, then the general solution to \cref{eqn:greensFunctionOverview:160} is
(FIXME: reference)

\boxedEquation{eqn:gradientGreensFunctionEuclidean:400}{
\begin{aligned}
F(\Bx) &=
\int_V G(\Bx, \Bx') B( \Bx' ) dV' \\
&+
\int_{\partial V} \lr{
 G(\Bx, \Bx') (\ncap' \cdot \spacegrad') F(\Bx')
-F(\Bx') (\ncap' \cdot \spacegrad') G(\Bx, \Bx')
} dA'.
\end{aligned}
}

We are also free to add in any specific solution \( F_0(\Bx) \) that satisfies the
homogeneous Helmholtz equation.
There is also freedom to add any solution of the homogeneous Helmholtz equation to the Green's function itself, so it is not unique.
For a bounded superposition we generally desire that the solution \( F \) and its normal derivative, or the Green's function \( G \) (and it's normal derivative) or an appropriate combination of the two are zero on the boundary, so that the surface integral is killed.

%}

      % redundant:
      %\subsection{First order gradient.}
      %  %
% Copyright © 2018 Peeter Joot.  All Rights Reserved.
% Licenced as described in the file LICENSE under the root directory of this GIT repository.
%
%{

\index{gradient!Green's function}
\index{Green's function!gradient representation}

We will see that the GA formulation of the statics equations (no time dependence), all have the form

\begin{dmath}\label{eqn:greensFunctionGradient:420}
\spacegrad F(\Bx) = J(\Bx),
\end{dmath}

where \( F, J \) are multivector fields and sources respectively.  We can assume an unbounded superposition solution

\begin{dmath}\label{eqn:greensFunctionGradient:440}
F(\Bx) = \int G(\Bx, \Bx') J(\Bx') dV' + F_0(\Bx),
\end{dmath}

where \( F_0 \) is any solution to the homogeneous gradient equation \( \spacegrad F_0 = 0 \), and operate on this presumed solution with the gradient to find

\begin{dmath}\label{eqn:greensFunctionGradient:460}
J(\Bx)
= \int \spacegrad G(\Bx, \Bx') J(\Bx') dV',
\end{dmath}

so the Green's function \( G \) for this system must satisfy

\begin{dmath}\label{eqn:greensFunctionGradient:480}
\spacegrad G(\Bx, \Bx') = \delta( \Bx - \Bx' ).
\end{dmath}

We will now show that this Green's function is vector valued as follows

\maketheorem{Green's function for the gradient}{thm:gradientGreensFunctionEuclidean:1}{
A Green's function that satisfies \cref{eqn:greensFunctionGradient:480} is
\begin{equation*}
   G(\Bx, \Bx') = \inv{4 \pi} \frac{\Bx - \Bx'}{\Norm{\Bx-\Bx'}^3}.
\end{equation*}
At points \( \Bx \ne \Bx' \), \( \spacegrad \wedge G = 0 \), or
\( \rspacegrad G = G \lspacegrad \).
} % theorem

To prove this, observe that we know the Laplacian representation of the delta function, so
the Green's function for the gradient can be written as

\begin{equation}\label{eqn:greensFunctionGradient:481}
\spacegrad G(\Bx, \Bx') = \delta( \Bx - \Bx' ) = \spacegrad^2 \lr{ -\inv{4\pi} \inv{ \Norm{\Bx - \Bx'} } }.
\end{equation}

GA provides us the rather beautiful and remarkable ability to factor the Laplacian in to a product of gradients \( \spacegrad^2 = \spacegrad \spacegrad \), so the gradient's Green's function is

\begin{dmath}\label{eqn:greensFunctionGradient:580}
G(\Bx, \Bx')
= \spacegrad \lr{ -\inv{4\pi} \inv{ \Norm{\Bx - \Bx'} } }
= -\inv{4\pi} \rcap \PD{r}{} \inv{r},
\end{dmath}

where \( \Br = \Bx - \Bx', r = \Norm{\Br} \), and \( \rcap = \Br/r \).  Proceeding with the derivatives, we find
\begin{dmath}\label{eqn:greensFunctionGradient:600}
G(\Bx, \Bx')
= -\inv{4\pi} \rcap \lr{ -\inv{r^2} }
= \frac{\rcap}{4 \pi r^2},
\end{dmath}

as claimed.  To show that the Green's function commutes with the gradient at points \( \Bx \ne \Bx' \) we can compute the (bivector) curl

\begin{dmath}\label{eqn:greensFunctionGradient:620}
\spacegrad \wedge \frac{\Br}{r^3}
=
\lr{ \spacegrad \inv{r^3}} \wedge \Br
+
\inv{r^3} \spacegrad \wedge \Br.
\end{dmath}

Since \( \spacegrad \inv{r^m} \propto \rcap \) the first wedge is zero.  The second wedge is also zero, which is easily demonstrated by coordinate expansion
\begin{dmath}\label{eqn:greensFunctionGradient:640}
\spacegrad \wedge \Br
=
\sum_{m,n} (\Be_m \partial_m) \wedge (\Be_n r_n)
=
\sum_{m,n} (\Be_m \wedge \Be_n) \partial_m r_n
=
\sum_{m,n} (\Be_m \wedge \Be_n) \delta_{m n}.
\end{dmath}

This last sum is zero since it is the symmetric sum of an antisymmetric quantity, which completes the proof.

We can determine the structure of an unbounded superposition solution through application of
%the fundamental theorem of geometric calculus
\cref{thm:fundamentalTheoremOfCalculus:1}

\begin{dmath}\label{eqn:greensFunctionGradient:500}
\int_V G(\Bx, \Bx') \lrspacegrad' F(\Bx') dV'
=
\int_V (G(\Bx, \Bx') \lspacegrad') F(\Bx') dV'
+
\int_V G(\Bx, \Bx') (\rspacegrad' F(\Bx')) dV'
=
\int_{\partial V} G(\Bx, \Bx') \ncap' F(\Bx') dA'.
\end{dmath}

As with the Helmholtz equation, we can presuming that the Green's function of
\cref{thm:gradientGreensFunctionEuclidean:1}
for the unbounded superposition solution also applies here.  Since

\begin{dmath}\label{eqn:greensFunctionGradient:520}
G(\Bx, \Bx') \lspacegrad'
=
-G(\Bx', \Bx) \lspacegrad'
=
-
\spacegrad' G(\Bx', \Bx)
= -\delta(\Bx - \Bx'),
\end{dmath}

we have
\begin{dmath}\label{eqn:greensFunctionGradient:540}
-F(\Bx)
+
\int_V G(\Bx, \Bx') J(\Bx') dV'
=
\int_{\partial V} G(\Bx, \Bx') \ncap' F(\Bx') dA',
\end{dmath}

or
\boxedEquation{eqn:gradientGreensFunctionEuclidean:560}{
F(\Bx)
=
\int_V G(\Bx, \Bx') J(\Bx') dV'
-
\int_{\partial V} G(\Bx, \Bx') \ncap' F(\Bx') dA'.
}

We are also free to add any specific solution \( F_0 \) to the gradient equation \( \spacegrad F_0 = 0 \).
Because the Green's function is not unique (we can add any solution \( G_0 \) of the gradient equation \( \spacegrad G_0 = 0 \)),
it may be desirable for bounded problems to construct Green's functions that are zero on the boundary of the integration volume.

%}

      \subsection{First order Helmholtz equation.}
         %
% Copyright © 2018 Peeter Joot.  All Rights Reserved.
% Licenced as described in the file LICENSE under the root directory of this GIT repository.
%
%{

The specialization of \cref{eqn:greensFunctionSolutions:400} to the first order Helmholtz equation \cref{eqn:greensFunctionOverview:240} is
\begin{dmath}\label{eqn:greensFunctionFirstOrderHelmholtz:700}
\lr{ \spacegrad + j k } G(\Bx, \Bx')  = \delta(\Bx - \Bx').
\end{dmath}

This Green's function is multivector valued

%
% Copyright � 2018 Peeter Joot.  All Rights Reserved.
% Licenced as described in the file LICENSE under the root directory of this GIT repository.
%
\maketheorem{Green's function for the first order Helmholtz operator.}{thm:gradientGreensFunctionEuclidean:720}{
The \textit{Green's function for the first order Helmholtz operator} \( \spacegrad + j k \) satisfies
\begin{equation*}
\lr{ \rspacegrad + j k } G(\Bx, \Bx') = G(\Bx, \Bx') \lr{ -\lspacegrad' + j k } = \delta(\Bx - \Bx'),
\end{equation*}
and has the value
\begin{equation*}
G(\Bx, \Bx') = \frac{e^{-j k r}}{4 \pi r} \lr{ j k \lr{ 1 + \rcap } + \frac{\rcap}{r} },
\end{equation*}
where \( \Br = \Bx - \Bx', r = \Norm{\Br} \) and \( \rcap = \Br/r \), and \( \spacegrad' \) denotes differentiation with respect to \( \Bx' \).
} % theorem


A special but important case is the \( k = 0 \) condition, which provides the
Green's function for the gradient, which is vector valued
\begin{equation}\label{eqn:greensFunctionFirstOrderHelmholtz:900}
G(\Bx, \Bx' ; k = 0) = \inv{4 \pi} \frac{\rcap}{r^2}.
\end{equation}

If we denote the (advanced) Green's function for the 2nd order Helmholtz operator
\cref{thm:gradientGreensFunctionEuclidean:3}
as \( \phi(\Bx, \Bx') \), we must have
\begin{equation}\label{eqn:greensFunctionFirstOrderHelmholtz:740}
\lr{ \rspacegrad + j k } G(\Bx, \Bx') = \delta(\Bx - \Bx') =
\lr{ \rspacegrad + j k } \lr{ \rspacegrad - j k } \phi(\Bx, \Bx'),
\end{equation}
we see that the Green's function is given by
\begin{dmath}\label{eqn:greensFunctionFirstOrderHelmholtz:760}
G(\Bx, \Bx')
=
\lr{ \rspacegrad - j k } \phi(\Bx, \Bx').
\end{dmath}

This can be computed directly
\begin{dmath}\label{eqn:greensFunctionFirstOrderHelmholtz:780}
G(\Bx, \Bx')
=
\lr{ \rspacegrad - j k } \lr{ -\frac{e^{-j k r}}{4 \pi r} }
=
\lr{ \rcap \PD{r}{} -j k } \lr{ -\frac{e^{-j k r}}{4 \pi r} }
=
\frac{-e^{-j k r}}{4 \pi}
\lr{
\rcap \lr{ -\frac{j k}{r} - \inv{ r^2 } } - \frac{j k}{r}
}
=
\frac{e^{-j k r}}{4 \pi}
\lr{
j k \lr{ 1 + \rcap } + \frac{\rcap}{r}
},
\end{dmath}
as claimed.
Observe that since \( \phi \) is scalar valued, we can also rewrite
\cref{eqn:greensFunctionFirstOrderHelmholtz:760} in terms of a right acting operator
\begin{dmath}\label{eqn:greensFunctionFirstOrderHelmholtz:800}
G(\Bx, \Bx')
=
\phi(\Bx, \Bx')
\lr{ \lspacegrad - j k }
=
\phi(\Bx, \Bx')
\lr{ -\lspacegrad' - j k },
\end{dmath}
so
\begin{equation}\label{eqn:greensFunctionFirstOrderHelmholtz:820}
G(\Bx, \Bx') \lr{ -\lspacegrad' + j k } =
\phi(\Bx, \Bx') \lr{ (\lspacegrad')^2 + k^2 }
=
\delta(\Bx - \Bx').
\end{equation}

This is relevant for bounded superposition states, which we will discuss next now that the proof of
\cref{thm:gradientGreensFunctionEuclidean:720} is complete.
In particular addition of
\( \int_V G(\Bx, \Bx') j k F(\Bx') dV' \) to both sides of \cref{lemma:greensFunctionOverview:420} gives
\begin{dmath}\label{eqn:greensFunctionFirstOrderHelmholtz:860}
\begin{aligned}
\int_V \lr{ G(\Bx, \Bx') \lr{ -\lspacegrad' + j k } } F(\Bx') dV'
&=
\int_V G(\Bx, \Bx') \lr{ \lr{ \rspacegrad' + j k } F(\Bx') } dV' \\
&-
\int_{\partial V} G(\Bx, \Bx') \ncap' F(\Bx') dA'.
\end{aligned}
\end{dmath}

Utilizing \cref{thm:gradientGreensFunctionEuclidean:720}, and substituting \( J(\Bx') \)
from \cref{eqn:greensFunctionOverview:240},
we find that one solution to the first order Helmholtz equation is
\begin{dmath}\label{eqn:greensFunctionFirstOrderHelmholtz:880}
F(\Bx)
=
\int_V G(\Bx, \Bx') J(\Bx') dV'
-
\int_{\partial V} G(\Bx, \Bx') \ncap' F(\Bx') dA'.
\end{dmath}

We are free to
add any specific solution \( F_0 \) that satisfies the homogeneous equation \( \lr{ \spacegrad + j k } F_0 = 0 \).
%}

      \subsection{Spacetime gradient.}
         %
% Copyright � 2018 Peeter Joot.  All Rights Reserved.
% Licenced as described in the file LICENSE under the root directory of this GIT repository.
%
%{
%%%\input{../latex/blogpost.tex}
%%%\renewcommand{\basename}{greensFunctionSpacetimeGradient}
%%%%\renewcommand{\dirname}{notes/phy1520/}
%%%\renewcommand{\dirname}{notes/ece1228-electromagnetic-theory/}
%%%%\newcommand{\dateintitle}{}
%%%%\newcommand{\keywords}{}
%%%
%%%\input{../latex/peeter_prologue_print2.tex}
%%%
%%%\usepackage{peeters_layout_exercise}
%%%\usepackage{peeters_braket}
%%%\usepackage{peeters_figures}
%%%\usepackage{siunitx}
%%%%\usepackage{mhchem} % \ce{}
%%%%\usepackage{macros_bm} % \bcM
%%%%\usepackage{macros_qed} % \qedmarker
%%%%\usepackage{txfonts} % \ointclockwise
%%%
%%%\beginArtNoToc
%%%
%%%\generatetitle{Green's function for the spacetime gradient}
%%%%\chapter{Green's function for the spacetime gradient}
%%%%\label{chap:greensFunctionSpacetimeGradient}
%%%% \citep{griffiths1999introduction}

We want to find the Green's function that solves spacetime gradient equations of the form \cref{eqn:greensFunctionOverview:220}.
That Green's function is multivector valued and given by

\maketheorem{Green's function for the spacetime gradient.}{thm:greensFunctionSpacetimeGradient:120}{
The Green's function for the spacetime gradient, satisfying
\begin{equation*}
\lr{ \spacegrad + \inv{c} \PD{t}{} } G(\Bx - \Bx', t - t') = \delta(\Bx - \Bx') \delta(t - t'),
\end{equation*}
is
\begin{equation*}
G(\Bx - \Bx', t - t')
=
\inv{4\pi} \lr{
- \frac{\rcap}{r62} \PD{r}{}
+ \frac{\rcap}{r}
+ \inv{c r} \PD{t}{}
}
\delta( -r/c + t - t' ),
\end{equation*}
where \( \Br = \Bx - \Bx', r = \Norm{\Br} \) and \( \rcap = \Br/r \).
} % theorem

With the help of \cref{eqn:derivativeOfDeltaFunction:140}
it is possible to further evaluate the delta function derivatives, however, we will defer doing so until we are ready to apply this Green's
function in a convolution integral to solve Maxwell's equation.

To prove this result, let \( \phi(\Bx - \Bx', t - t') \) be the retarded time (causal)
Green's function for the wave equation, satisfying

\begin{dmath}\label{eqn:greensFunctionSpacetimeGradient:40}
\lr{ \spacegrad + \inv{c} \PD{t}{} }
\lr{ \spacegrad - \inv{c} \PD{t}{} }
\phi(\Bx - \Bx', t - t') = \delta(\Bx - \Bx') \delta(t - t').
\end{dmath}

This function has the value
\begin{dmath}\label{eqn:greensFunctionSpacetimeGradient:60}
\phi(\Br, t - t')
=
-\inv{4 \pi r} \delta( -r/c + t - t' ),
\end{dmath}

where \( \Br = \Bx - \Bx', r = \Norm{\Br} \).  Derivations of this Green's function, and it's acausal advanced time friend, can be found in
\citep{schwinger1998classical}, \citep{jackson1975cew}, and use the usual Fourier transform and contour integration tricks.

Comparing \cref{eqn:greensFunctionSpacetimeGradient:40} to the defining statement of \cref{thm:greensFunctionSpacetimeGradient:120}, we see that the spacetime gradient Green's function is given by

\begin{dmath}\label{eqn:greensFunctionSpacetimeGradient:80}
G(\Bx - \Bx', t - t')
=
\lr{ \spacegrad - \inv{c} \PD{t}{} } \phi(\Br, t - t')
=
\lr{ \rcap \PD{r}{} - \inv{c} \PD{t}{} } \phi(\Br, t - t'),
\end{dmath}

where \( \rcap = \Br/r \).  Evaluating the derivatives gives

\begin{dmath}\label{eqn:greensFunctionSpacetimeGradient:100}
G(\Br, t - t')
=
-\inv{4\pi} \lr{ \rcap \PD{r}{} - \inv{c} \PD{t}{} } \frac{ \delta( -r/c + t - t' ) }{r}
=
-\inv{4\pi} \lr{
\frac{\rcap}{r} \PD{r}{} \delta( -r/c + t - t' )
- \frac{\rcap}{r^2} \delta( -r/c + t - t' )
- \inv{c r} \PD{t}{} \delta( -r/c + t - t' )
},
\end{dmath}

which completes the proof after some sign cancellation and minor rearrangement.
%}
%%%\EndArticle
%%%%\EndNoBibArticle

   \section{Helmholtz theorem.}
      %
% Copyright © 2016 Peeter Joot.  All Rights Reserved.
% Licenced as described in the file LICENSE under the root directory of this GIT repository.
%
In conventional electromagnetism Maxwell's equations are posed in terms of separate divergence and curl equations, so it is desirable to show that knowning the divergence and curl of a function and it's normal characteristics on the boundary of an integraion volume determine that function uniquely, known as the Helmholtz theorem
\maketheorem{Helmholtz first theorem.}{thm:helmholtzDerviationMultivectorStatement:1}{
If vector \(\BM\) is defined by its divergence

\begin{dmath}\label{eqn:helmholtzDerviationMultivectorStatement:20}
\spacegrad \cdot \BM = s
\end{dmath}

and its curl
\begin{dmath}\label{eqn:helmholtzDerviationMultivectorStatement:40}
\spacegrad \cross \BM = \BC
\end{dmath}

within a region and its normal component \( \BM_{\txtn} \) over the boundary, then \( \BM \) is
uniquely specified.
} % theorem

It could be argued that Helmholtz's theorem is irrelavent when using the GA formalism, since we consolidate the separate divergence and curl equations into one gradient operator.
We include a proof here regardless, since it can be performed in a compact and interesting fashion using
the fundamental theorem of geometric calculus \cref{thm:fundamentalTheoremOfCalculus:1}.


      %
% Copyright © 2016 Peeter Joot.  All Rights Reserved.
% Licenced as described in the file LICENSE under the root directory of this GIT repository.
%
The gradient of the vector \( \BM \) can be written as a single even grade multivector

\begin{equation}\label{eqn:helmholtzDerviationMultivector:60}
\spacegrad \BM
= \spacegrad \cdot \BM + I \spacegrad \cross \BM
= s + I \BC.
\end{equation}

%Observe that the Laplacian of \( \BM \) is vector valued
%
%\begin{dmath}\label{eqn:helmholtzDerviationMultivector:760}
%\spacegrad^2 \BM = \spacegrad s + I \spacegrad \BC.
%\end{dmath}
%
%This means that \( \spacegrad \BC \) must be a bivector \( \spacegrad \BC = \spacegrad \wedge \BC \), or that \( \BC \) has zero divergence
%
%\begin{dmath}\label{eqn:helmholtzDerviationMultivector:780}
%\spacegrad \cdot \BC = 0.
%\end{dmath}

This can be used to attempt to discover the relation between the vector \( \BM \) and its divergence and curl.
The vector \( \BM \) can be expressed at the point of interest as a convolution with the delta function at all other points in space

\begin{dmath}\label{eqn:helmholtzDerviationMultivector:80}
\BM(\Bx) = \int_V dV' \delta(\Bx - \Bx') \BM(\Bx').
\end{dmath}

The Laplacian representation of the delta function in \R{3} is

\begin{dmath}\label{eqn:helmholtzDerviationMultivector:100}
\delta(\Bx - \Bx') = -\inv{4\pi} \spacegrad^2 \inv{\Norm{\Bx - \Bx'}},
\end{dmath}

so \( \BM \) can be represented as the following convolution

\begin{dmath}\label{eqn:helmholtzDerviationMultivector:120}
\BM(\Bx) = -\inv{4\pi} \int_V dV' \spacegrad^2 \inv{\Norm{\Bx - \Bx'}} \BM(\Bx').
\end{dmath}

%As noted in \cref{eqn:helmholtzDerviationMultivector:460} the Laplacian of a vector can be factored as
%
%\begin{dmath}\label{eqn:helmholtzDerviationMultivector:140}
%\spacegrad^2 \Ba
%=
%\spacegrad (\spacegrad \cdot \Ba)
%-
%\spacegrad \cross (\spacegrad \cross \Ba).
%\end{dmath}
%
%Note that the last term can be written in cross product notation using \( \Bc \cdot (\Ba \wedge \Bb) = -\Bc \cross (\Ba \cross \Bb) \) if desired.

Using this relation and proceeding with a few applications of the chain rule, plus the fact that \( \spacegrad 1/\Norm{\Bx - \Bx'} = -\spacegrad' 1/\Norm{\Bx - \Bx'} \), we find
%
%I previously posted a Geometric Algebra attack on the Helmholtz theorem.  Here is
%
%Here's a third way of deriving the Helmholtz theorem inversion relation.  This is a refinement of the traditional vector algebra solution that led to \cref{eqn:helmholtzDerviationMultivector:200}, that uses a factorization of the Laplacian directly, deferring any expansion in terms of dot and cross (or wedge) products until the very end.
%
%Starting from the first line of \cref{eqn:helmholtzDerviationMultivector:160}, we have

\begin{dmath}\label{eqn:helmholtzDerviationMultivector:720}
-4 \pi \BM(\Bx)
= \int_V dV' \spacegrad^2 \inv{\Norm{\Bx - \Bx'}} \BM(\Bx')
= \gpgradeone{\int_V dV' \spacegrad^2 \inv{\Norm{\Bx - \Bx'}} \BM(\Bx')}
= -\gpgradeone{\int_V dV' \spacegrad \lr{ \spacegrad' \inv{\Norm{\Bx - \Bx'}}} \BM(\Bx')}
= -\gpgradeone{\spacegrad \int_V dV' \lr{
\spacegrad' \frac{\BM(\Bx')}{\Norm{\Bx - \Bx'}}
-\frac{\spacegrad' \BM(\Bx')}{\Norm{\Bx - \Bx'}}
} }
=
-\gpgradeone{\spacegrad \int_{\partial V} dA'
\ncap \frac{\BM(\Bx')}{\Norm{\Bx - \Bx'}}
 }
+\gpgradeone{\spacegrad \int_V dV'
\frac{s(\Bx') + I\BC(\Bx')}{\Norm{\Bx - \Bx'}}
 }
=
-\gpgradeone{\spacegrad \int_{\partial V} dA'
\ncap \frac{\BM(\Bx')}{\Norm{\Bx - \Bx'}}
 }
+\spacegrad \int_V dV'
\frac{s(\Bx')}{\Norm{\Bx - \Bx'}}
+\spacegrad \cdot \int_V dV'
\frac{I\BC(\Bx')}{\Norm{\Bx - \Bx'}}.
\end{dmath}

By inserting a no-op grade selection operation in the second step, the trivector terms that would show up in subsequent steps are automatically filtered out.
%the troublesome trivector term that shows up in my first purely Geometric Algebra
%attempt is eliminated.
This leaves us with a boundary term dependent on the surface and the normal and tangential components of \( \BM \).
Added to that is a pair of volume integrals that provide the unique dependence of \( \BM \) on its divergence and curl.
When the surface is taken to infinity, which requires \( \Norm{\BM}/\Norm{\Bx - \Bx'} \rightarrow 0 \), then the dependence of \( \BM \) on its divergence and curl is unique.

In order to express final result in traditional vector algebra form, a couple transformations are required.
The first is that

\begin{equation}\label{eqn:helmholtzDerviationMultivector:800}
\gpgradeone{ \Ba I \Bb } = I^2 \Ba \cross \Bb = -\Ba \cross \Bb.
\end{equation}

For the grade selection in the boundary integral, note that

\begin{dmath}\label{eqn:helmholtzDerviationMultivector:740}
\gpgradeone{ \spacegrad \ncap \BX }
=
\gpgradeone{ \spacegrad (\ncap \cdot \BX) }
+
\gpgradeone{ \spacegrad (\ncap \wedge \BX) }
=
\spacegrad (\ncap \cdot \BX)
+
\gpgradeone{ \spacegrad I (\ncap \cross \BX) }
=
\spacegrad (\ncap \cdot \BX)
-
\spacegrad \cross (\ncap \cross \BX).
\end{dmath}

These give

%\begin{dmath}\label{eqn:helmholtzDerviationMultivector:721}
\boxedEquation{eqn:helmholtzDerviationMultivector:721}{
\begin{aligned}
\BM(\Bx)
&=
\spacegrad \inv{4\pi} \int_{\partial V} dA' \ncap \cdot \frac{\BM(\Bx')}{\Norm{\Bx - \Bx'}}
-
\spacegrad \cross \inv{4\pi} \int_{\partial V} dA' \ncap \cross \frac{\BM(\Bx')}{\Norm{\Bx - \Bx'}} \\
&-\spacegrad \inv{4\pi} \int_V dV'
\frac{s(\Bx')}{\Norm{\Bx - \Bx'}}
+\spacegrad \cross \inv{4\pi} \int_V dV'
\frac{\BC(\Bx')}{\Norm{\Bx - \Bx'}}.
\end{aligned}
}
%\end{dmath}

   \section{Problem solutions.}
      \shipoutAnswer
%}
