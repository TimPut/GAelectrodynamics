%
% Copyright © 2016 Peeter Joot.  All Rights Reserved.
% Licenced as described in the file LICENSE under the root directory of this GIT repository.
%
\index{Helmholtz's theorem}
In conventional electromagnetism Maxwell's equations are posed in terms of separate divergence and curl equations.  It is therefore desirable to show that the divergence and curl of a function and it's normal characteristics on the boundary of an integration volume determine that function uniquely.  This is known as the Helmholtz theorem
\maketheorem{Helmholtz first theorem.}{thm:helmholtzDerviationMultivectorStatement:1}{
A vector \( \BM \) is uniquely determined by its
divergence
\begin{equation*}
\spacegrad \cdot \BM = s,
\end{equation*}
and curl
\begin{equation*}
\spacegrad \cross \BM = \BC,
\end{equation*}
and its value
over the boundary.
} % theorem

%It could be argued that Helmholtz's theorem is irrelavent when using the GA formalism, since we consolidate the separate divergence and curl equations into one gradient operator.
%We include a proof here regardless, since it can be performed in a compact and interesting fashion using
%%the fundamental theorem of geometric calculus
%\cref{thm:fundamentalTheoremOfCalculus:1}.
